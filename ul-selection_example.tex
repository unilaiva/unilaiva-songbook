 %% Selection of songs from unilaiva-songbook: an example
 %% =====================================================
 %%
 %% To create your own selection, just make a copy of this file named
 %% ul-selection_SOMETHING.tex in the project root, and edit the new
 %% file to have correct \mainbooktitle and \includeonlysongs, then
 %% run the compile script.
 %%
 %% To use a custom image as a cover, create a single page PDF of it,
 %% put it in content/img, and uncomment \coverpdf command below (in
 %% your new file) and set the correct file name (without path).
 %%
 %% The compile script will automatically compile every selection booklet,
 %% which is named ul-selection_*.tex and put in the root directory of the
 %% project.

\documentclass[twoside,10pt]{book}

%% Set the title of this document:
\newcommand{\mainbooktitle}{Example selection}

%% Uncomment this and set the filename without path (the file must be found
%% in content/img), to use a single-page PDF as a cover. Otherwise a cover
%% will be generated.
% \newcommand{\coverpdf}{my_selection_cover.pdf}

%% The following required package contains all the imports, style settings etc:
\usepackage{tex/unilaiva-songbook_common}

%% Settings (comment out as per your wishes):
\showtagsfalse    % Comment out to show tags in preludes
\showphfalse      % Comment out to show phase info in preludes
\showphintocfalse % Comment out to show phase info in TOC
%\shownotesfalse   % Comment out to show melody notes
%\showbeatsfalse   % Comment out to show beat marks
%\chordsoff        % Comment out to show chords

%% Here we set the song numbers included in this selection:
\includeonlysongs{107,232,333,334,502,703,105,324}

%% Required:
%% unilaiva-songbook_selections_include.tex
%% ========================================
%%
%% This file is a partial .tex file to be included with \input macro
%% to a Unilaiva songbook selection booklet main .tex file.
%%
%% See ul_selection_example.tex in the project's root directory for
%% an example and documentation on how to create these.
%%


\providecommand{\mainbooktitle}{Unilaiva Selection}
\providecommand{\bookmotto}{a selection of songs from Unilaiva Songbook}
\providecommand{\fullbookcoverimage}{Unilaiva-songbook_COVER.pdf}

% Disable Lilypond, as it doesn't work in selections for some reason. TODO: fix.
\showlilypondfalse

\upcasebooktitleinheadertrue
\upcasesectiontitleinheadertrue

% Do not show chapter nor section in the page headers, as they don't exist.
% Instead spread the book title and subtitle onto odd and even pages.
\fancypagestyle{unilaiva}[fancy]{%
  \fancyhead[CE]{%
    \headertitlestyle%
    \ifupcasebooktitleinheader%
      \MakeUppercase{\mainbooktitle}%
    \else%
      \mainbooktitle%
    \fi%
  }
  \fancyhead[CO]{%
    \headertitlestyle%
    \ifdefined\subbooktitle%
      \ifupcasesectiontitleinheader%
        \MakeUppercase{\subbooktitle}%
      \else%
        \subbooktitle%
      \fi%
    \fi
  }
}

% This will create the cover page for a selection booklet. If macro \coverpdf
% is defined, it's content will be assumed to be a file name of a PDF file in
% content/img, and will be used as a cover page. Otherwise a cover will be
% generated from \mainbooktitle, \subbooktitle (optional) and
% \fullbookcoverimage.
\newcommand{\coverpageforselection}{
  \ifdefined\coverpdf{
    \ulcoverpage[unilaiva-mode-icon-partial_512x512px.png]
                {\coverpdf}
                [unilaiva-mode-icon-music_512x512px]
  }\else{ % must be contained within a block for the settings to not bleed
    \thispagestyle{empty}
    \topskip0pt
    \vspace*{11.3ex} % align with "Contents" heading on the next page; BAD for any changes
    \begin{center}
      {\Huge \mainbooktitle}
      \ifdefined\bookmotto%
        {\par\normalsize\textit{\bookmotto}}
      \fi
      \par\vspace*{6.98ex}
      \ifdefined\subbooktitle%
        {\Large \textbf{\subbooktitle}}
      \fi
      \par\vspace*{\fill}\vspace*{\fill}
      \imagec[2]{\fullbookcoverimage}
      \par\vspace*{\fill}
    \end{center}
  }
  \fi
}

% This will create the second page for a selection booklet
\newcommand{\imprintpageforselection}{
  { % must be contained within a block for the settings to not bleed
    \thispagestyle{empty}
    \topskip0pt
    \normalsize
    \vspace*{11.3ex} % align with "Contents" heading on the next page; BAD for any changes
    \begin{center}
      {\Huge \mainbooktitle}
      \ifdefined\bookmotto% \bookmotto exists
        \par\textit{\normalsize\bookmotto}
      \fi
      \ifdefined\subbooktitle% \subbooktitle exists
        \par\vspace{2ex}
        {\Large \textbf{\subbooktitle}}
        \ifdefined\subsubbooktitle% \subsubbooktitle exists
          \par\vspace{0.618034ex}%
          {\normalsize \subsubbooktitle}
        \fi
      \fi
      \par\vspace*{\fill}
      \ifdefined\bookbytext% \bookbytext exists
        {\large \bookbytext}
        \par\vspace{2ex}
      \fi
      {\large \bookversionpretext}
      \par\vspace{0.618034ex}
      {\Large%
        %vvvv-mm-dd % Use hardcoded date date for tagged printout releases
        \songbookversion% current date, the default
        \ifchorded\else{\par\large\emph{\lyricsonlytext}}\fi% if non-chorded version, mention it here
      }
      \par\vspace*{\fill}
      {\normalsize Find the complete songbook at:}
      \\
      \ifdefined\bookwebsitelinkqrimage% \bookwebsitelink exists
        \imagel[3]{\bookwebsitelinkqrimage} % is already centered, so 'l' version ok
        \par
      \fi
      \ifdefined\bookwebsitelink% \bookwebsitelink exists
        {\small\url{\bookwebsitelink}}
      \fi
      % Add vspace once, if either \bookwebsitelink or \bookwebsitelinkqrimage,
      % or both, exist
      \ifdefined\bookwebsitelinkqrimage%
        \par\vspace*{\fill}
      \else\ifdefined\bookwebsitelink%
        \par\vspace*{\fill}
      \fi\fi
      \ifdefined\bookcompiledbytext% \bookcompiledbytext exists
        {\footnotesize\bookcompiledbytext}
        \par\vspace{4ex}
      \fi
      \ifdefined\imprintpagefootnote%
        {\scriptsize\imprintpagefootnote}%
      \fi
    \end{center}
    \vspace{-1.85ex}% workaround to align page bottom with other pages
  }
} % END \imprintpageforselection


\begin{document}

  \coverpageforselection % cover page here
  \clearpage
  \imprintpageforselection % the second (title) page here (verso)

  % TOC:
  \toc

  \clearpage

  \begin{songs}{}
    % Spanish (mostly) language songs
% ===============================
%
% The following sets the song number for the first song in this file.
% The number will automatically be incremented by one for each song.
% Please do not change this! Changing would make different versions of
% the songbook to have different numbers for the same songs, and it
% would totally mess up the selection booklets causing them to have
% wrong songs in them. (For the same reason, add new songs only to the
% end of each songs_ file.)
\setcounter{songnum}{100}


\beginsong{Elevo mi Canto}[by={Mariana Root}, ph={I}, key={Am}, sks={Am, Gm--Em}]
  \audio[key=Em]{https://soundcloud.com/user-624844191/elevo-mi-canto-i-raise-my-voice}
  \beginchorus\memorize
    \[^\mn{A}]E|\[\mnc{E}Am]levo mi can\[^\mn{C}]to al |\[\mnc{D}Em]cielo, yo el\[^ \mn{C}\mn{B}]evo |\[Am] \[^\mn{E}]elevo \[^\mn{C}]mi |\[\mnciii{D}{C}{B}Em]voz
    y |\[Am]busco con fe y con |\[Em]gracias |\[G] \up{*}la transforma|\[Am]ción
    ay yo busco |\[G] \up{*}la transforma|\[Am]ción | \e
  \endchorus
  \notesoff
  \altlyr{ay la unión, ay la sanación, ay la curación, ay la conexión, la iluminación, \ldots}
  \beginchorus
    |^Ay Pachamama, |^ay \up{*}madrecita |^ mil besos te |^doy
    |^porque eres músi|^ca \up{¤}sanadora |^ mil besos te |^doy
    \up{¤}sanadora |^ mil besos te |^doy | \e
  \endchorus
  \altlyr[*]{curandera, \ldots}\altlyrnospace[¤]{amorosa, \ldots}
  \begin{translation}
    I raise my song to the sky, I raise I raise my voice
    And with faith and thanks I seek the transformation
    Oh I seek the transformation (union, healing, connection, enlightenment)
    \nextverse
    Oh Pachamama, oh beloved mother, thousand kisses to you
    For your healing music thousand kisses to you
    Healer, a thousand kisses to you
  \end{translation}
\endsong


\beginsong{Cuatro Vientos}[by={Danit Treubig}, tags={wind}, ph={I, II}, key={Dm}, sks={C\#m, Bm--Em}]
  \audio[key=C\#m]{https://www.youtube.com/watch?v=Pclv31cDTTc}
  \audio[key=C\#m]{https://soundcloud.com/danit-treubig/cuatro-vientos}
  \transpose{5}
  \beginchorus\memorize
    |\[\mnc{E}Am]Vien\[^\mn{A}]to | que |\[\mnc{G}Em]viene de la \[^\mn{A}]mon|\[\mnciii{G}{F}{E}G]taña;
    |\[Am]Viento | tráe|\[Em]nos la clari|\[G]dad.
  \endchorus
  \notesoff
  \beginchorus
    |^Viento | que |^viene del |^mar; \altchords{\id{(Am)}|Am | |Em |G}
    |^Viento, | tráe|^nos la liber|^tad. \altchords{|Am | |Em |G}
  \endchorus
  \beginchorus\noteson
    \ind |\[\mnc{A}Am]Vuela vuela vuela vuela |\[\mn{E}]vuela vuela vue\[\mn{D}]la \altchords{|Am | \e}
    \ind vue|\[Em]la con no|\[G]sotros. \altchords {|Em |G}
  \endchorus
  \beginchorus
    |^Viento | que |^viene del de|^sierto;
    |^Viento, | tráe|^nos el si|^lencio. \goto{Vuela vuela}
  \endchorus
  \beginchorus
    |^Viento | que |^viene de la |^selva;
    |^Viento, tráe|^nos la me|^moria. \goto{Vuela vuela}
  \endchorus
  \begin{translation}
    Wind that comes from the mountain;
    Wind bring us clarity.
    \nextverse
    Wind that comes from the sea;
    Wind, bring us freedom.
    \nextverse
    \ind Fly, fly, fly, fly, fly, fly, fly, fly with us.
    \nextverse
    Wind that comes from the desert;
    Wind, bring us silence.
    \nextverse
    Wind that comes from the forest;
    Wind, bring us the memory.
  \end{translation}
\endsong


\beginsong{Ani Qu Ne'}[ex={tsalagi gawonihisdi, español}, tags={Moon, fire}, ph={I, II}, key={Am}, sks={Am, Am--Dm}]
  \audio[key=Cm]{https://soundcloud.com/minna_finland/a-li-kuni}
  \meter{4}{4}
  \beginchorus
    \ind |\[\mnc{A}Am]Ani \[\mnc{D}Dm]qu ne' |\[\mnc{E}E7]cha\[\mn{C}]wu'\[\mn{B}]na\[\mnc{A}Am]ni \rep{2}
    \ind |\[Dm]{A wa} \[\bm]wa bika |\[Am]na' kaye\[\bm]na \rep{2}
    \ind |\[Am]lyahuh\[G]thi' |\[Em]bisi\[Am]ti \rep{2}
  \endchorus
  \beginchorus
    |\[\mnc{A}Am]En \[\mn{B}]las \[\mnc{C}C]noches |\[\mnc{B}E7]cuan\[\mn{C}]do \[\mn{B}]la \[\mnc{A}Am]luna |co\[\mn{B}]mo \[\mnc{C}C]pla|\[\mnc{D}E7]ta \[\mn{C}]se \[\mn{B}]e\[\mnc{A}Am]leva
    |y la \[C]sel|\[G]va ilu\[Am]mina |y tam\[C]bién |\[G]la pra\[Am]dera
    |{ }{ } Los \[\bm]lobos |\[E7]en la \[Am]noche |llama\[C]rán al |\[E7]gran e\[Am]spíri-
    |tu\ldots \[G] |y al es\[Em]píri|tu del \[Am]fue|go \[\bm]
  \endchorus
  \goto{Ani qu ne'}
  \begin{translation}
    When evening descended upon the village
    The medicine man disappeared into the forest
    Touching the ground with his hands
    \nextverse
    In the nights when the silver moon rises
    And the forest and the meadow are illuminated
    Wolves in the night call the great spirit\ldots
    And the spirit of the fire
  \end{translation}
  \begin{explanation}
    The first part is a prayer of the \emph{Cherokee} people to call in the ancestors,
    honoring them and humbling to their wisdom. It is often sung as a lullaby.
  \end{explanation}
\endsong


\beginsong{Cuando la Luna}[by={Keya Maria}, tags={moon}, ph={II}, key={Em}, sks={Em, Dm--F\#m}]
  \audio[key=Em]{https://soundcloud.com/keyaiyapa/cuando-la-luna}
  \beginchorus
    \[\mn{B}]Cuando la |\[\mnc{E}Em]luna \[\mn{D}]re\[\mn{E}]donda \[\mn{F#}]es|\[\mnc{E}C \mn{D}]ta |\[\mn{E}] | \e
    Y se ilu|mina la oscuri|\[Em]dad | \up{1}(|) \e
  \endchorus\glueverses
  \beginchorus
    Vienen de |\[G]lejos a este lug|\[B]ar
    Magos, du|endes a concor|\[Em]dar \up{2}(| \e)
  \endchorus
  \beginchorus
    \ind |\[G]Que me \[D]crescan |\[Em]alas, |\[G]que me \[D]hablen los |\[Em]magos
    \ind |\[G]Quiro es\[D]tar pres|\[Em]ente |\[B7]en todo mo|\[Em]mento \up{2}(| | \e)
  \endchorus
  \notesoff
  \beginchorus
    Cuando la |\[Em]luna rendonda es|\[C]ta | | \e
    Cuando mil |luces mil colores |\[Em]caen | \up{1}(|) \e
  \endchorus\glueverses
  \beginchorus
    Vienen can|\[G]tando a este lug|\[B]ar
    Magos, du|endes a reali|\[Em]zar \up{2}(| \e)
  \endchorus
  \goto{Que me crescan alas}
  \begin{translation}
    \nextverse
    When the moon is round
    And it illuminates the darkness
    They come from far away to this place
    Wizards, elves to agree
    \nextverse
    \ind I grow wings, I speak magic
    \ind I want to be present at all times
    \nextverse
    When the moon is round
    When thousands of colored lights fall
    They come sing to this place
    Wizards, elves to make
  \end{translation}
\endsong


\beginsong{Lunita}[by={Danit Treubig},tags={moon},ph={II}, key={Am}, sks={Gm, Gm--Bm}]
  \audio[key=Gm]{https://www.youtube.com/watch?v=ZPC3P3ntzKY}
  \beginchorus
    \[\mn{E}]Lu|\[Am]ni\[\mn{C}]ta |\[\mnc{D}Dm]reina \[\mn{C}]de \[\mnc{B}Em]no\[\mn{C}]che
    Lu|\[Am]nita her|\[Dm]mana \[Em] de mi |\[Am]alma
  \endchorus
  \beginverse
    \ind \[\mn{C}]Tu |\[F]luz, \[\mn{E}]tu clari|\[C]dad, \[\mn{C}]tu vibra|\[F]ción, \[\mn{E}]tu armo|\[C]nía
    \ind Tu si|\[F]lencio, tu a|\[C]mor, pode|\[Em]rosa lu|\[Am]nita
  \endverse
  \beginverse
    \ind[3]|\[\mncii{D}{F}Dm]Rei\[\mn{E}]na \[\mn{F}]del |\[\mnc{G}G]cie\[\mn{D}]lo, \[\mn{E}\mn{D}]te |\[\mnc{E}C]doy \[\mn{D}]mi \[\mnc{E}Em]co\[\mn{F}]ra|\[\mnc{E}Am]zón
    \ind[3]|\[Dm]Reina del |\[G]cielo, compa|\[C]ñera de mi \[Em]vida, de mi |\[Am]alma
  \endverse
  \beginverse
    \ind \[\mn{C}]En tu |\[F]luz \[\mn{E}]todo el mundo |\[C]bril\[\mn{C}]la, en tu a|\[F]brazo \[\mn{E}]la tierra |\[C]can\[\mn{C}]ta
    \ind En tu |\[F]luz todo el mundo |\[C]brilla, pode|\[Em]rosa lu|\[Am]nita,
    \ind pode|\[Em]rosa abue|\[Am]lita
  \endverse
  \beginchorus
    \ind[2]\[\mn{E}]Dan|\[\mnc{F}F]zar con el universo, can|\[\mnc{C}C]tar en tu sonrisa
    \ind[2]Dan|\[F]zar con el universo y en tu |\[G]luz siempre fe|\[Am]liz
  \endchorus\glueverses\beginchorus
    \ind[2]En tu |\[G]luz siempre fe|\[Am]liz
  \endchorus
  \imagecc[3]{moon_bw_transparent_bg_939x939px.png}%
  \begin{translation}
    Moon, queen of night
    Moon, sister of my soul
    \nextverse
    Your light, your clarity, your vibration, your harmony
    Your silence, your love, powerful Moon
    \nextverse
    Queen of sky, I give you my heart
    Queen of sky, companion of my life, of my soul
    \nextverse
    In your light the whole world shines, in your embrace the Earth sings
    In your light the whole world shines, powerful Moon
    \nextverse
    To dance with the universe, to sing in your smile
    To dance with the universe and in your always happy light,
    in your always happy light
  \end{translation}
\endsong


\beginsong{En el Cielo y en la Tierra}[by={Isabel Ruiz},tags={Mother Earth, fire},ph={II}, key={Dm}, sks={Dm, Cm--Em}]
  \audio[key={Cm}]{https://soundcloud.com/lucas-rolim-126599764/en-el-cielo-pacha-mama-nessi-gomes-nn-uxhdayry}
  \transpose{5}
  \beginchorus\memorize
    |\[\mnc{E}Am]En \[^\mn{A}]el cielo y |\[\mnc{E}G]en \[^\mn{G}]la tierra |\[\mnc{E}Em]con \[^\mn{G}]el sol y |\[\mnc{A}Am]las \[^\mn{C}]es\[^\mn{A}]trellas
  \endchorus\glueverses
  \notesoff
  \beginchorus
    |^En el cielo y |^en la tierra, |^la lunita y |^las estrellas
  \endchorus
  \beginchorus
    |^Siento el fuego |^dentro dentro, |^siento el fuego a|^quí adentro
  \endchorus\glueverses
  \beginverse
    |^Vuela vuela |^aguilita, |^vuela vuela |^condorcito \replay
    |^Vuelan libres |^por nosotros, |^miren, cuiden |^todo todo
  \endverse
  \beginchorus
    |^Siento el fuego |^dentro dentro, |^siento el fuego a|^quí adentro
  \endchorus\glueverses
  \beginchorus
    |^Pachamama en |^este fuego, |^Pachamama a|^quí te encuentro
  \endchorus
  \begin{translation}
    In the sky and on earth with the sun and the stars
    In the sky and on earth, the moon and the stars
    \nextverse
    I feel the fire inside, I feel the fire in here
    Fly fly eagle, fly fly condor
    They fly free for us, looking, taking care of everything
    \nextverse
    I feel the fire inside, I feel the fire in here
    Pachamama in this fire, Pachamama here I find you
  \end{translation}
\endsong


\beginsong{Estrella Azul}[by={Narayan Dass}, tags={stars}, ph={II}, key={Am}, sks={Cm, Am--Dm}]
  \audio[key={Cm},pitch={432}]{https://maureenji.bandcamp.com/track/estrella-azul}
  \audio[]{https://soundcloud.com/bettinamaureenji/estrella-azul}
  \meter{3}{4}
  \beginchorus\memorize
    \[^\mn{E}]Es|\[\mnc{A}Am]toy a|\[^\mn{C}]qui en |\[^\mn{A}]su \[^\mn{C}]jar|\[\mnc{B}Em]din \altchords{\id{(Dm)}|Dm | | |Am}
    |\[Am]En su ser|\[C]vicio Se|\[Dm]ñor | \e \altchords{|Dm |F |Gm | \e}
    |Pregun|\[G/B]tando |\[C]eschu|\[Am]chando \altchords{| - |C/E |F |Dm}
    |\[C]Cami|\[E]nando a|\[Am]sí | \e \altchords{|F |A |Dm | \e}
  \endchorus
  \notesoff
  \beginchorus
    Yo |^veo su |cara es|trella a|^zul
    |^No mas mi|^edo tu|^ve | \e
    |Ya enten|^di |^como a|^sí
    Tu |^eres un |^padre para |^mi | \e
  \endchorus
  \beginchorus
    Me re|^galas su es|trella bri|llante a|^zul
    |^Yo la |^uso a|^si | \e
    Cal|mando a mis her|^manos con la |^fuerza del a|^mor
    a|^qui en |^este jar|^din | \e
  \endchorus
  \beginchorus
    |^Madre o |madre me co|noces a |^mi
    |^Eres como |^cuando na|^ci | \e
    |Gracias a|^mor por la |^vida a|^qui
    Yo |^can|^to a|^sí | \e
  \endchorus
  \begin{translation}
    I'm here in your garden
    In your service, Lord
    Asking, listening
    Walking like this
    \nextverse
    I see the face of blue star
    No more fear I have
    I already understand how
    You are a father to me
    \nextverse
    I am given your bright blue star
    I use it like that
    Calming my brothers with the force of love
    Here in this garden
    \nextverse
    Mother, oh mother, you know me
    You are like when I was born
    Thank you, love, for life here
    I sing like that
  \end{translation}
\endsong


\beginsong{Tierra, tan Sólo}[by={Marta Gómez, Federico García Lorca}, ph={II}, tags={earth}, key={Em}, sks={F\#m, Em--Gm}]
  \audio[key=Gm]{https://soundcloud.com/sudhir-ibiza/tierra}
  \audio[key=Gm]{https://www.youtube.com/watch?v=wS\_3tsJIKoM}
  \capo{2}
  \beginverse
    \[^\mn{B}]Ti|\[Em]er\[^\mn{E}]ra \[^\mn{G}]tan \[^\mn{F#}]só\[^\mn{E}]lo |\[\mnc{B}G/B]tierr\[^\mn{C}]a
    |\[Am] para las heridas re|\[F#\textdegree]cient\[B7]es
    Ti|\[Em]erra tan sólo |\[G/B]tierra
    |\[Am] para el humilde pensa|\[F#\textdegree]miento\[B7]
    Ti|\[Em]erra, tan sólo |\[D]tierra
    |\[Cmaj7] para el que huye de la |\[F#\textdegree]tie\[B7]{'r}|\[Em]ra | \e
  \endverse
  \notesoff
  \textnote{\musSegno}
  \beginverse
    Ti|^erra tan sólo |^tierra
    |^ tierra desnuda y ale|^{'gre} ^
    Ti|^erra, tan sólo |^tierra
    |^ tierra que ya no se mue|^{'ve} ^
    Ti|\[C]erra, tan sólo |\[D]tier\[C]ra
    de |\[B7]noches i|\[D]nme\[B7]{'n}|\[Em]sas
  \endverse
  \beginverse
    |\[D] No es la ce\[Cmaj7]{niza en} v|\[B7]ilo
    de las |\[C]cosas quemad|\[B7]as
    |\[Am] Lo que yo vengo busca|\[Em]ndo
    es |\[C]tie\[B7]{'r}|\[Em]ra
  \endverse
  \textnote{\up{1}Instrumental \emph{(1st time only)}}
  \textnote{\emph{D.S. al fine}}
  \beginverse
    Vi|\[Cmaj7]ento en el o|\[B7]{'livar}
    |\[Cmaj7]viento en la |\[B7]sier|\[Em]ra
  \endverse
  \begin{translation}
    Earth only earth
    for recent wounds
    Earth only earth
    for the humble thought
    Earth only earth
    for the one who runs away from the earth
    \nextverse
    Earth only earth
    bare and happy earth
    Earth only earth
    earth it doesn't move anymore
    Earth only earth
    of immense nights
    \nextverse
    It's not ash on the edge
    of the burned things
    What i come looking for
    is earth
    \nextverse
    Wind in the olive grove
    wind in the mountains
  \end{translation}
\endsong


\beginsong{Taita Inti Padre Sol}[by={Ramón Peregrino}, tags={Sun}, ph={II}, key={Dm}, sks={Dm, Bm--Em}]
  \audio[key=Am]{https://soundcloud.com/caminorojo/taita-inti-padre-sol-1}
  \transpose{5}
  \beginchorus\memorize
    \[^\mn{A}]Taita |\[\mnc{C}Am]Inti \[^\mn{A}]Padre |\[\mnc{E}Em]Sol,
    Ven ven |ven trae tu \up{1}ca|\[Am]lor \altlyr[2]{sa|ber}
  \endchorus
  \notesoff
  \beginchorus
    Está por el |^rio, por la tierra y por el |^mar,
    volando en el |viento el poder de Dios es|^tá
  \endchorus
  \beginchorus
    Eh Yage ya|^ge yage yage yage ya|^ho, \altchords{|Cm |Gm}
    Sol, Luna y Es|trellas, yo les canto otra |^vez \altchords{| |Cm}
  \endchorus
  \beginchorus
    Aya|^huasca, Caa|^pi,
    Inti inti|gua currupi curru|^pi
  \endchorus
  \beginchorus
    Como me en|^seño yo fui aquí y lo lla|^me,
    agradezco |siempre que nos abra su po|^der
  \endchorus
  \beginchorus
    Pájaro can|^tó, pájaro vo|^ló,
    lleva su pre|sencia quien aquí lo mere|^ció
  \endchorus
  \begin{translation}
    Taita Inti Father Sun,
    Come come come, bring your \up{1}warmth \up{2}{(knowledge)}.
    \nextverse
    The power of God is by the river, by the land
    and by the sea, flying in the wind.
    \nextverse
    Eh Yage yage yage yage yage yaho,
    Sun, Moon and Stars, I sing to you again.
    \nextverse
    Ayahuasca, Caapi,
    Inti intigua currupi currupi.
    \nextverse
    As he taught me I came here and called him,
    thank you for always opening your power to us.
    \nextverse
    Bird sang, bird flew,
    carried his presence to those who here deserved it.
  \end{translation}
\endsong


\beginsong{Aguilita}[by={Lua Maria}, tags={path, flying, wisdom}, ph={II}, key={Am}, sks={Am, Gm--Bm}]
  \audio[key=Gm]{https://soundcloud.com/lua-maria/abuelita-aguilita}
  \audio[]{https://adrianfreedman.com/product/the-phoenix-tree/}
  \transpose{5}
  \beginchorus
    |\[\mnc{E}Em]A\[\mn{F#}]bue\[\mn{G}]li\[\mn{A}]ta |\[\mn{B}]a\[\mn{E}]gui\[\mn{D}]li\[\mn{B}]ta, |\[\mnc{A}D]ven \[\mnc{G}C]ven |\[\mnc{E}Em]ven \rep{3} \altchords{\id{(Em)}|Em | |D C |Em}
  \endchorus\glueverses\beginchorus
    |\[D]Lleva nos por |\[Am]tu camin|\[C]o, sabi|\[Em]o \altchords{|D |Am |C |Em}
  \endchorus
  \beginchorus
    |\[Em]Vuela vuela |vuela vuela, |\[D]alas de \[C]uni|\[Em]dad \rep{3}
  \endchorus\glueverses\beginchorus
    |\[D]Lleva nos por |\[Am]tu camin|\[C]o, sabi|\[Em]o
  \endchorus
   \begin{translation}
     Grandmother eagle, come come come
     Take us on your path, wise
     \nextverse
     Fly fly fly fly, wings of unity
     Take us on your path, wise
   \end{translation}
\endsong


\beginsong{Aguila Aguilé}[ph={II}, key={Am}, sks={Am, Am--Dm}]
  \audio[key=Bm]{https://soundcloud.com/ta-li/aguila-aguile}
  \beginchorus\memorize
    \[^\mn{A}]Des\[^\mn{B}]de |\[\mnciii{C}{B}{A}Am]lejos, | \[^\mn{A}]des\[^\mn{B}]de \[^\mn{C}\mn{D}]lejos |\[\mnc{E}C]oigo | \e
    El |\[G]canto enamorado | de un |\[Am]pájaro | \e
  \endchorus
  \notesoff
  \beginverse
    Ese |^pájaro | es mi a|^buela | \e
    Es mi a|^buela \up{1}que canta, | |^canta enamorada | \e \altlyr[2]{Ayahuasca}
    \replay Ese |^pájaro | es mi a|^buelo | \e
    Es mi a|^buelo \up{1}que canta, | |^canta enamorado | \e \altlyr[2]{Peyote}
  \endverse
  \beginverse\noteson
    |\[\mnc{C}Am]Canta canta c\[\mn{B}]an\[\mn{A}]ta |\sublyrpush{canta canta canta} |\[\mnc{E}C]canta canta c\[\mn{D}]an\[\mn{C}]ta |\sublyr{canta canta canta} \e
    |\[G]Canta canta canta |\sublyrpush{canta canta canta} |\[Am]canta canta canta |\sublyr{canta canta canta} \e
  \endverse
  \beginchorus\noteson
    \ind |\[\mncii{E}{D}C]Águi\[\mn{E}]la | |\[\mnc{D}G]águila a\[\mn{E}]gui\[\mn{D}]lé | \e
    \ind |Águila agui|lé águila agui|\[Am]lé | \e
  \endchorus
  \begin{translation}
    From afar, from far away I hear
    The song in love with a bird
    \nextverse
    That bird is my grandmother
    It's my grandmother who sings, she sings in love
    \nextverse
    That bird is my grandfather
    It's my grandfather who sings, he sings in love
    \nextverse
    Sing sing sing sing sing
    Sing sing sing sing sing
    \nextverse
    Eagle eagle\ldots
  \end{translation}
\endsong


\beginsong{Abre tus Alas}[by={Nina},tags={path, flying},ph={II, III}]
  \beginchorus
    \[\mn{E}]Abre tus |\[\mncii{G}{E}Em]alas \[\mn{G}]pájaro vo|\[\mnc{F#}B7]lar
    Con plumas de co|\[Em]lores danzándole a la |\[B7]mar
  \endchorus
  \beginchorus
    \ind Abre tus |\[Am]alas libres y ha vo|\[Em]lar
    \ind El sol y las es|\[B7]trellas son mi guía cami|\[Em]nar \[\up{1}(E7)]
  \endchorus
  \beginchorus
    Abre tus |\[Em]alas sin miedo a sal|\[B7]tar
    El cielo nos en|\[Em]seña como aprender a|\[B7]mar
  \endchorus
  \beginchorus
    \ind Abre tus |\[Am]alas libres y ha vo|\[Em]lar
    \ind El sol y las es|\[B7]trellas son mi guía cami|\[Em]nar \[\up{1}(E7)]
  \endchorus
  \beginchorus
    \ind La luz del |\[Am]sol ilumina nuestro |\[Em]día
    \ind Emprendiendo el ca|\[B7]mino de amor y la ale|\[Em]gría \[\up{1}(E7)]
  \endchorus
\begin{translation}
  Open your wings, flying bird
  With colored feathers dancing to the sea
  \nextverse
  Open your wings free and have a flight
  The sun and the stars, they are the guide for my journey
  \nextverse
  Open your wings without fear of jumping to the unknown
  The sky teaches us how to learn to love
  \nextverse
  Open your wings free and have a flight
  The sun and the stars, they are the guide for my journey
  \nextverse
  The sunlight light up our day
  Starting the path of love and joy
\end{translation}
\endsong


\beginsong{Señora mi Señora}[by={Santiago Andrade, Irina Flores},ex={español, nahuatl},tags={Divine Mother},ph={II}]
  \beginchorus
    \[\mn{B}]Seño|\[\mnc{E}Em]ra, mi Seño|ra yarí
  \endchorus\glueverses
  \beginverse
    Pacha|\[Em]mama Nunkuy |\[G]Tierra sinchi |\[D]Tonāntzin coa|\[Em]tli yarí
    Pacha|\[Em]mama Nunkuy |\[G]Tierra sinchi |\[D]Tonāntzin coa|\[Em]tli yarí \brk| | \e
  \endverse
  \beginchorus
    El a|\[Em]güita de \up{1}la |vida hay si  \altlyr[2]{mi}
  \endchorus\glueverses
  \beginverse
    va cu|\[Em]rando, va sa|\[G]nando sinchi |\[D]Tonāntzin coa|\[Em]tli yarí
    va cu|\[Em]rando, va sa|\[G]nando sinchi |\[D]Tonāntzin coa|\[Em]tli yarí \brk| | \e
  \endverse
  \beginchorus
    Mai|\[Em]zito de \up{1}la |vida hay si  \altlyr[2]{mi}
  \endchorus\glueverses
  \beginverse
    va cu|\[Em]rando, va sa|\[G]nando sinchi |\[D]Tonāntzin coa|\[Em]tli yarí
    va cu|\[Em]rando va sem|\[G]brando sinchi |\[D]Tonāntzin coa|\[Em]tli yarí \brk| | \e
  \endverse
  \beginchorus
    Comi|\[Em]dita de \up{1}la |vida hay si  \altlyr[2]{mi}
  \endchorus\glueverses
  \beginverse
    va cu|\[Em]rando va sa|\[G]nando sinchi |\[D]Tonāntzin coa|\[Em]tli yarí
    va cu|\[Em]rando alimen|\[G]tando sinchi |\[D]Tonāntzin coa|\[Em]tli yarí \brk| | \e
  \endverse
  \beginchorus
    Ay los |\[Em]frutos de \up{1}la |vida hay si  \altlyr[2]{mi}
  \endchorus\glueverses
  \beginverse
    van cu|\[Em]rando van sa|\[G]nando sinchi |\[D]Tonāntzin coa|\[Em]tli yarí
    van cu|\[Em]rando hay endul|\[G]zando sinchi |\[D]Tonāntzin coa|\[Em]tli yarí \brk| | \e
  \endverse
  \begin{explanation}
    \begin{description}
      \item[Tonāntzin] is the title for the Aztec Mother Goddess, by which Goddesses such as
        \emph{Mother Earth}, the \emph{Goddess of Sustenance}, \emph{Honored Grandmother},
        \emph{Snake}, \emph{Brin\-ger of Maize} and \emph{Mother of Corn} can be called,
        as it is an honorific title comparable to ``Our Lady'' or ``Our Great Mother''.
      \item[Nunkuy:] Mother Earth
      \item[coatli:] ``water serpent'' or ``serpent water'', name for several medicinal plants.
        \emph{Coatl} means ``serpent'' or ``twin''. \emph{Cōātlīcue}, ``skirt of snakes'', is the
        primordial earth goddess, mother of the gods, the sun, the moon and the stars.
      \item[yarí:] ``heart''
    \end{description}
  \end{explanation}
  \imagecc[1]{snake_sa_bw_transparent_bg_960x262px.png}%
\endsong


\beginsong{Ayahuasca Ayní}[by={Diego Palma},tags={Aya},ph={II}]
  \meter{6}{8}
  \beginchorus
    |\[Am] \[\mn{A}]Narananai |\[\mn{E}]Narai Naira|\[Em]nai Naranai Na|\[Am]nai | \e
  \endchorus
  \beginchorus\memorize
    |\[Am] Ayahuasca ay|\[G]ní, | Ayahuasca |\[Am]cúranos
  \endchorus
  \beginchorus
    |^ Madrecita ay|^ní, | madrecita |^cúranos
  \endchorus
  \beginchorus
    |^ Medicina ay|^ní, | medicina |^cúranos
  \endchorus
  \beginchorus
    |^ Abuelita ay|^ní, | abuelita |^cúranos
  \endchorus
\endsong


\beginsong{Medicina que Trae los Cielos}[by={Hamilton Dielu},ph={II, III}]
  \audio[]{https://soundcloud.com/sacredvalleytribe/medicina-que-trae-los-cielos}
  \meter{3}{4}
  \newchords{chords_medicinaque_a}\newchords{chords_medicinaque_b}
  \beginchorus\memorize[chords_medicinaque_a]
    \[^\mn{E}]Medi|\[\mnc{A}Am]cina que |\[\mnc{B}E]tra\[^\mn{C}]e \[^\mn{B}]los |\[\mnc{A}Am]cielos des|\[^\mn{E}]de la flo|\[\mnc{D}Dm]resta a
    |\[E]mi cora|\[Am]zón |\[\up{2}(A7)] \e
  \endchorus\glueverses
  \notesoff
  \beginchorus\memorize[chords_medicinaque_b]
    Alum|\[Dm]bran|do ca|\[C]mi|no tray|\[E7]endo |libera|\[Am]ción |\[\up{1}A7] \e
  \endchorus
  \beginchorus\replay[chords_medicinaque_a]
    Doy |^gracias a |^todos los |^seres ma|estros di|^vinos
    |^de reden|^ción |^ \e
  \endchorus\glueverses
  \beginchorus\replay[chords_medicinaque_b]
    Humil|^dad |y pa|^cien|cia en|^señan |la compa|^sión |^ \e
  \endchorus
  \beginchorus\replay[chords_medicinaque_a]
    Aquí |^pido por |^todos los |^seres |desampa|^rados y
    |^sin protec|^ción |^ \e
  \endchorus\glueverses
  \beginchorus\replay[chords_medicinaque_b]
    Ofre|^cien|do este |^can|to desde el |^fondo de |mi cora|^zón |^ \e
  \endchorus
  \begin{translation}
    Medicine that brings the skies from the forest to my heart
    Lighting the path by bringing liberation
    \nextverse
    I give thanks to all divine beings of redemption
    Humility and patience teach compassion
    \nextverse
    Here I ask for all the helpless beings without protection
    Offering this song from the bottom of my heart
  \end{translation}
\endsong


\beginsong{Abrete Corazón}[by={Rosa Giove},tags={opening},ph={II, III}]
  \beginchorus
    |\[G] \[\mn{B}]Ábrete \[\mn{A}]co\[\mn{G}]ra|zón, |\[Em] ábrete senti|miento
    |\[C] ábrete entendi|miento, deja a un |\[G]lado la ra|zón
    y |\[D]deja brillar el |sol escon|\[C]dido | en tu inte|\[G]rior | \e
  \endchorus
  \beginverse
    |\[C]Ábrete memoria an|tigua escon|\[Bm]dida en la |tierra
    en las |\[C]plantas, | en el |\[D]aire | \e
    Re|\[C]cuerda lo que apren|diste, bajo |\[Bm]agua, bajo |fuego
    hace |\[C]ya, | ya mucho |\[D]tiempo | \e
  \endverse
  \beginverse
    Ya es |\[G]hora ya, ya es |\[Bm]hora |\[C] abre la mente y re|\[D]cuerda
    como el e|\[Em]spíritu cura, como el |\[Bm]amor sana
    |\[C] como el árbol flo|\[D]rece y la \lrep|\[C]vida | per|\[D]dura | \e \rrep
  \endverse
  % % extra verse:
  % \beginverse
  %   Respira profundo; y eleve-te hasta el Cielo
  %   Mira el Sol de frente; como las águilas
  %   No tengas temor; Confia en ti
  %   Confia en Dios; Confia en la Vida
  % \endverse
  \begin{translation}
    Open your heart, open to your feelings
    Open to your understanding, leave reason aside
    And let shine the Sun that is hidden within you
    \nextverse
    Open up ancient memory hidden in the Earth
    In the plants, in the air
    Remember what you learned, under water, under fire
    long, long time ago
    \nextverse
    It is time, now is the hour to open your mind and remember
    how Spirit cures, how love heals,
    how trees flourish and life endures
    % % extra verse:
    % \nextverse
    % Open your wings, Breathe deeply
    % And lift yourself up to the sky
    % Look at the sun in front of you like eagles do
    % Have no fear, Trust in yourself
    % Trust in God; Trust in Life
  \end{translation}
\endsong


\beginsong{El Viejo Tambor}[by={Alonso del Río},tags={path, heart},ph={II, III}]
  \audio[]{https://www.youtube.com/watch?v=PTU8DtcFKO4}
  \audio[]{https://soundcloud.com/makaruja/el-viejo-tambor}
  \beginverse
    |\[C] \[^\mn{C}]Desde los |tiempos en que mi a|\[\mnc{D}G]buelo so\[^\mn{G}]ñó
    Que un |día yo vivi|\[F]ría para can|tarles esta \[G]can|\[Am]ción | \e
    |\[C] Desde los |tiempos en que mi a|\[G]buelo soñó
    Que un |día yo mori|\[F]ría para entre|garles mi co\[G]ra|\[Am]zón | \e
  \endverse
  \notesoff
  \beginverse
    \ind |\[Dm] Y voy can|tando por mi ca|\[F]mino
    \ind Me va ale|grando un viejo \[G]tam|\[Am]bor | | | \e
    \ind |\[Dm] Vivo dan|zando por mi ca|\[F]mino
    \ind Me va gui|ando un viejo \[G]tam|\[Am]bor | | | \e
  \endverse
  \beginchorus
    |^ Por las que|bradas veo mi |^Cóndor volar
    Y |siento que voy a|^briendo
    Como el sus |alas mi co^ra|^zón | \e
  \endchorus
  \goto{Y voy cantando}
  \beginchorus
    |^ Por las mon|tañas veo que mi |^Aguila va
    Me |va mostrando el ca|^mino
    Camino |Rojo del co^ra|^zón | \e
  \endchorus
  \goto{Y voy cantando}
  \begin{translation}
    Since the time when my grandfather dreamed
    That one day I would live to sing this song
    Since the times when my grandfather dreamed
    That one day I would die to receive this in my heart
    \nextverse
    And I'm singing on my way
    An old drum brings me joy
    I live dancing on my way
    I am guided by this old drum
    \nextverse
    Through the ravines I see my Condor fly
    And I feel that I'm opening my heart
    Like his wings
    \nextverse
    Through the mountains I see that my Eagle is soaring
    Showing me the way
    The Red Path of the Heart
  \end{translation}
\endsong


\beginsong{Mira como Cura el Agua}[by={Alonso del Río},tags={water, air, fire, earth},ph={III}]
  \beginverse
    |\[\mnc{A}Am]Mira |como |cura el |\[^\mn{E}]agua
    |\[Dm] va la|vando como un |\[Am]rí|o | | \e
    |\[Am]Mira |como |cura el |aire
    |\[Dm] va can|tándote al o|\[Am]í|do | | \e
    |\[Am]Mira |como |cura el |fuego
    |\[Dm] va que|mando hasta el ol|\[Am]vi|do | | \e
    |\[Am]Como |la tier|ra le|vanta
    |\[Dm] tu cora|zón tan do|\[Am]li|do | | \e
  \endverse
  \beginverse
    \ind |\[F] Que Willka|mayu traiga las |\[C]notas
    \ind \[F]para que a|\[C]legres tu cora|zón
    \ind |\[F] Que el Apu |Linli todas las |\[C]tardes
    \ind \[F]venga so|\[C]plando su bendi|ción
    \ind |\[F] Que el Ch’eqta |Qaqa venga tra|\[C]yendo
    \ind \[F]el padre |\[C]rayo que enseña|rá
    \ind |\[F] Que Mama |Ñusta viene cui|\[C]dando
    \ind \[F]a sus hi|\[C]jitos \[E7]en la os|curi|\[Am]dad | | | \e
  \endverse
  \begin{translation}
    See how the water cures by cleansing like a river.
    Watch how the air cures by singing into your ear.
    See how the fire cures by burning to oblivion,
    How the earth raises your hurt heart.
    \nextverse
    May Willkamayu bring the notes so that you may rejoice in your heart.
    May Apu Linli come every evening to blow his blessing.
    May the Ch’eqta Qaqa come bringing the teaching father ray.
    May Mama Ñusta come taking care of her little children in the darkness.
  \end{translation}
  \begin{explanation}
    \begin{description}
      \item[Willkamayu] is the central river in \emph{El Valle Sacrado} close to
        \emph{Cusco, Peru}.
      \item[Apu Linli, Ch’eqta Qaqa and Mama Ñusta] are \emph{apu}s, sacred mountain spirits,
        around that valley.
    \end{description}
  \end{explanation}
\endsong


\beginsong{Amor y Unidad}[by={Bóveda Celeste},ph={III}]
  \meter{6}{8}
  \beginchorus
    |\[\mnc{E}Em]Corren dos liebres a|\[\mn{G}]tra\[\mn{F#}]ve\[\mn{E}]sando \[\mn{D}]el
    |\[Am]bosque donde na|cieron
    de|\[Em]scansan a un lado de un |lago donde
    pla|\[Am]tican con un jil|guero
  \endchorus
  \beginchorus
    \ind Que co|\[F]menta que el a|mor es
    \ind compar|\[C]tir lo que hay en |ti
    \ind es |\[Em]nuestra natura|leza, nuestra
    \ind e|\[Am]sencia más su|til
  \endchorus
  \beginchorus
    |\[Em]Águila y condor se en|cuentran en la
    |\[Am]cima de una cordi|llera
    sem|\[Em]brando un mensaje que |llega al
    inte|\[Am]rior de la madre |tierra
  \endchorus
  \beginchorus
    \ind Que nos |\[F]dice que ve|nimos de la misma
    \ind |\[C]fuente y del mismo lu|gar
    \ind to|\[Em]dos somos de la |tierra,
    \ind del |\[Am]agua, del fuego y del |aire
  \endchorus
  \begin{translation}
    Two hares run through the forest where they were born;
    they rest at the side of a lake where they talk with a goldfinch
    \nextverse
    Which says that love is to share what is in you
    It is our nature, our most subtle essence
    \nextverse
    Eagle and condor are at the top of a mountain range
    Sowing a message that reaches the interior of Mother Earth
    \nextverse
    According to it we come from the same source and from the same place;
    we are all from the earth, the water, the fire and the air
  \end{translation}
  \imagecc[3]{eagle_feather_bw_transparent_bg_1280x442px.png}%
\endsong


\beginsong{Que Florezca la Luz}[tags={light, love},ph={III}]
  \meter{6}{8}
  \beginchorus
    \[\mn{A}]Que flo|\[Am]rez\[\mn{E}]ca \[\mn{D}]la |\[G]luz, que flo|\[E7]rez\[\mn{F}]ca \[\mn{E}]la |\[Am]luz \[\up{2}(A7)]
  \endchorus\glueverses
  \notesoff
  \beginchorus
    |\[Dm]La luz del |\[Am]sol, |\[E7]la luz del a|\[Am]mor \[\up{1}A7] \up{2}(| \e)
  \endchorus
  \beginchorus
    |\[Am]Oh gran es|\[G]píritu, gran es|\[E7]píritu de a|\[Am]mor \[\up{2}(A7)]
  \endchorus\glueverses
  \beginchorus
    |\[Dm]Llénanos con tu |\[Am]luz, |\[E7]llena nuestros cora|\[Am]zon\[\up{1}A7]es \up{2}(| \e)
  \endchorus
  \begin{translation}
    Let the light bloom, let the light bloom
    \nextverse
    the light of the sun, the light of love
    \nextverse
    Oh great spirit, great spirit of love
    \nextverse
    fill us with your light, fill our hearts
  \end{translation}
\endsong


\beginsong{Vuela con el Viento}[by={Ayla Schafer},tags={flying},ph={III}]
  \meter{6}{8}
  \beginchorus\memorize
    |\[\mnc{E}Em]Llé\[^\mn{E}]vame con tus |\[\mnc{D}D]alas \[^\mn{E}]de \[^\mn{D}]luz
    |\[C]águila |\[D]tráenos medicina
    del |\[Em]viento, del aire, las es|\[D]trellas, del sol
    |\[C]brillando, |\[D]guías mi camino
  \endchorus
  \notesoff
  \beginchorus
    |\[Em]Cura, cura, cura, cúrame, |\[D]sana todo lo que yo llevo
    |\[C]agradesco por mi vida Pa|\[D]chamama yo te amo
  \endchorus
  \beginchorus
    |\[Em] Vuela con el |\[D]viento, |\[C] vuela con el |\[D]viento
  \endchorus
  \beginchorus
    |^Llévame con tus |^alas de amor
    |^condorcito |^tráenos medicina
    del |^cielo, ilumina |^mi interior
    |^volando ens|^éñame el camino
  \endchorus
  \goto{Cura, cura, cura}
  \goto{Vuela con el viento}
  \begin{translation}
    Carry me with your wings of light, eagle bring us the medicine of the
    wind, of the air, of the stars, of the sun, shining you guide my way.
    \nextverse
    Cure, cure, cure me, heal everything I carry.
    Giving gratitude for my life, Mother Earth I love you.
    \nextverse
    Fly with the wind, fly with the wind.
    \nextverse
    Carry me with your wings of love, Condor bring us the medicine of
    the sky, illuminate my interior, flying you show me the way.
  \end{translation}
\endsong


\beginsong{Abuelita Ayahuasquita}[tags={Aya, flying},ph={III}]
  \meter{6}{8}
  \beginchorus
    |\[G] \[\mn{G}]Ab\[\mn{D}]uelita A|\[D]yahuasquita en|\[C]séñame a vo|\[G]lar
    |\[G] Como paja|\[D]ritos que vuelan |\[C]libres en la inmensi|\[G]dad
    Cura |\[A]cura cura |\[A7]cura mi cora|\[D]zón |\[\up{2}(D7)]-
  \endchorus
  \beginchorus
    Y vo|\[C]lando encontra|\[G]ré paz y liber|\[D]tad | \e
    Limpia y |\[C]cambia mis tris|\[G]tezas a color |\[A]verde esperan|\[D]za | \e
  \endchorus
  \beginverse
    Vuela |\[C]vuela pa\[D]lomi|\[G]ta vuela |\[C]vuela con A\[D]ya|\[G]huasca
    Limpia y |\[C]cura mis \[D]triste|\[G]zas a co|\[C]lor verde es\[D]peran|\[G]za |\[D] |\[C] |\[G]
  \endverse
  \beginverse
    |\[C] Lara lara la|\[G]irai la|\[D]raira | \e
    |\[C] Lara lara la|\[G]irai rai
    |\[C] Lara lara la|\[G]irai rai
    |\[C] Lara lara la|\[G]irai la|\[D]raira | \e
  \endverse
  \begin{translation}
    Grandma beloved Ayahuasca teach me to fly
    Like little birds flying free in the immensity
    Heal, heal, heal, heal my heart
    \nextverse
    And flying I will find peace and freedom
    Cleanse and change my sadness to green hope
    \nextverse
    Fly, fly, little dove, fly, fly with Ayahuasca
    Cleanse and heal my sadness to green hope
  \end{translation}
\endsong


\beginsong{Pachamama Pachacamaq}[by={Kike Pinto},tags={Mother Earth, thankfulness, you},ph={III}]
  \meter{3}{4}
  % this song uses two sets of memorized chords. so initialize chord registers here:
  \newchords{chords_pachamamacamaq_a}\newchords{chords_pachamamacamaq_b}
  \beginverse\memorize[chords_pachamamacamaq_a]
    \[^\mn{B}]Pac\[^\mn{D}]ha|\[\mnc{E}C]mama |Pacha|camaq | luz de |vida, |\[Fmaj7]luz de a|\[C]mor | \e
    te agra|\[Am]dezco |\[Em]por mi |\[Am]vida | por mi a|\[F]liento y |por mi |\[Em]voz | \e
    te agra|\[Am]dezco |\[Em]por mis |\[Am]sueños | y el la|\[F]tir de un |cora|\[Em]zón
    | | | \e
  \endverse
  \notesoff
  \beginverse\replay[chords_pachamamacamaq_a]
    Por el |^sol de |cada |día | por el |agua y |^su sa|^bor | \e
    por el |^aire y |^por el |^viento | por el |^fuego y |su ca|^lor | \e
    por mi |^gente y |^por mi |^pueblo | y el can|^tar de es|ta can|^ción
    | | | \e
  \endverse
  \beginverse\memorize[chords_pachamamacamaq_b]
    \ind Por mi |\[F]tierra y |mi fa|\[C]milia, | porque |\[F]son par|te de |\[C]mi | \e
    \ind y yo |\[Am]siendo |\[Em]parte de |\[Am]ellos | puedo |\[F]ser par|te de |\[Em]ti | \e
    \ind y yo |\[Am]siendo |\[Em]parte de |\[Am]ellos | puedo |\[F]ser par|te de |\[Em]ti
    \ind | | | \e
  \endverse
  \beginverse\replay[chords_pachamamacamaq_b]
    \ind Palo|^mita |de mi |^vida, | palo|^mita |de mi a|^mor | \e
    \ind así |^quiso |^Pacha|^camaq | que nos |^junte|mos los |^dos | \e
    \ind y así |^quiere |^Pacha|^mama | que nos |^ame|mos los |^dos
    \ind | | | \e
  \endverse
  \brk
  \textnote{suomeksi:} % by: Malla
  \beginverse\replay[chords_pachamamacamaq_a]
    Äiti |^Maa, |Kaiken |Luoja, | lähde |valon, |^rakkau|^den | \e
    kii|^tän |^elä|^mästäin, | henges|^täin ja |äänes|^täin, | \e
    kii|^tän |^unel|^mistain, | syk|^keestä |sydä|^men | | | \e
  \endverse
  \beginverse\replay[chords_pachamamacamaq_a]
    Kii|^tän |aurin|gosta, | veden |raik|^kaudes|^ta | \e
    il|^masta |^ja tuu|^lesta, | läm|^möstä |liekki|^en | \e
    hei|^mosta |^ja ko|^dista, | lau|^lusta |sydä|^men | | | \e
  \endverse
  \beginverse\replay[chords_pachamamacamaq_b]
    \ind Kiitän |^maasta |ja per|^heestä, | sillä |^ne on |osa |^mua | \e
    \ind ja kos|^ka oon |^osa |^heitä, | voin myös |^olla |osa |^sua | \e
    \ind ja kos|^ka oon |^osa |^heitä, | voin myös |^olla |osa |^sua | | | \e
  \endverse
  \beginverse\replay[chords_pachamamacamaq_b]
    \ind E|^lä|mäni |^lintu, | rakkau|^teni |kyyhky|^nen, | \e
    \ind niin |^tahtoi |^Kaiken |^Luoja, | että |^yh|distym|^me | \e
    \ind ja niin |^tahtoo |^Äiti |^Maa, | että |^ra|kastam|^me | | | \e
  \endverse
  % % comment out English translation now that we have a Finnish version
  % \begin{translation}
  %   Pachamama Pachacamaq, light of life, light of love
  %   I thank you for my life, for my breath and for my voice
  %   I thank you for my dreams and the heartbeat
  %   \nextverse
  %   For the sun, for each day, for the water and its flavor
  %   For the air and for the wind, for the fire and its heat
  %   For my people (\emph{of the village}) and for the people and the singing of this song
  %   \nextverse
  %   For my land and my family, because they are part of me
  %   And by being part of them I can be part of you
  %   And by being part of them I can be part of you
  %   \nextverse
  %   Little dove of my life, little dove of my love
  %   So Pachacamaq wanted the two of us to join
  %   And so Pachamama wants us to love each other
  % \end{translation}
  \begin{explanation}
    \begin{description}
      \item[Pachamama:] the Goddess of the Earth in \emph{Inca} mythology, ``World Mother'',
        worshipped by indigenous people of the Andes
      \item[Pachacamaq:] a deity responsible for creating the first humans, originally worshipped
        in the \emph{Ichma} culture (which was later absorbed to the Incan empire)
    \end{description}
  \end{explanation}
\endsong


\beginsong{Machi}[by={Peia},ph={III}]
  \audio[pitch={432}]{https://www.youtube.com/watch?v=D7os9V-n7rs}
  \musicnote{Intro and outro repetitions are in 6/8, but the main part of the song is in 4/4.}
  \meter{6}{8}
  \beginchorus
    \ind |\[\mnc{A}F#m]Ma\[\mn{B}]chi |\[\mnc{C#}A]machi |\[E] \[\mn{B}]ma\[\mn{C#}]chi|\[\mnc{F#}F#m]ma \up{2}(| \e)
  \endchorus
  \beginverse
    |\[D] Machi |\[F#m]cura |\[D] Machi |\[F#m]sana
    |\[D] Machi |\[F#m]cántame |\[D]una |\[F#m]nana | \e
  \endverse
  \goto{Machi machi machima}
  \beginverse
    |^ Yo no |^lloro |^ yo sólo |^canto
    |^ Con tu en|^canta |^Pachamama |^Madre Tierra
  \endverse
  \goto{Machi machi machima}
  \begin{translation}
    Machi is curing, Machi is healing
    Machi sings me a lullaby
    \nextverse
    I do not cry, I just sing
    With your love Pachamama Mother Earth
  \end{translation}
  \begin{explanation}
    A \textbf{machi} is a traditional healer and religious leader in the Mapuche culture
    of Chile and Argentina, usually a woman.
  \end{explanation}
\endsong


\beginsong{Luz del Bosque\\I am the Light of the Forest}[by={Adrian Freedman},tags={light},ph={III, IV}]
  % This song is apparently originally in English and by Adrian Freedman.
  \audio[]{https://soundcloud.com/adrianfreedman/i-am-the-light-of-the-forest-la-luz-del-bosque}
  % See (Issa Elle's Spanish version): https://soundcloud.com/issa-elle-1/la-luz-del-bosque
  \newchords{chords_luzdelbosque_a}\newchords{chords_luzdelbosque_b}
  \beginchorus\memorize[chords_luzdelbosque_a]
    |\[\mnc{C}C]Somos \[^\mn{D}]la \[^\mn{E}]luz \[^\mn{C}]del |\[\mnc{G}G]bosque, e|\[Dm]spíritu de todas e|\[Am]dades
  \endchorus\glueverses
  \notesoff
  \beginchorus\memorize[chords_luzdelbosque_b]
    |\[Dm]Somos la luz di|\[C]vina, sabidu|\[G]ría de los |\[Am]mares
    \up{2}(| \e)
  \endchorus
  \beginchorus\replay[chords_luzdelbosque_a]
    Transfor|^mamos el dol|^or tray|^endo todo a la l|^uz
  \endchorus\glueverses
  \beginchorus\replay[chords_luzdelbosque_b]
    Con el e|^spíritu de mis \up{*}an|^cestros todo el |^dia y la |^noche
    \up{2}(| \e)
  \endchorus
  \altlyr{abuelitos}
  \textnote{suomeksi:}
  \beginchorus\replay[chords_luzdelbosque_a]
    |^Olemme valo |^metsän, |^henki kaikkien aiko|^jen
  \endchorus\glueverses
  \beginchorus\replay[chords_luzdelbosque_b]
    |^Olemme pyhä |^valo, |^viisaus meri|^en
    \up{2}(| \e)
  \endchorus
  \beginchorus\replay[chords_luzdelbosque_a]
    Muun|^namme kärsi|^myksen |^tuomalla kaiken val|^oon
  \endchorus\glueverses
  \beginchorus\replay[chords_luzdelbosque_b]
    \up{*}Esi|^vanhempain hengen |^kanssa koko |^päivän ja koko |^yön
    \up{2}(| \e)
  \endchorus
  \altlyr{Isovanhempain}
  %% Translation commented out for saving space, since we have the Finnish version
  % \begin{translation}
  %   We are the light of the forest, spirit of all ages
  %   \nextverse
  %   We are the divine light, wisdom of the seas
  %   \nextverse
  %   We transform the pain by bringing everything to light
  %   \nextverse
  %   With the spirit of the grandmother singing all night
  %   With the help of the mother singing all night
  %   \nextverse
  %   With the spirit of the grandfather singing all night
  %   With the help of the father singing all night
  % \end{translation}
\endsong


\beginsong{La Semilla}[by={Shimshai},tags={love, heart},ph={III}]
  \meter{6}{8}
  \newchords{chords_lasemilla_a}\newchords{chords_lasemilla_b}
  \beginverse\memorize[chords_lasemilla_a] % memorize chords into a named register
    \lrep \[^\mn{A}]En |\[\mnc{D}Dm]el pre\[A7]sente en|\[Dm]contrarás,
    en |\[Fmaj7]el cora\[C]zón tu ve|\[Dm]rás \rrep
    \lrep Este a|\[Fmaj7]mor que \[C]fluye bi|\[Dm]en adentro,
    a|\[Fmaj7]quí hay \[C]una se|\[Dm]milla \rrep
    \vspace{1em}
    \lrep \up{2}(La se|\[(Dm)]milla) ha \[A7]sido sem|\[Dm]brada en ti
    y el a|\[Fmaj7]mor es el \[C]agua que |\[Dm]la alimenta \rrep
  \endverse
  \notesoff
  \beginverse\memorize[chords_lasemilla_b] % memorize chords into a named register
    \ind \lrep Con a|\[F]mor e\[C]ternamen|\[Dm]te crecerá,
    \ind mi |\[F]madre lo \[C]hace a|\[Dm]sí \rrep
    \ind Con a|\[F]mor e\[C]ternamen|\[Dm]te crecerá,
    \ind mi |\[F]padre lo \[C]hace a|\[Dm]sí
    \ind Con a|\[F]mor e\[C]ternamen|\[Dm]te crecerá,
    \ind la |\[F]fuerza me \[C]hace a|\[Dm]sí \[A7]
  \endverse
  \brk
  \textnote{in English:}
  \beginverse\replay[chords_lasemilla_a] % replay chords from a named register
    \lrep |^Be in the ^now you will |^find,
    and |^be in the ^heart you will |^see \rrep
    \lrep This |^love that ^flows so |^deep inside,
    with|^in there ^lies a |^seed \rrep
    \vspace{1em}
    \lrep \up{2}(A |^seed) has been ^planted in|^side of your heart
    and |^love is the ^water that |^feeds \rrep
  \endverse
  \beginverse\replay[chords_lasemilla_b] % replay chords from a named register
    \ind \lrep With |^love e^ternal|^ly it will grow,
    \ind my |^mother she ^makes it |^so \rrep
    \ind With |^love e^ternal|^ly it will grow,
    \ind my |^father he ^makes it |^so
    \ind With |^love e^ternal|^ly it will grow,
    \ind the |^force she ^makes me |^so \[(A7)]
  \endverse
\endsong


\beginsong{Oso Blanco}[by={Bóveda Celeste},ph={III}]
  \audio[key=Bm]{https://soundcloud.com/bovedaceleste-1/oso-blanco-arnaldo-andressa}
  %% alt. chords: C -> G -> Em -> Am
  % \capo{2}
  \beginchorus\memorize
    \[^\mn{G}]En este \[^\mn{E}]ca|\[\mncii{G}{E}Am]mi|no rumbo a la |\[\mncii{A}{C}Fmaj7]selva | \e
    me he encon|\[G]trado un pajar|illo sona|\[F]jero | \e
    que inspi|\[Am]rado en el vuelo del |Águila encon|\[Fmaj7]{tró en su} cora|zón
    este |\[G]canto que brota del |Alma aquí y a|\[F]hora | \e
  \endchorus
  \notesoff
  \beginchorus
    Paja|^rillo sona|jero encon|^traste en la Visión | \e
    a un Oso |^blanco con alas de |Cóndor y mi|^rada de Jagu|ar
  \endchorus
  \beginchorus
    Entre|^lazas marirí | las plumas del |^Águila y del C|óndor
    donde |^llueven las gotas de |cielo que hoy fe|^cundan la Se|milla \up{2}(| \e)
    % 'llueven' (rain) or 'caen' (fall)?
  \endchorus
  \textnote{\emph{D.C. al Fine}}
  \beginchorus
    En este ca|^mi|no rumbo a la |^selva | \e
    me he encon|^trado un coraz|ón sonaj|^ero | \e
  \endchorus
  \begin{translation}
    On this road to the forest
    I have found a little bird rattle
    that inspired, by the flight of the Eagle found in its heart,
    this song that springs from the Soul here and now.
    \nextverse
    Little bird rattle, found in the Vision
    of a White Bear with wings of Condor and the gaze of Jaguar.
    \nextverse
    Intertwine \emph{marirí}, the feathers of the Eagle and the Condor,
    where the drops of sky rain that today fertilize the Seed.
    \nextverse
    On this road to the forest
    I have found a heart rattle.
  \end{translation}
  \begin{explanation}
    \begin{description}
      \item[Marirí:] see song \emph{Marirí}
    \end{description}
  \end{explanation}
\endsong


\beginsong{Nadi Wewe}[by={Kuitzi Moezzi},tags={Aya},ph={III}]
  \beginverse
    \[^\mn{A}]Santa |\[Am]madre A\[^\mn{C}]ya\[^\mn{A}]huas|\[\mncii{G}{E}C]quita
    Per|\[Em]mítame recibir|\[Am]te
  \endverse
  \notesoff
  \beginchorus
    \ind Nadi |\[Dm]wewe nadi |\[Am]wewe
    \ind Nadi |\[Em]wewe nadi |\[Am]we
  \endchorus
  \beginverse
    Aya|^huasca curande|^rita
    En|^séñame a curar|^me \goto{Nadi wewe}
  \endverse
  \beginverse
    Sabia |^madre Ayahuas|^quita
    Apr|^enderé tenerte |^fe \goto{Nadi wewe}
  \endverse
  \beginverse
    Aya|^huasca visio|^naria
    Áb|^reme la conscien|^cia \goto{Nadi wewe}
  \endverse
  \beginverse
    Aya|^huasca enlaza|^dora
    Mis rela|^ciones las cuida|^ré \goto{Nadi wewe}
  \endverse
  \beginverse
    Amo|^rosa madre Aya|^huasca
    Mi cora|^zón entrega|^ré \goto{Nadi wewe}
  \endverse
  \beginverse
    Bendita |^madre Aya|^huasca
    Tu bendi|^ción comparti|^ré \goto{Nadi wewe}
  \endverse
  \begin{translation}
    Holy mother Ayahuasca, allow me to receive you
    \nextverse
    Ayahuasca, healer, teach me to heal myself
    \nextverse
    Wise mother Ayahuasca, I will learn to have faith
    \nextverse
    Ayahuasca, visionary, open my conscience
    \nextverse
    Ayahuasca, uniter, look after my relationships
    \nextverse
    Loving mother Ayahuasca, my heart will surrender
    \nextverse
    Blessed mother Ayahuasca, your blessing I will share
  \end{translation}
\endsong


\beginsong{Sirenita Bobinzana}[by={Artur Mena},ph={III, IV}]
  \newchords{chords_sirenita_a}\newchords{chords_sirenita_b}
  \meter{3}{4}
  \beginchorus\memorize[chords_sirenita_a]
    \[^\mn{E}]Si\[^\mn{D}]re|\[\mnc{E}C]nita \[^\mn{G}]de \[^\mn{E}]los |\[\mnc{D}G]ríos, \[^\mn{G}]dan\[^\mn{A}]za |\[\mnc{B}B7]danza \[^\mn{A}]con \[^\mn{B}]el |\[\mnc{E}Em]viento | \e
  \endchorus\glueverses\beginchorus\memorize[chords_sirenita_b]
    Con  tus |\[G]flores y a|romas, per|fumas los \[B7]cora|\[Em]zones | \e
  \endchorus
  \notesoff
  \beginchorus\replay[chords_sirenita_a]
    Cura |^cura cuerpe|^citos, limpia |^limpia espiri|^titos | \e
  \endchorus\glueverses\beginverse\replay[chords_sirenita_b]
    Canta|^remos ica|ritos abue|lita cu^rande|^ra | \replay \e
    Danza|^remos muy jun|titos sire|nita ^Bobin|^zana | \e
  \endverse
  \beginchorus\replay[chords_sirenita_a]
    Raira |^rairai raira |^rairai raira |^rairai raira |^rairai
  \endchorus\glueverses\beginchorus\replay[chords_sirenita_b]
    Raira |^ra-irai raira |ra-irai raira |ra-irai ^raira|^rairai
  \endchorus
  \beginchorus\replay[chords_sirenita_a]
    Medi|^cina de la |^selva, eres |^tu Bobin|^zana | \e
  \endchorus\glueverses\beginchorus\replay[chords_sirenita_b]
    Curas |^males das visi|ones a tus |hijos ^en las |^dietas \e
  \endchorus
  \beginchorus\replay[chords_sirenita_a]
    Cura |^cura cuerpe|^citos, limpia |^limpia espiri|^titos | \e
  \endchorus\glueverses\beginverse\replay[chords_sirenita_b]
    Canta|^remos ica|ritos en sesion|cita de ^aya|^huasca | \replay \e
    Danza|^remos muy jun|titos sire|nita ^Bobin|^zana | \e
  \endverse
  \begin{translation}
    Little princess of the rivers, dance dance with the wind
    With your flowers and aromas, you perfume the hearts
    \nextverse
    Heal heal our little bodies; cleanse cleanse our spirits
    We will sing icaros; Grandmother, the healer
    We will dance much together, princess Bobinzana.
    \nextverse
    Jungle medicine are you, Bobinzana
    You cure ill visions in your dieting children
    \nextverse
    Heal heal our little bodies; cleanse cleanse our spirits
    We will sing icaros in ayahuasca sessions
    We will dance much together, princess Bobinzana.
  \end{translation}
  \brk
  \begin{explanation}
    \begin{description}
      \item[Bobinzana:] \emph{Calliandra angustifolia}, an Amazonian tree with healing properties, a plant teacher
    \end{description}
  \end{explanation}
\endsong


\beginsong{Agua de Estrellas}[by={Miguel Molina},tags={stars, Mother Earth},ph={III, IV}]
  \transpose{7}
  \beginchorus
    \[\mn{A}]En \[\mn{C}]tus |\[\mnc{D}Dm]ojos de agua \[\mn{C}]in\[\mn{A}]fi|\[\mnc{C}F]ni\[\mn{A}]ta
    Se |\[C]bañan las estrel|\[G]litas \[C]ma|\[Dm]ma | \e
  \endchorus
  \notesoff
  \beginchorus
    Agua de |\[F]luz agua de es|\[Am]trellas
    Pacha|\[G]mama vie\[C]nes del |\[Dm]cielo | \e
  \endchorus
  \beginchorus\memorize
    |\[Dm]Limpia |limpia
    |\[F]Limpia corazón |\[C]agua bril\[G]lante
     \replay |^Sana |sana
    |^Sana corazón |^agua ben^dita
     \replay |^Calma |calma
    |^Calma corazón |^agua del ^cielo ma|\[Dm]ma | \e
  \endchorus
  \begin{translation}
    In your eyes of eternal water
    the stars bathe, mother
    \nextverse
    Water of light, water of the stars
    Pachamama comes from the sky
    \nextverse
    Cleanse, cleanse, cleanse the heart, brilliant water
    Heal, heal, heal the heart, blessed water
    Calm, calm, calm the heart, water of the sky, mother
  \end{translation}
\endsong


\beginsong{Tzen Tze Re Rei}[by={Nase Chiriap},ex={shuar, español},ph={III, IV}]
  \audio[]{https://soundcloud.com/loli-cosmica/tzen-tze-re-re}
  \audio[]{https://www.youtube.com/watch?v=vlGsRDuLUe0}
  \audio[]{https://www.youtube.com/watch?v=PC26I9uFW\_Q}
  \beginverse
    |\[\mnc{E}Am]Y es \[\mn{G}]a|\[\mn{E}]quí \[\mn{G}]donde |\[\mnc{E}C]quiero es\[\mn{D}]tar |\[\mn{C}]junto a \[\mn{D}]ti
  \endverse
  \beginverse
    \ind |\[\mn{E}]Rei \[\mn{D}]rei |\[\mn{C}]rei
    \ind \up{¤}(Tzen tze re|\[C]rei rei |rei) \altlyr[¤]{\emph{Skip this line if coming from} ``\ldots junto a ti''}
    % Loli Cosmica's version: The next line is omitted on (only) the very first time
    \ind Tzen tze re|rei rei |rei
    \ind Tzen tze re|rei rei rei |rei
    \ind Rei |rei | Rei |\[Am]rei | | | \e
  \endverse
  % \goto{Rei rei rei} % This repeat is in Loli Cosmica's version
  \beginverse\memorize
    De tus |\[Am]ojos |salen los co|\[C]lores
    |Del uni|verso |pedimos |\[Am]todo | | | \e
    Y |esos co|lores |\[C]son el ali|mento
    |Para la exis|tencia |que somos |\[Am]todo | | | \e
  \endverse
  \goto{Rei rei rei}
  \beginverse
    De tu |^voz |salen los so|^nidos
    |Del uni|verso |que oímos |^todo | | | \e
    Y |esos so|nidos |^son el ali|mento
    |Para la exis|tencia |que somos |^todo | | | \e
  \endverse
  \goto{Rei rei rei}
  \beginverse
    De tus |^manos |salen las ca|^ricias
    |Del uni|verso |que sentimos |^todo | | | \e
    Y |esas ca|ricias |^son el ali|mento
    |Para la exis|tencia |que somos |^todo | | | \e
  \endverse
  \goto{Rei rei rei}
  \goto{Y es aquí}
  \goto{Rei rei rei}
  \goto{Y es aquí}
  \goto{Rei rei rei}
  \begin{translation}
    And it’s here that I want to be together with you
    \nextverse
    From your eyes come the colours of the universe
    We ask for all and those colours
    Are the food for the existence that we are all
    \nextverse
    From your voice come the sounds of the universe
    We hear all and those sounds
    Are the food for the existence that we are all
    \nextverse
    From your hands come the caresses of the universe
    We feel all and those caresses
    Are the food for the existence that we are all
  \end{translation}
  \begin{explanation}
    \textbf{Tzen tze re rei} \emph{(Shuar language)} means the arrow that advances and opens doors.
    Also, it is an onomatopoeia for the chattering of a chinchilla, a dormouse or a squirrel.
  \end{explanation}
\endsong


\beginsong{Sólo Dios Sabe si Vuelvo}[by={Julian Herreros Riveira},ph={III, IV}]
  \transpose{5}
  \beginverse
    |\[\mnc{E}Am]A|'\[^\mn{D}]bre\[^\mn{E}]te \[^\mn{D}]flor|\[\mnc{C}F]ci\[^\mn{A}]ta de los |\[^\mn{C}]cua\[^\mn{A}]tro vien\[^\mn{G}]tos |\[\mnc{E}C]yari | \e
    Oloro|cita perfu|\[G]mera doctor|\[Am]cita | \e
    Estrelli|ta de siete |\[Em]flechas Tonān|\[Am]tzina | \e
    Ay danza|\[C]remos hasta |\[G]las claras del |\[Am]día | \e
    para escu|char la voz de |\[Em]los que ya se |\[Am]fueron | \e
  \endverse
  \notesoff
  \beginverse
    \ind |^Naa|a na na na |^naana na na |naina na na |^naana | \e
    \ind Na na na |naana na na |^naina na na |^naana | \e
    \ind Na na na |naana na na |^naina na na |^naana | \e
  \endverse
  \beginverse
    |^So|'n tus aguas |^que le dan la |vida a mi |^pueblo | \e
    Ay las que |brotan por mis |^ojos triste|^mente | \e
    porque en mi |tierra ya no |^se le canta al |^agua | \e
    Son las que |^corren por tus |^ojos Pacha|^mama | \e
    y rever|decen en mi |^quebrada de a|^mor | \e \goto{Naa na na na}
  \endverse
  \beginverse
    |^Lle|'no de ale|^gría danza |la muerte a mi |^lado | \e
    Chuma bo|rracherita |^pinta gente |^yari | \e
    Cuatro oto|rongos magos |^llegan a la |^fiesta | \e
    Se alza en el |^cielo el brillo a|^zul de la flor |^blanca | \e
    Camino |rojo sólo |^Dios sabe si |^vuelvo | \e \goto{Naa na na na}
  \endverse
  \begin{translation}
    Open flower of the four winds \emph{yari}
    Dear smell perfumes the healer
    Little star of seven arrows, Mother Goddess
    Oh we will dance until the clear of the day
    to hear the voice of those who have already left
    \nextverse
    It's your waters that give life to my people
    Oh the ones that sprout sadly through my eyes
    because in my land water is no longer sung to
    They are the ones that run through your eyes Pachamama
    and revive in my stream of love
    \nextverse
    Full of joy dance, death by my side
    Drunken folly paints people \emph{yari}
    Four jaguars, wizards arrive at the party
    The blue glow of the white flower rises in the sky
    Red path, only God knows if I return
  \end{translation}
\endsong

\scleardpage
\beginsong{En la Selva un Río}[by={Kike Pinto},ph={III, IV}]
  \normalsize % make smaller to fit all verses on one spread
  \newchords{chords_enlaselva_a}\newchords{chords_enlaselva_b}
  \meter{6}{8}
  \beginchorus\memorize[chords_enlaselva_a]
    \[^\mn{E}]En |\[\mnc{E}Em]la sel\[^\mn{D}]va un |\[^\mn{B}]río en noche os|\[G]cura | \e
    |\[Am] va surcan|\[Bm]do una ca|\[Em]noa. | \e
  \endchorus\glueverses
  \beginverse\memorize[chords_enlaselva_b]
    la |\[Em]vigilan des|de la espe|\[G]sura, | \e
    |\[Am] los jagua|\[Bm]res y las |\[Em]boas. | \e
    la |\[Em]protegen des|de la espe|\[G]sura, | \e
    |\[Am] los jagua|\[Bm]res y las |\[Em]boas.
    \ind |Trai nanai na|\[Bm]nai na|nai |\[Em]nai Nanai nanai |\[G]nai
    \ind |\[Am] va surcan|\[Bm]do una ca|\[Em]noa
    \ind |Trai nanai na|\[Bm]nai na|nai |\[Em]nai Nanai nanai |\[G]nai
    \ind |\[Am] mil jagua|\[Bm]res y mil |\[Em]boas | |\[Em]Trai nanai na |\ldots | | | | | | \e
  \endverse
  \notesoff
  \beginchorus\replay[chords_enlaselva_a]
    U|^na voz me can|ta en el o|^ído | \e
    |^ una bel|^la melo|^día, | \e
  \endchorus\glueverses
  \beginverse\replay[chords_enlaselva_b]
    y u|^na luz dibu|ja ante mis |^ojos | \e
    |^ visiones de |^un nuevo |^día, | \e
    y u|^na luz dibu|ja ante tus |^ojos | \e
    |^ visiones de |^un nuevo |^día.
    \ind |Trai nanai na|^nai na|nai |^nai Nanai nanai |^nai
    \ind |^ una bel|^la melo|^día
    \ind |Trai nanai na|^nai na|nai |^nai Nanai nanai |^nai
    \ind |^ visiones de |^un nuevo |^día | |^Trai nanai na |\ldots | | | | | | \e
  \endverse
  \beginchorus\replay[chords_enlaselva_a]
    El |^viento me tra|e un dulce a|^roma | \e
    |^ alcanfo|^res y azu|^cenas, | \e
  \endchorus\glueverses
  \beginverse\replay[chords_enlaselva_b]
    y a|^lza el vuelo en mi al|ma una pa|^loma | \e
    |^ rompe |^rompe las ca|^denas, | \e
    y a|^lza el vuelo en mi al|ma una pa|^loma | \e
    |^ rompe |^rompe las ca|^denas.
    \ind |Trai nanai na|^nai na|nai |^nai Nanai nanai |^nai
    \ind |^ alcanfo|^res y azu|^cenas
    \ind |Trai nanai na|^nai na|nai |^nai Nanai nanai |^nai
    \ind |^ rompe |^rompe las ca|^denas | |^Trai nanai na |\ldots | | | | | | \e
  \endverse
  \beginchorus\replay[chords_enlaselva_a]
    Re|^ma rema re|ma cano|^ero | \e
    |^ por los |^ríos de mis |^venas. | \e
  \endchorus\glueverses
  \beginverse\replay[chords_enlaselva_b]
    can|^ta canta can|ta curan|^dero, | \e
    |^ cura cu|^rame mis |^penas. | \e
    so|^pla sopla so|pla perfu|^mero, | \e
    |^ rompe |^rompe mis ca|^denas.
    \ind |Trai nanai na|^nai na|nai |^nai Nanai nanai |^nai
    \ind |^ alcanfo|^res y azu|^cenas
    \ind |Trai nanai na|^nai na|nai |^nai Nanai nanai |^nai
    \ind |^ por los ri|^os de mis |^venas | |^Trai nanai na |\ldots | | | | | | \e
  \endverse
  \begin{translation}
    In the jungle on a river on a dark night rides a canoe. (x2)
    They watch it from the thicket, the jaguars and the boas.
    They protect it from the thicket, the jaguars and the boas.
    Trai nanai nanai nanai nai Nanai nanai nai, rides a canoe;
    Trai nanai nanai nanai nai Nanai nanai nai, a thousand jaguars and a thousand boas
    \nextverse
    A voice sings in my ear a beautiful melody, (x2)
    and a light draws before my eyes visions of a new day,
    and a light draws before my eyes visions of a new day.
    Trai nanai nanai nanai nai Nanai nanai nai, a beautiful melody;
    Trai nanai nanai nanai nai Nanai nanainai visions of a new day.
    \nextverse
    The wind brings a sweet aroma of camphora trees and lilies, (x2)
    and in my soul a dove rises to fly, breaking breaking the chains,
    and in my soul a dove rises to fly, breaking breaking the chains.
    Trai nanai nanai nanai nai Nanai nanai nai, camphora trees and lilies;
    Trai nanai nanai nanainai nanai nanainai, breaking the chains.
    \nextverse
    Row row row, canoteer, by the rivers of my veins. (x2)
    Sing sing sing, curandero, heal heal my pains.
    Blob blow blow, perfumer, break break my chains.
    Trai nanai nanai nanai nai Nanai nanai nai, camphores and lilies;
    Trai nanai nanai nanai nai nanai nanai nai, by the rivers of my veins.
  \end{translation}
  % Image downloaded from: https://openclipart.org/detail/120811/jaguar
  % Image license: in the public domain
  \imagecc[2]{jaguar_drawing_bw_transparent_bg_PD__1081px.png}%
\endsong


\beginsong{Corazón es lo Único que Tengo}[by={Nubia Rodriguez},tags={heart}, ph={III, IV}]
  %\transpose{5}
  \meter{6}{8}
  \beginchorus\memorize
    Gran E|\[\mnc{E}Em]spíritu, gran a|buela, gran a|\[\mnc{D}G]bu\[^\mn{B}]elo | \e
    Como |\[D]soy me pre|sento ante |\[Em]ti | \e
    Como |\[G]soy te pi|do bendi|\[Bm]cio|nes
    Y agra|\[D]dezco el cora|zón que has puesto en |\[Em]mi | \e
  \endchorus
  \notesoff
  \beginchorus
    Cuando |^vengo nomás |vengo, nomás |^vengo | \e
    Gran E|^spíritu sab|rás a lo que |^vengo | \e
    A entre|^gar mi cora|zón, mi cora|^zón | \e
    Cora|^zón que es lo |único que |^tengo | \e
  \endchorus
  \beginchorus
    Cora|^zón que es lo |único que |^tengo | \e
    Cora|^zón que es lo |único que |^tengo | \e
  \endchorus
  \begin{translation}
    Great Spirit, great grandmother, great grandfather
    As I am, I present myself to you
    As I am I ask you blessings
    And I thank for the heart that you have placed in me
    \nextverse
    When I come here I just come, I just come
    Great Spirit you will know why I am here
    To give my heart, my heart
    Heart that is all I have
    \nextverse
    Heart that is all I have
    Heart that is all I have
  \end{translation}
\endsong


\beginsong{Cura Sana}[by={Andrés Córdoba},ex={cofan, quechua, español},ph={III, IV}]
  \beginverse
    \[^\mn{E}]{O\ldots}|\[\mnc{A}Am]coremajaï jojore |\[F]jasparuñaï
    curaima|\[C]gente y pinta selva tierra |\[E]urumanyaï | \e
    I|\[Am]quisimanyairo amor |\[F]azul cielo
    celeste |\[C]azulejero que vie|\[E]ne y pinta | \e
  \endverse
  \notesoff
  \beginverse
    \ind |\[Am]Cura cura, |\[F]sana sana: |\[C]taita celoc |\[E]be licencialc | \e
    \ind |\[Am]Cura cura, |\[F]sana sana: |\[C]taita celoc |\[E]ashlepay | \e
  \endverse
  \beginverse
    I|^quisimanyairo uru|^maña y gente
    cordi|^llera montaña que vie|^ne y llegá | \e
    Ya|^ge yage pinta pinta |^azulejero,
    chuma |^rasca borrachera que |^viene y pinta | \e
  \endverse
  \goto{Cura cura}
  \beginverse
    O|^coremajaï jojore |^tacro yocro-
    manje|^ro pinta selva tierra |^urumanyaï | \e
    I|^quisimanyairo amor |^azul cielo
    celeste |^azulejero que vie|^ne y pinta | \e
  \endverse
  \goto{Cura cura}
  \beginverse
    O|^coremajaï jojore |^jasparuñaï
    curaima|^gente y pinta amor cielo |^tierra y vida
    |^Astro centro mundo cos|^mos seres
    interplane|^tario que llegá |^cura y sana
  \endverse
  \goto{Cura cura}
\endsong


\beginsong{Pura Medicina}[by={Mallkikuna},ph={IV}]
  \beginverse
    \[^\mn{E}]Pu\[^\mn{D}]ra \[^\mn{E}]me\[^\mn{D}]di|\[\mnc{E}Am]cina tus ojitos, | pura medi|cina el co|\[C]lor
    Pura medi|\[C]cina toda vida, | cuando sonri|\[G]e tu interi|\[Am]or
  \endverse
  \notesoff
  \beginverse
    Pura medi|^cina Taita Inti, | pura medi|cina Padre |^Sol
    Pura medi|^cina toda vida, | luz brillando |^de tu interi|^or
  \endverse
  \beginverse
    Pura medi|^cina la familia, | pura medi|cina sus ma|^nitos
    Pura medi|^cina la sonrisa, | juntos alum|^bramos esta |^vida
  \endverse
  \begin{translation}
    Pure medicine your eyes, pure medicine the color
    Pure medicine all life, when you smile inside
    \nextverse
    Pure medicine Taita Inti, pure medicine Father Sun
    Pure medicine all life, light shining from your interior
    \nextverse
    Pure medicine the family, pure medicine their little hands
    Pure medicine the smile, together we light this life
  \end{translation}
\endsong


\scleardpage
\beginsong{Guacamayo}[by={Danit Treubig},ph={IV}]
  \audio[key=C\#m]{https://danit.bandcamp.com/album/aliento}
  \audio[key=C\#m]{https://www.youtube.com/watch?v=4s3uheDMRl0}
  \audio[key=C\#m]{https://www.youtube.com/watch?v=oSgKCfJhWbM} % live
  \capo{4}
  \mnbeginchorus
    \[\mn{A}]Te agra|\[Am]dezco |\[\mnc{F}Fmaj7&5]{ por llegar} a mi |\[\mnc{G}G]co\[\mn{A}]ra\[\mn{G}]zón | \e
    her|\[\mnc{F}F]mosa criatura del |\[\mnc{E}Am]vien\[\mn{C}]to | \e
    \[\mn{A}]Te agra|dezco |\[\mnc{F}Fmaj7&5]{ por volar} en mi |\[\mnc{G}G]in\[\mn{A}]ter\[\mn{G}]ior | \e
    Tus co|\[\mnc{E}Em]lores me llevan \[\mn{D}]a|\[\mncii{C}{A}Am]dentro | \e\sublyr{\up{2}(hey hey hey)}
  \mnendchorus
  \mnbeginchorus
    \ind |\[\mnc{G}G]Pinta pinta pinta con el |\[Em]movimie\[\mn{E}]nto
    \ind |\[\mnc{B}G]de tus plumas sale |\[\mnc{G}Em]una vibración
    \ind |\[\mnc{B}G]que me abraza e|\[\mnc{G}Em]ternamente
    \ind |\[\mnc{B}G]que me enseña e|\[\up{1}\mnc{G}Em\up{2}\mnc{A}(Am)]ternamente
  \mnendchorus\glueverses
  \beginverse
    \ind | | \e
  \endverse
  \beginchorus
    Te agra|\[Am]dezco |\[Fmaj7&5]{ por entregar} tu |\[G]amistad | \e
    |\[F]aa-ee-ee-ee-|\[Am]{eh-eh-eh-eeh} | \e
    Agrade|ciendo por |\[Fmaj7&5]{esta hermosa} |\[G]claridad | \e
    Por la con|\[Em]fianza y en la |\[Am]luz | \e\sublyr{\up{2}(hey hey hey)}
  \endchorus
  \goto{Pinta pinta pinta}
  \mnbeginchorus
    \lrep |\[\mnc{A}Am]{ Guacamayo} |Gu'acama\[\mn{G}]yo
    |\[\mnc{C}C]heeheehee \[\mn{G}]poder|\[\mn{C}]oso Guacamayo |\[\mnc{A}Am]hee \[\mn{C}\mn{G}]hey|\[\mn{A}]hee \rrep \rep{4}
    \vspace{1.5em}
    |\[\mnc{G}Em]En tus ojos |\[\mnc{A}B7]brilla una estrella |\[\mnc{B}G]flecha de luz a|\[\mnc{C}Am]briendo la pinta
    |\[\mnc{G}Em]Tu coro|\[\mnc{A}B7]na de luz |\[\mnc{B}G]danzando |\[\mnc{C}Am]{en el} viento
    |\[\mnc{G}Em]Iluminando |\[\mnc{A}B7]todo el espacio |\[\mnc{B}G]{ ay} curando mi |\[\mnc{C}Am]corazón
    |\[\mnc{G}Em]Iluminando |\[\mnc{A}B7]todo el espacio |\[\mnc{B}G]{ ay} curan\[\mn{G}]do |\[\mnc{A}Am]{la família}
    | ya hee ya |\[\mnc{E}C]hee | ya hee ya |\[\mnc{E}Am]hee-\[\mn{C}\mn{B}]{e-e-}\[\mn{C}\mn{A}]{e-ee} | \e
  \mnendchorus
  \beginchorus
    |\[Am]{ Guacamayo} |Gu'acamayo
    |\[C]heeheehee poder|oso Guacamayo |\[Am]hee hey|hee
    \rep{4}
  \endchorus
  \beginverse
    \lrep |\[\mnc{C}C]hee|y y\[\mn{A}]a |\[Am]hee | \e \rrep \rep{6}
    |\sublyr{hee-}\[C]{ Guacamayo} |\sublyr{y} vuela con\sublyr{ya}migo |\sublyr{hee}\[Am]{ Guacamayo} | \e
    |\sublyr{hee-}\[C]{ Guacamahito} pode|\sublyr{y}roso guaca\sublyr{ya}mayo |\sublyr{hee}\[Am]{ guacamahito} | \e
    \lrep |\sublyr{hee-}\[C]{ Guacamayo} |\sublyr{y}\hspace{1.5em}\sublyr{ya}\hspace{.5em} |\sublyr{hee}\[Am]{ Guacamayo} | \e \rrep
    \lrep |\sublyr{hee-}\[C]{ Guacamayo} |\sublyr{y}\hspace{1.5em}\sublyr{ya}\hspace{.5em} |\sublyr{hee}\[Am] | \e \rrep
  \endverse
  \begin{translation}
    I thank you for reaching my heart
    Beautiful creature of the wind
    I thank you for flying inside me
    Your colors take me inside
    \nextverse
    Paint paint paint with the movement
    of your feathers a vibration comes out
    that embraces me eternally
    that teaches me eternally
    \nextverse
    I thank you for giving your friendship
    Thanking for this beautiful clarity
    By trust and in the light
    \nextverse
    Macaw, powerful Macaw
    \nextverse
    In your eyes shines a star arrow of light opening the vision
    Your crown of light dancing in the wind
    Lighting up the whole space healing my heart
    Lighting up the whole space healing the family
    \nextverse
    Macaw, powerful Macaw
    \nextverse
    Macaw, fly with me macaw
    Little Macaw, powerful little Macaw
    Macaw\ldots
  \end{translation}
\endsong


\beginsong{Colibrí Dorado}[by={Arnaldo Herrera},ph={III}]
  \audio[key=Em]{https://soundcloud.com/arnaldo-herrera-3/colibri-dorado-andressa-arnaldo}
  \audio[key=Cm]{https://www.youtube.com/watch?v=JX44ZNy7ubs}
  \audio[key=Am]{https://www.youtube.com/watch?v=EaksDP2jCNg}
  \beginchorus
    \[\mn{A}]A nuestro \[\mn{B}]ho|\[\mnc{C}Am]gar \[\mn{B}]llegó \[\mn{A}]vol|an\[\mn{E}]do
    Un |\[G]Colibrí dor|ado
    que |\[Fmaj7]trajo a nuestro al|\[G]tar
    un nuevo a|\[Am]manecer | \e
  \endchorus
  \beginverse
    \[\mn{E}]La luz se a|\[\mnc{D}Em]cerca ref\[\mn{C}]le|\[\mnc{D}G]jando \[\mn{E}]la a\[\mn{D}]le|\[\mnciii{A}{C}{A}Am]gría | \e
    El viento s|\[Em]opla desper|\[G]tando un nuevo |\[Am]día | \e
    Y la fres|\[Em]cura en la mon|\[G]taña va lle|\[Am]vando | \e
    La escencia |\[G]de un Pica|\[Fmaj7]flor | \e
    que a|sciende enamo|\[Am]rado | \e
  \endverse
  \begin{translation}
    He came flying to our home
    A golden hummingbird
    who brought to our altar
    A new sunrise
    \nextverse
    The light approaches reflecting the joy
    The wind blows waking up a new day
    And the freshness in the mountain is taking
    The essence of a Hummingbird
    that rises in love
  \end{translation}
\endsong


\beginsong{Espíritu Eterno}[index={Cayaríriŕi},by={Darío Poletti},ph={III, IV}]
  \audio[key=Gm]{https://www.youtube.com/watch?v=tIZ0Gp_ChGw}
  \audio[key=Am]{https://www.youtube.com/watch?v=4IJxTvTm6hI}
  \capo{3}
  \mnbeginverse
    |\[\mnc{B}Em]Cuando la luna se a|\[^\mn{E}]soma ilumi|\[\mnc{D}G]nando la noche \[^\mn{B}]ni|\[^\mn{D}]may
    |\[\mnc{A}Bm]{Y las} estrellas se |\[^\mn{B}]ven en el firma|\[Em]men\[^\mn{E}]to | \e
    |\[^\mn{B}]Azul profundo que |\[^\mn{E}]nos envuelve en el |\[\mnc{F#}G]gran \[^\mn{D}]misterio \[^\mn{B}]ni|\[^\mn{D}]may
    |\[\mnc{A}Bm]{El misterio} de la |\[^\mn{B}]noche bien a|\[Em]dentro | \e
  \mnendverse
  \mnbeginchorus
    \[\mn{E}]Ca\[\mn{F#}]ya|\[\mnc{G}C]rí \[\mn{F#}]ri\[\mn{G}]rí \[\mn{F#}]ri|\[\mn{G}]rí \[\mn{F#}]ri\[\mn{G}]rí \[\mn{A}]rí|\[\mnc{G}G]rí \[\mn{D}]rirí \[\mn{B}]riri|\[\mn{D}]rí
    |\[\mnc{A}Bm]Es la presencia de |\[\mn{B}]tu espíritu e|\[Em]ter\[\mn{E}]no | \e
  \mnendchorus
  \notesoff
  \beginverse
    |^Son las raíces pro|fundas las que nos |^dan su secretos ni|may
    |^Medicina de la |tierra bien a|^dentro | \e
    |Abriendo nuestras con|sciencias abriendo |^corazones ni|may
    |^Medicina de los |Andes bien a|^dentro | \e
  \endverse
  \goto{Cayarí}
  \beginverse
    |^Padres, hijos, her|manos y madres |^que nos miran cre|cer
    |^Estarán dando su |luz a nuestra |^senda | \e
    |No te lo olvides her|mano lo que entre|^gas con el cora|zón
    |^Nos volverá con a|mor, será nuestra e|^sencia | \e
  \endverse
  \goto{Cayarí}
  \begin{translation}
    When the moon rises lighting up the night \emph{nimay}
    And the stars are seen in the sky
    Deep blue that envelops us in the great mystery \emph{nimay}
    The mystery of the night deep inside
    \nextverse
    Cayarí rirí rirí rirí rirí rirí rirí
    It is the presence of your eternal spirit
    \nextverse
    It is the deep roots that give us their secrets \emph{nimay}
    Medicine from the land deep inside
    Opening our consciences opening hearts \emph{nimay}
    Medicine from the Andes deep inside
    \nextverse
    Parents, children, brothers and mothers who watch us grow
    They will be giving their light to our path
    Do not forget brother what you give with your heart
    It will return to us with love, it will be our essence
  \end{translation}
\endsong


\beginsong{Como Te Quiero Yo}[by={Irina Florez},ph={III, IV},tags={love}]
  \audio[key=C\#m]{https://irinaflorez.bandcamp.com/track/como-te-quiero-yo}
  \audio[key=C\#m]{https://soundcloud.com/dejahvuuu/como-te-quiero-yo-irina-florez}
  \audio[key=C\#m]{https://www.youtube.com/watch?v=p5kGatl8fmA}
  \newchords{chords_comotequieroyo_a}\newchords{chords_comotequieroyo_b}
  \beginverse\memorize[chords_comotequieroyo_a]
    \ind |\[\mnc{A}Am]Yana he ya\[^\mn{E}]na |\[C]yo, yana he |\[E7]ya\[^\mn{D}]na yo \[^\mn{C}]hi\[^\mn{B}]a|\[\mnc{A}Am]na
    \ind Yana he |yana \[C]yo, |yana he \[E7]yana |yo hia\[Am]na
    \ind |Yana he yana |\[E7]yo hiana, |yana he yana |\[Am]yo hiana | | \e
  \endverse
  \notesoff
  \beginverse\replay[chords_comotequieroyo_a]
    \ind |^Como te quiero |^yo, como te |^quiero yo-o-o-|^oo
    \ind Como te |quiero ^yo, |como te ^quiero |yo-o-o-^oo
  \endverse
  \noteson
  \beginverse\memorize[chords_comotequieroyo_b]
    |\[\mnc{E}Am]Como el agua que |\[C]fluye libre,
    |\[E7]agua clarita |\[Am]del manantial.
    |\[F]Como el viento que |\[C]canta y sopla
    |\[E7]en lo profundo |\[Am]del corazón.
  \endverse
  \notesoff
  \beginverse\replay[chords_comotequieroyo_a]
    \ind |^Como te quiero |^yo, como te |^quiero yo-o-o-|^oo
    \ind Como te |quiero ^yo, |como te ^quiero |yo-o-o-^oo
  \endverse
  \beginverse\replay[chords_comotequieroyo_b]
    |^Como el fuego que |^brilla claro,
    |^la llama antigua |^de la creación.
    |^Como la tierra que |^nace en flores,
    |^que llama el \up{1}canto |^del colibrí. \altlyr[2]{vuelo}
  \endverse
  \beginverse\replay[chords_comotequieroyo_a]
    \textnote{Instrumental, on 2nd playthrough:}
    \ind[3] |^{ }{ }{ }{ }{ } |^{ }{ }{ }{ }{ } |^{ }{ }{ }{ }{ } |^{ }{ }{ }{ }{ }
    \ind[3] |{ }{ }{ }{ }^{ }{ }{ }{ } | { }{ }{ }{ }^{ }{ }{ }{ } |{ }{ }{ }{ }^{ }{ }{ }{ }
    \ind[3] |{ }{ }{ }{ }{ } |^{ }{ }{ }{ }{ } |{ }{ }{ }{ }{ } |^{ }{ }{ }{ }{ } | \e
  \endverse
  \beginverse\replay[chords_comotequieroyo_a]
    \ind |^Como te quiero |^yo, como te |^quiero yo-o-o-|^oo
    \ind Como te |quiero ^yo, |así te ^quiero |yo-o-o-^oo
  \endverse
  \beginverse\replay[chords_comotequieroyo_b]
    |^Como es el paja|^rito le canta al
    |^cielo cuando nace el |^sol.
    ^Como es el |paja^rito le |canta al
    ^cielo cuan|do nace el ^sol. | \e
  \endverse
  \textnote{\emph{D.C. al Fine}}
  \goto{Yana he yana yo}
  \begin{translation}
    Yana he yana yo\ldots
    \nextverse
    How I love you\ldots
    \nextverse
    Like the water that flows free,
    clear spring water.
    Like the wind that sings and blows
    deep in the heart.
    \nextverse
    Like the fire that shines clear,
    the ancient flame of creation.
    Like the earth that grows flowers,
    which calls the song (flight) of the colibri.
    \nextverse
    Like the bird that sings
    to the sky when the sun rises.
  \end{translation}
\endsong


\beginsong{Todo es mi Familia}[tags={unity},ph={IV}]
  \beginchorus
    \[\mn{E}]Ca|\[\mnc{A}A]minaré en bell\[\mn{G}\mn{E}]eza, ca|\[\mn{A}]minaré \[\mn{G}]en \[\mn{A}]paz
  \endchorus\glueverses\beginchorus
    |\[G]Todo es mi fa|\[A]milia
  \endchorus
  \beginverse
    |\[A]Todo es sagrado, las |plantas y animales
    |Todo es sagrado, las mon|tañas, selva y el mar
  \endverse\glueverses\beginchorus
    |\[G]Todo es mi fa|\[A]milia
  \endchorus
  \beginchorus
    |\[A]Hey ah hey ah hey ah, |hey ah hey ah hey
  \endchorus\glueverses\beginchorus
    |\[G]Todo es mi fa|\[A]milia
  \endchorus
  \textnote{in English:}
  \beginchorus
    I |\[A]will walk in beauty, I |will walk in peace
  \endchorus\glueverses\beginchorus
    |\[G]Everything is my |\[A]family
  \endchorus
  \beginverse
    |\[A]Everything is sacred, |animals and plants
    |\[A]Everything is sacred, |mountains, forest and sea
  \endverse\glueverses\beginchorus
    |\[G]Everything is my |\[A]family
  \endchorus
  \beginchorus
    |\[A]Hey ah hey ah hey ah, |hey ah hey ah hey
  \endchorus\glueverses\beginchorus
    |\[G]Everything is my |\[A]family
  \endchorus
  \musicnote{outro, fade out:}
  \beginchorus
    |\[G]Todo somos fa|\[A]milia
  \endchorus
\endsong


\beginsong{Niño Salvaje}[index={Soy un Hijo de la Tierra},by={Alberto Kuselman},tags={unity},ph={IV}]
  \audio[]{https://www.youtube.com/watch?v=02DM-kESGzY}
  \audio[]{https://www.youtube.com/watch?v=OTJd02cGq10}
  \beginchorus\memorize
    |\[\mnc{E}C]El silencio es mi pa|\[\mnciii{F}{E}{D}Fmaj7]labra, la |\[\mnc{G}G]tier\[^\mn{D}]ra es mi |\[\mnciii{E}{D}{C}C]madre
    |los árboles mis her|\[Fmaj7]manos, las es|\[G]trellas mi desti|\[C]no
  \endchorus
  \notesoff
  \beginchorus
    |^Soy \up{*}un hijo de la |^tierra, mi |^corazón es una es|^trella
    |viajo a bordo del e|^spíritu ca|^mino a la eterni|^dad
  \endchorus
  \notesoff
  \beginchorus
    |^Soy \up{*}un niño sal|^vaje, ino|^cente, libre y sil|^vestre
    |tengo todas las e|^dades mis an|^cestros viven en |^mí
  \endchorus
  \beginchorus
    |^Soy \up{*}hermano de las |^nubes y |^sólo sé compar|^tir
    |sé que todo es de |^todos y que |^todo está vivo en |^mí
  \endchorus
  \vspace{1em}\altlyr{una hija, una niña, hermana}
  \begin{translation}
    Silence is my word, the earth is my mother,
    the trees my brothers, the stars my destiny.
    \nextverse
    I am a child of the earth, my heart is a star,
    I travel aboard the spirit on the way to eternity.
    \nextverse
    I am a wild, innocent, free and wild child,
    I have all ages, my ancestors live in me.
    \nextverse
    I'm a brother of the clouds and will only share,
    I know that everything belongs to everyone and that
    \ind everything is alive in me.
  \end{translation}
\endsong


\scleardpage % Needed here
\beginsong{Madre Selva}[by={Grupo Putumayo},tags={Mother Earth},ph={IV}]
  \newchords{chords_madreselva_a}\newchords{chords_madreselva_b}
  \beginchorus\memorize[chords_madreselva_a]
    \[^\mn{A}]Mad\[^\mn{B}]re |\[\bmc\mnc{C}Am]Tierra Madre \[\bm]Tierra yo te a|\[\bmc F]labo Madre \[\bm]Tierra
    Porque |\[\bmc G]eres el o\[\bmc Em]rigen \[\bmc G]de la |\[\bmc Am]vida \[\bm]
  \endchorus\glueverses\beginchorus\memorize[chords_madreselva_b]
    En tus |\[\bmc G]selvas en tus \[\bmc Em]selvas tus mon|\[\bmc Am]tañas \[\bm]
    Donde ha|\[\bmc G]bita el e\[\bmc Em]spíritu di|\[\bmc Am]vino \[\bm]
  \endchorus
  \notesoff
  \beginchorus\replay[chords_madreselva_a]
    Madre |^Selva Madre ^Selva yo te a|^labo Madre ^Selva
    Porque |^eres el o^rigen ^de la |^vida ^
  \endchorus\glueverses\beginverse\replay[chords_madreselva_b]
    En tus |^selvas en tus ^selvas tus mon|^tañas ^
    Donde |^nace el aire ^puro que re|^spiro ^\replay
    En tus |^selvas en tus ^selvas y tus |^ríos ^
    Donde |^nace el agua ^pura que be|^bemos ^
  \endverse
  \beginchorus\replay[chords_madreselva_a]
    Danos |^fuerza, danos ^fuerza Madre |^Selva, danos ^fuerza
    Ilumi|^na ilumi^na nues^tro ca|^mino ^
  \endchorus\glueverses\beginchorus\replay[chords_madreselva_b]
    Ilu|^mina ilumi^na nuestro ca|^mino, ^
    con tu e|^spíritu tu e^spíritu di|^vino ^
  \endchorus
  \beginchorus\replay[chords_madreselva_a]
    Yo te a|^labo yo te a^labo Madre |^Selva yo te a^labo
    Porque |^eres el o^rigen ^de la |^vida ^
  \endchorus\glueverses\beginchorus\replay[chords_madreselva_b]
    En tus |^selvas en tus ^selvas tus mon|^tañas ^
    Donde |^nace, crece y ^crece el yage|^cito ^
  \endchorus
  \beginchorus\replay[chords_madreselva_a]
    Mama |^Cocha con tus ^aguas crista|^linas
    Limpia, ^purifica |^cuerpo, el e^spíri^tu del |^hombre ^
  \endchorus\glueverses\beginchorus\replay[chords_madreselva_b]
    Al que |^pide sana^ción, paz y harmo|^nía ^
    Al que a|^bre su cora^zón a la ale|^gría. ^
  \endchorus
  \beginchorus\replay[chords_madreselva_a]
    Yo te a|^labo yo te a^labo Madre |^Selva yo te a^labo
    Porque |^eres el o^rigen ^de la |^vida ^
  \endchorus\glueverses\beginchorus\replay[chords_madreselva_b]
    En tus |^selvas en tus ^selvas tus mon|^tañas ^
    Donde |^nace, crece y ^crece el yage|^cito ^
    \rep{4}
  \endchorus
  \begin{translation}
    Mother Earth, Mother Earth, I praise you Mother Earth,
    because you are the origin of life.
    In your forests, in your forests, in your mountains,
    where the divine spirit dwells.
    \nextverse
    Mother Forest, Mother Forest, I praise you Mother Forest,
    because you are the origin of life.
    In your forests, in your forests, in your mountains,
    where the pure air that I breathe is born.
    In your forests, in your forests and in your rivers,
    where the pure water that we drink is born.
    \nextverse
    Give us strength, give us strength, Mother Forest, give us strength.
    Illuminate, illuminate our path.
    Illuminate, illuminate our path,
    with your spirit, your divine spirit.
    \nextverse
    I praise you, I praise you, Mother Forest I praise you,
    because you are the origin of life.
    In your forests, in your forests, in your mountains,
    where the \emph{yagecito} is born, growing and growing.
    \nextverse
    \emph{Mama Cocha}, with your crystalline waters:
    cleanse, purify the body, the spirit of (hu)man,
    of the one who asks for healing, peace and harmony,
    of the one who opens their heart to joy.
    \nextverse
    I praise you, I praise you, Mother Forest I praise you,
    because you are the origin of life.
    In your forests, in your forests, in your mountains,
    where the yagecito is born, growing and growing.
  \end{translation}
  \begin{explanation}
    \begin{description}
      \item[Mama Cocha:] an ancient Incan goddess of sea (also other bodies of water) and fishes,
        the guardian of fishermen and sailors
      \item[Yagé, yagecito:] another name for \emph{Ayahuasca}, used especially in areas around
        and within Colombia and Ecuador
    \end{description}
  \end{explanation}
\endsong


\beginsong{Cuñaq}[by={Shimshai},tags={water},ph={IV}]
  \beginchorus\memorize % memorize chords even though in "chorus"
    |\[\mnc{A}Am]Des\[^\mn{E}]de Cuñaq viene | a|güita serpen\[C]teando
    |Por las a\[Em]cequias |y en remo\[Am]linos
    |\[C]Hacia nuestras \[Am]vidas | \e
  \endchorus
  \notesoff
  \beginchorus
    \ind |\[Am]De cantar hua\[C]linas |y a la vez llo\[Em]rando
    \ind |Toditas mis \[G]pe|nas se aca\[Am]baron
    \ind | Pacha\[C]mama está de |\[Am]fiesta | \e
  \endchorus
  \beginchorus
    |^Una estrellita | que a|legre me de^cía
    |Canta cantor^cito |a la a^güita
    A la a|^güita madre ^Cuñaq | \e
  \endchorus
  \goto{De cantar}
  \begin{translation}
    From Cuñaq comes water (drink), meandering
    by the ditches and in eddies
    into our lives
    \nextverse
    Singing \emph{hualinas} while weeping
    All my sorrows are gone
    Pachamama is celebrating
    \nextverse
    A little star told me:
    sing to the little water,
    to water mother Cuñaq
  \end{translation}
  \begin{explanation}
    \textbf{Cuñaq cocha} is a lake high up in the mountains in Peru,
    of which it is said that water is born there. Each year there are
    celebrations honoring the water descending from that lake.
  \end{explanation}
\endsong


\beginsong{Luna Manta \\ Pachamama}[by={Andrés Córdoba},tags={Mother Earth},ph={IV}]
  \beginverse
    |\[\mnc{E}Em]Luna \[^\mn{G}]manta que a|\[\mnc{F#}Bm]lumbra, |\[Em]luna manta que alum|\[Bm]bró
    a |\[G]mi tierrita |\[D]bella, |\[C]a mi tierra \[B7]echo una |\[Em]flor | \e
    |\[Em]Luna manta que a|\[Bm]lumbra, |\[Em]Taita Inti dio ca|\[Bm]lor
    a |\[G]mi tierrita |\[D]bella, |\[C]a mi tierra \[B7]echo una |\[Em]flor | \e
  \endverse
  \notesoff
  \beginchorus
    \ind Que |\[D]lindo es ver mi |\[Em]tierra, que |\[D]bello es ver mi |\[Em]tierra
    \ind Que |\[D]lindo es ver mi |\[C]tierra, mi Pacha que|\[B7]rida, mi Pachama|\[Em]ma | \e
  \endchorus
  \beginverse
    Su |^siete colores que a|^lumbran, |^siete colores que alum|^bró
    na|^ció de una que|^brada encan|^tada ^en a|^mor | \e
    Su |^siete colores que a|^lumbran, un |^arco iris alum|^bró
    na|^ció en una la|^guna bende|^cida ^por el |^sol | \e
  \endverse
  \vspace{1em}\goto{Que lindo}
  \textnote{\emph{D.C. al Fine}}
  \beginverse
    Que |\[D]lindo es ver mi |\[Em]tierra, los |\[D]apus de las cordi|\[Em]lleras
    |\[C]donde se junta fa|\[Em]milia, silvan melo|\[B7]días, mi Pachama|\[Em]ma | \e
    Que |\[D]lindo es ver mi |\[Em]tierra, los |\[D]apus de las cordi|\[Em]lleras
    |\[C]donde hay selvas encan|\[Em]tadas, y plantas sa|\[B7]gradas, mi Pachama|\[Em]ma | \e
  \endverse
  \begin{translation}
    Moon that illuminates, moon that illuminated
    To my beautiful earth, to my earth I throw a flower
    Moon that illuminates, Taita Inti gave warmth
    To my beautiful earth, to my earth I throw a flower
    \nextverse
    \ind How lovely it is to see my earth, how beautiful it is to see my earth
    \ind How lovely it is to see my earth, my beloved Pacha, my Pachamama
    \nextverse
    Its seven colors that illuminate, seven colors that illuminated
    Born of an enchanted gully in love
    Its seven colors that illuminate, a rainbow illuminated
    Was born in a sun-blessed lagoon
    \nextverse
    How lovely to see my earth, the apus of the mountain ranges
    Where family reunites, poetic melodies, my Pachamama
    How lovely to see my earth, the apus of the mountain ranges
    Where there are enchanted forests, and sacred plants, my Pachamama
  \end{translation}
  \begin{explanation}
    \begin{description}
     \item[Apu] is a mountain spirit in the mythology of Peru, Ecuador and Bolivia; the term
       dates back to the Inca empire.
    \end{description}
  \end{explanation}
\endsong


\beginsong{Sol de la Mañana}[by={Bóveda Celeste},ph={IV}]
  \audio[key={Bm}]{https://www.youtube.com/watch?v=gTGygs833d0}
  % \capo{2}
  \beginchorus
    |\[\mnc{E}Am\mn{C}]Bri|lla, |\hspace{0.5em} \[\mn{E}]bril\[\mn{c}]la y |\[\mn{D}]can\[\mn{E}]ta un \[\mn{C}]nue\[\mn{B}]vo |\[\mn{A}]día | \e
    |\hspace{0.5em} Brilla el |sol de la ma|ñana | \e
    |\hspace{0.5em} Movi|miento de ésta |\[Dm]tie|'rra,
    |\hspace{0.5em} palpi|tando en la flo|\[Am]re|'sta
    |\hspace{0.5em} Donde un lla|mado nos re|úne | \e
    |En el cora|zón de una fa|milia | | | \e
    Her|mano, |Huitzillín y que |\[C]sue|ñas
    En |\[Dm]la nueva au|rora y la armo|\[Am]nía | \e
    Sim|pleza y |sutileza ins|\[C]pi|ra
    El |\[Dm]flujo de una |fuerza de ale|\[Am]grí|a
    Vo|lando |a los cuatro |\[C]vien|tos
    Plu|\[Dm]majes que se |sienten voy re|\[Am]zando | \e
    Y el |sona|jero agrade|\[C]ci|do
    Hoy |\[Dm]siempre está ofren|dita la fa|\[Am]mili|a
    |Canta cantor|cito mari|ri-ri-ri-ri-|ri-ri-ri-ri-|riii | | | \e
    |Canta cantor|cito mari|ri-ri-ri-ri-|ri-ri-ri-ri-|riii | | | \e
  \endchorus
  \beginchorus
    \[\mn{E}]Can|\[\mnc{A}Am]tar, |\hspace{0.5em} \[\mn{E}]can|\[\mn{A}]tar | \e
    \lrep |\[C]Vas vibrando |alto
    |\[Dm]Vas tocando en |\[E]la profundi|\[Am]dad del cora|zón
    Vas te|jiendo cantor|cito \rrep
    |\[C]Vas vibrando |alto
    |\[Dm]Vas tocando en |\[E]la profundi|\[F]dad del cora|\[E]zón
    Donde la |\[Am]sangre hierve |\[C]con amor
    |\[F]No hay más pensa|\[E]mientos
    Sólo el |\[Am]flujo de la e|\[C]volución
    |\[F]Deja las ra|\[E]zones, y em|\[Am]pápate en acc|\[C]iones
    Que el |tiempo que |pasa es u|\[G]na i\[E]lu|\[Am]sión | | | \e
    |Canta cantor|cito mari|ri-ri-ri-ri-|ri-ri-ri-ri-|riii | | | \e
    |Canta cantor|cito mari|ri-ri-ri-ri-|ri-ri-ri-ri-|riii | | | \e
  \endchorus
  \begin{translation}
    Shine, shine and sing a new day
    The morning sun shines
    Movement of this earth,
    throbbing in the forest
    Where a call brings us together
    In the heart of a family
    Brother, \emph{Huitzillín} and what you dream
    In the new dawn and harmony
    Simplicity and subtlety inspires
    The flow of a force of joy
    Flying to the four winds
    Plumages that feel I'm praying
    And the grateful rattle
    Today the family is always offered
    Sing, singer, mari-ri-ri-ri-ri-ri-ri-ri-ri-riiii
    Sing, singer, mari-ri-ri-ri-ri-ri-ri-ri-ri-riiii
    \nextverse
    To sing, to sing
    \lrep You are vibrating high
    You are playing in the depth of the heart
    You are weaving a little song \rrep
    You are vibrating high
    You are playing in the depth of the heart
    Where the blood boils with love
    There are no more thoughts
    Just the flow of evolution
    Leave the reasons, and immerse yourself in actions
    That the time that passes is an illusion
    Sing, singer, mari-ri-ri-ri-ri-ri-ri-ri-ri-riiii
    Sing, singer, mari-ri-ri-ri-ri-ri-ri-ri-ri-riiii
  \end{translation}
  \begin{explanation}
    \begin{description}
      \item[Huitzillín:] Nahuatl word for ``hummingbird''
      \item[Marirí:] see song \emph{Marirí}
    \end{description}
  \end{explanation}
\endsong


\begin{intersong}
  \begin{feeler}
    ``Animals are something invented by plants to move seeds around. An extremely yang solution to a peculiar problem which they faced.''\\
    --- \emph{Terence McKenna} (1946--2000)
  \end{feeler}
  \vfill
\end{intersong}


\beginsong{Plantas Sagradas}[by={Danit Treubig},ph={IV}]
  \audio[]{https://www.youtube.com/watch?v=tCiDdrh5oks}
  \beginverse* % Show the beginning chant small and without chords, to fit on page
    \tiny\chordsoff
    |\[Dm]Hey hey hey ha |hey hey hey ha, |\[Am]hey ram hey ram |hey ram hey cu|\[Dm]ra y plumajero croma|jerui pinta a|\[Am]buelo | chai
    |\[Dm]bí bibibí bibi|bí ja cha rai |\[Am]pinta y cura y |cura |\[Dm]medicinai |taita y curan|\[Am]dera | \e
  \endverse
  \beginchorus\memorize
    \[^\mn{F}]Cami|\[\mnc{D}Dm]nando \[^\mn{F}]por la |\[^\mn{D}]selva si|\[\mnc{C}Am]gui\[^\mn{A}]endo \[^\mn{C}]la be|\[^\mn{A}]lleza,
    si|\[Dm]guiendo las can|ciones de las |\[Am]plantas. | \e
  \endchorus
  \notesoff
  \beginchorus
    \ind |\[F] Siento |vivo, |\[Am]siento la |paz,
    \ind |\[F]siento el po|der de la |\[Am]mad|re.
  \endchorus
  \beginchorus
    \ind\ind |\[Em]Siento las |\[G]plantas sa|\[Am]gra|das.
  \endchorus
  \beginchorus
    Ya|^gé yagé ya|gé, ya|^gé yagé ya|gé,
    ya|^gé yagé ya|gé, ya|^gé. | \e
  \endchorus
  \beginchorus
    |^Siempre desper|tando, |^siempre agrade|ciendo
    |^por este po|der, esta be|^lleza. | \e
  \endchorus
  \goto{Siento vivo}
  \goto{Siento las plantas}
  \goto{Yagé, yagé \rep{4}}
  \goto{Siento las plantas \rep{5}}
  \begin{translation}
    Walking through the jungle following the beauty,
    following the songs of the plants.
    \nextverse
    I feel alive, I feel peace, I feel the power of the mother.
    \nextverse
    I feel the sacred plants.
    %\nextverse % Skip this to fit the song on one page
    %Yagé yagé yagé, yagé yagé yagé, yagé yagé yagé, yagé.
    \nextverse
    Always waking up, always grateful for this power, this beauty.
  \end{translation}
\endsong


\beginsong{Buscando el Camino}[by={Karin Micha},tags={path},ph={IV}]
  \transpose{5} % To Am, where melody lies between low G and high B
  \beginverse
    \[^\mn{E}]Bus|\[C]can\[^\mn{F#}]do \[^\mn{E}]el \[^\mn{D}]ca|\[\mnc{B}G]mino que |\[Am]lleva a la e|\[Em]sencia
    Es|\[C]cucho el lla|\[G]mado |\[Am]de la Madre |\[Em]Tierra
    Voy recor|\[C]riendo los |\[G]valles voy sal|\[Am]tando las mon|\[Em]tañas
    Voy vo|\[C]lando por el |\[G]cielo voy dan|\[Am]zando con el |\[Em]agua! | \e
  \endverse
  \notesoff
  \beginverse
    Y |^sigo este ca|^mino de i|^magia y mis|^terio
    |^Voy sin equi|^paje llevo |^solo mis res|^petos
    Voy se|^guro voy sin |^miedo Gran E|^spíritu me |^guia
    Tran|^quilo y sin a|^puros hoy la |^Tierra me da el |^pulso! | \e
  \endverse
  \beginverse
    \ind \lrep |^Laa la lai la |^laa laa |^laa la lai lai |^laa laa\ldots \rrep\rep{4} | \e
    % % You can figure it out!
    % \ind |^Laa la lai la |^laa laa |^laa la lai lai |^laa
    % \ind la la la la la |^lai la lai la |^laa la lai |^lai la la lai lai |^laa laa
    % \ind la |^lai la lai la |^laa la lai |^lai la la lai lai |^laa! | \e
  \endverse
  \beginverse
    |^Voy recor|^dando |^a mis abue|^litas
    |^Voy agrade|^ciendo |^a mis abue|^litos
    Nos mos|^traron el ca|^mino nos de|^jaron sus te|^soros
    Sa|^gradas ceremo|^nias pode|^rosas medi|^cinas! | \e
  \endverse
  \beginverse
    Y |^asi voy apren|^diendo y compar|^tiendo este men|^saje
    Somos |^hijos de la |^Tierra y de|^bemos prote|^ger la
    Traba|^jando con mis her|^manos somos |^guardianes y guer|^reros
    Cu|^ramos con nuestros |^cantos y los |^rezos a la |^Tierra! | \e
  \endverse
  \goto{Laa la lai la}
  \begin{translation}
    I look for the path that leads to the essence.
    I hear the call of Mother Earth.
    I go traversing the valleys, I go jumping the mountains.
    I go flying through the sky, I go dancing with the water!
    \nextverse
    And I follow this path of imagination and mystery.
    I go without luggage, I only take my respects.
    I go safe without fear, Great Spirit guides me.
    Calm and without hurry, today the Earth gives me the pulse!
    \nextverse
    I'm remembering my grandmothers.
    I'm thankful to my grandparents.
    They showed us the way, they left us their treasures:
    sacred ceremonies, powerful medicines!
    \nextverse
    And so I'm learning and sharing this message.
    We are children of Earth and we must protect her.
    Working with my brothers, we are guardians and warriors.
    We heal with our songs and the prayers to the Earth!
  \end{translation}
\endsong


\beginsong{El Duende}[by={Camino Rojo}, ph={IV}]
  \meter{6}{8}
  \beginchorus\memorize
    \[^\mn{E}]El |\[\mnc{A}Am]duende \[^\mn{B}]de la |\[^\mn{C}]aire \[^\mn{A}]me |\[\mnc{B}E]vino a \[^\mn{G#}]de|\[^\mn{E}]cir
    |que si no can|\[Dm]taba yo me |\[C]iba mo|\[Am]rir
  \endchorus\glueverses\beginchorus
    \[\mn{E}]Can|^tar, \[\mn{B}]can|\[\mn{C}]tar \[\mn{A}]pa|^ra vi|vir
    Can|tar, can|^tar pa|^ra vi|^vir
  \endchorus
  \notesoff
  %\textnote{Instrumental}
  \beginchorus
    El |^duende del |agua me |^vino a de|cir
    |que si no flu|^ya yo me |^iba mo|^rir
  \endchorus\glueverses\beginchorus
    Flu|^ir, flu|ir pa|^ra vi|vir
    Flu|ir, flu|^ir pa|^ra vi|^vir
  \endchorus
  %\textnote{Instrumental}
  \beginchorus
    El |^duende del |fuego me |^vino a de|cir
    |que si no dan|^zaba yo me |^iba mo|^rir
  \endchorus\glueverses\beginchorus
    Dan|^zar, dan|zar pa|^ra vi|vir
    Dan|zar, dan|^zar pa|^ra vi|^vir
  \endchorus
  \textnote{Instrumental}
  \beginchorus
    El |^duende de la |tierra me |^vino a de|cir
    |que si no sem|^braba yo me |^iba mo|^rir
  \endchorus\glueverses\beginchorus
    Sem|^brar, sem|brar pa|^ra vi|vir
    Sem|brar, sem|^brar pa|^ra vi|^vir
  \endchorus
  %\textnote{Instrumental}
  \beginchorus
    El |^duende del |bosque me |^vino a de|cir
    |que si no a|^maba yo me |^iba mo|^rir
  \endchorus\glueverses\beginchorus
    A|^mar, a|mar, pa|^ra vi|vir
    A|mar, a|^mar, pa|^ra vi|^vir
  \endchorus
  \textnote{Instrumental}
  \beginchorus
    Da |^rai rai da |rai rai da |^rai rai da |rai
    Da |rai rai da |^rai rai da |^rai ray da |^rai
  \endchorus
  \begin{translation}
    The spirit of the air came to tell me
    that if I didn't sing I would die
    \nextverse
    Sing, sing to live
    \nextverse
    The spirit of the water came to tell me
    that if I didn't flow I would die
    \nextverse
    Flow, flow to live
    \nextverse
    The spirit of the fire came to tell me
    that if I didn't dance I would die
    \nextverse
    Dance, dance to live
    \nextverse
    The spirit of the earth came to tell me
    that if I didn't sow I would die
    \nextverse
    Sow, sow to live
    \nextverse
    The spirit of the forest came to tell me
    that if I didn't love I would die
    \nextverse
    Loving, loving, to live
  \end{translation}
\endsong


\beginsong{Camino Rojo}[tags={path},ph={IV}]
  \newchords{chords_caminorojo_a}\newchords{chords_caminorojo_b}
  \beginchorus\memorize[chords_caminorojo_a]
    \[^\mn{E}]Yo |\[\mnc{A}Am]sigo un camino de l|uz
    Yo |\[Dm]sigo un camino de a|\[Am]mor
  \endchorus\glueverses
  \beginchorus\memorize[chords_caminorojo_b]
    Camino |\[Dm]rojo camino |\[Am]rojo
    Camino |\[G]rojo de \[E]mi cora|\[Am]zón
  \endchorus
  \notesoff
  \beginchorus\replay[chords_caminorojo_a]
    Temaz|^calli, peyote, tip|i
    Temaz|^calli, peyote, tip|^i
  \endchorus\glueverses
  \beginchorus\replay[chords_caminorojo_b]
    Son los al|^tares, son los al|^tares
    Son los al|^tares que ^yo cono|^cí
  \endchorus
  \beginchorus\replay[chords_caminorojo_a]
    Aya|^huasca, San Pedro, niñ|os
    Taba|^quito, Maria, tam|^bor
  \endchorus\glueverses
  \beginchorus\replay[chords_caminorojo_b]
    Son sacra|^mentos, son sacra|^mentos
    Son sacra|^mentos para ^el cora|^zón
  \endchorus
  \begin{translation}
    I follow a path of light
    I follow a path of love
    Red path, red path
    Red path of my heart
    \nextverse
    Sweat lodge, Peyote, tipi
    Sweat lodge, Peyote, tipi
    They are the altars, they are the altars,
    They are the altars that I know
    \nextverse
    Ayahuasca, San Pedro, children \emph{(mushrooms)}
    Tobacco, Maria, the drum
    They are sacraments, they are sacraments
    They are sacraments for the heart
  \end{translation}
\endsong


\beginsong{Tlazocamati}[ex={español, nahuatl},tags={Aya},ph={IV}]
  \beginchorus\memorize % memorize chords even though in "chorus"
    \[^\mn{A}]A|\[\mnc{E}Am]ya Aya A|yahuas\[C]ca
    A|\[C]ya A\[Em]ya A|\[G]yahuas\[Am]ca \up{2}(| | \e)
  \endchorus
  \notesoff
  \beginchorus
   Ya|^ge Yage Ya|ge Yage ^he
   Ya|^ge Ya^ge |^Yage ^he \up{2}(| | \e)
  \endchorus
  \beginchorus
    |^Voy por el camino de |luz en la ^vida
    |^Sigo ade^lante |^con la medi^cina \up{2}(| | \e)
  \endchorus
  \beginchorus
    |^Tlazocam\[\bmadj{-.5ex}]ati to|pixin me^xica
    |^Toyo lo ^tatzin |^toyo lo ^nantzin \up{2}(| | \e)
  \endchorus
  \beginchorus
    |^We ya hey y\[\bmadj{-.5ex}]a yo |We ya hey ^yo
    |^We ya hey ^ya |^We ya hey ^yo \up{2}(| | \e)
  \endchorus
  \beginchorus
    A|^quí Aya A|yahuas^ca
    A|^quí A^ya y |^en el más al^lá \up{2}(| | \e)
  \endchorus
  \begin{explanation}
    \textbf{Tlazocamati} is a Nahuatl word for ``thank you''. Nahuatl (also known as Aztec)
    language is nowadays spoken in what is currently central Mexico.
  \end{explanation}
\endsong


\beginsong{Wirikuta}[tags={Mother Earth, Sun},ph={IV}]
  \beginverse
    |\[\mnc{A}Am]Pacha|\[\mn{C}]mama |\[G]pacha|\[Am]mama
    |\[Am]Pacha|mama |\[G]madre |\[Am]tierra
  \endverse
  \beginchorus\memorize % memorize chords here instead of the non-chorus verse above
    |\[Am]Wirikuta |\[G]Wiriku\[Am]ta |\[C]Wiriku\[Dm]ta |\[E7]Gran Espíri\[Am]tu
  \endchorus
  \vspace{1em}\goto{Pachamama}
  \beginchorus
    |^Taita Inti |^Taita In^ti |^Taita In^ti |^Gran Espíri^tu
  \endchorus
  \vspace{1em}\goto{Pachamama}
  \begin{explanation}
    \begin{description}
      \item[Wirikuta:] a site in central Mexico, sacred to the Wixarrica Huichol people,
          where the world was created; \textbf{Wiracocha} is a creator deity in the mythology
          of the Andes.
      \item[(Taita) Inti:] Incan Sun God
    \end{description}
  \end{explanation}
\endsong


\beginsong{Gracias}[tags={thankfulness, love},ph={III, V}]
  \meter{6}{8}
  \beginchorus
    |\[\mnc{E}Am]A\[\mn{C}]buelitas |\up{*}piedras las gracias te |\[E]doy
    Las gracias te |\[Am]doy \[\up{2}(A7)]
  \endchorus
  \altlyr{tierra, agua, fuego\ldots}
  \beginchorus
    |\[Dm]Por brimer cora|\[Am]zón a la sana|\[E]cion
    Ab\[E7]rirme al a|\[Am]mor \[\up{1}A7]
  \endchorus
  \begin{translation}
    Grandmother stones, I offer you thanks
    I offer you thanks
    \nextverse
    By opening the heart to healing
    Love opens to me
  \end{translation}
\endsong


\beginsong{Bienvenidos Hermanitos}[by={Daniel Osorio},tags={morning, Mother Earth, Sun},ph={IV, V}]
  \beginchorus\memorize
    \[^\mn{A}]Ya va |\[Am]florecien\[^\mn{C}]do \[^\mn{D}]el |\[\mnc{E}C]día,
    la ale|\[G]gría va desper|\[C]tando
    Pacha|\[F]mama ya nos le|\[C]vantas,
    Madre |\[E7]Tierra canto de a|\[Am]mor | \e
  \endchorus
  \notesoff
  \beginchorus
    Estamos |^vivos con ganas de a|^marnos,
    de la |^vida enamo|^rados
    Taita |^Inti que estás en los |^cielos
    ya ca|^lientas nuestro cora|^zón | \e
  \endchorus
  \beginchorus
    \ind |\[F]Lailaralai Lara|\[C]lai Laralai |\[G]Lai Larala|\[C]lai
    \ind Melo|\[F]día sin fin |\[C]esta vibran|\[E7]do en mi cora|\[Am]zón | \e
  \endchorus
  \beginchorus
    Paja|^rito, selva, mon|^taña,
    |^canto de fuerza y |^vida
    mara|^villas hoy nos es|^peran 
    bendi|^ciones del crea|^dor | \e
  \endchorus
  \beginchorus
    \ind |\[F]Bienve|nidos herma|\[C]nitos |\[F]bienve|nidos hoy
    \ind |\[C] \[F]Bienve|nida \[E]vida bu|\[Am]ena, |\[E7]bienvenida |\[Am]tu canción | \e
  \endchorus
  \vspace{1em}\goto{Lailaralai}
  \brk
  \imagecc[2]{sunrise_mountain_bw_transparent_bg_CC0_1280px.png}%
  \begin{translation}
    The day is already blossoming,
    joy is awakening.
    Pachamama lift us up.
    Mother Earth, I sing of love.
    \nextverse
    We are alive wanting to love each other,
    with the desire for life in love.
    Taita Inti (Sun God) in the skies
    already warms our hearts.
    \nextverse
    Lailaralai Laralai Laralai Lai Laralai.
    Endless melody is vibrating in my heart.
    \nextverse
    Little bird, forest, mountain,
    I sing of strength and life.
    Marveling this day, we wait for
    the blessings of the creator.
    \nextverse
    Welcome siblings, welcome this day.
    Welcome good life, welcome your song.
  \end{translation}
\endsong


%%%%%%%%%%%%%%%%%%%%%%%%%%%%%%%%%%%%%%%%%%%%%%%%%%%%%%%%%%%%%%%%%%%
%%% LATEST PRINTOUT CONTAINED THE SONGS ABOVE.                  %%%
%%%%%%%%%%%%%%%%%%%%%%%%%%%%%%%%%%%%%%%%%%%%%%%%%%%%%%%%%%%%%%%%%%%
%%% Please try to not change the song numbers above this point. %%%
%%% Add new songs only after this point.                        %%%
%%%%%%%%%%%%%%%%%%%%%%%%%%%%%%%%%%%%%%%%%%%%%%%%%%%%%%%%%%%%%%%%%%%


% % TODO: this is not done, needs work!
% \beginsong{Arbolito Divino}[by={Nick Barbachano},ph={II, III}]
%   \audio[key=Am]{https://soundcloud.com/nickbarbachano/arbolito-divino-feat-danit}
%   \audio[key=Am]{https://www.youtube.com/watch?v=nmboXwt3Fc8}
%   \beginverse
%     |\[\mnc{A}Am]Arboli\[^\mn{G}]to \[^\mn{A}]di|\[\mnc{D}G]vino, raíces \[^\mn{E}]pro|\[\mnc{C}Am]fundos
%     Abraza el |\[G]mundo, nuestro ca|\[Am]mino, con tu medi|\[G]cina
%     Todas las |\[Am]plantas, las plantas di|\[G]vinas mari-ma|ri-ri-ri-ri-ri
%   \endverse
%   \notesoff
%   \beginverse
%     |^Yagesito cu|^raca, de la selva|^cita
%     La pachama|^mita, trae sabidu|^ría a nuestra fa|^milia,
%     Trae ale|^gría la liana que |^guía, Caapi caa|pi caapi caapi caapi
%   \endverse
%   \beginverse
%     |^Ayahuasquita |^cura, cura mi |^cuerpo
%     Con tu amor|^cito, cura mi |^mente viejo doctor|^cito
%     Cura mi |^gente, poderoso abue|^lito, vovo vo|vo vovo vo vo
%   \endverse
%   \beginverse
%     |^Trai nai nai con la |^vida, vida de la |^baila
%     Cura con la |^waira, limpia limpia |^taita vibrando con la |^walca
%     Con la madre |^selva, Taita cu|^raca ara a|ra ra ra ra ra\replay
%     |^Trai nai nai nai nai |^nai nai Tra na na na
%     |^nai nai Trai nai nai |^nai nai Tra na na na
%     |^nai nai Trai nai nai |^nai, Tra na na na
%     |^nai nai nai nai nai nai |^nai nai nai na |na na na na nai
%   \endverse
%   \goto{Arbolito divino}
%   \goto{Trai nai nai con la vida}
%   \musicnote{instrumental: Am G F G}
%   \beginchorus
%     \ind |\[Am]Trai nai nai nai nai |\[G]nai nai Tra na na na
%     \ind |\[Am]nai nai Trai nai nai |\[G]nai nai Tra na na na
%     \ind |\[F]nai nai Trai nai nai |\[G]nai, Tra na na na
%     \ind |\[F]nai nai nai nai nai nai |\[G]nai nai nai na |na na na na nai
%   \endchorus
%   \begin{translation}
%     Divine little tree, deep roots
%     Embrace the world and our path, with your medicine
%     All plants, divine plants mari-mari-ri-ri-ri-ri
%     \nextverse
%     Yage, from the forest,
%     The Mother Earth, bring wisdom to our family,
%     Bring joy, the liana that guides, Caapi caapi caapi caapi caapi
%     \nextverse
%     Ayahuasquita heal, heal my body,
%     With your love, heal my mind, old doctor
%     Heal my people Mighty grandpa, vovo-vovo-vovo-vo-vo
%     \nextverse
%     Trai nai nai with the life, life of the dance
%     Cure with the waira, clean clean taita vibrating with the walca
%     With the mother jungle, Taita curaca-ara-ara-ra-ra-ra-ra
%     Trai nai nai nai nai\ldots
%   \end{translation}
% \endsong

    % Portuguese language songs from the Americas

\beginsong{Defumação \\ ``Suitsutus''}[ah={201}]
  \meter{4}{4}
  \beginchorus
    \lrep |\[Am] Defuma de|\[E]fuma\[Am]dor
    E|\[C]sta \[Dm]casa de Nos|\[E]so Sen\[Am]hor\rrep
    \lrep |\[Am]Leva pra's on|\[A7]das do \[Dm]mar
    O |\[Dm]mal que a\[E]qui |\[E7]posso es\[Am]tar\rrep
  \endchorus
  \meter{3}{4}
  \beginchorus
    \lrep De|\[Am]fuma esta |casa |\[A7]bem defu|\[Dm]mada
    Com |\[Dm]a Cruz de |Deus ela |\[E7]vai ser re|\[Am]zada\rrep
    \lrep Eu |\[C]sou reza|dor sou fil|ho de Um|\[Dm]banda
    Com |\[Dm]a Cruz de |\[E7]Deus todo |mal se ab|\[Am]randa\rrep
  \endchorus
  \vfill
  \begin{translation}
    Cense, incenser
    This house of our Lord
    Take to the  waves of the sea
    the evil that might be here
    \nextverse
    Cense this incensed house
    With the Cross of God will be prayed
    I am the prayer, the son of Umbanda
    With the Cross of God all evil relents
  \end{translation}
\endsong


\beginsong{Sou Brilho Do Sol \\ Olen paiste auringon}[ah={193}]
  \beginchorus\memorize
    \lrep Eu |\[A7]sou brilho do |\[D]sol |\[A7]sou brilho da |\[D]lua \rrep
    \lrep Dou |\[Bm]brilho às est|\[Em]relas porque |\[A7]todas me acompa|\[D]nham \rrep
  \endchorus
  \beginchorus
    \lrep Eu |^sou brilho do |^mar eu |^vivo no ven|^to \rrep
    \lrep Eu |^brilho na flo|^resta porque |^ela me perten|^ce \rrep
  \endchorus
  \beginchorus
    \lrep Olen |^paiste aurin|^gon, |^sekä loiste |^kuun \rrep
    \lrep Va|^laisen tähdis|^tön, sillä se |^on mun seura|^nain \rrep
  \endchorus
  \beginchorus
    \lrep Olen |^välke meren|^pinnan, hen|^gitän tuules|^sa \rrep
    \lrep Loistan |^metsän sydä|^messä, sillä se |^kuuluu minul|^le \rrep
  \endchorus
\endsong


\beginsong{Força Da Floresta}[ah={168}]
  \beginverse
    Chamo |\[Dm]força| la da flo|\[Am]resta
    E a força |\[Em]vem| para nos ensi|\[Am]nar |
  \endverse
  \beginverse % not a chorus to disable the vertical line
    Yana|^heê| Yana|^heê
    Yana|^heê| Yana|^heê
  \endverse
  \beginverse
    Chamo a |^cura| la da flo|^resta
    E a cura |^vem| para nos cu|^rar | \hfill Yanaheê\ldots
  \endverse
  \beginverse
    Segura |^firme| que eu vou te |^levar
    E te mostrar |^el | minha mãe Ye|^manjá | \hfill Yanaheê\ldots
  \endverse
  \beginverse
    Chamo a |^força| da linha de |^tucum
    E a força |^vem| força de ê O|^gum | \hfill Yanaheê\ldots
  \endverse
  \begin{translation}
    I call the power of the forest
    And power comes to teach us
    \nextverse
    I call the healing of the forest
    And it comes to heal us
    \nextverse
    Hold on tight, I'll take you
    And show you my mother Yemanjá (Queen of the Ocean)
    \nextverse
    I call the power of the tucum line
    And power comes, power of Ogúm (Warrior Spirit)
  \end{translation}
\endsong

\sclearpage
\beginsong{Flor Das Águas}[by={Mestre Irineu Serra},ah={170}]
  \capo{5}
  \beginchorus
    |\[Am] Flor das |\[G]águas da onde |\[E7]vens, para onde |\[Am]vais
    Vou fa|\[Am]zer a mi\[C]nha lim|\[E7]peza \[F]no cora|\[C]ção \[E7]está meu |\[Am]Pai
  \endchorus
  \beginchorus\memorize
    A mo|\[Am]rada \[E7]do meu |\[Am]Pai é no |coração do |\[G]mundo
    Aonde e|\[F]xiste \[C]todo a|\[E7]mor e \[F]tem um |\[C]segre\[E7]do pro|\[Am]fundo    
  \endchorus
  \beginchorus
    Este |^segre^do pro|^fundo está em |toda humani|^dade
    Se to|^dos se ^conhe|^cerem ^ aqui |^dentro ^da ver|^dade.
  \endchorus
  \begin{translation}
    Flower of waters, from where do you come, where do you go
    I will do my cleansing, in the heart is my Father
    \nextverse
    My father resides in the heart of the world
    Where there is love and all have a profound secret
    \nextverse
    This profound secret is for all humanity
    If everyone knew the truth.
  \end{translation}
\endsong


\beginsong{Guerreiro da Paz \\ Warrior of Peace}[by={Santo Daime Rio Grande do Sul}]
  \chordsoff
  \beginchorus
    El que une la verdad 
    Eu chamo a força, eu chamo a força 
    eu chamo a força 
    força das pedras para me firmar 
    Eu chamo a terra, eu chamo a terra 
    eu chamo a terra 
    eu chamo a terra para me enraizar 
  \endchorus
  \beginchorus
    Eu chamo o vento, eu chamo o vento 
    eu chamo o vento 
    eu chamo o vento vem me elevar 
    Eu chamo o fogo, eu chamo o fogo 
    eu chamo o fogo 
    eu chamo o fogo para me purificar 
  \endchorus
  \beginchorus
    Eu chamo a Lua, chamo o Sol 
    chamo as estrelas 
    Chamo o universo para me iluminar 
    Eu chamo a água, chamo a chuva 
    e chamo o rio 
    Eu chamo todos para me lavar 
  \endchorus
  \beginchorus
    Eu chamo o raio, o relâmpago e o trovão 
    Eu chamo todo o poder da criação 
    Eu chamo o mar, chamo o céu e o infinito 
    Eu chamo todos para nos libertar 
  \endchorus
  \beginchorus
    Eu chamo Cristo, eu chamo Budha 
    Eu chamo Krishna 
    Eu chamo a força de todos orixás 
    Eu chamo todos com suas forças divinas 
    Eu quero ver o universo iluminar 
  \endchorus
  \beginchorus
    Eu agradeço pela vida e a coragem 
    Ao universo pela oportunidade 
    E a minha vida eu dedico com amor 
    Ao sonho vivo da nossa humanidade 
  \endchorus
  \beginchorus
    Sou mensageiro, sou cometa, eu sou indígena 
    Eu sou filho da nação do Arco Íris 
    Com meus irmãos eu vou ser mais um 
    guerreiro 
    Na nobre causa do Inka Redentor 
  \endchorus
  \beginchorus
    Eu sou guerreiro, eu sou guerreiro e vou lutando 
    A minha espada é a palavra do amor 
    O meu escudo é a bondade no meu peito 
    E o meu elmo são os dons do meu senhor 
  \endchorus
  \beginchorus
    Eu agradeço a nossa Mãe e ao nosso Pai 
    E aos meus irmãos por todos me ajudar 
    A minha glória para todos eu entrego 
    Porque nós todos somos um nesta união 
  \endchorus
  \beginchorus
    Ñdarei a sã 
    ñdarei a sã 
    ñdarei a sã 
    Desde o principio 
    todos nós somos irmãos! 
    Orei ouá 
    Orei ouá 
    Orei ouá 
    Viva o Poder de todo o universo!
  \endchorus

  \begin{translation}
    I call the force, I call the force 
    I call the force 
    forces of the stones to firm me 
    I call the earth, I call the earth 
    I call the earth 
    I call the earth to root me 
    \nextverse
    I call the wind, I call the wind 
    I call the wind 
    I call the wind comes to rise me up 
    I call the fire, I call the fire 
    I call the fire 
    I call the fire to become purified 
    \nextverse
    I call the Moon, I call the Sun 
    I call the stars 
    I call the universe to illuminate me 
    I call the water, I call the rain 
    and I call the river 
    I call all to wash me 
    \nextverse
    I call the ray, the lightning and the thunder 
    I call all the power of the creation 
    I call the sea, I call the sky and the infinite 
    I call all to free us
    \nextverse
    I call Christ, I call Budha 
    I call Krishna 
    I call the force of all Orixás 
    I call all with their divine forces 
    I want to see the universe light up
    \nextverse
    I thank for the life and the courage 
    To the universe for the opportunity 
    And my life I dedicate with love 
    To our humanity's alive dream
    \nextverse
    I am messenger, I am comet, I am indigenous 
    I am son of the nation of the rainbow 
    With my siblings I will be one more 
    warrior 
    In the noble cause of Inka Redentor
    \nextverse
    I am warrior, I am warrior and I am going struggling 
    My sword is the word of the love 
    My shield is the kindness in my chest 
    And my helmet is my gentleman's talents
    \nextverse
    I thank our Mother and to our Father 
    And to all my brothers for to help me 
    My glory for all I give 
    Because us all are one in this union
    \nextverse
    Ñdarei a sã   
    ñdarei a sã    ñdarei a sã 
    ñdarei a sã 
    From the begining
    all of us are brothers! 
    Orei ouá 
    Orei ouá 
    Orei ouá 
    Live the Power of the whole universe!
  \end{translation}
\endsong


\beginsong{Luz Amor e Paz}[ah={175}]
  \beginchorus\memorize % memorize chords, even though inside 'chorus'
    Vem |\[Dm]luz m\[.]e ilu|\[Am]min\[.]a sonho dou|\[F]rado d\[.]o as|\[Am]tral \[.]
    Me con|\[F]duz e m\[.]e en|\[G]sin\[.]a o teu |\[G7sus4]brilho é o \[G7]meu si|\[C]nal\[.]
  \endchorus
  \beginchorus
    Vem a|^mor m^e faz vi|^ver c^om tua |^força ^divi|^nal ^
    Eu sou |^filho des^se po|^der ^ meu co|^ração é o ^meu si|^nal ^
  \endchorus
  \beginchorus
    Vem |^paz ^ vem |^paz ^ tua |^vida é n^atu|^ral ^
    Bem que |^o am^or me |^traz ^ meu si|^léncio é ^teu si|^nal ^
  \endchorus
  \begin{translation}
    Come, light, illuminate me, golden astral dream.
    Lead me and teach me, your brilliance is my sign.
    \nextverse
    Come, love, make me live with your divine strength.
    I am the son of this power, my heart is my sign.
    \nextverse
    Come, peace, come, peace, your life is natural.
    So what love brings me, my silence is your sign.
  \end{translation}
\endsong


\beginsong{Coração Do Mundo}[ah={187}]
\meter{3}{4}
  \beginverse
    Meu |\[C]cora|\[G/B]ção meu |\[C]cora|\[G/B]ção| 
    |\[C]Daime |\[G/B]tua a|\[Am]ju|da|
    |\[C]Falar co|\[G/B]migo |\[C]abre |\[G/B]ti
    Pra |\[C]deixar |\[G/B]me en|\[Am]trar| |
  \endverse
  \beginchorus
    |\[Dm]Cora|\[Em]ção do |\[Am]mun|do
    Em |\[F]ti que|\[Dm]ro es|\[Em]tar| |
    |\[Dm]Cora|\[Em]ção do |\[Am]mun|do|
    |\[F]Vamos |\[Em]a cu|\[Am]rar (bailar)²| |  
  \endchorus
  \beginverse
    A|^i den|^tro de |^teu a|^mor
    En|^contro |^uma |^luz | |
    |^Esta |^luz vem |^abra|^çar
    Me |^dar tran|^qüili|^da|de|  \hfill Coração\ldots
  \endverse
  \beginverse
    Com |^este a|^braço vou |^camin|^hando
    Ao |^centro do |^sofri|^men|to
    Flu|^indo can|^tando le|^vando a |^luz
    Que |^nos da |^ale|^gri|a|  \hfill Coração\ldots
  \endverse
  \ldots Transformando transformando\ldots

  \textnote{suomeksi:}
  \beginverse
    |^Sydäme|^ni |^sydäme|^ni
    |^Sydäme|^ni auta |^minu|a|
    |^Puhu |^minulle, |^avau|^du|
    |^Pääs|^tä mut si|^sälle|si|
  \endverse
  \beginchorus
    |\[Dm]Maa|\[Em]ilman |\[Am]sydän |\[Am]
    Si|\[F]nussa |\[Dm]tahdon |\[Em]elää | |
    |\[Dm]Maa|\[Em]ilman |\[Am]sydän | |
    |\[F]Paran|\[Em]tukaam|\[Am]me (tanssikaamme)² | |
  \endchorus
  \beginverse
    |^Siellä |^rakkautes |^sisäl|^lä|
    |^Ko-|^ohtaan |^valon| |
    |^Valo |^syleilee |^minu|^a
    ja |^sä|^teilee |^rauhaa | |   \hfill Maailman\ldots
  \endverse
  \beginverse
    |^Syleilys |^kanssa mä |^kul|^jen|
    |^Kärsi|^myksen kes|^kellä| |
    |^Lentäen |^laulan |^seuraten |^valoa|
    |^joka |^tuo meille |^ilo|a|   \hfill Maailman\ldots
  \endverse
\endsong



    % Sanskrit language bhajans and mantras from the Indian subcontinent
% ==================================================================
%
% The following sets the song number for the first song in this file.
% The number will automatically be incremented by one for each song.
% Please do not change this! Changing would make different versions of
% the songbook to have different numbers for the same songs, and it
% would totally mess up the selection booklets causing them to have
% wrong songs in them. (For the same reason, add new songs only to the
% end of each songs_ file.)
\setcounter{songnum}{400}


\beginsong{Pūrṇamadaḥ}[by={trad., Shantala},ex={from Upaniṣad, the first prayer},ph={I, II},tags={unity},key={Am},gk={Am, any}]
  \audio[key=Am]{https://www.youtube.com/watch?v=S0nUTYWqlJA}
  \audio[key=Am]{https://soundcloud.com/yogacat-1/purnamadah}
  \audio[key=Am]{https://soundcloud.com/kirtanforthespirit/purnamadah}
  \mnbeginverse
    |\[\mnc{A}Am]Pūr\[\mn{B}]ṇa\[\mn{C}]ma\[\mnc{B}Am/B]daḥ |\[\mnc{A}Am]Pūr\[\mn{B}]ṇa\[\mn{C}]mi\[\mnc{B}Am/B]dam \altchords{\id[1]{(Em) \capo{5}}|Em Em/F\shrp{} |Em Em/F\shrp{}}
    |\[\mnc{A}Am]Pūr\[\mn{B}]ṇa\[\mncii{C}{B}Am/B]{t Pūr}ṇa|\[\mnc{A}Am]mu\[\mn{C}]dacya\[\mnc{B}Am/B]te \altchords{|Em Em/F\shrp{} |Em Em/F\shrp{}}
    |\[\mnc{A}Am]Pūr\[\mn{B}]ṇas\[\mncii{C}{B}Am/B]{ya Pūr}\[\mn{A}]ṇa|\[\mn{B}]mā\[\mn{C}]dā\[Fmaj7]ya \altchords{|Em Em/F\shrp{} | - Cmaj7}
    |\[\mnc{A}Am]Pūr\[\mn{B}]ṇa\[\mn{C}\mncadj{2ex}{B}Am/B]mevā\[\mn{A}]va|\[\mnc{C}Fmaj7]shiṣya\[\mn{B}]te\[G6] \altchords{|Em Em/F\shrp{} |Cmaj7 D6}
  \mnendverse
  \mnbeginverse
    |\[\mnc{A}Am]Oṃ \[\bm]{}{ } |\[Fmaj7]{} \[G6]{} |\[Am]{} \[\bm]{}{ } |\[Fmaj7]{} \[G6] \altchords{|Em |Cmaj7 D6 |Em |Cmaj7 D6}
  \mnendverse
  \begin{lilywrap}\begin{lilypond}[]
    \include "tex/lp-include-head.ly"
    theMelody = \relative a' {
      \key a \minor \time 4/4
      \set melismaBusyProperties = #'()
      \repeat volta 2 {
        | a4 b8 c b2 | a4 b8 c b2
        | a4 b4( c8) b4 b8 | a8 c4 c8 b2
        | a4 b4 c8 b4 a8 | b4 c4 c2
        | a4 b8 c4  b4 a8 | c4 c8 b8~2
        | a1~ | a1 | r1 | r1
      }
    }
    theLyricsOne = \lyricmode {
      \repeat volta 2 {
        | Pūr -- ṇa -- ma -- daḥ | Pūr -- ṇa -- mi -- dam;
        | Pūr -- ṇa -- t Pūr -- ṇa | mu -- da -- cya -- te;
        | Pūr -- ṇas -- ya Pūr -- ṇa | mā -- dā -- ya;
        | Pūr -- ṇa -- me -- vā -- va | shiṣ -- ya -- te. _
        | Oṃ
      }
    }
    theChords = \chordmode {
      \repeat volta 2 {
        | a2:m a2:m/b | a2:m a2:m/b
        | a2:m a2:m/b | a2:m a2:m/b
        | a2:m a2:m/b | a2:m/b f2:maj7
        | a2:m a2:m/b | f2:maj7 g:6
        | a1:m | f2:maj7 g2:6 | a1:m | f2:maj7 g2:6
      }
    }
    \layout { #(layout-set-staff-size 14) } % for better fit
   \include "tex/lp-include-tail-notab.ly"
  \end{lilypond}\end{lilywrap}
  \begin{feeler}
    That is Pūrṇa; this is Pūrṇa.\\
    From Pūrṇa comes Pūrṇa.\\
    Taking Pūrṇa from Pūrṇa, Pūrṇa remains.\\
  \end{feeler}
  \begin{explanation}
    \begin{description}
      \item[Pūrṇa:] full with Divine Consciousness
    \end{description}
    The outer world is full with Divine Consciousness \emph{(Pūrṇa)};
    The inner world also is full with Divine Consciousness.
    From the fullness of Divine Consciousness the world is manifested.
    Taking \emph{Pūrṇa} from \emph{Pūrṇa}, \emph{Pūrṇa} remains,
    (because Divine Consciousness is non-dual and infinite).
  \end{explanation}
\endsong


\beginsong{Mahāmṛtyuñjaya Mantra \\ Oṃ Tryambakaṃ}[ph={III},tags={health, liberation},ex={from Rigveda 7.59.12}]
  \beginverse% \quad on the first line is there to not extend melody notes to the next bar
    |\[\mnc{E}C]Oṃ Tryamba\[\mn{D}]ka|\[G]ṃ \[\mn{C}]Ya\[\mn{D}]jā\[\mn{E}]mah|\[\mnciii{D}{C}{A}Dm]e \quad |\[Fmaj7]{} \e
    |\[C]{} Sugandhiṃ |\[G]Puṣṭivardha|\[Dm]nam | \e
    Ur|\[F]vārukam Iva |\[C]Bandhanān Mṛ|\[G]tyor | \e
    Muk|\[Dm]ṣīya Mā 'Mrtāt \echo{Muk|ṣīya Mā 'Mṛtāt}
    Muk|\[G]ṣīya Mā 'Mrtāt \echo{Muk|ṣīya Mā 'Mṛtāt}
  \endverse
  \begin{feeler}
    We Meditate on the Three-eyed reality\\
    Which permeates and nourishes all like a fragrance.\\
    May we be liberated from death for the sake of immortality,\\
    Even as the cucumber is severed from bondage to the creeper.
  \end{feeler}
\endsong


\beginsong{Mahāmṛtyuñjaya Mantra 2 \\ Oṃ Tryambakaṃ 2}[ph={III}, tags={health, liberation}, key={Am}, gk={Cm, Am--Em},ex={from Rigveda 7.59.12}]
  \beginverse
    |\[\mnc{A}Am]Oṃ \[\mn{B}]Tr\[\mn{C}]yam\[\mn{B}]ba\[\mn{A}]ka|ṃ \[\mn{B}]Ya\[\mn{C}]jā\[\mn{A}]ma\[\mn{B}]h|\[Em]e
    Sugandhiṃ |Puṣṭivardhana|\[Dm]m
    Urvārukam |Iva Bandhanā|\[G]n
    Mṛtyor |Mukṣīya Mā 'Mṛ|\[Am]tāt | \e
  \endverse
  \begin{explanation}
    Mahāmṛtyuñjaya Mantra \emph{(lit. ``Great death-defeating mantra``)} is one of the more potent
    of the ancient Sanskrit mantras. Maha Mrityunjaya is a call for enlightenment and is a practice
    of purifying the karmas of the soul at a deep level. It is also said to be quite beneficial for
    mental, emotional, and physical health.
  \end{explanation}
\endsong


\beginsong{Moola Mantra}[index={Om Satchitananda Parabrahma}, tags={source}, ph={I, II}, key={Dm}, gk={Cm, Cm--F\shrp{}m}]
  \transpose{5}\preferflats
  \mnbeginchorus
    \[\mn{E}]Om |\[\mnc{A}Am]Satchitanan\[\mn{G}]da |\[\mn{A}]Para\[\mn{B}]brah\[\mn{A}]ma \altchords{\id[1]{(Am)}|Am | \e}
    |Purushothama Pa\[\mn{G}]ra|\[\mn{A}\mn{B}]mat\[\mn{A}]ma \altchords{| - | \e}
    \[\mn{C}]Sri |\[\mnc{D}F]Bhagavati \[\mn{C}]Sa\[\mn{B}]{me}|\[\mncii{A}{G}G]tha \altchords{|F |G}
    \[\mn{B}]Sri |\[\mnc{A}Am]Bhagavate \[\mn{B}]Na\[\mn{A}]ma|ha \altchords{|Am | \e}
  \mnendchorus
  \mnbeginchorus
    \[\mn{E}]Hari |\[\mnc{A}Am]Om \[\mn{B}]Tat |\[\mn{A}]Sat, Hari |\[\mnc{B}G]Om \[\mn{C}]Tat |\[\mn{B}]Sat \altchords{|Am | - |G | \e}
    \[\mn{A}]Ha\[\mn{B}]ri |\[\mnc{C}F]Om \[\mn{D}]Tat |\[\mnc{B}G]Sat, \[\mn{C}]Ha\[\mn{B}]ri |\[\mncii{A}{G}Em]Om \[\mn{B}]Tat |\[\mnc{A}Am]Sat \altchords{|F |G |Em |Am}
  \mnendchorus
  \begin{feeler}
    Oh Divine Force, Spirit of All Creation,\\
    Highest Personality, Divine Presence,\\
    manifest in every living being.\\
    Supreme Soul manifested\\
    as the Divine Mother and\\
    as the Divine Father.\\
    I bow in deepest reverence.\\
  \end{feeler}
  \begin{explanation}
    Moola mantra evokes the living God, asking protection and freedom from all sorrow
    and suffering. It is a prayer that adores the great creator and liberator, who out of love and
    compassion manifests, to protect us, in an earthly form.  The calmness that the mantra can
    give is to be experienced, not spoken about. Here is the key with which any door to spiritual
    treasure could be opened. A tool which can be used to achieve all desires. A medicine which
    cures all ills. Just like when you call a person he comes and makes you feel his presence, the
    same manner when you chant this mantra, the supreme energy manifests everywhere around you. As
    the Universe is Omnipresent, the supreme energy can manifest anywhere and any time. It is also
    very important to know that the invocation with all humility, respect and with great necessity
    makes the presence stronger.
    \begin{description}
      \item[Om:] Calling on the highest energy of all there is. It is said 'In the beginning was the
        Supreme word and the word created every thing. That word is Om'. If you are meditating in
        silence deeply, you can hear the sound Om within. The whole creation emerged from the sound
        Om. It is the primordial sound or the Universal sound by which the whole universe vibrates.
        This divine sound has the power to create, sustain and destroy, giving life and movement to
        all that exist.
      \item[Sat:] Truth. The formless. The all penetrating existence that is formless, shapeless,
        omnipresent, attributeless, and qualityless aspect of the Universe, experienced as emptiness
        of the Universe. The body of the Universe that is static. Everything that has a form and can
        be sensed evolved out of this. So subtle that it is beyond all perceptions. It can only be
        seen when it has become gross and has taken form. We are in the Universe and the Universe is
        in us. We are the effect and Universe is the cause and the cause manifests itself as the
        effect.
      \item[Chit:] The Pure Consciousness of the Universe that is infinite, omni-present
        manifesting power of the Universe. Out of this is evolved everything that we call Dynamic
        energy or force. It can manifest in any form or shape. It is the consciousness manifesting
        as motion, as gravitation, as magnetism, etc. Also manifesting as the actions of the body,
        as thought force. The Supreme Spirit.
      \item[Ananda:] Pure bliss, love, joy and friendship nature of the Universe. When you experience
        either the Supreme Energy in this Creation (Sat) and become one with the Existence or
        experience the aspect of Pure Consciousness (Chit), you enter into a state of Divine Bliss
        and eternal happiness (Ananda).
      \item[Parabrahma:] The Supreme creator being in his Absolute aspect; beyond space and time.
        The essence of the Universe that is with and without form.
      \item[Purushothama:] The energy that incarnates as an Avatar in human form to help and guide
        mankind and relate closely to the beloved creation.  This has different meanings. Purusha
        means soul and Uthama means the supreme, the Supreme spirit. It also means the supreme
        energy of force guiding us from the highest world. Purusha also means Man, and Purushothama
        is the energy that incarnates as an Avatar to help and guide Mankind and relate closely to
        the beloved Creation.
      \item[Paramatma:] Supreme inner energy that is immanent in every creature and in all beings,
        living and non-living. Who comes to me in my heart, and becomes my inner voice whenever I
        ask. It's the force that can come to you whenever you want and wherever you want to guide
        and help you.
      \item[Sri Bhagavathi:] The divine mother, the power aspect of creation. The female aspect,
        which is characterized as the Supreme Intelligence in action, the Power (The Shakti). It is
        referred to the Mother Earth (Divine Mother) aspect of the creation.
      \item[Sametha:] Together or in communion with.
      \item[Sri Bhagavathe:] The Male aspect of the Creation, which is unchangeable and permanent.
      \item[Namaha:] Salutations or prostrations to the Universe that is Om and also has the
        qualities of Sat Chit Ananda, that is omnipresent, unchangeable and changeable at the same
        time, the supreme spirit in a human form and formless, the indweller that can guide and help
        in the feminine and masculine forms with the supreme intelligence. I thank you and
        acknowledge this presence in my life. I seek your presence and guidance all the time.
      \item[Hari om tat sat:] God is the truth. Hari is another name of Lord Vishnu.
    \end{description}
  \end{explanation}
\endsong


\beginsong{Gāyatrī Mantra \\ Sāvitrī Mantra}[index={Om Bhūr Bhuvaḥ Svaha},tags={wisdom, liberation, source, Sun},ph={III},ex={from Rigveda 3.62.10},key={Am},gk={Bm, Am--Em}]
  \mnbeginverse
    |\[\mnc{A}Am]Oṃ |\[\mnc{B}G]Bhūr Bhuvaḥ \[\mn{C}]Sva|\[\mncii{B}{A}Am]ha \altchords{\id[1]{(Bm)}|Bm |A |Bm}
    \[\mn{C}]Tat |\[\mnc{B}G]Sa\[\mn{C}]vi|\[\mncii{B}{A}Am]tur \[\mn{C}]Va|\[\mnc{B}G\mn{A}\mn{G}\mn{A}]reṇ|\[F]yaṃ \altchords{|A |Bm |A |G}
    Bhar|\[\mnc{B}G]gho De\[\mn{A}]va\[\mn{B}]sya |\[\mnc{C}C]Dhī\[\mn{D}]ma\[\mn{E}]hi \altchords{|A |D}
    Dhi|\[\mncii{D}{B}G]yo \[\mn{D}]Yo |\[\mncadj{1ex}{C}C]Naḥ Pra|\[\mnc{B}G]cho\[\mn{C}]da|\[\mncii{B}{A}Am]yāt \altchords{|A |D |A |Bm}
  \mnendverse
  \begin{feeler}
    We meditate on the brilliance of that\\
    Being who has produced this universe;\\
    may She enlighten our minds.
  \end{feeler}
  \begin{explanation}
    A prayer of praise that awakens the vital energies and gives liberation and deliverance from
    ignorance; it directs one's energies from harsh towards subtle. This mantra is known to impart
    wisdom, understanding, and enlightenment. This is said to be the oldest and most powerful of
    mantras, being thousands of years old. It purifies the person chanting it as well as the
    listener as it creates a tangible sense of well being in whoever comes across it.

    We meditate on that most adorable, desirable and enchanting luster and brilliance of
    our Supreme Being, our Source Energy, our Collective Consciousness who is our creator,
    inspirer and source of eternal Joy. May this Light inspire and guide our mind and open
    our hearts. That Divine Illumination which pervades the physical plane, astral plane and
    the celestial plane. That which is the most adorable. On that Divine Radiance we Meditate.
    May that Enlighten our Intellect and Awaken our Spiritual Wisdom.

    The mantra has been translated in many ways. The literal meaning of the words are below:
    \begin{description}
      \item[Oṃ:] the sacred syllable, pranava;
      \item[Bhūr:] Bhuloka (physical plane);
      \item[Bhuvaḥ:] Antariksha (space);
      \item[Svaha:] Svarga Loga (Heaven);
      \item[Tat:] that;
      \item[Savitur:] vivifying power of the Sun, source of life (also identified with Surya, Sun God);
      \item[Vareṇyaṃ:] greatest;
      \item[Bhargho:] brilliance;
      \item[Devasya:] of a Divine Entity;
      \item[Dhīmahi:] ``we meditate upon'' (knowledge imparted/understood);
      \item[Dhiyo:] Buddhi (intellect, our understanding of reality);
      \item[Yo:] which;
      \item[Naḥ:] our;
      \item[Prachodayāt:] enlighten, inspire, propel.
    \end{description}
  \end{explanation}
  \yesendsongvfill
\endsong


\beginsong{Pavamāna Mantra \\ Om Asato Mā}[ex={from Bṛhadāraṇyaka Upaniṣad},tags={transcendence},ph={I, II, III},key={Dm},gk={Dm, Cm--F\shrp{}m}]
  \mnbeginchorus\memorize
    |\[\mnc{D}Dm]Om \[\bm]{}\[^\mn{C}]{ } |\[\mnc{A}Am] asa\[\bmc\mn{D}]to \[^\mn{E}]mā |\[\mnc{D}Dm]sadga\[^\mn{C}]ma\[\bmc\mn{D}]ya, |\[Am]\e
    \[^\mn{A}]Tama\[\bmc\mn{D}]so \[^\mn{E}]mā |\[\mnc{F}F]jyoti\[\bm]r\[^\mn{G}]ga\[^\mn{F}]ma|\[\mnc{E}C]ya, \[\bm]
    Mṛ\[^\mn{F}]t|\[\mnc{E}Am]yor\[^\mn{D}]mā\[\bmc\mn{C}]{'m}\[^\mn{E}]ṛ|\[\mnc{D}Dm]taṃ ga\[^\mn{C}]ma\[\bmc\mn{D}]ya. | \e\[\bm]
  \mnendchorus
  \notesoff
  \textnotefornext{suomeksi:}
  \beginchorus
    |^Joh^{-}da|^ta ^minut |^totuu^teen, |^ \e
    ^Pimeydes|^tä ^kirkkau|^teen,
    ^Tiedotto|^muudes^ta tie|^toisuu^teen. | \e^
  \endchorus
  \begin{lilywrap}\begin{lilypond}[]
    % transcribed by larva, latest update on 2024-02
    % based on Deva Premal's version, but not exactly it
    \include "tex/lp-include-head.ly"
    % \header {
    %   title = "Pavamāna Mantra"
    % }
    theMelody = \relative d'' {
      \key d \minor \slurDashed
      \set melismaBusyProperties = #'()
      %\tempo 4 = 110
      \time 4/4
      \repeat volta 2 {
        | d2.( c4 | a4)( a8 a8) d4 e4 | d8( d8)( d8)( c8) d2~ | d4 \parenthesize a8 \parenthesize a8
        d8( d8) e4 | f4( f4)( f4) g8( f8) | e2( e8 e8)( e8)( f8)
        | e4 d4 c4 e4 | d4 d8( c8) d2~ | d1
      }
      % Deva Premal's version:
      % \repeat volta 2 {
      %   | d2.( c4 | a4)( a8 a8) d4 e4 | d8( d8)( d8)( c8) d2~ | d4 \parenthesize a8 \parenthesize a8
      %   d8( d8) e4 | d2 d4 c8( c8) | a2( a8) e'8( e8)( e8)
      %   | e4 d4 c4 e4 | d4 d8( c8) d2~ | d1
      % }
      \fine
    }
    theLyricsOne = \lyricmode {
      \set stanza = "1."
      \repeat volta 2 {
        | Om __ _ | _ a -- sa -- to mā | sad -- _ ga -- ma -- ya, | _
        Ta -- ma -- so __ _ mā | jyo -- tir -- _ ga -- ma -- | ya, __ _ _
        Mṛt -- _ | yor -- mā -- 'm -- ṛ -- | taṃ ga -- ma -- ya. | _
      }
    }
    theLyricsTwo = \lyricmode {
      % Finnish translation and phrasing by S.
      \set stanza = "2."
      \repeat volta 2 {
        | Joh -- da -- | ta __ _ _ mi -- nut | to -- tuu -- _ _ teen, | _ \skip 1 \skip 1
        Pi -- mey -- des -- | tä __ _ kirk -- kau -- _ | teen,
        Tie -- dot -- _ to -- | muu -- des -- ta tie -- | toi -- suu -- _ teen. | _
      }
    }
    theChords = \chordmode {
      \repeat volta 2 {
        | d1:m | a:m | d:m | a:m
        | f1 | c | a:m | d:m | d:m
      }
      % % Deva Premal's version:
      % \repeat volta 2 {
      %   | d1:m | a:m | d:m | a:m
      %   | f | a:m | a:m | d:m | d:m
      % }
    }
    \layout { #(layout-set-staff-size 15) } % for better fit
   \include "tex/lp-include-tail-notab.ly"
  \end{lilypond}\end{lilywrap}
  \begin{translation}
    Lead me from illusion to reality \emph{(of eternal self)},
    from darkness \emph{(ignorance)} to light \emph{(spiritual understanding)},
    from \emph{(the world of)} death to immortality \emph{(of self-realization)}.
  \end{translation}
\endsong


\beginsong{Prabhu Aap Jago}[by={Carioca},tags={transcendence},ph={III}]
  \newchords{chords_prabhu_a}\newchords{chords_prabhu_b}
  \beginchorus\memorize[chords_prabhu_a]
    \[^\mn{C}]Pra\[^\mn{D}]bhu |\[\mnc{E}C]Aap Jago \[^\mn{D}]Pra\[^\mn{E}]bhu |\[\mnc{F}Fmaj7]Aap \[^\mn{E}]Ja\[^\mn{C}]go
    Prabhu |\[Dm]Aap Jago Para|\[G]mathma Jago
  \endchorus
  \notesoff
  \beginverse\memorize[chords_prabhu_b]
    Mere |\[Em]Sarva Jago Sar|\[Am]vatra Jago
    Prabhu |\[Dm]Aap Jago Para|\[G]mathma Jago
    Mere |\[Em]Sarva Jago Sar|\[Am]vatra Jago
    Prabhu |\[Dm]Aap Jago Para|\[G]mathma Ja|\[C]go | \e
  \endverse
  \textnotefornext{in English:}
  \beginchorus\replay[chords_prabhu_a]
    |^Cease the cause of |^suffering
    Illumi|^nate the cause of |^joy
  \endchorus
  \beginverse\replay[chords_prabhu_b]
    |^Cease the cause of |^suffering
    Illumi|^nate the cause of |^love
    |^Cease the cause of |^suffering
    Illumi|^nate the |^cause of |^love | \e
  \endverse
  \begin{feeler}
    God awaken, God awaken in me, God awaken everywhere.\\
    May love awaken; may love awaken everywhere.
  \end{feeler}
\endsong


\beginsong{Shakti Kundalini \\ Om Mata Om Kali}[tags={Divine Mother},ph={I, IV}, key={Dm}, gk={Dm, Gm--F\shrp{}m}]
  \meter{4}{4}
  \mnbeginchorus
    \[\mn{F}]Om |\[\mncii{E}{D}Dm]Mata \[\mn{F}]Om |\[\mn{E}\mn{D}]Kali \altchords{\id[1]{(Am)}|Am | \e}
    |\[\mnc{C}C]Durga \[\mn{E}]Devi \[\mn{F}]na|\[\mnc{E}Dm]mo \[\mn{D}]namaha \altchords{|G |Am}
  \mnendchorus
  \mnbeginchorus
    |\[\mnc{D}Dm]Shakti \[\mn{E}]kun\[\mn{D}]da|\[\mnc{C}C]lini |\[\mnc{B&}B&]Jaga\[\mn{D}]dam\[\mn{C}]be \[\mn{B&}]Ma|\[\mnciii{A}{C#}{E}A]ta \altchords{|Am |G |F |E}
    |\[\mnc{D}Dm]Shakti \[\mn{E}]kun\[\mn{F}]da|\[\mncii{G}{F}C]li\[\mn{E}]ni |\[\mnc{D}B&]Jaga\[\mn{F}]dam\[\mnc{E}Am]be \[\mn{C}]Ma|\[\mnc{D}Dm]ta \altchords{|Am |G |F Em |Am}
  \mnendchorus
  \begin{feeler}
    I bow unto the Divine Mother and Her many feminine aspects: Kali, remover of delusion and
    ignorance; Divine Goddess Durga; Shakti, universal life force and consort to Shiva; and
    Kundalini, the Goddess energy that rises within us. Praise to the Mother of the World!
  \end{feeler}
\endsong


\beginsong{Jay Shri Ma \\ Ananda Ma \\ Kali Ma}[tags={Divine Mother},ph={IV}]
  \beginchorus
    \[\mn{D}]Jay \[\mn{F}]Shri |\[\mnc{A}F]Ma Kali Ka\[\mn{G}]li |\[Gm]Ma
    Jay Shri |\[Dm]Ma | \e
  \endchorus
  \beginchorus
    |\[F]{} Ananda Ma |\[C]Durga Devi
    |\[Gm]Jagadambe Shri |\[Dm]Ma
  \endchorus
\endsong


\beginsong{Jay Ambe}[tags={Divine Mother},ph={II, IV}]
  \beginchorus
    |\[\mnc{D}Dm]Jay Am\[\mn{A}]be |\[\mnc{G}C]Jaga\[\mn{A}]dam\[\mnc{E}Am]be
    |\[F]Mata Bha\[C]vani ki |\[Dm]Jay Ambe
  \endchorus
  \beginchorus
    |\[Dm]{} Durgati Nashini |\[F]Durga Jaya Jaya
    |\[C]{} Kala Vinashini |\[Dm]Kali Jaya Jaya
  \endchorus
  \beginchorus
    |\[C]Uma Rama Brah|\[F]mani Jaya Jaya
    |\[C]Radha \[Am]Rukamani |\[Dm]Sita Jaya Jaya
  \endchorus
\endsong


\beginsong{Saraswati}[tags={Divine Mother, learning},ph={II, III}]
  \beginverse
    \[\mn{D}]Sa\[\mn{E}]ras|\[\mncii{F}{E}Dm]wa\[\mn{D}]ti | \[\mn{D}]Ma\[\mn{E}]ha|\[\mn{F}\mn{E}]lax\[\mn{D}]mi | \e
    Durga |\[F]Devi |\[C]Nama|\[Dm]ha | \e
  \endverse
  \beginverse
    Saras|\[Gm]wati | Maha|\[F]laxmi | \e
    Durga |\[B&]Devi |\[C]Nama|\[Dm]ha | \e
  \endverse
\endsong

\beginsong{Devi Mantra \\ Sarva Mangala}[tags={Divine Mother, Shiva},ph={III, IV}]
  % \capo{3}
  \textnotefornext{part A: Devi Mantra}
  \beginchorus
    |\[\mnc{A}Am]Sarva Mangala |Man\[\mn{C}]ga\[\mn{A}]ly\[\mnc{B}Em]e
    |\[G]Shive Sarv|\[Em]artha Sadhi\[Am]ke
    |Sharanye Tryambake |Gau\[Em]ri
    Nara|\[Am]yan\[G]i Na|\[Em]mostut\[Am]e
    Nara|\[Am]yan\[G]i Na|\[Em]mostut\[Am]e | | \e
  \endchorus
  \textnotefornext{part B: Om Namah Shivaya}
  \beginchorus
    \ind |\[Am]Om Namah Shivaya, |\[C]Om \[G]Nama Shiva \rep{3}
  \endchorus\glueverses\beginverse
    \ind |\[Am]Om Namah Shivaya, |\[C]Om \[G]Nama Shiv|\[Am]aya \up{1}(| \e)
  \endverse
  \dacapo
  \beginchorus
    \ind |\[Am]Shivaya, Shivaya, |\[Em]Shivay\[Am]a \rep{3}
  \endchorus\glueverses\beginverse
    \ind |Shivaya, Shivaya
  \endverse\glueverses\beginchorus
    \ind |\[C]Om \[G]Namah Shiv|\[Am]aya
  \endchorus
  \begin{feeler}
     \textbf{Devi Mantra:}\par
     Welcome to you O Narayani; who is the positiveness in all the auspicious,
     one who is so auspicious herself and has all auspicious qualities,
     The provider of protection, the one with three eyes and a beautiful face;
     we salute you, O Narayani.
  \end{feeler}
\endsong


\beginsong{Mataji}[by={trad., Elisabet Just},tags={Divine Mother},ex={saṃskṛtam, português},ph={IV}]
  \audio[]{https://www.youtube.com/watch?v=\_UEUQZsaDZ4}
  %\capo{3}
  \beginverse
    |\[\mnc{A}Am]Ayi Ayi Ayi Ayi Di|wa\[\mn{B}]li \[\mn{A}]Hai \[\mn{G}]Ye \[\mn{A}\mn{B}\mn{C}]Ayi
    |\[G]Aise Shubhawa\[C]sar Par Hum |\[Em]Puje Maha \[Am]Lakshmi
    |\[Am]Sabhi Devi Devata |Aapa Hi Ko Puje
    |\[C]Nirmala \[E]Ma, O Maiya |\[G]Nirmala \[Am]Ma
  \endverse\glueverses
  \beginchorus
    O Chindra|wara, Wali Maha, Laksh|\[(C)]mi Mata\[Am]ji
  \endchorus
  \beginchorus
    Ô \sublyr{\up{2}(mi)}iê iê i|\sublyr{Ayi Ayi Ayi}\[Am]ê, ô \sublyr{Ayi}iê \sublyr{Di-}iê |\sublyr{wali}Xo\sublyr{Hai}roo\sublyr{Ye}dô \sublyr{A-}
    Ô \sublyr{y-}iê \sublyr{i}iê i|\sublyr{Ayi Ayi Ayi}\[Am]ê, ô \sublyr{Ayi}iê \sublyr{Di-}iê |\sublyr{wali}Xo\sublyr{Hai}roo\sublyr{Ye}dô \sublyr{A-}
    Olomi ai|\sublyr{Aise}\[G]ê \sublyr{Shubhawa}Xorô \sublyr{sar}\[C]{} \sublyr{Par}Ó\sublyr{Hum}manfé|\sublyr{Pu-}\[Em]é \sublyr{je}Xo\sublyr{Ma-}roo \sublyr{ha}dô\sublyr{Laksh-}\[Am]
  \endchorus\glueverses
  \beginverse
    Ô \sublyr{mi}iê iê i|\sublyr{Sabhi}\[Am]ê, \sublyrpush{Devi De}\sublyr{va-}ô \sublyr{ta}iê iê |\sublyr{Aapa}Xo\sublyr{Hi Ko}roodô\sublyr{Puje}
    Ô iê iê i|\sublyr{Nir-}\[C]ê, \sublyrpush{mala} \sublyr{Ma,}\[E]{} \sublyr{O}ô \sublyr{Mai-}iê \sublyr{ya}iê |\sublyr{Nir-}\[G]Xo\sublyr{mala}roodô\sublyr{Ma}\[Am]
  \endverse\glueverses
  \beginchorus
    Ô iê iê i|\sublyr{Ayi Ayi Ayi}\[Am]ê, ô \sublyr{Ayi}iê \sublyr{Di-}iê |\sublyr{wali}Xo\sublyr{Hai}roo\sublyr{Ye}dô \sublyr{A(yi)}
  \endchorus
  \beginchorus\musicnotefornext{decelerando}
    \sublyrpush{Ô iê iê i} |\sublyr{ê,}\[Am]Om Mani Pad\sublyr{ô}me \sublyr{iê iê}Hum |\sublyr{Xoroodô} \e
  \endchorus
  \beginchorus\rep{4}
    \ind \sublyrpush{\up{\textbf{1}}Ô iê iê i} |\sublyr{ê,}\[Am]Om Mani Pad\sublyr{ô}me \sublyr{iê iê}Hum |\sublyr{Xo}\[G]Om \sublyr{roo}Ma\sublyr{dô}ni Pad\sublyr{ô}me \sublyr{iê iê i}Hum
    \ind |\sublyr{ê}\[F]Om Mani \[Em]Pa\sublyr{ô iê iê}dme |\sublyr{Xoroodô}\[Am]Hum
  \endchorus
  \beginchorus\rep{4}
    |\[Am]Om Om Mani |\[G]Padme \[Am]Hum
  \endchorus
  \beginchorus
    \ind |\[Am]Om Mani Padme Hum |\[G]Om Mani Padme Hum
    \ind |\[F]Om Mani \[Em]Padme |\[Am]Hum
  \endchorus
  \beginchorus
    Eu |\sublyr{\up{\textbf{1}}Om}\[Am]vi ma\sublyr{Ma-}mãe \sublyr{ni}O\sublyr{Pad-}xum \sublyr{me}na \sublyr{Hum}cacho|\sublyr{Om}\[G6]ei\sublyr{Mani}ra \sublyrpush{Padme Hum}sen-
    |\sublyr{Om}\[F]tada \sublyr{Ma-}na \sublyr{ni}bei\sublyr{Pad-}\[G7]ra \sublyr{me}do r|\sublyr{Hum}\[Am]io
  \endchorus
  \beginchorus\rep{4}
    Colhendo |\[Dm]lírio, lírio lê \[G]{} colhendo
    |\[C]lírio, lírio lá \[F]{} colhendo
    |\[B\textdegree7]lírio pra enfei\[E7]tar o seu con|\[Am]gá
  \endchorus
  \goto{Om Mani Padme Hum}
  \begin{translation}
    The day of Diwali has arrived.
    On this auspicious occasion allow us to worship Shri Mahalaxmi.
    All the Gods and Goddesses worship you,
    oh Mahalaxmi Mataji Goddess of Chindwara.
    \nextverse
    I saw Mother Oxum at the waterfall
    sitting on the river's edge
    gathering lilies, there picking lilies,
    lilies to decorate her altar.
    % Image downloaded from: https://openclipart.org/detail/185904/lily
    % Original: Drinks of the World --- James Mew and John Ashton, 1892
    % Image license: Public Domain
    \imager[5]{lily_drawing_bw_transparent_bg_PD__1025px.png}%
  \end{translation}
  \begin{explanation}
    \begin{description}
      \item[Mataji:] a Hindi term meaning ``respected mother''
      \item[Diwali:] the yearly Hindu festival of lights, when prayers are offered to
        \textbf{Lakshmi}, goddess of prosperity and fortune
      \item[Oxum:] see song \emph{Ide Were} for explanation
    \end{description}
  \end{explanation}
\endsong


\beginsong{Shiva Shambho}[tags={Shiva},ph={II, IV}]
  \audio[]{https://www.youtube.com/watch?v=YBso7TPtvJU}
  \beginchorus
    |\[\mnc{A}Am]Jaya jaya Shiva \[\mn{B}]Sham|\[\mn{C}\mn{B}\mn{A}]bho, |\[\mnc{G}G]jaya jaya \[\mn{C}]Shiva \[\mn{B}]sham|\[\mnc{A}Am]bho
  \endchorus
  \beginchorus
    |\[Am]Mahadeva sham|bho, |\[G]Mahadeva sham|\[Am]bho
  \endchorus
  \begin{explanation}
    \begin{description}
      \item[Mahadeva:] a title for Lord Shiva, meaning ``Great God``
      \item[Shambho:] ``the auspicious one``
    \end{description}
  \end{explanation}
\endsong


\beginsong{Haidakandhi}[tags={Shiva, Vishnu}, ph={III, IV}, key={Am}, gk={Am, Gm--Bm}]
  \meter{4}{4}
  \beginchorus
    |\[Am]{} \[\mn{A}]Om Namah \[\mn{B}]Shi|\[\mnc{C}F]va\[\mn{A}]ya \[\mn{C}]Na\[\mn{D}]mah |\[\mnc{E}C]Om |\[\mnc{D}G]Haida\[\mn{C}]kan\[\mn{B}]dhi
  \endchorus
  \beginchorus
    |\[Am]{} \[\mn{A}]Hari Ha\[\mn{G}\mn{F}]ri |\[F]{} Hari \[\mn{G}]Ha\[\mn{F}\mn{E}]ri |\[C]{} Hari \[\mn{F}]Ha\[\mn{E}]ri |\[\mnc{D}G]Shan\[\mn{C}]ka\[\mn{B}]ra
  \endchorus
  \begin{explanation}
    \begin{description}
      \item[Hari] is a name for the supreme absolute in the \emph{Vedas} (also
      in \emph{Guru Granth Sahib} and many other sacred texts of South Asia).
      Hari refers to \emph{Vishnu} who takes away all the sorrows of his
      devotees.
      \item[Shankara] is one of the names for Shiva.
    \end{description}
  \end{explanation}
\endsong


\begin{intersong} % A quote from the Upanishads about the fourth state of consciousness
  \begin{feeler}
    ``There must be a fourth state beyond the waking, dreaming and dreamless states in which
    the absolute oneness of Brahman-Atman is what should be known.''\\
    --- \emph{The Upanishads}
  \end{feeler}
  \begin{explanation}
    \begin{description}
      \item[Brahman] is the highest Universal Principle, the Ultimate Reality. It is the final
        cause of all that exists.
      \item[Atman] is the true inner self, the soul, of an individual.
      \item[The Upanishads] are ancient Sanskrit texts that contain some of the central
        philosophical concepts of Hinduism.
    \end{description}
  \end{explanation}
\end{intersong}


\beginsong{Om Namah Shivaya}[tags={Shiva},ph={IV}]
  \beginchorus\memorize % memorize chords even though in 'chorus'
    \textnotefornext{intro:}*
    |\[\mnc{A}Am]Om Namah Shi|\[\mnc{C}F]vaya; |\[\mnc{D}G]Om Na\[^\mn{C}]mah \[^\mn{B}]Shi|\[\mnc{A}Am]vaya
  \endchorus
  \notesoff
  \beginchorus
    |^ Shivaya |^Namaha; |^ Shivaya |^Namaho
  \endchorus
  \beginchorus
    |^Sham Bol ^Shankara |^Namah ^Shivaya; \replay
    |^Girija ^Shankara |^Namah ^Shivaya
  \endchorus
  \beginchorus
    |^Aruna^chala Shiva |^Namah Shi^vaya; \replay
    |^Aruna^chala Shiva |^Namah ^Shivaya
  \endchorus
  \beginchorus
    Hari |\[C]Om Namah Shi|\[G]vaya; |\[F]Om Namah Shi|\[Am]vaya
  \endchorus
  \beginchorus
    \textnotefornext{outro:}
    |^Om Namah Shi|^vaya; |^Om Namah Shi|^vaya
  \endchorus
  % Image downloaded from: https://imgbin.com/png/VxTaSrxf/shiva-hanuman-art-ganesha-sai-baba-of-shirdi-png
  % Image license: Free for non-commercial use
  \imagecc[3]{shiva_bw_transparent_bg_760x859px.png}%
\endsong


\beginsong{Ganesha Mantra \\ Removing of obstacles Mantra }[index={Om Gam Ganapatayei Namaha},by={Prembabanda},tags={Ganesha},ph={I, IV}]
  \showmantra{Om Gam Ganapatayei Namaha}
  \begin{feeler}
    Salutations to the remover of obstacles.
  \end{feeler}
  \begin{explanation}
    This sound formula assists us in the removal of obstacles. In order for that to happen there
    is no need to know the exact nature of the hindrances. Just the awareness and recognition that
    there are obstacles and then chanting this mantra with the intention for resolve is enough.
    This mantra unifies us within. When there is oneness there are no obstacles. This mantra is
    also used for the beginning of any endeavor. Whenever we start anything anew we can bless the
    project with the energy of Ganesh through this mantra.
    \vspace{2em}
    \begin{description}
      \item[Gam:] the seed sound for Ganesh
      \item[Ganapati:] another name for Ganesh --- the Remover of Obstacles, and of Oneness/Unity
      \item[Yei:] a sound that activates shakti/energy
    \end{description}
  \end{explanation}
  \imagecb[2]{ganesha_bw_transparent_background_1280x1232.png}%
  \textnotefornext{song: part A}
  \beginchorus\memorize % memorize chords even though in 'chorus'
    |\[\mnc{E}Em]Om \[^\mn{B}]Parvati Patayei |\[\mnc{A}Bm]Hara Ha\[^\mn{F#}]ra \[^\mn{A}]Hara Ma\[^\mn{B}]ha\[^\mn{A}]de\[^\mn{G}]va
    |\[C]{} Gajana\[D]nam Bu|\[Em]ta
  \endchorus
  \notesoff
  \beginchorus
    |^Ganadi Sevatam |^ Kapitha Jambu
    |^ Phalacha^ru |^Bhakshanam
  \endchorus
  \beginchorus
    |^Umasutam Shoka |^ Vinasha Karakam
    |^ Namami ^Vigneshvara |^Pada Pankajam
  \endchorus
  \textnotefornext{part B}
  \beginchorus
    |\[Em]Om Gam Ganapata|yei Nama\[Bm]ha
    |\[Em]Om Gam Ganapata|yei Nama\[Bm]ha
    |\[G]Om Gam Ganapata|yei Nama\[Bm]ha
    |\[Em]Om Gam Ganapata|yei Nama\[Bm]ha
  \endchorus
  \begin{translation}
    O elephant---faced God, Ganesha,
    You are served by the attendants of Shiva.
    \nextverse
    And you eat forest apples and blackberries.
    \nextverse
    You are \emph{Uma}'s son, the destroyer of sorrows.
    I bow to the lotus feet of the remover of obstacles.
  \end{translation}
  \begin{explanation}
    \begin{description}
      \item[Uma:] ``light'', Lady of the Mountains, also known as \emph{Parvati}
    \end{description}
  \end{explanation}
\endsong


\beginsong{Ganesha Sharanam}[tags={Ganesha},ph={I, IV}]
  \audio[]{https://soundcloud.com/sound-of-light/61-ganesh-sharanam}
  \beginverse
    |\[\mnc{E}Em]Om Bom |Hare
    Na|'maha Shi|\[\mnlow{G}\mnlow{F#}]va\[\mnlow{E}]ya
    |\[C]{} Ganga Par|vati Ma
    |\[D]{} Ganesha |\[B7]Sharanam
  \endverse
\endsong


\beginsong{Jaya Ganesha}[tags={Ganesha},ph={I, IV}]
  \audio[]{https://soundcloud.com/sound-of-light/60-jai-ganesh}
  \beginchorus
    |\[\mnc{C}C]Jaya Ga\[\mn{D}]nesha |\[\mn{E}]jaya Ga\[\mn{F}]nesha |\[\mn{E}]jaya Ga\[\mn{F}]ne\[\mn{E}]sha |\[G]de\[\mn{D}]va
  \endchorus
  \beginchorus
    |\[F]Mata jaki |\[G]Parvati |\[Dm]pita \[G]Maha|\[C]deva
  \endchorus
  \begin{translation}
    Glory to You, O Lord Ganesha!
    \nextverse
    Born of Parvati, daughter of the Himalayas, and the great Shiva.
  \end{translation}
\endsong


\beginsong{Govinda Hari Om}[tags={Vishnu, Krishna},ph={III},key={Am},gk={Am, Gm--C\shrp{}m}]
  % in Am the notes range from A to G'
  \meter{4}{4}
  \beginverse
    |\[\mnc{A}Am]Go\[\mn{E}\mn{D}]vin|\[Dm]da | \[\mn{C}]Ha\[\mn{B}]ri |\[\mnc{A}Am]Om Hari |\[\mnc{B}E]Hari | \altchords{\id[1]{(Bm)}|Bm |Em | - |Bm |F\shrp{}}
    |\[Am]Gopa|\[Dm]la | Hari |\[Am]Om|\[G]{} | \altchords{|Bm |Em | - |Bm |A}
    |\[C]Sada |\[Dm]Sadhana | Ananda |\[Am]Bhavana| \altchords{|D |Em | - |Bm}
    \endverse\glueverses\beginchorus
    |\[Dm]Vishnu |\[Am]Sadhana |\[E]Hari |\[Am]Om|\altchords{|Em |Bm |F\shrp{} |Bm}
  \endchorus
\endsong


\beginsong{Guru Brahma}[by={Adi Sankaracharya},tags={teacher},ph={II, III}]
  \audio[]{https://soundcloud.com/bastiaan-yansa/guru-brahma}
  \transpose{5}
  \beginverse
    \[\mn{E}]Gur|\[C]ur Brahm|\[Am]a Gur|\[\mnc{F#}B7]ur \[\mn{E}]Vishn|\[\mnc{G}Em]u
    Gur|\[C]ur Dev|\[Am]o Mah|\[B7]eshwar|\[Em]ah
    Gu|\[C]ru Saak|\[Am]shaat Par|\[B7]a Brahm|\[Em]a
    Tas|\[C]mai S|\[Am]hri Guruv|\[B7]e Namah|\[Em]a | \e
  \endverse
  \begin{translation}
    Guru that is Brahma, Guru that is Vishnu,
    Guru that is Lord Maheshwara \emph{(Shiva)}.
    Guru that is verily the supreme reality.
    Sublime prostrations to that Guru.
  \end{translation}
  \begin{explanation}
    This \emph{shloka} (category of verse line developed from the Vedic Anuṣṭubh poetic meter)
    is by \emph{Adi Sankaracharya} (788--820), a Hindu mystic, as a part of \emph{Guru strotam},
    a sacred prayer dedicated to his spiritual guide \emph{Govinda Bhagwadpada}.
  \end{explanation}
\endsong


\beginsong{Om Shanti Om \\ Peace Mantra}[tags={peace},ph={I, II},key={Am},gk={Am, Am--D\shrp{}m}]
  % in Am the notes range from G to F'
  \mnbeginchorus
    |\[\mnc{A}Am]Om |\[\mnc{G}Em]Shan\[\mn{C}\mn{B}]ti |\[\mnc{A}Am]Om | \e \altchords{\id[1]{(Bm)}|Bm |F\shrp{}m |Bm | \e}
    \mnendchorus\glueverses\mnbeginverse
    \[\mn{A}]Om |\[\mnc{D}Dm]Shanti |\[\mnc{F}Dm/F]Shanti |\[\mnc{E}Am/E]Shan\[\mn{D}]ti\[\mn{E}]hi | \e \altchords{|Em |Em/G |Bm/F\shrp{} | \e}
  \mnendverse
  \notesoff
  \textnotefornext{outro:}
  \beginverse
    |\[Am]Om |\[Em]Shanti |\[Am]Om | \e \altchords{|Bm | F\shrp{}m | Bm | \e}
  \endverse
  \begin{feeler}
    Peace in my heart, peace with each other, peace in the cosmos.
  \end{feeler}
\endsong


\beginsong{Sudhossi Budhossi \\ Forever Pure}[by={trad., Shimshai},ex={saṃskṛtam, english, español}, tags={liberation, transcendence},ph={III}]
  \audio[]{https://www.youtube.com/watch?v=AMH7WaiLqWk}
  \newchords{sudhossi_a}\newchords{sudhossi_b}
  \meter{3}{4}
  \beginverse\memorize[sudhossi_a]
    \[^\mn{B}]Su|\[\mnc{E}Em]dhos\[^\mn{F#}]si \[^\mn{E}]bu|\[\mnc{G}C]dhossi ni|\[\mnc{A}D]ran\[^\mn{G}]ja\[^\mn{F#}]no|\[\mnc{E}Em]si
    Sam|\[Am]sara may|\[Em]a pari|\[D]var jito|\[Em]si
  \endverse\glueverses\beginchorus\memorize[sudhossi_b]
    Sam|\[Am]sara swapa|\[Em]nam tria|\[Am]ja mohan ni|\[Em]dram
    Na|\[Am]jamna mri|\[C]tyor tat|\[D]sat swa ru|\[Em]pe
  \endchorus
  \notesoff
  \beginchorus
    \ind |\[D]Na na na\ldots |\[Em]{} |\[D]{} |\[Em]
  \endchorus
  \beginverse\replay[sudhossi_a]
    You |^are forever |^pure, you |^are forever |^true
    And the |^dream of this |^world can |^never touch |^you
  \endverse\glueverses\beginchorus\replay[sudhossi_b]
    So |^give up your at|^tachment and |^give up your con|^fusion
    And |^fly to that |^space that's be|^yond all il|^lusion
    % % Alternate last line (both been used by the author):
    %And a|^bide in the |^truth that's be|^yond all il|^lusion
  \endchorus
  \goto{Na na na}
  \beginverse\replay[sudhossi_a]
    E|^res siempre |^puro e|^res verda|^dero
    Y el |^sueño del |^mundo no |^te toca|^rá
  \endverse\glueverses\beginchorus\replay[sudhossi_b]
    De|^ja los a|^pegos de|^ja la confu|^sión
    Y vi|^ve en la ver|^dad más al|^lá de la ilu|^sión\goto{Na na na}
  \endchorus
  \begin{feeler}
    It is said that this Sanskrit mantra was originally sung every night as\\
    a lullaby by an Indian mother to her 12 children who all became sadhus.
  \end{feeler}
\endsong


\beginsong{Dhanvantre Mantra \\ Healing Mantra}[index={Om Shree Dhanvantre},tags={health},ph={III}]
  \showmantra{Om Shree Dhanvantre Namaha}
  \begin{feeler}
    Salutations to the being and power of the Celestial Healer.
  \end{feeler}
  \begin{explanation}
    \textbf{Dhanvantari} is the celestial healer. This mantra helps us find the right path to 
    healing, or directs us to the right health practitioner. In India it is also commonly chanted 
    during cooking in order for the food to be charged with healing vibrations – either to prevent 
    disease or assist in healing for those who are sick. This mantra can be chanted for any 
    situation that one would like to be healed or remedied. Good to remember and be open to the 
    path of healing not necessarily looking the way we expect it!
  \end{explanation}
\endsong


\beginsong{Om Namo Bhagavate Vasudevaya}[tags={liberation},ph={II, III}]
  \showmantra{Om Namo Bhagavate Vasudevaya}
  \begin{feeler}
    Salutations to the Indweller who is omnipresent, omnipotent, immortal and divine.
  \end{feeler}
  \begin{explanation}
    \textbf{Vasudeva} is the individual aspect of God that dwells inside of us. This mantra frees 
    our minds and spirits from negative patterns in this life. Regular and consistent practice of 
    this mantra gives us a complete spiritual freedom: it frees us from the cycle of rebirth and 
    helps us realize ourselves as a manifestation of transcendent divinity. It can also help bring 
    in an advanced spiritual soul if chanted by the mother during pregnancy.
  \end{explanation}
\endsong


\beginsong{Hari Om Shiva Om\ldots \\ Cosmic vibration Mantra}[tags={Vishnu, Shiva},ph={II}]
  \showmantra{Hari Om Shiva Om Shiva Om Hari Om}
  \begin{explanation} 
    \textbf{Hari} is another name of Lord Vishnu. Can also be translated as The Remover of ego. 
    Universal mantra of cosmic vibration.
  \end{explanation}
\endsong


\beginsong{Om Eim Saraswatyei Namaha}[tags={learning},ph={II}]
  \showmantra{Om Eim Saraswatyei Namaha}
  \begin{explanation}
    Salutations to Saraswati, the goddess of music, poetry, the arts, education, 
    learning and divine speech. Opens us towards education, learning, and the artistic world of 
    music and poetry. Whenever you find yourself moved to tears by a piece of music, or touched 
    by the words of the great poets and sages, you are in the presence of Saraswati. May we be 
    at ease while learning the wonders of the unfolding mystery of Life.
  \end{explanation}
\endsong


\beginsong{Om Namo Narayana}[tags={transcendence},ph={II}]
  \showmantra{Om Namo Narayana}
  \begin{explanation}
    I bow to the divine. Salutes the all-pervading aspect of the Great Spirit anchored 
    in our hearts and in all beings. Destroys barriers, obstacles, afflictions, and difficulties. 
    Leads to self-realisation. Traditionally chanted to assist the dying as they make their 
    transition, the mantra asks prayerfully, that we may all merge into the grace of divine light.
  \end{explanation}
\endsong


% Image to show on the empty recto page (remove if no longer empty!)
\begin{intersong}%
  \subsection*{Chakras}
  \imagecc[1]{Chakras_map_997x1132px.png}%
  \begin{description}
    \item[Sahasrāra:] ``thousand-petaled'', the crown chakra
    \item[Ājñā:] ``command'', the third eye chakra
    \item[Viśuddha:] ``especially pure'', the throat chakra
    \item[Anāhata:] ``unstruck'', the heart chakra
    \item[Maṇipūra:] ``jewel city'', the solar plexus chakra
    \item[Svādhiṣṭhāna:] ``one's own base'', the sacral chakra
    \item[Mūlādhāra:] ``root support'', the root chakra
  \end{description}
\end{intersong}


\beginsong{Rigveda: Soma}[by={translated to Finnish by Klaus Karttunen},ex={from Rigveda (book 8, hymn 48), ca 1000 BCE}]
  \chordsoff % no vertical space for non-existing chords
  \normalsize % to fit on one spread
  {\noindent\textbf{Somalle:}}\vspace{2em}
  \beginverse
    Makeasta ravinnosta olen ollut osallinen, viisaana,
    hyvin huolehtivasta, parhaan vapauden löytäjästä,
    jonka luo kokoontuvat jumalat ja ihmiset,
    medeksi he sitä nimittävät.
  \endverse
  \beginverse
    Kun olet tunkeutunut sisään, olet kuin Aditi,
    jumalallisen vihan karkottaja,
    oi mehu, Indran kumppanuudesta iloiten
    aja meidät rikkauteen kuin tottelevainen tamma valjaisiin.
  \endverse
  \beginverse
    Olemme juoneet Somaa, olemme tulleet kuolemattomiksi,
    olemme menneet valoon, olemme löytäneet jumalat.
    Mitä nyt meille tekisi vihamielisyys?
    Mitä kuolevaisen pahuus, oi kuolematon?
  \endverse
  \beginverse
    Siunaukseksi tule juotuna sydämellemme, oi mehu,
    ystävällinen ole kuin isä pojalle, oi Soma,
    ymmärtävä kuin ystävä ystävälle, laajamaineinen,
    pidennä elämäämme, Soma, elääksemme.
  \endverse
  \beginverse
    Juotuani nämä ihanat, vapauttavat mehut,
    niveliäni ne sitovat kuin vaunuja nahkahihnat,
    suojelkoot ne minua jalan murtumalta,
    pitäkööt mehut minut erossa katkeamisesta.
  \endverse
  \beginverse
    Sytytä minut kuin sytytetty tuli,
    tee kauasnäkeväksi, tee meidät paremmiksi,
    sillä silloin sinun juopumuksessasi, Soma,
    kuin rikkaana itseäni pidän --- etene menestykseksi.
  \endverse
  \beginverse
    Sinusta puserretusta, innokkain mielin
    nauttisimme kuin perintöomaisuudesta;
    kuningas Soma, pidennä elämäämme,
    kuin Aurinko keväisiä päiviä.
  \endverse
  \beginverse
    Kuningas Soma, ole meille lempeä onneksemme,
    sinun palvojiasi me olemme, tiedä se!
    Nousee kyky ja into, oi mehu,
    älä anna meitä pois vihollisen tahdosta.
  \endverse
  \beginverse
    Sillä sinä olet ruumiittemme suojelija, Soma,
    miesten valvojana olet asettunut kaikkiin jäseniin,
    jos me sinun lupauksesi rikkoisimme,
    ole meille lempeä, hyvä ystävä onneksemme, oi jumala.
  \endverse
  \beginverse
    Olisimmepa lempeän ystävämme kanssa,
    älköön hän minua juotuna vahingoittako, keltakasvoinen,
    tämä Soma, joka on asetettu meihin,
    siksi menen pyytämään Indralta elämäni pidennystä.
  \endverse
  \beginverse
    Pois ovat menneet uupumukset, sairaudet
    vavisten hävisivät, pimentäjättäret ovat pelästyneet,
    mahtava Soma on meihin noussut,
    olemme tulleet sinne, missä elämä pidentyy.
  \endverse
  \beginverse
    Mehu joka juotuna sydämiimme, oi isät,
    kuolematon kuolevaisiin on saapunut,
    sitä Somaa vuodatuksi haluamme kunnioittaa,
    olla hänen armossaan ja suopeamielisyydessään.
  \endverse
  \beginverse
    Sinä, Soma, olet yhdessä isien kanssa
    ulottanut itsesi halki taivaan ja maan,
    sellaista sinua me vuodatuksin haluamme kunnioittaa,
    oi mehu, olisimmepa rikkauksien valtiaita.
  \endverse
  \beginverse
    Puhukaa puolestamme, te suojelevat jumalat,
    älköön meitä uni kukistako, älköön tyhjä puhe,
    me Somalle joka päivä rakkaina
    kokouksessa haluamme puhua hyvine poikinemme.
  \endverse
  \beginverse
    Sinä meille, Soma, joka puolelta voiman antaja,
    sinä valon löytäjä, astu meihin, miesten valvoja,
    sinä meitä, oi mehu, apulaistesi kanssa
    suojele takaapäin ja myöskin edestäpäin.
  \endverse
  \begin{explanation}
    \begin{description}
      \item[Rigveda] on kokoelma varhaisia intialaisia uskonnollisia hymnejä. Sitä pidetään
        vanhimpana Intian pyhistä teksteistä; se lienee laadittu joskus 1700--1000 eaa.
      \item[Soma] on kasviperäinen enteogeeninen rituaalijuoma, jonka sisältämistä kasveista
        ei ole varmaa tietoa. Rigvedassa Somalle on osoitettu lukuisia hymnejä.
    \end{description}
  \end{explanation}
\endsong


%%%%%%%%%%%%%%%%%%%%%%%%%%%%%%%%%%%%%%%%%%%%%%%%%%%%%%%%%%%%%%%%%%%
%%% LATEST PRINTOUT CONTAINED THE SONGS ABOVE.                  %%%
%%%%%%%%%%%%%%%%%%%%%%%%%%%%%%%%%%%%%%%%%%%%%%%%%%%%%%%%%%%%%%%%%%%
%%% Please try to not change the song numbers above this point. %%%
%%% Add new songs only after this point.                        %%%
%%%%%%%%%%%%%%%%%%%%%%%%%%%%%%%%%%%%%%%%%%%%%%%%%%%%%%%%%%%%%%%%%%%


    \beginsong{Compassion Mantra \\ Mantra of Avalokiteshvara}[index={Om Mani Padme Hum},tags={compassion 1},ph={I, II}]
  \showmantra{Om Mani Padme Hum}
  {\small\textnote{Tibetan:} }
  % move the next one up (this is a special case, using two \showmantra after each other):
  \vspace{-2em}
  \showmantra{Om Mani Peme Hung}
  \begin{feeler}
    OM,\\
    the jewel (method; MANI)\\
    in the lotus (wisdom; PADME)\\
    indivisible (HUM).\\\vspace{1em}
    Hail to the Jewel in the Lotus.
  \end{feeler}
  \begin{explanation}
    % From a lecture given by The Dalai Lama at the Kalmuck Mongolian Buddhist Center, New Jersey:
    \textbf{The 14th Dalai Lama:} ``It is very good to recite the mantra OM MANI PADME HUM, but
    while doing it, think the meaning of the six syllables which is great and vast.\par
    The first, OM [\ldots] symbolizes the practitioner's impure body, speech, and mind; it also
    symbolizes the pure exalted body, speech, and mind of a Buddha. [\ldots]\par
    The path is indicated by the next four syllables. MANI, meaning jewel, symbolizes the
    factors of method: the altruistic intention to become enlightened, compassion, and
    love. [\ldots]\par
    The two syllables, PADME, meaning lotus, symbolize wisdom. Just as a lotus grows forth
    from mud but is not sullied by the faults of mud, so wisdom is capable of putting you in
    a situation of non-contradiction where as there would be contradiction if you did not have
    wisdom. [\ldots]\par
    Purity must be achieved by an indivisible unity of method and wisdom, symbolized by the
    final syllable HUM, which indicates indivisibility. [\ldots]\par
    Thus the six syllables mean that in dependence on the practice of a path which is an
    indivisible union of method and wisdom, you can transform your impure body, speech, and
    mind into the pure exalted body, speech, and mind of a Buddha. [\ldots]''
    % % Commented out for ..mm.. reasons
    %\textbf{Lama Thubken Trinley:} ``These six syllables prevent rebirth into the six realms of
    %cyclic existence. It translates as 'OM the jewel in the lotus HUM'. OM prevents rebirth
    %in the God realm, MA prevents rebirth in the Asura (Titan) realm, NI prevents rebirth in
    %the Human realm, PA prevents rebirth in the Animal realm, ME prevents rebirth in the
    %Hungry Ghost realm, and HUM prevents rebirth in the Hell Realm.''
  \end{explanation}
\endsong


\beginsong{Jewel in the Lotus Flower}[index={Om Mani Padme Hum},ph={I, II}]
  \meter{4}{4}
  \beginverse
    \[\mn{A}]There's \[^\mn{C}]a |\[\mnc{D}Dm]jewel in the Lotus |flower
    Unfolding |\[C]deep with\[Am]in my |\[Dm]soul
    To be a |jewel in a Lotus |flower
    Unfolding |\[C]is the \[Am]highest |\[Dm]goal
  \endverse
  \notesoff
  \beginchorus
    ^Hari |^Om Mani Padme |Hum
    Om Mani |^Om Mani ^Padme |^Hum
  \endchorus
  \imagecc[1]{om_mani_padme_hum_script_bw_transparent_bg_2000px.png}
\endsong


\beginsong{Padmasambhava Mantra \\ Vajra Guru Mantra}[index={Om Ah Hum},ph={I, II}]
  \showmantra{Om Ah Hum Vajra Guru Padme Siddhi Hum}
  {\small\textnote{Tibetan:} }
  % move the next one up (this is a special case, using two \showmantra after each other):
  \vspace{-2em}
  \showmantra{Om Ah Hung Benza Guru Peme Siddhi Hung}
  \vspace{2em}
  \textnote{song:}
  \beginchorus
    |\[\mnc{B}Em]Om A\[\mn{A}]h |\[\mn{B}]Hum |Vajra \[\mn{C}]Gu\[\mn{A}]ru |\[Asus2]Padme \[\mn{C}]Sid\[\mn{D}]dhi |\[\mnc{B}Em]Hum | \e
  \endchorus
  \begin{explanation}
    \textbf{Padmasambhava} was a historical teacher in the 8th century, who is regarded
    as the founder of the Nyingma tradition. He is said to have been a renowned
    scholar, meditator, and magician --- the 'second Buddha' in the minds of many
    in Tibet.
    \begin{description}
      \item Dilgo Khyentse Rinpoche:

        ``It is said that the twelve syllables Om Ah Hum Vajra Guru Padme Siddhi Hum carry
        the entire blessing of the twelve types of teaching taught by Buddha, which are the
        essence of His 84000 Dharmas\ldots''
      \item Jamyang Khyentse Wangpo:

        ``It begins with \textbf{OM AH HUM}, which are the seed syllables of the three vajras (of body,
        speech and mind).

        \textbf{VAJRA} signifies the \emph{dharmakaya} [\emph{Truth body} which embodies the very
        principle of enlightenment and knows no limits or boundaries] since, like the adamantine vajra,
        it cannot be 'cut' or destroyed by the elaborations of conceptual thought.

        \textbf{GURU} signifies the \emph{sambhogakaya} [\emph{body of mutual enjoyment} which is
        a body of bliss or clear light manifestation], which is 'heavily' laden with the qualities of the
        seven aspects of union.

        \textbf{PADME} signifies the \emph{nirmanakaya} [\emph{created body} which manifests in time
        and space], the radiant awareness of the wisdom of discernment arising as the lotus family of
        enlightened speech.

        Remembering the qualities of the great Guru of Oddiyana [Padmasambhava], who is inseparable from these
        three kayas, pray with the continuous devotion that is the intrinsic display of the nature
        of mind, free from the elaboration of conceptual thought.

        All the supreme and ordinary accomplishments — \textbf{SIDDHI} — are obtained through the power of
        this prayer, and by thinking, `\textbf{HUM}! May they be bestowed upon my mindstream, this very
        instant!'''
    \end{description}
  \end{explanation}
\endsong


\beginsong{Om Namo Amitābhaya}[ph={I}]
  \beginchorus
    |\[\bmc\mnc{A}Am]Om na\[\mnc{E}]mo\[\bmadj{-.5ex}] Ami|\[\bmc G]tābhaya\[\bmadj{-.7ex}]
    |\[\bmc C]Buddhaya, \[\bmc Em7]Dharmaya, |\[\bmc Am]Sanghaya\[\bmadj{-.7ex}]
  \endchorus
  \beginchorus
    |\[\bmc Am]Om na\[\bm]mo, |\[\bmc G]Om na\[\bm]mo
    |\[\bmc F]Om na\[\bm]mo Ami|\[\bmc E]tābhaya\[\bmadj{-.7ex}]
  \endchorus
  \begin{explanation}
    \textbf{Amitābha} is the principal buddha in Pure Land Buddhism, a branch of East Asian Buddhism.
    In Vajrayana Buddhism, Amitābha is known for his longevity attribute, magnetising red fire
    element, the aggregate of discernment, pure perception and the deep awareness of emptiness of
    phenomena. According to these scriptures, Amitābha possesses infinite merits resulting from good
    deeds over countless past lives as a bodhisattva named Dharmakāra. Amitābha means ``Infinite Light''
    so Amitābha is also called ``The Buddha of Immeasurable Life and Light''.\\
    Buddhists take refuge in the \emph{Three Jewels} or \emph{Triple Gem}, which are:
    \begin{description}
      \item[\hspace{2em} Buddha:] the fully enlightened one
      \item[\hspace{2em} Dharma:] the teachings (expounded by the Buddha)
      \item[\hspace{2em} Sangha:] the spiritual community (the monastic order of Buddhism that practice the Dharma)
    \end{description}
  \end{explanation}
\endsong


\beginsong{Perfection Mantra \\ Gate Gate}[index={Teyata Gate Gate},ph={II}]
  \showmantra{Teyata Gate Gate Paragate Para Samgate Bodhi So Ha}
  \begin{feeler}
    Gone, gone, gone far beyond to the awakened state.
  \end{feeler}
  \vfill
  \textnote{song:}
  \meter{6}{8}
  \beginchorus
    \[\mn{B}]Ga\[\mn{D}]te |\[\mnc{E}Em]Gate Para|\[D]gate
    Para Sam|\[Bm]gate Bodhi |\[Em]So Ha
  \endchorus
  \beginchorus
    Gate |\[G]Gate Para|\[D]gate
    Para Sam|\[Bm]gate Bodhi |\[Em]So Ha
  \endchorus
  \begin{explanation}
    The path that takes us to enlightenment comprises the six arts of perfection. This mantra
    helps us to be generous, patient, conscientious, diligent, focused and wise.
  \end{explanation}
\endsong


\beginsong{Tara Mantra}[index={Om Tare Tu Tare},by={traditional, Deva Premal},ph={III}]
  \showmantra{Om Tare Tu Tare Ture Mama Ah Yuh Pune Jana Putim Kuru So Ha}
  \begin{feeler}
    The liberator of suffering shines light upon me to create\\
    an abundance of merit and wisdom for long life and happiness.
  \end{feeler}
  \vfill
  \textnote{song:}
  \beginchorus
    \[\mn{E}]Om |\[\mnc{A}Am]Tare Tu |\[Fmaj7]Tare Tu|\[G]re So |\[C]Ha
    Om |\[Dm]Tare Tu |\[Em]Tare Tu|\[Fmaj7]re So |\[Am]Ha
  \endchorus
  \begin{explanation}
    Long life and good health for oneself and others is sought through recitation of this mantra
    thus making one’s life and particularly the spiritual journey meaningful.

    \emph{Tara}, who Tibetans also call \emph{Dolma}, is commonly thought to be a Bodhisattva or
    Buddha of compassion and action, a protector who comes to our aid to relieve us of physical,
    emotional and spiritual suffering.

    Tara has 21 forms, of which two are especially popular among Tibetan people: \emph{White Tara},
    who is associated with compassion and long life, and \emph{Green Tara}, who is associated with
    enlightened activity and abundance.
  \end{explanation}
\endsong


\beginsong{Medicine Buddha Mantra \\ Healing Mantra}[index={Teyata Om Bekanze Bekanze},tags={health 1},ph={III}]
  \showmantra{Teyata Om Bekanze Bekanze Maha Bekanze Radza Samut Gate So Ha}
  \begin{feeler}
    I invoke the healing buddha inside me by going all the way to the supreme heights to remove 
    the pain of illness and spiritual ignorance.
  \end{feeler}
  \vfill
  \textnote{song:}
  \beginchorus
    |\[\mnc{D}Dm]Teyata Om |\[\mnc{F}F]Bekanze Bekanze |\[C] Maha Bekanze
    | Radza Samut Gate |\[Dm]So Ha | \e
  \endchorus
  \begin{explanation}
    The practical purpose of spirituality is to help others deal with their various life issues.
    Sickness represents a major problem. Reciting this mantra may contribute to healing on
    many levels adding to the effectiveness of medical treatment and medicines.
  \end{explanation}
\endsong


\beginsong{Purification Mantra}[index={Om Benza Satto Hung},ph={I}]
  \showmantra{Om Benza Satto Hung}
  \begin{explanation}
    \ldots which is the short version of the 100 syllable Mantra: \\
    OM BENZA SATVO SA MA YA MA NU PALA YA BHENZA SATTO TENO PA TISHTHA DRIDHO ME BHAWA SUTOKHAYO ME 
    BHAWA SUPOKHAYO ME BHAWA ANURAKTO ME BHAWA SARVA SIDDHI ME PRAYACCHA SARVA KARMA SUTSA ME
    TSITTAM SHREYANG KURU HUNG HA HA HA HA HO BHAGWAN SARVA TATHAGATA BENZA MA ME MUCCHA BHENZE 
    BHAWA MAHA SAMAYASATTVA AH HUNG PHET
  \end{explanation}
  \begin{feeler}
    Buddha of Purification within me, embodying all the Buddhas, please protect my resolve to 
    purify all my karmas and always bestow on me the ability to make my mind good, virtuous, 
    auspicious and immeasurably loving with the indestructible strength of a diamond.
  \end{feeler}
  \begin{explanation}
    Even though our potential remains obscure in the darkness of negativity, it need not be
    permanent. This mantra helps transform negative karma created over many lifetimes.  
  \end{explanation}
\endsong


\beginsong{Teacher Buddha Mantra}[index={Om Muni Muni},tags={teacher 1},ph={I}]
  \showmantra{Om Muni Muni Maha Muni So Ha}
  \begin{feeler}
    To the teacher, teacher, the great teacher, I pay homage.
  \end{feeler}
  \begin{explanation}
    Shakyamuni, the historical Buddha, cast as the overall teacher of the tradition, illustrates 
    the point that without a good teacher in the beginning there can be no success in spiritual 
    training. Reciting this mantra therefore helps us find a good teacher to lead us towards 
    clarity of mind and ultimately discovery of our own pure consciousness which is the real guru.
  \end{explanation}
\endsong


\beginsong{Wisdom Mantra}[index={Om Ah Ra Pa Tsa Na},tags={wisdom 1},ph={II}]
  \showmantra{Om Ah Ra Pa Tsa Na Dhi Dhi Dhi\ldots}
  \begin{feeler}
    Amidst the chaos, everything is pure by nature.
  \end{feeler}
  \begin{explanation}
    The pinnacle of spiritual success is to achieve enlightenment. This depends on recognition of 
    our potential. The mantra confirms that each of us has the capacity to replace ignorance with 
    wisdom.
  \end{explanation}
\endsong


\begin{intersong}%
  \imagecc[0]{Prokofiev_-_Golden_Ratio_fractal_bw_transparent_bg_1543x910px.png}%
  {\scriptsize Fractal pattern with Hausdorff dimension \(\frac{log \varphi}{log \sqrt[\varphi]{\varphi}} = \varphi \approx 1.618034 \),
  the Golden Ratio. The construction is similar to Heighway's dragon, except for the similarity
  ratios. It is generated from an IFS consisting of two similarities of ratios: \(r\) and \(r^2\),
  with \(r=\frac{1}{\varphi^{\left(\frac{1}{\varphi}\right)}}\). By: Prokofiev.}
\end{intersong}


    % English (mostly) language songs

\beginsong{We Are One in Harmony}[by={Michael Stillwater}]
  \beginchorus
    |\[G]We are one in |\[F]harmony |\[C]singing in cele|\[G]bration|
    |\[G]We are one in |\[F]harmony |\[C]singing in |\[G]love|
  \endchorus
  \beginverse
    We are |\[G]o|\[F]ne |\[C]singing in cele|\[G]bration
    We are |\[G]o|\[F]ne |\[C]singing in |\[G]love|
  \endverse
\endsong

\scleardpage
\beginsong{Mother I Feel You}[by={Dianne Martin}]
  \beginchorus\memorize
    |\[Em]Mother I feel you |\[Bm]under my \[Em]feet|
    |\[Em]Mother I feel your |\[D]heart \[Em]beat|
  \endchorus
  \beginchorus
    |^Heya heya heya heya |^heya heya ^ho|
    |^Heya heya heya heya |^heya ^ho|  
  \endchorus
  \beginchorus
    |^Sister I hear you |^in the river ^song|
    |^Eternal waters |^flowing on and ^on|  
  \endchorus  
  \hfill Heya heya\ldots  
  \beginchorus
    |^Father I see you |^when the eagles ^fly|
    |^Light of the spirit gonna |^take us ^high|
  \endchorus
  \hfill Heya heya\ldots  
  \textnote{suomeksi:}
  \beginchorus
    |^Maa, äitini, sun |^tunnen alla ^jalkojen|
    |^Tunnen ja kuulen, |^sykkeen sun ^sydämen|
  \endchorus
  \beginchorus
    |^Heii-jannaa hoii-jannaa |^heii-jannaa ^hou|
    |^Heii-jannaa hoii-jannaa |^heii ^hou|
  \endchorus
  \beginchorus
    |^Tuli, veljeni, polta |^vanhat liat ^pois|
    |^jotta tilalle |^uutta ^tulla vois|  
  \endchorus  
  \hfill Heii-jannaa...
  \beginchorus
    |^Vesi, siskoni, virtaa |^joet mielee^ni|
    |^Tunnen sun voiman |^vesipisa^rassakin|  
  \endchorus
  \hfill Heii-jannaa...
  \beginchorus
    |^Ilma, isäni, tuule |^läpi ruumii^ni,|
    |^jotta sieluni |^vapaa ^olisi|  
  \endchorus
  \hfill Heii-jannaa...

\endsong


\beginsong{Strong Wind \\ Pure Earth}
  \beginchorus
    Strong |\[Em]wind, deep |\[Bm]water,
    tall |\[Em]trees, warm |\[Bm]fire
    I can |\[C]feel it in my |\[D]body
    I can |\[C]feel it \[Bm]in my |\[Em]soul
  \endchorus
  \beginchorus
    \lrep Heya |\[Em]heya heya heya
    Heya |\[Bm]heya heya ho\rrep\rep{2}
    Heya |\[C]heya heya |\[D]heya
    Heya |\[C]heya \[D]heya |\[Em]ho
  \endchorus
  \beginverse
    \textnote{suomeksi:}
    Puhdas |\[Em]maa, lämmin |\[Bm]tuli,
    kova |\[Em]tuuli, syvä |\[Bm]vesi
    Mä vain |\[C]tunnen ihol|\[D]lani
    ne mä |\[C]tunnen \[Bm]sisäs|\[Em]säin    
  \endverse
  \beginverse
    \small
    \textnote{Alternate lyrics:}  
    Pure earth, warm fire,
    strong wind, deep water
    I can feel them in my body
    I can feel them in my soul
  \endverse
\endsong


\beginsong{One Planet \\ Our Planet is Turning \\ Winter Solstice Chant}
  \beginverse
    |\[Dm] One planet is |turning | circle on her|
    |\[F]path a\[C]round the |\[Dm]sun
    Earth mother is |\[Dm]calling|
    |\[Dm] her \[C]children |\[Dm]home|
  \endverse
  \beginverse
    |^ The light is re|turning | although it|
    |^is the ^darkest |^hour
    No one can |^hold|
    |^ ^back the |^dawn|
  \endverse
  \beginverse
    |^ Let's keep it |burning, | let's keep the|
    |^flame of the ^hope a|^live
    Make safe our |^journey|
    |^ ^through the |^storm|
  \endverse
\endsong

    % Songs in other languages

\beginsong{Ancient Aramaic Prayer}[ph={I}]
  \chordsoff % there are no chords
  \vskip 1em
  \begin{center} % center the lines
    Abwûn d'bwaschmâja
    \vskip 1em
    Nethkâdasch schmach
    \vskip 1em
    Têtê malkuthach.
    \vskip 1em
    Nehwê tzevjânach aikâna d'bwaschmâja af b'arha.
    \vskip 1em
    Hawvlân lachma d'sûnkanân jaomâna.
    \vskip 1em
    Waschboklân chaubên wachtahên aikâna \\
    daf chnân schwoken l'chaijabên.
    \vskip 1em
    Wela tachlân l'nesjuna
    \vskip 1em
    ela patzân min bischa.
    \vskip 1em
    Metol dilachie malkutha wahaila wateschbuchta l'ahlâm almîn.
    \vskip 1em
    Amên.
    \vskip 2em
    % Ancient aramaic symbol:
    %   - upper dot: God (mind)
    %   - left dot:  Son (wisdom)
    %   - right dot: Spirit (life)
    %   - below(?): One Universal God
    % The symbol has been used by ancient Near Eastern scribes
    % to indicate that the writing was of a sacred nature
    % 0.618^4 ~ 0.146
    \includegraphics[width=0.146\textwidth]{ancient_aramaic_symbol_bw_transparent_bg_184x225px.png}
  \end{center}
  \brk % to suggest putting a page break here
  \begin{translation}
    \vskip 8em % try to align with the original prayer
    %\normalsize % scale up from ordinary translation, to align (there is space on the page)
    \begin{center}
      Oh Thou, from whom the breath of life comes,
      who fills all realms of sound, light and vibration.
      \nextverse
      May Your light be experienced in my utmost holiest.
      \vskip 0.5em % to align
      \nextverse
      Your Heavenly Domain approaches.
      \vskip 0.5em % to align
      \nextverse
      Let Your will come true --- in the universe \emph{(all that vibrates)}
      just as on earth \emph{(that is material and dense)}.
      \nextverse
      Give us wisdom \emph{(understanding, assistance)}
      for our daily need.
      \nextverse
      Detach the fetters of faults that bind us \emph{(karma)},
      like we let go the guilt of others.
      \nextverse
      Let us not be lost in superficial things
      \emph{(materialism, common temptations)},
      \nextverse
      but let us be freed from that what keeps us off from
      our true purpose.
      \nextverse
      From You comes the all-working will, the lively strength to act,
      the song that beautifies all and renews itself from age to age.
      \nextverse
      Sealed in trust, faith and truth.
      \emph{(I confirm with my entire being.)}
    \end{center}
  \end{translation}
\endsong


\beginsong{Lecha Eli}[by={Rabbi Avraham Iebn Ezra, Yair Gadassi},ex={hebrew},tags={source 1},ph={II}]
  \beginverse
    |\[Am] \[^\noteU{A}]Lecha \[^\noteU{E}]E|li | teshuka|\[G]ti |
    | Becha chesh|\[Dm]ki |\[Em] ve'ahava|\[Am]ti | |
    |\[Am] Lecha li|bi | vechilyo|\[G]tai |
    | Lecha ru|\[Dm]chi |\[Em] venishma|\[Am]ti | |
  \endverse
  \beginchorus
    \chorusindent |\[Dm] Hashive|ni va'ashu|\[G]va | \textsuperscript{2}( | |)
    \chorusindent | Vetirtzeh |\[Dm] |\[Em]et teshuva|\[Am]ti| |
  \endchorus
  \beginverse
    |^ Lecha ya|dai | lecha rag|^lai |
    | Umimach |^hee |^ techuna|^ti | |
    |^ Lecha atz|mi | lecha da|^mi |
    | Ve'ori |^im |^ geviya|^ti | | \gotochorus{Hashiveni}
  \endverse
  \beginchorus
    |\[Am]Oh |ho oh ho ho ho |\[G]ho | |
    |\[Dm]Oh |\[Em]ho oh ho ho ho |\[Am]ho | |
  \endchorus
  \beginverse
    |^ Lecha ez'|ak | becha ed|^bak |
    | Adei shu|^vi |^ le'adma|^ti | |
    |^ Lecha a|ni | be'odi |^chai |
    | Ve'af ki |^a- |^ charei mo|^ti | | \gotochorus{Hashiveni}
  \endverse
  \begin{translation}
    For You my God is my passion
    In You is my desire and my love
    Yours are my heart and my organs
    Yours are my spirit and my soul
    \nextverse
    \chorusindent Bring me back to You and I will return
    \chorusindent And You shall want my repentance
    \nextverse
    Yours are my hands and legs
    And from You is my character
    Yours are my bones and my blood
    And my skin and my body
    \nextverse
    Oh ho oh ho ho ho ho
    \nextverse
    To You I will call and to You I will cling
    Until I return to my land
    I give myself to You whilst I still live
    And even after I die
  \end{translation}
\endsong


\beginsong{Ishq Allāh\\Love, Lover and Beloved}[by={James Burgess},tags={source 1, love 1},ex={arabic, english},ph={IV}]
  \beginchorus
    \chorusindent |\[Bm\noteULL{B}]Ishq \[\noteU{F#}]Allāh ma'|būd Allāh
    \chorusindent Ishq Al|lāh ma'\[A]būd Al|\[Bm]lāh |
  \endchorus
  \beginverse
    |\[A]God is Love, |\[Bm]Lover and Beloved |
    |\[A] Love, Lover and Be|\[Bm]loved |
    |\[A]I am Love, |\[Bm]Lover and Beloved |
    |\[A] Love, Lover and Be|\[Bm]loved |
  \endverse
  \begin{explanation}
    \begin{description}
      \item[Ishq Allāh ma'būd Allāh] translates literally to "love God adored God"
        which can be interpreted as "God is Love and God is the Beloved" --- and more poetically
        as "God is Love, Lover and Beloved".
    \end{description}
  \end{explanation}
\endsong


\beginsong{Beautiful Names of God}[tags={source 1},ex={arabic},ph={I}]
  \meter{3}{4}
  \beginverse
    \[ .\noteUL{B}]Bis\[\noteU{A}]mil|\[Am]lah, \[ .] \[ .]Al|\[C]lāh, \[ .] \[ .]Raḥ|\[G]mān, \[ .] \[ .]Ra|\[Am]ḥīm \[ .]
    \[ .]Mā|\[ .]lik, \[ .] \[ .]Qud|\[Dm]dūs, \[ .] \[ .]Sa|\[C]laām, \[ .]Mu’\[ .]min, |\[E]Muhay\[ .]min \[ .] | \[ .]- \[ .]
    \[ .]A|\[Am]zīz, \[ .] \[ .]Jab|\[E]bār, \[ .] \[ .]Muta |\[C]kab\[Dm]bir, \[ .]Khā|\[E]liq \[ .] \[ .] | \[ .]- \[ .]
%    % Original(?), rhythmically stranger version below:
%    \[ .\noteUL{B}]Bis\[\noteU{A}]mil|\[Am]lah, \[ .] \[ .]Al|\[C]lāh, \[ .] \[ .]Raḥ|\[G]mān, \[ .] \[ .]Ra|\[Am]ḥīm \[ .]
%    \[ .]Mā|\[ .]lik, \[ .] \[ .]Qud|\[Dm]dūs, \[ .] \[ .]Sa|\[C]laām, \[ .]Mu’\[ .]min, |\[E]Muhay\[ .]min \[ .] |
%    |\[ .] \[ .]A\[Am]zīz, |\[ .] \[ .]Jab\[E]bār, |\[ .] \[ .]Muta \[C]kab|\[Dm]bir, \[ .]Khā\[E]liq | \[ .] \[ .]
  \endverse
  \begin{translation}
    \emph{In Qur'an:} Begin in the name of God, the One, Compassion, Mercy;
    Sovereign, Holy, Peace, Guarantor, Guardian; 
    Allmighty, Powerful, Tremendous, Creator
  \end{translation}
\endsong


\beginsong{Mash Allah}[tags={you 1, source 1},ex={arabic, english},ph={IV}]
  \beginchorus
    \[\noteU{E}]Through \[\noteU{F#}]your |\[Em\noteULL{G}]eyes shines the light
    Mash Al|lah mash Allah |
    |\[D]Wonder of \[B7]God in |\[Em]You
  \endchorus
  \beginverse
    |\[G]Mash Al|\[Am]lah mash Allah |
    |\[D7]Mash Al|\[Em]lah mash Allah |
    |\[G]Mash Al|\[Am]lah mash Allah |
    |\[B7]Wonder of God in |\[Em]You |
    |\[B7]Wonder of God in |\[Em]You |
  \endverse
  \begin{explanation}
    \begin{description}
      \item[Mash Allah] is Arabic and means "as God willed it". It is used to express thankfulness,
        appreciation or joy for what was just mentioned.
    \end{description}
  \end{explanation}
\endsong


\beginsong{Asse Wana Hey Wana \\ Hey Niketi}[ex={hopílavayi, english},tags={heart 1, circle 1},ph={III, IV}]
  \beginchorus
    |\[Em\noteULL{B}]Asse \[\noteU{A}]wa\[\noteU{G}]na |\[Am\noteULL{A}]hey wana |\[D]asse wana |\[Em]hey wana |
  \endchorus
  \notesoff
  \beginchorus
    |\[Em]Hey niketi |\[D]hey wana |\[Bm]hey niketi |\[Em]hey wana |
  \endchorus
  \beginchorus
    |\[Em]Hey sister |\[Am]we are one, |\[D]hey brother |\[Em]we are one |
  \endchorus
  \beginchorus
    |\[Em]No matter |\[D]where we're going to |\[Bm]no matter |\[Em]where we're coming from |
  \endchorus
  \begin{explanation}
    \begin{description}
     \item[Wana] is a Hopi word for "heart". We are all connected in our hearts.
    \end{description}
  \end{explanation}
\endsong


\beginsong{Weha Ehayo}[by={Lakota},ex={lakȟótiyapi, español, english},ph={III}]
  \beginverse % 25 beats in this verse
    \chorusindent \[D\noteUL{D}]Weha eh\[.]ay\[.]o \[A\noteUL{C#}]weha eh\[.]ay\[.]o
    \chorusindent W\[.]eha e\[C]hay\[.]o \[G]weha eh\[.]ay\[.]o
    \chorusindent W\[.]eha e\[C]hay\[.]o \[G]weha \[.]eha\[D]yo! \[.] \[.] \[.] \[.] \[.] \[.] \[.]
  \endverse
  \beginverse\memorize % 36 beats in this verse
    \[D]Gran Esp\[.]írit\[.]u \[A]yo voy \[.]a ped\[.]ir, ó\[.]yem\[.]e \[.] \[.]
    A\[.]l uni\[C]vers\[.]o \[G]yo voy \[.]a ped\[.]ir, ó\[.]yem\[.]e \[.] \[.]
    Par\[.]a mi \[C]puebl\[.]o \[G]que sobrev\[.]iv\[.]a
    y\[.]o he d\[.]icho \[D]hey! \[.] \[.] \[.] \[.] \[.] \[.] \[.] \gotochorus{Weha ehayo}
  \endverse
  \beginverse
    ^Pacham^am^a ^yo voy ^a ped^ir, ó^yem^e ^ ^
    ^a Wira^coch^a ^yo voy ^a ped^ir, ó^yem^e ^ ^
    par^a mi ^puebl^o ^que siempre v^iv^a
    y^o he d^icho ^hey! \[.] \[.] \[.] \[.] \[.] \[.] \[.] \gotochorus{Weha ehayo}
  \endverse
  \beginverse
    ^Great Sp^ir^it ^I am ^going to ^plead, ^hear my ^call ^ ^
    T^o the ^univ^erse ^I am ^going to ^plead, ^hear my ^call ^ ^
    F^or the sur^viv^al ^of our p^eop^le
    ^I am s^aying ^hey! \[.] \[.] \[.] \[.] \[.] \[.] \[.] \gotochorus{Weha ehayo}
  \endverse
  \begin{explanation}
    \begin{description}
     \item[Wiracocha] is the great creator deity in the pre-Inca and Inca mythology in the Andes.
    \end{description}
  \end{explanation}
\endsong


\beginsong{Eskawatã kayawey}[by={pajé Agostinho Inkamuru},ex={hancha kuin},ph={III, IV}]
  \transpose{7}
  \beginchorus\memorize
    |\[Dm] \[^\noteU{A}]Nukun |mãnã Yu\[^\noteU{C}]xi|\[^\noteU{A}]bu bu \[^\noteU{G}]be\[F\noteUL{F}]tã |
    |Eskawatã kaya|\[Am]wey ki|\[Dm]ki | |
  \endchorus\glueverses
  \notesoff
  \beginchorus
    |^ Eskawatã ka|^ya, kaya|wey kaya|^wey, | kayawey ki|^ki | |
  \endchorus
  \beginchorus
    |^ Nukun |\textsuperscript{*}niwe Yuxi|bu bu be^tã |
    |Eskawatã kaya|^wey ki|^ki | |
  \endchorus\glueverses
  \beginchorus
    |^ Eskawatã ka|^ya, kaya|wey kaya|^wey, | kayawey ki|^ki | |
  \endchorus
  \vspace{1em}\hfill{\small \textsuperscript{*} shina, kãna, bari, ushe, vixi, yamã, shava, yura, muká\ldots}
  \begin{explanation}
    This song from the \textit{Huni Kuin (Kaxinawá)} tribe calls the spirits of nature, the elements, the sun, the moon, the stars, and \textit{Yuxibu} to bring transformation.
    \begin{description}
      \item[Eskawatã kayawey:] transformation
      \item[Yuxibu:] the creator (energy)
    \end{description}
  \end{explanation}
  % The version above is the Curawaka's version.
  % See: https://www.youtube.com/watch?v=DU64jmOPL5k
  % Lyrics for another, more original(?) version, go something like this:
  %   Nukun mana Yoxibu, yubã mana Yoxibu, mana Yoxibu bãta,
  %     eskawata kayawê Eskawata kayawá (2x)
  %   Nukun shina Yoxibu, yubã shina Yoxibu, shina Yoxibu bãta,
  %     eskawata kayawê Eskawata kayawá (2x)
  %   Nukun kãna Yoxibu, yubã kãna Yoxibu, kãna Yoxibu bãta,
  %     eskawata kayawê Eskawata kayawá (2x)
  %   Nukun bari Yoxibu, yubã bari Yoxibu, bari Yoxibu bãta,
  %     eskawata kayawê Eskawata kayawá
  %   Nukun ushe Yoxibu, yubã ushe Yoxibu, ushe Yoxibu bãta,
  %     eskawata kayawê Eskawata kayawá
  %   Nukun vixi Yoxibu, yubã vixi Yoxibu, vixi Yoxibu bãta,
  %     eskawata kayawê Eskawata kayawá
  %   Nukun yamã Yoxibu, yubã yamã Yoxibu, yamã Yoxibu bãta,
  %     eskawata kayawê Eskawata kayawá
  %   Nukun shava Yoxibu, yubã shava Yoxibu, shava Yoxibu bãta,
  %     eskawata kayawê Eskawata kayawá
  %   Nukun yura Yoxibu, yubã yura Yoxibu, yura Yoxibu bãta,
  %     eskawata kayawê Eskawata kayawá
  %   Nukun muká Yoxibu, yubã muká Yoxibu, muká Yoxibu bãta,
  %     eskawata kayawê Eskawata kayawá
  % See: https://www.youtube.com/watch?v=1xRclkh6kUg
\endsong


\beginsong{Marirí}[ex={quechua},tags={protection 1},ph={I, II}]
  \beginchorus\meter{5}{4}
    \textsuperscript{*}\[ .\noteUL{G}]Lu\[ .]pu|\[C]nita \[ .]supay \[ .]callam\[ .]puntay \[ .]man|\[Fmaj7\noteUDELTAX{A}{-3em}]tay \[ .\noteUL{G}]mar\[C\noteUL{E}]í
  \endchorus\glueverses
  \beginchorus
    \textsuperscript{*}\[ .]Lu\[ .]pu|\[Am]nita \[ .]supay \[ .]callam\[ .]puntay \[ .]man|\[Dm]tay \[ .]mar\[Am]í \[.]
  \endchorus
  \beginchorus\meter{4}{4}
    \[ .\noteUL{E}]Mari|\[C]rí \[ .]mari\[Fmaj7]rí \[ .]mari|\[Am]rí \[.] \[.] \rep{3}
  \endchorus\glueverses
  \beginverse
    \[.] |\[C] \[.] \[Fmaj7] \[.] |\[Am] \[.] -
  \endverse\chordsoff
  \vspace{1em}\hfill{\small\textsuperscript{*} Irapay, Ashpasuri, Bobinsana, Chiringa, Ayahuasca, Chacrunera,
    Chaliponga, Huanilla, Catahua, Chihuahuaco, Tomapende, Aguaje, Palmicha, Wicungo, Remocaspi,
    Huachumita, Iboga, Peyotito, Hongocitos, Sapotito, Boa Boa, Otorongo, Urcututo, Yanguntoro,
    Aguilita, Condorcito, Lucero, Quillaruna, Chulla Chaqui, Wiracocha, Pachamama, Taita Inti\ldots
  }
  \begin{translation}
    \textsuperscript{*}\textit{Lupunita}, with the tip of my tongue I call on your power.
  \end{translation}
  \begin{explanation}
    \begin{description}
      \item[Marirí] is a powerful healing spirit that lives in \textit{yachay}, the phlegm that
        contains the essence of a curandero's power. The spirit can be passed to another by a
        curandero, who regurgitates it, or by another nature spirit. It is nurtured by
        tobacco smoke. Marirí is very important in the practices of curanderos of the Upper
        Amazon.
      \item[Lupunita:] wolf
      \item[Irapay:] plant spirits
      \item[Ashpasuri:] animal spirits
      \item[Bobinsana \ldots  Sapotito:] plants
      \item[Boa Boa \ldots  Condorcito:] animals
      \item[Lucero:] stars
      \item[Quillaruna:] Moon
      \item[Chulla Chaqui \ldots Taita Inti:] deities
    \end{description}
  \end{explanation}
\endsong


\beginsong{Ide Were}[by={traditional},tags={water 1},ex={yoruba},ph={IV}]
  \beginverse
    \[\noteU{C}]Ide |\[Em\noteULL{E}]were were nita Osh|\[D]un
    Ide |\[C]were were | -
    Ide |\[Em]were were nita Osh|\[D]un
    Ide |\[C]were were nita ya |
    |\[C] Ocha kini|ba nita Osh|\[D]un
    Chek|\[G]e cheke chek|\[C]e nita |\[D]ya
    Ide |\[C]were were | | -
  \endverse
  \begin{explanation}
    This Yoruba chant is dedicated to \textbf{Oshun}, the Goddess of Love,
    happiness and prosperity. She brings to us all the good things of life,
    and is defender of the poor and the mother of all orphans; Goddess
    Oshun brings to them their needs in this life.
    \par
    Oshun (also known as Ochun or Oxum in Latin America) is an orisha, a highly
    benevolent spirit or deity that reflects one of the expressions of God in
    the Ifa and Yoruba religions (Nigeria).
    \par
    Thought to be the most beautiful of the female Orixas. No one can resist
    her charming laugh, her graceful dancing, and her lips that taste like
    honey. She has a lush womanly figure with full hips, which suggest
    fertility and eroticism.
    \par
    She exhibits all of the attributes connected with fresh flowing water:
    lively, sparking, refreshing, vivacious. She is the \underline{Goddess
    of sweet water} and can be discovered where there is fresh water, at
    rivers, ponds, lakes, and particularly waterfalls.
    \par
    She is also a healer of the sick. Teacher, who taught the Yoruba culture,
    agriculture and mysticism. The art of divination using cowrie shells. The
    bringer of song, music and dance, healing chants and meditations taught
    to her by her father Obatala, the first of the created Orishi.
    \par
    This chant speaks of a necklace as a symbol of initiation into love.
    \par
    According to the Yoruba elders, Oshun [also Osun, Oxum] is the "unseen
    mother present at every gathering", because Oshun is the Yoruba
    understanding of the cosmological forces of water, moisture, and
    attraction. Therefore she is omnipresent and omnipotent. Her power is
    represented in another Yoruba scripture which reminds us that "no one is
    an enemy to water" and therefore everyone has need of and should respect
    and revere Oshun, as well as her followers.
    \par
    Oshun is the force of harmony. Harmony we see as beauty, feel as love,
    and experience as ecstasy. Oshun according to the ancients was the only
    female Irunmole amongst the original 16 sent from the spirit realm to
    create the world. As such, she is revered as "Yeye" — the sweet mother
    of us all. When the male Irunmole attempted to subjugate Oshun due to
    her femaleness she removed her divine energy, called ase by the Yoruba,
    from the project of creating the world and all subsequent efforts at
    creation were in vain. It was not until visiting with the Supreme Being,
    Olodumare, and begging Oshun pardon under the advice of Olodumare that
    the world could continue to be created. But not before Oshun had given
    birth to a son. This son became Elegba, the great conduit of ase in the
    Universe and also the eternal and infernal trickster.
    \par
    Oshun is known as Iyalode, the "(explicitly female) chief of the realm."
    She is also known as Laketi, she who has ears, because of how quickly
    and effectively she answers prayers. When she possesses her followers,
    she dances, flirts and then weeps — because no one can love her enough
    and the world is not as beautiful as she knows it could be.
  \end{explanation}

  \begin{center}%
    \vfill%
    \includegraphics[width=0.618\textwidth]{oshun_bw_transparent_bg_650x811px.png}%
    \vfill
  \end{center}
\endsong


\beginsong{O la Mama \\ Ancient Mother}[tags={Divine Mother 1, Mother Earth 1},by={trad. African},ex={some african language, english},ph={I, II}]
  \beginchorus\memorize
    |\[Em] \[^\noteU{E}]O \[^\noteU{B}]la |\[Am\noteULL{A}]Mama, |\[D] wa ha su |\[Em]kola |
    |\[Bm11/D] O la |\[C]Mama, |\[D] wa ha su |\[Em]wam |
    \vspace{1em}
    |\[Em] O la |\[Am]Mama, |\[D] kow wey ha |\[Em]ha ha ha |
    |\[Bm11/D] O la |\[C]Mama, |\[D] ta te ka|\[Em]yee |
  \endchorus
  \notesoff
  \beginchorus
    |^ Ancient |^Mother, |^ I hear you |^calling |
    |^ Ancient |^Mother, |^ I hear your |^song |
    \vspace{1em}
    |^ Ancient |^Mother, |^ I hear your |^laughter |
    |^ Ancient |^Mother, |^ I taste your |^tears |
  \endchorus
  % This second verse is a more unknown addon by somebody:
  \beginchorus
    |^ Ancient |^Mother, |^ I feel you |^calling |
    |^ Ancient |^Mother, |^ I sing your |^song |
    \vspace{1em}
    |^ Ancient |^Mother, |^ I share your |^laughter |
    |^ Ancient |^Mother, |^ I dry your |^tears |
  \endchorus
\endsong


\beginsong{Emamaa \\ Moder Jord}[by={trad., Tane Mahuta},ex={eesti, suomi, svenska; based on a Swedish folk song},tags={Mother Earth 1},ph={III}]
  % TODO: check song origin, find out the full swedish lyrics,
  %       and perhaps better finnish ones as well; ting?
  \textnote{eesti keeles:}
  \beginchorus\memorize
    \lrep \[^\noteU{A}]Ema |\[Dm\noteULL{D}]Maa, ema \[C]Maa, Maa|\[Dm]e\[Am]ma \rrep
    Su |\[F]sees elab \[C]ürgne |\[Dm]vägi mis |toob mul \[Am]lohu|\[Dm]tust
  \endchorus
  \notesoff
  \beginchorus
    |\[Dm]Kummar\[Am]dan su |\[Dm]ees kallis \[Am]ema
    Et |\[F]südames \[C]mind ikka |\[Dm]kannad
  \endchorus
  \textnote{suomeksi:}
  \beginchorus
    \lrep Äiti |^Maa, äiti ^Maa, Maa|^äi^ti \rrep
    Sun |^sisälläs on ^väkevä |^voima, se |minua ^lohdut|^taa
  \endchorus
  \beginchorus
    |\[Dm]Kumar\[Am]ran edes|\[Dm]säsi rakas \[Am]äiti
    Sua |\[F]sydämes\[C]säin aina |\[Dm]kannan
  \endchorus
  \textnote{på svenska:}
  \beginchorus
    \lrep Moder |^Jord, moder ^Jord, Jord|^mo^der \rrep
    Du |^glöder så ^varmt i ditt |^inre, din |hetta ^ger mig |^lust
  \endchorus
\endsong


    % Finnish songs
% =============
%
% The following sets the song number for the first song in this file.
% The number will automatically be incremented by one for each song.
% Please do not change this! Changing would make different versions of
% the songbook to have different numbers for the same songs, and it
% would totally mess up the selection booklets causing them to have
% wrong songs in them. (For the same reason, add new songs only to the
% end of each songs_ file.)
\setcounter{songnum}{600}


\beginsong{Kalevala-sävelmä}[key={Am}, gk={Dm, Gm--Em}]
  % Present the melody on a staff using Lilypond
  \begin{lilywrap}\begin{lilypond}[] \include "tex/lp-include-head.ly"
    theMelody = {
      \key a \minor \time 2/4
      |a'4 a' |b' b' |c'' e'' |b'2 |b'2
      |c''4 a' |d'' c'' |b' c'' |a'2 a'2 \bar "|."
    }
    theLyricsOne = \lyricmode {
      Va -- ka van -- ha Väi -- nä -- möi -- nen,
      tie -- tä -- jä i -- än i -- kui -- nen
    }
    theChords = \chordmode {
     |a2:m |e:m/b |c |e:m/b |e:m
     |a:m |d:m7 |e:m |a:m |a:m
    }
    \include "tex/lp-include-tail.ly" % Don't create guitar tabs
  \end{lilypond}\end{lilywrap}
  %% Commented out for space and boringness reasons
  % \begin{lilywrap}\textnote{Haikea versio:}\begin{lilypond}[]
  %   \include "tex/lp-include-head.ly"
  %   theMelody = {
  %     \key a \minor \time 2/4
  %     |d''4 d'' |d'' a' |d'' c'' |b'2 |b'2
  %     |d''4 d''|d'' c'' |b' c'' |a'2 a'2 \bar "|."
  %   }
  %   theLyricsOne = \lyricmode {
  %     Va -- ka van -- ha Väi -- nä -- möi -- nen,
  %     tie -- tä -- jä i -- än i -- kui -- nen
  %   }
  %   theChords = \chordmode {
  %     |d2:m |d:m |d:m7 |e:m/b |e:m
  %     |d:m | d:m7 |e:m |a:m |a:m
  %   }
  %   \include "tex/lp-include-tail-notab.ly"
  % \end{lilypond}\end{lilywrap}
  \yesendsongvfill% to balance vspace before and after lilypond, as there is no other content
\endsong


\beginsong{Juurilaulu \\ Kuulumme piiriin}[tags={piiri}, ph={I, V}, key={Am}, gk={Bm, Am--G\shrp{}m}]
  % Present the melody on a staff using Lilypond
  \begin{lilywrap}\begin{lilypond}[] \include "tex/lp-include-head.ly"
    theMelody = \relative a' {
      \key a \minor \time 4/4 \partial 2
      \repeat volta 2 {
        a4 a8 a8 | g8( a8) a2.~
        | a2 a4 a8 b | c( a) a2.~
        | a2 a4 a8 b | c c~ c2.~
        | c2 c8 b4 g8 | g4 a2.~ | a2
      }
    }
    theLyricsOne = \lyricmode {
      Kuu -- lum -- me pii -- riin __
      Kuu -- lum -- me pii -- riin __
      Ai -- ko -- jen ta -- kaa __
      Ta -- kai -- sin kier -- toon __
    }
    theChords = \chordmode {
      s2 | a1:m | a:m | a:m | a:m | c | c2 g2 | a1:m | a:m
    }
    \include "tex/lp-include-tail.ly"
  \end{lilypond}\end{lilywrap}
  \yesendsongvfill% to balance vspace before and after lilypond, as there is no other content
\endsong


\beginsong{Lampaanpolska \\ Kekrilaulu \\ Yksi kaksi kolme neljä}[ph={III}, key={Am}, gk={Am, G\shrp{}m--Em}]
  \meter{3}{4}
  \beginverse
    |\[\mnc{A}Am]Yksi kaksi kolme |\[\mnc{B}E]neljä, |\[\mnc{C}Am]anna \[\mn{A}]i\[\mnc{G#}E]lon |\[\mnc{A}Am]olla.
    Ja |\[Am]kun suru |\[E]tulee, |\[Am]anna hä\[E]nen |\[Am]mennä.
  \endverse
  \beginverse
    |\[Am/E]Paarmat ne |\[Dm]laulaa, |\[C]neljä hiirtä |\[E]hyppelee.
    |\[Am]Kissi lyöpi |\[E]trummun päälle ja |\[Am]koko maa\[E]ilma |\[Am]pauhaa.
  \endverse
  % Image downloaded from: https://www.maxpixel.net/Musical-Instruments-Drum-Music-Jazz-Cat-1287910
  \imagecc[2]{cat_drumming_ed_by_larva__aa_transbg_CC0_1264x1788px.png}
\endsong


\beginsong{Laulu oravasta}[by={Otto Kotilainen, Aleksis Kivi}, key={Dm}, gk={Dm, Dm--Em}]
  % NOTE: this is based on the version by Aapo Similä
  \transpose{5} % transpose to Dm: then lowest note is G, the highest A
  \beginverse
    \musicnote{intro:}
    \up{*}\meter{3}{4}|\[C] \[F] \[Em]
  \endverse
  \beginverse\memorize
    \meter{3}{4}|\[\mnc{E}Am]Make\[^\mn{A}]as\[^\mn{C}]ti |\[\mncii{B}{A}Em]ora\[^\mn{G}]vai\[^\mn{E}]nen
    \meter{5}{4}|\[F]Makaa sammalhuonees\[G]sansa;
    \meter{3}{4}|\[Em]Sinnepä ei |\[C]Hallin hammas
    \meter{5}{4}|\[F]Eikä metsä\[Em]miehen \[Am]ansa
    \up{*}\meter{3}{4}|\[C]Ehtineet \[F]milloin\[Em]kaan, |\[C]ei \[F]milloin\[Em]kaan
  \endverse
  \notesoff
  \beginverse
    \meter{3}{4}|^Kammiostaan |^korkeasta
    \meter{5}{4}|^Katselee hän mailman ^piirii,
    \meter{3}{4}|^Taisteloa |^allans' monta;
    \meter{5}{4}|^Havuoksan ^rauhan^viiri
    \up{*}\meter{3}{4}|^Päällänsä ^liepoit^taa. |^ ^ ^
  \endverse
  \beginverse
    \meter{3}{4}|^Mikä elo |^onnellinen
    \meter{5}{4}|^Keinuvassa kehto^linnass'!
    \meter{3}{4}|^Siellä kiikkuu |^oravainen
    \meter{5}{4}|^Armaan kuusen ^äitin^rinnass':
    \up{*}\meter{3}{4}|^Metsolan ^kantele ^soi! |^ ^ ^
  \endverse
  \beginverse
    \meter{3}{4}|^Siellä torkkuu |^heiluhäntä
    \meter{5}{4}|^Akkunalla pienoi^sella,
    \meter{3}{4}|^Linnut laulain |^taivaan alla
    \meter{5}{4}|^Saattaa hänen ^ilta^sella
    \up{*}\meter{3}{4}|^Unien ^Kulta^laan. |^ ^ ^
  \endverse
  \musicnote{\up{*}grave}
\endsong


\beginsong{Taivas on sininen ja valkoinen}[by={Kansanlaulu}, ph={II}, key={Am}, gk={Am, G\shrp{}m--Am}]
  \meter{4}{4}
  % declare new (global) named chord-replay registers:
  \newchords{chords_taivas_a}\newchords{chords_taivas_b}
  \mnbeginchorus\memorize[chords_taivas_a] % memorize chords into a named register
    |\[\mnc{A}Am]Taivas \[^\mn{C}]on \[\bmc\mn{E}]sininen \[^\mn{D}]ja |\[\mncii{C}{B}E7]val\[^\mn{A}\mn{G#}]koi\[\mnc{A}Am]nen \[^\mn{C}\mn{D}]ja
    |\[^\mn{E}]tähtö\[\mnc{F}Dm]si\[^\mn{E}\mn{D}]ä|\[\mnc{E}C]täyn\[E7]nä
    \mnendchorus\glueverses\mnbeginchorus\memorize[chords_taivas_b] % memorize chords into a named register
    |\[\mnc{E}Am]Niin on \[\mnc{B}Bm7&5]nuo\[^\mn{A}]ri |\[\mncii{G}{F}G7]sy\[^\mn{E}\mn{D}]dä\[\mnc{E}C]me\[^\mn{A}\mn{B}]ni
    |\[\mncii{C}{D}Am]a\[^\mn{E}\mn{D}]ja\[\mnc{C}E7]tuk\[^\mn{B}]sia |\[\mnc{A}Am]täyn\[\bm]nä
  \mnendchorus
  \notesoff
  \beginchorus\replay[chords_taivas_a] % replay chords from a named register
    |^Enkä mä ^muille |^ilmoi^{ta mun}
    |sydän^suru|^ja^ni
    \endchorus\glueverses\beginchorus\replay[chords_taivas_b] % replay chords from a named register
    |^Synkkä ^metsä ja |^kirkas ^taivas ne
    |^tuntee mun ^huoli|^a^ni
  \endchorus
\endsong


\beginsong{Haltin häät}[by={Hannu Seppänen, Arto Alaspää}]
  \ifchorded\baselineadj=-.2ex\fi % to fit on one page, TODO: could this be the default?
  \beginverse\memorize
    Kun |\[\mnc{C}C]ihmiskunnan aamu \[^\mn{D}]vas\[^\mn{E}]ta |\[Em]alkoi sarastaa
    Ja |\[F]Lappi oli jättiläisten |\[G]maana
    |\[C]Kaunis Malla-neito alkoi |\[Em]häitään valmistaa
    |\[F]Sulhasenaan nuori uljas |\[G]Saana
  \endverse
  \notesoff
  \beginverse
    |^Kaikkialta kansaa saapui |^Haltiin juhlimaan
    Ja |^kirkonkellot häitä alkoi |^soittaa
    |^Silloin astui kirkkoon tumma |^Pältsa Ruotsinmaan
    Hän |^vaimokseen myös Mallan tahtoi |^voittaa
  \endverse
  \beginverse
    \ind Hän |\[Am]aikoi estää häät ja kutsui |\[Em]velhot avukseen
    \ind Ja |\[Am]pian saikin juhlakansa |\[Em]kuulla kauhukseen
    \ind Kun |\[F]pohjoisesta vyöryi |\[C/G]jää ja yltyi |\[G]tuuli
  \endverse
  \beginverse
    |^Kirkkokansa pakeni ja |^Mallaa sylissään
    Myös |^Saana alkoi juosten turvaan |^kantaa
    He |^kauas eivät ehtineet kun |^jäivät alle jään
    Ja |^jähmettyivät Kilpisjärven |^rantaan
  \endverse
  \beginverse
    \ind On |\[Am]aikakaudet tuntureiksi |\[Em]heidät muuttaneet
    \ind Ja |\[Am]Kilpisjärven kasvattaneet |\[Em]Mallan kyyneleet
    \ind Kun |\[F]jäinen pohjoistuuli |\[C/G]soi myös itkee |\[G]Saana
  \endverse
\endsong


\beginsong{Finlandia-hymni}[by={Jean Sibelius, Veikko Koskenniemi}]
  % TODO: better chords?
  \beginverse
     \[\mnc{F#}D]Oi \[\mnc{E}A]Suo\[\mnc{F#}D]mi, |\[\mnc{G}G]kat\[^\mn{F#}]so, |\[\mnc{E}A]si\[^\mn{F#}]nun \[\mnc{D}G]päi\[^\mn{E}]väs' |\[A]koit\[\mnc{F#}D]taa,
    |\[D] yön \[A]uh\[D]ka |\[G]karkoi|\[A]tettu \[G]on \[A]jo |\[D]pois,
    |\[D] ja \[A/C#]aamun |\[Bm]kiuru |\[D/F#]kirkkaudessa |\[A]soit\[Em]taa
    |\[Em] kuin \[D/F#]itse |\[G]taiva|\[D]han kan\[A]si |\[F#]sois.
    |\[D] Yön \[A/C#]vallat |\[Bm]aamun |\[D/F#]valkeus \[A]jo |voit\[Em]taa,
    |\[Em] sun \[D/F#]päiväs |\[G]koittaa, |\[A7]oi syn\[D]nyin|maa! | \e
  \endverse
  \notesoff
  \beginverse
    ^Oi ^nou^se, |^Suomi, |^nosta ^korke|^al^le
    |^ pääs' ^sep^pe|^löimä |^suurten ^muis^to|^jen,
    |^ oi ^nouse, |^Suomi, |^näytit maail|^mal^le
    |^ sa ^että |^karkoi|^tit or^juu|^den
    |^ ja ^ettet |^taipu|^nut sa sor^ron |al^le,
    |^ on ^aamus |^alka|^nut, syn^nyin|maa! | \e
  \endverse
  % \begin{translation} % comment out: takes too much space
  %   Finland, behold, your day has now come dawning;
  %   Banished is night, its menace gone with light,
  %   Larks' song again in morning-brightness ringing,
  %   Filling the air to heaven's great height,
  %   And morning's glow, night's darkness overcoming;
  %   Your day is come, o my native land.
  %   \nextverse
  %   O Finland, rise, stand proud, the future facing,
  %   Your valiant deeds recalling, once again;
  %   O Finland rise, in the world's sight erasing
  %   From your fair brows vile slavery's stain.
  %   You were not broken by oppressors ruling;
  %   Your morning's come, o my native land.
  % \end{translation}
\endsong


\beginsong{Päivänsäde ja menninkäinen}[by={Reino Helismaa},ph={IV},key={C},gk={C, (B)--(C)}]
  \beginverse
    |\[\mnc{C}Am]Aurin\[^\mn{B}]ko \[^\mn{A}]kun |\[\mnc{D}Dm]päätti \[^\mn{C}]ret\[^\mn{B}]ken, |\[\mnc{C}Am]siskois\[^\mn{B}]taan \[^\mn{A}]jäi |\[\mnc{D}Dm]jälkeen \[^\mn{C}]het\[^\mn{B}]ken
    |\[Am]päivänsäde |\[E7]viimei|\[Am]nen. | \e
    |\[Dm]Hämärä jo |\[Am]metsään hiipi, |\[Dm]päivänsäde |\[Am]kultasiipi
    |\[D]aikoi juuri |\[D7]lentää eestä |\[G]sen, | \e
    kun |\[C]menninkäisen |\[Am]pienen näki |\[Dm]vastaan tule|\[G7]van;
    se |\[C]juuri oli |\[D7]noussut luolas|\[G]taan. | \e
    Kas |\[C]menninkäinen |\[C7]ennen päivän |\[F]laskua ei |\[F#\textdegree7]voi | \e
    mil|\[C]loinkaan elää |\[Dm7]pääl\[G7]lä |\[C]maan. | \e
  \endverse
  \notesoff
  \beginverse
    |^Katselivat |^toisiansa; |^menninkäinen |^rinnassansa
    |^tunsi kummaa |^leiskun|^taa. | \e
    |^Sanoi: poltat |^silmiäni, |^mut' en ole |^eläissäni
    |^nähnyt mitään |^yhtä iha|^naa! | \e
    Ei |^haittaa vaikka |^loisteesi mun |^sokeaksi |^saa;
    on |^pimeässä |^helppo taival|^taa. | \e
    Jää |^luokseni, niin |^kotiluolaan |^näytän sulle |^tien | \e
    ja |^sinut armaak|^se^ni |^vien! | \e
  \endverse
  \beginverse
    |^Säde vastas: |^peikko kulta, |^pimeys vie |^hengen multa,
    |^enkä toivo |^kuole|^maa. | \e
    |^Pois mun täytyy |^heti mennä, |^ellen kohta |^valoon lennä,
    |^niin en hetke|^äkään elää |^saa. | \e
    Niin |^lähti kaunis |^päivänsäde, |^mutta vielä|^kin,
    kun |^menninkäinen |^yksin tallus|^taa, | \e
    hän |^miettii, miksi |^toinen täällä |^valon lapsi |^on, | \e
    ja |^toinen yötä |^ra^kas|^taa. | \e
  \endverse
\endsong


% Force this song on its own page to not have an empty page after
% the song after this one. Remove this when chords are added or the
% songs are rearranged.
\sclearpage
\beginsong{Viatonten valssi}[by={Einojuhani Rautavaara, Eila Kivikk'aho},tags={(chords missing)}]
  \chordsoff % do not show empty line for non-existing chords
  \beginverse
    Kun kesäinen yö oli kirkkain ja tyyninä valvoivat veet
    ja helisi soittimet sirkkain kuin viulut ja kanteleet.
    Viisi pientä piru parkaa aivan ujoa ja arkaa
    sievin kumarruksin tohti käydä enkeleitä kohti.
  \endverse
  \beginverse
    Univormunsa karvaiset heitti, he sarvet ja saparovyön,
    oli lanteilla vain lukinseitti ja helisi harput yön.
    Enkelitkin sulkapaidan jätti tuonne, taakse aidan.
    Siellä häntä, siellä siipi toisiansa tervehtiipi.
  \endverse
  \beginverse
    Ja niinhän he, nostaen jalkaa, niin nätisti tanssia alkaa
    yli kallion kasteisen. Ja se yö oli onnellinen.
    Missäs sika, --- jos ei kerää kärsäänsä se yhtäperää ---
    siivet karvat, ynnä muuta, vielä maiskutellen suuta.
  \endverse
  \beginverse
    Sill' aikaa enkelit tanssi niin ujosti varpaillaan
    Vain pukuna pikkuinen kranssi, viis pirua toverinaan.
    Oi, pienoiset, ettehän arvaa, moni vaihtaa nahkaa ja karvaa.
    Mut harppua sirkat lyö, yhä kun on kesäyö.
  \endverse
  \beginverse
    Kerran tuli Aamunkoitto. Loppui tanssi, loppui soitto.
    Pirut, niinkuin enkelitkin, tunnusmerkkejänsä itki.  
  \endverse
\endsong


\beginsong{Täss' on nainen}[by={Hedningarna},ph={II},key={Am},gk={Am, Am--Em}]
    % TODO: chords (note: melody is good for Kalevala-type stuff)
  \meter{5}{8}
  \mnbeginchorus\memorize
    \lrep \[^\mn{A}]Täss' \[^\mn{G}]on |\[\mnc{A}Am]nainen tuu\[^\mn{G}]len |\[^\mn{A}]tuoma
    \[\mnc{B&}Amadd9-]Tuulen |\[\mnc{C}C]tuoma \[\mnc{B&}C7]ve'en \[^\mn{C}]ve|\[\mnc{A}Am]tämä \rrep
    \lrep \[^\mn{E}]Me\[^\mn{D}]ren |\[\mncii{E}{D}Am/E]aal\[^\mn{E}]to\[\mnc{D}(B&)]jen \[^\mn{E}]a|\[\mnc{A}Am]jama
    \[\mnc{B&}Amadd9-]Meren |\[\mnc{C}C]tyrskyn \[\mnc{B&}C7]työn\[^\mn{C}]te|\[\mnc{A}Am]lemä \rrep
  \mnendchorus
  \notesoff
  \beginchorus\replay
    \lrep Kuin mie |^käynen laula|mahan
    ^Laulan |^mie me^ret me|^siksi \rrep
    \lrep Suoloik|^si me^ren so|^merot
    ^Meren |^hiekat ^herne|^hiksi \rrep
  \endchorus
  \beginchorus\replay
    \lrep Yhen |^vyöni vyötän|nällä
    ^Yhen |^paita^ni pa|^nolla \rrep
    \lrep Solke|^ni so^litta|^malla
    ^Polki|^meni ^paina|^malla \rrep
  \endchorus
  \beginverse\replay
    \lrep Nouse |^luontoni lo|vesta
    ^Synty|^ni sy^västä |^maasta \rrep
    \lrep Synty|^ni sy^västä |^maasta
    ^Haavan |^alta ^halti|^ainen! \rrep
  \endverse
\endsong


\beginsong{Sadelaulu}[by={Sanna Kurki-Suonio},tags={vesi},ph={II},key={\~Dm}, gk={\~Dm, \~Gm--\~Dm}]
  \audio[key=Dm]{https://soundcloud.com/sannakurki-suonio/sadelaulu}
  % in ~Dm, the notes range from D to C'
  \beginverse
    |\[\mnc{D}Dm]Sa\[^\mn{A}]de syöksy\[^\mn{E}]y|\[\mnc{F}F]vi \[^\mn{A}]sy\[\mncii{G}{F}C]li\[^\mn{E}]hin |\[\mnc{D}Dm]pi\[^\mn{A}]saraise\[^\mn{E}]t |\[\mnc{F}F]pai\[^\mn{A}]an \[\mncii{G}{F}C]pääl\[^\mn{E}]le
    |\[Dm]Vesi vihmo|\[F]en ve\[C]tävi |\[Dm]kaiken alleen |\[F]kaste\[C]levi
    |\[Dm]Minä vain sa|\[F]teessa \[C]seison |\[Dm]satehessa |\[F]suloi\[C]sessa
  \endverse
  \notesoff
  \beginverse
    |^Oi kaalinna |^ti-moo-^jaa |^Oi maalinna |^ti-moo-^ja-a-aa
  \endverse
  \beginverse
    |^Vesi mulle |^voiman ^tuopi |^voiman vahvan |^ja vä^kevän
    |^Pyyhkii pois pö|^lyiset ^mietteet |^ajatukset |^auvot^taapi
  \endverse
  \beginchorus\noteson
    \ind |\[\mnc{A}Am]Aaa-\[\mn{B}]aa |\[\mnc{G#}E]Aaa-\[\mn{E}]aa |\[\mnc{A}Am]Aaa-\[\mn{G}]aa-\[\mn{A}]aa-\[\mn{B}]aa |\[\mncii{C}{B}E]Aai-\[\mn{A}\mn{B}]aai-\[\mn{G#}]aaa-\[\mn{E}]aaa
  \endchorus
  \beginverse
    |^Ukko heittävi |^vasa^moitaan |^säästele ei |^sala^moitaan
    |^Minä vain sa|^teessa ^seison |^satehessa |^suloi^sessa
  \endverse
  \beginchorus
    |^Oi kaalinna |^ti-moo-^jaa |^Oi maalinna |^ti-moo-^ja-a-aa
  \endchorus
  \beginchorus
    \ind |\[Am]Aaa-aa |\[E]Aaa-aa |\[Am]Aaa-aa-aa-aa |\[E]Aai-aai-aaa-aaa \rep{4}
  \endchorus
  \beginverse
    |^Puhdista ve|^si puh^dista |^ajatukse|^ni kir^kasta
    |^Syän surusta |^sulat^tele |^tuskan tunteet |^tunnol^tani
    |^Aatteet alhaiset |^aivois^tani |^puhdista pi|^sara ^pieni
  \endverse
  \beginchorus
    \ind |\[Am]Aaa-aa |\[E]Aaa-aa |\[Am]Aaa-aa-aa-aa |\[E]Aai-aai-aaa-aaa
  \endchorus
  \beginverse
    |^Virtaa vesi, |^vihmo ^vesi |^voimaa tuot sä |^mulle ^vesi
    |^Virtaa vesi, |^vihmo ^vesi |^voimaa tuot sä |^mulle ^vesi
  \endverse
  \beginchorus
    |^Oi kaalinna |^ti-moo-^jaa |^Oi maalinna |^ti-moo-^ja-a-aa \rep{5}
  \endchorus % after this there are four measures with just chord C
  % Image downloaded from: https://imgbin.com/png/psfPETyB/water-png
  % Image license: Free for non-commercial use
  \imagecc[4]{water_transparent_bg_254x784px.png}
\endsong


\beginsong{Sisältäni portin löysin}[by={Pekka Streng},ph={III, IV},key={D},gk={D, C--D}]
  % in D the notes range from A to B'
  \beginverse\memorize
    |\[A] \[^\mn{A}]Sisäl|täni \[^\mn{B}]por\[^\mn{A}]tin |\[D]löy\[^\mn{F#}]sin | \e
    |\[A] melkein |huomaamattom|\[D]an. | \e
    |\[G] Kun sen |läpi hiljaa |\[D]nousen, | \e
    |\[G] näen |toisin \[A] maail|\[D]man. | \e
  \endverse
  \notesoff
  \beginverse
    |^ Värit k|auniit vasta h|^uomaan, | \e
    |^ kuulen |äänet kirkkaam|^mat. | \e
    |^ Jätän |soinnuttomat lu|^olat, | \e
    |^ jätän |varjot ^ hoippuv|^at. | \e
  \endverse
  \noteson
  \beginverse
    \ind |\[\mnc{A}A]Aaa\ldots |\[\mn{E}] |\[\mnc{F#}D] |\[\mn{D}] \[\mn{A}] |\[\mnc{C#}A] | |\[D]\[\mn{D}] |\[\mn{F#}] \[\mn{A}]
    \ind |\[A]Aaa\ldots | |\[D] | |\[A] | |\[D] | \e
  \endverse
  \notesoff
  \beginverse
    |^ Jokin |säteilee ja loi|^staa | \e
    |^ alta k|uoren synkänk|^in. | \e
    |^ Kun sen |huomaa kevyem|^min | \e
    |^ ajatuk|set ^ liikkuv|^at. | \e
  \endverse
  \beginverse
    |^ Meidän |värit ylös vi|^rtaa | \e
    |^ ja |yhteen sulaut|^uu. | \e
    |^ Kaikki to|istaan kosket|^taa, | \e
    |^ kaikki a|amuun ^ kurkot|^tuu. | \e \goto{Aaa}
  \endverse
\endsong


\beginsong{Nouse luontoni lovesta}[by={Antti Tuonela},ph={I}]
  \beginverse
    |\[\mnc{A}Dm]Nouse \[\mn{G}]luon\[\mn{F}]toni |\[\mn{G}]lo\[\mn{D}]vesta \echo{|Nouse luontoni |lovesta}
    |Syntyni syvästä |maasta \echo{|Syntyni syvästä |maasta}
    |Nouse niin kuin |nousit ennen \echo{|Nouse niin kuin |nousit ennen}
    |Minun nostate|llessani \echo{|Minun nostate|llessani}
  \endverse
  \beginverse
    Nosta |\[Dm]Ukon voima |taivahas\sublyr{Nosta}ta \echo{|Ukon voima |taivahasta}
    |Maasta Maan E|moisen voima \echo{|Maasta Maan E|moisen voima}
    |Nouse niin kuin |nousit ennen \echo{|Nouse niin kuin |nousit ennen}
    |Minun nostate|llessani \echo{|Minun nostate|llessani}
  \endverse
  \beginchorus
    \ind |\[\mnc{G}E&]Tuekseni turvakseni |väekseni \[\mn{F}]voi\[\mn{E&}]makse|\[\mnc{D}Dm]ni | \e
    \ind |\[E&]Tuekseni turvakseni |väekseni voimakse|\[Dm]ni | \e
  \endchorus
  \textnote{\emph{D.C. al Fine}}
  \beginchorus
    |\[Dm]Nouse luontoni |lovesta \echo{|Nouse luontoni |lovesta}
    |\[Dm]Nouse luontoni |lovesta \echo{|Nouse luontoni |lovesta}
  \endchorus
\endsong


\beginsong{Äidin laulu}[index={Laulan sinulle lapsoseni},by={Marika Salo},tags={Äiti Maa},ph={III}]
  \meter{4}{4}
  \beginverse
    |\[\mnc{A}Am]Laulan \[\mnc{B}G]sinulle |\[Am]lapsoseni |\[Em]laulan sinulle |\[Am]laulun
  \endverse\glueverses
  \beginchorus
    |\[Am]Kuule \[G]minua |\[Am]lapsoseni, kun |\[Em]äitisi laulaa |\[Am]sulle
  \endchorus
  \notesoff
  \beginverse
    |^Missä ^ikinä |^kuljetkin |^siellä olen |^aina
  \endverse\glueverses
  \beginchorus
    |^Olen ^jalkojes |^alla |^metsän puissa ja |^tuulessa
  \endchorus
  \beginverse
    |^Vuoret on ^syntyneet |^kupeistani |^laaksot rintojen |^välistä
  \endverse\glueverses
  \beginchorus
    |^Meret ja ^joet |^kohdustani |^veri on värjännyt |^maan
  \endchorus
  \beginverse
    |^Hyvä sun on ^täällä |^kulkea |^maan ja taivaan |^väliä
  \endverse\glueverses
  \beginchorus
    |^Äitisi ^silittää |^varpaitasi ja |^isäs silittää |^päätä
  \endchorus
  \beginverse
    Ja |^jos sattuis ^lapseni |^käymään niin, että |^ilmaan tipah|^taisit
  \endverse\glueverses
  \beginchorus
    Niin |^älä sinä ^lapseni |^huolta kanna |^isäs ottaa |^kopin
  \endchorus
  \beginverse
    |^Ei ole ^harha-|^askelia |^ei ole |^virheitä
  \endverse\glueverses
  \beginchorus
    |^Kauneutta ^kohti |^kuljet vain täällä |^ikuisessa |^sylissä
  \endchorus
  \beginverse
    Ja |^vielä ^kerron |^sinulle |^kuuntele vielä |^hetki
  \endverse\glueverses
  \beginchorus
    |^Aina oot ^ollut |^toivottu ja |^tänne terve|^tullut
  \endchorus
\endsong


\beginsong{Olkoon niin}[by={Laura Iso-Metsälä},ph={III}]
  \audio[]{https://soundcloud.com/arulai/olkoon-niin-live-in-temppeli}
  \beginverse
    |\[\mnc{B}Bm]Ol\[\mn{C#}]koon |\[\mn{D}]niin, olkoon |\[A]ni\[\mn{C#}]in | \e
    |\[Bm]Olkoon |niin, olkoon |\[A]niin | \e
    |\[Bm]Olkoon |niin, olkoon |\[A]niin | \e
    Että olet |\[Bm]terve, tur|vassa ja |\[A]vapaa | \e
  \endverse
  \beginverse
    Että olet |\[D]onnellin|en ja |\[F#m]rauhallin|en
    Olet |\[A]turvas|sa nyt ja |\[Bm]aina | \e
    Että olet |\[D]onnellin|en ja |\[F#m]rauhallin|en
    Olet |\[A]ter|ve ja |\[Bm]vapaa | \e
  \endverse
  \beginchorus
    \ind Ahee a|\[Bm]hoo, hee a|hoo, hee a|\[A]hoo | \e
    \rep{4}
  \endchorus
  \begin{feeler}
    May all beings be peaceful.\\
    May all beings be happy.\\
    May all beings be safe.\\
    May all beings awaken to\\
    the light of their true nature.\\
    May all beings be free.
  \end{feeler}
\endsong


\beginsong{Koti mun luona}[by={Malla Maanpiiri}, ph={III, IV}, key={G}, gk={G, G--B}]
  \mnbeginchorus\memorize
    |\[\mnc{C}C]Sulla sulla |\[\mnc{G}G]siskokulta on
    |\[\mnc{C}Am]aina koti mun |\[\mnc{G}Em/G]luona | \e
  \mnendchorus
  \notesoff
  \beginchorus\replay
    \ind Me |^ollaan täällä |^näyttämässä,
    \ind |^mitä rakkaus |^on | \e
  \endchorus
  \beginchorus\replay
    |^Sulla sulla |^velikulta on
    |^aina koti mun |^luona | \e
  \endchorus
  \goto{Me ollaan täällä}
  \beginchorus\replay
    |^Sulla sulla |^lapsikulta on
    |^aina koti mun |^luona | \e
  \endchorus
  \goto{Me ollaan täällä}
\endsong


\beginsong{Siunattu voima}[by={Lotta Maija}, ph={III}, key={Am}, gk={Am, Am--Em}]
  \audio[key=Am]{https://www.youtube.com/watch?v=PHNaFQiBNuU}
  \mnbeginverse
    |\[\mncii{A}{B}Am]Illan \[^\mn{A}]suus\[^\mn{G}]sa |\[^\mn{A}]va\[^\mn{B}\mnc{C}]lon \[^\mn{B}]voi\[^\mn{G}]ma, |\emph{\[^\mn{A}\mn{B}\mn{C}]siunat\[^\mn{D}]tu |\[^\mn{E}]voima} \altchords{\id[1]{(Bm)}|Bm | - | - | \e }
    |\[\mnc{D}G]Las\[^\mn{C}]ku \[^\mn{B}]au\[^\mn{A}]rin|koi\[^\mn{B}]sen \[^\mn{A}\mn{G}]illan, |\emph{\[\mnciii{A}{B}{A}\emph{Am}]siunat\[^\mn{G}]tu |v\[^\mn{A}]oima} \altchords{|A | - |Bm | \e}
    |Aurinkoinen |lehvän leuto, |\emph{siunattu |voima} \altchords{|Bm | - | - | \e }
    |\[G]Kesän henki |koivun hento, |\emph{\[\emph{Am}]siunattu |voima} \altchords{|A | - |Bm | \e}
  \mnendverse
  \notesoff
  \beginverse
    |^Vetten päälle |valon loiste, |\emph{siunattu |voima}
    |^Valon loiste |sielun kirkkaus, |\emph{^siunattu |voima}
    |Kirkastelee |kimmellellen, |\emph{siunattu |voima}
    |^Valon kanssa |värähdellen, |\emph{^siunattu |voima}
  \endverse
  \beginverse
    |^Syömeen paistaa |herätellen, |\emph{siunattu |voima}
    |^Herätellen |sytytellen, |\emph{^siunattu |voima}
    |Ikiajan |valon voima, |\emph{siunattu |voima}
    |^Kipunoita |kaukaa tuolta, |\emph{^siunattu |voima}
  \endverse
  \beginverse
    |^Lämmön synnyn |loiste ompi, |\emph{siunattu |voima}
    |^Ajan takaa |ikuisempi, |\emph{^siunattu |voima}
  \endverse
\endsong


\beginsong{Kiitos elämälle}[by={Tiia Ilomäki},tags={kiitollisuus},ph={V}]
  \beginverse
    |\[Am] \[\mn{E}]Tämä on |lau\[\mn{A}]lu e\[\mn{C}]lä|\[Em]mäl\[\mn{B}]le, | \e
    |\[Am] sen valoil|le ja |\[Em]varjoil|le.
    |\[Am] Tämä on |laulu tun|\[Em]teille, | \e
    |\[Am] joskus niin |helvetin |\[Em]tukahdutta|ville.
    |\[Am] | | | \e
    |\[Am] Tämä on |laulu tans|\[Em]sille, | \e
    |\[Am] liik|keelle niin |\[Em]kauniil|le.
    |\[Am] Tämä on |laulu nau|\[Em]rulle, | \e
    |\[Am] het|kille |\[Em]yhtei|sille.
    |\[Am] Tämä on |laulu ihmi|\[Em]sille, | \e
    |\[Am] rak|kaudelle |\[Em]jaetul|le.
    |\[Am] |\[C] |\[Am] |\[C] \e
    |\[Em] | | | \e
  \endverse
  \beginchorus
    \[\mn{B}]Mä laulan |\[G]kii\[\mn{E}]tos | \[\mn{G}]elä|\[\mnc{F#}Em]mäl\[\mn{E}]le | \e
    Mä laulan |\[G]kiitos | tun|\[Em]teille | \e
    Mä laulan |\[G]kiitos | tans|\[Em]sille | \e
    Mä laulan |\[G]kiitos | nau|\[Em]rulle | \e
    Mä laulan |\[G]kiitos | ihmi|\[Em]sille | \e
    |\[C] Olemme |tulleet tänne | luomaan |uutta
    |\[Em] maail|maa | | \e
    \up{*}\echo{maail|\[G]maa, | | | \e
    maail|\[Em]maa, | | | \e
    maail|\[G]maa, | | | \e
    maail|\[Em]maa | | | \e} \altlyr[*]{Vocalize on 2nd repeat only}
  \endchorus
  \beginverse
    |\[Am] Tämä on |laulu elä|\[Em]mälle, | \e
    |\[Am] sen valoil|le ja |\[Em]varjoil|le.
  \endverse
\endsong


\beginsong{Nyt on laulut laulettu}[ex={based on a Hungarian folk song},ph={V},key={Am},gk={Am, Gm--D\shrp{}m}]
  \audio{https://www.youtube.com/watch?v=voIa8mZ3UDc}
  \mnbeginverse
    |\[\mnc{C}Am]Nyt on \[\bmc\mn{B}]laulut |\[^\mn{A}]laulet\[\bm]tu ja |\[\mnc{E}C]lähdet\[\mnc{D}G]tävä |\[\mnc{C}C]on. \[\bm]
    |\[^\mn{E}]Hei \[\mnc{F}Fmaj7]vaan, |\[\mnc{E}C]hei\[^\mn{D}]pä \[\bmc\mn{C}]hei \[^\mn{A}]ja |\[^\mn{C}]lähdet\[\mnc{B}E]tävä |\[\mnc{A}Am]on. \[\bm]
  \mnendverse
  \notesoff
  \beginverse
    |^Toisen ^kerran |tava^tessa |^laulut ^uudet |^on. ^
    |Hei ^vaan, |^heipä ^hei ja |laulut ^uudet |^on. ^
  \endverse
%   % Lilypond notation commented out to save space for this simple song
%   \begin{lilywrap}\begin{lilypond}[] \include "tex/lp-include-head.ly"
%     theMelody = \relative c'' {
%       \key a \minor \time 4/4
%       % https://kansalliskirjasto.finna.fi/Record/fikka.4746565
%       % Unkarilainen kansansävelmä
%       % Suom. sanat: Liisa Tenkku
%       \repeat volta 2 {
%         | c4 c b b | a a a a | e' e d d | c1
%         | e2 f2 | e4. d8 c4 a | c c b b | a1
%       }
%     }
%     theLyricsOne = \lyricmode {
%       \set stanza = "1."
%       | Nyt on lau -- lut | lau -- let -- tu ja | läh -- det -- tä -- vä | on.
%       | Hei vaan, | hei -- pä hei ja | läh -- det -- tä -- vä | on.
%     }
%     theLyricsTwo = \lyricmode {
%       \set stanza = "2."
%       | Toi -- sen ker -- ran | ta -- va -- tes -- sa | lau -- lut uu -- det | on.
%       | Hei vaan, | hei -- pä hei ja | lau -- lut uu -- det | on.
%     }
%     theChords = \chordmode {
%       \repeat volta 2 {
%         | a1:m | a:m | c2 g2 | c1
%         | c2 f2:maj7 | c1 | c2 e2 | a1:m
%       }
%     }
%     \include "tex/lp-include-tail.ly"
%   \end{lilypond}\end{lilywrap}
\endsong


\beginsong{Tupakkarulla}[by={Kansanlaulu},tags={uni}]
  \meter{2}{4}
  \beginverse
    |\[\mnc{D}Dm]Tuu |\[^\mn{A}]tuu |tupakka|\[A]rul|la \brk|\[Gm]mistäs |\[Dm]tiesit |\[A7]tänne |\[Dm]tul|la?
    |\[Dm]Tulin |pitkin |\[Gm7]Turun |\[A]tie|tä, \brk|\[A7]hämä|\[Dm]läisten |\[A7]härkä|\[Dm]tie|tä.
  \endverse
  \notesoff
  \beginverse
    |^Mistäs |tunsit |meidän |^por|tin? \brk|^Siitä |^tunsin |^uuden |^por|tin:
    |^haka |alla, |^pyörä |^pääl|lä \brk|^karhun |^talja |^portin |^pääl|lä
  \endverse
  \beginverse
    |^Uni |kysyi |uunin |^pääl|tä, \brk|^unen |^poika |^porstu|^as|ta:
    |^Onko |lasta |^kätky|^es|sä, \brk|^pientä |^peittei|^den si|^säs|sä?
  \endverse
  \beginverse
    |^Tuoppa |unta |tuokko|^ses|sa, \brk|^kanna |^vaski |^vakka|^ses|sa,
    |^sillä |silmät |^sive|^le,| \brk|^näky|^miset |^näppä|^e|le.
  \endverse
  \beginverse
    |^Nuku |nuku |nurmi|^lin|tu, \brk|^väsy |^väsy |^västä|^räk|ki,
    |^nuku |kun mi|^nä nu|^ku|tan, \brk|^väsy |^kun mi|^nä vä|^sy|tän.
  \endverse
  % Present the melody on a staff using Lilypond
  \begin{lilywrap}\begin{lilypond}[] \include "tex/lp-include-head.ly"
    {\key d \minor \time 2/4
      d'2 | a'2 | a'8 a'4 f'8 | e'2 | e'2
      g'4 g'4 | a'4 f'4 | e'4 f'4 | d'2 | d'2
      f'4 d'4 | e'4 f'4 | g'4 f'4 | e'2 | e'2
      a'4. g'8 | f'4 f'4 | e'4 f'4 | d'2 | d'2 \bar "|."
    }\addlyrics{
      Tuu tuu tu -- pak -- ka -- rul -- la,
      mis -- täs tie -- sit tän -- ne tul -- la?
      Tu -- lin pit -- kin Tu -- run tie -- tä,
      hä -- mä -- läis -- ten här -- kä -- tie -- tä.
    }
  \end{lilypond}\end{lilywrap}
  % Nicely align music notation on both songs of this spread to the bottom.
  % for symmetry. (This is actually the default if Lilypond block is last,
  % but in if it changes, it doesn't have to change here.)
  \noendsongvfill
\endsong


\beginsong{Leppäkerttu}[by={Kansanlaulu},tags={uni},ph={V}]
  \meter{4}{4}
  \beginverse
    |\[\mnc{D}Dm]Lennä, \[^\mn{A}]lennä |\[A]leppäkerttu, |\[Gm]ison \[Dm]kiven |\[A7]juu\[Dm]reen.
    |\[Dm]Lennä leikki|\[Gm7]kedon \[A]kautta |\[A7]unipuuhun |\[Dm]suureen.
  \endverse
  \notesoff
  \beginverse
    |^Kulta-kulta|^lehden alla |^äiti ^puuron |^keit^tää.
    |^Unituutu |^leppä^kertun |^lämpimästi |^peittää.
  \endverse
  \beginverse
    |^Laula, laula, |^unilintu, |^tuoksu, ^tuomen|^tert^tu.
    |^Nuku, puna|^paitu^lainen, |^pikku leppä|^kerttu.
  \endverse
  % Present the melody on a staff using Lilypond
  \begin{lilywrap}
    \imagerb[5]{leppakerttu_transparent_bg_353x279px.png}
    \begin{lilypond}[] \include "tex/lp-include-head.ly"
      {\key d \minor \time 4/4
        d'4 d'4 a'4 a'4 | a'4 a'4 e'4 e'4
        g'4 g'4 f'4 f'4 | e'2 d'2
        f'4 d'4 e'4 f'4 | g'4 f'4 e'4 e'4
        a'4 g'4 f'4 e'4 | d'2 d'2 \bar "|."
      }\addlyrics{
        Len -- nä, len -- nä lep -- pä -- kert -- tu,
        i -- son ki -- ven juu -- reen.
        Len -- nä leik -- ki -- ke -- don kaut -- ta
        u -- ni -- puu -- hun suu -- reen.
      }
    \end{lilypond}
  \end{lilywrap}
  % Nicely align music notation on both songs of this spread to the bottom.
  % for symmetry. (This is actually the default if Lilypond block is last,
  % but in if it changes, it doesn't have to change here.)
  \noendsongvfill
\endsong


\beginsong{Pieni tytön tylleröinen}[tags={uni},key={Dm},gk={Dm, Cm--D\shrp{}m}]
  \audio[key=Dm]{https://www.youtube.com/watch?v=iFEi8XTWSRM}
  \mnbeginverse
    |\[\mnc{D}Dm]Pie\[^\mn{A}]ni \up{*}ty\[^\mn{B&}]tön |\[\mnc{A}Gm]tyl\[^\mn{G}]leröi\[^\mn{E}]nen |\[\mnc{F}Dm]tietä pit\[^\mn{E}]kin |\[^\mn{F}]kul\[^\mn{A}]ki.
    |\[\mnc{F}B&]Saapui sinne |\[Gm]Nuk\[^\mn{E}]ku-\[E7]Mat\[^\mn{D}]ti, |\[\mnc{A}A7]silmät \[^\mn{C#}]pienet |\[\mnc{D}Dm]sulki.
  \mnendverse
  \notesoff
  \beginverse
    |^Kasvoi kuusi |^kukkalatva, |^käki siinä |kukkui.
    |^Mutta \up{*}tytön |^tylle^röinen |^nurmikolla |^nukkui.
  \endverse
  \beginverse
    |^Pieni \up{*}tytön |^tylleröinen |^sievää unta |näki
    |^että hänen |^ympä^rilleen |^tuli metsän |^väki.
  \endverse
  \beginverse
    |^Tapio ja |^Tellervo ja |^Sinipiika |pieni,
    |^Mustikka ja |^Mansik^ka ja |^suuri metsän |^sieni.
  \endverse
  \beginverse
    |^Sipsutteli |^Sinipiika |^pienen \up{*}tytön |luokse;
    |^otti kiinni |^kädes^tä, |^hyppeli ja |^juoksi.
  \endverse
  \beginverse
    |^Eipä \up{*}tytön |^tylleröinen |^ollut mitään |vailla.
    |^Hauska oli |^oles^kella |^Nukku-Matin |^mailla.
  \endverse
  \altlyr{pojan (palleroinen)}
  \imagecc[4]{sleeping_baby_bw_transparent_bg_1280px.png}%
\endsong


\beginsong{Tuuin yössä muukalaista}[by={Neilikka},tags={uni},key={Bm},gk={Cm, Cm--D\shrp{}m}]
  \audio[key=Cm]{https://www.youtube.com/watch?v=OR7YYnrGvPg}
  \meter{3}{4}
  \transpose{2} % in Am the notes range from E to F', in Dm from A to Bb', in Bm from F# to G'
  \mnbeginverse
    |\[\mnc{E}Am]Tuu\[^\mn{C}]in \[^\mn{E}]yössä |\[Dm]muu\[^\mn{D}]ka\[^\mn{A}]laista, |\[\mnc{C}Am]tum\[^\mn{A}]ma\[^\mn{C}]sil\[^\mn{E}]mää |\[\mnc{B}E]lasta \altchords{\id[1]{(Am)}|Am |Dm |Am |E}
    |\[\mnc{D}Dm]Äsken \[^\mn{E}]tän\[^\mn{F}]ne |\[\mnc{E}Am]kut\[^\mn{C}]su\[^\mn{E}]maani, |\[\mnc{E}E]maasta \[^\mn{C}]i\[^\mn{B}]ha|\[Am]nas\[^\mn{A}]ta \altchords{|Dm |Am |E |Am}
  \mnendverse
  \notesoff
  \beginverse
    |^Unenrihmat |^sinne sitoo |^sen ken tuli |^vasta
    |^Nuku paluun |^sinne laulan |^tuulen suhi|^nasta
  \endverse
  \beginverse
    |^Suvisirkan |^soittelosta, |^hämärästä |^illan
    |^Laulan pilven, |^taivaan mieltä, |^laulan seitti|^sillan
  \endverse
  \beginverse
    |^Lapsen käydä |^univarpain |^kohti onnen |^maata \altchords{\id[2]{(Dm)}|Dm |Gm |Dm |A}
    |^Sinne äitis |^murhemieli |^seurata ei |^saata \altchords{|Gm |Dm |A |Dm}
  \endverse
  \beginverse
    \musicnote{interlude:}
    |^ |^ |^ |^
    |^ |^ |^ |^
  \endverse
  \beginverse
    |^Siel ei tunnu |^talven tuskat, |^siel on aina |^kesä
    |^Siellä metsän |^joka puussa |^ilolla on |^pesä
  \endverse
  \beginverse
    |^Polut syliin |^satumetsän |^houkuttaa ja |^hukkuu
    |^Siellä lapsi |^itkutonna |^niittyvillaan |^nukkuu
  \endverse
\endsong


\beginsong{Suojelusenkeli}[by={P. J. Hannikainen},tags={suojelus}]
  \meter{3}{8}
  \beginverse
    \[^\mn{B}]Maan |\[\mnc{E}Em]korvessa |kulkevi |\[B]lapsosen |\[Em]tie.
    \[B]Hänt' |\[Em]ihana |\[G]enkeli |\[D]kotihin |\[G]vie.
    Niin |\[Em]pitkä \[C]on |\[D]matka, \[B]ei |\[Em]kotia |\[B]näy, | \e
    vaan |\[Em]ihana |\[C]enke\[B]li |\[Em]vieres\[Am]sä |\[B]käy,
    vaan |\[Em]ihana |\[Am]enkeli |\[Em/B]vieres\[B7]sä |\[Em]käy.
  \endverse
  \notesoff
  \beginverse
    On |^pimeä |korpi ja |^kivinen |^tie,
    ^ja |^usein se |^käytävä |^liukaskin |^lie.
    Oi, |^pian^han |^lapso^nen |^langeta |^vois, | \e
    jos |^käsi ei |^enke^lin |^kädes^sä |^ois,
    jos |^käsi ei |^enkelin |^kädes^sä |^ois.
  \endverse
  \beginverse
    % original: Ja syntikin mustia verkkoja vaan
    Ja |^mielikin |mustia |^verkkoja |^vaan
    ^on |^laajalle |^laskenut |^korpehen |^maan.
    Niin |^pian^han |^niihin^kin |^tarttua |^vois, | \e
    jos |^käsi ei |^enke^lin |^kädes^sä |^ois,
    jos |^käsi ei |^enkelin |^kädes^sä |^ois.
  \endverse
  \beginverse
    Maan |^korvessa |kulkevi |^lapsosen |^tie.
    ^Hänt' |^ihana |^enkeli |^kotihin |^vie.
    Oi, |^laps' et^hän |^milloin^kaan |^ottaa sä |^vois | \e
    sä |^kättäsi |^enke^lin |^kädes^tä |^pois.
    sä |^kättäsi |^enkelin |^kädes^tä |^pois.
  \endverse
\endsong


\beginsong{En etsi valtaa loistoa}[by={Jean Sibelius, Sakari Topelius}]
  \meter{4}{4}
  \beginverse
    \[^\mn{F#}]En |\[D]etsi \[^\mn{G}]val\[^\mn{F#}]taa, |\[Em]loisto\[A7]a, en |\[D]kaipaa \[A7]kul\[D]taa|\[A]kaan,
    mä |\[Em]pyydän \[A7]taivaan |\[D]valoa ja |rauhaa \[Gm]päälle |\[D]maan.
    Se |\[Em]joulu suo, mi |\[F#\textdegree7]onnen tuo ja |\[B7]mielet nostaa |\[Em]Luojan luo.
    Ei |\[A7]val\[D]taa \[A7]ei\[D]kä |\[Em]kultaa\[A7]kaan, vaan |\[D]rauhaa \[A7]päälle |\[D]maan.
  \endverse
  \notesoff
  \beginverse
    Suo |^mulle maja |^rauhai^sa ja |^lasten ^jou^lu|^puu,
    Ju|^malan ^sanan |^valoa, joss' |sieluin ^kirkas|^tuu.
    Tuo |^kotihin, nyt |^pieneenkin, nyt |^joulujuhla |^suloisin,
    Ju|^ma^lan ^sa^nan |^valo^a ja |^mieltä ^jalo|^a.
  \endverse
  \beginverse
    Luo |^köyhän niinkuin |^rikka^han saa, |^joulu ^i^ha|^na!
    Pi|^mey^tehen |^maailman tuo |taivaan ^valo|^a!
    Sua |^halajan, sua |^odotan, sä |^Herra maan ja |^taivahan.
    Nyt |^köy^hän ^niin^kuin |^rikkaan ^luo su|^loinen ^joulus |^tuo!
  \endverse
\endsong


\beginsong{Kosketa minua henki}[by={Ilkka Kuusisto 1979},ex={Virsi 125},ph={III},key={G},gk={G, E\flt{}--G}]
  % in Bb the notes range from D to D, in transposed G they range from B to B
  \transpose{-3} % to G (3)
  \meter{3}{4}
  \mnbeginverse
    |\[\mnc{D}B&]Kosketa |\[\mnc{F}Dm]minua, |\[\mnc{G}Gm]Hen|\[^\mn{D}]ki, |\[\mnc{E&}Cm7]kos\[^\mn{F}]ke\[^\mn{G}]ta |\[\mnc{A}Dm]kirk\[^\mn{F}]ka|\[\mnc{G}E&]us! |\[G7/D]
    |\[\mnc{G}Cm]An\[^\mn{B&}]na |\[\mncii{D}{C}F7]e\[^\mn{B&}]lä|\[\mncii{A}{F}B&maj7]mäl|\[\mnc{D}G]le |\[\mnc{G}Cm7]suun\[^\mn{F}]ta \[^\mn{G}]ja |\[\mnc{F}F7]tar\[^\mn{E&}]koi|\[\mnc{D}B&]tus. | \e
  \mnendverse
  \notesoff
  \beginverse
    |^Kosketa, |^Jumalan |^Hen|ki, |^syvälle |^sydä|^meen. |^
    |^Sinne |^paina |^hil|^jaa |^luottamus |^rakkau|^teen. | \e
  \endverse
  \beginverse
    |^Rohkaise |^minua, |^Hen|ki, |^murenna |^pelko|^ni. |^
    |^Tässä |^maail|^mas|^sa |^osoita |^paikka|^ni. | \e
  \endverse
  \beginverse
    |^Valaise, |^Jumalan |^Hen|ki, |^silmäni |^aukai|^se, |^
    |^että |^voisin |^ol|^la |^ystävä |^toisil|^le. | \e
  \endverse
  \beginverse
    |^Kosketa |^minua, |^Hen|ki! |^Herätä |^kiittä|^mään, |^
    |^sinun |^lähel|^lä|^si |^armosta |^elä|^mään. | \e
  \endverse
  \beginverse % for alt key (F) chords only
    \altchords{\id[1]{(F)}|F |Am |Dm | - |Gm7 |Am |B\flt{} |D7/A}
    \altchords{|Gm |C7 |Fmaj7 |D |Gm7 |C7 |F | \e}
  \endverse
\endsong


\beginsong{Mörri-Möykky}[by={Marjatta Pokela},ph={IV}]
  \newchords{chords_morrimoykky_a}\newchords{chords_morrimoykky_b}
  \beginverse\memorize[chords_morrimoykky_a]
    |\[\mnc{B}Em]Kor\[^\mn{A}]pi\[^\mn{G}]kuusen |\[\mnc{F#}Am]kannon \[\mnc{E}Em]alla on |\[\mnc{F#}B7]Mörri-\[^\mn{B}]Möy\[^\mn{D#}]kyn |\[\mnc{E}Em]kolo
  \endverse\glueverses\beginchorus\memorize[chords_morrimoykky_b]
    |\[Am]Siellä on koti ja |\[Em]siellä on peti
    ja |\[B7]peikolla pehmoinen |\[\up{1}Em\up{2}(E)]olo
  \endchorus
  \notesoff
  \beginverse
    \ind |\[E]Tiu tau tiu tau |tili tali tittan
    \ind |Sirkat soittaa |\[B]salolla
  \endverse\glueverses\beginchorus
    \ind |\[A]Pikkuiset peikot ne |\[E]piilossa pysyy
    \ind |\[B7]kirkkaalla päivän |\[E]valolla
  \endchorus\glueverses\beginverse
    \ind |\[Am] |\[Em] |\[B7] |\[Em]
  \endverse
  \beginverse\replay[chords_morrimoykky_a]
    |^Syksyn tullen |^sieniä ^kasvaa |^karhunkanka|^halla
  \endverse\glueverses\beginchorus\replay[chords_morrimoykky_b]
    |^Mörri-Möykky se |^sateessa istuu
    |^kärpässienen |^alla \goto{Tiu tau tittan}
  \endchorus
  \beginverse\replay[chords_morrimoykky_a]
    |^Ottaisin minä |^Mörri-^Möykyn, |^jos vain kiinni |^saisin
  \endverse\glueverses\beginchorus\replay[chords_morrimoykky_b]
    |^Pieneen koriin |^pistäisin ja
    |^kotiin kuljet|^taisin \goto{Tiu tau tittan}
  \endchorus
  \beginverse\replay[chords_morrimoykky_a]
    Vaan |^eipä taida |^meidän ^äiti |^peikkolasta |^ottaa
  \endverse\glueverses\beginchorus\replay[chords_morrimoykky_b]
    |^Eikä se edes |^usko, että
    |^Mörri-Möykky on |^totta \goto{Tiu tau tittan}
  \endchorus
\endsong

%%%%%%%%%%%%%%%%%%%%%%%%%%%%%%%%%%%%%%%%%%%%%%%%%%%%%%%%%%%%%%%%%%%
%%% LATEST PRINTOUT CONTAINED THE SONGS ABOVE.                  %%%
%%%%%%%%%%%%%%%%%%%%%%%%%%%%%%%%%%%%%%%%%%%%%%%%%%%%%%%%%%%%%%%%%%%
%%% Please try to not change the song numbers above this point. %%%
%%% Add new songs only after this point.                        %%%
%%%%%%%%%%%%%%%%%%%%%%%%%%%%%%%%%%%%%%%%%%%%%%%%%%%%%%%%%%%%%%%%%%%


\beginsong{Rakastan sinua elämä}[by={Eduard Komanovski, Konstantin Vanshenkin}, key={Am}, gk={Am, (Am)--(Cm)}]
  % Finnish words by: Pauli Salonen
  \newchords{chords_rakastan_elamaa_a}\newchords{chords_rakastan_elamaa_b}
  \mnbeginverse\memorize[chords_rakastan_elamaa_a]
    \[^\mn{E}]Päättyy |\[\mnc{A}Am]yö, \[\bm] \[^\mn{C}]aa\[^\mn{A}]mu |\[\mnc{F}Dm]saa,\[\bm] uusi |\[\mnc{E}E7]päi\[^\mn{D}]vä \[^\mn{C}]kun \[\bmc\mn{B}]kirk\[^\mn{A}]kaa\[^\mn{B}]na |\[\mnc{C}Am]lois\[^\mn{A}]taa.\[\bm]
    \[^\mn{E}]Sulle |\[^\mn{A}]oi, \[\bm] \[^\mn{C}]ko\[^\mn{A}]ti|\[\mnc{A}F]maa, \[\bm] sävel |\[\mnc{G}G7]kau\[^\mn{F}]ne\[^\mn{E}]hin \[\bmc\mn{D}]tuu\[^\mn{C}]les\[^\mn{D}]sa |\[\mnc{F}C]soit\[^\mn{E}]taa.\[\bm]
    \mnendverse\glueverses\mnbeginchorus\memorize[chords_rakastan_elamaa_b]
    \[^\mn{E}]Rakas|\[\mnc{B}E7]tan\[\bm] \[^\mn{A}]e\[^\mn{G#}]lä|\[\mnc{A}F]mää, \[Am\mn{E}]{{ }{ }{ }{ }{ }{ }jo}\[^\mn{F}]ka |\[\mnc{G}Gm6]uute\[^\mn{A}]na \[\mnc{G}A7]aa\[^\mn{F}]mus\[^\mn{E}]sa |\[\mnc{A}Dm]au\[^\mn{D}]kee.\[\bm]
    \[^\mn{E}]Ra\[^\mn{D}]kas|\[\mnc{F}Dm6]tan \[E7\mn{E}]{ }{ }{ }{ }{e}\[^\mn{D}]lä|\[\mnc{C}Am]mää, \[D7\mn{D}]{{ }{ }{ }{ }{ }jo}\[^\mn{B}]ka |\[\mnc{E}Am]uu\[^\mn{C}]pu\[^\mn{A}]en \[\mnc{E}E7]il\[^\mn{C}]las\[^\mn{B}]sa |\[\mnc{C}Am]rau\[^\mn{A}]kee.\[\bm]
  \mnendchorus
  \notesoff
  \beginverse\replay[chords_rakastan_elamaa_a]
    Kirkka|^hin ^ päivä |^ei, ^ aina |^parhainta ^loistetta |^suone.^
    Unel|man ^ usein |^vei, ^ eikä |^ystävä ^lohtua |^tuone.^
    \endverse\glueverses\beginchorus\replay[chords_rakastan_elamaa_b]
    Rakas|^tan ^ elä|^mää, ^{{ }{ }{ }{ }{ }jo}ka |^kyynelten ^helminä |^hohtaa.^
    Rakas|^tan ^ elä|^mää, ^{{ }{ }{ }{ }jo}ka |^myrskyihin ^tietäni |^johtaa.^
  \endchorus
  \beginverse\replay[chords_rakastan_elamaa_a]
    Jäänyt |^on ^ päivän |^työ, ^ ilta |^varjoja ^tielleni |^siirtää.^
    Kaupun|gin ^ sydän |^lyö, ^ valot |^laineille ^siltoja |^piirtää.^
    \endverse\glueverses\beginverse\replay[chords_rakastan_elamaa_b]
    Rakas|^tan ^ elä|^mää, ^{{ }{ }{ }{ }{ }jo}ka |^nuoruuden ^haaveita |^kantaa.^
    Rakas|^tan ^ elä|^mää, ^{{ }{ }{ }{ }jo}ka |^muistojen ^hetkiä |^antaa.^ \replay[chords_rakastan_elamaa_b]
    Rakas|^tan ^ elä|^mää, ^{{ }{ }{ }{ }{ }sil}le |^lempeni ^tahdon mä |^antaa.^
    Rakas|^tan ^ elä|^mää, ^{{ }{ }{ }{ }jo}ka |^muistojen ^hetkiä |^kantaa.^
  \endverse
%   %% Lilypond commented out for space saving reasons in the main songbook
%   \begin{lilywrap}\begin{lilypond}[]
%     % transcribed by larva, latest update on 2023-06
%     \include "tex/lp-include-head.ly"
%     theMelody = \relative c' {
%       \key a \minor \time 4/4 \partial 4
%       \set melismaBusyProperties = #'() \slurDashed
%       e8. \sectionmark "A" e16
%       | a2. c8. a16 | f'2. 8. 16 | e4 d8. c16 b4 a8. b16 | c4 a4 r4
%       e8. e16
%       | a2. c8. a16 | a'2. 8. 16 | g4 f8. e16 d4 c8. d16 | f4 e4 r4 \break
%       \repeat volta 2 {
%         e8 \sectionmark "B" e8
%         | b'2. a8 gis8 | a2. e8 f8 | g4 8. a16 g4 f8. e16 | a4 d,4 r4
%         e8 d8
%         | f2. e8 d8 | c2. d8 b8
%         | e4 c8 a8 e4 c'8. b16 | c4 a4 r4
%       }
%     }
%     theLyricsOne = \lyricmode {
%       \set stanza = "1.A"
%       Päät -- tyy | yö, aa -- mu | saa, uu -- si | päi -- vä kun kirk -- kaa -- na | lois -- taa.
%       Sul -- le | oi, ko -- ti -- | maa, sä -- vel | kau -- ne -- hin tuu -- les -- sa | soit -- taa.
%       \repeat volta 2 {
%         \set stanza = "1.B"
%         Ra -- kas -- | tan e -- lä -- | mää, jo -- ka | uu -- te -- na aa -- mus -- sa | au -- kee.
%         Ra -- kas -- | tan e -- lä -- | mää, jo -- ka | uu -- pu -- en il -- las -- sa | rau -- kee.
%       }
%     }
%     theLyricsTwo = \lyricmode {
%       \set stanza = "2.A"
%       Kirk -- ka -- | hin päi -- vä | ei, ai -- na | par -- hain -- ta lois -- tet -- ta | suo -- ne.
%       U -- nel -- | man u -- sein | vei, ei -- kä | ys -- tä -- vä loh -- tu -- a | tuo -- ne.
%       \repeat volta 2 {
%         \set stanza = "2.B"
%         Ra -- kas -- | tan e -- lä -- | mää, jo -- ka | kyy -- nel -- ten hel -- mi -- nä | hoh -- taa.
%         Ra -- kas -- | tan e -- lä -- | mää, jo -- ka | myrs -- kyi -- hin tie -- tä -- ni | joh -- taa.
%       }
%     }
%     theLyricsThree = \lyricmode {
%       \set stanza = "3.A"
%       Jää -- nyt | on päi -- vän | työ, il -- ta | var -- jo -- ja tiel -- le -- ni | siir -- tää.
%       Kau -- pun -- | gin sy -- dän | lyö, va -- lot | lai -- neil -- le sil -- to -- ja | piir -- tää.
%       \repeat volta 2 {
%         <<
%           {
%             \set stanza = "3.B:i"
%             Ra -- kas -- | tan e -- lä -- | mää, jo -- ka | nuo -- ruu -- den haa -- vei -- ta | kan -- taa.
%             Ra -- kas -- | tan e -- lä -- | mää, jo -- ka | muis -- to -- jen het -- ki -- ä | an -- taa.
%           }
%           \new Lyrics { \set associatedVoice = "theMelody"
%             \set stanza = "3.B:ii"
%             \override LyricText.color = #grey Ra -- kas -- | tan e -- lä -- | mää, \override LyricText.color = #black sil -- le | lem -- pe -- ni tah -- don mä | an -- taa.
%             \override LyricText.color = #grey Ra -- kas -- | tan e -- lä -- | mää, jo -- ka | muis -- to -- jen het -- ki -- ä \override LyricText.color = #black | kan -- taa.
%           }
%         >>
%       }
%     }
%     theChords = \chordmode {
%       s4
%       | a1:m | d:m | e:7 | a:m
%       | a:m | f | g:7 | c2.
%       \repeat volta 2 {
%         s4
%         | e1:7 | f2 a2:m | g2:m6 a2:7 | d1:m
%         | d2:m6 e2:7 | a2:m7 d2:7 | a2:m e2:7 | a2.:m
%       }
%     }
%     %\layout { #(layout-set-staff-size 15) } % for better fit
%     \include "tex/lp-include-tail-lyricsbelow.ly"
%   \end{lilypond}\end{lilywrap}
\endsong

\beginsong{Tuuditan tulisoroista}[by={Kansanlaulu}, key={Em}, gk={Gm, Gm--Bm}, tags={uni}]
  \audio[key=B\flt{}m]{https://www.youtube.com/watch?v=mpHEc8SjfZc}
  \newchords{chords_tulisoroinen_a}\newchords{chords_tulisoroinen_b}
  \capo{3}
  \mnbeginverse\memorize[chords_tulisoroinen_a]
    |\[\mnc{B}Em]Tuuditan \[^\mn{A}]tu|\[^\mn{G}]li\[^\mn{B}]so\[^\mn{E}]rois\[^\mn{F#}]ta, |\[^\mn{G}]kipunaista \[\mnc{A}Am]kiikut|\[\mnc{B}Em]telen \altchords{\id[1]{Am}|Am | - | - Dm |Am}
    \mnendverse\glueverses\mnbeginchorus\memorize[chords_tulisoroinen_b]
    |\[\mnc{B}Em]Vaa\[^\mn{E}]lin \[^\mn{B}]pien\[^\mn{A}]tä |\[^\mn{G}]val\[^\mn{B}]ke\[^\mn{E}]ais\[^\mn{F#}]ta, |\[\mnc{G}G]luojan lasta \[\mnc{F#}Bm]liikut|\[\mnc{E}Em]telen \altchords{|Am | - |C Em |Am}
  \mnendchorus
  \notesoff
  \beginverse\replay[chords_tulisoroinen_a]
    |^Taivahast' on |lahja tullut, |taivahan tu^len ki|^soista
    \endverse\glueverses\beginchorus\replay[chords_tulisoroinen_b]
    |^Luojan lemmen |leikinnöistä, |^ei vahingon ^valke|^oista
  \endchorus
  \beginverse\replay[chords_tulisoroinen_a]
    |^Tuudin lasta |maan valoksi, |en vahingon ^valke|^aksi
    \endverse\glueverses\beginchorus\replay[chords_tulisoroinen_b]
    |^Tuudin toivo|jen tuleksi, |^työn jaloisen ^jatka|^jaksi
  \endchorus
  \beginverse\replay[chords_tulisoroinen_a]
    |^Senpä tuudin |tuikkeheksi, |tähdeksi tä^hän ta|^lohon
    \endverse\glueverses\beginchorus\replay[chords_tulisoroinen_b]
    |^Maan hyväksi |maineheksi, |^suurionnis^ten i|^lohon
  \endchorus
  \begin{lilywrap}\begin{lilypond}[]
    % transcribed by larva, latest update on 2023-07
    \include "tex/lp-include-head.ly"
    theMelody = \relative b' {
      \key e \minor \time 4/4
      \set melismaBusyProperties = #'() \slurDashed
      | b4 b b a | g b e,4. fis8 | g8 g g g a4 a | b2 b2
      \repeat volta 2 {
        | b4 e b4. a8 | g4 b e,4. fis8 | g8 g g g fis4 fis | e2 e2
      }
    }
    theLyricsOne = \lyricmode {
      \set stanza = "1."
      | Tuu -- di -- tan tu -- | li -- so -- rois -- ta, | ki -- pu -- nais -- ta kii -- kut -- | te -- len;
      \repeat volta 2 {
        | Vaa -- lin pien -- tä | val -- ke -- ais -- ta, | luo -- jan las -- ta lii -- kut -- | te -- len.
      }
    }
    theLyricsTwo = \lyricmode {
      \set stanza = "2."
      | Tai -- va -- hast' on | lah -- ja tul -- lut, | tai -- va -- han tu -- len ki -- | sois -- ta;
      \repeat volta 2 {
        | Luo -- jan lem -- men | lei -- kin -- nöis -- tä, | ei va -- hin -- gon val -- ke -- | ois -- ta.
      }
    }
    theLyricsThree = \lyricmode {
      \set stanza = "3."
      | Tuu -- din las -- ta | maan va -- lok -- si, | en va -- hin -- gon val -- ke -- | ak -- si;
      \repeat volta 2 {
        | Tuu -- din toi -- vo -- | jen tu -- lek -- si, | työn ja -- loi -- sen jat -- ka -- | jak -- si.
      }
    }
    theLyricsFour = \lyricmode {
      \set stanza = "4."
      | Sen -- pä tuu -- din | tuik -- ke -- hek -- si, | täh -- dek -- si tä -- hän ta -- | lo -- hon;
      \repeat volta 2 {
        | Maan hy -- väk -- si | mai -- ne -- hek -- si, | suu -- ri -- on -- nis -- ten i -- | lo -- hon.
      }
    }
    theChords = \chordmode {
      | e1:m | e:m | e2:m a2:m | e1:m
      \repeat volta 2 {
        | e1:m | e:m | g2 b2:m | e1:m
      }
    }
    %\layout { #(layout-set-staff-size 15) } % for better fit
    \include "tex/lp-include-tail-notab.ly"
  \end{lilypond}\end{lilywrap}
\endsong


% \beginsong{O Kriste, kunnian kuningas}
%   \audio[key=Am]{https://www.youtube.com/watch?v=P_elenZRgAA}
%   \newchords{chords_okriste_a}\newchords{chords_okriste_b}
%   \meter{3}{4}
%   \beginverse\memorize[chords_okriste_a]
%     O |\[Am]Kriste, |\[C]kunnian |\[Dm]Kunin|\[Am]gas,
%     Ja |\[Dm]lunas|\[Am]taja |\[Em]laupi|\[Am]as!
%     Kuu|\[Am]le si|\[C]nua |\[Dm]rukoi|\[Am]len,
%     Ve|\[Dm]relläs’ |\[Am]ostet|\[Em]tu o|\[Am]len.
%   \endverse
%   \beginverse\memorize[chords_okriste_b]
%     \ind Suu|\[C]ret syn|\[F]tin’ kyl|\[G]läs tien|\[C]net,
%     \ind Joi|\[Dm]ta vas|\[Am]taas teh|\[Em]nyt lie|\[Am]nen,
%     \ind Koht’ |\[Am]äitin’ |\[C]kohdust’ |\[Dm]tultu|\[Am]an’,
%     \ind Ah |\[Dm]armahda |\[Am]päällen’, |\[Em]Juma|\[Am]la!
%   \endverse
%   \beginverse\replay[chords_okriste_a]
%     |^Muista, |^Herra, va|^las pääl|^len,
%     jon|^ka vah|^vast’ van|^noit meil|^len:
%     Et|^tes suo |^syntist’ |^hukku|^van,
%     vaan |^katu|^van ja |^elä|^vän.
%   \endverse
%   \beginverse\replay[chords_okriste_b]
%     \ind Tun|^nustan, |^synti|^nen o|^len,
%     \ind Huo|^kaan, ka|^dun ty|^kös tu|^len,
%     \ind Ja |^anteeks’ |^anta|^vas toi|^von,
%     \ind Mi|^tä vas|^taas ri|^koin vai|^voin.
%   \endverse
%   \beginverse\replay[chords_okriste_a]
%     Ku|^ningas |^kaikki|^valti|^as,
%     O |^Jesu |^aina |^laupi|^as!
%     Kuu|^le mi|^nua |^vaivais|^ta!
%     Ru|^koilen |^sinua |^hartaas|^ta,
%   \endverse
%   \beginverse\replay[chords_okriste_b]
%     \ind Ken |^paits’ |^sua minu|^a kuu|^lee?
%     \ind Ken |^apuun |^paits’ |^sinua tu|^lee?
%     \ind Jos et |^kuule, |^auta |^minu|^a,
%     \ind Ei |^muilta |^ole |^apu|^a.
%   \endverse
%   \beginverse\replay[chords_okriste_a]
%     Ol|^koon si|^nun, Je|^su kii|^tos!
%     Ai|^na ja |^ijät’ |^ylis|^tys,
%     Kuin |^kaikkein |^päälle |^armah|^dat,
%     Jotk’ |^tykös |^hartaast’ |^huuta|^vat.
%   \endverse
%   \beginverse\replay[chords_okriste_b]
%     \ind Se |^sama |^kunnia |^Isäl|^le,
%     \ind Ja |^ynnä |^Pyhäll’ |^Hengel|^le,
%     \ind Yh|^dell’ kol|^minai|^sell’ Her|^rall’,
%     \ind I|^jäisest’ |^aina |^hallitse|^vall’.
%   \endverse
%   \beginverse\replay[chords_okriste_a]
%     Ó |^Cristo, |^Rei da |^Glóri|^a
%     e |^misericordi|^oso |^Reden|^tor!
%     Es|^cute-me |^por |^fa|^vor
%     Fui |^comprado |^com seu |^san|^gue.
%   \endverse
%   \beginverse\replay[chords_okriste_b]
%     \ind Vo|^cê con|^hece meus |^grandes peca|^dos,
%     \ind que eu |^come|^ti con|^tra vo|^cê
%     \ind des|^de que |^saí do ú|^tero da minha |^mãe.
%     \ind Ah, |^tem misericór|^dia de |^mim, ó De|^us!
%   \endverse
%   \beginverse\replay[chords_okriste_a]
%     Lem|^bre-se, |^ó |^Se|^nhor,
%     o |^pacto |^que você |^fez por |^nós:
%     Que |^você não |^vai deixar um |^pecador mo|^rrer,
%     mas |^se ele se a|^rrepender, e|^le vive|^rá.
%   \endverse
%   \beginverse\replay[chords_okriste_b]
%     \ind Eu |^con|^fesso que sou |^um peca|^dor
%     \ind e |^eu suspiro, eu |^me arrependo |^e vou para vo|^cê
%     \ind pa|^ra buscar |^o seu |^perdã|^o,
%     \ind con|^tra o qual |^eu trans|^gre|^di.
%   \endverse
%   \beginverse\replay[chords_okriste_a]
%     Ó Rei Todo-Poderoso,
%     Jesus sempre misericordioso!
%     Ouça este pobre homem!
%     Eu te imploro com fervor,
%   \endverse
%   \beginverse\replay[chords_okriste_b]
%     \ind Quem mais pode me ouvir?
%     \ind Quem mais pode me ajudar?
%     \ind Se você não me ouvir e me ajudar,
%     \ind Ninguém mais pode me ajudar.
%   \endverse
%   \beginverse\replay[chords_okriste_a]
%     Louvado seja você, Jesus!
%     Louvado seja para todo o sempre
%     porque você é misericordioso
%     para todos aqueles que clamam por você.
%   \endverse
%   \beginverse\replay[chords_okriste_b]
%     \ind E glória ao Pai
%     \ind e ao Espírito Santo,
%     \ind para o Deus triuno
%     \ind que reina para sempre.
%   \endverse
% \endsong

% % Commented out, as not needed for page fill currently
% \begin{intersong}
%   % Original image downloaded from: https://commons.wikimedia.org/wiki/File:Harmonic_series_to_32.png
%   % Image license: Public Domain
%   % Edited by: larva
%   \imagecc[0]{harmonic_series_to_32_PD__1641x1299px.png}
%   % Harmonic series (sum):
%   \begin{center}%
%     $$\sum_{n=1}^{\infty} \frac{1}{n}$$
%     \vfill
%   \end{center}
% \end{intersong}

    % This file contains some work-in-progress scribblings from larva;
% perhaps to be removed in the more final editions


\beginsong{Puhdista}[by={larva},tags={suitsutus 1, suojelus 1},ph={I}]
  \beginchorus
    |\[Am\noteULL{A}]Happi yhtyy |\[Dm7\noteULL{C}]hiileen, |\[C]lämpö nousee | |
    |\[Am]Henki koskee |\[Dm7]ainetta, |\[F]tietoisuus \[G]koho|\[Am]aa |
  \endchorus
  \beginchorus
    \chorusindent |\[Am]Puhdis|\[Em]ta, |\[G]puhdis|\[Am]ta |
    \chorusindent |\[Dm]Savuna ilmaan, |\[Am]uhrilahja |
    \chorusindent |\[G]Puhdis|\[Am]ta |
    |\[Em]Kutsun suojelusta, |kutsun suojelusta |
    |\[Am]Paikka on pyhä | |
    |\[Em]Kutsun suojelusta, |kutsun suojelusta |
    |\[Am]Aika on pyhä | |
  \endchorus
  \beginverse
    \chorusindent |\[Am]Puhdis|\[Em]ta, |\[G]puhdis|\[Am]ta |
    \chorusindent |\[Dm]Savuna ilmaan, |\[Am]uhrilahja |
    \chorusindent |\[G]Puhdis|\[Am]ta |
  \endverse
  \textnote{outro, fade out:}
  \beginchorus
    |\[G]Puhdis|\[Am]ta |
  \endchorus
\endsong


\beginsong{Matkustan}[by={larva},tags={sydän 1},ph={II}]
  \beginverse
    |\[Am\noteULL{A}]Mat\[^\noteU{C}]kus|\[Em]tan, |\[Am]matkus|\[Em]tan |
    |\[Am]Mieleni |\[Em]sisään, |\[Dm]syvemmäl|\[Am]le |
    |\[Am]Matkus|\[Em]tan, |\[Am]matkus|\[Em]tan |
    |\[Am]Ajatusten |\[Em]taakse, |\[Dm]yti|\[Am]meen |
  \endverse
  \beginverse
    |\[Am]Kaikenlaista |\[Dm]tulee vastaan; |
    |\[C]mikä siitä \[Em]on tärke|\[Am]ää? |
  \endverse
  \beginchorus
    \lrep |\[Am/E]Sydämeen voi |\[G]luottaa \rrep
    |\[Em]Siellä se \textsuperscript{1}sisällä \textsuperscript{2}(syvällä) |\[Am]on |
  \endchorus
\endsong


\begin{intersong}
  \vfill
  \begin{feeler}
    "An old alchemist gave the following consolation to one of his disciples: 'No matter how
    isolated you are and how lonely you feel, if you do your work truly and conscientiously,
    unknown friends will come and seek you.'"\\
    --- \emph{Carl Jung} (1875--1961)
  \end{feeler}
  %\vfill
\end{intersong}


\beginsong{Avaruus aukeaa sisältä}[by={larva},tags={sydän 1, avaruus 1},ph={II, III}]
  \beginchorus
    \lrep |\[Em\noteULL{E}]Joskus luulen \[ .]olevani |\[D\noteUL{F#}]jotain mitä \textsuperscript{1}\[.]min' en |\[Em]oo | | \rrep \hfill{\footnotesize\emph{\textsuperscript{2}en vaan}}
    |\[Am]Kun sen \[ .]huomaan niin |\[C]suuntaan ta\[Bm]kaisin ko|\[Em]tiin: | |
    |\[D]Sy-\[ .]ydä|\[Em]meen, |\[D]ke-\[ .]eskel|\[Em]le |
    |\[D]Sy-\[ .]y-|\[ .]y-\[ .]ydä|\[Em]meen | |
    |\[D]Ke-\[ .]e-|\[ .]e-\[ .]eskel|\[Em]le | |
    \lrep |\[Am] \[C] |\[Bm] |\[Em] | | \rrep
  \endchorus
  \beginchorus
    \lrep |\[G]A-\[ .]va|\[C]ru-\[ .]u-|\[Em]uus | | \rrep
    |\[G]A-\[ .]va|\[C]ruus \[ .]auke|\[Am]aa \[Am6/E]keskel|\[Em]tä |
    |\[D]Sy-\[ .]yväl|\[Em]tä |\[D]si-\[ .]isäl|\[Em]tä |
    |\[D]Sy-\[ .]y-|\[ .]y-\[ .]yväl|\[Em]tä | |
    |\[D]Si-\[ .]i-|\[ .]i-\[ .]isäl|\[Em]tä | |
    \lrep |\[Am] \[C] |\[Bm] |\[Em] | | \rrep
  \endchorus
\endsong


\beginsong{Ajan takaa}[by={larva},tags={rakkaus 1, lähde 1}]
  \beginchorus
    \lrep |\[Am\noteULL{E}]A-\[ .] \[Am7\noteUDELTAX{D}{-2em}]jan |\[Em]takaa,\[ .] |\[Dm]totu\[ .]uden |\[Am]luota\[ .] | \rrep
    \lrep |\[Em]To-\[ .] |\[ .] \[ .]otuu|\[Am]den \[ .]aijjai|\[ .]jajajaja\[ .]jajajaja | \rrep
  \endchorus
  \beginchorus
    \lrep |\[Am]A-\[ .] \[Am7]jan |\[Em]takaa,\[ .] |\[Dm]totu\[ .]uden |\[Am]luota\[ .] | \rrep
    \lrep |\[Dm]Jospa saisin \[ .]sieltä |\[G]mukaani \[ .]palan |\[Am]rakkaut\[ .]ta | \[ .] \[ .] |
    |\[Em]Sitä kylväi\[ .]sin |\[G] \[ .]maail|\[Am]maan\[ .] | \[ .] \[ .] | \rrep
  \endchorus
  \begin{center}%
    \vfill%
    % 0.618^3 ~ 0.236 (Golden Ratio)
    \includegraphics[width=0.236\textwidth]{bufo_alvarius_bw_transparent_bg_300px.png}
    \vfill%
  \end{center}
\endsong


\beginsong{Hetki}[by={larva},tags={kiitollisuus 1},ph={III, IV}]
  \beginchorus\memorize
    |\[Dm\noteULL{D}]Mie\[^\noteU{A}]li |\[Gm]kuljettaa |\[Am]mennee\[Am7]seen ja |\[Dm]tulevaan |
    |\[Dm]Hetki |\[Gm]katoaa |\[Am]aja\[C]tusten |\[Dm]mukana |
  \endchorus
  \notesoff
  \beginchorus
    |\[Dm]Oi kuinka \[C]kaipaan |\[Dm]niin |
    |\[B&]Sitä mitä \[C]en osaa |\[Dm]saavuttaa
  \endchorus
  \beginchorus
    \textsuperscript{1}(mutta) |\[C#]Kiitos kiitos kiitos kiitos |kiitos tästä hetkes|\[Dm]tä | -
  \endchorus
  \beginchorus
    Se |^opettaa |^elämään |^elämää ^ |^tässä vaan |
    |^Opettaa |^elämään |^e-^elä|^mää!
  \endchorus
\endsong


\beginsong{Minne olenkaan matkalla}[by={larva},ph={III}]
  \beginchorus
    |\[Am\noteULL{A}]Minne mä \[\noteU{E}]olenkaan |\[F]matkalla, |
    |\[C]tiedä sitä |\[Em]en
  \endchorus
  \beginverse
    Mutta |\[C]tahdon |\[G]luottaa,
    sillä |\[Em]kaikkeus ihme |\[Am]on
  \endverse
  \beginchorus
    pada-\[Dm7]diida-diida-|Diida-diida-diida-diida-|Daida-
    \[Em]dam-pada-|Dam padadam-|Padam-\[Am]dam | - \textsuperscript{2}(| | |)
  \endchorus
  \beginverse
    \chorusindent |\[Dm]Sisältäni | löydän |\[Am]surua, | |
    \chorusindent |\[Dm]paljon | on myös |\[Am]iloa | |
    \chorusindent \lrep |\[C]Elämä |\[G]kaunis |\[Am]on | | \rrep
  \endverse
  \beginverse
    \chorusindent |\[Dm]Ympärilläni | kohtaan |\[Am]pelkoa, | |
    \chorusindent |\[Dm]kaikkialla | kuitenkin |\[Am]rakkautta | |
    \chorusindent \lrep |\[C]Elämä |\[G]kaunis |\[Am]on | | \rrep
  \endverse
\endsong


\beginsong{Lämpö, löyly}[by={larva},tags={sauna 1}]
  \beginchorus\memorize
    |\[Am] \[^\noteU{A}]Lämpö, |\[^\noteU{E}]löyly, \[G]iha|\[C]nainen |
    |\[Dm] Poista |\[E]kiire aina|\[Am]hinen |
  \endchorus
  \beginchorus
    |^ Täytä s|ielu, t^yhjää |^mieli |
    |^ Anna |^ajatusten |^sulaa |
  \endchorus
  \beginchorus
    |^ Tuli, |vesi, ^vasta|^jaiset |
    |^ Yhes |^löylyn tänne |^tuovat |
  \endchorus
  \beginchorus
    \chorusindent |\[F] Ilon, |\[E]rauhan, meille |\[Am]suovat |
  \endchorus
\endsong

\nextcol % Jump to the next page; can be removed when there wouldn't be an empty page
\beginsong{Kyynikolle pelastus}[by={larva}]
  \beginchorus
    |\[C\noteUL{C}]Mikä tääll' \[\noteU{B}]on |\[F]aitoo?
    Ku |\[C]rakkauskin vaatii |\[G]taitoo. |
  \endchorus
  \beginchorus
    |\[C]Anna mulle |\[F]jotakin pientä: |
    |\[G]lientä tai vaikkapa |\[C]sientä. |
  \endchorus
\endsong


\begin{intersong} % Fibonacci sequence
  \footnotesize
  \vfill
  \begin{center}
    1, 1, 2, 3, 5, 8, 13, 21, 34, 55, 89, 144, 233, 377, 610, 987, 1597, \ldots
  \end{center}
  \vfill
\end{intersong}


    \chordsoff % songs: do not show (empty line for non-existing) chords
    % songs: Increase line spacing for better readability
    \baselineadj=+1pt plus 0pt minus 0pt%
    \renewcommand{\lyricfont}{\small} % songs: use smaller font
    \songcolumns{2} % songs: two columns per page
    \songpos{1} % songs: avoid ONLY page-turns within songs
    % songs: make penalty for breaking column/page at any line of lyrics to be the same:
    % (The default for \interlinepenalty is 1000, and for all the others 200.)
    \interlinepenalty=200 %
    \songcolumns{2} % two columns!

\beginsong{Aamulla}[tags={Aurinko 1}]
  \beginverse
    Terve kasvos näyttämästä,
    Päivä kulta koittamasta,
    Aurinko ylenemästä!
    Pääsit ylös alltoin alta
    Yli männistön ylenit,
    Nousit kullaisna käkenä,
    Hopeaisna kyyhkyläisnä
    Tasaiselle taivahalle,
    Elollesi entiselle,
    Matkoillesi muinaisille.
  \endverse
  \beginverse
    Nouse aina aikoinasi
    Perästä tämänki päivän,
    Tuo meille tuliaisiksi
    Anna täyttä terveyttä,
    Siirrä saama saatavihin,
    Pyytö päähän peukalomme,
    Onni onkemme nenähän;
    Käy kaaresi kaunihisti,
    Päätä päivän matkuesi,
    Pääse illalla ilohon!
  \endverse
\endsong


\beginsong{Tuulen sanat}[tags={tuuli 1}]
  \beginverse
    Terve kuu, terve päivä,
    Terve ilma, terve tuulet,
    Pohjois- ja etelätuuli,
    Itätuuli, länsituuli
    Lapintuuli, luoetuuli
    Suvituuli, lounaistuuli,
    Päivän nousu- ja laskutuuli
    Ja kaikki väliset tuulet!
    Lepy tuuli leppeäksi
    Lauhu ilma lauhkeaksi
    Kuu kirkas kumottamahan,
    Päivä lämmin paistamahan;
    Sivu tuulet tuulekohot,
    Sivu saakohot satehet,
    Kohti kuut kumottakohot,
    Kohti päivät paistakohot!
  \endverse
\endsong


\beginsong{Löylyn sanat: terve löyly}[tags={sauna 1}]
  \beginverse
    Terve löyly, terve lämmin
    terve henkäys kiukainen,
    kylpy lämpimäin kivisten,
    hiki vanhan Väinämöisen.
    Löylystä vihannan vihdan,
    tervan voimasta terveiden.
  \endverse
  \beginverse
    Löyly kiukahan kivestä,
    löyly saunan sammalista.
    Tervehyttä tekemähän,
    rauhoa rakentamahan,
    kipehille voitehiksi,
    pahoille parantehiksi.  
  \endverse 
\endsong


\beginsong{Löylyn sanat: tule löylyhyn}[tags={sauna 1}]
  \musicnote{Melodia: Kalevala-sävelmä tai esim. Hedingarna: Täss' on nainen}
  \beginverse
    Tule löylyhyn, Jumala, 
    Iso ilman, lämpimähän,
    Terveyttä tekemähän,
    Rauhoa rakentamahan
  \endverse
  \beginverse
    Lyötä maahan liika löyly
    Paha löyly pois lähetä
    Ettei polta tyttöjäsi
    Turmele tekemiäsi
  \endverse
  \beginverse
    Minkä vettä viskaelen
    Noille kuumille kivillen
    Se medeksi muuttukohon
    Simaksi sirahtakohon
  \endverse
  \beginverse
    Juoskohon joki metinen
    Simalampi laikkukohon
    Läpi kiukahan kivisen
    Läpi saunan sammalisen! 
  \endverse 
\endsong


\beginsong{Varjele vakainen luoja}[by={Kalevala: 43. runo}]
  \beginverse
    Anna Luoja, suo Jumala
    anna onni ollaksemme.
    Hyvin ain’ eleäksemme,
    Kunnialla kuollaksemme.
    Suloisessa Suomenmaassa
    Kaunihissa Karjalassa!
  \endverse
  \beginverse
    Varjele, vakainen Luoja
    Kaitse, kaunoinen Jumala,
    Ole puolla poikiesi,
    Aina lastesi apuna,
    Aina yöllisnä tukena,
    Päivällisnä vartiana.
  \endverse  
\endsong


\beginsong{Ihmisen synty}[]
  \beginverse
    Ihminen ihala ilme,
    Sukukunnan suuri luomus,
    Tehty on mullan kakkarasta,
    Mullan kaakusta rakettu,
    (Sille Herra hengen antoi,
    Luoja suustahan sukesi.)
  \endverse
\endsong


\beginsong{Karhun synty}[]
  \beginverse
    Otsoseni, ainoiseni,
    Mesikämmen kaunoiseni,
    Kyllä mä sukusi tieän,
    Miss' oot otso syntynynnä,
    Saatuna sinisaparo,
    Jalka kyntinen kyhätty:
    Tuoll' oot otso syntynynnä
    Ylähällä taivosessa,
    Kuun kukuilla, päällä päivän,
    Seitsentähtien selällä,
    Ilman impien tykönä,
    Luona luonnon tyttärien.
  \endverse
  \beginverse
    Tuli läikkyi taivahasta,
    Ilma kääntyi kehrän päällä,
    Otsoa suettaessa,
    Mesikkiä luotaessa.
    Sieltä maahan laskettihin
    Vierehen metisen viian,
    Hongattaren huolitella,
    Tuomettaren tuu'itella,
    Juurella nyrynärehen,
    Alla haavan haaralatvan,
    Metsän linnan liepehellä,
    Korven kultaisen kotona.
  \endverse
  \beginverse
    Siitä otso ristittihin,
    Karvahalli kastettihin,
    Metisellä mättähällä,
    Sarajoen salmen suulla,
    Pohjan tyttären sylissä.
    Siinä se valansa vannoi
    Pohjan eukon polven päässä,
    Essä julkisen Jumalan,
    Alla parran autuahan,
    Tehä ei syytä syyttömälle,
    Vikoa viattomalle,
    Käyä kesät kaunihisti,
    Soreasti sorkutella,
    Elellä ajat iloiset
    Suon selillä, maan navoilla,
    Kilokangasten perillä;
    Käyä kengättä kesällä,
    Sykysyllä syylingittä,
    Asua ajat pahemmat,
    Talvikylmät kyhmästellä,
    Tammisen tuvan sisässä,
    Havulinna liepehellä,
    Kengällä komean kuusen,
    Katajikon kainalossa.
  \endverse
\endsong


\beginsong{Kiven synty}[]
  \beginverse
    Ken kiven kiveksi tiesi,
    Kun oli otraisna jyvänä,
    Nousi maasta mansikkana,
    Puun juuresta puolukkana,
    Taikka häilyi hattarassa,
    Piili pilvien sisässä,
    Tuli maahan taivahasta,
    Putosi punakeränä,
    Kaaloi kakraisna kapuna,
    Vieri vehnäisnä mykynä,
    Läpi pilvipatsahien,
    Puhki kaarien punaisten,
    Hullu huutavi kiveksi,
    Maan munaksi mainitsevi.
  \endverse
\endsong


\beginsong{Noidan synty}[]
  \beginverse
    Kyllä tieän noian synnyn,
    Sekä alun arpojia:
    Tuoll' on noita syntynynnä,
    Tuolla alku arpojien,
    Pohjan penkeren takana,
    Lapin maassa laakeassa;
    Siell' on noita syntynynnä,
    Siellä arpoja sikesi,
    Hakoisella vuotehella,
    Kivisellä pääalalla.
  \endverse
\endsong


\beginsong{Puiden synty}[]
  \beginverse
    Sampsa poika Pellervoinen
    Kesät kentällä makasi
    Keskellä jyväketoa,
    Jyväparkan parmahalla;
    Otti kuusia jyviä,
    Seitsemiä siemeniä,
    Yhen nää'än nahkasehen,
    Koipehen kesäoravan,
    Läksi maita kylvämähän,
    Toukoja tihittämähän.
  \endverse
  \beginverse
    Kylvi maita kyyhätteli,
    Kylvi maita, kylvi soita,
    Kylvi auhtoja ahoja,
    Panettavi paasikoita.
    Kylvi kummut kuusikoiksi,
    Mäet kylvi männiköiksi,
    Kankahat kanervikoiksi,
    Notkont nuoriksi vesoiksi.
    Noromaille koivut kylvi,
    Lepät maille leyhkeille,
    Kylvi tuomet tuorehille,
    Pihlajat pyhille maille,
    Pajut maille paisuville,
    Raiat nurmien rajoille,
    Katajat karuille maille,
    Tammet virran vierimaille.
  \endverse
  \beginverse
    Läksi puut ylenemähän,
    Vesat nuoret nousemahan,
    Tuuliaisen tuu'ittaissa,
    Ahavaisen liekuttaissa,
    Kasvoi kuuset kukkalatvat,
    Lautui lakkapäät petäjät,
    Nousi koivuset noroilla,
    Lepät mailla leyhkeillä,
    Tuomet mailla tuorehilla,
    Pihlajat pyhillä mailla,
    Pajut mailla paisuvilla,
    Raiat mailla raikkahilla,
    Katajat karuilla mailla,
    Tammet virran vieremillä.
  \endverse
\endsong


\sclearpage
\beginsong{Höyhensaaret}[by={Eino Leino}]
  \beginverse
    Mitä siitä jos nuorna ma murrunkin
    tai taitun ma talvisäihin,
    moni murtunut onpi jo ennemmin
    ja jäätynyt elämän jäihin.
    Kuka vanhana vaappua tahtoiskaan?
    Ikinuori on nuoruus laulujen vaan
    ja kerkät lemmen ja keväimen,
    ilot sammuvi ihmisten.
  \endverse
  \beginverse
    Mitä siitä jos en minä sammukaan
    kuin rauhainen, riutuva liesi,
    jos sammun kuin sammuvat tähdet vaan
    ja vaipuvi merillä miesi.
    Kas, laulaja tähtiä laulelee
    ja hän meriä suuria seilailee
    ja hukkuvi hyrskyhyn, ennen kuin
    käy purjehin reivatuin.
  \endverse
  \beginverse
    Mitä siitä jos en minä saanutkaan,
    mitä toivoin ma elämältä,
    kun sain minä toivehet suuret vaan
    ja kaihojen kantelen hältä.
    Ja vaikka ma laps olen pieni vain,
    niin jumalten riemut ma juoda sain
    ja juoda ne täysin siemauksin--
    niin riemut kuin murheetkin.
  \endverse
  \beginverse
    Ja vaikka ma laps olen syksyn vaan
    ja istuja pitkän illan,
    sain soittaa ma kielillä kukkivan maan
    ja hieprukan hivuksilla.
    Niin mustat, niin mustat ne olivat;
    ja suurina surut ne tulivat,
    mut kaikuos riemu nyt kantelen
    vielä kertasi viimeisen!
  \endverse
  \beginverse
    Oi, kantelo pitkien kaihojen,
    sinä aarteeni omani, ainoo!
    Me kaksi, me kuulumme yhtehen,
    jos kuin mua kohtalot vainoo.
    Me kuljemme kylästä kylähän näin,
    ohi kylien koirien räkyttäväin,
    ja keskellä raition raakuuden
    sävel soipa on keväimen.
  \endverse
  \beginverse
    Me kaksi, me tulemme metsästä
    ja me metsien ilmaa tuomme,
    me laulamme nuoresta lemmestä
    ja lempemme kuvan me luomme,
    me luomme sen maailman tomusta niin
    kuin Luoja loi ihmisen Eedeniin
    ja korvesta kohoitamme me sen
    kun vaskisen käärmehen.
  \endverse
  \beginverse
    Te ystävät, joiden rinnassa kyyt
    yön-pitkät pistää ja kalvaa,
    te, joita jäytävi sydämen syyt
    ja elämä harmaja halvaa,
    oi, helise heille mun kantelein,
    oi, helise onnea haavehein
    ja unta silmihin unettomiin
    mun silmäni suljit sa niin.
    Kas, ylläpä mustien murheiden
    on kaunihit taivaankaaret
    ja kaukana keskellä aaltojen
    on haaveiden höyhensaaret
    ja ken sinne lapsosen kaarnalla käy,
    ei sille ne aavehet yölliset näy,
    vaan rinnoin hän uinuvi rauhaisin
    kuin äitinsä helmoihin.
  \endverse
  \beginverse
    Mitä siitä jos valhetta onkin ne vaan
    ja kestä ei päivän terää!
    Me uinumme siksi kuin valveutaan
    ja vaivat ne jällehen herää.
    Moni nukkui nuorihin toiveisiin
    ja heräsi hapsihin hopeisiin;
    hän katsahti ympäri kummissaan
    ja -- uinahti uudestaan.
    Miks ihmiset tahtoa, taistella
    ja koittaa korkealle?
    Me olemme kaikki vain lapsia
    ja murrumme murheen alle.
    Miks emme me kaikki vois uinahtaa
    ja hyviä olla ja hymytä vaan
    ja katsoa katsehin kirkkahin
    vain sielumme syvyyksiin?
  \endverse
  \beginverse
    Oi, unessa murheet ne unhottuu
    ja rauhaton rauhan saapi,
    oi, unessa vankikin vapautuu,
    sen kahlehet katkeaapi,
    ja köyhä on rikas kuin kuningas maan
    ja kevyt on valtikka kuninkaan
    ja kaikki, kaikki on veljiä vaan--
    oi, onnea unelmain!
  \endverse
  \beginverse
    Oi, onnea uinua uudelleen
    ne lapsuen päivät lauhat
    ja itkeä jällehen yksikseen
    ne riemut ja rinnan rauhat;
    taas uskoa, että on lapsi vaan
    ja että voi alkaa uudestaan
    ja uskoa uusihin toiveisiin
    sekä vanhoihin ystäviin!
  \endverse
  \beginverse
    Taas uskoa riemuhun, keväimeen
    ja lippuhun pilvien linnan
    ja uskoa lempehen puhtaaseen
    taas kahden puhtahan rinnan,
    taas uskoa itsensä rikkahaks
    ja maailman suureks ja avaraks--
    voi, kuinka se sentään on ihanaa,
    kun sen nuorena uskoa saa!
  \endverse
  \beginverse
    Voi, kuinka se sille on ihanaa,
    joka kaiken sen kadotti kerran,
    joka häkistä katseli maailmaa
    ja näki vain vaaksan verran,
    joka etsi kauneutta, elämää,
    ja näki vain markkinavilinää,
    ja näki räyhäävän raakuuden, tyhmyyden--
    niit' aikoja unhota en.
  \endverse
  \beginverse
    Kun muistelen, kuinka ma kerjännyt
    olen koirana lempeä täällä,
    miten rikasten portailla pyydellyt
    olen tuiskulla, tuulissäällä,
    vain lämpöä hiukkasen, hiukkasen vain
    ja kun minä muistelen, mitä mä sain
    ja mitä mä nielin ja vaikenin
    ja mitä mä ajattelin!
  \endverse
  \beginverse
    Miten olen minä kulkenut, uskonut,
    ett'eivät ne unhoitukaan!
    Ja sentään ne olen minä unhoittanut
    kuin unhoittaa voi kukaan.
    Ja sentään se nousi, niin kohtalot kaas,
    ja sentään ma seppona seison taas
    ja taivahan kansia taon ja lyön--
    oi, onnea tähtisen yön!
  \endverse
  \beginverse
    Ne saapuvat, saapuvat uudestaan
    mun onneni orhit valkeet,
    ne painavat vanhalla voimallaan
    mun rintani jättipalkeet.
    Ja kirkas on taivas ja kukkii maa
    ja säkenet suustani suitsuaa
    ja ääneni on kuni ukkosen--
    oi, onnea unelmien!
  \endverse
  \beginverse
    Mitä siitä jos haaveeni verkot vaan
    on verkkoja hämähäkin!
    Mitä siitä jos omieni viittova vaan
    on laulua laineiden näkin!
    Moni nukkui nuorihin toiveisiin
    ja heräsi hapsihin hopeisiin
    tai herännyt täällä ei milloinkaan.
    Missä? Milloin? Helmassa maan.
    Minä tahdon riemuja keväimen
    ja onnesta osani kerta!
    Olen imenyt rintoja totuuden,
    mut niistä vaan tuli verta.
    Siis, tulkaa te utaret unelmien,
    minä vaivun riemunne rinnoillen
    ja uskon päivähän, aurinkohon.
    Unen maito on loppumaton.
  \endverse
  \beginverse
    Oi, kauniisti mulle te kaartukaa,
    mun syömeni sateenkaaret!
    Mua hiljaa, hiljaa tuudittakaa,
    te haaveiden höyhensaaret!
    Mua katsokaa: olen lapsi vaan,
    olen riisunut päältäni riemut maan
    ja pyytehet kullan ja kunnian.
    Uni onni on laulajan.
  \endverse
  \beginverse
    Minä tahdon vain uinua yksikseen.
    En tahtois vielä mä kuolla.
    Mut kuulkaa, jo äitini huhuilee
    Tuonen aaltojen tuolla puolla.
    Oi, odota hetkinen, äityein!
    En viel' olis valmis ma matkallein,
    mun syömeni on niin syyllinen.
    Suo että mä pesen sen.
  \endverse
  \beginverse
    Suo että mä ensin huuhdon vaan
    nämä synkeät, huonot aatteet,
    suo että mä päälleni ensin saan
    ne puhtahat, valkeat vaatteet,
    jotk' ompeli onneni impynen,
    hän, hämärän impeni ihmeellinen,
    min kuvaa kannan ma sydämessäin
    siit' asti kuin hänet mä näin.
  \endverse
  \beginverse
    Me tulemme, äitini armahain!
    Oi katso, meitä on kaksi!
    Oi katso, mik' on mulla rinnassain!
    Niin oisitko rikkahaksi
    sinä uskonut koskaan kuopustas?
    Ja katso, me pyydämme siunaustas,
    sun poikasi synkeä, syyllinen,
    ja mun impeni puhtoinen.
  \endverse
  \beginverse
    Katso, kuin hän on kaunis ja valkoinen
    ja muistuttaa niin sua!
    Hän on niin hellä ja herttainen,
    vaikk'ei hän lemmi mua,
    Elä kysele hältä, miks tänne mun toi,
    mut usko, se niin oli parhain, oi!
    Ja usko, nyt ett' olen onnellinen
    kuin aikoina lapsuuden.
  \endverse
  \beginverse
    Elä kysele multa sa laaksoista maan!
    Ei olleet ne luodut mulle.
    Mut jos sinun silmäsi tutkii vaan,
    voin laulaa ma laulun sulle
    kuin lauloin ma lapsen aikoihin--
    kas, lauluna sujuu se paremmin
    ja kyynelet kuuluvat kantelehen.
    Niitä muuten ma ilmoita en.
  \endverse
\endsong


\beginsong{Hymyilevä Apollo}[by={Eino Leino}]
  \beginverse
    Näin lauloin ma kuolleelle äidillein
    ja äiti mun ymmärsi heti.
    Hän painoi suukkosen otsallein
    ja sylihinsä mun veti:
    »Ken uskovi toteen, ken unelmaan, --
    sama se, kun täysin sa uskot vaan!
    Sun uskos se juuri on totuutes.
    Usko poikani unehes!»
  \endverse
  \beginverse
    Miten mielelläin, niin mielelläin
    hänen luoksensa jäänyt oisin
    luo Tuonen virtojen viileäin,
    mut kohtalot päätti toisin.
    Vielä viimeisen kerran viittasi hän
    kuin hän vain viitata tiesi.
    Taas seisoin ma rannalla elämän,
    mut nyt olin toinen miesi.
  \endverse
  \beginverse
    Nyt tulkaa te murheet ja vastukset,
    niin saatte te vasten suuta!
    Nyt raudasta mulla on jänteret,
    nyt luuni on yhtä luuta.
    Kas, Apolloa, joka hymyilee,
    sitä voita ei Olympo jumalineen,
    ei Tartarus, Pluto, ei Poseidon.
    Hymyn voima on voittamaton.
  \endverse
  \beginverse
    Meri pauhaa, ukkonen jylisee,
    Apollo saapuu ja hymyy.
    Ja katso! Ukkonen vaikenee,
    tuul' laantuu, lainehet lymyy.
    Hän hymyllä maailman hallitsee,
    hän laululla valtansa vallitsee,
    ja laulunsa korkea, lempeä on.
    Lemmen voima on voittamaton.
  \endverse
  \beginverse
    Kun aavehet mieltäsi ahdistaa,
    niin lemmi! -- ja aavehet haihtuu.
    Kun murheet sun sielusi mustaks saa,
    niin lemmi! -- ja iloks ne vaihtuu.
    Ja jos sua häpäisee vihamies,
    niin lemmellä katko sen kaunan ies
    ja katso, hän kasvonsa kääntää pois
    kuin itse hän hävennyt ois.
  \endverse
  \beginverse
    Kuka taitavi lempeä vastustaa?
    Ketä voita ei lemmen kieli?
    Sitä kuulee taivas ja kuulee maa
    ja ilma ja ihmismieli.
    Kas, povet se aukovi paatuneet,
    se rungot nostavi maatuneet
    ja kutovi lehtihin, kukkasiin
    ja uusihin unelmiin.
  \endverse
  \beginverse
    Ei paha ole kenkään ihminen,
    vaan toinen on heikompi toista.
    Paljon hyvää on rinnassa jokaisen,
    vaikk' ei aina esille loista.
    Kas, hymy jo puoli on hyvettä
    ja itkeä ei voi ilkeä;
    miss' ihmiset tuntevat tuntehin,
  \endverse
\endsong


\beginsong{Soutaja}[by={Unto Kupiainen}]
  \beginverse
    Vieras on virta ja vieras on vene, 
    eivät ne unelmies uomia mene. 
    Ilta on ihmisessä ja aamu on outo; 
    illasta aamuun on ihmisen souto. 
    Illasta aamuun on yöllistä matkaa; 
    jos jaksat uskoa, jaksat jatkaa. 
    Taapäin tuijotat, soudat eteen 
    outoa venettä outoon veteen. 
  \endverse
\endsong

\songcolumns{1} % back to one column

  \end{songs}

\end{document}


%% That's it! Nothing more is needed.
