 %% setup_astral-A5-fullmusic_include.tex
 %% =====================================
 %%
 %% Common settings for unilaiva-astral-* songbooks. Also imports the
 %% unilaiva-songbook_common.sty package.
 %%
 %%
 %% Input this file in the header of a main songbook file, right after
 %% optionally giving different versions (with \newcommand{}) for the macros
 %% provided below with \providecommand. The \show??? settings, can be changed
 %% after including this file.
 %%
 %% For astral books, the sub book title must be set before calling this,
 %% as shown below.
 %%
 %% Usage:
 %%
 %%   \newcommand{\subbooktitle}{Book name}
 %%    %% setup_astral-A5-fullmusic_include.tex
 %% =====================================
 %%
 %% Common settings for unilaiva-astral-* songbooks. Also imports the
 %% unilaiva-songbook_common.sty package.
 %%
 %%
 %% Input this file in the header of a main songbook file, right after
 %% optionally giving different versions (with \newcommand{}) for the macros
 %% provided below with \providecommand. The \show??? settings, can be changed
 %% after including this file.
 %%
 %% For astral books, the sub book title must be set before calling this,
 %% as shown below.
 %%
 %% Usage:
 %%
 %%   \newcommand{\subbooktitle}{Book name}
 %%    %% setup_astral-A5-fullmusic_include.tex
 %% =====================================
 %%
 %% Common settings for unilaiva-astral-* songbooks. Also imports the
 %% unilaiva-songbook_common.sty package.
 %%
 %%
 %% Input this file in the header of a main songbook file, right after
 %% optionally giving different versions (with \newcommand{}) for the macros
 %% provided below with \providecommand. The \show??? settings, can be changed
 %% after including this file.
 %%
 %% For astral books, the sub book title must be set before calling this,
 %% as shown below.
 %%
 %% Usage:
 %%
 %%   \newcommand{\subbooktitle}{Book name}
 %%    %% setup_astral-A5-fullmusic_include.tex
 %% =====================================
 %%
 %% Common settings for unilaiva-astral-* songbooks. Also imports the
 %% unilaiva-songbook_common.sty package.
 %%
 %%
 %% Input this file in the header of a main songbook file, right after
 %% optionally giving different versions (with \newcommand{}) for the macros
 %% provided below with \providecommand. The \show??? settings, can be changed
 %% after including this file.
 %%
 %% For astral books, the sub book title must be set before calling this,
 %% as shown below.
 %%
 %% Usage:
 %%
 %%   \newcommand{\subbooktitle}{Book name}
 %%   \input{tex/setup_astral-A5-fullmusic_include.tex}
 %%

% These \providecommands can be set differently before including this file,
% in the main document.

\providecommand{\basicfontsize}{10pt} % 10pt, 11pt, 12pt
\providecommand{\ulbindingoffset}{7mm} % 7 mm for 2:1 wire binding, 25 mm wires

\documentclass[twoside,\basicfontsize]{book}

% Set the title of this document:
\providecommand{\mainbooktitle}{Unilaiva no Astral}

% Set things shown on the second page:
\providecommand{\bookmotto}{\ifchorded{com informação musical}\fi}
\providecommand{\bookbytext}{%
  Recebido por:\par\vspace{0.618034ex}\textbf{\Large humanidade}%
}%
\providecommand{\bookversionpretext}{Versão:}
\providecommand{\bookvariantpretext}{Variante:}
\providecommand{\variantdefaulttext}{padrão}
\providecommand{\variantlyricsonlytext}{somente letras}

% Created with: qrencode -s 32 -l Q -o content/img/QR_https_unilaiva-astral_aavalla_net.png "https://unilaiva-astral.aavalla.net/"
% --> then added the Unilaiva no Astral icon in the middle as 320x320 image with borders
\providecommand{\bookwebsitelinkqrimage}{QR_https_unilaiva-astral_aavalla_net.png}
\providecommand{\bookwebsitelink}{https://unilaiva-astral.aavalla.net/}

\providecommand{\bookcompiledbytext}{%
  Compilado por: larva (\href{mailto:lari.natri@iki.fi}{lari.natri@iki.fi})%
}%
\providecommand{\imprintpagefootnote}{%
  O material deste livro foi coletado e transcrito, com a ajuda de uma multidão
  de pessoas, de várias fontes de áudio e texto --- e por meio de cantar, tocar,
  dançar, compartilhar, estudar e aprender. Este é um trabalho em andamento,
  feito com Gratidão e Amor.
}

\providecommand\chaptersymbol{%
  {\symboliifont 🟌}%
}

% Setup PDF metadata:
\providecommand{\pdfauthor}{larva}
\providecommand{\pdfsubject}{Songbook}
\providecommand{\pdfkeywords}{songs spiritual entheogens ceremony}

\providecommand{\textnextsong}{seguinte hino}
\providecommand{\preludetextkey}{Tom:} % 'tom' instead of 'tonalidade' for brevity
\providecommand{\preludetextgood}{boas:} % ', ' is prepended before this
\providecommand{\preludetextgoodkeysalone}{Boas tons:}

\providecommand\preludekeylinestyle{\bfseries\scriptsize} % make key line in the prelude larger
\providecommand\preludetagstyle{\bfseries\scriptsize} % make tag list in the prelude bold and larger


% The following package contains all the imports, style settings etc:
\usepackage{tex/unilaiva-songbook_common}


% BEGIN helpers for prece chapters

% Use this to specify a subtitle -- used in Pai nosso + Ave Maria
\newcommand{\precesubtitle}[1]{
  \vspace{2.618034ex}\textbf{\textit{#1}}\vspace{1.618034ex}
}

% Use this to specify a translation title within a verse, eg
% \precetranstitle{EN}{Greed}
\newcommand{\precetranstitle}[2]{
  \brkpenalty=-9000\brk%
  \vspace{2.618034ex}%
  \textbf{[#1]} \textit{#2}%
  \vspace{1.618034ex}%
}
% Use this to create a vertical space between "paragraphs"
% within verses:
\newcommand{\preceparspace}{\vspace{1ex}}

% Setup the color for note mark superscripts to be the same as chords'
% color:
\definecolor{precenotemarkcolor}{rgb}{0,0,0}%
\colorlet{precenotemarkcolor}{chordcolor}%

% Superscipt with notemarkcolor
\newcommand{\preceup}[1]{
  \textsuperscript{\color{precenotemarkcolor}#1}%
}

% Environment for a multiline note to be used within a song environment:
\newenvironment{precenote}{%
  \yesendsongvfill%
  \vfill%
  \beginverse*%
  \small%
  \begin{itshape}%
}{%
  \end{itshape}%
  \endverse%
}

% END helpers for prece chapters


% Setup the icons displayed on the front cover
\renewcommand{\nolyricsicon}{unilaiva-mode-icon-somenteletras_512x512px.png}
\renewcommand{\insertcovericons}{\insertcovericonsvtl}%

% unilaiva-songbook_common: setup

% polyglossia: language settings
\setmainlanguage{portuguese}
\setotherlanguages{english, finnish}

\showphfalse
\showphintocfalse
\showkeytrue
\showgoodkeystrue
\showtranslationtrue
\showofftrue
\showextrue
\showtagstrue
\shownotestrue
\showbeatstrue
\showauthtrue
\showlilypondtrue
\showexplanationtrue
\showtranslationtrue
\showfeelertrue
\showaltchordstrue

\renewcommand\translationstyle{\footnotesize} % make translated text smaller
\renewcommand\mncirclestyle{\small} % make melody note's circle larger
\renewcommand\mnsymbolstyle{\scriptsize} % make the melody note name text larger
% larger chord font:
\renewcommand{\printchord}[1]{\sffamily\bfseries\normalsize\color{chordcolor}#1}

% Colors used for different languages (at least) in the prayers section
\definecolor{englishcolor}{rgb}{0,0.26,0.26} %
\definecolor{finnishcolor}{rgb}{0,0.05,0.38} %

% Setup chapter colors
\definecolor{precescolor}{rgb}{.196,.5176,.749} % from the image
\definecolor{oracaocolor}{rgb}{.749,.749,.749}
\definecolor{concentracaocolor}{rgb}{.2078,.4156,.3725}
\definecolor{concentracaodiversoscolor}{rgb}{.3803,.7607,.6784}
\definecolor{daimecolor}{rgb}{0,.5,0}
\definecolor{santamariacolor}{rgb}{0,0,.5}
\definecolor{curaicolor}{rgb}{.1725,.5686,.0039}
\definecolor{curaiicolor}{rgb}{.2627,.8627,.0078}
\definecolor{curadiversoscolor}{rgb}{.6627,.8627,0}
\definecolor{saomiguelcolor}{rgb}{.5490,0,1}
\definecolor{diversoscolor}{rgb}{.7843,.7843,.5490}
\definecolor{cruzeirinhocolor}{rgb}{.9529,.7843,0}
\definecolor{encerramentocolor}{rgb}{0,.7647,1}
\definecolor{defumacaocolor}{rgb}{.8475,.3019,.848}
\definecolor{aniversariocolor}{rgb}{1,.5,0}
\definecolor{mestrediversoescolor}{rgb}{1..9921,.7058}
\definecolor{novadimensaocolor}{rgb}{.9,.5,0}
\definecolor{ochaveiraocolor}{rgb}{.3529,.98,.4078}
\definecolor{revelacaocolor}{rgb}{1,1,1} % = white = invisible
\definecolor{ultioscolor}{rgb}{0,.4039,.933}
\definecolor{uldiversoscolor}{rgb}{.4313,0,.49}

 %%

% These \providecommands can be set differently before including this file,
% in the main document.

\providecommand{\basicfontsize}{10pt} % 10pt, 11pt, 12pt
\providecommand{\ulbindingoffset}{7mm} % 7 mm for 2:1 wire binding, 25 mm wires

\documentclass[twoside,\basicfontsize]{book}

% Set the title of this document:
\providecommand{\mainbooktitle}{Unilaiva no Astral}

% Set things shown on the second page:
\providecommand{\bookmotto}{\ifchorded{com informação musical}\fi}
\providecommand{\bookbytext}{%
  Recebido por:\par\vspace{0.618034ex}\textbf{\Large humanidade}%
}%
\providecommand{\bookversionpretext}{Versão:}
\providecommand{\bookvariantpretext}{Variante:}
\providecommand{\variantdefaulttext}{padrão}
\providecommand{\variantlyricsonlytext}{somente letras}

% Created with: qrencode -s 32 -l Q -o content/img/QR_https_unilaiva-astral_aavalla_net.png "https://unilaiva-astral.aavalla.net/"
% --> then added the Unilaiva no Astral icon in the middle as 320x320 image with borders
\providecommand{\bookwebsitelinkqrimage}{QR_https_unilaiva-astral_aavalla_net.png}
\providecommand{\bookwebsitelink}{https://unilaiva-astral.aavalla.net/}

\providecommand{\bookcompiledbytext}{%
  Compilado por: larva (\href{mailto:lari.natri@iki.fi}{lari.natri@iki.fi})%
}%
\providecommand{\imprintpagefootnote}{%
  O material deste livro foi coletado e transcrito, com a ajuda de uma multidão
  de pessoas, de várias fontes de áudio e texto --- e por meio de cantar, tocar,
  dançar, compartilhar, estudar e aprender. Este é um trabalho em andamento,
  feito com Gratidão e Amor.
}

\providecommand\chaptersymbol{%
  {\symboliifont 🟌}%
}

% Setup PDF metadata:
\providecommand{\pdfauthor}{larva}
\providecommand{\pdfsubject}{Songbook}
\providecommand{\pdfkeywords}{songs spiritual entheogens ceremony}

\providecommand{\textnextsong}{seguinte hino}
\providecommand{\preludetextkey}{Tom:} % 'tom' instead of 'tonalidade' for brevity
\providecommand{\preludetextgood}{boas:} % ', ' is prepended before this
\providecommand{\preludetextgoodkeysalone}{Boas tons:}

\providecommand\preludekeylinestyle{\bfseries\scriptsize} % make key line in the prelude larger
\providecommand\preludetagstyle{\bfseries\scriptsize} % make tag list in the prelude bold and larger


% The following package contains all the imports, style settings etc:
\usepackage{tex/unilaiva-songbook_common}


% BEGIN helpers for prece chapters

% Use this to specify a subtitle -- used in Pai nosso + Ave Maria
\newcommand{\precesubtitle}[1]{
  \vspace{2.618034ex}\textbf{\textit{#1}}\vspace{1.618034ex}
}

% Use this to specify a translation title within a verse, eg
% \precetranstitle{EN}{Greed}
\newcommand{\precetranstitle}[2]{
  \brkpenalty=-9000\brk%
  \vspace{2.618034ex}%
  \textbf{[#1]} \textit{#2}%
  \vspace{1.618034ex}%
}
% Use this to create a vertical space between "paragraphs"
% within verses:
\newcommand{\preceparspace}{\vspace{1ex}}

% Setup the color for note mark superscripts to be the same as chords'
% color:
\definecolor{precenotemarkcolor}{rgb}{0,0,0}%
\colorlet{precenotemarkcolor}{chordcolor}%

% Superscipt with notemarkcolor
\newcommand{\preceup}[1]{
  \textsuperscript{\color{precenotemarkcolor}#1}%
}

% Environment for a multiline note to be used within a song environment:
\newenvironment{precenote}{%
  \yesendsongvfill%
  \vfill%
  \beginverse*%
  \small%
  \begin{itshape}%
}{%
  \end{itshape}%
  \endverse%
}

% END helpers for prece chapters


% Setup the icons displayed on the front cover
\renewcommand{\nolyricsicon}{unilaiva-mode-icon-somenteletras_512x512px.png}
\renewcommand{\insertcovericons}{\insertcovericonsvtl}%

% unilaiva-songbook_common: setup

% polyglossia: language settings
\setmainlanguage{portuguese}
\setotherlanguages{english, finnish}

\showphfalse
\showphintocfalse
\showkeytrue
\showgoodkeystrue
\showtranslationtrue
\showofftrue
\showextrue
\showtagstrue
\shownotestrue
\showbeatstrue
\showauthtrue
\showlilypondtrue
\showexplanationtrue
\showtranslationtrue
\showfeelertrue
\showaltchordstrue

\renewcommand\translationstyle{\footnotesize} % make translated text smaller
\renewcommand\mncirclestyle{\small} % make melody note's circle larger
\renewcommand\mnsymbolstyle{\scriptsize} % make the melody note name text larger
% larger chord font:
\renewcommand{\printchord}[1]{\sffamily\bfseries\normalsize\color{chordcolor}#1}

% Colors used for different languages (at least) in the prayers section
\definecolor{englishcolor}{rgb}{0,0.26,0.26} %
\definecolor{finnishcolor}{rgb}{0,0.05,0.38} %

% Setup chapter colors
\definecolor{precescolor}{rgb}{.196,.5176,.749} % from the image
\definecolor{oracaocolor}{rgb}{.749,.749,.749}
\definecolor{concentracaocolor}{rgb}{.2078,.4156,.3725}
\definecolor{concentracaodiversoscolor}{rgb}{.3803,.7607,.6784}
\definecolor{daimecolor}{rgb}{0,.5,0}
\definecolor{santamariacolor}{rgb}{0,0,.5}
\definecolor{curaicolor}{rgb}{.1725,.5686,.0039}
\definecolor{curaiicolor}{rgb}{.2627,.8627,.0078}
\definecolor{curadiversoscolor}{rgb}{.6627,.8627,0}
\definecolor{saomiguelcolor}{rgb}{.5490,0,1}
\definecolor{diversoscolor}{rgb}{.7843,.7843,.5490}
\definecolor{cruzeirinhocolor}{rgb}{.9529,.7843,0}
\definecolor{encerramentocolor}{rgb}{0,.7647,1}
\definecolor{defumacaocolor}{rgb}{.8475,.3019,.848}
\definecolor{aniversariocolor}{rgb}{1,.5,0}
\definecolor{mestrediversoescolor}{rgb}{1..9921,.7058}
\definecolor{novadimensaocolor}{rgb}{.9,.5,0}
\definecolor{ochaveiraocolor}{rgb}{.3529,.98,.4078}
\definecolor{revelacaocolor}{rgb}{1,1,1} % = white = invisible
\definecolor{ultioscolor}{rgb}{0,.4039,.933}
\definecolor{uldiversoscolor}{rgb}{.4313,0,.49}

 %%

% These \providecommands can be set differently before including this file,
% in the main document.

\providecommand{\basicfontsize}{10pt} % 10pt, 11pt, 12pt
\providecommand{\ulbindingoffset}{7mm} % 7 mm for 2:1 wire binding, 25 mm wires

\documentclass[twoside,\basicfontsize]{book}

% Set the title of this document:
\providecommand{\mainbooktitle}{Unilaiva no Astral}

% Set things shown on the second page:
\providecommand{\bookmotto}{\ifchorded{com informação musical}\fi}
\providecommand{\bookbytext}{%
  Recebido por:\par\vspace{0.618034ex}\textbf{\Large humanidade}%
}%
\providecommand{\bookversionpretext}{Versão:}
\providecommand{\bookvariantpretext}{Variante:}
\providecommand{\variantdefaulttext}{padrão}
\providecommand{\variantlyricsonlytext}{somente letras}

% Created with: qrencode -s 32 -l Q -o content/img/QR_https_unilaiva-astral_aavalla_net.png "https://unilaiva-astral.aavalla.net/"
% --> then added the Unilaiva no Astral icon in the middle as 320x320 image with borders
\providecommand{\bookwebsitelinkqrimage}{QR_https_unilaiva-astral_aavalla_net.png}
\providecommand{\bookwebsitelink}{https://unilaiva-astral.aavalla.net/}

\providecommand{\bookcompiledbytext}{%
  Compilado por: larva (\href{mailto:lari.natri@iki.fi}{lari.natri@iki.fi})%
}%
\providecommand{\imprintpagefootnote}{%
  O material deste livro foi coletado e transcrito, com a ajuda de uma multidão
  de pessoas, de várias fontes de áudio e texto --- e por meio de cantar, tocar,
  dançar, compartilhar, estudar e aprender. Este é um trabalho em andamento,
  feito com Gratidão e Amor.
}

\providecommand\chaptersymbol{%
  {\symboliifont 🟌}%
}

% Setup PDF metadata:
\providecommand{\pdfauthor}{larva}
\providecommand{\pdfsubject}{Songbook}
\providecommand{\pdfkeywords}{songs spiritual entheogens ceremony}

\providecommand{\textnextsong}{seguinte hino}
\providecommand{\preludetextkey}{Tom:} % 'tom' instead of 'tonalidade' for brevity
\providecommand{\preludetextgood}{boas:} % ', ' is prepended before this
\providecommand{\preludetextgoodkeysalone}{Boas tons:}

\providecommand\preludekeylinestyle{\bfseries\scriptsize} % make key line in the prelude larger
\providecommand\preludetagstyle{\bfseries\scriptsize} % make tag list in the prelude bold and larger


% The following package contains all the imports, style settings etc:
\usepackage{tex/unilaiva-songbook_common}


% BEGIN helpers for prece chapters

% Use this to specify a subtitle -- used in Pai nosso + Ave Maria
\newcommand{\precesubtitle}[1]{
  \vspace{2.618034ex}\textbf{\textit{#1}}\vspace{1.618034ex}
}

% Use this to specify a translation title within a verse, eg
% \precetranstitle{EN}{Greed}
\newcommand{\precetranstitle}[2]{
  \brkpenalty=-9000\brk%
  \vspace{2.618034ex}%
  \textbf{[#1]} \textit{#2}%
  \vspace{1.618034ex}%
}
% Use this to create a vertical space between "paragraphs"
% within verses:
\newcommand{\preceparspace}{\vspace{1ex}}

% Setup the color for note mark superscripts to be the same as chords'
% color:
\definecolor{precenotemarkcolor}{rgb}{0,0,0}%
\colorlet{precenotemarkcolor}{chordcolor}%

% Superscipt with notemarkcolor
\newcommand{\preceup}[1]{
  \textsuperscript{\color{precenotemarkcolor}#1}%
}

% Environment for a multiline note to be used within a song environment:
\newenvironment{precenote}{%
  \yesendsongvfill%
  \vfill%
  \beginverse*%
  \small%
  \begin{itshape}%
}{%
  \end{itshape}%
  \endverse%
}

% END helpers for prece chapters


% Setup the icons displayed on the front cover
\renewcommand{\nolyricsicon}{unilaiva-mode-icon-somenteletras_512x512px.png}
\renewcommand{\insertcovericons}{\insertcovericonsvtl}%

% unilaiva-songbook_common: setup

% polyglossia: language settings
\setmainlanguage{portuguese}
\setotherlanguages{english, finnish}

\showphfalse
\showphintocfalse
\showkeytrue
\showgoodkeystrue
\showtranslationtrue
\showofftrue
\showextrue
\showtagstrue
\shownotestrue
\showbeatstrue
\showauthtrue
\showlilypondtrue
\showexplanationtrue
\showtranslationtrue
\showfeelertrue
\showaltchordstrue

\renewcommand\translationstyle{\footnotesize} % make translated text smaller
\renewcommand\mncirclestyle{\small} % make melody note's circle larger
\renewcommand\mnsymbolstyle{\scriptsize} % make the melody note name text larger
% larger chord font:
\renewcommand{\printchord}[1]{\sffamily\bfseries\normalsize\color{chordcolor}#1}

% Colors used for different languages (at least) in the prayers section
\definecolor{englishcolor}{rgb}{0,0.26,0.26} %
\definecolor{finnishcolor}{rgb}{0,0.05,0.38} %

% Setup chapter colors
\definecolor{precescolor}{rgb}{.196,.5176,.749} % from the image
\definecolor{oracaocolor}{rgb}{.749,.749,.749}
\definecolor{concentracaocolor}{rgb}{.2078,.4156,.3725}
\definecolor{concentracaodiversoscolor}{rgb}{.3803,.7607,.6784}
\definecolor{daimecolor}{rgb}{0,.5,0}
\definecolor{santamariacolor}{rgb}{0,0,.5}
\definecolor{curaicolor}{rgb}{.1725,.5686,.0039}
\definecolor{curaiicolor}{rgb}{.2627,.8627,.0078}
\definecolor{curadiversoscolor}{rgb}{.6627,.8627,0}
\definecolor{saomiguelcolor}{rgb}{.5490,0,1}
\definecolor{diversoscolor}{rgb}{.7843,.7843,.5490}
\definecolor{cruzeirinhocolor}{rgb}{.9529,.7843,0}
\definecolor{encerramentocolor}{rgb}{0,.7647,1}
\definecolor{defumacaocolor}{rgb}{.8475,.3019,.848}
\definecolor{aniversariocolor}{rgb}{1,.5,0}
\definecolor{mestrediversoescolor}{rgb}{1..9921,.7058}
\definecolor{novadimensaocolor}{rgb}{.9,.5,0}
\definecolor{ochaveiraocolor}{rgb}{.3529,.98,.4078}
\definecolor{revelacaocolor}{rgb}{1,1,1} % = white = invisible
\definecolor{ultioscolor}{rgb}{0,.4039,.933}
\definecolor{uldiversoscolor}{rgb}{.4313,0,.49}

 %%

% These \providecommands can be set differently before including this file,
% in the main document.

\providecommand{\basicfontsize}{10pt} % 10pt, 11pt, 12pt
\providecommand{\ulbindingoffset}{7mm} % 7 mm for 2:1 wire binding, 25 mm wires

\documentclass[twoside,\basicfontsize]{book}

% Set the title of this document:
\providecommand{\mainbooktitle}{Unilaiva no Astral}

% Set things shown on the second page:
\providecommand{\bookmotto}{\ifchorded{com informação musical}\fi}
\providecommand{\bookbytext}{%
  Recebido por:\par\vspace{0.618034ex}\textbf{\Large humanidade}%
}%
\providecommand{\bookversionpretext}{Versão:}
\providecommand{\bookvariantpretext}{Variante:}
\providecommand{\variantdefaulttext}{padrão}
\providecommand{\variantlyricsonlytext}{somente letras}

% Created with: qrencode -s 32 -l Q -o content/img/QR_https_unilaiva-astral_aavalla_net.png "https://unilaiva-astral.aavalla.net/"
% --> then added the Unilaiva no Astral icon in the middle as 320x320 image with borders
\providecommand{\bookwebsitelinkqrimage}{QR_https_unilaiva-astral_aavalla_net.png}
\providecommand{\bookwebsitelink}{https://unilaiva-astral.aavalla.net/}

\providecommand{\bookcompiledbytext}{%
  Compilado por: larva (\href{mailto:lari.natri@iki.fi}{lari.natri@iki.fi})%
}%
\providecommand{\imprintpagefootnote}{%
  O material deste livro foi coletado e transcrito, com a ajuda de uma multidão
  de pessoas, de várias fontes de áudio e texto --- e por meio de cantar, tocar,
  dançar, compartilhar, estudar e aprender. Este é um trabalho em andamento,
  feito com Gratidão e Amor.
}

\providecommand\chaptersymbol{%
  {\symboliifont 🟌}%
}

% Setup PDF metadata:
\providecommand{\pdfauthor}{larva}
\providecommand{\pdfsubject}{Songbook}
\providecommand{\pdfkeywords}{songs spiritual entheogens ceremony}

\providecommand{\textnextsong}{seguinte hino}
\providecommand{\preludetextkey}{Tom:} % 'tom' instead of 'tonalidade' for brevity
\providecommand{\preludetextgood}{boas:} % ', ' is prepended before this
\providecommand{\preludetextgoodkeysalone}{Boas tons:}

\providecommand\preludekeylinestyle{\bfseries\scriptsize} % make key line in the prelude larger
\providecommand\preludetagstyle{\bfseries\scriptsize} % make tag list in the prelude bold and larger


% The following package contains all the imports, style settings etc:
\usepackage{tex/unilaiva-songbook_common}


% BEGIN helpers for prece chapters

% Use this to specify a subtitle -- used in Pai nosso + Ave Maria
\newcommand{\precesubtitle}[1]{
  \vspace{2.618034ex}\textbf{\textit{#1}}\vspace{1.618034ex}
}

% Use this to specify a translation title within a verse, eg
% \precetranstitle{EN}{Greed}
\newcommand{\precetranstitle}[2]{
  \brkpenalty=-9000\brk%
  \vspace{2.618034ex}%
  \textbf{[#1]} \textit{#2}%
  \vspace{1.618034ex}%
}
% Use this to create a vertical space between "paragraphs"
% within verses:
\newcommand{\preceparspace}{\vspace{1ex}}

% Setup the color for note mark superscripts to be the same as chords'
% color:
\definecolor{precenotemarkcolor}{rgb}{0,0,0}%
\colorlet{precenotemarkcolor}{chordcolor}%

% Superscipt with notemarkcolor
\newcommand{\preceup}[1]{
  \textsuperscript{\color{precenotemarkcolor}#1}%
}

% Environment for a multiline note to be used within a song environment:
\newenvironment{precenote}{%
  \yesendsongvfill%
  \vfill%
  \beginverse*%
  \small%
  \begin{itshape}%
}{%
  \end{itshape}%
  \endverse%
}

% END helpers for prece chapters


% Setup the icons displayed on the front cover
\renewcommand{\nolyricsicon}{unilaiva-tag-icon-lyrics-only_PT_512x512px.png}
\renewcommand{\insertcovericons}{\insertcovericonsvtl}%

% unilaiva-songbook_common: setup

% polyglossia: language settings
\setmainlanguage{portuguese}
\setotherlanguages{english, finnish}

\showphfalse
\showphintocfalse
\showkeytrue
\showgoodkeystrue
\showtranslationtrue
\showofftrue
\showextrue
\showtagstrue
\shownotestrue
\showbeatstrue
\showauthtrue
\showlilypondtrue
\showexplanationtrue
\showtranslationtrue
\showfeelertrue
\showaltchordstrue

\renewcommand\translationstyle{\footnotesize} % make translated text smaller
\renewcommand\mncirclestyle{\small} % make melody note's circle larger
\renewcommand\mnsymbolstyle{\scriptsize} % make the melody note name text larger
% larger chord font:
\renewcommand{\printchord}[1]{\sffamily\bfseries\normalsize\color{chordcolor}#1}

% Colors used for different languages (at least) in the prayers section
\definecolor{englishcolor}{rgb}{0,0.26,0.26} %
\definecolor{finnishcolor}{rgb}{0,0.05,0.38} %

% Setup chapter colors
\definecolor{precescolor}{rgb}{.196,.5176,.749} % from the image
\definecolor{oracaocolor}{rgb}{.749,.749,.749}
\definecolor{concentracaocolor}{rgb}{.2078,.4156,.3725}
\definecolor{concentracaodiversoscolor}{rgb}{.3803,.7607,.6784}
\definecolor{daimecolor}{rgb}{0,.5,0}
\definecolor{santamariacolor}{rgb}{0,0,.5}
\definecolor{curaicolor}{rgb}{.1725,.5686,.0039}
\definecolor{curaiicolor}{rgb}{.2627,.8627,.0078}
\definecolor{curadiversoscolor}{rgb}{.6627,.8627,0}
\definecolor{saomiguelcolor}{rgb}{.5490,0,1}
\definecolor{diversoscolor}{rgb}{.7843,.7843,.5490}
\definecolor{cruzeirinhocolor}{rgb}{.9529,.7843,0}
\definecolor{encerramentocolor}{rgb}{0,.7647,1}
\definecolor{defumacaocolor}{rgb}{.8475,.3019,.848}
\definecolor{aniversariocolor}{rgb}{1,.5,0}
\definecolor{mestrediversoescolor}{rgb}{1..9921,.7058}
\definecolor{novadimensaocolor}{rgb}{.9,.5,0}
\definecolor{ochaveiraocolor}{rgb}{.3529,.98,.4078}
\definecolor{revelacaocolor}{rgb}{1,1,1} % = white = invisible
\definecolor{ultioscolor}{rgb}{0,.4039,.933}
\definecolor{uldiversoscolor}{rgb}{.4313,0,.49}
