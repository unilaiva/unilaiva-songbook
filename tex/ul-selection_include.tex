%% unilaiva-songbook_selections_include.tex
%% ========================================
%%
%% This file is a partial .tex file to be included with \input macro
%% to a Unilaiva songbook selection booklet main .tex file.
%%
%% See ul_selection_example.tex in the project's root directory for
%% an example and documentation on how to create these.
%%


\providecommand{\mainbooktitle}{Selection}

% Do not show chapter nor section in the page headers
\fancyhead[LO]{}
\fancyhead[RE]{}

% This will create the cover page for a selection booklet. If macro \coverpdf
% is defined, it's content will be assumed to be a file name of a PDF file in
% content/img, and will be used as a cover page. Otherwise a cover will be
% generated.
\newcommand{\coverpageforselection}{
  \ifdefined\coverpdf{
    \includepdf[]{content/img/\coverpdf}
  }\else{ % must be contained within a block for the settings to not bleed
    \thispagestyle{empty}
    \topskip0pt
    \vspace*{5.05em} % align with "Contents" heading on the next page; BAD for any changes
    \begin{center}
      \Huge \mainbooktitle\\
      \vspace*{\fill}
      \imagec[2]{Unilaiva-songbook_COVER.pdf}
      %\vspace*{\fill}
    \end{center}
  }
  \fi
}

% This will create the second page for a selection booklet
\newcommand{\imprintpageforselection}[2]{
  { % must be contained within a block for the settings to not bleed
    \thispagestyle{empty}
    \topskip0pt
    \vspace*{5.05em} % align with "Contents" heading on the next page; BAD for any changes
    \begin{center}
      \Huge \mainbooktitle\\
      \normalsize\textit{a selection of songs from Unilaiva songbook}\\
      \vspace*{\fill}
      \imagec[4]{Unilaiva-songbook_COVER.pdf}
      \ifx&#1&% #1 is empty
      \else % #1 not empty
        \vspace{1em}
        \Large \textbf{#1}\\
        \ifx&#2&% #2 is empty
        \else % #2 not empty
          \vspace{0.3em}%
          \normalsize #2\\
        \fi
      \fi
      \vspace*{\fill}
      \large By:\\
      \Large \textbf{humankind}\\
      \vspace{1em}
      \large Version:\\
      % vvvv-mm-dd % Use hardcoded date for tagged printout releases
      \Large \the\year-\ifnum\month<10 0\fi\the\month-\ifnum\day<10 0\fi\the\day% current date, the default
      \ifchorded\else{\\\emph{(without chords)}}\fi% if non-chorded version, mention it here
      \\
      \vspace*{\fill}
      {\normalsize Find the complete songbook at:}
      \\
      % Created with: qrencode -s 32 -l Q -o content/img/QR_https_unilaiva_aavalla_net.png "https://unilaiva.aavalla.net/"
      % --> then added the Unilaiva icon in the middle as transparent 320x320 image
      \imagel[3]{QR_https_unilaiva_aavalla_net.png} % Already centered, so l-version ok
      \\
      {\small\url{https://unilaiva.aavalla.net/}}
      \\
      \vspace*{\fill}
      {\footnotesize Compiled by: larva (\href{mailto:lari.natri@iki.fi}{lari.natri@iki.fi})}
      \\
      \vspace{1.5em}
      {\scriptsize
        The material in this book is collected from various sources over time.\\
        \vspace{-1em} % workaround for too large line spacing
        This is a work in progress, and made for the contributors' private use.
        \vspace{-1em} % workaround for too large line spacing
      }
    \end{center}
  }
} % END \imprintpageforselection


\begin{document}

  % Remove chapter & section name from page headers, as they do not exist
  % in the selection format.
  \fancypagestyle{unilaiva}[fancy]{%
    \setlength{\footskip}{3.7pt} % to work around a warning
    \setlength{\headheight}{12.3pt} % to work around a warning
    \renewcommand{\headrulewidth}{0pt}
    \renewcommand{\footrulewidth}{0pt}
    \fancyhf{}
    \fancyhead[LE,RO]{\rmfamily\small\thepage} % page nr
    \fancyhead[CE]{\rmfamily\small\itshape{\thisbooktitle}} % even center: book name
    \fancyhead[LO]{\rmfamily\small\itshape{}} % odd left: empty
    \fancyhead[CO]{\rmfamily\small\itshape{}} % odd center: empty
  }

  \coverpageforselection % cover page here
  \clearpage
  \imprintpageforselection{}{} % the second (title) page here (verso)

  % TOC:
  \toc

  \clearpage
  \begin{songs}{}
    % Spanish (mostly) language songs
% ===============================
%
% The following sets the song number for the first song in this file.
% The number will automatically be incremented by one for each song.
% Please do not change this! Changing would make different versions of
% the songbook to have different numbers for the same songs, and it
% would totally mess up the selection booklets causing them to have
% wrong songs in them. (For the same reason, add new songs only to the
% end of each songs_ file.)
\setcounter{songnum}{100}


\beginsong{Elevo mi Canto}[by={Mariana Root}, ph={I}, key={Am}, sks={Am, Gm--Em}]
  \audio[key=Em]{https://soundcloud.com/user-624844191/elevo-mi-canto-i-raise-my-voice}
  \beginchorus\memorize
    \[^\mn{A}]E|\[\mnc{E}Am]levo mi can\[^\mn{C}]to al |\[\mnc{D}Em]cielo, yo el\[^ \mn{C}\mn{B}]evo |\[Am] \[^\mn{E}]elevo \[^\mn{C}]mi |\[\mnciii{D}{C}{B}Em]voz
    y |\[Am]busco con fe y con |\[Em]gracias |\[G] \up{*}la transforma|\[Am]ción
    ay yo busco |\[G] \up{*}la transforma|\[Am]ción | \e
  \endchorus
  \notesoff
  \altlyr{ay la unión, ay la sanación, ay la curación, ay la conexión, la iluminación, \ldots}
  \beginchorus
    |^Ay Pachamama, |^ay \up{*}madrecita |^ mil besos te |^doy
    |^porque eres músi|^ca \up{¤}sanadora |^ mil besos te |^doy
    \up{¤}sanadora |^ mil besos te |^doy | \e
  \endchorus
  \altlyr[*]{curandera, \ldots}\altlyrnospace[¤]{amorosa, \ldots}
  \begin{translation}
    I raise my song to the sky, I raise I raise my voice
    And with faith and thanks I seek the transformation
    Oh I seek the transformation (union, healing, connection, enlightenment)
    \nextverse
    Oh Pachamama, oh beloved mother, thousand kisses to you
    For your healing music thousand kisses to you
    Healer, a thousand kisses to you
  \end{translation}
\endsong


\beginsong{Cuatro Vientos}[by={Danit Treubig}, tags={wind}, ph={I, II}, key={Dm}, sks={C\#m, Bm--Em}]
  \audio[key=C\#m]{https://www.youtube.com/watch?v=Pclv31cDTTc}
  \audio[key=C\#m]{https://soundcloud.com/danit-treubig/cuatro-vientos}
  \transpose{5}
  \beginchorus\memorize
    |\[\mnc{E}Am]Vien\[^\mn{A}]to | que |\[\mnc{G}Em]viene de la \[^\mn{A}]mon|\[\mnciii{G}{F}{E}G]taña;
    |\[Am]Viento | tráe|\[Em]nos la clari|\[G]dad.
  \endchorus
  \notesoff
  \beginchorus
    |^Viento | que |^viene del |^mar; \altchords{\id{(Am)}|Am | |Em |G}
    |^Viento, | tráe|^nos la liber|^tad. \altchords{|Am | |Em |G}
  \endchorus
  \beginchorus\noteson
    \ind |\[\mnc{A}Am]Vuela vuela vuela vuela |\[\mn{E}]vuela vuela vue\[\mn{D}]la \altchords{|Am | \e}
    \ind vue|\[Em]la con no|\[G]sotros. \altchords {|Em |G}
  \endchorus
  \beginchorus
    |^Viento | que |^viene del de|^sierto;
    |^Viento, | tráe|^nos el si|^lencio. \goto{Vuela vuela}
  \endchorus
  \beginchorus
    |^Viento | que |^viene de la |^selva;
    |^Viento, tráe|^nos la me|^moria. \goto{Vuela vuela}
  \endchorus
  \begin{translation}
    Wind that comes from the mountain;
    Wind bring us clarity.
    \nextverse
    Wind that comes from the sea;
    Wind, bring us freedom.
    \nextverse
    \ind Fly, fly, fly, fly, fly, fly, fly, fly with us.
    \nextverse
    Wind that comes from the desert;
    Wind, bring us silence.
    \nextverse
    Wind that comes from the forest;
    Wind, bring us the memory.
  \end{translation}
\endsong


\beginsong{Ani Qu Ne'}[ex={tsalagi gawonihisdi, español}, tags={Moon, fire}, ph={I, II}, key={Am}, sks={Am, Am--Dm}]
  \audio[key=Cm]{https://soundcloud.com/minna_finland/a-li-kuni}
  \meter{4}{4}
  \beginchorus
    \ind |\[\mnc{A}Am]Ani \[\mnc{D}Dm]qu ne' |\[\mnc{E}E7]cha\[\mn{C}]wu'\[\mn{B}]na\[\mnc{A}Am]ni \rep{2}
    \ind |\[Dm]{A wa} \[\bm]wa bika |\[Am]na' kaye\[\bm]na \rep{2}
    \ind |\[Am]lyahuh\[G]thi' |\[Em]bisi\[Am]ti \rep{2}
  \endchorus
  \beginchorus
    |\[\mnc{A}Am]En \[\mn{B}]las \[\mnc{C}C]noches |\[\mnc{B}E7]cuan\[\mn{C}]do \[\mn{B}]la \[\mnc{A}Am]luna |co\[\mn{B}]mo \[\mnc{C}C]pla|\[\mnc{D}E7]ta \[\mn{C}]se \[\mn{B}]e\[\mnc{A}Am]leva
    |y la \[C]sel|\[G]va ilu\[Am]mina |y tam\[C]bién |\[G]la pra\[Am]dera
    |{ }{ } Los \[\bm]lobos |\[E7]en la \[Am]noche |llama\[C]rán al |\[E7]gran e\[Am]spíri-
    |tu\ldots \[G] |y al es\[Em]píri|tu del \[Am]fue|go \[\bm]
  \endchorus
  \goto{Ani qu ne'}
  \begin{translation}
    When evening descended upon the village
    The medicine man disappeared into the forest
    Touching the ground with his hands
    \nextverse
    In the nights when the silver moon rises
    And the forest and the meadow are illuminated
    Wolves in the night call the great spirit\ldots
    And the spirit of the fire
  \end{translation}
  \begin{explanation}
    The first part is a prayer of the \emph{Cherokee} people to call in the ancestors,
    honoring them and humbling to their wisdom. It is often sung as a lullaby.
  \end{explanation}
\endsong


\beginsong{Cuando la Luna}[by={Keya Maria}, tags={moon}, ph={II}, key={Em}, sks={Em, Dm--F\#m}]
  \audio[key=Em]{https://soundcloud.com/keyaiyapa/cuando-la-luna}
  \beginchorus
    \[\mn{B}]Cuando la |\[\mnc{E}Em]luna \[\mn{D}]re\[\mn{E}]donda \[\mn{F#}]es|\[\mnc{E}C \mn{D}]ta |\[\mn{E}] | \e
    Y se ilu|mina la oscuri|\[Em]dad | \up{1}(|) \e
  \endchorus\glueverses
  \beginchorus
    Vienen de |\[G]lejos a este lug|\[B]ar
    Magos, du|endes a concor|\[Em]dar \up{2}(| \e)
  \endchorus
  \beginchorus
    \ind |\[G]Que me \[D]crescan |\[Em]alas, |\[G]que me \[D]hablen los |\[Em]magos
    \ind |\[G]Quiro es\[D]tar pres|\[Em]ente |\[B7]en todo mo|\[Em]mento \up{2}(| | \e)
  \endchorus
  \notesoff
  \beginchorus
    Cuando la |\[Em]luna rendonda es|\[C]ta | | \e
    Cuando mil |luces mil colores |\[Em]caen | \up{1}(|) \e
  \endchorus\glueverses
  \beginchorus
    Vienen can|\[G]tando a este lug|\[B]ar
    Magos, du|endes a reali|\[Em]zar \up{2}(| \e)
  \endchorus
  \goto{Que me crescan alas}
  \begin{translation}
    \nextverse
    When the moon is round
    And it illuminates the darkness
    They come from far away to this place
    Wizards, elves to agree
    \nextverse
    \ind I grow wings, I speak magic
    \ind I want to be present at all times
    \nextverse
    When the moon is round
    When thousands of colored lights fall
    They come sing to this place
    Wizards, elves to make
  \end{translation}
\endsong


\beginsong{Lunita}[by={Danit Treubig},tags={moon},ph={II}, key={Am}, sks={Gm, Gm--Bm}]
  \audio[key=Gm]{https://www.youtube.com/watch?v=ZPC3P3ntzKY}
  \beginchorus
    \[\mn{E}]Lu|\[Am]ni\[\mn{C}]ta |\[\mnc{D}Dm]reina \[\mn{C}]de \[\mnc{B}Em]no\[\mn{C}]che
    Lu|\[Am]nita her|\[Dm]mana \[Em] de mi |\[Am]alma
  \endchorus
  \beginverse
    \ind \[\mn{C}]Tu |\[F]luz, \[\mn{E}]tu clari|\[C]dad, \[\mn{C}]tu vibra|\[F]ción, \[\mn{E}]tu armo|\[C]nía
    \ind Tu si|\[F]lencio, tu a|\[C]mor, pode|\[Em]rosa lu|\[Am]nita
  \endverse
  \beginverse
    \ind[3]|\[\mncii{D}{F}Dm]Rei\[\mn{E}]na \[\mn{F}]del |\[\mnc{G}G]cie\[\mn{D}]lo, \[\mn{E}\mn{D}]te |\[\mnc{E}C]doy \[\mn{D}]mi \[\mnc{E}Em]co\[\mn{F}]ra|\[\mnc{E}Am]zón
    \ind[3]|\[Dm]Reina del |\[G]cielo, compa|\[C]ñera de mi \[Em]vida, de mi |\[Am]alma
  \endverse
  \beginverse
    \ind \[\mn{C}]En tu |\[F]luz \[\mn{E}]todo el mundo |\[C]bril\[\mn{C}]la, en tu a|\[F]brazo \[\mn{E}]la tierra |\[C]can\[\mn{C}]ta
    \ind En tu |\[F]luz todo el mundo |\[C]brilla, pode|\[Em]rosa lu|\[Am]nita,
    \ind pode|\[Em]rosa abue|\[Am]lita
  \endverse
  \beginchorus
    \ind[2]\[\mn{E}]Dan|\[\mnc{F}F]zar con el universo, can|\[\mnc{C}C]tar en tu sonrisa
    \ind[2]Dan|\[F]zar con el universo y en tu |\[G]luz siempre fe|\[Am]liz
  \endchorus\glueverses\beginchorus
    \ind[2]En tu |\[G]luz siempre fe|\[Am]liz
  \endchorus
  \imagecc[3]{moon_bw_transparent_bg_939x939px.png}%
  \begin{translation}
    Moon, queen of night
    Moon, sister of my soul
    \nextverse
    Your light, your clarity, your vibration, your harmony
    Your silence, your love, powerful Moon
    \nextverse
    Queen of sky, I give you my heart
    Queen of sky, companion of my life, of my soul
    \nextverse
    In your light the whole world shines, in your embrace the Earth sings
    In your light the whole world shines, powerful Moon
    \nextverse
    To dance with the universe, to sing in your smile
    To dance with the universe and in your always happy light,
    in your always happy light
  \end{translation}
\endsong


\beginsong{En el Cielo y en la Tierra}[by={Isabel Ruiz},tags={Mother Earth, fire},ph={II}, key={Dm}, sks={Dm, Cm--Em}]
  \audio[key={Cm}]{https://soundcloud.com/lucas-rolim-126599764/en-el-cielo-pacha-mama-nessi-gomes-nn-uxhdayry}
  \transpose{5}
  \beginchorus\memorize
    |\[\mnc{E}Am]En \[^\mn{A}]el cielo y |\[\mnc{E}G]en \[^\mn{G}]la tierra |\[\mnc{E}Em]con \[^\mn{G}]el sol y |\[\mnc{A}Am]las \[^\mn{C}]es\[^\mn{A}]trellas
  \endchorus\glueverses
  \notesoff
  \beginchorus
    |^En el cielo y |^en la tierra, |^la lunita y |^las estrellas
  \endchorus
  \beginchorus
    |^Siento el fuego |^dentro dentro, |^siento el fuego a|^quí adentro
  \endchorus\glueverses
  \beginverse
    |^Vuela vuela |^aguilita, |^vuela vuela |^condorcito \replay
    |^Vuelan libres |^por nosotros, |^miren, cuiden |^todo todo
  \endverse
  \beginchorus
    |^Siento el fuego |^dentro dentro, |^siento el fuego a|^quí adentro
  \endchorus\glueverses
  \beginchorus
    |^Pachamama en |^este fuego, |^Pachamama a|^quí te encuentro
  \endchorus
  \begin{translation}
    In the sky and on earth with the sun and the stars
    In the sky and on earth, the moon and the stars
    \nextverse
    I feel the fire inside, I feel the fire in here
    Fly fly eagle, fly fly condor
    They fly free for us, looking, taking care of everything
    \nextverse
    I feel the fire inside, I feel the fire in here
    Pachamama in this fire, Pachamama here I find you
  \end{translation}
\endsong


\beginsong{Estrella Azul}[by={Narayan Dass}, tags={stars}, ph={II}, key={Am}, sks={Cm, Am--Dm}]
  \audio[key={Cm},pitch={432}]{https://maureenji.bandcamp.com/track/estrella-azul}
  \audio[]{https://soundcloud.com/bettinamaureenji/estrella-azul}
  \meter{3}{4}
  \beginchorus\memorize
    \[^\mn{E}]Es|\[\mnc{A}Am]toy a|\[^\mn{C}]qui en |\[^\mn{A}]su \[^\mn{C}]jar|\[\mnc{B}Em]din \altchords{\id{(Dm)}|Dm | | |Am}
    |\[Am]En su ser|\[C]vicio Se|\[Dm]ñor | \e \altchords{|Dm |F |Gm | \e}
    |Pregun|\[G/B]tando |\[C]eschu|\[Am]chando \altchords{| - |C/E |F |Dm}
    |\[C]Cami|\[E]nando a|\[Am]sí | \e \altchords{|F |A |Dm | \e}
  \endchorus
  \notesoff
  \beginchorus
    Yo |^veo su |cara es|trella a|^zul
    |^No mas mi|^edo tu|^ve | \e
    |Ya enten|^di |^como a|^sí
    Tu |^eres un |^padre para |^mi | \e
  \endchorus
  \beginchorus
    Me re|^galas su es|trella bri|llante a|^zul
    |^Yo la |^uso a|^si | \e
    Cal|mando a mis her|^manos con la |^fuerza del a|^mor
    a|^qui en |^este jar|^din | \e
  \endchorus
  \beginchorus
    |^Madre o |madre me co|noces a |^mi
    |^Eres como |^cuando na|^ci | \e
    |Gracias a|^mor por la |^vida a|^qui
    Yo |^can|^to a|^sí | \e
  \endchorus
  \begin{translation}
    I'm here in your garden
    In your service, Lord
    Asking, listening
    Walking like this
    \nextverse
    I see the face of blue star
    No more fear I have
    I already understand how
    You are a father to me
    \nextverse
    I am given your bright blue star
    I use it like that
    Calming my brothers with the force of love
    Here in this garden
    \nextverse
    Mother, oh mother, you know me
    You are like when I was born
    Thank you, love, for life here
    I sing like that
  \end{translation}
\endsong


\beginsong{Tierra, tan Sólo}[by={Marta Gómez, Federico García Lorca}, ph={II}, tags={earth}, key={Em}, sks={F\#m, Em--Gm}]
  \audio[key=Gm]{https://soundcloud.com/sudhir-ibiza/tierra}
  \audio[key=Gm]{https://www.youtube.com/watch?v=wS\_3tsJIKoM}
  \capo{2}
  \beginverse
    \[^\mn{B}]Ti|\[Em]er\[^\mn{E}]ra \[^\mn{G}]tan \[^\mn{F#}]só\[^\mn{E}]lo |\[\mnc{B}G/B]tierr\[^\mn{C}]a
    |\[Am] para las heridas re|\[F#\textdegree]cient\[B7]es
    Ti|\[Em]erra tan sólo |\[G/B]tierra
    |\[Am] para el humilde pensa|\[F#\textdegree]miento\[B7]
    Ti|\[Em]erra, tan sólo |\[D]tierra
    |\[Cmaj7] para el que huye de la |\[F#\textdegree]tie\[B7]{'r}|\[Em]ra | \e
  \endverse
  \notesoff
  \textnote{\musSegno}
  \beginverse
    Ti|^erra tan sólo |^tierra
    |^ tierra desnuda y ale|^{'gre} ^
    Ti|^erra, tan sólo |^tierra
    |^ tierra que ya no se mue|^{'ve} ^
    Ti|\[C]erra, tan sólo |\[D]tier\[C]ra
    de |\[B7]noches i|\[D]nme\[B7]{'n}|\[Em]sas
  \endverse
  \beginverse
    |\[D] No es la ce\[Cmaj7]{niza en} v|\[B7]ilo
    de las |\[C]cosas quemad|\[B7]as
    |\[Am] Lo que yo vengo busca|\[Em]ndo
    es |\[C]tie\[B7]{'r}|\[Em]ra
  \endverse
  \textnote{\up{1}Instrumental \emph{(1st time only)}}
  \textnote{\emph{D.S. al fine}}
  \beginverse
    Vi|\[Cmaj7]ento en el o|\[B7]{'livar}
    |\[Cmaj7]viento en la |\[B7]sier|\[Em]ra
  \endverse
  \begin{translation}
    Earth only earth
    for recent wounds
    Earth only earth
    for the humble thought
    Earth only earth
    for the one who runs away from the earth
    \nextverse
    Earth only earth
    bare and happy earth
    Earth only earth
    earth it doesn't move anymore
    Earth only earth
    of immense nights
    \nextverse
    It's not ash on the edge
    of the burned things
    What i come looking for
    is earth
    \nextverse
    Wind in the olive grove
    wind in the mountains
  \end{translation}
\endsong


\beginsong{Taita Inti Padre Sol}[by={Ramón Peregrino}, tags={Sun}, ph={II}, key={Dm}, sks={Dm, Bm--Em}]
  \audio[key=Am]{https://soundcloud.com/caminorojo/taita-inti-padre-sol-1}
  \transpose{5}
  \beginchorus\memorize
    \[^\mn{A}]Taita |\[\mnc{C}Am]Inti \[^\mn{A}]Padre |\[\mnc{E}Em]Sol,
    Ven ven |ven trae tu \up{1}ca|\[Am]lor \altlyr[2]{sa|ber}
  \endchorus
  \notesoff
  \beginchorus
    Está por el |^rio, por la tierra y por el |^mar,
    volando en el |viento el poder de Dios es|^tá
  \endchorus
  \beginchorus
    Eh Yage ya|^ge yage yage yage ya|^ho, \altchords{|Cm |Gm}
    Sol, Luna y Es|trellas, yo les canto otra |^vez \altchords{| |Cm}
  \endchorus
  \beginchorus
    Aya|^huasca, Caa|^pi,
    Inti inti|gua currupi curru|^pi
  \endchorus
  \beginchorus
    Como me en|^seño yo fui aquí y lo lla|^me,
    agradezco |siempre que nos abra su po|^der
  \endchorus
  \beginchorus
    Pájaro can|^tó, pájaro vo|^ló,
    lleva su pre|sencia quien aquí lo mere|^ció
  \endchorus
  \begin{translation}
    Taita Inti Father Sun,
    Come come come, bring your \up{1}warmth \up{2}{(knowledge)}.
    \nextverse
    The power of God is by the river, by the land
    and by the sea, flying in the wind.
    \nextverse
    Eh Yage yage yage yage yage yaho,
    Sun, Moon and Stars, I sing to you again.
    \nextverse
    Ayahuasca, Caapi,
    Inti intigua currupi currupi.
    \nextverse
    As he taught me I came here and called him,
    thank you for always opening your power to us.
    \nextverse
    Bird sang, bird flew,
    carried his presence to those who here deserved it.
  \end{translation}
\endsong


\beginsong{Aguilita}[by={Lua Maria}, tags={path, flying, wisdom}, ph={II}, key={Am}, sks={Am, Gm--Bm}]
  \audio[key=Gm]{https://soundcloud.com/lua-maria/abuelita-aguilita}
  \audio[]{https://adrianfreedman.com/product/the-phoenix-tree/}
  \transpose{5}
  \beginchorus
    |\[\mnc{E}Em]A\[\mn{F#}]bue\[\mn{G}]li\[\mn{A}]ta |\[\mn{B}]a\[\mn{E}]gui\[\mn{D}]li\[\mn{B}]ta, |\[\mnc{A}D]ven \[\mnc{G}C]ven |\[\mnc{E}Em]ven \rep{3} \altchords{\id{(Em)}|Em | |D C |Em}
  \endchorus\glueverses\beginchorus
    |\[D]Lleva nos por |\[Am]tu camin|\[C]o, sabi|\[Em]o \altchords{|D |Am |C |Em}
  \endchorus
  \beginchorus
    |\[Em]Vuela vuela |vuela vuela, |\[D]alas de \[C]uni|\[Em]dad \rep{3}
  \endchorus\glueverses\beginchorus
    |\[D]Lleva nos por |\[Am]tu camin|\[C]o, sabi|\[Em]o
  \endchorus
   \begin{translation}
     Grandmother eagle, come come come
     Take us on your path, wise
     \nextverse
     Fly fly fly fly, wings of unity
     Take us on your path, wise
   \end{translation}
\endsong


\beginsong{Aguila Aguilé}[ph={II}, key={Am}, sks={Am, Am--Dm}]
  \audio[key=Bm]{https://soundcloud.com/ta-li/aguila-aguile}
  \beginchorus\memorize
    \[^\mn{A}]Des\[^\mn{B}]de |\[\mnciii{C}{B}{A}Am]lejos, | \[^\mn{A}]des\[^\mn{B}]de \[^\mn{C}\mn{D}]lejos |\[\mnc{E}C]oigo | \e
    El |\[G]canto enamorado | de un |\[Am]pájaro | \e
  \endchorus
  \notesoff
  \beginverse
    Ese |^pájaro | es mi a|^buela | \e
    Es mi a|^buela \up{1}que canta, | |^canta enamorada | \e \altlyr[2]{Ayahuasca}
    \replay Ese |^pájaro | es mi a|^buelo | \e
    Es mi a|^buelo \up{1}que canta, | |^canta enamorado | \e \altlyr[2]{Peyote}
  \endverse
  \beginverse\noteson
    |\[\mnc{C}Am]Canta canta c\[\mn{B}]an\[\mn{A}]ta |\sublyrpush{canta canta canta} |\[\mnc{E}C]canta canta c\[\mn{D}]an\[\mn{C}]ta |\sublyr{canta canta canta} \e
    |\[G]Canta canta canta |\sublyrpush{canta canta canta} |\[Am]canta canta canta |\sublyr{canta canta canta} \e
  \endverse
  \beginchorus\noteson
    \ind |\[\mncii{E}{D}C]Águi\[\mn{E}]la | |\[\mnc{D}G]águila a\[\mn{E}]gui\[\mn{D}]lé | \e
    \ind |Águila agui|lé águila agui|\[Am]lé | \e
  \endchorus
  \begin{translation}
    From afar, from far away I hear
    The song in love with a bird
    \nextverse
    That bird is my grandmother
    It's my grandmother who sings, she sings in love
    \nextverse
    That bird is my grandfather
    It's my grandfather who sings, he sings in love
    \nextverse
    Sing sing sing sing sing
    Sing sing sing sing sing
    \nextverse
    Eagle eagle\ldots
  \end{translation}
\endsong


\beginsong{Abre tus Alas}[by={Nina Almar}, tags={path, flying}, ph={II, III}, key={Em}, sks={Em, Dm--Gm}]
  \audio[key=Em]{https://soundcloud.com/taotxana/abre-tus-alas-niniana-jaylan}
  \audio[key=Em]{https://www.youtube.com/watch?v=oIsmh6xAhxE}
  \beginchorus
    \[\mn{E}]Abre tus |\[\mncii{G}{E}Em]alas \[\mn{G}]pájaro \[\mn{B}]vo|\[B7]lar
    Con plumas de co|\[Em]lores danzándole a la |\[B7]mar
  \endchorus
  \beginchorus
    \ind \[\mn{B}]Abre tus |\[\mnc{A}Am]a\[\mn{E}]las libres y ha \[\mn{F#}]vo|\[\mncii{G}{E}Em]lar
    \ind El sol y las es|\[B7]trellas son mi guía cami|\[Em]nar \[\up{1}(E7)]
  \endchorus
  \beginchorus
    Abre tus |\[Em]alas sin miedo a sal|\[B7]tar
    El cielo nos en|\[Em]seña como aprender a|\[B7]mar
  \endchorus
  \beginchorus
    \ind Abre tus |\[Am]alas libres y ha vo|\[Em]lar
    \ind El sol y las es|\[B7]trellas son mi guía cami|\[Em]nar \[\up{1}(E7)]
  \endchorus
  \beginchorus
    \ind La luz del |\[Am]sol ilumina nuestro |\[Em]día
    \ind Emprendiendo el ca|\[B7]mino de amor y la ale|\[Em]gría \[\up{1}(E7)]
  \endchorus
\begin{translation}
  Open your wings, flying bird
  With colored feathers dancing to the sea
  \nextverse
  Open your wings free and have a flight
  The sun and the stars, they are the guide for my journey
  \nextverse
  Open your wings without fear of jumping to the unknown
  The sky teaches us how to learn to love
  \nextverse
  Open your wings free and have a flight
  The sun and the stars, they are the guide for my journey
  \nextverse
  The sunlight light up our day
  Starting the path of love and joy
\end{translation}
\endsong


\beginsong{Señora mi Señora}[by={Santiago Andrade, Irina Flores}, ex={español, nahuatl}, tags={Divine Mother}, ph={II}, key={Dm}, sks={Dm, Cm--Gm}]
  \audio[key=Dm]{https://soundcloud.com/cuatro-vientos-pur-pecha/senor-mi-senora}
  \transpose{-2}
  \beginchorus
    \[\mn{B}]Seño|\[\mnc{E}Em]ra, mi \[\mn{G}]Se\[\mn{F#}]ño|\[\mn{E}]ra yarí
  \endchorus\glueverses
  \beginverse
    Pacha|\[Em]mama Nunkuy |\[G]Tierra sinchi |\[D]Tonāntzin coa|\[Em]{tli yarí}
    Pacha|\[Em]mama Nunkuy |\[G]Tierra sinchi |\[D]Tonāntzin coa|\[Em]{tli yarí} \brk| | \e
  \endverse
  \beginchorus
    El a|\[Em]güita de \up{1}la |vida hay si  \altlyr[2]{mi}
  \endchorus\glueverses
  \beginverse
    va cu|\[Em]rando, va sa|\[G]nando sinchi |\[D]Tonāntzin coa|\[Em]{tli yarí}
    va cu|\[Em]rando, va sa|\[G]nando sinchi |\[D]Tonāntzin coa|\[Em]{tli yarí} \brk| | \e
  \endverse
  \beginchorus
    Mai|\[Em]zito de \up{1}la |vida hay si  \altlyr[2]{mi}
  \endchorus\glueverses
  \beginverse
    va cu|\[Em]rando, va sa|\[G]nando sinchi |\[D]Tonāntzin coa|\[Em]{tli yarí}
    va cu|\[Em]rando va sem|\[G]brando sinchi |\[D]Tonāntzin coa|\[Em]{tli yarí} \brk| | \e
  \endverse
  \beginchorus
    Comi|\[Em]dita de \up{1}la |vida hay si  \altlyr[2]{mi}
  \endchorus\glueverses
  \beginverse
    va cu|\[Em]rando va sa|\[G]nando sinchi |\[D]Tonāntzin coa|\[Em]{tli yarí}
    va cu|\[Em]rando alimen|\[G]tando sinchi |\[D]Tonāntzin coa|\[Em]{tli yarí} \brk| | \e
  \endverse
  \beginchorus
    Ay los |\[Em]frutos de \up{1}la |vida hay si  \altlyr[2]{mi}
  \endchorus\glueverses
  \beginverse
    van cu|\[Em]rando van sa|\[G]nando sinchi |\[D]Tonāntzin coa|\[Em]{tli yarí}
    van cu|\[Em]rando hay endul|\[G]zando sinchi |\[D]Tonāntzin coa|\[Em]{tli yarí} \brk| | \e
  \endverse
  \begin{explanation}
    \begin{description}
      \item[Tonāntzin] is the title for the Aztec Mother Goddess, by which Goddesses such as
        \emph{Mother Earth}, the \emph{Goddess of Sustenance}, \emph{Honored Grandmother},
        \emph{Snake}, \emph{Brin\-ger of Maize} and \emph{Mother of Corn} can be called,
        as it is an honorific title comparable to ``Our Lady'' or ``Our Great Mother''.
      \item[Nunkuy:] Mother Earth
      \item[coatli:] ``water serpent'' or ``serpent water'', name for several medicinal plants.
        \emph{Coatl} means ``serpent'' or ``twin''. \emph{Cōātlīcue}, ``skirt of snakes'', is the
        primordial earth goddess, mother of the gods, the sun, the moon and the stars.
      \item[yarí:] ``heart''
    \end{description}
  \end{explanation}
  \imagecc[1]{snake_sa_bw_transparent_bg_960x262px.png}%
\endsong


\beginsong{Ayahuasca Ayní}[by={Diego Palma}, tags={Aya}, ph={II}, key={Bm}, sks={C\#m, Bm--Dm}]
  \audio[key=C\#m]{https://soundcloud.com/sacredvalleytribe/ayahuasca-ayni}
  \transpose{2}
  \meter{6}{8}
  \beginchorus
    \ind[2]|\[Am] \[\mn{E}]Nara\[\bm]naina |\[\mn{D}]Nai\[\mn{C}]ra \[\bmc\mn{B}]Nai\[\mn{A}]ra|\[\mnc{E}Em]nai Nara\[\bm]nai \[\mn{B}]Na|\[\mnc{A}Am]nai | \e
  \endchorus
  \textnote{\musSegno}
  \beginchorus
    |\[Am] \[\mn{C}]Ayahuasca ay|\[\mnc{B}G]ní, | Ayahuasca |\[\mnc{C}Am]cú\[\mn{B}]ra\[\mn{A}]nos
  \endchorus
  \beginchorus
    \ind |\[Am] \[\mn{E}]Madrecita ay|\[\mnc{D}G]ní, | madrecita |\[\mnc{E}Am]cú\[\mn{D}]ra\[\mn{C}]nos
  \endchorus
  \beginchorus
    |\[Am] Medicina ay|\[G]ní, | medicina |\[Am]cúranos
  \endchorus
  \beginchorus
    \ind |\[Am] Abuelita ay|\[G]ní, | abuelita |\[Am]cúranos
  \endchorus
  \textnote{\emph{D.S. al fine}}
  \goto{Naranaina}
\endsong


\beginsong{Medicina que Trae los Cielos}[by={Hamilton Dielu},ph={II, III}, key={Am}, sks={Cm, Bm--Dm}]
  \capo{3}
  \audio[key=Cm]{https://soundcloud.com/sacredvalleytribe/medicina-que-trae-los-cielos}
  \meter{3}{4}
  \newchords{chords_medicinaque_a}\newchords{chords_medicinaque_b}
  \beginchorus\memorize[chords_medicinaque_a]
    \[^\mn{E}]Medi|\[\mnc{A}Am]cina que |\[\mnc{B}E]tra\[^\mn{C}]e \[^\mn{B}]los |\[\mnc{A}Am]cielos des|\[^\mn{E}]de la flo|\[\mnc{D}Dm]resta a
    |\[E]mi cora|\[Am]zón |\[\up{2}(A7)] \e
  \endchorus\glueverses
  \notesoff
  \beginchorus\memorize[chords_medicinaque_b]
    Alum|\[Dm]bran|do ca|\[C]mi|no tray|\[E7]endo |libera|\[Am]ción |\[\up{1}A7] \e
  \endchorus
  \beginchorus\replay[chords_medicinaque_a]
    Doy |^gracias a |^todos los |^seres ma|estros di|^vinos
    |^de reden|^ción |^ \e
  \endchorus\glueverses
  \beginchorus\replay[chords_medicinaque_b]
    Humil|^dad |y pa|^cien|cia en|^señan |la compa|^sión |^ \e
  \endchorus
  \beginchorus\replay[chords_medicinaque_a]
    Aquí |^pido por |^todos los |^seres |desampa|^rados y
    |^sin protec|^ción |^ \e
  \endchorus\glueverses
  \beginchorus\replay[chords_medicinaque_b]
    Ofre|^cien|do este |^can|to desde el |^fondo de |mi cora|^zón |^ \e
  \endchorus
  \begin{translation}
    Medicine that brings the skies from the forest to my heart
    Lighting the path by bringing liberation
    \nextverse
    I give thanks to all divine beings of redemption
    Humility and patience teach compassion
    \nextverse
    Here I ask for all the helpless beings without protection
    Offering this song from the bottom of my heart
  \end{translation}
\endsong


\beginsong{Abrete Corazón}[by={Rosa Giove}, tags={opening}, ph={II, III}, key={F}, sks={F, D\#--G}]
  \audio[key=F]{https://soundcloud.com/sacredvalleytribe/abrete-corazon}
  \audio{https://soundcloud.com/madre-sigal-lia/abrete-coraz-n}
  \transpose{-2}
  \beginchorus
    |\[G] \[\mn{B}]Ábrete \[\mn{A}]co\[\mn{G}]ra|zón, |\[Em] \[\mn{B}]ábrete \[\mn{A}]sen\[\mn{G}]ti|\[\mn{B}]mien\[\mn{G}]to
    |\[C] ábrete entendi|miento, deja a un |\[G]lado la ra|zón
    y |\[D]deja brillar el |sol escon|\[C]dido | en tu inte|\[G]rior | \e
  \endchorus
  \beginverse
    |\[\mnc{E}C]Ábrete memoria an|tigua escon|\[\mnc{F#}Bm]dida en \[\mn{G}]la |ti\[\mn{F#}]erra
    en las |\[C]plantas, | en el |\[D]aire | \e
    Re|\[C]cuerda lo que apren|diste, bajo |\[Bm]agua, bajo |fuego
    hace |\[C]ya, | ya mucho |\[D]tiempo | \e
  \endverse
  \beginverse
    \[\mn{B}]Ya es |\[G]h\[\mn{G}]ora ya, ya es |\[\mncii{F#}{D}Bm]hora |\[C] \[\mn{E}]abre la mente y re|\[\mnc{F#}D]cuerda
    como el e|\[Em]spíritu cura, como el |\[Bm]amor sana
    |\[C] como el árbol flo|\[D]rece y la \lrep|\[C]vida | per|\[D]dura | \e \rrep
  \endverse
  % % extra verse:
  % \beginverse
  %   Respira profundo; y eleve-te hasta el Cielo
  %   Mira el Sol de frente; como las águilas
  %   No tengas temor; Confia en ti
  %   Confia en Dios; Confia en la Vida
  % \endverse
  \begin{translation}
    Open your heart, open to your feelings
    Open to your understanding, leave reason aside
    And let shine the Sun that is hidden within you
    \nextverse
    Open up ancient memory hidden in the Earth
    In the plants, in the air
    Remember what you learned, under water, under fire
    long, long time ago
    \nextverse
    It is time, now is the hour to open your mind and remember
    how Spirit cures, how love heals,
    how trees flourish and life endures
    % % extra verse:
    % \nextverse
    % Open your wings, Breathe deeply
    % And lift yourself up to the sky
    % Look at the sun in front of you like eagles do
    % Have no fear, Trust in yourself
    % Trust in God; Trust in Life
  \end{translation}
\endsong


\beginsong{El Viejo Tambor}[by={Alonso del Río}, tags={path, heart}, ph={II, III}, key={Am}, sks={Bm, Am--Dm}]
  \audio[key=B&m]{https://soundcloud.com/makaruja/el-viejo-tambor}
  \audio[]{https://www.youtube.com/watch?v=PTU8DtcFKO4}
  \beginverse
    |\[C] \[^\mn{C}]Desde los |tiempos en que mi a|\[\mnc{D}G]buelo so\[^\mn{G}]ñó
    Que un |día yo vivi|\[F]ría para can|tarles esta \[G]can|\[Am]ción | \e
    |\[C] Desde los |tiempos en que mi a|\[G]buelo soñó
    Que un |día yo mori|\[F]ría para entre|garles mi co\[G]ra|\[Am]zón | \e
  \endverse
  \notesoff
  \beginverse\noteson
    \ind |\[Dm] \[\mn{D}]Y voy can|tando por mi \[\mn{E}]ca|\[\mnc{C}F]mi\[\mn{A}]no
    \ind Me va ale|grando un viejo \[G]tam|\[Am]bor | | | \e
    \ind |\[Dm] Vivo dan|zando por mi ca|\[F]mino
    \ind Me va gui|ando un viejo \[G]tam|\[Am]bor | | | \e
  \endverse
  \beginchorus
    |^ Por las que|bradas veo mi |^Cóndor volar
    Y |siento que voy a|^briendo
    Como el sus |alas mi co^ra|^zón | \e
  \endchorus
  \goto{Y voy cantando}
  \beginchorus
    |^ Por las mon|tañas veo que mi |^Aguila va
    Me |va mostrando el ca|^mino
    Camino |Rojo del co^ra|^zón | \e
  \endchorus
  \goto{Y voy cantando}
  \begin{translation}
    Since the time when my grandfather dreamed
    That one day I would live to sing this song
    Since the times when my grandfather dreamed
    That one day I would die to receive this in my heart
    \nextverse
    And I'm singing on my way
    An old drum brings me joy
    I live dancing on my way
    I am guided by this old drum
    \nextverse
    Through the ravines I see my Condor fly
    And I feel that I'm opening my heart
    Like his wings
    \nextverse
    Through the mountains I see that my Eagle is soaring
    Showing me the way
    The Red Path of the Heart
  \end{translation}
\endsong


\beginsong{Mira como Cura el Agua}[by={Alonso del Río},tags={water, air, fire, earth},ph={III}]
  \beginverse
    |\[\mnc{A}Am]Mira |como |cura el |\[^\mn{E}]agua
    |\[Dm] va la|vando como un |\[Am]rí|o | | \e
    |\[Am]Mira |como |cura el |aire
    |\[Dm] va can|tándote al o|\[Am]í|do | | \e
    |\[Am]Mira |como |cura el |fuego
    |\[Dm] va que|mando hasta el ol|\[Am]vi|do | | \e
    |\[Am]Como |la tier|ra le|vanta
    |\[Dm] tu cora|zón tan do|\[Am]li|do | | \e
  \endverse
  \beginverse
    \ind |\[F] Que Willka|mayu traiga las |\[C]notas
    \ind \[F]para que a|\[C]legres tu cora|zón
    \ind |\[F] Que el Apu |Linli todas las |\[C]tardes
    \ind \[F]venga so|\[C]plando su bendi|ción
    \ind |\[F] Que el Ch’eqta |Qaqa venga tra|\[C]yendo
    \ind \[F]el padre |\[C]rayo que enseña|rá
    \ind |\[F] Que Mama |Ñusta viene cui|\[C]dando
    \ind \[F]a sus hi|\[C]jitos \[E7]en la os|curi|\[Am]dad | | | \e
  \endverse
  \begin{translation}
    See how the water cures by cleansing like a river.
    Watch how the air cures by singing into your ear.
    See how the fire cures by burning to oblivion,
    How the earth raises your hurt heart.
    \nextverse
    May Willkamayu bring the notes so that you may rejoice in your heart.
    May Apu Linli come every evening to blow his blessing.
    May the Ch’eqta Qaqa come bringing the teaching father ray.
    May Mama Ñusta come taking care of her little children in the darkness.
  \end{translation}
  \begin{explanation}
    \begin{description}
      \item[Willkamayu] is the central river in \emph{El Valle Sacrado} close to
        \emph{Cusco, Peru}.
      \item[Apu Linli, Ch’eqta Qaqa and Mama Ñusta] are \emph{apu}s, sacred mountain spirits,
        around that valley.
    \end{description}
  \end{explanation}
\endsong


\beginsong{Amor y Unidad}[by={Bóveda Celeste}, ph={III}, key={Am}, sks={Am, Em--Am}]
  \audio[key=Am]{https://bovedaceleste.bandcamp.com/track/amor-y-unidad}
  \meter{6}{8}
  \beginchorus
    |\[\mnc{E}Em]Corren dos liebres a|\[\mn{G}]tra\[\mn{F#}]ve\[\mn{E}]sando \[\mn{D}]el
    |\[Am]bosque donde na|cieron
    de|\[Em]scansan a un lado de un |lago donde
    pla|\[Am]tican con un jil|guero
  \endchorus
  \beginchorus
    \ind \[\mn{C}]Que \[\mn{B}]co|\[\mnc{A}F]men\[\mn{F}]ta que el a|mor es
    \ind compar|\[C]tir lo que hay en |ti
    \ind es |\[Em]nuestra natura|leza, nuestra
    \ind e|\[Am]sencia más su|til
  \endchorus
  \beginchorus
    |\[Em]Águila y condor se en|cuentran en la
    |\[Am]cima de una cordi|llera
    sem|\[Em]brando un mensaje que |llega al
    inte|\[Am]rior de la madre |tierra
  \endchorus
  \beginchorus
    \ind Que nos |\[F]dice que ve|nimos de la misma
    \ind |\[C]fuente y del mismo lu|gar
    \ind to|\[Em]dos somos de la |tierra,
    \ind del |\[Am]agua, del fuego y del |aire
  \endchorus
  \begin{translation}
    Two hares run through the forest where they were born;
    they rest at the side of a lake where they talk with a goldfinch
    \nextverse
    Which says that love is to share what is in you
    It is our nature, our most subtle essence
    \nextverse
    Eagle and condor are at the top of a mountain range
    Sowing a message that reaches the interior of Mother Earth
    \nextverse
    According to it we come from the same source and from the same place;
    we are all from the earth, the water, the fire and the air
  \end{translation}
  \imagecc[3]{eagle_feather_bw_transparent_bg_1280x442px.png}%
\endsong


\beginsong{Que Florezca la Luz}[tags={light, love}, ph={III}, key={Am}, sks={Am, Gm--Dm}]
  \audio[key=Am]{https://soundcloud.com/user-731402154/que-florezca-la-luz1}
  \meter{6}{8}
  \beginchorus
    \[\mn{A}]Que flo|\[Am]rez\[\mn{E}]ca \[\mn{D}]la |\[G]lu\[\mn{C}\mn{B}]z, \[\mn{D}]que flo|\[E7]rez\[\mn{F}]ca \[\mn{E}]la |\[Am]luz \[\up{2}(A7)]
  \endchorus\glueverses
  \notesoff
  \beginchorus
    |\[Dm]La luz del |\[Am]sol, |\[E7]la luz del a|\[Am]mor \[\up{1}A7] \up{2}(| \e)
  \endchorus
  \beginchorus
    |\[Am]Oh gran es|\[G]píritu, gran es|\[E7]píritu de a|\[Am]mor \[\up{2}(A7)]
  \endchorus\glueverses
  \beginchorus
    |\[Dm]Llénanos con tu |\[Am]luz, |\[E7]llena nuestros cora|\[Am]zon\[\up{1}A7]es \up{2}(| \e)
  \endchorus
  \begin{translation}
    Let the light bloom, let the light bloom
    \nextverse
    the light of the sun, the light of love
    \nextverse
    Oh great spirit, great spirit of love
    \nextverse
    fill us with your light, fill our hearts
  \end{translation}
\endsong


\beginsong{Vuela con el Viento}[by={Ayla Schafer},tags={flying},ph={III}, key={Em}, sks={Em, Cm--F\#m}]
  \audio[key=Em]{https://soundcloud.com/aylamusic/vuela-con-el-viento}
  \meter{6}{8}
  \beginchorus\memorize
    |\[\mnc{B}Em]Llé\[^\mn{E}]vame con tus |\[\mnc{D}D]alas \[^\mn{E}]de \[^\mn{D}]luz
    |\[C]águila |\[D]tráenos medicina
    del |\[Em]viento, del aire, las es|\[D]trellas, del sol
    |\[C]brillando, |\[D]guías mi camino
  \endchorus
  \beginchorus
    \ind |\[\mnc{G}Em]Cu\[\mn{F#}]ra, \[\mn{G}]cu\[\mn{F#}]ra, \[\mn{G}]cu\[\mn{F#}]ra, \[\mn{G}]cú\[\mn{B}]ra\[\mn{G}]me, |\[\mnc{F#}D]sa\[\mn{E}]na \[\mn{F#}]to\[\mn{E}]do \[\mn{F#}]lo \[\mn{E}]que \[\mn{F#}]yo \[\mn{D}]llevo
    \ind |\[C]agradesco por mi vida Pa|\[D]chamama yo te amo
  \endchorus
  \beginchorus
    \ind[2] |\[Em] \[\mn{A}]Vue\[\mn{G}]la \[\mn{F#}]con \[\mn{G}]el |\[\mnc{A}D]vien\[\mn{D}]to, |\[C] \[\mn{A}]vue\[\mn{G}]la \[\mn{F#}]con \[\mn{G}]el |\[\mnc{A}D]vien\[\mn{D}]to
  \endchorus
  \notesoff
  \beginchorus
    |^Llévame con tus |^alas de amor
    |^condorcito |^tráenos medicina
    del |^cielo, ilumina |^mi interior
    |^volando ens|^éñame el camino
  \endchorus
  \goto{Cura, cura, cura}
  \goto{Vuela con el viento}
  \begin{translation}
    Carry me with your wings of light, eagle bring us the medicine of the
    wind, of the air, of the stars, of the sun, shining you guide my way.
    \nextverse
    Cure, cure, cure me, heal everything I carry.
    Giving gratitude for my life, Mother Earth I love you.
    \nextverse
    Fly with the wind, fly with the wind.
    \nextverse
    Carry me with your wings of love, Condor bring us the medicine of
    the sky, illuminate my interior, flying you show me the way.
  \end{translation}
\endsong


\beginsong{Abuelita Ayahuasquita}[tags={Aya, flying},ph={III}]
  \meter{6}{8}
  \beginchorus
    |\[G] \[\mn{G}]Ab\[\mn{D}]uelita A|\[D]yahuasquita en|\[C]séñame a vo|\[G]lar
    |\[G] Como paja|\[D]ritos que vuelan |\[C]libres en la inmensi|\[G]dad
    Cura |\[A]cura cura |\[A7]cura mi cora|\[D]zón |\[\up{2}(D7)]-
  \endchorus
  \beginchorus
    Y vo|\[C]lando encontra|\[G]ré paz y liber|\[D]tad | \e
    Limpia y |\[C]cambia mis tris|\[G]tezas a color |\[A]verde esperan|\[D]za | \e
  \endchorus
  \beginverse
    Vuela |\[C]vuela pa\[D]lomi|\[G]ta vuela |\[C]vuela con A\[D]ya|\[G]huasca
    Limpia y |\[C]cura mis \[D]triste|\[G]zas a co|\[C]lor verde es\[D]peran|\[G]za |\[D] |\[C] |\[G]
  \endverse
  \beginverse
    |\[C] Lara lara la|\[G]irai la|\[D]raira | \e
    |\[C] Lara lara la|\[G]irai rai
    |\[C] Lara lara la|\[G]irai rai
    |\[C] Lara lara la|\[G]irai la|\[D]raira | \e
  \endverse
  \begin{translation}
    Grandma beloved Ayahuasca teach me to fly
    Like little birds flying free in the immensity
    Heal, heal, heal, heal my heart
    \nextverse
    And flying I will find peace and freedom
    Cleanse and change my sadness to green hope
    \nextverse
    Fly, fly, little dove, fly, fly with Ayahuasca
    Cleanse and heal my sadness to green hope
  \end{translation}
\endsong


\beginsong{Pachamama Pachacamaq}[by={Kike Pinto},tags={Mother Earth, thankfulness, you},ph={III}]
  \meter{3}{4}
  % this song uses two sets of memorized chords. so initialize chord registers here:
  \newchords{chords_pachamamacamaq_a}\newchords{chords_pachamamacamaq_b}
  \beginverse\memorize[chords_pachamamacamaq_a]
    \[^\mn{B}]Pac\[^\mn{D}]ha|\[\mnc{E}C]mama |Pacha|camaq | luz de |vida, |\[Fmaj7]luz de a|\[C]mor | \e
    te agra|\[Am]dezco |\[Em]por mi |\[Am]vida | por mi a|\[F]liento y |por mi |\[Em]voz | \e
    te agra|\[Am]dezco |\[Em]por mis |\[Am]sueños | y el la|\[F]tir de un |cora|\[Em]zón
    | | | \e
  \endverse
  \notesoff
  \beginverse\replay[chords_pachamamacamaq_a]
    Por el |^sol de |cada |día | por el |agua y |^su sa|^bor | \e
    por el |^aire y |^por el |^viento | por el |^fuego y |su ca|^lor | \e
    por mi |^gente y |^por mi |^pueblo | y el can|^tar de es|ta can|^ción
    | | | \e
  \endverse
  \beginverse\memorize[chords_pachamamacamaq_b]
    \ind Por mi |\[F]tierra y |mi fa|\[C]milia, | porque |\[F]son par|te de |\[C]mi | \e
    \ind y yo |\[Am]siendo |\[Em]parte de |\[Am]ellos | puedo |\[F]ser par|te de |\[Em]ti | \e
    \ind y yo |\[Am]siendo |\[Em]parte de |\[Am]ellos | puedo |\[F]ser par|te de |\[Em]ti
    \ind | | | \e
  \endverse
  \beginverse\replay[chords_pachamamacamaq_b]
    \ind Palo|^mita |de mi |^vida, | palo|^mita |de mi a|^mor | \e
    \ind así |^quiso |^Pacha|^camaq | que nos |^junte|mos los |^dos | \e
    \ind y así |^quiere |^Pacha|^mama | que nos |^ame|mos los |^dos
    \ind | | | \e
  \endverse
  \brk
  \textnote{suomeksi:} % by: Malla
  \beginverse\replay[chords_pachamamacamaq_a]
    Äiti |^Maa, |Kaiken |Luoja, | lähde |valon, |^rakkau|^den | \e
    kii|^tän |^elä|^mästäin, | henges|^täin ja |äänes|^täin, | \e
    kii|^tän |^unel|^mistain, | syk|^keestä |sydä|^men | | | \e
  \endverse
  \beginverse\replay[chords_pachamamacamaq_a]
    Kii|^tän |aurin|gosta, | veden |raik|^kaudes|^ta | \e
    il|^masta |^ja tuu|^lesta, | läm|^möstä |liekki|^en | \e
    hei|^mosta |^ja ko|^dista, | lau|^lusta |sydä|^men | | | \e
  \endverse
  \beginverse\replay[chords_pachamamacamaq_b]
    \ind Kiitän |^maasta |ja per|^heestä, | sillä |^ne on |osa |^mua | \e
    \ind ja kos|^ka oon |^osa |^heitä, | voin myös |^olla |osa |^sua | \e
    \ind ja kos|^ka oon |^osa |^heitä, | voin myös |^olla |osa |^sua | | | \e
  \endverse
  \beginverse\replay[chords_pachamamacamaq_b]
    \ind E|^lä|mäni |^lintu, | rakkau|^teni |kyyhky|^nen, | \e
    \ind niin |^tahtoi |^Kaiken |^Luoja, | että |^yh|distym|^me | \e
    \ind ja niin |^tahtoo |^Äiti |^Maa, | että |^ra|kastam|^me | | | \e
  \endverse
  % % comment out English translation now that we have a Finnish version
  % \begin{translation}
  %   Pachamama Pachacamaq, light of life, light of love
  %   I thank you for my life, for my breath and for my voice
  %   I thank you for my dreams and the heartbeat
  %   \nextverse
  %   For the sun, for each day, for the water and its flavor
  %   For the air and for the wind, for the fire and its heat
  %   For my people (\emph{of the village}) and for the people and the singing of this song
  %   \nextverse
  %   For my land and my family, because they are part of me
  %   And by being part of them I can be part of you
  %   And by being part of them I can be part of you
  %   \nextverse
  %   Little dove of my life, little dove of my love
  %   So Pachacamaq wanted the two of us to join
  %   And so Pachamama wants us to love each other
  % \end{translation}
  \begin{explanation}
    \begin{description}
      \item[Pachamama:] the Goddess of the Earth in \emph{Inca} mythology, ``World Mother'',
        worshipped by indigenous people of the Andes
      \item[Pachacamaq:] a deity responsible for creating the first humans, originally worshipped
        in the \emph{Ichma} culture (which was later absorbed to the Incan empire)
    \end{description}
  \end{explanation}
\endsong


\beginsong{Machi}[by={Peia},ph={III}]
  \audio[pitch={432}]{https://www.youtube.com/watch?v=D7os9V-n7rs}
  \musicnote{Intro and outro repetitions are in 6/8, but the main part of the song is in 4/4.}
  \meter{6}{8}
  \beginchorus
    \ind |\[\mnc{A}F#m]Ma\[\mn{B}]chi |\[\mnc{C#}A]machi |\[E] \[\mn{B}]ma\[\mn{C#}]chi|\[\mnc{F#}F#m]ma \up{2}(| \e)
  \endchorus
  \beginverse
    |\[D] Machi |\[F#m]cura |\[D] Machi |\[F#m]sana
    |\[D] Machi |\[F#m]cántame |\[D]una |\[F#m]nana | \e
  \endverse
  \goto{Machi machi machima}
  \beginverse
    |^ Yo no |^lloro |^ yo sólo |^canto
    |^ Con tu en|^canta |^Pachamama |^Madre Tierra
  \endverse
  \goto{Machi machi machima}
  \begin{translation}
    Machi is curing, Machi is healing
    Machi sings me a lullaby
    \nextverse
    I do not cry, I just sing
    With your love Pachamama Mother Earth
  \end{translation}
  \begin{explanation}
    A \textbf{machi} is a traditional healer and religious leader in the Mapuche culture
    of Chile and Argentina, usually a woman.
  \end{explanation}
\endsong


\beginsong{Luz del Bosque\\I am the Light of the Forest}[by={Adrian Freedman},tags={light},ph={III, IV}]
  % This song is apparently originally in English and by Adrian Freedman.
  \audio[]{https://soundcloud.com/adrianfreedman/i-am-the-light-of-the-forest-la-luz-del-bosque}
  % See (Issa Elle's Spanish version): https://soundcloud.com/issa-elle-1/la-luz-del-bosque
  \newchords{chords_luzdelbosque_a}\newchords{chords_luzdelbosque_b}
  \beginchorus\memorize[chords_luzdelbosque_a]
    |\[\mnc{C}C]Somos \[^\mn{D}]la \[^\mn{E}]luz \[^\mn{C}]del |\[\mnc{G}G]bosque, e|\[Dm]spíritu de todas e|\[Am]dades
  \endchorus\glueverses
  \notesoff
  \beginchorus\memorize[chords_luzdelbosque_b]
    |\[Dm]Somos la luz di|\[C]vina, sabidu|\[G]ría de los |\[Am]mares
    \up{2}(| \e)
  \endchorus
  \beginchorus\replay[chords_luzdelbosque_a]
    Transfor|^mamos el dol|^or tray|^endo todo a la l|^uz
  \endchorus\glueverses
  \beginchorus\replay[chords_luzdelbosque_b]
    Con el e|^spíritu de mis \up{*}an|^cestros todo el |^dia y la |^noche
    \up{2}(| \e)
  \endchorus
  \altlyr{abuelitos}
  \textnote{suomeksi:}
  \beginchorus\replay[chords_luzdelbosque_a]
    |^Olemme valo |^metsän, |^henki kaikkien aiko|^jen
  \endchorus\glueverses
  \beginchorus\replay[chords_luzdelbosque_b]
    |^Olemme pyhä |^valo, |^viisaus meri|^en
    \up{2}(| \e)
  \endchorus
  \beginchorus\replay[chords_luzdelbosque_a]
    Muun|^namme kärsi|^myksen |^tuomalla kaiken val|^oon
  \endchorus\glueverses
  \beginchorus\replay[chords_luzdelbosque_b]
    \up{*}Esi|^vanhempain hengen |^kanssa koko |^päivän ja koko |^yön
    \up{2}(| \e)
  \endchorus
  \altlyr{Isovanhempain}
  %% Translation commented out for saving space, since we have the Finnish version
  % \begin{translation}
  %   We are the light of the forest, spirit of all ages
  %   \nextverse
  %   We are the divine light, wisdom of the seas
  %   \nextverse
  %   We transform the pain by bringing everything to light
  %   \nextverse
  %   With the spirit of the grandmother singing all night
  %   With the help of the mother singing all night
  %   \nextverse
  %   With the spirit of the grandfather singing all night
  %   With the help of the father singing all night
  % \end{translation}
\endsong


\beginsong{La Semilla}[by={Shimshai},tags={love, heart},ph={III}]
  \meter{6}{8}
  \newchords{chords_lasemilla_a}\newchords{chords_lasemilla_b}
  \beginverse\memorize[chords_lasemilla_a] % memorize chords into a named register
    \lrep \[^\mn{A}]En |\[\mnc{D}Dm]el pre\[A7]sente en|\[Dm]contrarás,
    en |\[Fmaj7]el cora\[C]zón tu ve|\[Dm]rás \rrep
    \lrep Este a|\[Fmaj7]mor que \[C]fluye bi|\[Dm]en adentro,
    a|\[Fmaj7]quí hay \[C]una se|\[Dm]milla \rrep
    \vspace{1em}
    \lrep \up{2}(La se|\[(Dm)]milla) ha \[A7]sido sem|\[Dm]brada en ti
    y el a|\[Fmaj7]mor es el \[C]agua que |\[Dm]la alimenta \rrep
  \endverse
  \notesoff
  \beginverse\memorize[chords_lasemilla_b] % memorize chords into a named register
    \ind \lrep Con a|\[F]mor e\[C]ternamen|\[Dm]te crecerá,
    \ind mi |\[F]madre lo \[C]hace a|\[Dm]sí \rrep
    \ind Con a|\[F]mor e\[C]ternamen|\[Dm]te crecerá,
    \ind mi |\[F]padre lo \[C]hace a|\[Dm]sí
    \ind Con a|\[F]mor e\[C]ternamen|\[Dm]te crecerá,
    \ind la |\[F]fuerza me \[C]hace a|\[Dm]sí \[A7]
  \endverse
  \brk
  \textnote{in English:}
  \beginverse\replay[chords_lasemilla_a] % replay chords from a named register
    \lrep |^Be in the ^now you will |^find,
    and |^be in the ^heart you will |^see \rrep
    \lrep This |^love that ^flows so |^deep inside,
    with|^in there ^lies a |^seed \rrep
    \vspace{1em}
    \lrep \up{2}(A |^seed) has been ^planted in|^side of your heart
    and |^love is the ^water that |^feeds \rrep
  \endverse
  \beginverse\replay[chords_lasemilla_b] % replay chords from a named register
    \ind \lrep With |^love e^ternal|^ly it will grow,
    \ind my |^mother she ^makes it |^so \rrep
    \ind With |^love e^ternal|^ly it will grow,
    \ind my |^father he ^makes it |^so
    \ind With |^love e^ternal|^ly it will grow,
    \ind the |^force she ^makes me |^so \[(A7)]
  \endverse
\endsong


\beginsong{Oso Blanco}[by={Bóveda Celeste},ph={III}]
  \audio[key=Bm]{https://soundcloud.com/bovedaceleste-1/oso-blanco-arnaldo-andressa}
  %% alt. chords: C -> G -> Em -> Am
  % \capo{2}
  \beginchorus\memorize
    \[^\mn{G}]En este \[^\mn{E}]ca|\[\mncii{G}{E}Am]mi|no rumbo a la |\[\mncii{A}{C}Fmaj7]selva | \e
    me he encon|\[G]trado un pajar|illo sona|\[F]jero | \e
    que inspi|\[Am]rado en el vuelo del |Águila encon|\[Fmaj7]{tró en su} cora|zón
    este |\[G]canto que brota del |Alma aquí y a|\[F]hora | \e
  \endchorus
  \notesoff
  \beginchorus
    Paja|^rillo sona|jero encon|^traste en la Visión | \e
    a un Oso |^blanco con alas de |Cóndor y mi|^rada de Jagu|ar
  \endchorus
  \beginchorus
    Entre|^lazas marirí | las plumas del |^Águila y del C|óndor
    donde |^llueven las gotas de |cielo que hoy fe|^cundan la Se|milla \up{2}(| \e)
    % 'llueven' (rain) or 'caen' (fall)?
  \endchorus
  \textnote{\emph{D.C. al Fine}}
  \beginchorus
    En este ca|^mi|no rumbo a la |^selva | \e
    me he encon|^trado un coraz|ón sonaj|^ero | \e
  \endchorus
  \begin{translation}
    On this road to the forest
    I have found a little bird rattle
    that inspired, by the flight of the Eagle found in its heart,
    this song that springs from the Soul here and now.
    \nextverse
    Little bird rattle, found in the Vision
    of a White Bear with wings of Condor and the gaze of Jaguar.
    \nextverse
    Intertwine \emph{marirí}, the feathers of the Eagle and the Condor,
    where the drops of sky rain that today fertilize the Seed.
    \nextverse
    On this road to the forest
    I have found a heart rattle.
  \end{translation}
  \begin{explanation}
    \begin{description}
      \item[Marirí:] see song \emph{Marirí}
    \end{description}
  \end{explanation}
\endsong


\beginsong{Nadi Wewe}[by={Kuitzi Moezzi},tags={Aya},ph={III}]
  \beginverse
    \[^\mn{A}]Santa |\[Am]madre A\[^\mn{C}]ya\[^\mn{A}]huas|\[\mncii{G}{E}C]quita
    Per|\[Em]mítame recibir|\[Am]te
  \endverse
  \notesoff
  \beginchorus
    \ind Nadi |\[Dm]wewe nadi |\[Am]wewe
    \ind Nadi |\[Em]wewe nadi |\[Am]we
  \endchorus
  \beginverse
    Aya|^huasca curande|^rita
    En|^séñame a curar|^me \goto{Nadi wewe}
  \endverse
  \beginverse
    Sabia |^madre Ayahuas|^quita
    Apr|^enderé tenerte |^fe \goto{Nadi wewe}
  \endverse
  \beginverse
    Aya|^huasca visio|^naria
    Áb|^reme la conscien|^cia \goto{Nadi wewe}
  \endverse
  \beginverse
    Aya|^huasca enlaza|^dora
    Mis rela|^ciones las cuida|^ré \goto{Nadi wewe}
  \endverse
  \beginverse
    Amo|^rosa madre Aya|^huasca
    Mi cora|^zón entrega|^ré \goto{Nadi wewe}
  \endverse
  \beginverse
    Bendita |^madre Aya|^huasca
    Tu bendi|^ción comparti|^ré \goto{Nadi wewe}
  \endverse
  \begin{translation}
    Holy mother Ayahuasca, allow me to receive you
    \nextverse
    Ayahuasca, healer, teach me to heal myself
    \nextverse
    Wise mother Ayahuasca, I will learn to have faith
    \nextverse
    Ayahuasca, visionary, open my conscience
    \nextverse
    Ayahuasca, uniter, look after my relationships
    \nextverse
    Loving mother Ayahuasca, my heart will surrender
    \nextverse
    Blessed mother Ayahuasca, your blessing I will share
  \end{translation}
\endsong


\beginsong{Sirenita Bobinzana}[by={Artur Mena},ph={III, IV}]
  \newchords{chords_sirenita_a}\newchords{chords_sirenita_b}
  \meter{3}{4}
  \beginchorus\memorize[chords_sirenita_a]
    \[^\mn{E}]Si\[^\mn{D}]re|\[\mnc{E}C]nita \[^\mn{G}]de \[^\mn{E}]los |\[\mnc{D}G]ríos, \[^\mn{G}]dan\[^\mn{A}]za |\[\mnc{B}B7]danza \[^\mn{A}]con \[^\mn{B}]el |\[\mnc{E}Em]viento | \e
  \endchorus\glueverses\beginchorus\memorize[chords_sirenita_b]
    Con  tus |\[G]flores y a|romas, per|fumas los \[B7]cora|\[Em]zones | \e
  \endchorus
  \notesoff
  \beginchorus\replay[chords_sirenita_a]
    Cura |^cura cuerpe|^citos, limpia |^limpia espiri|^titos | \e
  \endchorus\glueverses\beginverse\replay[chords_sirenita_b]
    Canta|^remos ica|ritos abue|lita cu^rande|^ra | \replay \e
    Danza|^remos muy jun|titos sire|nita ^Bobin|^zana | \e
  \endverse
  \beginchorus\replay[chords_sirenita_a]
    Raira |^rairai raira |^rairai raira |^rairai raira |^rairai
  \endchorus\glueverses\beginchorus\replay[chords_sirenita_b]
    Raira |^ra-irai raira |ra-irai raira |ra-irai ^raira|^rairai
  \endchorus
  \beginchorus\replay[chords_sirenita_a]
    Medi|^cina de la |^selva, eres |^tu Bobin|^zana | \e
  \endchorus\glueverses\beginchorus\replay[chords_sirenita_b]
    Curas |^males das visi|ones a tus |hijos ^en las |^dietas \e
  \endchorus
  \beginchorus\replay[chords_sirenita_a]
    Cura |^cura cuerpe|^citos, limpia |^limpia espiri|^titos | \e
  \endchorus\glueverses\beginverse\replay[chords_sirenita_b]
    Canta|^remos ica|ritos en sesion|cita de ^aya|^huasca | \replay \e
    Danza|^remos muy jun|titos sire|nita ^Bobin|^zana | \e
  \endverse
  \begin{translation}
    Little princess of the rivers, dance dance with the wind
    With your flowers and aromas, you perfume the hearts
    \nextverse
    Heal heal our little bodies; cleanse cleanse our spirits
    We will sing icaros; Grandmother, the healer
    We will dance much together, princess Bobinzana.
    \nextverse
    Jungle medicine are you, Bobinzana
    You cure ill visions in your dieting children
    \nextverse
    Heal heal our little bodies; cleanse cleanse our spirits
    We will sing icaros in ayahuasca sessions
    We will dance much together, princess Bobinzana.
  \end{translation}
  \brk
  \begin{explanation}
    \begin{description}
      \item[Bobinzana:] \emph{Calliandra angustifolia}, an Amazonian tree with healing properties, a plant teacher
    \end{description}
  \end{explanation}
\endsong


\beginsong{Agua de Estrellas}[by={Miguel Molina},tags={stars, Mother Earth},ph={III, IV}]
  \transpose{7}
  \beginchorus
    \[\mn{A}]En \[\mn{C}]tus |\[\mnc{D}Dm]ojos de agua \[\mn{C}]in\[\mn{A}]fi|\[\mnc{C}F]ni\[\mn{A}]ta
    Se |\[C]bañan las estrel|\[G]litas \[C]ma|\[Dm]ma | \e
  \endchorus
  \notesoff
  \beginchorus
    Agua de |\[F]luz agua de es|\[Am]trellas
    Pacha|\[G]mama vie\[C]nes del |\[Dm]cielo | \e
  \endchorus
  \beginchorus\memorize
    |\[Dm]Limpia |limpia
    |\[F]Limpia corazón |\[C]agua bril\[G]lante
     \replay |^Sana |sana
    |^Sana corazón |^agua ben^dita
     \replay |^Calma |calma
    |^Calma corazón |^agua del ^cielo ma|\[Dm]ma | \e
  \endchorus
  \begin{translation}
    In your eyes of eternal water
    the stars bathe, mother
    \nextverse
    Water of light, water of the stars
    Pachamama comes from the sky
    \nextverse
    Cleanse, cleanse, cleanse the heart, brilliant water
    Heal, heal, heal the heart, blessed water
    Calm, calm, calm the heart, water of the sky, mother
  \end{translation}
\endsong


\beginsong{Tzen Tze Re Rei}[by={Nase Chiriap},ex={shuar, español},ph={III, IV}]
  \audio[]{https://soundcloud.com/loli-cosmica/tzen-tze-re-re}
  \audio[]{https://www.youtube.com/watch?v=vlGsRDuLUe0}
  \audio[]{https://www.youtube.com/watch?v=PC26I9uFW\_Q}
  \beginverse
    |\[\mnc{E}Am]Y es \[\mn{G}]a|\[\mn{E}]quí \[\mn{G}]donde |\[\mnc{E}C]quiero es\[\mn{D}]tar |\[\mn{C}]junto a \[\mn{D}]ti
  \endverse
  \beginverse
    \ind |\[\mn{E}]Rei \[\mn{D}]rei |\[\mn{C}]rei
    \ind \up{¤}(Tzen tze re|\[C]rei rei |rei) \altlyr[¤]{\emph{Skip this line if coming from} ``\ldots junto a ti''}
    % Loli Cosmica's version: The next line is omitted on (only) the very first time
    \ind Tzen tze re|rei rei |rei
    \ind Tzen tze re|rei rei rei |rei
    \ind Rei |rei | Rei |\[Am]rei | | | \e
  \endverse
  % \goto{Rei rei rei} % This repeat is in Loli Cosmica's version
  \beginverse\memorize
    De tus |\[Am]ojos |salen los co|\[C]lores
    |Del uni|verso |pedimos |\[Am]todo | | | \e
    Y |esos co|lores |\[C]son el ali|mento
    |Para la exis|tencia |que somos |\[Am]todo | | | \e
  \endverse
  \goto{Rei rei rei}
  \beginverse
    De tu |^voz |salen los so|^nidos
    |Del uni|verso |que oímos |^todo | | | \e
    Y |esos so|nidos |^son el ali|mento
    |Para la exis|tencia |que somos |^todo | | | \e
  \endverse
  \goto{Rei rei rei}
  \beginverse
    De tus |^manos |salen las ca|^ricias
    |Del uni|verso |que sentimos |^todo | | | \e
    Y |esas ca|ricias |^son el ali|mento
    |Para la exis|tencia |que somos |^todo | | | \e
  \endverse
  \goto{Rei rei rei}
  \goto{Y es aquí}
  \goto{Rei rei rei}
  \goto{Y es aquí}
  \goto{Rei rei rei}
  \begin{translation}
    And it’s here that I want to be together with you
    \nextverse
    From your eyes come the colours of the universe
    We ask for all and those colours
    Are the food for the existence that we are all
    \nextverse
    From your voice come the sounds of the universe
    We hear all and those sounds
    Are the food for the existence that we are all
    \nextverse
    From your hands come the caresses of the universe
    We feel all and those caresses
    Are the food for the existence that we are all
  \end{translation}
  \begin{explanation}
    \textbf{Tzen tze re rei} \emph{(Shuar language)} means the arrow that advances and opens doors.
    Also, it is an onomatopoeia for the chattering of a chinchilla, a dormouse or a squirrel.
  \end{explanation}
\endsong


\beginsong{Sólo Dios Sabe si Vuelvo}[by={Julian Herreros Riveira},ph={III, IV}]
  \transpose{5}
  \beginverse
    |\[\mnc{E}Am]A|'\[^\mn{D}]bre\[^\mn{E}]te \[^\mn{D}]flor|\[\mnc{C}F]ci\[^\mn{A}]ta de los |\[^\mn{C}]cua\[^\mn{A}]tro vien\[^\mn{G}]tos |\[\mnc{E}C]yari | \e
    Oloro|cita perfu|\[G]mera doctor|\[Am]cita | \e
    Estrelli|ta de siete |\[Em]flechas Tonān|\[Am]tzina | \e
    Ay danza|\[C]remos hasta |\[G]las claras del |\[Am]día | \e
    para escu|char la voz de |\[Em]los que ya se |\[Am]fueron | \e
  \endverse
  \notesoff
  \beginverse
    \ind |^Naa|a na na na |^naana na na |naina na na |^naana | \e
    \ind Na na na |naana na na |^naina na na |^naana | \e
    \ind Na na na |naana na na |^naina na na |^naana | \e
  \endverse
  \beginverse
    |^So|'n tus aguas |^que le dan la |vida a mi |^pueblo | \e
    Ay las que |brotan por mis |^ojos triste|^mente | \e
    porque en mi |tierra ya no |^se le canta al |^agua | \e
    Son las que |^corren por tus |^ojos Pacha|^mama | \e
    y rever|decen en mi |^quebrada de a|^mor | \e \goto{Naa na na na}
  \endverse
  \beginverse
    |^Lle|'no de ale|^gría danza |la muerte a mi |^lado | \e
    Chuma bo|rracherita |^pinta gente |^yari | \e
    Cuatro oto|rongos magos |^llegan a la |^fiesta | \e
    Se alza en el |^cielo el brillo a|^zul de la flor |^blanca | \e
    Camino |rojo sólo |^Dios sabe si |^vuelvo | \e \goto{Naa na na na}
  \endverse
  \begin{translation}
    Open flower of the four winds \emph{yari}
    Dear smell perfumes the healer
    Little star of seven arrows, Mother Goddess
    Oh we will dance until the clear of the day
    to hear the voice of those who have already left
    \nextverse
    It's your waters that give life to my people
    Oh the ones that sprout sadly through my eyes
    because in my land water is no longer sung to
    They are the ones that run through your eyes Pachamama
    and revive in my stream of love
    \nextverse
    Full of joy dance, death by my side
    Drunken folly paints people \emph{yari}
    Four jaguars, wizards arrive at the party
    The blue glow of the white flower rises in the sky
    Red path, only God knows if I return
  \end{translation}
\endsong

\scleardpage
\beginsong{En la Selva un Río}[by={Kike Pinto},ph={III, IV}]
  \normalsize % make smaller to fit all verses on one spread
  \newchords{chords_enlaselva_a}\newchords{chords_enlaselva_b}
  \meter{6}{8}
  \beginchorus\memorize[chords_enlaselva_a]
    \[^\mn{E}]En |\[\mnc{E}Em]la sel\[^\mn{D}]va un |\[^\mn{B}]río en noche os|\[G]cura | \e
    |\[Am] va surcan|\[Bm]do una ca|\[Em]noa. | \e
  \endchorus\glueverses
  \beginverse\memorize[chords_enlaselva_b]
    la |\[Em]vigilan des|de la espe|\[G]sura, | \e
    |\[Am] los jagua|\[Bm]res y las |\[Em]boas. | \e
    la |\[Em]protegen des|de la espe|\[G]sura, | \e
    |\[Am] los jagua|\[Bm]res y las |\[Em]boas.
    \ind |Trai nanai na|\[Bm]nai na|nai |\[Em]nai Nanai nanai |\[G]nai
    \ind |\[Am] va surcan|\[Bm]do una ca|\[Em]noa
    \ind |Trai nanai na|\[Bm]nai na|nai |\[Em]nai Nanai nanai |\[G]nai
    \ind |\[Am] mil jagua|\[Bm]res y mil |\[Em]boas | |\[Em]Trai nanai na |\ldots | | | | | | \e
  \endverse
  \notesoff
  \beginchorus\replay[chords_enlaselva_a]
    U|^na voz me can|ta en el o|^ído | \e
    |^ una bel|^la melo|^día, | \e
  \endchorus\glueverses
  \beginverse\replay[chords_enlaselva_b]
    y u|^na luz dibu|ja ante mis |^ojos | \e
    |^ visiones de |^un nuevo |^día, | \e
    y u|^na luz dibu|ja ante tus |^ojos | \e
    |^ visiones de |^un nuevo |^día.
    \ind |Trai nanai na|^nai na|nai |^nai Nanai nanai |^nai
    \ind |^ una bel|^la melo|^día
    \ind |Trai nanai na|^nai na|nai |^nai Nanai nanai |^nai
    \ind |^ visiones de |^un nuevo |^día | |^Trai nanai na |\ldots | | | | | | \e
  \endverse
  \beginchorus\replay[chords_enlaselva_a]
    El |^viento me tra|e un dulce a|^roma | \e
    |^ alcanfo|^res y azu|^cenas, | \e
  \endchorus\glueverses
  \beginverse\replay[chords_enlaselva_b]
    y a|^lza el vuelo en mi al|ma una pa|^loma | \e
    |^ rompe |^rompe las ca|^denas, | \e
    y a|^lza el vuelo en mi al|ma una pa|^loma | \e
    |^ rompe |^rompe las ca|^denas.
    \ind |Trai nanai na|^nai na|nai |^nai Nanai nanai |^nai
    \ind |^ alcanfo|^res y azu|^cenas
    \ind |Trai nanai na|^nai na|nai |^nai Nanai nanai |^nai
    \ind |^ rompe |^rompe las ca|^denas | |^Trai nanai na |\ldots | | | | | | \e
  \endverse
  \beginchorus\replay[chords_enlaselva_a]
    Re|^ma rema re|ma cano|^ero | \e
    |^ por los |^ríos de mis |^venas. | \e
  \endchorus\glueverses
  \beginverse\replay[chords_enlaselva_b]
    can|^ta canta can|ta curan|^dero, | \e
    |^ cura cu|^rame mis |^penas. | \e
    so|^pla sopla so|pla perfu|^mero, | \e
    |^ rompe |^rompe mis ca|^denas.
    \ind |Trai nanai na|^nai na|nai |^nai Nanai nanai |^nai
    \ind |^ alcanfo|^res y azu|^cenas
    \ind |Trai nanai na|^nai na|nai |^nai Nanai nanai |^nai
    \ind |^ por los ri|^os de mis |^venas | |^Trai nanai na |\ldots | | | | | | \e
  \endverse
  \begin{translation}
    In the jungle on a river on a dark night rides a canoe. (x2)
    They watch it from the thicket, the jaguars and the boas.
    They protect it from the thicket, the jaguars and the boas.
    Trai nanai nanai nanai nai Nanai nanai nai, rides a canoe;
    Trai nanai nanai nanai nai Nanai nanai nai, a thousand jaguars and a thousand boas
    \nextverse
    A voice sings in my ear a beautiful melody, (x2)
    and a light draws before my eyes visions of a new day,
    and a light draws before my eyes visions of a new day.
    Trai nanai nanai nanai nai Nanai nanai nai, a beautiful melody;
    Trai nanai nanai nanai nai Nanai nanainai visions of a new day.
    \nextverse
    The wind brings a sweet aroma of camphora trees and lilies, (x2)
    and in my soul a dove rises to fly, breaking breaking the chains,
    and in my soul a dove rises to fly, breaking breaking the chains.
    Trai nanai nanai nanai nai Nanai nanai nai, camphora trees and lilies;
    Trai nanai nanai nanainai nanai nanainai, breaking the chains.
    \nextverse
    Row row row, canoteer, by the rivers of my veins. (x2)
    Sing sing sing, curandero, heal heal my pains.
    Blob blow blow, perfumer, break break my chains.
    Trai nanai nanai nanai nai Nanai nanai nai, camphores and lilies;
    Trai nanai nanai nanai nai nanai nanai nai, by the rivers of my veins.
  \end{translation}
  % Image downloaded from: https://openclipart.org/detail/120811/jaguar
  % Image license: in the public domain
  \imagecc[2]{jaguar_drawing_bw_transparent_bg_PD__1081px.png}%
\endsong


\beginsong{Corazón es lo Único que Tengo}[by={Nubia Rodriguez},tags={heart}, ph={III, IV}]
  %\transpose{5}
  \meter{6}{8}
  \beginchorus\memorize
    Gran E|\[\mnc{E}Em]spíritu, gran a|buela, gran a|\[\mnc{D}G]bu\[^\mn{B}]elo | \e
    Como |\[D]soy me pre|sento ante |\[Em]ti | \e
    Como |\[G]soy te pi|do bendi|\[Bm]cio|nes
    Y agra|\[D]dezco el cora|zón que has puesto en |\[Em]mi | \e
  \endchorus
  \notesoff
  \beginchorus
    Cuando |^vengo nomás |vengo, nomás |^vengo | \e
    Gran E|^spíritu sab|rás a lo que |^vengo | \e
    A entre|^gar mi cora|zón, mi cora|^zón | \e
    Cora|^zón que es lo |único que |^tengo | \e
  \endchorus
  \beginchorus
    Cora|^zón que es lo |único que |^tengo | \e
    Cora|^zón que es lo |único que |^tengo | \e
  \endchorus
  \begin{translation}
    Great Spirit, great grandmother, great grandfather
    As I am, I present myself to you
    As I am I ask you blessings
    And I thank for the heart that you have placed in me
    \nextverse
    When I come here I just come, I just come
    Great Spirit you will know why I am here
    To give my heart, my heart
    Heart that is all I have
    \nextverse
    Heart that is all I have
    Heart that is all I have
  \end{translation}
\endsong


\beginsong{Cura Sana}[by={Andrés Córdoba},ex={cofan, quechua, español},ph={III, IV}]
  \beginverse
    \[^\mn{E}]{O\ldots}|\[\mnc{A}Am]coremajaï jojore |\[F]jasparuñaï
    curaima|\[C]gente y pinta selva tierra |\[E]urumanyaï | \e
    I|\[Am]quisimanyairo amor |\[F]azul cielo
    celeste |\[C]azulejero que vie|\[E]ne y pinta | \e
  \endverse
  \notesoff
  \beginverse
    \ind |\[Am]Cura cura, |\[F]sana sana: |\[C]taita celoc |\[E]be licencialc | \e
    \ind |\[Am]Cura cura, |\[F]sana sana: |\[C]taita celoc |\[E]ashlepay | \e
  \endverse
  \beginverse
    I|^quisimanyairo uru|^maña y gente
    cordi|^llera montaña que vie|^ne y llegá | \e
    Ya|^ge yage pinta pinta |^azulejero,
    chuma |^rasca borrachera que |^viene y pinta | \e
  \endverse
  \goto{Cura cura}
  \beginverse
    O|^coremajaï jojore |^tacro yocro-
    manje|^ro pinta selva tierra |^urumanyaï | \e
    I|^quisimanyairo amor |^azul cielo
    celeste |^azulejero que vie|^ne y pinta | \e
  \endverse
  \goto{Cura cura}
  \beginverse
    O|^coremajaï jojore |^jasparuñaï
    curaima|^gente y pinta amor cielo |^tierra y vida
    |^Astro centro mundo cos|^mos seres
    interplane|^tario que llegá |^cura y sana
  \endverse
  \goto{Cura cura}
\endsong


\beginsong{Pura Medicina}[by={Mallkikuna},ph={IV}]
  \beginverse
    \[^\mn{E}]Pu\[^\mn{D}]ra \[^\mn{E}]me\[^\mn{D}]di|\[\mnc{E}Am]cina tus ojitos, | pura medi|cina el co|\[C]lor
    Pura medi|\[C]cina toda vida, | cuando sonri|\[G]e tu interi|\[Am]or
  \endverse
  \notesoff
  \beginverse
    Pura medi|^cina Taita Inti, | pura medi|cina Padre |^Sol
    Pura medi|^cina toda vida, | luz brillando |^de tu interi|^or
  \endverse
  \beginverse
    Pura medi|^cina la familia, | pura medi|cina sus ma|^nitos
    Pura medi|^cina la sonrisa, | juntos alum|^bramos esta |^vida
  \endverse
  \begin{translation}
    Pure medicine your eyes, pure medicine the color
    Pure medicine all life, when you smile inside
    \nextverse
    Pure medicine Taita Inti, pure medicine Father Sun
    Pure medicine all life, light shining from your interior
    \nextverse
    Pure medicine the family, pure medicine their little hands
    Pure medicine the smile, together we light this life
  \end{translation}
\endsong


\scleardpage
\beginsong{Guacamayo}[by={Danit Treubig},ph={IV}]
  \audio[key=C\#m]{https://danit.bandcamp.com/album/aliento}
  \audio[key=C\#m]{https://www.youtube.com/watch?v=4s3uheDMRl0}
  \audio[key=C\#m]{https://www.youtube.com/watch?v=oSgKCfJhWbM} % live
  \capo{4}
  \mnbeginchorus
    \[\mn{A}]Te agra|\[Am]dezco |\[\mnc{F}Fmaj7&5]{ por llegar} a mi |\[\mnc{G}G]co\[\mn{A}]ra\[\mn{G}]zón | \e
    her|\[\mnc{F}F]mosa criatura del |\[\mnc{E}Am]vien\[\mn{C}]to | \e
    \[\mn{A}]Te agra|dezco |\[\mnc{F}Fmaj7&5]{ por volar} en mi |\[\mnc{G}G]in\[\mn{A}]ter\[\mn{G}]ior | \e
    Tus co|\[\mnc{E}Em]lores me llevan \[\mn{D}]a|\[\mncii{C}{A}Am]dentro | \e\sublyr{\up{2}(hey hey hey)}
  \mnendchorus
  \mnbeginchorus
    \ind |\[\mnc{G}G]Pinta pinta pinta con el |\[Em]movimie\[\mn{E}]nto
    \ind |\[\mnc{B}G]de tus plumas sale |\[\mnc{G}Em]una vibración
    \ind |\[\mnc{B}G]que me abraza e|\[\mnc{G}Em]ternamente
    \ind |\[\mnc{B}G]que me enseña e|\[\up{1}\mnc{G}Em\up{2}\mnc{A}(Am)]ternamente
  \mnendchorus\glueverses
  \beginverse
    \ind | | \e
  \endverse
  \beginchorus
    Te agra|\[Am]dezco |\[Fmaj7&5]{ por entregar} tu |\[G]amistad | \e
    |\[F]aa-ee-ee-ee-|\[Am]{eh-eh-eh-eeh} | \e
    Agrade|ciendo por |\[Fmaj7&5]{esta hermosa} |\[G]claridad | \e
    Por la con|\[Em]fianza y en la |\[Am]luz | \e\sublyr{\up{2}(hey hey hey)}
  \endchorus
  \goto{Pinta pinta pinta}
  \mnbeginchorus
    \lrep |\[\mnc{A}Am]{ Guacamayo} |Gu'acama\[\mn{G}]yo
    |\[\mnc{C}C]heeheehee \[\mn{G}]poder|\[\mn{C}]oso Guacamayo |\[\mnc{A}Am]hee \[\mn{C}\mn{G}]hey|\[\mn{A}]hee \rrep \rep{4}
    \vspace{1.5em}
    |\[\mnc{G}Em]En tus ojos |\[\mnc{A}B7]brilla una estrella |\[\mnc{B}G]flecha de luz a|\[\mnc{C}Am]briendo la pinta
    |\[\mnc{G}Em]Tu coro|\[\mnc{A}B7]na de luz |\[\mnc{B}G]danzando |\[\mnc{C}Am]{en el} viento
    |\[\mnc{G}Em]Iluminando |\[\mnc{A}B7]todo el espacio |\[\mnc{B}G]{ ay} curando mi |\[\mnc{C}Am]corazón
    |\[\mnc{G}Em]Iluminando |\[\mnc{A}B7]todo el espacio |\[\mnc{B}G]{ ay} curan\[\mn{G}]do |\[\mnc{A}Am]{la família}
    | ya hee ya |\[\mnc{E}C]hee | ya hee ya |\[\mnc{E}Am]hee-\[\mn{C}\mn{B}]{e-e-}\[\mn{C}\mn{A}]{e-ee} | \e
  \mnendchorus
  \beginchorus
    |\[Am]{ Guacamayo} |Gu'acamayo
    |\[C]heeheehee poder|oso Guacamayo |\[Am]hee hey|hee
    \rep{4}
  \endchorus
  \beginverse
    \lrep |\[\mnc{C}C]hee|y y\[\mn{A}]a |\[Am]hee | \e \rrep \rep{6}
    |\sublyr{hee-}\[C]{ Guacamayo} |\sublyr{y} vuela con\sublyr{ya}migo |\sublyr{hee}\[Am]{ Guacamayo} | \e
    |\sublyr{hee-}\[C]{ Guacamahito} pode|\sublyr{y}roso guaca\sublyr{ya}mayo |\sublyr{hee}\[Am]{ guacamahito} | \e
    \lrep |\sublyr{hee-}\[C]{ Guacamayo} |\sublyr{y}\hspace{1.5em}\sublyr{ya}\hspace{.5em} |\sublyr{hee}\[Am]{ Guacamayo} | \e \rrep
    \lrep |\sublyr{hee-}\[C]{ Guacamayo} |\sublyr{y}\hspace{1.5em}\sublyr{ya}\hspace{.5em} |\sublyr{hee}\[Am] | \e \rrep
  \endverse
  \begin{translation}
    I thank you for reaching my heart
    Beautiful creature of the wind
    I thank you for flying inside me
    Your colors take me inside
    \nextverse
    Paint paint paint with the movement
    of your feathers a vibration comes out
    that embraces me eternally
    that teaches me eternally
    \nextverse
    I thank you for giving your friendship
    Thanking for this beautiful clarity
    By trust and in the light
    \nextverse
    Macaw, powerful Macaw
    \nextverse
    In your eyes shines a star arrow of light opening the vision
    Your crown of light dancing in the wind
    Lighting up the whole space healing my heart
    Lighting up the whole space healing the family
    \nextverse
    Macaw, powerful Macaw
    \nextverse
    Macaw, fly with me macaw
    Little Macaw, powerful little Macaw
    Macaw\ldots
  \end{translation}
\endsong


\beginsong{Colibrí Dorado}[by={Arnaldo Herrera},ph={III}]
  \audio[key=Em]{https://soundcloud.com/arnaldo-herrera-3/colibri-dorado-andressa-arnaldo}
  \audio[key=Cm]{https://www.youtube.com/watch?v=JX44ZNy7ubs}
  \audio[key=Am]{https://www.youtube.com/watch?v=EaksDP2jCNg}
  \beginchorus
    \[\mn{A}]A nuestro \[\mn{B}]ho|\[\mnc{C}Am]gar \[\mn{B}]llegó \[\mn{A}]vol|an\[\mn{E}]do
    Un |\[G]Colibrí dor|ado
    que |\[Fmaj7]trajo a nuestro al|\[G]tar
    un nuevo a|\[Am]manecer | \e
  \endchorus
  \beginverse
    \[\mn{E}]La luz se a|\[\mnc{D}Em]cerca ref\[\mn{C}]le|\[\mnc{D}G]jando \[\mn{E}]la a\[\mn{D}]le|\[\mnciii{A}{C}{A}Am]gría | \e
    El viento s|\[Em]opla desper|\[G]tando un nuevo |\[Am]día | \e
    Y la fres|\[Em]cura en la mon|\[G]taña va lle|\[Am]vando | \e
    La escencia |\[G]de un Pica|\[Fmaj7]flor | \e
    que a|sciende enamo|\[Am]rado | \e
  \endverse
  \begin{translation}
    He came flying to our home
    A golden hummingbird
    who brought to our altar
    A new sunrise
    \nextverse
    The light approaches reflecting the joy
    The wind blows waking up a new day
    And the freshness in the mountain is taking
    The essence of a Hummingbird
    that rises in love
  \end{translation}
\endsong


\beginsong{Espíritu Eterno}[index={Cayaríriŕi},by={Darío Poletti},ph={III, IV}]
  \audio[key=Gm]{https://www.youtube.com/watch?v=tIZ0Gp_ChGw}
  \audio[key=Am]{https://www.youtube.com/watch?v=4IJxTvTm6hI}
  \capo{3}
  \mnbeginverse
    |\[\mnc{B}Em]Cuando la luna se a|\[^\mn{E}]soma ilumi|\[\mnc{D}G]nando la noche \[^\mn{B}]ni|\[^\mn{D}]may
    |\[\mnc{A}Bm]{Y las} estrellas se |\[^\mn{B}]ven en el firma|\[Em]men\[^\mn{E}]to | \e
    |\[^\mn{B}]Azul profundo que |\[^\mn{E}]nos envuelve en el |\[\mnc{F#}G]gran \[^\mn{D}]misterio \[^\mn{B}]ni|\[^\mn{D}]may
    |\[\mnc{A}Bm]{El misterio} de la |\[^\mn{B}]noche bien a|\[Em]dentro | \e
  \mnendverse
  \mnbeginchorus
    \[\mn{E}]Ca\[\mn{F#}]ya|\[\mnc{G}C]rí \[\mn{F#}]ri\[\mn{G}]rí \[\mn{F#}]ri|\[\mn{G}]rí \[\mn{F#}]ri\[\mn{G}]rí \[\mn{A}]rí|\[\mnc{G}G]rí \[\mn{D}]rirí \[\mn{B}]riri|\[\mn{D}]rí
    |\[\mnc{A}Bm]Es la presencia de |\[\mn{B}]tu espíritu e|\[Em]ter\[\mn{E}]no | \e
  \mnendchorus
  \notesoff
  \beginverse
    |^Son las raíces pro|fundas las que nos |^dan su secretos ni|may
    |^Medicina de la |tierra bien a|^dentro | \e
    |Abriendo nuestras con|sciencias abriendo |^corazones ni|may
    |^Medicina de los |Andes bien a|^dentro | \e
  \endverse
  \goto{Cayarí}
  \beginverse
    |^Padres, hijos, her|manos y madres |^que nos miran cre|cer
    |^Estarán dando su |luz a nuestra |^senda | \e
    |No te lo olvides her|mano lo que entre|^gas con el cora|zón
    |^Nos volverá con a|mor, será nuestra e|^sencia | \e
  \endverse
  \goto{Cayarí}
  \begin{translation}
    When the moon rises lighting up the night \emph{nimay}
    And the stars are seen in the sky
    Deep blue that envelops us in the great mystery \emph{nimay}
    The mystery of the night deep inside
    \nextverse
    Cayarí rirí rirí rirí rirí rirí rirí
    It is the presence of your eternal spirit
    \nextverse
    It is the deep roots that give us their secrets \emph{nimay}
    Medicine from the land deep inside
    Opening our consciences opening hearts \emph{nimay}
    Medicine from the Andes deep inside
    \nextverse
    Parents, children, brothers and mothers who watch us grow
    They will be giving their light to our path
    Do not forget brother what you give with your heart
    It will return to us with love, it will be our essence
  \end{translation}
\endsong


\beginsong{Como Te Quiero Yo}[by={Irina Florez},ph={III, IV},tags={love}]
  \audio[key=C\#m]{https://irinaflorez.bandcamp.com/track/como-te-quiero-yo}
  \audio[key=C\#m]{https://soundcloud.com/dejahvuuu/como-te-quiero-yo-irina-florez}
  \audio[key=C\#m]{https://www.youtube.com/watch?v=p5kGatl8fmA}
  \newchords{chords_comotequieroyo_a}\newchords{chords_comotequieroyo_b}
  \beginverse\memorize[chords_comotequieroyo_a]
    \ind |\[\mnc{A}Am]Yana he ya\[^\mn{E}]na |\[C]yo, yana he |\[E7]ya\[^\mn{D}]na yo \[^\mn{C}]hi\[^\mn{B}]a|\[\mnc{A}Am]na
    \ind Yana he |yana \[C]yo, |yana he \[E7]yana |yo hia\[Am]na
    \ind |Yana he yana |\[E7]yo hiana, |yana he yana |\[Am]yo hiana | | \e
  \endverse
  \notesoff
  \beginverse\replay[chords_comotequieroyo_a]
    \ind |^Como te quiero |^yo, como te |^quiero yo-o-o-|^oo
    \ind Como te |quiero ^yo, |como te ^quiero |yo-o-o-^oo
  \endverse
  \noteson
  \beginverse\memorize[chords_comotequieroyo_b]
    |\[\mnc{E}Am]Como el agua que |\[C]fluye libre,
    |\[E7]agua clarita |\[Am]del manantial.
    |\[F]Como el viento que |\[C]canta y sopla
    |\[E7]en lo profundo |\[Am]del corazón.
  \endverse
  \notesoff
  \beginverse\replay[chords_comotequieroyo_a]
    \ind |^Como te quiero |^yo, como te |^quiero yo-o-o-|^oo
    \ind Como te |quiero ^yo, |como te ^quiero |yo-o-o-^oo
  \endverse
  \beginverse\replay[chords_comotequieroyo_b]
    |^Como el fuego que |^brilla claro,
    |^la llama antigua |^de la creación.
    |^Como la tierra que |^nace en flores,
    |^que llama el \up{1}canto |^del colibrí. \altlyr[2]{vuelo}
  \endverse
  \beginverse\replay[chords_comotequieroyo_a]
    \textnote{Instrumental, on 2nd playthrough:}
    \ind[3] |^{ }{ }{ }{ }{ } |^{ }{ }{ }{ }{ } |^{ }{ }{ }{ }{ } |^{ }{ }{ }{ }{ }
    \ind[3] |{ }{ }{ }{ }^{ }{ }{ }{ } | { }{ }{ }{ }^{ }{ }{ }{ } |{ }{ }{ }{ }^{ }{ }{ }{ }
    \ind[3] |{ }{ }{ }{ }{ } |^{ }{ }{ }{ }{ } |{ }{ }{ }{ }{ } |^{ }{ }{ }{ }{ } | \e
  \endverse
  \beginverse\replay[chords_comotequieroyo_a]
    \ind |^Como te quiero |^yo, como te |^quiero yo-o-o-|^oo
    \ind Como te |quiero ^yo, |así te ^quiero |yo-o-o-^oo
  \endverse
  \beginverse\replay[chords_comotequieroyo_b]
    |^Como es el paja|^rito le canta al
    |^cielo cuando nace el |^sol.
    ^Como es el |paja^rito le |canta al
    ^cielo cuan|do nace el ^sol. | \e
  \endverse
  \textnote{\emph{D.C. al Fine}}
  \goto{Yana he yana yo}
  \begin{translation}
    Yana he yana yo\ldots
    \nextverse
    How I love you\ldots
    \nextverse
    Like the water that flows free,
    clear spring water.
    Like the wind that sings and blows
    deep in the heart.
    \nextverse
    Like the fire that shines clear,
    the ancient flame of creation.
    Like the earth that grows flowers,
    which calls the song (flight) of the colibri.
    \nextverse
    Like the bird that sings
    to the sky when the sun rises.
  \end{translation}
\endsong


\beginsong{Todo es mi Familia}[tags={unity},ph={IV}]
  \beginchorus
    \[\mn{E}]Ca|\[\mnc{A}A]minaré en bell\[\mn{G}\mn{E}]eza, ca|\[\mn{A}]minaré \[\mn{G}]en \[\mn{A}]paz
  \endchorus\glueverses\beginchorus
    |\[G]Todo es mi fa|\[A]milia
  \endchorus
  \beginverse
    |\[A]Todo es sagrado, las |plantas y animales
    |Todo es sagrado, las mon|tañas, selva y el mar
  \endverse\glueverses\beginchorus
    |\[G]Todo es mi fa|\[A]milia
  \endchorus
  \beginchorus
    |\[A]Hey ah hey ah hey ah, |hey ah hey ah hey
  \endchorus\glueverses\beginchorus
    |\[G]Todo es mi fa|\[A]milia
  \endchorus
  \textnote{in English:}
  \beginchorus
    I |\[A]will walk in beauty, I |will walk in peace
  \endchorus\glueverses\beginchorus
    |\[G]Everything is my |\[A]family
  \endchorus
  \beginverse
    |\[A]Everything is sacred, |animals and plants
    |\[A]Everything is sacred, |mountains, forest and sea
  \endverse\glueverses\beginchorus
    |\[G]Everything is my |\[A]family
  \endchorus
  \beginchorus
    |\[A]Hey ah hey ah hey ah, |hey ah hey ah hey
  \endchorus\glueverses\beginchorus
    |\[G]Everything is my |\[A]family
  \endchorus
  \musicnote{outro, fade out:}
  \beginchorus
    |\[G]Todo somos fa|\[A]milia
  \endchorus
\endsong


\beginsong{Niño Salvaje}[index={Soy un Hijo de la Tierra},by={Alberto Kuselman},tags={unity},ph={IV}]
  \audio[]{https://www.youtube.com/watch?v=02DM-kESGzY}
  \audio[]{https://www.youtube.com/watch?v=OTJd02cGq10}
  \beginchorus\memorize
    |\[\mnc{E}C]El silencio es mi pa|\[\mnciii{F}{E}{D}Fmaj7]labra, la |\[\mnc{G}G]tier\[^\mn{D}]ra es mi |\[\mnciii{E}{D}{C}C]madre
    |los árboles mis her|\[Fmaj7]manos, las es|\[G]trellas mi desti|\[C]no
  \endchorus
  \notesoff
  \beginchorus
    |^Soy \up{*}un hijo de la |^tierra, mi |^corazón es una es|^trella
    |viajo a bordo del e|^spíritu ca|^mino a la eterni|^dad
  \endchorus
  \notesoff
  \beginchorus
    |^Soy \up{*}un niño sal|^vaje, ino|^cente, libre y sil|^vestre
    |tengo todas las e|^dades mis an|^cestros viven en |^mí
  \endchorus
  \beginchorus
    |^Soy \up{*}hermano de las |^nubes y |^sólo sé compar|^tir
    |sé que todo es de |^todos y que |^todo está vivo en |^mí
  \endchorus
  \vspace{1em}\altlyr{una hija, una niña, hermana}
  \begin{translation}
    Silence is my word, the earth is my mother,
    the trees my brothers, the stars my destiny.
    \nextverse
    I am a child of the earth, my heart is a star,
    I travel aboard the spirit on the way to eternity.
    \nextverse
    I am a wild, innocent, free and wild child,
    I have all ages, my ancestors live in me.
    \nextverse
    I'm a brother of the clouds and will only share,
    I know that everything belongs to everyone and that
    \ind everything is alive in me.
  \end{translation}
\endsong


\scleardpage % Needed here
\beginsong{Madre Selva}[by={Grupo Putumayo},tags={Mother Earth},ph={IV}]
  \newchords{chords_madreselva_a}\newchords{chords_madreselva_b}
  \beginchorus\memorize[chords_madreselva_a]
    \[^\mn{A}]Mad\[^\mn{B}]re |\[\bmc\mnc{C}Am]Tierra Madre \[\bm]Tierra yo te a|\[\bmc F]labo Madre \[\bm]Tierra
    Porque |\[\bmc G]eres el o\[\bmc Em]rigen \[\bmc G]de la |\[\bmc Am]vida \[\bm]
  \endchorus\glueverses\beginchorus\memorize[chords_madreselva_b]
    En tus |\[\bmc G]selvas en tus \[\bmc Em]selvas tus mon|\[\bmc Am]tañas \[\bm]
    Donde ha|\[\bmc G]bita el e\[\bmc Em]spíritu di|\[\bmc Am]vino \[\bm]
  \endchorus
  \notesoff
  \beginchorus\replay[chords_madreselva_a]
    Madre |^Selva Madre ^Selva yo te a|^labo Madre ^Selva
    Porque |^eres el o^rigen ^de la |^vida ^
  \endchorus\glueverses\beginverse\replay[chords_madreselva_b]
    En tus |^selvas en tus ^selvas tus mon|^tañas ^
    Donde |^nace el aire ^puro que re|^spiro ^\replay
    En tus |^selvas en tus ^selvas y tus |^ríos ^
    Donde |^nace el agua ^pura que be|^bemos ^
  \endverse
  \beginchorus\replay[chords_madreselva_a]
    Danos |^fuerza, danos ^fuerza Madre |^Selva, danos ^fuerza
    Ilumi|^na ilumi^na nues^tro ca|^mino ^
  \endchorus\glueverses\beginchorus\replay[chords_madreselva_b]
    Ilu|^mina ilumi^na nuestro ca|^mino, ^
    con tu e|^spíritu tu e^spíritu di|^vino ^
  \endchorus
  \beginchorus\replay[chords_madreselva_a]
    Yo te a|^labo yo te a^labo Madre |^Selva yo te a^labo
    Porque |^eres el o^rigen ^de la |^vida ^
  \endchorus\glueverses\beginchorus\replay[chords_madreselva_b]
    En tus |^selvas en tus ^selvas tus mon|^tañas ^
    Donde |^nace, crece y ^crece el yage|^cito ^
  \endchorus
  \beginchorus\replay[chords_madreselva_a]
    Mama |^Cocha con tus ^aguas crista|^linas
    Limpia, ^purifica |^cuerpo, el e^spíri^tu del |^hombre ^
  \endchorus\glueverses\beginchorus\replay[chords_madreselva_b]
    Al que |^pide sana^ción, paz y harmo|^nía ^
    Al que a|^bre su cora^zón a la ale|^gría. ^
  \endchorus
  \beginchorus\replay[chords_madreselva_a]
    Yo te a|^labo yo te a^labo Madre |^Selva yo te a^labo
    Porque |^eres el o^rigen ^de la |^vida ^
  \endchorus\glueverses\beginchorus\replay[chords_madreselva_b]
    En tus |^selvas en tus ^selvas tus mon|^tañas ^
    Donde |^nace, crece y ^crece el yage|^cito ^
    \rep{4}
  \endchorus
  \begin{translation}
    Mother Earth, Mother Earth, I praise you Mother Earth,
    because you are the origin of life.
    In your forests, in your forests, in your mountains,
    where the divine spirit dwells.
    \nextverse
    Mother Forest, Mother Forest, I praise you Mother Forest,
    because you are the origin of life.
    In your forests, in your forests, in your mountains,
    where the pure air that I breathe is born.
    In your forests, in your forests and in your rivers,
    where the pure water that we drink is born.
    \nextverse
    Give us strength, give us strength, Mother Forest, give us strength.
    Illuminate, illuminate our path.
    Illuminate, illuminate our path,
    with your spirit, your divine spirit.
    \nextverse
    I praise you, I praise you, Mother Forest I praise you,
    because you are the origin of life.
    In your forests, in your forests, in your mountains,
    where the \emph{yagecito} is born, growing and growing.
    \nextverse
    \emph{Mama Cocha}, with your crystalline waters:
    cleanse, purify the body, the spirit of (hu)man,
    of the one who asks for healing, peace and harmony,
    of the one who opens their heart to joy.
    \nextverse
    I praise you, I praise you, Mother Forest I praise you,
    because you are the origin of life.
    In your forests, in your forests, in your mountains,
    where the yagecito is born, growing and growing.
  \end{translation}
  \begin{explanation}
    \begin{description}
      \item[Mama Cocha:] an ancient Incan goddess of sea (also other bodies of water) and fishes,
        the guardian of fishermen and sailors
      \item[Yagé, yagecito:] another name for \emph{Ayahuasca}, used especially in areas around
        and within Colombia and Ecuador
    \end{description}
  \end{explanation}
\endsong


\beginsong{Cuñaq}[by={Shimshai},tags={water},ph={IV}]
  \beginchorus\memorize % memorize chords even though in "chorus"
    |\[\mnc{A}Am]Des\[^\mn{E}]de Cuñaq viene | a|güita serpen\[C]teando
    |Por las a\[Em]cequias |y en remo\[Am]linos
    |\[C]Hacia nuestras \[Am]vidas | \e
  \endchorus
  \notesoff
  \beginchorus
    \ind |\[Am]De cantar hua\[C]linas |y a la vez llo\[Em]rando
    \ind |Toditas mis \[G]pe|nas se aca\[Am]baron
    \ind | Pacha\[C]mama está de |\[Am]fiesta | \e
  \endchorus
  \beginchorus
    |^Una estrellita | que a|legre me de^cía
    |Canta cantor^cito |a la a^güita
    A la a|^güita madre ^Cuñaq | \e
  \endchorus
  \goto{De cantar}
  \begin{translation}
    From Cuñaq comes water (drink), meandering
    by the ditches and in eddies
    into our lives
    \nextverse
    Singing \emph{hualinas} while weeping
    All my sorrows are gone
    Pachamama is celebrating
    \nextverse
    A little star told me:
    sing to the little water,
    to water mother Cuñaq
  \end{translation}
  \begin{explanation}
    \textbf{Cuñaq cocha} is a lake high up in the mountains in Peru,
    of which it is said that water is born there. Each year there are
    celebrations honoring the water descending from that lake.
  \end{explanation}
\endsong


\beginsong{Luna Manta \\ Pachamama}[by={Andrés Córdoba},tags={Mother Earth},ph={IV}]
  \beginverse
    |\[\mnc{E}Em]Luna \[^\mn{G}]manta que a|\[\mnc{F#}Bm]lumbra, |\[Em]luna manta que alum|\[Bm]bró
    a |\[G]mi tierrita |\[D]bella, |\[C]a mi tierra \[B7]echo una |\[Em]flor | \e
    |\[Em]Luna manta que a|\[Bm]lumbra, |\[Em]Taita Inti dio ca|\[Bm]lor
    a |\[G]mi tierrita |\[D]bella, |\[C]a mi tierra \[B7]echo una |\[Em]flor | \e
  \endverse
  \notesoff
  \beginchorus
    \ind Que |\[D]lindo es ver mi |\[Em]tierra, que |\[D]bello es ver mi |\[Em]tierra
    \ind Que |\[D]lindo es ver mi |\[C]tierra, mi Pacha que|\[B7]rida, mi Pachama|\[Em]ma | \e
  \endchorus
  \beginverse
    Su |^siete colores que a|^lumbran, |^siete colores que alum|^bró
    na|^ció de una que|^brada encan|^tada ^en a|^mor | \e
    Su |^siete colores que a|^lumbran, un |^arco iris alum|^bró
    na|^ció en una la|^guna bende|^cida ^por el |^sol | \e
  \endverse
  \vspace{1em}\goto{Que lindo}
  \textnote{\emph{D.C. al Fine}}
  \beginverse
    Que |\[D]lindo es ver mi |\[Em]tierra, los |\[D]apus de las cordi|\[Em]lleras
    |\[C]donde se junta fa|\[Em]milia, silvan melo|\[B7]días, mi Pachama|\[Em]ma | \e
    Que |\[D]lindo es ver mi |\[Em]tierra, los |\[D]apus de las cordi|\[Em]lleras
    |\[C]donde hay selvas encan|\[Em]tadas, y plantas sa|\[B7]gradas, mi Pachama|\[Em]ma | \e
  \endverse
  \begin{translation}
    Moon that illuminates, moon that illuminated
    To my beautiful earth, to my earth I throw a flower
    Moon that illuminates, Taita Inti gave warmth
    To my beautiful earth, to my earth I throw a flower
    \nextverse
    \ind How lovely it is to see my earth, how beautiful it is to see my earth
    \ind How lovely it is to see my earth, my beloved Pacha, my Pachamama
    \nextverse
    Its seven colors that illuminate, seven colors that illuminated
    Born of an enchanted gully in love
    Its seven colors that illuminate, a rainbow illuminated
    Was born in a sun-blessed lagoon
    \nextverse
    How lovely to see my earth, the apus of the mountain ranges
    Where family reunites, poetic melodies, my Pachamama
    How lovely to see my earth, the apus of the mountain ranges
    Where there are enchanted forests, and sacred plants, my Pachamama
  \end{translation}
  \begin{explanation}
    \begin{description}
     \item[Apu] is a mountain spirit in the mythology of Peru, Ecuador and Bolivia; the term
       dates back to the Inca empire.
    \end{description}
  \end{explanation}
\endsong


\beginsong{Sol de la Mañana}[by={Bóveda Celeste},ph={IV}]
  \audio[key={Bm}]{https://www.youtube.com/watch?v=gTGygs833d0}
  % \capo{2}
  \beginchorus
    |\[\mnc{E}Am\mn{C}]Bri|lla, |\hspace{0.5em} \[\mn{E}]bril\[\mn{c}]la y |\[\mn{D}]can\[\mn{E}]ta un \[\mn{C}]nue\[\mn{B}]vo |\[\mn{A}]día | \e
    |\hspace{0.5em} Brilla el |sol de la ma|ñana | \e
    |\hspace{0.5em} Movi|miento de ésta |\[Dm]tie|'rra,
    |\hspace{0.5em} palpi|tando en la flo|\[Am]re|'sta
    |\hspace{0.5em} Donde un lla|mado nos re|úne | \e
    |En el cora|zón de una fa|milia | | | \e
    Her|mano, |Huitzillín y que |\[C]sue|ñas
    En |\[Dm]la nueva au|rora y la armo|\[Am]nía | \e
    Sim|pleza y |sutileza ins|\[C]pi|ra
    El |\[Dm]flujo de una |fuerza de ale|\[Am]grí|a
    Vo|lando |a los cuatro |\[C]vien|tos
    Plu|\[Dm]majes que se |sienten voy re|\[Am]zando | \e
    Y el |sona|jero agrade|\[C]ci|do
    Hoy |\[Dm]siempre está ofren|dita la fa|\[Am]mili|a
    |Canta cantor|cito mari|ri-ri-ri-ri-|ri-ri-ri-ri-|riii | | | \e
    |Canta cantor|cito mari|ri-ri-ri-ri-|ri-ri-ri-ri-|riii | | | \e
  \endchorus
  \beginchorus
    \[\mn{E}]Can|\[\mnc{A}Am]tar, |\hspace{0.5em} \[\mn{E}]can|\[\mn{A}]tar | \e
    \lrep |\[C]Vas vibrando |alto
    |\[Dm]Vas tocando en |\[E]la profundi|\[Am]dad del cora|zón
    Vas te|jiendo cantor|cito \rrep
    |\[C]Vas vibrando |alto
    |\[Dm]Vas tocando en |\[E]la profundi|\[F]dad del cora|\[E]zón
    Donde la |\[Am]sangre hierve |\[C]con amor
    |\[F]No hay más pensa|\[E]mientos
    Sólo el |\[Am]flujo de la e|\[C]volución
    |\[F]Deja las ra|\[E]zones, y em|\[Am]pápate en acc|\[C]iones
    Que el |tiempo que |pasa es u|\[G]na i\[E]lu|\[Am]sión | | | \e
    |Canta cantor|cito mari|ri-ri-ri-ri-|ri-ri-ri-ri-|riii | | | \e
    |Canta cantor|cito mari|ri-ri-ri-ri-|ri-ri-ri-ri-|riii | | | \e
  \endchorus
  \begin{translation}
    Shine, shine and sing a new day
    The morning sun shines
    Movement of this earth,
    throbbing in the forest
    Where a call brings us together
    In the heart of a family
    Brother, \emph{Huitzillín} and what you dream
    In the new dawn and harmony
    Simplicity and subtlety inspires
    The flow of a force of joy
    Flying to the four winds
    Plumages that feel I'm praying
    And the grateful rattle
    Today the family is always offered
    Sing, singer, mari-ri-ri-ri-ri-ri-ri-ri-ri-riiii
    Sing, singer, mari-ri-ri-ri-ri-ri-ri-ri-ri-riiii
    \nextverse
    To sing, to sing
    \lrep You are vibrating high
    You are playing in the depth of the heart
    You are weaving a little song \rrep
    You are vibrating high
    You are playing in the depth of the heart
    Where the blood boils with love
    There are no more thoughts
    Just the flow of evolution
    Leave the reasons, and immerse yourself in actions
    That the time that passes is an illusion
    Sing, singer, mari-ri-ri-ri-ri-ri-ri-ri-ri-riiii
    Sing, singer, mari-ri-ri-ri-ri-ri-ri-ri-ri-riiii
  \end{translation}
  \begin{explanation}
    \begin{description}
      \item[Huitzillín:] Nahuatl word for ``hummingbird''
      \item[Marirí:] see song \emph{Marirí}
    \end{description}
  \end{explanation}
\endsong


\begin{intersong}
  \begin{feeler}
    ``Animals are something invented by plants to move seeds around. An extremely yang solution to a peculiar problem which they faced.''\\
    --- \emph{Terence McKenna} (1946--2000)
  \end{feeler}
  \vfill
\end{intersong}


\beginsong{Plantas Sagradas}[by={Danit Treubig},ph={IV}]
  \audio[]{https://www.youtube.com/watch?v=tCiDdrh5oks}
  \beginverse* % Show the beginning chant small and without chords, to fit on page
    \tiny\chordsoff
    |\[Dm]Hey hey hey ha |hey hey hey ha, |\[Am]hey ram hey ram |hey ram hey cu|\[Dm]ra y plumajero croma|jerui pinta a|\[Am]buelo | chai
    |\[Dm]bí bibibí bibi|bí ja cha rai |\[Am]pinta y cura y |cura |\[Dm]medicinai |taita y curan|\[Am]dera | \e
  \endverse
  \beginchorus\memorize
    \[^\mn{F}]Cami|\[\mnc{D}Dm]nando \[^\mn{F}]por la |\[^\mn{D}]selva si|\[\mnc{C}Am]gui\[^\mn{A}]endo \[^\mn{C}]la be|\[^\mn{A}]lleza,
    si|\[Dm]guiendo las can|ciones de las |\[Am]plantas. | \e
  \endchorus
  \notesoff
  \beginchorus
    \ind |\[F] Siento |vivo, |\[Am]siento la |paz,
    \ind |\[F]siento el po|der de la |\[Am]mad|re.
  \endchorus
  \beginchorus
    \ind\ind |\[Em]Siento las |\[G]plantas sa|\[Am]gra|das.
  \endchorus
  \beginchorus
    Ya|^gé yagé ya|gé, ya|^gé yagé ya|gé,
    ya|^gé yagé ya|gé, ya|^gé. | \e
  \endchorus
  \beginchorus
    |^Siempre desper|tando, |^siempre agrade|ciendo
    |^por este po|der, esta be|^lleza. | \e
  \endchorus
  \goto{Siento vivo}
  \goto{Siento las plantas}
  \goto{Yagé, yagé \rep{4}}
  \goto{Siento las plantas \rep{5}}
  \begin{translation}
    Walking through the jungle following the beauty,
    following the songs of the plants.
    \nextverse
    I feel alive, I feel peace, I feel the power of the mother.
    \nextverse
    I feel the sacred plants.
    %\nextverse % Skip this to fit the song on one page
    %Yagé yagé yagé, yagé yagé yagé, yagé yagé yagé, yagé.
    \nextverse
    Always waking up, always grateful for this power, this beauty.
  \end{translation}
\endsong


\beginsong{Buscando el Camino}[by={Karin Micha},tags={path},ph={IV}]
  \transpose{5} % To Am, where melody lies between low G and high B
  \beginverse
    \[^\mn{E}]Bus|\[C]can\[^\mn{F#}]do \[^\mn{E}]el \[^\mn{D}]ca|\[\mnc{B}G]mino que |\[Am]lleva a la e|\[Em]sencia
    Es|\[C]cucho el lla|\[G]mado |\[Am]de la Madre |\[Em]Tierra
    Voy recor|\[C]riendo los |\[G]valles voy sal|\[Am]tando las mon|\[Em]tañas
    Voy vo|\[C]lando por el |\[G]cielo voy dan|\[Am]zando con el |\[Em]agua! | \e
  \endverse
  \notesoff
  \beginverse
    Y |^sigo este ca|^mino de i|^magia y mis|^terio
    |^Voy sin equi|^paje llevo |^solo mis res|^petos
    Voy se|^guro voy sin |^miedo Gran E|^spíritu me |^guia
    Tran|^quilo y sin a|^puros hoy la |^Tierra me da el |^pulso! | \e
  \endverse
  \beginverse
    \ind \lrep |^Laa la lai la |^laa laa |^laa la lai lai |^laa laa\ldots \rrep\rep{4} | \e
    % % You can figure it out!
    % \ind |^Laa la lai la |^laa laa |^laa la lai lai |^laa
    % \ind la la la la la |^lai la lai la |^laa la lai |^lai la la lai lai |^laa laa
    % \ind la |^lai la lai la |^laa la lai |^lai la la lai lai |^laa! | \e
  \endverse
  \beginverse
    |^Voy recor|^dando |^a mis abue|^litas
    |^Voy agrade|^ciendo |^a mis abue|^litos
    Nos mos|^traron el ca|^mino nos de|^jaron sus te|^soros
    Sa|^gradas ceremo|^nias pode|^rosas medi|^cinas! | \e
  \endverse
  \beginverse
    Y |^asi voy apren|^diendo y compar|^tiendo este men|^saje
    Somos |^hijos de la |^Tierra y de|^bemos prote|^ger la
    Traba|^jando con mis her|^manos somos |^guardianes y guer|^reros
    Cu|^ramos con nuestros |^cantos y los |^rezos a la |^Tierra! | \e
  \endverse
  \goto{Laa la lai la}
  \begin{translation}
    I look for the path that leads to the essence.
    I hear the call of Mother Earth.
    I go traversing the valleys, I go jumping the mountains.
    I go flying through the sky, I go dancing with the water!
    \nextverse
    And I follow this path of imagination and mystery.
    I go without luggage, I only take my respects.
    I go safe without fear, Great Spirit guides me.
    Calm and without hurry, today the Earth gives me the pulse!
    \nextverse
    I'm remembering my grandmothers.
    I'm thankful to my grandparents.
    They showed us the way, they left us their treasures:
    sacred ceremonies, powerful medicines!
    \nextverse
    And so I'm learning and sharing this message.
    We are children of Earth and we must protect her.
    Working with my brothers, we are guardians and warriors.
    We heal with our songs and the prayers to the Earth!
  \end{translation}
\endsong


\beginsong{El Duende}[by={Camino Rojo}, ph={IV}]
  \meter{6}{8}
  \beginchorus\memorize
    \[^\mn{E}]El |\[\mnc{A}Am]duende \[^\mn{B}]de la |\[^\mn{C}]aire \[^\mn{A}]me |\[\mnc{B}E]vino a \[^\mn{G#}]de|\[^\mn{E}]cir
    |que si no can|\[Dm]taba yo me |\[C]iba mo|\[Am]rir
  \endchorus\glueverses\beginchorus
    \[\mn{E}]Can|^tar, \[\mn{B}]can|\[\mn{C}]tar \[\mn{A}]pa|^ra vi|vir
    Can|tar, can|^tar pa|^ra vi|^vir
  \endchorus
  \notesoff
  %\textnote{Instrumental}
  \beginchorus
    El |^duende del |agua me |^vino a de|cir
    |que si no flu|^ya yo me |^iba mo|^rir
  \endchorus\glueverses\beginchorus
    Flu|^ir, flu|ir pa|^ra vi|vir
    Flu|ir, flu|^ir pa|^ra vi|^vir
  \endchorus
  %\textnote{Instrumental}
  \beginchorus
    El |^duende del |fuego me |^vino a de|cir
    |que si no dan|^zaba yo me |^iba mo|^rir
  \endchorus\glueverses\beginchorus
    Dan|^zar, dan|zar pa|^ra vi|vir
    Dan|zar, dan|^zar pa|^ra vi|^vir
  \endchorus
  \textnote{Instrumental}
  \beginchorus
    El |^duende de la |tierra me |^vino a de|cir
    |que si no sem|^braba yo me |^iba mo|^rir
  \endchorus\glueverses\beginchorus
    Sem|^brar, sem|brar pa|^ra vi|vir
    Sem|brar, sem|^brar pa|^ra vi|^vir
  \endchorus
  %\textnote{Instrumental}
  \beginchorus
    El |^duende del |bosque me |^vino a de|cir
    |que si no a|^maba yo me |^iba mo|^rir
  \endchorus\glueverses\beginchorus
    A|^mar, a|mar, pa|^ra vi|vir
    A|mar, a|^mar, pa|^ra vi|^vir
  \endchorus
  \textnote{Instrumental}
  \beginchorus
    Da |^rai rai da |rai rai da |^rai rai da |rai
    Da |rai rai da |^rai rai da |^rai ray da |^rai
  \endchorus
  \begin{translation}
    The spirit of the air came to tell me
    that if I didn't sing I would die
    \nextverse
    Sing, sing to live
    \nextverse
    The spirit of the water came to tell me
    that if I didn't flow I would die
    \nextverse
    Flow, flow to live
    \nextverse
    The spirit of the fire came to tell me
    that if I didn't dance I would die
    \nextverse
    Dance, dance to live
    \nextverse
    The spirit of the earth came to tell me
    that if I didn't sow I would die
    \nextverse
    Sow, sow to live
    \nextverse
    The spirit of the forest came to tell me
    that if I didn't love I would die
    \nextverse
    Loving, loving, to live
  \end{translation}
\endsong


\beginsong{Camino Rojo}[tags={path},ph={IV}]
  \newchords{chords_caminorojo_a}\newchords{chords_caminorojo_b}
  \beginchorus\memorize[chords_caminorojo_a]
    \[^\mn{E}]Yo |\[\mnc{A}Am]sigo un camino de l|uz
    Yo |\[Dm]sigo un camino de a|\[Am]mor
  \endchorus\glueverses
  \beginchorus\memorize[chords_caminorojo_b]
    Camino |\[Dm]rojo camino |\[Am]rojo
    Camino |\[G]rojo de \[E]mi cora|\[Am]zón
  \endchorus
  \notesoff
  \beginchorus\replay[chords_caminorojo_a]
    Temaz|^calli, peyote, tip|i
    Temaz|^calli, peyote, tip|^i
  \endchorus\glueverses
  \beginchorus\replay[chords_caminorojo_b]
    Son los al|^tares, son los al|^tares
    Son los al|^tares que ^yo cono|^cí
  \endchorus
  \beginchorus\replay[chords_caminorojo_a]
    Aya|^huasca, San Pedro, niñ|os
    Taba|^quito, Maria, tam|^bor
  \endchorus\glueverses
  \beginchorus\replay[chords_caminorojo_b]
    Son sacra|^mentos, son sacra|^mentos
    Son sacra|^mentos para ^el cora|^zón
  \endchorus
  \begin{translation}
    I follow a path of light
    I follow a path of love
    Red path, red path
    Red path of my heart
    \nextverse
    Sweat lodge, Peyote, tipi
    Sweat lodge, Peyote, tipi
    They are the altars, they are the altars,
    They are the altars that I know
    \nextverse
    Ayahuasca, San Pedro, children \emph{(mushrooms)}
    Tobacco, Maria, the drum
    They are sacraments, they are sacraments
    They are sacraments for the heart
  \end{translation}
\endsong


\beginsong{Tlazocamati}[ex={español, nahuatl},tags={Aya},ph={IV}]
  \beginchorus\memorize % memorize chords even though in "chorus"
    \[^\mn{A}]A|\[\mnc{E}Am]ya Aya A|yahuas\[C]ca
    A|\[C]ya A\[Em]ya A|\[G]yahuas\[Am]ca \up{2}(| | \e)
  \endchorus
  \notesoff
  \beginchorus
   Ya|^ge Yage Ya|ge Yage ^he
   Ya|^ge Ya^ge |^Yage ^he \up{2}(| | \e)
  \endchorus
  \beginchorus
    |^Voy por el camino de |luz en la ^vida
    |^Sigo ade^lante |^con la medi^cina \up{2}(| | \e)
  \endchorus
  \beginchorus
    |^Tlazocam\[\bmadj{-.5ex}]ati to|pixin me^xica
    |^Toyo lo ^tatzin |^toyo lo ^nantzin \up{2}(| | \e)
  \endchorus
  \beginchorus
    |^We ya hey y\[\bmadj{-.5ex}]a yo |We ya hey ^yo
    |^We ya hey ^ya |^We ya hey ^yo \up{2}(| | \e)
  \endchorus
  \beginchorus
    A|^quí Aya A|yahuas^ca
    A|^quí A^ya y |^en el más al^lá \up{2}(| | \e)
  \endchorus
  \begin{explanation}
    \textbf{Tlazocamati} is a Nahuatl word for ``thank you''. Nahuatl (also known as Aztec)
    language is nowadays spoken in what is currently central Mexico.
  \end{explanation}
\endsong


\beginsong{Wirikuta}[tags={Mother Earth, Sun},ph={IV}]
  \beginverse
    |\[\mnc{A}Am]Pacha|\[\mn{C}]mama |\[G]pacha|\[Am]mama
    |\[Am]Pacha|mama |\[G]madre |\[Am]tierra
  \endverse
  \beginchorus\memorize % memorize chords here instead of the non-chorus verse above
    |\[Am]Wirikuta |\[G]Wiriku\[Am]ta |\[C]Wiriku\[Dm]ta |\[E7]Gran Espíri\[Am]tu
  \endchorus
  \vspace{1em}\goto{Pachamama}
  \beginchorus
    |^Taita Inti |^Taita In^ti |^Taita In^ti |^Gran Espíri^tu
  \endchorus
  \vspace{1em}\goto{Pachamama}
  \begin{explanation}
    \begin{description}
      \item[Wirikuta:] a site in central Mexico, sacred to the Wixarrica Huichol people,
          where the world was created; \textbf{Wiracocha} is a creator deity in the mythology
          of the Andes.
      \item[(Taita) Inti:] Incan Sun God
    \end{description}
  \end{explanation}
\endsong


\beginsong{Gracias}[tags={thankfulness, love},ph={III, V}]
  \meter{6}{8}
  \beginchorus
    |\[\mnc{E}Am]A\[\mn{C}]buelitas |\up{*}piedras las gracias te |\[E]doy
    Las gracias te |\[Am]doy \[\up{2}(A7)]
  \endchorus
  \altlyr{tierra, agua, fuego\ldots}
  \beginchorus
    |\[Dm]Por brimer cora|\[Am]zón a la sana|\[E]cion
    Ab\[E7]rirme al a|\[Am]mor \[\up{1}A7]
  \endchorus
  \begin{translation}
    Grandmother stones, I offer you thanks
    I offer you thanks
    \nextverse
    By opening the heart to healing
    Love opens to me
  \end{translation}
\endsong


\beginsong{Bienvenidos Hermanitos}[by={Daniel Osorio},tags={morning, Mother Earth, Sun},ph={IV, V}]
  \beginchorus\memorize
    \[^\mn{A}]Ya va |\[Am]florecien\[^\mn{C}]do \[^\mn{D}]el |\[\mnc{E}C]día,
    la ale|\[G]gría va desper|\[C]tando
    Pacha|\[F]mama ya nos le|\[C]vantas,
    Madre |\[E7]Tierra canto de a|\[Am]mor | \e
  \endchorus
  \notesoff
  \beginchorus
    Estamos |^vivos con ganas de a|^marnos,
    de la |^vida enamo|^rados
    Taita |^Inti que estás en los |^cielos
    ya ca|^lientas nuestro cora|^zón | \e
  \endchorus
  \beginchorus
    \ind |\[F]Lailaralai Lara|\[C]lai Laralai |\[G]Lai Larala|\[C]lai
    \ind Melo|\[F]día sin fin |\[C]esta vibran|\[E7]do en mi cora|\[Am]zón | \e
  \endchorus
  \beginchorus
    Paja|^rito, selva, mon|^taña,
    |^canto de fuerza y |^vida
    mara|^villas hoy nos es|^peran 
    bendi|^ciones del crea|^dor | \e
  \endchorus
  \beginchorus
    \ind |\[F]Bienve|nidos herma|\[C]nitos |\[F]bienve|nidos hoy
    \ind |\[C] \[F]Bienve|nida \[E]vida bu|\[Am]ena, |\[E7]bienvenida |\[Am]tu canción | \e
  \endchorus
  \vspace{1em}\goto{Lailaralai}
  \brk
  \imagecc[2]{sunrise_mountain_bw_transparent_bg_CC0_1280px.png}%
  \begin{translation}
    The day is already blossoming,
    joy is awakening.
    Pachamama lift us up.
    Mother Earth, I sing of love.
    \nextverse
    We are alive wanting to love each other,
    with the desire for life in love.
    Taita Inti (Sun God) in the skies
    already warms our hearts.
    \nextverse
    Lailaralai Laralai Laralai Lai Laralai.
    Endless melody is vibrating in my heart.
    \nextverse
    Little bird, forest, mountain,
    I sing of strength and life.
    Marveling this day, we wait for
    the blessings of the creator.
    \nextverse
    Welcome siblings, welcome this day.
    Welcome good life, welcome your song.
  \end{translation}
\endsong


%%%%%%%%%%%%%%%%%%%%%%%%%%%%%%%%%%%%%%%%%%%%%%%%%%%%%%%%%%%%%%%%%%%
%%% LATEST PRINTOUT CONTAINED THE SONGS ABOVE.                  %%%
%%%%%%%%%%%%%%%%%%%%%%%%%%%%%%%%%%%%%%%%%%%%%%%%%%%%%%%%%%%%%%%%%%%
%%% Please try to not change the song numbers above this point. %%%
%%% Add new songs only after this point.                        %%%
%%%%%%%%%%%%%%%%%%%%%%%%%%%%%%%%%%%%%%%%%%%%%%%%%%%%%%%%%%%%%%%%%%%


% % TODO: this is not done, needs work!
% \beginsong{Arbolito Divino}[by={Nick Barbachano},ph={II, III}]
%   \audio[key=Am]{https://soundcloud.com/nickbarbachano/arbolito-divino-feat-danit}
%   \audio[key=Am]{https://www.youtube.com/watch?v=nmboXwt3Fc8}
%   \beginverse
%     |\[\mnc{A}Am]Arboli\[^\mn{G}]to \[^\mn{A}]di|\[\mnc{D}G]vino, raíces \[^\mn{E}]pro|\[\mnc{C}Am]fundos
%     Abraza el |\[G]mundo, nuestro ca|\[Am]mino, con tu medi|\[G]cina
%     Todas las |\[Am]plantas, las plantas di|\[G]vinas mari-ma|ri-ri-ri-ri-ri
%   \endverse
%   \notesoff
%   \beginverse
%     |^Yagesito cu|^raca, de la selva|^cita
%     La pachama|^mita, trae sabidu|^ría a nuestra fa|^milia,
%     Trae ale|^gría la liana que |^guía, Caapi caa|pi caapi caapi caapi
%   \endverse
%   \beginverse
%     |^Ayahuasquita |^cura, cura mi |^cuerpo
%     Con tu amor|^cito, cura mi |^mente viejo doctor|^cito
%     Cura mi |^gente, poderoso abue|^lito, vovo vo|vo vovo vo vo
%   \endverse
%   \beginverse
%     |^Trai nai nai con la |^vida, vida de la |^baila
%     Cura con la |^waira, limpia limpia |^taita vibrando con la |^walca
%     Con la madre |^selva, Taita cu|^raca ara a|ra ra ra ra ra\replay
%     |^Trai nai nai nai nai |^nai nai Tra na na na
%     |^nai nai Trai nai nai |^nai nai Tra na na na
%     |^nai nai Trai nai nai |^nai, Tra na na na
%     |^nai nai nai nai nai nai |^nai nai nai na |na na na na nai
%   \endverse
%   \goto{Arbolito divino}
%   \goto{Trai nai nai con la vida}
%   \musicnote{instrumental: Am G F G}
%   \beginchorus
%     \ind |\[Am]Trai nai nai nai nai |\[G]nai nai Tra na na na
%     \ind |\[Am]nai nai Trai nai nai |\[G]nai nai Tra na na na
%     \ind |\[F]nai nai Trai nai nai |\[G]nai, Tra na na na
%     \ind |\[F]nai nai nai nai nai nai |\[G]nai nai nai na |na na na na nai
%   \endchorus
%   \begin{translation}
%     Divine little tree, deep roots
%     Embrace the world and our path, with your medicine
%     All plants, divine plants mari-mari-ri-ri-ri-ri
%     \nextverse
%     Yage, from the forest,
%     The Mother Earth, bring wisdom to our family,
%     Bring joy, the liana that guides, Caapi caapi caapi caapi caapi
%     \nextverse
%     Ayahuasquita heal, heal my body,
%     With your love, heal my mind, old doctor
%     Heal my people Mighty grandpa, vovo-vovo-vovo-vo-vo
%     \nextverse
%     Trai nai nai with the life, life of the dance
%     Cure with the waira, clean clean taita vibrating with the walca
%     With the mother jungle, Taita curaca-ara-ara-ra-ra-ra-ra
%     Trai nai nai nai nai\ldots
%   \end{translation}
% \endsong

    % Portuguese language songs
% =========================
%
% The following sets the song number for the first song in this file.
% The number will automatically be incremented by one for each song.
% Please do not change this! Changing would make different versions of
% the songbook to have different numbers for the same songs, and it
% would totally mess up the selection booklets causing them to have
% wrong songs in them. (For the same reason, add new songs only to the
% end of each songs_ file.)
\setcounter{songnum}{200}


\beginsong{Defumação}[tags={censing, Santo Daime}, ph={I}, key={Am}, sks={Am, Gm--Am}]
  \meter{4}{4}
  \mnbeginchorus
    \lrep \[\bm] \[\mn{E}]De|\[\mnc{A}Am]fuma \[\mn{B}]de\[\bmc\mn{C}]fu\[\mn{D}]ma|\[\mnc{E}E]dor;
    \[\bm]Esta |\[\mnc{A}Am]casa de \[\mn{B}]Nos\[\bmc\mn{C}]so \[\mn{B}]Se|\[\mn{A}]nhor \rrep
    \lrep \[\bmc\mn{E}]Leva pra's |\[\mnc{A}Am]on\[\mnc{E}A7]das \[\mn{F}]do |\[\mnc{D}Dm]mar;
    O \[\bmc\mn{E}]mal \[\mn{F}]que a|\[\mnc{E}E]qui \[\bmc\mn{B}]pos\[\mn{C}]so es|\[\mnc{A}Am]tar \rrep
  \mnendchorus
  \meter{3}{4}
  \mnbeginchorus
    \lrep \[\mn{E}]De|\[\mnc{A}Am]fuma es\[\mn{E}]ta |\[\mn{A}]ca\[\mn{E}]sa |\[A7]bem \[\mn{F}]de\[\mn{E}]fu|\[\mnc{D}Dm]ma\[\mn{B}]da
    \[\mn{D}]Com |\[\mnc{B}E/B]a \[\mnc{G#}]Cruz \[\mn{E}]de |\[\mnc{B}E]Deus \[\mn{G#}]e\[\mn{E}]la |\[\mnc{D}E7/D]vai \[\mn{C}]ser \[\mn{B}]re|\[Am]za\[\mn{A}]da \rrep
    \lrep \[\mn{C}]Eu |\[C]sou reza|dor sou |fi\[\mn{D}]lho \[\mn{E}]de Um|\[\mnc{D}Dm]ban\[\mn{B}]da
    \[\mn{D}]Com |a \[\mn{E}]Cruz \[\mn{F}]de |\[\mnc{E}E]Deus \[\mn{F}]to\[\mn{E}]do |\[\mnc{D}E7/D]mal \[\mn{C}]se \[\mn{B}]a|\[Am]bran\[\mn{A}]da \rrep
  \mnendchorus
  \begin{translation}
    Cense, incenser
    This house of our Lord
    Take to the  waves of the sea
    the evil that might be here
    \nextverse
    Cense this incensed house
    With the Cross of God will be prayed
    I am the prayer, the son of Umbanda
    With the Cross of God all evil relents
  \end{translation}
  \begin{explanation}
    \textbf{Umbanda} is a syncretic Brazilian religion that blends African traditions
    with Roman Catholicism, Spiritism, and Indigenous American beliefs.
  \end{explanation}
\endsong


\beginsong{Forças Encantadas}[by={Vinícius Rocha}, ph={I}, key={A}, sks={B, G--D\shrp{}}]
  \audio[key=B]{https://soundcloud.com/vinicius-rocha-908652537/20-forc-as-encantadas}
  \transpose{-2}
  \mnbeginchorus\memorize
    \[\bmc\mn{F#}]Ve\[^\mn{D#}]nho |\[\mnc{B}B]cha\[^\mn{D#}]man\[\bmc\mnc{F#}]do \[^\mn{G}]bai|\[^\mn{F#}]xinho \[\bm]pra \[^\mn{D#}]che|\[^\mn{B}]gar \[^\mn{D#}]de\[\bmc\mn{F#}]va\[^\mn{G}]ga|\[\mnc{E}Em]rinho
    \lrep \[\bm]Es\[^\mn{F#}]sas |\[\mnc{G}G]for\[^\mn{F#}]ças \[\bmc\mnc{E}Em]en\[^\mn{D#}]can|\[\mnc{F#}B]ta\[^\mn{D#}]das \[\bmc\mn{F#}]da \[^\mn{D#}]Ra|\[^\mn{B}]in\[^\mn{D#}]ha \[\bmc\mn{F#}]da \[^\mn{/}]Flo|\[\mnc{/}\up{1}(Em)]resta \rrep
  \mnendchorus
  \notesoff
  \beginchorus
    ^Heya |^heya ^curan|deiros, ^apren|dizes ^de Tu|^pã
    \lrep ^Vem tra|^zendo as ^folhas |^verdes, ^netos |de ^Yori|^mã \rrep
  \endchorus
  \beginchorus
    ^Madre|^sita ^Chacro|nita ^vem das |matas ^encan|^tadas
    \lrep ^Serpen|^teia o ^meu Ja|^gube, ^mari|rí-ri-^ri-ri-|^ri \rrep
  \endchorus
  \beginchorus
    ^Heya |^heya ^meus ca|boclos, ^mestres |de sa^bedo|^ria
    \lrep ^Das cu|^ras da ^natu|^reza, ^trainai|nai-nai-^nai-nai-|^nai \rrep
  \endchorus
  \begin{translation}
    I come calling softly to arrive slowly
    These enchanted forces of the Queen of the Forest
    \nextverse
    Heya heya healers, Tupã's apprentices
    Come bring the green leaves, grandchildren of Yorimã
    \nextverse
    Beloved mother Chacruna comes from the enchanted woods
    Meander my Jagube, marirí-ri-ri-ri-ri
    \nextverse
    Heya heya my caboclos, masters of wisdom
    Of nature's cures, trainainai-nai-nai-nai-nai
  \end{translation}
  \begin{explanation}
    \begin{description}
      \item[Tupã] is the word for God in the \emph{Tupi} and \emph{Guarani}
      languages. Tupã is considered to be the creator of the universe, of light,
      of humanity and of the spirits of good and evil.
      \item[Yorimã] is a sacred term, which represents a Cosmic Power, an
      \emph{Orixá}. The \emph{Yorimã Vibration} is composed of several Entities
      that have reached spiritual maturity through experience.
    \end{description}
  \end{explanation}
\endsong


\beginsong{Guerreiro da Paz}[ititle={Warrior of Peace}, by={Orestes Grokar}, tags={Santo Daime}, ph={I}, key={Em}, sks={Em, Em--Em}]
  \audio[key=Bm]{https://www.youtube.com/watch?v=qy8G2J6ZHlw}
  \transpose{5} % in Em the notes go from G to B'
  \beginchorus
    % El que une la verdad % is this line supposed to be included in the song?
    \[\mn{G}]Eu chamo a |\[G]força, eu chamo a força, eu chamo a |\[F#m]for\[\mn{F#}]ça
    Força das |pedras para \[\mn{E}]me \[\mn{D}]fir|\[\mnc{E}Em]mar
    \[\mn{G}]Eu chamo a |\[G]terra, eu chamo a terra, eu chamo a |\[F#m]ter\[\mn{F#}]ra
    eu chamo a |terra para me en\[\mn{E}]ra\[\mn{D}]i|\[\mnc{E}Em]zar
  \endchorus
  \beginchorus
    \ind[1]\[\mn{B}]Eu chamo o |\[Bm]vento, eu chamo o vento, eu chamo o |\[F#m]ven\[\mn{A}]to
    \ind[1]Eu chamo o |vento vem me \[\mn{G}]e\[\mn{A}]le|\[\mnc{B}Bm]var
    \ind[1]Eu chamo o |fogo, eu chamo o fogo, eu chamo o |\[F#m]fo\[\mn{A}]go
    \ind[1]Eu chamo o |fogo para me pu\[\mn{G}]ri\[\mn{A}]fi|\[\mnc{B}Bm]car
  \endchorus
  \beginchorus
    \ind[2]\[\mn{D}]Eu chamo a |\[Bm]Lua, chamo o Sol, chamo as es|\[A]tre\[\mn{C#}]las
    \ind[2]Chamo o uni|verso para me i\[\mn{B}]lu\[\mn{C#}]mi|\[\mnc{D}Bm]nar
    \ind[2]Eu chamo a |água, chamo a chuva, e chamo o |\[A]ri\[\mn{C#}]o
    \ind[2]Eu chamo |todos para \[\mn{B}]me \[\mn{A}]la|\[\mnc{B}Bm]var
  \endchorus
  \beginchorus
    \ind[3]\[\mn{F#}]Eu chamo o |\[D]raio, o relâmpago e o tro|\[A]vã\[\mn{E}]o
    \ind[3]Eu chamo |todo o poder da \[\mn{D}]cri\[\mn{E}]a|\[\mnc{F#}Bm]ção
    \ind[3]Eu chamo o |\[D]mar, chamo o céu e o infi|\[A]ni\[\mn{E}]to
    \ind[3]Eu chamo |todos para nos \[\mn{D}]li\[\mn{C#}]ber|\[\mnc{B}Bm]tar
  \endchorus
  \brk
  \beginchorus
    \ind[2]\[\mn{D}]Eu chamo |\[Bm]Cristo, eu chamo Budha, eu chamo |\[A\mn{C#}]Krishna
    \ind[2]Eu chamo a |força de todos ori|\[Bm]xás
    \ind[2]Eu chamo |todos com suas forças di|\[A]vinas
    \ind[2]Eu quero |ver o universo ilumi|\[Bm]nar
  \endchorus
  \beginchorus
    \ind[3]\[\mn{F#}]Eu agra|\[D]deço pela vida e a co|\[A\mn{E}]ragem
    \ind[3]Ao uni|verso pela oportuni|\[Bm]dade
    \ind[3]E a minha |\[D]vida eu dedico com a|\[A]mor
    \ind[3]Ao sonho |vivo da nossa humani|\[Bm]dade
  \endchorus
  \beginchorus
    \ind[2]\[\mn{D}]Sou mensa|\[Bm]geiro, sou cometa, eu sou indí|\[A]ge\[\mn{C#}]na
    \ind[2]Eu sou |filho da nação do Arco |\[Bm]Íris
    \ind[2]Com meus ir|mãos eu vou ser mais um guer|\[A]reiro
    \ind[2]Na nobre |causa do Inka Reden|\[Bm]tor
  \endchorus
  \beginchorus
    \ind[3]\[\mn{F#}]Eu sou guer|\[D]reiro, eu sou guerreiro e vou lu|\[A]tan\[\mn{E}]do
    \ind[3]A minha e|spada é a palavra do a|\[Bm]mor
    \ind[3]O meu e|\[D]scudo é a bondade no meu |\[A]peito
    \ind[3]E o meu |elmo são os dons do meu sen|\[Bm]hor
  \endchorus
  \beginchorus
    \ind[2]\[\mn{D}]Eu agra|\[Bm]deço a nossa Mãe e ao nosso |\[A]Pa\[\mn{C#}]i
    \ind[2]E aos meus ir|mãos por todos me aju|\[Bm]dar
    \ind[2]A minha |glória para todos eu en|\[A]trego
    \ind[2]Porque nós |todos somos um nesta uni|\[Bm]ão
  \endchorus
  \beginchorus
    \ind[1]\[\mn{B}]Ñdarei a |\[Bm]sã, ñdarei a sã, ñdarei a |\[F#m]sã\[\mn{A}]'
    \ind[1]Desde o prin|cipio, todos nós somos ir|\[Bm]mãos!
    \ind[1]Orei ou|á, orei ouá, orei ou|\[F#m]á
    \ind[1]Viva o Po|der de todo o uni|\[Bm]verso!
    \vspace{1em}
    \ind[2]\[\mn{D}]Ñdarei a |\[Bm]sã, ñdarei a sã, ñdarei a |\[A]sã\[\mn{C#}]'
    \ind[2]Desde o prin|cipio, todos nós somos ir|\[Bm]mãos!
    \ind[2]Orei ou|á, orei ouá, orei ou|\[A]á
    \ind[2]Viva o Po|der de todo o uni|\[Bm]verso!
    \vspace{1em}
    \ind[3]\[\mn{F#}]Ñdarei a |\[D]sã, ñdarei a sã, ñdarei a |\[A]sã\[\mn{E}]'
    \ind[3]Desde o prin|cipio, todos nós somos ir|\[Bm]mãos!
    \ind[3]Orei ou|\[D]á, orei ouá, orei ou|\[A]á
    \ind[3]Viva o Po|der de todo o uni|\[Bm]verso!
    \vspace{1em}
    \musicnote{Fade out\ldots}
  \endchorus
  \begin{translation}
    I call the force, I call the force, I call the force
    Forces of the stones to firm me
    I call the Earth, I call the Earth, I call the Earth
    I call the Earth to root me
    \nextverse
    I call the wind, I call the wind, I call the wind
    I call the wind comes to rise me up
    I call the fire, I call the fire, I call the fire
    I call the fire to become purified
    \nextverse
    I call the Moon, I call the Sun, I call the stars
    I call the universe to illuminate me
    I call the water, I call the rain and I call the river
    I call all to wash me
    \nextverse
    I call the ray, the lightning and the thunder
    I call all the power of the creation
    I call the sea, I call the sky and the infinite
    I call all to free us
    \nextverse
    I call Christ, I call Buddha, I call Krishna
    I call the force of all Orixás
    I call all with their divine forces
    I want to see the universe light up
    \nextverse
    I thank for the life and the courage
    To the universe for the opportunity
    And my life I dedicate with love
    To our humanity's alive dream
    \nextverse
    I am messenger, I am comet, I am indigenous
    I am son of the nation of the rainbow
    With my siblings I will be one more warrior
    In the noble cause of Inka Redentor
    \nextverse
    I am warrior, I am warrior and I am going struggling
    My sword is the word of the love
    My shield is the kindness in my chest
    And my helmet is my gentleman's talents
    \nextverse
    I thank our Mother and to our Father
    And to all my brothers for to help me
    My glory for all I give
    Because us all are one in this union
    \nextverse
    Ñdarei a sã, ñdarei a sã, ñdarei a sã
    From the beginning all of us are brothers!
    Orei ouá, orei ouá, orei ouá
    Live the Power of the whole universe!
  \end{translation}
\endsong


\beginsong{Força da Floresta}[ph={I, II}, key={Dm}, sks={Dm, Cm--Em}]
  \transpose{5} % in Dm the notes go from A to A
  \beginchorus\memorize
    \[^\mn{A}]Chamo |\[\mnc{D}Dm]força| la da \[^\mn{E}]flo|\[\mnc{C}Am]res\[^\mn{A}]ta \altchords{\id[1]{(Am)}|Dm | - |Am}
    E a força |\[Em]vem| para nos ensi|\[Am]nar | \e \altchords{|Em | - |Am | \e}
  \endchorus
  \notesoff
  \beginchorus
    \ind Yana|^heê| Yana|^heê
    \ind Yana|^heê| Yana|^heê | \e
  \endchorus
  \beginchorus
    Chamo a |^cura| la da flo|^resta
    E a cura |^vem| para nos cu|^rar | \e \goto{Yanaheê}
  \endchorus
  \beginchorus
    Segura |^firme| que eu vou te |^levar
    E te mostrar |^el | minha mãe Ye|^manjá | \e \goto{Yanaheê}
  \endchorus
  \beginchorus
    Chamo a |^força| da linha de |^tucum
    E a força |^vem| força de ê O|^gum | \e \goto{Yanaheê}
  \endchorus
  \begin{translation}
    I call the power of the forest
    And power comes to teach us
    \nextverse
    I call the healing of the forest
    And it comes to heal us
    \nextverse
    Hold on tight, I'll take you
    And show you my mother \textbf{Yemanjá} \emph{(Queen of the Ocean)}
    \nextverse
    I call the power of the tucum line
    And power comes, power of \textbf{Ogum} \emph{(Warrior Spirit)}
  \end{translation}
\endsong


\beginsong{Companheiro na Escuridão}[by={Bettina Maureenji, Amu Ahava}, ph={II}, key={Am}, sks={Am, Gm--Bm}]
  \audio[]{https://maureenji.bandcamp.com/track/companheiro-na-escuridao-2}
  \meter{3}{4}
  %\capo{3}
  \beginchorus\memorize
    |\[\mnc{C}Am]Sinto a |\[\mnc{B}Em]tu\[^\mn{C}]a |\[\mnc{A}Am]mã|o to|\[\mnc{A}F]cando meu |\[\mnc{G}G]co\[^\mn{A}]ra|\[\mnc{E}Am]ção | \e
    Ca|\[Em]lor que |\[F]limpa minha |\[Am]al|ma cla|\[F]rea a |\[Em]men|\[Am]te | \e
  \endchorus
  \notesoff
  \beginchorus
    |^Vejo mun|^do univer|^sal | |^sinto a |^uni|^ão | \e
    |^Força da |^luz afa|^gai |me dai-|^me con|^cien|^çia | \e
  \endchorus
  \beginchorus
    |^Oh tran|^quili|^da|de to|^cando meu |^cora|^ção | \e
    |^Como |^um a|^mi|go compa|^nheiro na es|^curi|^dão | \e
  \endchorus\glueverses
  \beginverse
    |\[Em]Como |\[F]um a|\[Am]mi|go \lrep compa|\[F]nheiro na es|\[Em]curi|\[Am]dão | \e \rrep\rep{4}
  \endverse
  \begin{translation}
    I feel your hand touching my heart
    Warmth that cleanses my soul, clears the mind
    \nextverse
    I see universal world, I feel the union
    Force of light, fondle me, give me conscience
    \nextverse
    Oh tranquility touching my heart
    Like a friend, companion in the dark
  \end{translation}
\endsong


\scleardpage
\beginsong{Silencio}[by={Elves do Juruá}, ph={II}, key={Am}, sks={B\flt{}m, Gm--Bm}]
  \audio[key=B\&m]{https://www.youtube.com/watch?v=g7RpwoRxiDA}
  \audio[key=B\&m]{https://soundcloud.com/amarcca/elves-do-jurua-silencio}
  \newchords{chords_silencio_a}\newchords{chords_silencio_b}\newchords{chords_silencio_c}
  \beginchorus\memorize[chords_silencio_a]
    \[^\mn{A}]Silenci|\[Am]ou|\[\mnc{G}G]{'}, agora \[\bm]vai \[^\mn{F}]si\[^\mn{E}]len\[^\mn{D}]ci|\[\mnc{E}Am]ar | \e
    \endchorus\glueverses\beginverse\memorize[chords_silencio_b]
    |\[C] Ouço \[\bm]canto na flo|\[Dm]resta
    Agora \[\bm]quem vem coman|dando é \[\bm]Papai Oxa|\[Am]lá | \e
  \endverse
  \beginchorus\memorize[chords_silencio_c]
    \ind \[^\mn{A}]E \[^\mn{B}]traz \[^\mn{C}]a |\[\mnc{D}Dm]Fo|'rça, e traz \[^\mn{F}]a |\[\mnc{E}Am]Paz | \e
    \ind |\[C] Eu an\[\bm]dando com si|\[Dm]lencio
    \ind Pois so\[\bm]mente o si|lencio é que sem\[\bm]pre me satis|\[Am]faz | \e
  \endchorus
  \notesoff
  \beginchorus\replay[chords_silencio_a]
    É Oxa|^lá|^{'}, que ago^ra vai coman|^dar | \e
    \endchorus\glueverses\beginverse\replay[chords_silencio_b]
    |^ Ele ^chega no si|^lencio
    Ele ^vem trazendo a |paz para ^nós ressusci|^tar | \e
  \endverse
  \beginchorus\replay[chords_silencio_c]
    \ind E a traz |^Lu|'z, e traz o A|^mor | \e
    \ind |^ Vou an^dando nessa |^vida
    \ind Cami^nhando com ca|rinho nos ca^minhos do Se|^nhor | \e
  \endchorus
  \beginchorus\replay[chords_silencio_a]
    Silenci|^ou|^{'}, ago^ra silenci|^ou | \e
    \endchorus\glueverses\beginverse\replay[chords_silencio_b]
    |^ No si^lencio da minha |^mente
    Agra^deço a Oxa|lá por ^todo seu a|^mor | \e
  \endverse
  \beginchorus\replay[chords_silencio_c]
    \ind Silenci|^ou|' Silenci|^ou|' \e
    \ind |^ Es^tou em harmo|^nia
    \ind Sen^tindo muita |paz e vi^vendo com A|^mor | \e
  \endchorus
  \beginchorus\replay[chords_silencio_a]
    E vai vo|^ar|^{'} Beija-^flor que vai vo|^ar | \e
    \endchorus\glueverses\beginverse\replay[chords_silencio_b]
    |^ Ele ^voa com do|^çura
    Pa^ra o seu per|fume ^poder espa|^lhar | \e
  \endverse
  \beginchorus\replay[chords_silencio_c]
    \ind Delica|^de|'za e muito A|^mor | \e
    \ind |^ Com ^a força da |^paz
    \ind Vou vo^ar com beija-|\[Am]flor | \e % One measure less in the last verse
  \endchorus
  \begin{translation}
    Silenced, now will silence
    I hear a song in the forest
    Now who's in charge is Father Oxalá
    \nextverse
    And brings Strength, and brings Peace
    I'm walking with silence
    Because only silence always satisfies me
    \nextverse
    It is Oxalá who will now command
    He arrives in silence
    He has been bringing peace to resurrect us
    \nextverse
    And brings the Light, and brings the Love
    I'm walking in this life
    Walking with affection in the ways of the Lord
    \nextverse
    Silenced, now silent
    In the silence of my mind
    I thank Oxalá for all your love
    \nextverse
    Silenced, silenced
    I'm in harmony
    Feeling very peaceful and living with Love
    \nextverse
    And it will fly, hummingbird that will fly
    He flies with sweetness
    so your perfume can spread
    \nextverse
    Delicacy and Love
    With the force of Peace
    I will fly with the hummingbird
  \end{translation}
  \begin{explanation}
    \begin{description}
      \item[Oxalá] is the orixá associated with the creation of the world and
        the human species. He is considered and worshiped as the greatest
        and most respected of all the orixás of the African pantheon. He
        symbolizes peace. Oxalá means white \emph{(oxa)} light \emph{(alá)}.
    \end{description}
  \end{explanation}
\endsong


\beginsong{Mãe Jurema}[by={Maria Alice}, ph={II}, tags={Santo Daime}, key={Em}, sks={Em, Dm--G\shrp{}m}]
  \audio[key=Em]{https://www.nossairmandade.com/hymn/1293/MãeJurema}
  \audio[key=Em]{https://soundcloud.com/kaktus-raam/mae-jurema}
  \transpose{2} % Dm->Em, in Em notes go from A to G'
  \newchords{chords_maejurema_a}\newchords{chords_maejurema_b}
  \transpose{-2}\preferflats
  \meter{3}{4}
  \beginchorus\memorize[chords_maejurema_a]
    \[^\mn{B}]Ju|\[\mnc{E}Em]rema \[^\mn{F#}]vós |\[^\mn{G}]sois \[^\mn{F#}]min\[^\mn{E}]ha |\[\mnc{D}D]Mãe, | Ra|\[\mnc{C}C]inha do |\[^\mn{E}]meu \[^\mn{D}]Ja\[^\mn{C}]cu|\[\mnc{B}Bm]tá | \e
  \endchorus\glueverses\beginchorus\memorize[chords_maejurema_b]
    Ma|\[Am]mãe de |toda flo|\[Em]res|ta, Ju|\[B7]rema, |ôh Jure|\[Em]ma | \e
  \endchorus
  \notesoff
  \beginchorus\replay[chords_maejurema_a]
    Ju|^rema, vós |sois caça|^do|ra das |^almas na |escuri|^dão | \e
  \endchorus\glueverses\beginchorus\replay[chords_maejurema_b]
    Vós |^caça com |a vossa |^fle|cha, illu|^mina com |o cora|^ção | \e
  \endchorus
  \beginchorus\replay[chords_maejurema_a]
    O |^vosso co|ração di|^vi|no tem a |^chave do |conheci|^men|to
  \endchorus\glueverses\beginchorus\replay[chords_maejurema_b]
    Vós |^sois es|trela ca|^den|te, que |^cai lá |no firma|^men|to
  \endchorus
  \beginchorus\replay[chords_maejurema_a]
    Oh! |^Min|ha Mãe Ju|^re|ma, a|^qui ven|ho vos lou|^var | \e
  \endchorus\glueverses\beginchorus\replay[chords_maejurema_b]
    Tra|^zendo |o meu Ro|^sá|rio, pa|^ra vos |apresen|^tar | \e
  \endchorus
  \beginchorus\replay[chords_maejurema_a]
    O |^meu roga|tivo é de |^luz | para |^toda a |escuri|^dão | \e
  \endchorus\glueverses\beginchorus\replay[chords_maejurema_b]
    Ju|^rema dai-|me a vossa |^cha|ve, para |^eu cumprir |minha mis|^são | \e
  \endchorus
  \begin{translation}
    Jurema you are my Mother, Queen of my Jacutá
    Mother of the whole forest, Jurema oh Jurema
    \nextverse
    Jurema, you are hunter of the souls of the darkness
    You hunt with your arrow, illuminate with the heart
    \nextverse
    Your divine heart has the key of knowledge
    You are shooting star that falls there of the firmament
    \nextverse
    Oh! My Mother Jurema, here I come to praise you
    Bringing my Rosary to present to you
    \nextverse
    My prayer is of light for all the darkness
    Jurema give me your key for me to accomplish my mission
  \end{translation}
  %% Explanation commented out because of lack of space on the page
  %\begin{explanation}
  %  \begin{description}
  %    \item[Jacutá] Altar designation. House of the saint.
  %      In \emph{Candomblé} it is a title given to \emph{Xangô}, which means ``to fight with the stones''.
  %  \end{description}
  %\end{explanation}
\endsong


\beginsong{Oh Jurema}[ph={II}, key={Em}, sks={Em, Am--F\shrp{}m}]
  \beginchorus\memorize
    \[^\mn{E}]Tava \[^\mn{F#}]na |\[\mnc{G}Em]mata com minha fle\[^\mn{F#}]cha \[^\mn{E}]na |\[\mnc{A}D]mão
    E Mamãe Ju|\[B7]rema dentro do meu cora|\[Em]ção
  \endchorus
  \notesoff
  \beginchorus
    E lá na |^mata encontrei Tupinam|^bá
    Que mandou Ju|^rema para vir me acompan|^har
  \endchorus
  \beginchorus
    \noteson
    \ind \[\mn{E}]Ju|\[Em]rema \[\mn{G}]ôh \[\mn{F#}]Ju\[\mn{E}]re|\[\mnc{A}D]ma
    \ind Ju|\[B7]rema ôh Jure|\[Em]ma
  \endchorus
  \beginchorus
    Foi lá na |^mata que encontrei a inspira|^ção
    Para eu se|^guir no caminho do cora|^ção
  \endchorus
  \goto{Jurema}
  \beginchorus
    Cabocla Ju|^rema, Cabocla guerre|^iro
    Oh Jur|^ema feitice|^ira
  \endchorus
  \goto{Jurema}
  \begin{translation}
    I was in the woods with my arrow in my hand,
    and Mama Jurema inside my heart.
    \nextverse
    And there in the woods I found Tupinambá
    who sent Jurema to come with me.
    \nextverse
    \ind Jurema oh Jurema
    \ind Jurema oh Jurema
    \nextverse
    It was in the woods that I found the inspiration
    to follow in the way of the heart.
    \nextverse
    Cabocla Jurema, Cabocla warrior,
    Oh Jurema, sorceress.
  \end{translation}
  \begin{explanation}
    \begin{description}
      \item[Caboclo Tupinambá] is a spirit that personifies the orixá Oxóssi
        (the orixá of hunt, forest, animals and wealth). He is firm and 
        combative without ceasing to be cheerful and affectionate. As wise,
        patient and charismatic, he is like a father.
      \item[Tupinambá, the people:] one of the ethnic groups that inhabited the
        northeastern coast of present-day Brazil before the conquest by the
        Portuguese
      \item[Cabocla Jurema] (the spirit) is the beautiful Queen of the Forest,
        the eldest daughter of Caboclo Tupinambá. She conveys energy and
        courage, always comforting those in need.
      \item[Jurema the plant] (\emph{Mimosa tenuiflora}) is a teacher
        plant, which is traditionally used in an entheogenic decoction, also
        called Jurema, in northeastern Brazil. The plant (especially its root
        bark) contains dimethyltryptamine, which seems to be orally active
        even without additional plants, though no beta-carbolines have been
        found in it.
    \end{description}
  \end{explanation}
  \yesendsongvfill
\endsong


\beginsong{Calma e Tranquilidade}[by={Deva Premal, Miten}, ph={II, III}, key={Am}, sks={Bm, Am--Em}]
  \audio[]{https://www.youtube.com/watch?v=arNT11J2b-k}
  \audio[]{https://soundcloud.com/caminhantecosmico/calma-e-tranquilidade}
  \transpose{5} % in Am the notes go from G to E
  \beginverse
    \[^\mn{E}]Cal\[^\mn{F#}]ma |\[\mnc{G}Em]e Tran\[\mnc{F#}D]qui\[^\mn{E}]li|\[\mnc{G}Em]dade, são as |ordens \[D]do Sen|\[Em]hor
    Calma |e Tranquili|\[Am]dade para |\[B7]receber o A|\[Em]mor | \e
    \[D]Hoo-|\[Em]oo |\hspace{1em}\[D]Hoo-|\[Em]oo | \e
  \endverse
  \begin{translation}
    Calm and Tranquility, these are the Lord's orders
    Calm and Tranquility to receive Love
  \end{translation}
\endsong


\beginsong{Nova Consciência}[by={Tiago Lucci}, key={Bm}, sks={C\shrp{}m, Cm--E\flt{}m}, ph={II}]
  \audio[key=Em]{https://soundcloud.com/sergio-moreira-slmore/10_nova-consciencia}
  \transpose{-5}
  \mnbeginverse
    |\[\mnc{B}Em]{No caminho} pa\[^\mn{C}]ra o |\[^\mn{B}]des\[^\mn{G}]per\[^\mn{E}]tar |\[^\mn{B}]é preciso re\[^\mn{G}]u|\[\mnc{A}Am]nir
    |Tudo que um di\[^\mn{B}]a |\[^\mn{A}]se \[^\mn{G}]pas\[^\mn{F#}]sou |\[^\mn{A}]e você não quis \[^\mn{F#}]sen|\[\mnc{G}Em]tir
    |Cada nó nesse \[^\mn{F#}]ca|\[^\mn{E}]min\[^\mn{B}]ho |\[^\mn{G}]é preciso en\[^\mn{E}]xer|\[\mnc{F#}B7]gar
    |\[^\mn{A}]Desatar o i\[^\mn{G}]ni|\[^\mn{F#}]mi\[^\mn{B}]go |\[\mnc{C}C]para a luz reen\[^\mn{B}]con|\[\mnc{G}Em]trar
    |Cada porta que s\[^\mn{F#}]e a|\[^\mn{G}]brir |outra vai se re\[^\mn{F#}]ve|\[\mnc{D#}B7]lar
    |\[^\mn{A}]Sei que devo pros\[^\mn{G}]se|\[^\mn{F#}]gui\[^\mn{B}]r |\[^\mn{B}]sei que um dia vou \[^\mn{G}]che|\[\mncii{F#}{E}Em]gar
  \mnendverse
  \notesoff
  \beginverse
    |^Vamos todos juntos |meus irmãos |replantar a santa |^luz
    |Dissolvendo toda a |ilusão |que ainda nos con|^duz
    |O acordo que fi|zemos |para nunca mais sen|^tir
    |Precisa ser desco|berto |^para um novo exis|^tir
    |Nessa nova consci|ência |você pode perce|^ber
    |Como é bela a sua |vida |o poder de escol|^her
  \endverse
  \beginverse
    |^Todos que aqui se |adentrar |devem receber a |^luz
    |Firmem na força do |meu amor |esse brilho que re|^luz
    |É preciso paci|ência |para tu par|ter o |^não
    |É preciso obedi|ência |^para estar nesse sa|^lão
    |Quem recebe uma |vida |junto ganha uma mis|^são
    |Resgatar toda a ver|dade |replantar a uni|^ão
  \endverse
  \begin{translation}
    On the way to awakening it is necessary to gather
    Everything that passed in a day and you didn't want to feel
    Every node on this path needs to be seen
    Untie the enemy for the light to find again
    Every door that opens another will reveal itself
    I know that I must go on, I know that one day I will arrive
    \nextverse
    Let's all go together, my sisters and brothers, to replant the holy light
    Dissolving all the illusion that still drives us
    The deal we made never to feel again
    Needs to be discovered for a new one to exist
    In this new consciousness you can see
    How beautiful is your life, the power to choose
    \nextverse
    All who enter here must receive the light
    Firm in the strength of my love this glow that shines
    It takes patience for you to break the no
    It takes obedience to be in this hall
    Whoever gets a life together gets a mission
    Rescue all the truth, replant the union
  \end{translation}
\endsong


\beginsong{Forças da Rainha da Floresta}[index={Santa Maria vem chegando}, by={Irineu Barsé}, tags={Santo Daime}, ph={II, III}, key={Am}, sks={Bm, (Am)--(Cm)}]
  \meter{3}{4}
  \beginchorus\memorize
    \[^\mn{E}]Santa Ma|\[\mnc{E}Am]ria vem che|\[\mnc{D}E]gando nes\[^\mn{E}]se \[^\mn{F}]ba\[^\mn{E}]ta|\[\mnc{C}Am]lhã\[^\mn{A}]o |\[A7] \e
    Trazendo as |\[Dm]forcas da Ra|\[A7]inha da flo|\[Dm]resta | \e
    São essas |\[Dm]forcas vinda |\[G]de nossa Se|\[C]nhora |\[F] \e
    Que centra|\[Dm7]lizam essa |\[E]luz aqui na |\[Am]terra | \e
  \endchorus
  \notesoff
  \beginchorus
    Eu vou pe|^dindo sempre |^a minha co|^ragem |^ \e
    Para se|^guir nesta ba|^talha do a|^mor | \e
    Com o com|^forto |^da minha mãe|^zinha |^ \e
    Com sua |^luz |^com seu resplen|^dor | \e
  \endchorus
  \beginchorus
    \ind Eu vou se|\[Dm]guindo sempre no ca|\[G]minho
    \ind Sempre gui|\[C]ado pela luz di|\[F]vina
    \ind Sempre bus|\[Dm7]cando apren|\[E]der nesta dou|\[Am]trina | \e
  \endchorus
  \begin{translation}
    Santa Maria has been arriving in this company
    Bringing the forces of the Queen of the Forest
    They are these forces that come from our Lady
    That centralize this Light here on Earth
    \nextverse
    I will always ask for my courage
    To follow in this battle of love
    With the comfort of my mother
    With her light, with her radiance
    \nextverse
    I will always follow the Path
    Always guided by the Divine Light
    Always seeking to learn in this doctrine
  \end{translation}
\endsong


\beginsong{Linda Verdade}[ititle={Beautiful Truth}, tags={source}, ph={II, III}, key={Am}, sks={Bm, G\shrp{}m--Em}]
  \beginchorus
    |\[\mnc{A}Am]Fazei de \[\mnc{B}G]Deus uma |\[\mnc{C}F]reali\[\mnciii{B}{A}{G#}E]dade \altchords{\id[1]{(D)}|Dm C |B\flt{} A}
    \endchorus\glueverses\beginchorus
    |\[C]Que Ele fa\[Dm]{rá de} ti |\[G]lin\[(Em)]{da ver}\[\bmc{}Am]dade \altchords{|F Gm |C (Am) Dm}
  \endchorus
  \textnote{in English:}
  \beginchorus
    |\[Am]Allow \[G]God to be|\[F]come a re\[E]ality
    \endchorus\glueverses\beginchorus
    |\[C]And He will make \[Dm]out of you a |\[G]beau\[(Em)]tiful \[\bmc{}Am]truth
  \endchorus
  \textnote{suomeksi:}
  \beginchorus
    |\[Am]Salli \[G]Jumalan |\[F]tulla \[E]todeksi
  \endchorus\glueverses
  \beginchorus
    |\[C]Ja hän tekee \[Dm]sinusta |\[G]kau\[(Em)]{niin to}\[\bmc{}Am]tuuden
  \endchorus
\endsong


\scleardpage
\beginsong{Ilumina Minha Mãe}[by={Marie Gabriella}, ph={II, III}, key={Am}, sks={Am, Am--Bm}]
  \audio[key=Am]{https://soundcloud.com/mariegabriellaoficial/iluminaminhamae}
  \audio[key=Am]{https://www.youtube.com/watch?v=dh5q0YcI-1k}
  \beginchorus\memorize
    \[^\mn{E}]Ilu|mina minha |\[\mnc{A}Am]mãe \[^\mn{E}]esse |medo, \[^\mn{F}]por \[^\mn{G}]fa|\[\mnc{F}Dm]vor
    \[^\mn{E}]Me |\[^\mn{D}]mostra a liber|\[\mnc{G}G]da\[^\mn{D}]de de vi|ver no \[^\mn{E}]seu \[^\mn{F}]a|\[\mnc{E}C]mor \replay\notesoff
    Me |leva as profun|^dezas das |minhas emo|^ções
    Para |eu ver com cla|^reza inconsci|entes nega|^ções \replay
    Que |me deixam dor|^mindo em dis|torcido pra|^zer
    Se|guindo distra|^ído tão dis|tante de vo|^cê \replay
    Como |vós eu quero |^ser como a |lua clare|^ar
    Refle|tindo a luz do |^sol para a |noite ilumi|^nar \replay
    |Rumo ao oce|^ano nas su|as águas brin|^car
    Sua be|leza apreci|^ando para a |vida cele|^brar
  \endchorus
  \beginchorus\memorize
    | |\[Am] | |\[Dm] | |\[G] | |\[C] \e
  \endchorus
  \beginchorus\memorize
    \ind \[^\mn{C}]Óh |\[\mnc{A}Am]mãe, | \[^\mn{E}]óh mãe \[^\mn{C}]de |\[\mnc{D}Dm]De|us
    \ind \[^\mn{B}]Pro|\[\mnc{G}G]te|ja \[^\mn{D}]os fil\[^\mn{B}]hos |\[\mnc{C}Am]se|us \replay\notesoff
    \ind Óh |^mãe, | óh mãe de |^De|us
    \ind Per|^do|e os filhos |^se|us
  \endchorus
  \beginchorus\memorize
    \ind\ind \[^\mn{D}]Dei|\[\mnc{F}Dm]xa a \[^\mn{E}]luz \[^\mn{D}]bril|\[^\mn{F}]har, |\[^\mn{A}]a flor \[^\mn{G}]flo\[^\mn{F}]res|\[\mnc{E}Am]cer
    \ind\ind A |vida reve|\[E]lar a ver|dade do meu |\[Am]ser \replay\notesoff
    \ind\ind Dei|xa o sol nas|^cer dentro |do meu cora|^ção
    \ind\ind O a|mor manifes|^tar a mais |pura grati|^dão \up{2}( | \e )
  \endchorus
  \beginchorus\memorize
    \ind Óh |\[Am]mãe, | mãe natu|\[Dm]re|za
    \ind Most|\[G]rai | vossa be|\[Am]le|za\replay
    \ind Óh |^mãe, | mãe natu|^re|za
    \ind Lem|^brai-|nos a nossa es|^sên|cia
  \endchorus
  \beginchorus\memorize
    \ind\ind Inter|\[Dm]ceda junto ao |pai levando a |minha ora|\[Am]ção
    \ind\ind O pe|dido é a |\[E]benção que me |trás a aceita|\[Am]ção \replay
    \ind\ind A |vós quero ser|^vir, junto a |ti quero se|^guir
    \ind\ind Apren|dendo a dizer |^sim para |tudo que a de |^vir \up{2}( | \e )
  \endchorus
  \beginverse\memorize
    \ind Óh |\[Am]mãe, | óh mãe de |\[Dm]De|us
    \ind Óh |\[G]mãe, | óh mãe de |\[Am]De|us \replay
    \ind Óh |^mãe, | óh minha |^mãe | \e
    \ind Óh |^mãe, | óh minha |^mãe | \e
  \endverse
  \begin{translation}
    Illuminate my mother this fear, please
    Show me the freedom to live in your love
    Take me to the depths of my emotions
    For me to clearly see unconscious denials
    That leave me sleeping in distorted pleasure
    Following distracted so far from you
    Like you I want to be like the moon light
    Reflecting sunlight for night illuminate
    Towards the ocean in its waters to play
    Your beauty appreciating for life to celebrate
  \nextverse
    Oh mother, oh mother of God
    Protect your children
    Oh mother, oh mother of God
    Forgive your children
  \nextverse
    Let the light shine, the flower bloom
    Life reveals the truth of my being
    Let the sun rise inside my heart
    Love manifests the purest gratitude
  \nextverse
    Oh mother, mother nature
    Show your beauty
    Oh mother, mother nature
    Remind us of our essence
  \nextverse
    Intercede with the father leading my prayer
    The request is the blessing that brings me acceptance
    I want to serve you, I want to follow you
    Learning to say yes to everything to come
  \nextverse
    Oh mother, oh mother of God
    Oh mother, oh mother of God
    Oh mother, oh my mother
    Oh mother, oh my mother
  \end{translation}
\endsong


\beginsong{Seres Vivos da Floresta}[by={Giti Bond?}, ph={II, III}, key={Am}, sks={Bm, Gm--Em}]
  \audio[key=Em]{https://www.youtube.com/watch?v=gXMoVIK7BMg}
  \audio[]{https://www.youtube.com/watch?v=uNv1ez7qlrk}
  \audio[]{https://soundcloud.com/radiomadrinha/seres-vivos-da-floresta}
  \beginchorus\memorize
    |\[\mnc{C}Am]Seres vivos \[^\mn{B}]da \[^\mn{A}]flor|\[\mnc{D}Dm]esta \altchords{|Bm |Em}
    |\[G]Venham me ilumin|\[Am]ar \altchords{|A |Bm}
    \lrep Es|\[Am]tou aqui est|\[Dm]ou cantando \altchords{|Bm |Em}
    Estou a|\[G]berto para me cur|\[Am]ar \rrep \altchords{|A |Bm}
  \endchorus
  \notesoff
  \beginchorus\replay
    |^A floresta traz mist|^érios
    E pe|^dimos vossa proteç|^ão
    \lrep |^Nossos medos v|^ão sumindo
    E vai se a|^brindo nosso coraç|^ão \rrep
  \endchorus
  \beginchorus\replay
    |^Toda luz se revela|^ndo
    Nos damos |^conta tudo é am|^ar
    \lrep |^A clareza va|^i surgindo
    E agora |^sei que posso vo|^ar \rrep
  \endchorus
  \begin{translation}
    Living beings of the forest, come to enlighten me.
    I am here, I am singing, I am open to be healed.
    \nextverse
    The forest brings mysteries and we ask for your protection.
    Our fears are disappearing and our hearts are opening.
    \nextverse
    All the light is revealed and we realize that everything is love.
    Clarity is rising and now I know that I can fly.
  \end{translation}
\endsong


\beginsong{Está em Você}[by={Elisa Cristal}, ph={II}, key={Em}, sks={Gm, Fm--G\shrp{}m}]
  \audio[]{https://www.youtube.com/watch?v=Ax6Dkn9dtVM}
  \transpose{-5}
  \capo{3}
  \newchords{chords_estaemvoce_a}\newchords{chords_estaemvoce_b}
  \beginverse\memorize[chords_estaemvoce_a]
    |\[Dm] |\[^\mn{A}]A for\[^\mn{C}]ça \[^\mn{G}]do |\[\mnc{A}Am]mar | |\[G] |\[^\mn{G}]po\[^\mn{F}]de \[^\mn{E}]most|\[\mnc{C}F]rar | \e
    |\[Dm] Abre os |olhos para |\[Am]ver | |\[G] o mis|tério reve|\[F]lar | \e
  \endverse\glueverses
  \beginchorus\memorize[chords_estaemvoce_b]
    |\[C] Es|tá em vo|\[F]cê | |\[C] |todo po|\[F]der | \e
    De a|\[G]mar | a|\[Am]mar | \e
  \endchorus
  \notesoff
  \beginverse\replay[chords_estaemvoce_a]
    |^ |Deixa o Sol bril|^har | |^ |ilumi|^nar | \e
    |^ É |só você que|^rer | |^ |simplesmente |^ser | \e
  \endverse\glueverses
  \beginchorus\replay[chords_estaemvoce_b]
    |^ As |nuvens podem che|^gar | |^ não |deixe se aba|^lar | \e
    É |^só | obser|^var | \e
  \endchorus
  \beginverse\replay[chords_estaemvoce_a]
    |^ |Abre o cora|^ção | |^ não |há outro lu|^gar | \e
    |^ |Cante esta can|^ção | |^ pra |vida seme|^ar | \e
  \endverse\glueverses
  \beginchorus\replay[chords_estaemvoce_b]
    |^ |Flor linda |^flor | |^ |flor formosa |^flor | \e
    Flor de a|^mor | beija |^flor | \e
  \endchorus
  \begin{translation}
    The force of the sea can show
    Open your eyes to see the mystery reveal
    All power is in you: of loving, loving
    \nextverse
    Let the sun shine, illuminate
    It's just that you simply want to be
    The clouds may come, do not let it unsettle you; just watch
    \nextverse
    Open the heart, there is no other place
    Sing this song to sow life
    Lovely flower bloom, beautiful flower bloom; flower of love, kiss a flower
  \end{translation}
\endsong


\beginsong{O Lua}[by={Cristina Tati}, tags={Moon, Santo Daime}, ph={II}, key={Am}, sks={Am, G\shrp{}m--(Am)}]
  \audio[]{http://www.nossairmandade.com/hymn.php?hid=3658}
  \newchords{chords_olua_a}\newchords{chords_olua_b}
  \transpose{5}
  \beginchorus\memorize[chords_olua_a]
    \[^\mn{B}]Ó |\[\mnc{E}Em]Lua! | Tu\[^\mn{F#}]a \[^\mn{G}]luz \[^\mn{F#}]vei\[^\mn{E}]o e |\[^\mn{B}]me \[^\mn{G}]i\[^\mn{E}]lu\[^\mn{G}]min|\[\mnc{B}Bm]ou.
  \endchorus\glueverses
  \beginchorus\memorize[chords_olua_b]
    Re|\[Am]cebe | esta prenda de A|\[Em]mor no Coraç|ão.
    E |\[D]canta |\[B] para \[B7]os teus ir|\[Em]mãos. \[\up{1}E7]
  \endchorus
  \notesoff
  \beginchorus\replay[chords_olua_a]
    Ó |^Lua! | Quem te fez foi o |Deus Onipot|^ente.
  \endchorus\glueverses
  \beginchorus\replay[chords_olua_b]
    Re|^cebe | esta força e |^firma tua m|ente.
    E |^canta |^ para ^toda esta |^gente. ^
  \endchorus
  \beginchorus\replay[chords_olua_a]
    Ó |^Lua! | Teu luar a Ter|ra ilumin|^ou.
  \endchorus\glueverses
  \beginchorus\replay[chords_olua_b]
    Ma|^mãe | no espelho da |^águas se ol|hou.
    So|^rriu, |^ e a Te^rra se ale|^grou. ^
  \endchorus
  \begin{translation}
    Oh Moon! Your light came and illuminated me.
    Receive this gift of Love in the Heart.
    And sing to your brothers and sisters.
    \nextverse
    Oh Moon! Who made you was the Almighty God.
    Receive this strength and firm your mind.
    And sing for all these people.
    \nextverse
    Oh Moon! Your light illuminated the Earth.
    Mother looked in the mirror of the waters.
    She smiled, and the Earth rejoiced.
  \end{translation}
\endsong


\beginsong{Gira o Mundo}[by={Cristina Tati}, tags={Santo Daime}, ph={II}, key={Dm}, sks={Cm, Cm--D\shrp{}m}]
  \audio[]{http://www.nossairmandade.com/hymn.php?hid=3667}
  \newchords{chords_giraomundo_a}\newchords{chords_giraomundo_b}
  \transpose{5} % in Dm the notes go from A to Bb'
  \beginchorus\memorize[chords_giraomundo_a]
    \[^\mn{E}]Eu vou \[^\mn{C}]mos|\[\mnc{E}Am]trar pa\[^\mn{C}]ra \[^\mn{E}]to\[^\mn{C}]do \[^\mn{F}]mun\[^\mn{E}]do |\[\mnc{D}Dm]ver. \altchords{\id[1]{(Am) \capo{3}}|Am |Dm}
    Esta Ver|\[E7]dade, Esta Força, Este Po|\[Am]der. \altchords{|E7 |Am}
  \endchorus\glueverses\beginchorus\memorize[chords_giraomundo_b]
    Estou no |\[Am]Céu, Estou na Terra, Estou no |\[E]Mar. \altchords{|Am |E}
    Quem me pro|\[E7]cura, nunca Eu ei de fal|\[Am]tar. \altchords{|E7 |Am}
  \endchorus
  \notesoff
  \beginchorus\replay[chords_giraomundo_a]
    Eu sou a |^Luz do Sol Resplande|^cente.
    Sou Eu quem |^dou o Brilho do Lu|^ar.
  \endchorus\glueverses\beginchorus\replay[chords_giraomundo_b]
    Eu sou a |^Lua, do Sol a compa|^nheira.
    Minha Be|^leza os poetas vão can|^tar.
  \endchorus
  \beginchorus\replay[chords_giraomundo_a]
    E gira o |^mundo, continua gi|^rando. \altchords{\id[2]{(Bm)}|Bm |Em}
    Sempre gi|^rou, e sempre vai gi|^rar. \altchords{|F\shrp{}7 |Bm}
  \endchorus\glueverses\beginchorus\replay[chords_giraomundo_b]
    Mas tem uns |^tantos que são cabe|^çudos. \altchords{|Bm |F\shrp{}}
    Dão muita |^volta e não saem do lu|^gar. \altchords{|F\shrp{}7 |Bm}
  \endchorus
  \beginchorus\replay[chords_giraomundo_a]
    Eu vou can|^tando, a minha vida é |^esta.
    Só me es|^cuta quem quer escu|^tar.
  \endchorus\glueverses\beginchorus\replay[chords_giraomundo_b]
    Eu quero |^ver esse povo bai|^lando.
    Que nóis só |^pode mesmo é se ale|^grar.
  \endchorus
  \begin{translation}
    I will show for the whole world to see. This Truth, This Strength, This Power.
    I'm in heaven, I'm on earth, I'm in the sea.  Who seeks me, I will never fail.
    \nextverse
    I am the Resplendent Sunlight. It is I who give the Moonlight Shine.
    I am the moon, a companion to the sun. My beauty the poets will sing.
    \nextverse
    And spin the world, keep spinning. It always has, and it always will.
    But there are a few who are stubborn. They go around a lot and don't leave the place.
    \nextverse
    I'm singing, my life is this. Only listen to me who wants to listen.
    I want to see these people dancing. What we can only really do is rejoice.
  \end{translation}
\endsong


\beginsong{Ilumina}[by={Guilherme Henrique Mendonça da Silva}, tags={Sun, Moon, stars, Aya},ph={II, III}, key={C}, sks={D, C--F\shrp{}}]
  \audio[key=D]{https://www.youtube.com/watch?v=v-xedZAX0hU}
  \beginchorus\memorize
    %% with simplified chords:
    |\[\mnc{C}C]Ó, \[\bm] \[^\mn{B}]gran\[^\mn{C}]di|\[^\mn{E}]o\[\bm]-\[^\mn{D}]so |\[\mnc{C}F]sol, \[\bm] \[^\mn{D}]sol \[^\mn{C}]cen|\[\mncii{A}{G}G]tral \[\bm] \altchords{\id[1]{(C)}|C F |C Am7 |F Dm |G G7}
    %% with full chords:
    %|\[C]Ó, \[F] grandi|\[C]o\[Am7]-so |\[F]sol, \[Dm] sol cen|\[G]tral \[G7]
  \endchorus
  \notesoff
  \beginchorus
    %% with simplified chords:
    \ind |\[C]Me \[\bm] i\[\mn{D}]lu|\[\mnc{F}F]mi\[\bm]- |\[\mnc{E}Am]na \[\bm] \[\mn{F}]i\[\mn{E}]lu|\[\mnc{D}G]mi\[\bm]- |\[\mnc{C}C]na \[\bm] \altchords{|Am Am7 |Dm G |C Am |Dm G |C C7}
    \ind ilu|\[F]mi\[\bm]- |\[Am]na \[\bm] ilu|\[G]mi\[\bm]- |\[C]na \[\bm] \altchords{|Dm G |C Am |Dm G |C \up{1}F}
    %% with full chords:
    %\ind |\[Am]Me \[Am7] ilu|\[Dm]mi\[G]- |\[C]na \[Am] ilu|\[Dm]mi\[G]- |\[C]na \[C7]
    %\ind ilu|\[Dm]mi\[G]- |\[C]na \[Am] ilu|\[Dm]mi\[G]- |\[C]na \[F]
  \endchorus
  \beginchorus
    |^Ó, ^ grandi|o^-sa |^lu^-a no |^céu ^ \goto{Me ilumina}
  \endchorus
  \beginchorus
    |^Ó, ^ grandi|o^-sa es|^tre^-la no |^céu ^ \goto{Me ilumina}
  \endchorus
  \beginchorus
    |^Ó, ^ grandi|o^-sa |^rai^-nha da |^flore^sta \goto{Me ilumina}
  \endchorus
  \begin{translation}
    Oh, great sun, central sun
    \nextverse
    \ind Enlighten me, enlighten, enlighten, enlighten
    \nextverse
    Oh, great moon in the sky\ldots
    \nextverse
    Oh, great star in the sky\ldots
    \nextverse
    Oh, great queen of the forest\ldots
  \end{translation}
\endsong


\beginsong{Luz \& Chuva de Esmeralda}[by={João Pedro G Afonso}, ph={II}, key={Am}, sks={Cm, Am--C\shrp{}m}]
  \audio[key=Dm]{https://soundcloud.com/dois-sois/luz-e-chuva-de-esmeralda}
  \audio[key=Dm]{https://open.spotify.com/track/0vQpKrzVtA7jva7AGHjhKL}
  \transpose{-5} % in Am the notes go from G to G'
  \textnote{Luz}
  \beginchorus
    \[\bmc\mnc{D}(Dm)]{A força} \[\mn{F}]che|\[\mnc{A}A]gou, essa \[\mnc{C}C]luz é Es\[\mn{B&}]me|\[\mnc{A}F]ral\[\mn{F}]da
    Vem dos \[\bm]raios das al|\[A]turas, dos rei\[A7]nos celesti|\[Dm]ais
  \endchorus\glueverses\beginchorus
    Estou cha|\[C]ma\[C7]ndo |\[F]Al\[\bm]to, eu quero |\[A]ver An\[\bm]jos des|\[Dm]cer\[\bm]
    Curando |\[C]to\[C7]{da a} consci|\[F]enc\[\bm]ia, livrando |\[A]o A\[A7]doe|\[Dm]cer\[\bm]
  \endchorus
  \beginchorus
    Eu chamo a |\[A]Cura, eu chamo os \[C]médicos do |\[F]céu
    Para liv\[\bm]rar qualquer do|\[A]ença e desfa\[A7]zer todo esse |\[Dm]véu\[\bm]
  \endchorus\glueverses\beginchorus
    Aqui es|\[C]to\[C7]{u na} Nova |\[F]E\[\bm]ra, aqui em |\[A]outra \[\bm]Dime|\[Dm]nsão
    Coman|\[C]da\[C7]{do por} Jesus |\[F]Cris\[\bm]to, na Luz da |\[A]Virgem \[A7]Concei|\[Dm]ção
  \endchorus
  \musicnote{instrumental}
  \textnote{Chuva}
  \beginchorus
    \[\mn{A}]Ó \[\bmc\mn{G}]san\[\mn{F}]ta \[\mn{E}]Mi\[\mn{D}]la|\[\mnc{C}C]grosa, \[\mn{D}]Vir\[\bmc\mn{E}]gem \[\mn{C}]da \[\mn{D}]Con\[\mn{E}]cei|\[\mnc{F}F]ção
    Tu me \[\bm]dá a Santa |\[A]Luz, eu te \[\bm]dou meu cora|\[Dm]ção
    Sal\[\bm]ve as Santas |\[C]forças da Ra\[\bm]inha da Flo|\[F]resta
    Para \[\bm]sempre aben|\[A]çoa em no\[A7]me das Santas |\[Dm]Rezas
  \endchorus
  \beginchorus
    |\[\mnc{A}Dm]Chu\[\bmc\mn{G}]va \[\mn{F}]de Es\[\mn{E}]me|\[\mnc{C}C]ral\[\bm]das
    |Cai sobre \[\bm]esse Co|\[F]ma\[A]{'n}|\[Dm]do
    \[\bm]Pétalas de |\[C]Luzes\[\bm] do Ar|\[F]ca\[A7]{njo Rafa}|\[Dm]el \[\bm]
  \endchorus
  \begin{translation}
    \definecolor{esmeraldacolor}{rgb}{0.314,0.784,0.471} % #50c878
    The strength has arrived, that light is {\color{esmeraldacolor}Emerald}
    Comes from the rays of heights, of the celestial kingdoms
    \nextverse
    I'm calling out loud, I want to see angels come down
    Healing all consciousness, ridding the sick
    \nextverse
    I call the Healing, I call the sky doctors
    To rid any disease and undo that veil
    \nextverse
    Here I am in the New Era, here in another Dimension
    Commanded by Jesus Christ, in the light of the Virgin of Conception
    \nextverse
    O miraculous saint, Virgin of Conception
    You give me the Holy Light, I give you my heart
    Save the holy forces of the Queen of the Forest
    Bless forever in the name of Holy Prayers
    \nextverse
    Rain of Emeralds
    Fall on that Command
    Light petals of Archangel Raphael
  \end{translation}
\endsong


\beginsong{Rosário da Floresta}[by={Antonio Malavoglia}, ph={II, III}, key={Am}, sks={Am, Gm--G\shrp{}m}--(Am)]
  \transpose{-2} % in Am the notes go from A to C'
  \audio[key=Am]{https://soundcloud.com/user-219518970/rosario-da-floresta-am}
  \audio[key=Bm]{https://soundcloud.com/user-55157743/18-rosario-da-floresta}
  \newchords{rosario_da_floresta_a}\newchords{rosario_da_floresta_b}
  \beginverse\memorize[rosario_da_floresta_a]
    |\[\mnc{B}Bm]Oh \[^\mn{A}]mi\[^\mn{B}]nha \[^\mn{C#}]sen|\[^\mn{B}]hora, tu \[^\mn{D}]sois \[^\mn{C#}]a \[^\mn{B}]mãe \[^\mn{A}]di|\[\mnc{F#}F#m]vina \altchords{\id[1]{(Gm)}|Gm | - |Dm}
    Que ilu|mina e guia os meus |\[Bm]passos. | \e \altchords{| - |Gm | \e}
    |Vento que sopra |forte, água da corren|\[F#m]teza \altchords{| - | - |Dm}
    E a Natu|reza da nossa mãe sa|\[Bm]grada. | | \e \altchords{| - |Gm | - | \e}
  \endverse
  \beginverse\memorize[rosario_da_floresta_b]
    \ind \[^\mn{F#}]Vem \[^\mn{G}]lá \[^\mn{F#}]das \[^\mn{E}]Es|\[\mnc{D}D]trelas Do \[^\mn{A}]Mar \[^\mn{D}]a|\[\mnc{E}A]zul do \[^\mn{A}]Infi|\[\mnciii{D}{C#}{B}Bm]nito | \e \altchords{|B\flt{} |F |Gm | \e}
    \ind És a Ver|\[G]dade, vem cá Pa|\[F#m]pai vem cá Ma|\[A]mãe. | \e \altchords{|E\flt{} |Dm |F | \e}
    \ind Brilha na Flo|\[D]resta o seu Ro|\[A]sário feito de |\[Bm]rosas | \e \altchords{|B\flt{} |F |Gm | \e}
    \ind Flores de |\[G]Luz, floriu a |\[F#m]Paz, floriu a |\[A]Calma. | \e \altchords{|E\flt{} |Dm |F | \e}
  \endverse
  \notesoff
  \beginverse\replay[rosario_da_floresta_a]
    |^Reza, Reza Ben|dita, escuta o meu pe|^dido
    De Ora|ção, pedindo Prote|^ção. | \e
    Pro|teja todos seus |filhos das dores do Ca|^minho
    Faz mere|cer o seu Amor Sa|^grado. | | \e
  \endverse
  \beginverse\replay[rosario_da_floresta_b]
    \ind Tu estás em |^tudo, vem do |^Sol, vai pra |^Lua | \e
    \ind Estou a|^qui com Ale|^gria, eu vim can|^tar. | \e
    \ind Canta Passa|^rinho, canta Ver|^dade tira men|^tira | \e
    \ind Para so|^rrir, com Ale|^gria, com Ale|^gria. | \e
  \endverse
  \begin{translation}
    Oh my lady, you are the divine mother
    Who illuminates and guides my steps.
    Strong wind, flowing water
    And the nature of our sacred mother.
    \nextverse
    Coming from the blue Starfishes of Infinity
    is the Truth, come here Father come here Mother.
    Her Rosary made of roses shines in the Forest
    Flowers of Light, blooming Peace, flowering Calm.
    \nextverse
    Pray, Pray Blessed, listen to my prayer
    request, asking for Protection.
    Protect all your children from the pain of the Way
    They deserve your Sacred Love.
    \nextverse
    You are in everything, come from the Sun, go to the Moon
    I'm here with joy, I came to sing.
    Sing Little Bird, sing Truth shred lies
    To smile, with Joy, with Joy.
  \end{translation}
\endsong


\beginsong{Rainha das Águas}[by={Adrian Freedman}, ph={II, III}, key={Bm}, sks={Bm, Am--C\shrp{}m}]
  \audio[key=Bm]{https://soundcloud.com/adrianfreedman/36-rainha-das-guas}
  \audio[key=Bm]{https://adrianfreedman.bandcamp.com/track/rainha-das-guas}
  \meter{3}{4}
  \beginchorus\memorize
    |\[^(Bm)] \[^\mn{B}]Ra|inha \[^\mn{D}]das |\[\mnc{F#}Bm]Á|guas, | Ra|inha \[^\mn{A}]do |\[^\mn{F#}]Mar | \e
    | Rain|ha das Es|\[Em]tre|las, que |\[A]brilham |na beira-|\[D]mar | \e
    | Rain|ha das Es|\[Em]tre|las, que |\[A]brilham |na beira-|\[D]mar | \e
  \endchorus
  \beginchorus
    \ind | \[\mn{D}]Ra|i\[\mn{C#}]nha \[\mn{B}]das |\[\mnc{A}A]Á|guas, | \[\mn{G}]Ra|in\[\mn{F#}]ha \[\mn{E}]do |\[\mnc{F#}D]Mar | \e
    \ind | Ra|inha de |\[Em]to|dos, |\[A]Ela é quem |vem nos cu|\[D]rar | \e
  \endchorus
  \beginverse
    \ind[2] |\[\mn{D}]Ra\[\mn{B}]in\[\mn{D}]ha|' \[\mn{B}]Ie\[\mn{D}]man|\[\mnc{B}Bm]já | \e
    \ind[2] |Rainha|' Ieman|já | \e
    \ind[2] |Rainha|' Ieman|já | \e
  \endverse
  \notesoff
  \beginchorus
    | Ra|inha das |^flo|res, | Rain|ha Divi|nal | \e
    | Rain|ha dos pri|^mo|res; |^ Branca, a|zul e crys|^tal | \e
    | Rain|ha dos pri|^mo|res; |^ Branca, a|zul e crys|^tal | \e
  \endchorus
  \goto{Rainha das Aguas \ldots vem nos curar}
  \goto{Rainha Iemanjá}
  \beginchorus
    | Ra|inha da |^Lu|a, | Rain|ha do Ast|ral | \e
    | Rain|ha das Al|^tu|ras, Ra|^inha Ce|lesti|^al | \e
    | Rain|ha Mamãe |^Pu|ra, Ra|^inha |Univer|^sal | \e
  \endchorus
  \goto{Rainha das Aguas \ldots vem nos curar}
  \goto{Rainha Iemanjá}
  \begin{translation}
    Queen of Waters, Queen of the Sea
    Queen of the Stars, that shine on the seashore
    \nextverse
    Queen of Waters, Queen of Sea
    Queen of all, She is the one who comes to heal us
    \nextverse
    Queen Iemanjá
    Queen Iemanjá
    Queen Iemanjá
    \nextverse
    Queen of flowers, Divine Queen
    Queen of perfections; white, blue and crystal
    \nextverse
    Queen of the Moon, Queen of the Astral
    Queen of Heights, Heavenly Queen
    Pure Mother Queen, Universal Queen
  \end{translation}
\endsong


\beginsong{Brilho da Verdade}[by={Chandra Lacombe}, ph={II, III}, key={Bm}, sks={Bm, (Bm)--(Cm)}]
  \audio[key=Dm]{https://www.youtube.com/watch?v=-wK2sENHmGU}
  \audio[]{https://soundcloud.com/paulmallon-2/brilho-da-verdade}
  \transpose{2} % in Bm the notes go from F# to B'
  \beginchorus\memorize
    \[^\mn{E}]Che|\[^\mn{C}]gou na \[^\mn{B}]voz do |\[\mnc{A}Am]vento, veio |\[^\mn{C}]pa\[^\mn{A}]ra e\[^\mn{C}]nun\[^\mn{A}]ci|\[\mnc{D}Dm]ar
    A torm|enta desse |\[C]tempo, e ninguém |queira duvi|\[E]dar
  \endchorus
  \beginchorus
    \ind \[\mn{E}]Quem a|inda está dor|\[\mnc{A}Am]min\[\mn{E}]do, na ilus|ão a se de\[\mn{G}]mo|\[\mnc{F}Dm]rar
    \ind Não dá |mais pra viver fu|\[C]gindo, é pre|ciso desape|\[E]gar
  \endchorus
  \notesoff
  \beginchorus
    Aqui |vai doer um |^pouco, para a |cura depu|^rar:
    Segue so|frendo orgu|^lhoso, que se re|cusa a se traba|^lhar
  \endchorus
  \beginchorus
    \ind Se a |ordem é se|^vera, é para |quem não se fir|^mou
    \ind Quando o pen|sar é como |^vela, a luz dis|sipa o te|^rror
  \endchorus
  \beginchorus
    É no co|mando de São Mi|^guel essa vi|tória aconte|^cer
    Tudo o que |não reflete |^Deus aqui não |vai mais se esten|^der
  \endchorus
  \beginchorus
    \ind Com o |brilho da ver|^dade divina es|trela vem confir|^mar
    \ind Que o a|mor e a bon|^dade nessa es|fera é que vão rei|^nar
  \endchorus
  \begin{translation}
    Arrived as the voice of the wind, came to enunciate.
    The storm of this time, and no one wants to doubt.
    \nextverse
    Who is still sleeping in the illusion to delay:
    you can no longer live on the run, you have to let go.
    \nextverse
    Here it will hurt a little for the cure to purify.
    Still suffering proud, refusing to work.
    \nextverse
    If the order is severe, it is for those who have not secured.
    When thinking is like candle, light dispels terror.
    \nextverse
    It's in Saint Michael's command that this victory happens.
    Everything that does not reflect God here will no longer extend.
    \nextverse
    With the brighness of the divine star truth comes to confirm
    that love and goodness in this sphere will reign.
  \end{translation}
\endsong


\beginsong{Aqui na Terra}[index={Here on the Earth}, by={Joseph Sulla},ex={english, português}, tags={Santo Daime}, ph={II}, key={Am}, sks={Bm, G\shrp{}m--Bm}]
  \audio{https://www.nossairmandade.com/hymn.php?hid=3814}
  \newchords{chords_aquinaterra_a}\newchords{chords_aquinaterra_b}
  \meter{3}{4}
  \beginchorus\memorize[chords_aquinaterra_a]
    |\[\mnc{C}Am]Here on \[^\mn{A}]the |\[^\mn{C}]Earth \[^\mn{A}]I |\[^\mn{C}]see so \[^\mn{A}]much |\[^\mn{C}]beauty \altchords{\id[1]{(Bm)}|Bm | - | - | \e}
    My |\[Dm]Mother she |shows it to |\[Am]me | \e \altchords{|Em | - |Bm}
  \endchorus\glueverses
  \beginchorus\memorize[chords_aquinaterra_b]
    For|\[Dm]ever I |want, for|\[Am]ever to |be \altchords{|Em | - |Bm | \e}
    With my |\[E]Mother |next to |\[Am]me | \e \altchords{|F\shrp{} | - |Bm | \e}
  \endchorus
  \notesoff
  \beginchorus\replay[chords_aquinaterra_a]
    This |^power, this |force it |flows like a |river
    Sur|^render and |flow to the |^sea | \e
  \endchorus\glueverses
  \beginchorus\replay[chords_aquinaterra_b]
    For|^ever I |want, for|^ever to |be
    With my |^Mother |next to |^me | \e
  \endchorus
  \beginchorus\replay[chords_aquinaterra_a]
    |^Rei Ja|gube e Ma|mãe Ra|inha
    From the |^forest a|cross the |^sea | \e
  \endchorus\glueverses
  \beginchorus\replay[chords_aquinaterra_b]
    For|^ever I |am and for|^ever shall |be
    With my |^Mother |next to |^me | \e
  \endchorus
  \brk
  \beginchorus\replay[chords_aquinaterra_a]
    A|^qui na |Terra eu vejo |tanta be|leza
    Minha |^Mãe ela |mostra para |^mim | \e
  \endchorus\glueverses
  \beginchorus\replay[chords_aquinaterra_b]
    Para |^sempre eu |quero, para |^sempre e|star
    Com minha |^Mãe jun|tinho a |^mim | \e
  \endchorus
  \beginchorus\replay[chords_aquinaterra_a]
    Este po|^der, esta |força flui |como um |rio
    Se en|^trega e flui |para o |^mar | \e
  \endchorus\glueverses
  \beginchorus\replay[chords_aquinaterra_b]
    Para |^sempre eu |quero, para |^sempre e|star
    Com minha |^Mãe jun|tinho a |^mim | \e
  \endchorus
  \beginchorus\replay[chords_aquinaterra_a]
    |^Rei Ja|gube e Ma|mãe Ra|inha
    Da Flo|^resta do outro |lado do |^mar | \e
  \endchorus\glueverses
  \beginchorus\replay[chords_aquinaterra_b]
    Para |^sempre e|stou e para |^sempre esta|rei
    Com minha |^Mãe jun|tinho a |^mim | \e
  \endchorus
  \begin{explanation}
    (The spirit of) \emph{Banisteriopsis Caapi} is also known as \textbf{(Rei) Jagube}.
  \end{explanation}
\endsong


\beginsong{Jardim do Universo}[by={Bettina Maureenji}, ph={II}, key={Dm}, sks={Cm--F\shrp{}m}]
  \meter{3}{4}
  \transpose{5} % in Dm the notes go from A to G'
  \newchords{chords_jardim_a}\newchords{chords_jardim_b}
  \beginchorus\memorize[chords_jardim_a]
    \[^\mn{E}]Vejo as |\[\mnc{A}Am]flores \[^\mn{B}]do \[^\mn{C}]ast|\[\mnc{D}G]ral, \altchords{\id[1]{(Am) \capo{5}}|Am |G}
    o Rei Ja|\[Am]gube e Mãe Ra|inha \altchords{|Am | \e}
  \endchorus\glueverses
  \notesoff
  \beginchorus\memorize[chords_jardim_b]
    Na cacho|\[Dm]eira a lim|\[Am]par, \altchords{|Dm |Am}
    Mamãe O|\[G]xum vem me ilumi|\[Am]nar \altchords{|G |Am}
  \endchorus
  \beginchorus\replay[chords_jardim_a]
    No cora|^ção trago a ver|^dade,
    fogo que |^arde pra me lem|brar
  \endchorus\glueverses
  \beginchorus\replay[chords_jardim_b]
    Eu chamo a |^força, eu chamo |^força,
    eu chamo a|^gora, eu chamo a|^qui
  \endchorus
  \noteson
  \beginchorus
    \ind \[\mn{A}]Neste jar|\[\mnc{F}Dm]dim do \[\mn{G}\mn{F}]uni|\[\mnc{E}Am]verso,
    \ind luz di|\[G]vina a me ilumi|\[Am]nar
    \ind Neste jar|\[Dm]dim do uni|\[Am]verso,
    \ind sempre, |\[G]sempre a te a|\[Am]mar
  \endchorus
  \notesoff
  \beginchorus\replay[chords_jardim_a]
    Vou se|^guindo o meu ca|^minho,
    vou can|^tar neste lu|gar
  \endchorus\glueverses
  \beginchorus\replay[chords_jardim_b]
    Dai-me fir|^meza, prote|^ção,
    ale|^gria e a compai|^xão
  \endchorus
  \brk
  \beginchorus\replay[chords_jardim_a]
    É beija-|^flor, é beija-|^flor
    que minha |^Mãe me entre|gou
  \endchorus\glueverses
  \beginchorus\replay[chords_jardim_b]
    Divina |^Mãe, Flor de Jas|^mim,
    vos perfu|^mai o meu jar|^dim
  \endchorus
  \goto{Neste jardim do universo}
  \goto{Divina Mãe, Flor de Jasmim}
  % Original image downloaded from: https://pxhere.com/en/photo/151450
  % Edited by: larva
  % Image license: CC0 (public domain)
  \imagecc[3]{jasmin_flower_transparent_bg_ed_by_larva_CC0_798x576px.png}%
  \begin{translation}
    I see the flowers of the astral, King Jagube and Mother Queen.
    At the waterfall to be cleaned, Mother Oxum comes to enlighten me.
    \nextverse
    In my heart I bring the truth, fire that burns to remind me.
    I call the force, I call the force, I call now, I call here.
    \nextverse
    In this garden of the universe, divine light illuminates me.
    In this garden of the universe, always, always to love you.
    \nextverse
    I'm going my way, I'm going to sing in this place.
    Give me firmness, protection, joy and compassion.
    \nextverse
    It's hummingbird, it's hummingbird which my Mother gave me.
    Divine Mother, Jasmine Flower, I have perfumed my garden.
    \nextverse
    In this garden of the universe, divine light illuminates me.
    In this garden of the universe, always, always to love you.
    \nextverse
    Divine Mother, Jasmine Flower, I have perfumed my garden.
  \end{translation}
\endsong


\beginsong{Coração do Mundo}[by={Bettina Maureenji, Amu Ahava}, tags={Mother Earth, heart}, ph={III}, key={Am}, sks={Am, Am--Bm}]
  \audio[key=Am]{https://soundcloud.com/bettinamaureenji/1-03-coracao-do-mundo}
  \meter{3}{4}
  \beginverse
    \[^\mn{G}]Meu |\[\mnc{C}C]cora|\[\mnc{B}G/B]ção \[^\mn{G}]meu |\[\mnc{C}C]cora|\[\mnc{B}G/B]ção
    |\[C]Daime |\[G/B]tua a|\[Am]ju|da
    Fa|\[C]lar co|\[G/B]migo |\[C]abre |\[G/B]ti
    |\[C]Pra dei|\[G/B]xar me en|\[Am]trar | \e
  \endverse
  \beginchorus
    \ind |\[\mnc{D}Dm]Cora|\[\mnc{E}Em]ção \[\mn{G}]do |\[\mnc{A}Am]mun|do
    \ind Em |\[F]ti que|\[Dm]ro es|\[Em]tar | \e
    \ind |\[Dm]Cora|\[Em]ção do |\[Am]mun|do
    \ind |\[F]Vamos |\[Em]a \up{1}cu|\[Am]rar | \e \altlyr[2]{bailar}
  \endchorus
  \notesoff
  \beginverse
    A|^i den|^tro de |^teu a|^mor
    En|^contro |^uma |^luz | \e
    |^Esta |^luz vem |^abra|^çar
    Me |^dar tran|^qüili|^da|de  \goto{Coração}
  \endverse
  \beginverse
    Com |^este a|^braço vou |^cami|^nhando
    Ao |^centro do |^sofri|^men|to
    Flu|^indo can|^tando le|^vando a |^luz
    Que |^nos da |^ale|^gri|a  \goto{Coração}
  \endverse
  \beginverse
    \ldots Transformando transformando\ldots
  \endverse
  \brk
  \textnote{suomeksi:}
  \beginverse
    Sy|^däme|^ni sy|^däme|^ni
    |^Sydäme|^ni auta |^minu|a
    |^Puhu |^minulle, |^avau|^du
    |^Päästä |^minut si|^sälle|si
  \endverse
  \beginchorus
    \ind |\[Dm]Maa|\[Em]ilman |\[Am]sydän | \e
    \ind Si|\[F]nussa |\[Dm]tahdon |\[Em]elää | \e
    \ind |\[Dm]Maa|\[Em]ilman |\[Am]sydän | \e
    \ind |\up{1}\[F]Paran|\[Em]tukaam|\[Am]me | \e \altlyr[2]{tanssikaamme}
  \endchorus
  \beginverse
    Si|^ellä |^rakkautes |^sisäl|^lä
    |^Ko-|^ohtaan |^valon | \e
    |^Valo |^syleilee |^minu|^a
    ja |^sä|^te-eilee |^rauhaa | \e  \goto{Maailman}
  \endverse
  \beginverse
    Sy|^leilys |^kanssa mä |^kul|^jen
    Kär|^simyk|^se-en kes|^kellä | \e
    |^Lentäen |^laulaen |^seuraten |^valoa
    |^joka |^tuo meille |^ilo|a  \goto{Maailman}
  \endverse
\endsong


\beginsong{Luz de Vida}[by={Amu Ahava},tags={source, light}, ph={II, III}, key={Dm}, sks={Dm, Dm--Em}]
  \audio[key=Dm]{https://www.youtube.com/watch?v=9gzYnEuHNMc}
  \audio[key=Fm]{https://soundcloud.com/astral-flowers-music/luz-de-vida-demo-composer-amu-ahava}
  \meter{6}{8}
  \beginchorus
    \ind |\[\mnc{D}Dm]Luz \[\mnc{F}Dm/F]de |\[\mnc{E}Cadd9/E]Vi\[\mnc{C}Cadd9]da
    \ind |\[Dm]Luz \[Dm/F]Di|\[Cadd9/E]vi\[Cadd9]na
    \ind |\[Dm]Sem\[Dm/F]pre e|\[Cadd9/E]sta \[Cadd9]
    \ind a|\[Gm6/B&]qui |\[Gm6] pra |\up{1}\[Asus4]mim \up{2}(ti)|\[A]
  \endchorus
  \beginverse\memorize
    |\[\mnc{A}Dm]Fon\[\mnc{F}Dm/F]te \[^\mn{G}]da \[^\mn{A}]con|\[\mnc{G}Cadd9/E]cien\[^\mn{E}]çia \[Cadd9]
    |\[Dm]Fon\[Dm/F]te do A|\[Cadd9/E]mor \[Cadd9]
    |\[Dm]Sem\[Dm/F]pre estou |\[Cadd9/E]livre \[Cadd9]
    Porque |\[Dm]Deus \[Dm/F]vibra em |\[Cadd9/E]mim \[Cadd9]
    Em |\[Gm6/B&]mim, |\[Gm6] em |\[Asus4]ti |\[A]
  \endverse
  \goto{Luz de Vida}
  \notesoff
  \beginverse
    No si|^lencio ^com humil|^dade ^
    O mis|^tério se ^revela a |^mim ^
    |^Toda essa ^reali|^dade ^
    Cria|^ção de um ^sonho |^meu ^
    |^Meu, |^ |^teu |^
  \endverse
  \goto{Luz de Vida}
  \noteson
  \beginchorus
    |\[\mnc{G}Gm]Se\[\mn{B&}]re-|\[\mn{D}]e\[\mn{F}]ni|\[\mncii{E}{A}Asus4]da|\[A]de\ldots
  \endchorus
  \begin{translation}
    Light of Life, Divine light
    Always it is here for me, \up{2}for you
    \nextverse
    Source of consciousness, source of Love
    I'm always free because God vibrates in me, in you
    \nextverse
    In silence with humility the mystery reveals itself to me
    All this reality creation of a dream of mine, of yours
    \nextverse
    Serenity\ldots
  \end{translation}
\endsong


\beginsong{Luz Amor e Paz}[by={Chandra Lacombe, Fernando Beltran}, tags={love, light, peace}, ph={III}, key={G}, sks={D\shrp{}--A}]
  \audio[key=C]{https://soundcloud.com/ariella-marques/chandra-e-carioca-luz-amor-e}
  \audio[key=C]{https://www.youtube.com/watch?v=EQKun-2b4XE}
  \transpose{-5} % in C the notes go from E to D', in G from B to A
  \mnbeginchorus\memorize % memorize chords, even though inside 'chorus'
    \[^\mn{G}]Vem |\[\mnc{F}Dm]Luz me ilu|\[\mnc{E}Am]mina sonho dou|\[\mnc{A}F]rado \[^\mnc{G}]do \[^\mn{F}]as|\[\mnc{E}Am]tral
    \[^\mn{A}]Me \[^\mn{B}]con|\[\mnc{C}F]duz e \[^\mn{B}]me \[^\mn{A}]en|\[\mnc{B}G]sina o \[^\mn{C}]teu |\[\mnc{D}G7sus4]brilho é o \[\mnc{C}G7]meu \[^\mn{B}]si|\[\mnc{C}C]nal
  \mnendchorus
  \notesoff
  \beginchorus
    Vem A|^mor me faz vi|^ver com tua |^força divi|^nal
    Eu sou |^filho desse po|^der meu cora|^ção é o ^meu si|^nal
  \endchorus
  \beginchorus
    Vem |^Paz vem |^Paz tua |^vinda é natu|^ral
    Bem |^que o amor me |^traz meu si|^léncio é ^teu si|^nal
  \endchorus
  \begin{translation}
    Come, Light, illuminate me, golden astral dream.
    Guide me and teach me, your brilliance is my sign.
    \nextverse
    Come, Love, with your divine force make me live.
    I am a child of this power, my heart is my sign.
    \nextverse
    Come, Peace, come, Peace, your coming is natural.
    So you bring me love, my silence is your sign.
  \end{translation}
\endsong


\beginsong{Pachamama}[index={Os Segredos Vem da Floresta}, ph={II, III}, key={Bm}, sks={Bm, Gm--D\shrp{}m}]
  \audio[key=Em]{https://www.youtube.com/watch?v=5hVHVYdifRs}
  \newchords{chords_ossegredos_a}\newchords{chords_ossegredos_b}
  \meter{6}{8}
  \beginchorus\memorize[chords_ossegredos_a]
    \[^\mn{B}]Os se|\[\mnc{E}Em]gre\[^\mn{F#}]dos \[^\mn{G}]vem da \[^\mn{F#}]flo|\[\mnc{D}G]res\[^\mn{B}]ta de luz
    Pacha|\[Bm]mama, Pacha|\[Em]mama
  \endchorus\glueverses
  \notesoff
  \beginchorus\memorize[chords_ossegredos_b]
    |\[D]Abre a consci|\[Em]ência
    Dos seus |\[G]filhos a cres|\[Bm]cer
  \endchorus
  \beginchorus\replay[chords_ossegredos_a]
    A ver|^dade traz rea|^leza aqui
    Ali|^menta o cora|^ção
  \endchorus\glueverses
  \beginchorus\replay[chords_ossegredos_b]
    Dissol|^vendo as tor|^mentas
    Deste |^mundo de ilu|^são
  \endchorus
  \beginchorus\replay[chords_ossegredos_a]
    Renas|^cendo das cinzas da his|^tória
    Mãe da |^sua forta|^leza
  \endchorus\glueverses
  \beginchorus\replay[chords_ossegredos_b]
    As vir|^tudes clareiam o cris|^tal
    Prima |^graça tão bri|^lhante
  \endchorus
  \beginchorus\replay[chords_ossegredos_a]
    Pacha|^mama abraça seus |^filhos
    Na jor|^nada do a|^mor
  \endchorus\glueverses
  \beginchorus\replay[chords_ossegredos_b]
    Da |^sua fonte crista|^lina
    Correm |^asas de esplen|^dor
  \endchorus
  \begin{translation}
    Secrets come from the forest of light:
    Pachamama, Pachamama
    Open the awareness
    Of your children so it grows
    \nextverse
    Truth brings nobility here
    Nourishing the heart
    Dissolving all the torments
    Of this illusory world
    \nextverse
    Being reborn from the ashes of history
    Mother of your strength
    Virtues brighten the crystal
    Shining perfect grace
    \nextverse
    Pachamama embraces her children
    On this journey of Love
    From your spring of crystalline water
    Flowing glorious wings
  \end{translation}
  % Original image downloaded from: https://imgbin.com/png/76Nznkqy/crystal-drawing-mineral-quartz-amethyst-png
  % Colorized
  % Image license: Free for non-commercial use
  \imagerb[3]{amethyst_drawing_transparent_bg_472x470px.png}
\endsong


\beginsong{Iemanjá}[by={Léo Artése},tags={Santo Daime, sea},ph={III}]
  \audio[]{http://www.nossairmandade.com/hymn.php?hid=4249}
  \newchords{chords_iemanja_a}\newchords{chords_iemanja_b}
  \beginchorus\memorize[chords_iemanja_a]
    \[^\mn{G}]Lu|\[\mnc{A}Am]ar \[^\mn{G}]se |\[^\mn{A}]fez \[^\mn{G}]um |\[^\mn{A}]rai\[^\mn{B}]o \[^\mn{A}]pra\[^\mn{G}]te|\[\mnc{E}C]a\[^\mn{C}]do
    Ilumi|\[Dm]nando o |céu e as |\[G]espumas do |\[C]Mar \up{2}( | \e)
  \endchorus\glueverses
  \beginchorus\memorize[chords_iemanja_b]
    Lindo cla|\[Dm]rão à |\[Am]beira-|\[Em]mar | \e
    Vejo Ma|\[Dm]mãe |\[F]Ieman|\[C]já | \e
  \endchorus
  \notesoff
  \beginchorus\replay[chords_iemanja_a]
    Lá |^vem, lá |vem jun|to com suas se|^reias
    Nos a|^benço|ar Ra|^inha Ieman|^já \up{2}( | \e)
  \endchorus\glueverses
  \beginchorus\replay[chords_iemanja_b]
    Dona das |^águas |^tu és |^Mãe | \e
    Oh! Jana|^ina |^Odoi|^á | \e
  \endchorus
  \beginchorus\replay[chords_iemanja_a]
    I|^lumi|nai min|has profundas |^águas
    Para eu |^deci|frar mis|^térios de meu |^mar \up{2}( | \e)
  \endchorus\glueverses
  \beginchorus\replay[chords_iemanja_b]
    Desse meu |^mar de |^emo|^ções | \e
    Rainha |^vem i|^lumi|^nar | \e
  \endchorus
  \beginchorus\replay[chords_iemanja_a]
    I|^eman|já prin|cipio gera|^dor
    Amor fun|^damen|tal tão |^puro e mater|^nal \up{2}( | \e)
  \endchorus\glueverses
  \beginchorus\replay[chords_iemanja_b]
    Ieman|^já vem |^confor|^tar | \e
    Oh! Jana|^ina |^Odoi|^á | \e
  \endchorus
  \begin{translation}
    The moonlight shines a silver beam
    Illuminating the sky and the froths of the sea
    Beautiful moonlight on the sea shore
    I see Mama Iemanjá
    \nextverse
    She comes, she comes with her mermaids
    To bless us, Queen Iemanjá
    Mistress of the waters, you are the Mother
    Oh! Janaina Odoiá
    \nextverse
    Illuminate my deep waters
    So I can decipher the mysteries of my sea
    Of my ocean of emotions
    The Queen comes to illuminate
    \nextverse
    Iemanjá, generating principle
    Fundamental love so pure and maternal
    Iemanjá comes to comfort
    Oh! Janaina Odoiá
  \end{translation}
\endsong


\scleardpage
\beginsong{Borboleta}[by={Aliança Arco Íris},ph={III}]
  \audio[key=Bm]{https://www.youtube.com/watch?v=6LwjPeqtTXE}
  \transpose{-2}
  %\capo{2} % doesn't fit here!
  \mnbeginverse
    \ind |\[G] \[\mn{D}]Está na |\[\mnc{E}A]ho\[\mn{D}]ra \[\mn{C#}]de \[\mn{D}]rom|\[\mnc{B}Bm]per | \e
    \ind |\[G]\ind \[\mn{B}]O ca|\[\mnc{A}A]su\[\mn{E}]lo que há em \[\mn{G}]vo|\[\mnc{F#}Bm]cê | \e
    \ind |\[G] \[\mn{B}]Está na |\[\mnc{C#}A]ho\[\mn{B}]ra \[\mn{A}]de \[\mn{G}]vo|\[\mnc{F#}Bm]ar | \e
    \ind |\[G] \[\mn{D}]Ser borbo|\[\mnc{E}A]le\[\mn{C#}]ta e asas \[\mn{D}]ba|\[\mnc{B}Bm]ter | \e
    \ind |\[G] \[\mn{B}]Está na |\[\mnc{C#}A]ho\[\mn{B}]ra \[\mn{A}]de \[\mn{G}]vo|\[\mnc{F#}Bm]ar | \e
    \ind |\[G] \[\mn{D}]Bater suas a|\[\mnc{E}A]sas \[\mn{C#}]{e borbole}\[\mn{D}]ta |\[\mnc{B}Bm]ser | \e
  \mnendverse
  \mnbeginverse
    \[\mn{B}]Bor\[\mn{D}]bo|\[\mnc{B}G]leta | \sublyrpush{borbo}|\sublyrpush{leta}{ } | \e
    Po\[\mn{D}]des \[\mn{C#}]vo|\[\mnc{B}Bm]ar | \sublyrpush{podes vo}|\sublyrpush{ar}{ }{ }{ } | \e
    \[\mn{D}]{É seu} mereci|\[\mnc{E}Em]men\[\mn{G}]to | \sublyrpush{é seu mereci}|\sublyrpush{mento}{ } | \e
    \[\mn{B}]Por saber e\[\mn{C#}]spe|\[\mnc{B}Bm]rar | \sublyrpush{por saber espe}|\sublyrpush{rar}{ }{ }{ } | \e
    \[\mn{F#}]Antes \[\mn{E}\mn{D}]eras |\[\mnc{E}Em]larv\[\mn{G}]a | \sublyrpush{antes eras} |\sublyrpush{larva}{ } | \e
    \[\mn{B}]{A ras}\[\mn{D}]te|\[\mnc{B}Bm]jar | \sublyrpush{a raste}|\sublyrpush{jar}{ }{ }{ } | \e
    \[\mn{F#}]Hoje és \[\mn{E}]bor\[\mn{D}]bo|\[\mncii{E}{G}Em]leta | \sublyrpush{hoje és borbo}|\sublyrpush{leta}{ } | \e
    \[\mn{B}]A \[\mn{D}]vo|\[\mnc{B}Bm]ar! | { }\sublyrpush{a vo}|\sublyrpush{ar!}{ }{ }{ } | \e
  \mnendverse
  \notesoff
  \beginchorus\memorize
    \ind |\[G] Está na |\[A]hora de rei|\[Bm]nar \sublyrpush{está na} |\sublyr{hora de rei-} \e\replay
    \ind |\sublyr{nar}^ Os pensa|^mentos que há em vo|^cê \sublyrpush{os pensa}|\sublyr{mentos que há em vo-} \e\replay
    \ind |\sublyr{cê}^ Está na |^hora de impe|^rar \sublyrpush{está na} |\sublyr{hora de impe-} \e\replay
    \ind |\sublyr{rar}^ O grande |^reino que é seu |^ser \sublyrpush{o grande} |\sublyr{reino que é seu} \e\replay
    \ind |\sublyr{ser}^ Está na |^hora de impe|^rar \sublyrpush{está na} |\sublyr{hora de impe-} \e\replay
    \ind |\sublyr{rar}^ O grande |^reino que és vo|^cê \sublyrpush{o grande} |\sublyr{reino} \e
    \vspace{1em}
    \sublyr{ que és}Bor\sublyr{  vo-}bo|\sublyr{cê}\[G]leta | \sublyrpush{borbo}|\sublyrpush{leta}{ } | \e
    Podes vo|\[Bm]ar | \sublyrpush{podes vo}|\sublyrpush{ar}{ }{ }{ } | \e
    É seu mereci|\[Em]mento | \sublyrpush{é seu mereci}|\sublyrpush{mento}{ } | \e
    Por saber espe|\[Bm]rar | \sublyrpush{por saber espe}|\sublyrpush{rar}{ }{ }{ } | \e
    Antes eras |\[Em]larva | \sublyrpush{antes eras} |\sublyrpush{larva}{ } | \e
    A raste|\[Bm]jar | \sublyrpush{a raste}|\sublyrpush{jar}{ }{ }{ } | \e
    Hoje és borbo|\[Em]leta | \sublyrpush{hoje és borbo}|\sublyrpush{leta}{ } | \e
    A impe|\[Bm]rar! | \sublyrpush{a impe}|\sublyrpush{rar!}{ }{ }{ } | \e
  \endchorus
  \capo{2}
  \begin{translation}
    It's time to break up
    The cocoon in you
    It's time to fly
    Be butterfly and flapping wings
    It's time to fly
    Flap your wings and butterfly be
    \nextverse
    Butterfly, you can fly
    It is your merit for knowing how to wait
    Before you were crawling larva
    Today you are a butterfly flying!
    \nextverse
    It's time to reign
    The thoughts in you
    It's time to rule
    The great kingdom that is your being
    It's time to rule
    The great kingdom that you are
    \nextverse
    Butterfly, you can fly
    It is your merit for knowing how to wait
    Before you were crawling larva
    Today you are the reigning butterfly! 
  \end{translation}
\endsong


\sclearpage
\begin{intersong}
  % Original image downloaded from: https://www.publicdomainpictures.net/en/view-image.php?image=279229&picture=stripe-butterfly
  % Edited by: larva (just cropped and set the transparent background)
  % Image license: Public Domain
  \imagecc[2]{stripe-butterfly_transparent_bg_PD_1200x1150px.png}%
\end{intersong}


\beginsong{Cura do Beija-flor}[by={Chandra Lacombe},ph={III, IV}]
  \audio[]{https://www.youtube.com/watch?v=9Vatao6HRJU}
  \audio[]{https://soundcloud.com/vasques22/cura-do-beija-flor}
  \newchords{chords_cura_do_bf_a}\newchords{chords_cura_do_bf_b}
  \transpose{5}
  \beginverse\memorize[chords_cura_do_bf_a]
    \[^\mn{G}]Cora|\[G]ção que vai se a|\[\mnc{F#}D]brindo em mil |\[\mnc{G}Em]péta\[^\mn{A}]las \[^\mn{B}]de |\[\mnc{E}C]flor
    eu te |\[Am]sinto, eu te re|\[D7]cebo Beija-|\[G]flor | \e
  \endverse
  \notesoff
  \beginverse\replay[chords_cura_do_bf_a]
    Nas a|^sas desta pu|^reza tem um |^brilho encanta|^dor
    pode |^ver aquele |^que já se entre|^gou | \e
  \endverse
  \noteson
  \beginverse\memorize[chords_cura_do_bf_b]
    \ind \[^\mn{B}]Vem sur|\[G]gindo um a|\[\mnc{A}D]migo no mo|\[\mnc{C}Am]mento \[^\mn{D}]es\[^\mn{C}]col|\[\mnc{B}Em]hido
    \ind pelo |\[C]mestre que re|\[D7]tira o te|\[G]mor | \e
  \endverse
  \notesoff
  \beginverse\replay[chords_cura_do_bf_b]
    \ind Um pre|^sente tão di|^vino, fruto |^de puro ca|^rinho
    \ind que re|^vela os te|^souros do a|^mor | \e
  \endverse
  \begin{translation}
    Heart that begins to open in a thousand flower petals;
    I feel you, I receive you, Hummingbird.
    \nextverse
    In the wings of this purity he has a charming glow;
    those who surrender themselves can see it.
    \nextverse
    A friend is coming by in the chosen moment,
    through the master that takes fear away.
    \nextverse
    Such a divine gift, fruit of pure tenderness,
    which reveals the treasures of love.
  \end{translation}
  \textnote{suomeksi:} % by: Outi
  \beginverse\replay[chords_cura_do_bf_a]
    Sydä|^meni aukea|^massa tuhan|^nessa kukas|^sa;
    Sinut |^vastaanottaa |^tahdon Koli|^bri | \e
  \endverse
  \beginverse\replay[chords_cura_do_bf_a]
    Näill' |^siivill' kirkka|^uden olet |^valo loista|^va;
    % edited line by larva:
    Sen voi |^nähdä kun vain |^sille antau|^tuu | \e
    % original line by outi:
    % Nähdä |^pystyvät he |^ken sen ymmär|^tää | \e
  \endverse
  \beginverse\replay[chords_cura_do_bf_b]
    \ind Ystä|^vä on saapu|^massa hetkel|^lä luva|^tulla:
    \ind Mesta|^ri joka |^pelon karkoit|^taa | \e
  \endverse
  \beginverse\replay[chords_cura_do_bf_b]
    \ind Tämä |^lahja juma|^lainen, hedel|^mä hellyyden |^aidon,
    \ind Joka |^paljastaa aar|^tehet rakkau|^den | \e
  \endverse
\endsong


\beginsong{Voo do Beija-flor}[by={Elisa Cristal},ph={II, III, IV}]
  \transpose{1}
  \beginchorus\memorize
    \[\bmc\mnc{E}]Vôo silenci|\[\mnc{D#}B]oso do mis\[^\mn{E}]tério \[^\mn{D#}]do a|\[\mnc{C#}C#m]mor
    Fecho os |\[A]olhos para ver aonde |\[E]vou
  \endchorus\glueverses
  \notesoff
  \beginchorus
    Vo^ar pelo infi|^nito daquilo que eu |^sou
    \up{1}desven|^dar o oceano interi|^or \altlyr[2]{mergulhar no}
  \endchorus
  \beginchorus
    \ind Beija-|\[B]flor me |\[C#m]le|\[G#m]va |\[E] \e \replay
    \ind Beija-|\[B]flor des|\[C#m]per|\[A]ta (em) |\[E](mim)
  \endchorus
  \beginchorus
    Me ^leva nas |^águas deste rio encanta|^dor
    Vale dou|^rado do meu lindo beija-|^flor
  \endchorus\glueverses
  \beginchorus
    Vo^ar neste a|^zul, o sol a se |^pôr
    Vento su|^ave me traz o fres|^cor
  \endchorus
  \goto{Beija-flor}
  \beginchorus
    Beija su^ave e faz a|^brir todas as pétalas desta |^flor
    Brilho da |^mata que incendia o busca|^dor
  \endchorus\glueverses
  \beginchorus
    Passa^rinho que me en|^canta, canta \[\bm]o canto do a|^mor
    Me |^leva para onde você |^for
  \endchorus
  \goto{Beija-flor}
  \begin{translation}
    Silent flight of the mystery of love
    I close my eyes to see where I am going
    \nextverse
    Flying through the infinite of what I am
    unraveling \up{2}(diving in) the inner ocean
    \nextverse
    \ind Hummingbird takes me
    \ind Hummingbird awakens (in me)
    \nextverse
    Take me in the waters of this lovely river
    Golden valley of my beautiful hummingbird
    \nextverse
    Fly in this blue, the sun to set
    Gentle wind brings me the freshness
    \nextverse
    Soft kiss and all the petals of this flower open
    Brightness of the forest that sets the seeker on fire
    \nextverse
    Bird that I love, sing the song of love
    Take me where you are
  \end{translation}
  % Original image downloaded from: https://commons.wikimedia.org/wiki/File:Hummingbird_(PSF).png
  % Edited by: larva (just cropped and set the transparent background)
  % Image license: Public Domain
  \imagecc[2]{hummingbird_drawing_bw_transparent_bg_PD_991x711px.png}%
\endsong


\beginsong{Sou Beija-flor}[by={Cristina Tati},tags={Santo Daime},ph={II, III}]
  \audio[]{http://www.nossairmandade.com/hymn.php?hid=3664}
  \newchords{chords_soubeijaflor_a}\newchords{chords_soubeijaflor_b}
  \beginchorus\memorize[chords_soubeijaflor_a]
    \[^\mn{E}]Sou canta|\[\mnc{A}Am]dor; minha \[E7]vida é can|\[Am]tar
  \endchorus\glueverses
  \beginchorus\memorize[chords_soubeijaflor_b]
    Vamos meus ir|\[Dm]mãos; \[G] va\[E7]mos can|\[Am]tar
  \endchorus
  \notesoff
  \beginchorus\replay[chords_soubeijaflor_a]
    Sou Beija-|^flor; minha ^vida é vo|^ar
  \endchorus\glueverses
  \beginchorus\replay[chords_soubeijaflor_b]
    E beijar as |^flores ^ ^e vo|^ar
  \endchorus
  \beginchorus\replay[chords_soubeijaflor_a]
    Sou reza|^dor; minha ^vida é re|^zar
  \endchorus\glueverses
  \beginchorus\replay[chords_soubeijaflor_b]
    E pedir a |^Deus pa^ra nos ^aju|^dar
  \endchorus
  \textnote{in English:}
  \beginchorus\replay[chords_soubeijaflor_a]
    I am a |^singer and my ^life is a |^song
  \endchorus\glueverses
  \beginchorus\replay[chords_soubeijaflor_b]
    Come brothers and |^sisters, ^ ^let us |^sing
  \endchorus
  \beginchorus\replay[chords_soubeijaflor_a]
    I am Humming|^bird and my ^life is a |^flight
  \endchorus\glueverses
  \beginchorus\replay[chords_soubeijaflor_b]
    And I kiss the |^flowers ^ ^and I |^fly
  \endchorus
  \beginchorus\replay[chords_soubeijaflor_a]
    I am |^praying and my ^life is a |^prayer
  \endchorus\glueverses
  \beginchorus\replay[chords_soubeijaflor_b]
    And I pray to |^God ^ ^to bless |^us
  \endchorus
\endsong


\beginsong{Cura}[by={Rainer Scheurenbrand},tags={learning, sea},ph={III}]
  \audio[]{https://rainerscheurenbrand.bandcamp.com/track/cura}
  \audio[]{https://soundcloud.com/jee-red/rainer-scheurenbrand-cura}
  % Original in Dm, so: \capo{4}
  \newchords{chords_cura_a}\newchords{chords_cura_b}
  \meter{3}{4}
  \beginchorus\memorize[chords_cura_a]
    \[^\mn{A}]E|\[Am]stou ap\[^\mn{B}]ren|\[\mnc{C}F]den\[^\mn{A}]do a |\[\mnc{B}G]me \[^\mn{G}]cu|\[\mnc{A}Am]rar
  \endchorus\glueverses\notesoff\beginchorus\memorize[chords_cura_b]
    Eu |\[Dm]peço con|\[Am]forto das |\[Dm]Águas do |\[Am]Mar
    Eu |\[Dm]peço con|\[G]forto da |\[E7]Mãe Yeman|\[Am]já
  \endchorus\glueverses\beginchorus\replay[chords_cura_a]
    |^ |^ |^ |^
  \endchorus
  \beginchorus\replay[chords_cura_a]
    E|^stou apren|^dendo dei|^xar a ilu|^são
  \endchorus\glueverses\beginchorus\replay[chords_cura_b]
    Eu |^peço a cla|^reza das |^Águas do |^Mar
    Eu |^peço a cla|^reza da |^Mãe Yeman|^já
  \endchorus\glueverses\beginchorus\replay[chords_cura_a]
    |^ |^ |^ |^
  \endchorus
  \beginchorus\replay[chords_cura_a]
    E|^stou apren|^dendo a |^me fir|^mar
  \endchorus\glueverses\beginchorus\replay[chords_cura_b]
    Eu |^peço a fir|^meza das |^Águas do |^Mar
    Eu |^peço a |^força da |^Mãe Yeman|^já
  \endchorus\glueverses\beginchorus\replay[chords_cura_a]
    |^ |^ |^ |^
  \endchorus
  \begin{translation}
    I am learning to heal myself
    I ask for comfort from the Waters of the Sea
    I ask for the comfort of Mother Yemanja
    \nextverse
    I am learning to let go of the illusion
    I ask for clarity from the Waters of the Sea
    I ask for clarity from Mother Yemanja
    \nextverse
    I am learning to secure myself
    I ask for strength from the Waters of the Sea
    I ask for strength from Mother Yemanja
  \end{translation}
\endsong


\beginsong{Passarinho}[by={Rainer Scheurenbrand},ph={III}]
  \audio[]{https://www.youtube.com/watch?v=4mTqEPQDlls}
  \newchords{chords_passarinho_a}\newchords{chords_passarinho_b}
  \beginchorus\memorize[chords_passarinho_a]
    \[^\mn{A}]Es\[^\mn{C}]tou \[^\mn{B}]a|\[\mnc{A}Am]qui, a\[^\mn{E}]qui com a min\[^\mn{D}]ha |\[\mnc{B}Em]dor
    No mundo sofre|dor, dentro da ilu|\[Am]são
  \endchorus
  \notesoff
  \beginchorus\replay[chords_passarinho_a]
    Estou a|^qui, un dia vou sa|^ir
    Un dia vou su|bir na estrela do a|^mor
  \endchorus
  \beginverse\memorize[chords_passarinho_b]
    \ind Estou aqu|\[Dm]i para aprend|\[Am]er
    \ind o mist|\[Em]ério do meu s|\[Am]er
    \ind Estou aqu|\[Dm]i para cur|\[Am]ar
    \ind Estou a|\[Em]berto\ldots | para camin|\[Am]har
  \endverse
  \beginchorus\replay[chords_passarinho_a]
    Sha la la la la |^la Sha la la la la
    |^la Sha la la la la |lai la lai lai |^la
  \endchorus
  \beginchorus\replay[chords_passarinho_a]
    Estou a|^qui, uma voz me fa|^lou
    Olha aqui es|ta a estrela do a|^mor
  \endchorus
  \beginchorus\replay[chords_passarinho_a]
    Estou a|^qui, vim para acor|^dar
    Vim para clare|ar, o mundo de ilu|^são
  \endchorus
  \beginverse\replay[chords_passarinho_b]
    \ind Meu passar|^inho, meu cantad|^or
    \ind Me l|^eva na estrela do am|^or
    \ind Meu passar|^inho, meu profess|^or
    \ind Me en|^sina\ldots | o canto do a|^mor
  \endverse
  \goto{Sha la la la la}
  \goto{Meu passarinho}
  \beginchorus\replay[chords_passarinho_a]
    \musicnote{outro, repeat and fade out:}
    |\[Am] Estas aqui para apren|\[F]der
    O mis|\[G]tério do seu |\[Am]ser
  \endchorus
  \begin{translation}
    I'm here, here with my pain, in the suffering world, within the illusion.
    \nextverse
    I'm here, I'm going out one day. One day I'll rise to the star of love.
    \nextverse
    I'm here to learn the mystery of my being.
    I'm here to heal. I'm open\ldots to the little bird.
    \nextverse
    Sha la la la la la\ldots
    \nextverse
    I'm here, a voice told me: ``look, here is the star of love.''
    \nextverse
    I'm here, I came to wake up. I came to clear the world of illusion.
    \nextverse
    My little bird, my singer, take me to the star of love.
    My little bird, my teacher, teach me\ldots the song of love.
    \nextverse
    Sha la la la la la\ldots
    \nextverse
    You're here to learn the mystery of your being.
  \end{translation}
\endsong


\beginsong{Valor do Pedido}[by={Rainer Scheurenbrand},ph={IV}]
  \audio[]{https://rainerscheurenbrand.bandcamp.com/track/valor-do-pedido}
  \audio[]{https://www.youtube.com/watch?v=aapo3wVFhk8}
  %\capo{4} % original is in C#m
  \meter{3}{4}
  \beginchorus\memorize
    |\[\mnc{C}Am]Vamos |\[\mnc{B}Em]todos \[^\mn{A}]fes\[^\mn{G}]te|\[\mnc{A}Am]jar | \e
    Com |\[Dm]todos |seres di|\[Am]vi|nos
    Com |\[Dm]todos |seres di|\[Am]vinos com a|mor
    Lá no |\[F]alto |\[E]do as|\[Am]tral | \e
    Lá no |\[F]alto |\[E]do as|\[Am]tral | \e
  \endchorus
  \notesoff
  \beginchorus
    Es|^tamos |^dando as |^mã|os
    Para |^todos |os ir|^mã|os
    Para |^todos |os ir|^mãos com a|mor
    Neste |^mundo |^e no |^out|ro
    Neste |^mundo |^e no |^out|ro
  \endchorus
  \beginchorus
    Es|^tamos |^agrade|^cen|do
    Ao |^sol, a |lua e as es|^tre|las
    Ao |^sol, a |lua e as es|^trelas com a|mor
    Para ilumi|^nar |^o meu ca|^mi|nho
    Para ilumi|^nar |^o meu ca|^mi|nho
  \endchorus
  % instrumental here
  \beginchorus
    Meu |^Pai tem |^tudo o que eu pre|^ci|so
    Quero |^nunca |mais esque|^cer | \e
    Quero |^nunca |mais esque|^cer o cora|ção
    O va|^lor |^do pe|^di|do
    O va|^lor |^do pe|^di|do
  \endchorus
  \begin{translation}
    Let us all celebrate
    With all divine beings
    With all divine beings with love
    There in good mood
    There in good mood
    \nextverse
    We are holding hands
    For all the brothers
    To all the brothers with love
    In this world and the other
    In this world and the other
    \nextverse
    We are thanking
    In the sun, the moon and the stars
    In the sun, the moon and the stars with love
    To light my way
    To light my way
    \nextverse
    My Father has everything I need
    I want never to forget
    I want never to forget the heart
    The value of request
    The value of request
  \end{translation}
\endsong


\beginsong{Minha Mãe Oxumaré}[ph={II, III}]
  \audio[key=Fm]{https://soundcloud.com/arkana_music/minha-mae-oxumare}
  \audio[key=Fm]{https://www.youtube.com/watch?v=FzRhzoazJ2g}
  \audio[key=Am]{https://www.youtube.com/watch?v=F7mP7yp8BVA}
  \transpose{-5}
  \capo{1}
  \beginchorus
    \[\mnc{A}]Se |\[Am]min\[\mn{B}]ha \[\mnc{C}]mãe \[\mn{B}]é \[\mn{A}]O|\[\mn{B}\mn{A}]xum
    Na Um|\[Dm]banda e no Candom|\[Am]blé
    \lrep Aye|\[Dm]yeo ayeyeo, aye|\[Am]yeo ayeyeo
    Minha |\[C]mãe O\[E7]xuma|\[Am]ré \[\up{1}(A7)] \rrep
  \endchorus
  \beginchorus
    \[\mn{A}]Ela |\[\mnc{G}G]vem \[\mn{B}]bei\[\mn{D}]ran\[\mn{B}]do o |\[Am]rí\[\mn{A}]o
    Colhendo |\[Dm]lirios pra nos ofer|\[Am]tar
    \lrep Mamãe O|\[Dm]xum, ayeyeo
    Mamãe O|\[Am]xum, Orixá
    Desce |\[C]venha nos a\[E7]benço|\[Am]ar \[\up{1}(A7)] \rrep
  \endchorus
  \begin{translation}
    My mother is Oxum
    In Umbanda and Candomblé
    Ayeyeo, ayeyeo, ayeyeo, ayeyeo
    My mother Oxumaré
    \nextverse
    She comes along the river
    Gathering lilies to offer to us
    Mother Oxum, ayeyeo
    Mother Oxum, Orixá
    She comes down to bless us
  \end{translation}
\endsong


\beginsong{Cachoeira de Oxum}[ph={III}]
  \audio[]{https://soundcloud.com/flormosura/cachoeira-de-oxum}
  \beginchorus
    \[\mn{E}]Na cacho|\[\mnc{A}Am]eira de \[\mn{B}]ma\[\mnc{C}Em]mãe \[\mn{D}]O|\[\mnc{E}Am]xum
    Corre agua crista|\[Dm]lina
    No tem\[E]plo pai Olo|\[Am]rum
  \endchorus
  \beginverse
    Mamãe O|\[Dm]xum
    Fez a cacho|\[Am]eira |\[Dm] \e
    Pai Olo\[E7]rum abenco|\[Am]ou
  \endverse
  \beginchorus
    Eu vou pe|\[Dm]dir permissao a Oxa|\[Am]lá
    Pra banhar na cacho|\[E7]eira
    Para todo mal le|\[Am]var
  \endchorus
  \begin{translation}
    In mother Oxum's waterfall
    Cristalline water runs
    In the temple of father Olorum
    \nextverse
    Mother Oxum
    Made the waterfall
    Father Olorum gave his blessing
    \nextverse
    I will ask Oxalá's permission
    To bath in the waterfall
    To take away all the bad things
  \end{translation}
\endsong


\beginsong{Corações da Luz}[by={Corações da Luz}, ph={III}]
  \audio[key={Bm}]{https://soundcloud.com/user-259793369/coracoes-da-luz}
  \meter{3}{4}
  \beginchorus\memorize
    \[^\mn{B}]Ó |\[\mnc{D}D]grande mis|\[\mnc{E}Em]terio do \[^\mn{D}]u\[^\mn{C#}]ni|\[\mnc{B}Bm]ver|so
    Eu |vou cho|\[G]rar de ale|\[Em]gri|a
    \lrep |\[G]Lindo Passa|\[A]rinho beija-|\[Bm]flor | \e
    % maybe: \lrep |\[\up{1}G (\up{2}Em)]Lindo Passa|\[\up{1}A]rinho beija-|\[Bm]flor | \e
    Vo|\[F#]{a\ldots} vo|\[F#7]{a\ldots} vo|\[Bm]{a\ldots} | \e \rrep
  \endchorus
  \notesoff
  \beginchorus
    |^Voa a|^través das es|^tre|las
    E de|fuma os |^nossos cora|^çõe|s
    \lrep A |^lua illumi|^nando as |^flo|res
    E as mon|^tanhas can|^tam de a|^mor | \e \rrep
  \endchorus
  \beginchorus
    Uma |^chuva le|^ve aca|^ricia-|me
    E o |sol co|^meça a bril|^har | \e
    \lrep Se faz sen|^tir o po|^der do arco-|^i|ris
    Que |^traz a |^luz do cria|^dor | \e \rrep
  \endchorus
  \beginchorus
    Ou|^ve e can|^ta essa mu|^si|ca
    E |levan|^ta as al|^tu|ras
    \lrep |^Vamos bril|^har meus ir|^mã|os
    Nos |^somos co|^rações da |^luz | \e \rrep
  \endchorus
  \begin{translation}
    O great mystery of the universe
    I'll cry for joy
    Beautiful bird hummingbird
    Fly\ldots fly\ldots fly\ldots
    \nextverse
    Fly through the stars
    And smudge our hearts
    The moon illuminating the flowers
    And the mountains sing of love
    \nextverse
    A light rain caresses me
    And the sun starts to shine
    The power of the rainbow is felt
    That brings the light of the creator
    \nextverse
    Listen and sing this song
    And lift the heights
    Let's shine my brothers and sisters
    We are hearts of light
  \end{translation}
\endsong


\beginsong{O Bálsamo do Céu}[by={Baixinha}, tags={Santo Daime}, ph={III}, key={Dm}, sks={Dm, Cm--D\shrp{}m}]
  % Also in songs_astral_unilaiva_diversos.tex (with lilypond notation)
  % Could this hymn be received by Isabel Barsé instead, as stated in nossairmandade.com?
  \audio[key=Dm]{https://www.nossairmandade.com/hymn.php?hid=2097}
  \audio[key=Dm]{https://soundcloud.com/lvaro-xi/b-lsamo-do-c-u}
  \meter{3}{4}
  \mnbeginchorus
    \lrep \[\mn{A}]O |\[\mnc{D}Dm]bálsamo \[\mn{E}]do |\[\mnc{F}F]Céu \[\mn{A}]des|\[\mnc{E}A]ceu \[\mn{F}]à |\[\mnc{E}Dm]Ter\[\mn{D}]ra \rrep
    \lrep \[\mn{A}]As |\[\mnc{B&}Gm]flo\[\mn{A}]res \[\mn{G}]da |\[\mnc{F}F]Ter\[\mn{E}]ra \[\mn{D}]rece|\[\mnc{A}A7]be\[\mn{C#}]ram \[\mn{E}]a |\[\mnc{D}Dm]Luz \rrep
  \mnendchorus
  \beginchorus\memorize
    \ind \lrep \[\mn{A}]Os |\[\mnc{D}Dm]homens \[\mn{E}]na |\[\mn{F}]Ter\[\mn{D}]ra \[\mn{F}]pre|\[\mnc{E}A]ci\[\mn{C#}]sam \[\mn{E}]do |\[\mnc{D}Dm]Bem \rrep
    \ind \lrep Meu |\[Gm]Pai é quem |\[F]manda mensa|\[A7]geiros do A|\[Dm]lém \rrep
  \endchorus
  \beginchorus
    \ind \lrep ^Os |^seres ^di|^vi^nos ^quem |^vem ^nos ^cu|^rar \rrep\notesoff
    \ind \lrep Com seus |^raios de |^Luz vem nos |^ilumi|^nar \rrep
  \endchorus
  \begin{translation}
    The balsam of Heaven came down to Earth
    The flowers of the Earth received the Light
    \nextverse
    The men of the Earth need the goodness
    My Father is who orders the messengers from beyond
    \nextverse
    The divine beings that come to heal us
    With their rays of light they come to illuminate us
  \end{translation}
\endsong


\beginsong{Elevei o Pensamento}[by={Ricardo Morais}, tags={Santo Daime}, key={D}, sks={D, C--A}, ph={III}]
  \audio[key={D}]{https://www.nossairmandade.com/hymn/25/EleveiOPensamento}
  \audio[key={D}]{https://soundcloud.com/beijamima/elevei-meu-pensamento}
  \mnbeginchorus\memorize
    \[^\mn{D}]Ele\[\bm]vei o \[^\mn{C#}]pensa|\[\mnc{B}Bm]men\[\bm]to | \e \altchords{\id[1]{(C)}|Am | \e}
    A \[\bm]Deus la \[^\mn{C#}]nas \[^\mn{D}]al|\[\mnc{E}Em]tu\[\bm]ras | \e \altchords{|Dm | \e}
    \[^\mn{A}]Para \[\bmc\mn{C#}]eu re\[^\mn{B}]conhe|\[\mnc{A}A]cer\[\bm] | \e \altchords{|G | \e}
    O bri\[\bmc\mn{B}]lho da \[^\mn{C#}]for\[^\mn{E}]mo|\[\mnc{D}D]su\[\bm]ra | \e \altchords{|C | \e}
  \mnendchorus
  \notesoff
  \beginchorus
    O bri^lho da formo|^su^ra | \e \altchords{\id[2]{(F)}|Dm | \e}
    É u^ma luz de a|^len^to | \e \altchords{|Gm | \e}
    Que lim^pa meu cora|^ção^ | \e \altchords{|C | \e}
    E vigo^ra meu pensa|^men^to | \e \altchords{|F | \e}
  \endchorus
  \beginchorus
    A^qui eu vou can|^tan^do | \e \altchords{\id[3]{(G)}|Em | \e}
    Que a ^minha lida e |^es^sa | \e \altchords{|Am | \e}
    Viva ^Rei Jurami|^dam^ | \e \altchords{|D | \e}
    E a Ra^inha da flo|^res^ta | \e \altchords{|G | \e}
  \endchorus
  \begin{translation}
    I raised up my thought to God there in the heights,
    For me to recognize the brilliance of beauty
    \nextverse
    The brilliance of beauty, it is a light of vital breath
    Who cleans my heart and invigorates my thoughts
    \nextverse
    Here I go singing because this is my task
    Viva King Juramidam and the Queen of the Forest
  \end{translation}
\endsong


\beginsong{Mamãe dos Ventos}[by={Madr. Maria Alice},ph={III},tags={Santo Daime},key={Em},sks={Gm, Gm--G\shrp{}m}]
  \audio[key=Gm]{https://www.nossairmandade.com/hymn/1291/MamãeDosVentos}
  \audio[key=Gm]{https://www.youtube.com/watch?v=PBpoenC25FE}
  \newchords{chords_mamaedosventos_a}\newchords{chords_mamaedosventos_b}
  \capo{3}
  \mnbeginchorus\memorize[chords_mamaedosventos_a]
    \[^\mn{B}]Mamãe dos |\[Em]Ventos, a\[\bm]qui eu vou cha|\[^\mn{E}]mar\[\bm] \altchords{\id[1]{(Gm)}|Gm| \e}
    Chaman\[^\mn{F#}]do o |\[\mnc{G}Am]tempo pa\[^\mn{F#}]ra \[\mnc{E}B7]vir \[^\mn{B}]se adian|\[Em]tar\[\bm] \altchords{|Cm D7 |Gm}
    \mnendchorus\glueverses\mnbeginchorus\memorize[chords_mamaedosventos_b]
    \[^\mn{B}]Justi|\[\mnc{A}Am]ceira, \[^\mn{E}]Vós vem\[\bm], \[^\mn{C}]na \[^\mn{B}]tem\[^\mn{A}]pes|\[\mnc{G}Em]tade, \[^\mn{E}]Vós vem\[\bm] \altchords{|Cm |Gm}
    \[^\mn{B}]Sop\[^\mn{A}]ran\[^\mn{G}]do o |\[\mnc{F#}B7]vento, com \[^\mn{A}]vos\[^\mn{C}]so \[\bmc\mn{B}]rai\[^\mn{A}]o pu\[^\mn{G}]ri\[^\mn{/}]fi|\[\mnc{/}Em]car\[\bm] \altchords{|D7 |Gm}
  \mnendchorus
  \notesoff
  \beginchorus\replay[chords_mamaedosventos_a]
    O vosso |^raio, é que ^rompe a imensi|dão^
    Para eu po|^der ouvir ^{a voz} do tro|^vão^
    \endchorus\glueverses\beginchorus\replay[chords_mamaedosventos_b]
    Epa|^rrê, Iansã^! Epa|^rrê, Iansã^!
    Epa|^rrê! dentro ^do meu cora|^ção^
  \endchorus
  \beginchorus\replay[chords_mamaedosventos_a]
    Minha Santa |^Bárbara, com a vos^sa chama na |mão^
    Vós sois Ra|^inha da ^comunica|^ção^
    \endchorus\glueverses\beginchorus\replay[chords_mamaedosventos_b]
    Estais no |^fogo da fo^gueira de São |^João
    ^Vós limpai a |^casa para a ^justifica|^ção
  \endchorus
  \brk
  \beginchorus\replay[chords_mamaedosventos_a]
    Mamãe dos |^Ventos, a ^vós eu vou lou|var^
    E o meu Ro|^sário também ^vos apresen|^tar^
    \endchorus\glueverses\beginchorus\replay[chords_mamaedosventos_b]
    Varrei as |^trevas, Iansã^, na varre|^ção, Iansã^
    Ilumi|^nai toda ^esta escuri|^dão^
  \endchorus
  \begin{translation}
    Mother of the Winds, here I will call
    Calling the time to come ahead
    \nextverse
    Righteous, You come, in the storm, You come
    Blowing the wind, purifying with your ray
    \nextverse
    Your ray is that it breaks the immensity
    So I can hear the voice of thunder
    \nextverse
    Eparrê, Iansã! Eparrê, Iansã!
    Eparrê! inside my heart
    \nextverse
    My Holy Barbara, with your flame in your hand
    You are Queen of communication
    \nextverse
    You are in the fire of the fire of São João
    You clean the house for righteousness
    \nextverse
    Mama of the Winds, I will praise you
    And my Rosary will also introduce you
    \nextverse
    Swept the darkness, Iansã, in the sweeping, Iansã
    Illuminate all this darkness
  \end{translation}
  \begin{explanation}
    \begin{description}
      \item[Iansã] is the Orisha of winds, storms, lightning, death and rebirth.
        She is a warrior and is unbeatable. Attributes of Iansã include great
        intensity of feelings, sensations, and charm. Another ability attributed
        to Iansã is control over the mysteries that surround the dead. She is
        syncretized with Santa Barbara. Her salutation is "Eparrê!" or
        "Eparrê, Oiá!
    \end{description}
  \end{explanation}
\endsong


\beginsong{Acreditar}[by={Vinicius Seki},ph={III}]
  \audio[key=Bm]{https://soundcloud.com/vinicius-seki/acreditar-vinicius-seki}
  \audio[key=Bm]{https://www.youtube.com/watch?v=d3qwlnuylx0}
  \beginverse
    \[\mnc{F#}Bm]{É a} força que |\[^\mn{G}]move e nos \[^\mn{F#}]en\[^\mn{C#}]sina a ser \[^\mn{D}]ir|\[\mnc{E}A]mao
    A ter simplici|\[F#m]dade acredi\[A]tar no cora|\[Bm]ção
    Coragem humil|dade pra servir em uni|\[A]ão
    E ter fé de ver|\[F#m]dade pra sa\[A]ir da ilu|\[Bm]sao
  \endverse
  \notesoff
  \beginverse
    ^{O amor} é o grande |mestre nos ensina e faz can|^tar
    Cantar a Madre |^Terra e com ^Deus sintoni|^zar
    Com graça e harmo|nia nao importa aonde |^for
    O bem é quem nos |^guia com sa^úde e com vi|^gor
  \endverse
  \beginverse
    ^{Ele quem} nos en|sina e nos inspira a trabal|^har
    E a ter sua von|^tade pra po^der compartil|^har
    A força que ha|bita bem dentro do cora|^ção
    Te entrego essa |^chave preste ^muita aten|^ção
  \endverse
  \beginverse
    ^Aprenda a silenci|ar o ego tao engana|^dor
    Fazendo aflo|^rar a luz di^vina que eu |^sou
  \endverse
  \begin{translation}
    It is the force that moves and teaches us to be a brother
    To have simplicity to believe in the heart
    Courage humility to serve in unity
    And have real faith to get out of illusion
    \nextverse
    Love is the great teacher that teaches us and makes us sing
    Sing Mother Earth and tune with God
    With grace and harmony no matter where you go
    Good is the one who guides us with health and vigor
    \nextverse
    He who teaches us and inspires us to work
    And having your will to be able to share
    The strength that dwells deep within the heart
    I give you this key pay close attention
    \nextverse
    Learn to silence the deceiving ego
    Bringing out the divine light that I am
  \end{translation}
\endsong


\scleardpage
\beginsong{Chama}[by={Nei Zigma},ph={III}]
  % No repeats in this one by the author:
  \audio[key={E&m}]{https://www.youtube.com/watch?v=mpW5eVY-4Jw}
  \audio[key=Am]{https://soundcloud.com/e-mohic/chama-cancao-de-nei-zigma}
  \capo{5}
  \mnbeginchorus\memorize
    \[^\mn{E}]Te |\[Em]chamo na força do |\[^\mn{B}]vento
    Para |\[\mnc{A}C]me \[Am] \[^\mn{B}]en\[^\mn{A}]si|\[\mnc{F#}B7]nar
    A flu|\[\mnc{E}Em]ir como tu f\[E7]luis nesse ele|\[\mnc{E}Am]mento
    A sabedo|\[\mnc{D}D7]ria do nos\[^\mn{E}]so \[^\mn{D}]a|\[\mnc{B}G]mar
    Te |\[\mnc{D}Bm]chamo na \[\mnc{C}C]força \[^\mn{B}]das |\[\mnc{F#}B7]Águas
    \[^\mn{E}]Pa\[^\mn{F#}]ra |\[\mnc{G}C]me \[Am] \[^\mn{A}]en\[^\mn{G}]gre|\[\mnc{F#}B7]nar \[^\mn{E}]nas e\[^\mn{F#}]mo|\[\mnc{G}Am]ções
    Em que tu fluis des\[^\mn{A}]se e\[^\mn{G}]le|\[\mnc{F#}B7]mento
    \[^\mn{E}]Quero a\[^\mn{F#}]pren|\[\mnc{G}C]der a \[\mnc{F#}Am]sem\[^\mn{E}]pre es|\[\mnc{D#}B7]tar
  \mnendchorus
  \notesoff
  \beginchorus
    Te |^chamo na força das |pedras
    Para |^me ^ silenci|^ar
    Nas profun|^dezas ^do teu mis|^tério
    Ser paci|^ência pra te escu|^tar
    Te |^chamo na ^força da |^terra
    Pro teu a|^mor ^ me fecun|^dar
    Luz - no infi|^nito do nosso mis|^tério
    Intui|^ção ^sempre a cri|^ar
  \endchorus
  \beginchorus
    Te |^chamo na força do |fogo
    Para |^eu ^ apren|^der
    Na co|^ragem em que tu f^luis
    Como gue|^rreiro ser teu |^spelho
    De amor sem |^medo
    Eterna |^chama quei^mando |^dentro
    Que se ali|^menta da ^força do |^vento
  \vspace{1em}\replay
    Oh! |^Força do vento que e|spalha
    As se|^mentes ^ nesta |^terra
    Que a|^guarda as ^águas do |^céu
    Deixando o |^solo bem mais |^fértil
    Que of |^fruta na ^terra flo|^resça
    En|^chendo ^nossos |^olhos
    Oh! |^Força do vento que ali|^menta
    Eterna |^chama ^ queimando |^dentro
    Queimando |\[Em]dentro
  \endchorus
  \begin{translation}
    I call you in the force of the wind
    To teach me
    To flow the way you flow in this elemento
    The wisdom of our love
    I call you in the force of the waters
    To connect me to the emotions
    The element in which you flow
    I want to learn to forever be
    \nextverse
    I call you in the force of the stones
    To silence me
    In the depths of your mystery
    Patience to hear your
    I call you in the force of the earth
    To be fertilized by your love
    Light - in the infinity of our mystery
    Intuition always in creation
    \nextverse
    I call you in the force of the fire
    So that I may learn
    From the courage in which you flow
    To be a warrior reflecting in your mirror
    Love without fear
    And in the earth a flame burning inside
    Nourished by the force of the wind
    \nextverse
    Oh! Force of the wind that scatters
    The seeds on this earth
    That receives the waters from the sky
    Leaving the soil richer
    May the fruits of the earth blossom
    Filling our eyes
    Oh! Force of the wind that nourishes
    Eternal flame burning inside
    Burning inside
  \end{translation}
\endsong


\beginsong{Cavalo Marinho}[ph={III, IV}]
  \audio[key=C]{https://rainerscheurenbrand.bandcamp.com/track/cavalo-marinho}
  \audio[key=C]{https://www.youtube.com/watch?v=-8bfTg6Re8g}
  \beginchorus\memorize
    \[\bm\mn{G}]Ga\meter{8}{8}|\[\bmc\mnc{C}C\bm]lop\[\bm]a \[\bm]ca\[\bmc\mnc{D}G]va\[\bmc\mn{C}]lo \[\bmc\mn{D}]ma\[\bmc\mn{E}]rin|\[\bmc\mnc{C}C\bm\bm]ho
    M\[\bm]e en\[\bmc F]sin\[\bm]a o \[\bm]cam\[\bm]in\meter{2}{8}|\[\bmc C]ho \[\bm]que \meter{3}{8}|\[\bmc Dm]dev\[\bm]o \[\bm]to\meter{8}{8}|\[\bmc Am\bm\bm\bm]mar \[\bm\bm\bm\bm]
    |\[\bmc F \bm]Solt\[\bm]a \[\bm]as \[\bmc G]cri\[\bm]nas \[\bm]no \[\bm]ven\meter{2}{8}|\[\bmc C]to
    \[\bm]Ga\meter{4}{8}|\[\bmc F]lo\[\bm]pa \[\bm]no \[\bm]ven\meter{2}{8}|\[\bmc C]to \[\bm]ca\meter{3}{8}|\[\bmc G]va\[\bm]lo \[\bm]do \meter{8}{8}|\[\bmc C\bm\bm\bm]mar \[\bm\bm\bm]
  \endchorus
  \notesoff
  \textnote{suomeksi:} % by: Outi
  \beginchorus
    ^ \meter{8}{8}|^Lauk^ka^a ^me^ri^he^vo|^nen ^
    ^Näy^tä ^mi^kä \meter{2}{8}|^tie ^mun \meter{3}{8}|^tu^lee ^ot\meter{8}{8}|^taa ^
    |^Hei^tä^ ^har^ja ^tuu^le\meter{2}{8}|^hen ^
    \meter{4}{8}|^Me^ri^he^vo\meter{2}{8}|^nen, ^lauk\meter{3}{8}|^kaa ^tuu^les\meter{8}{8}|^sa! ^
  \endchorus
  %% Different way of showing the rhythm (not as clear), using /4 and /8:
  % \beginchorus
  %   \[^\mn{G}]Ga\meter{4}{4}|\[\bmc\mnc{C}C]lo\[\bm]pa ca\[\bmc\mnc{D}G]va\[^\mn{C}]lo \[\bm\mn{D}]ma\[^\mn{E}]rin|\[\bmc\mnc{C}C]ho \[\bm]
  %   Me en\[\bmc F]sina o \[\bm]camin\meter{5}{8}|\[\bmc C]ho \[\bm]que \[\bmc Dm]de\[\bm]vo \[\bm]to\meter{4}{4}|\[\bmc Am]mar \[\bm\bm\bm]
  %   |\[\bmc F]Solt\[\bm]a as \[\bmc G]crinas \[\bm]no ven\meter{3}{4}|\[\bmc C]to
  %   Ga\[\bmc F]lopa \[\bm]no ven\meter{5}{8}|\[\bmc C]to \[\bm]ca\[\bmc G]va\[\bm]lo \[\bm]do \meter{4}{4}|\[\bmc C]mar \[\bm\bm\bm]
  % \endchorus
  % \beginchorus
  %   \meter{4}{4}|^Lauk^kaa ^meri^hevo|^nen ^
  %   ^Näytä ^mikä \meter{5}{8}|^tie ^mun ^tu^lee ^ot\meter{4}{4}|^taa ^
  %   |^Hei^tä ^harja ^tuule\meter{3}{4}|^hen
  %   ^Meri^hevo\meter{5}{8}|^nen, ^lauk^kaa ^tuu^les\meter{4}{4}|^sa! ^
  % \endchorus
  \begin{translation}
    Gallop, sea horse
    Teach me the way that I must take
    Set your mane free in the wind
    Gallop in the wind, horse of the sea
  \end{translation}
  \vfill
  \musicnote{Alternate, simplified rhythm: make each chord last for 2 beats,
    and double that to 4 for the `Am' and the final `C' chords.}
  \noendsongvfill
\endsong


\beginsong{Santa Maria}[by={Yatra},ph={III, IV}]
  \audio[key=C]{https://soundcloud.com/sound-of-light/santa-maria-266-florestral}
  \newchords{chords_santamaria_a}\newchords{chords_santamaria_b}
  \beginverse\memorize[chords_santamaria_a]
    \[^\mn{G}]Santa Ma|\[\mnc{E}C]ria fez bri\[\mnc{D}G]lhar a \[^\mn{E}]Lua |\[\mnc{D}C]Bran\[^\mn{C}]ca
    Quando che|\[C7]gou com Sua Força do As|\[F]tral
  \endverse
  \beginchorus\memorize[chords_santamaria_b]
    \[^\mn{D}\mn{C}]Ela |\[\mncii{D}{C}Dm]veio \[^\mn{D}\mn{C}]atra\[\mnc{D}G]vés \[^\mn{C}]da \[^\mn{B}]na\[^\mn{C}]tu|\[\mnc{G}C]re\[^\mn{E}]za
    Real\[Am]çar es\[A7]ta be|\[Dm]leza
    Deste \[G]Reino Imperi|\[C]al
  \endchorus
  \notesoff
  \beginverse\replay[chords_santamaria_a]
    Santa Ma|^ria é a ^Flor que desa|^brocha
    Neste Jar|^dim de Consciência Univer|^sal
  \endverse
  \beginchorus\replay[chords_santamaria_b]
    Ela é a |^Luz que tem a ^Força Cura|^dora
    Ela é a ^Santa ^Prote|^tora
    De Po^der Celesti|^al
  \endchorus
  \beginverse\replay[chords_santamaria_a]
    Santa Ma|^ria é a ^Força que con|^sola
    É a bon|^dade desta Grande Imensi|^dão
  \endverse
  \beginchorus\replay[chords_santamaria_b]
    Santa Ma|^ria com su^a Força Di|^vina
    É a Es^trela ^Matu|^tina
    Que ilu^mina esta ses|^são
  \endchorus
  \begin{translation}
    Santa Maria made the White Moon shine
    When she came with Her Power of the Astral
    \nextverse
    She came through nature
    Realizing this beauty
    Of this Imperial Reign
    \nextverse
    Santa Maria is the Flower that blooms
    In this Garden of Universal Consciousness
    \nextverse
    She is the Light that has the Healing Power
    She is the Saint Protectress
    Of Celestial Power
    \nextverse
    Santa Maria is the Force that consoles
    She is the kindness of this Great Vastness
    \nextverse
    Santa Maria with her Divine Strength
    Is the Morning Star
    That illuminates this session
  \end{translation}
\endsong


\beginsong{Caminhando com Fé}[by={Ambika},ph={IV}]
  \audio[]{https://soundcloud.com/mantric-mambo/01-seguindo-com-fe}
  \newchords{chords_caminhando_a}\newchords{chords_caminhando_b}
  %\capo{3} % no space on the spread for this
  \meter{2}{4}
  \beginverse\memorize[chords_caminhando_a]
    \[^\mn{B}]Eu |\[\mnc{E}Em]vou cami\[^\mn{B}]nhando com |\[^\mn{G}]fé si\[^\mn{E}]go \[^\mn{G}]nes\[^\mn{E}]ta |\[\mnc{G}B7]mis\[^\mn{F#}]são | \e
    Lou|vando tudo que é sa|grado com os meus pés |\[Em]no chão | \e
  \endverse
  \notesoff
  \beginverse\replay[chords_caminhando_a]
    Me |^firmo na força do |vento que sopra |^além | \e
    E |vou balançando nas |ondas num eterno |^vai e vem | \e
  \endverse
  \beginverse\memorize[chords_caminhando_b]
    \ind E |\[Am]ando na terra a |cantar me |\[Em]banho nas águas do |mar
    \ind Ol|\[C]hando o horizonte re|cebo minha mãe Ie|\[B7]manjá | \e
  \endverse
  \beginverse\replay[chords_caminhando_b]
    \ind E |^olho as estrelas |no céu, a |^lua e o sol a bri|lhar
    \ind Ma|^gia que faz o infi|nito se manife|^star | \e
  \endverse
  \beginverse\replay[chords_caminhando_a]
    Me en|^trego na dança da |vida com satis|^fação | \e
    Dan|çando com tranquili|dade neste fura|^cão | \e
  \endverse
  \beginverse\replay[chords_caminhando_a]
    |^Tudo vai passando aos |olhos e aquilo que |^fica | \e
    É a es|sência divina que |faz essa vida bo|^nita | \e
  \endverse
  \beginverse\replay[chords_caminhando_b]
    \ind Vi|^rando os olhos pra |frente man|^tendo focada a |mente
    \ind No mis|^tério que faz o uni|verso flore|^scer | \e
  \endverse
  \beginverse\replay[chords_caminhando_b]
    \ind Vi|^vendo o momento |como é com as |^cores de Oxuma|ré
    \ind Dei|^xando que a vida nos |faça um instrumento di|^vino | \e
  \endverse
  \brk
  \beginverse\replay[chords_caminhando_a]
    Mamãe Ju|^rema chega a|qui nesta mo|^rada
    Aben|çoai esta jor|nada confor|tai este vi|^ver | \e
  \endverse
  \beginverse\replay[chords_caminhando_a]
    Seu Sete |^Flechas que nos |dá a dire|^ção
    Ele en|sina segue em |frente e com a|mor agrade|^cer | \e
  \endverse
  \begin{translation}
    I will walk with faith, I follow this mission
    Praising all that is sacred with my feet on the floor
    \nextverse
    I hold on to strength from the wind that blows beyond
    And I'm rocking in the waves in an eternal come and go
    \nextverse
    And I walk the earth singing, I bathe in the waters of the sea
    Looking at the horizon I receive my mother Iemanjá
    \nextverse
    And look at the stars in the sky, the moon and the sun shining
    Magic that makes infinity manifest
    \nextverse
    I surrender in the dance of life with satisfaction
    Dancing with tranquility in this hurricane
    \nextverse
    Everything is passing by the eyes and what stays
    It is the divine essence, what makes this life beautiful
    \nextverse
    Turning eyes forward, keeping the mind focused
    In the mystery that makes the universe flourish
    \nextverse
    Living the moment as it is with the colors of Oxumaré \emph{(green, yellow, rainbow)}
    Letting life make us a divine instrument
    \nextverse
    Mother Jurema, get here at this address
    Bless this journey, comfort this life
    \nextverse
    Your Seven Arrows that give us the direction
    She teaches to move forward with love and gratitude
  \end{translation}
  \begin{explanation}
    \begin{description}
      \item[Oxumaré] is the \emph{orixá} of all movements, of all cycles. If one
        day Oxumaré loses its strength, the world will end, because the universe
        is dynamic and the Earth is also in constant motion. The rainbow is
        associated with Oxumaré.
      \item[Òrìṣà] (original spelling in the Yoruba language), known as
        \emph{orichá} or \emph{orixá} in Latin America, are the human form of the
        spirits \emph{(Irunmọlẹ)} sent by Olodumare/Olorun/Olofi in the Yoruba
        traditional account of the dawn of time. The Irunmọlẹ are meant to guide
        creation and particularly humanity on how to live and succeed on Earth
        \emph{(Ayé)}.
    \end{description}
  \end{explanation}
\endsong


\beginsong{Deixa a Gira Girar}[by={Mateus Alelulia, Heraldo, Dadinho}, ph={IV}]
  \audio[]{https://www.vagalume.com.br/os-tincoas/deixa-a-gira-girar.html}
  \audio[]{https://www.youtube.com/watch?v=zrymjzHlYEw}
  \audio[]{https://soundcloud.com/irineubarse-1/deixa-gira-girar-iemanja-find}
  \meter{2}{4}
  \capo{2}
  \beginchorus
    |\[Em] \[\mn{E}]Meu \[\mn{F#}]pai |\[\mnc{G}B7]veio \[\mn{F#}]da A\[\mn{E}]ru|\[\mncii{E}{B}Em]an\[\mn{E}]da \notesoff
    | e a nossa |\[Am]mãe |\[\mnc{D}D]é \[\mn{C}]I\[\mn{B}]an|\[\mnc{G}Em]sã |\[\up{1}B7] \e
  \endchorus
  \beginverse
    \[\mn{B}]Ô, |\[\mnc{A}Am]gira, deixa a |gira gira|r | \e
    Ô, |\[Em]gira, deixa a |gira gira|r | \e
    Ô, |\[F]gira, deixa a |gira gira|r | \e
    Ô, |\[Em]gira, deixa a |gira gira|r |\[B7] \e
  \endverse
  \beginverse
    |\[Em] \[\mn{G}]Dei\[\mn{F#}]xa a |\[\mn{E}]gi\[\mn{G}]ra \[\mn{B}]gi|\[\mn{E}]rar | \e
    | Sara|vá, I|ansã! | \e
    | É Xan|gô e Ieman|\[Am]já, iê | \e
    | Deixa a |\[B7]gira gi|\[Em]rar |\[B7] \e
    i|\[Em]ê |\[B7] i|\[Em]ô |\[B7] \e
  \endverse
  \begin{translation}
    My father came from Aruanda and our mother is Iansã.
    \nextverse
    Oh, gira, let the gira spin.
    \nextverse
    Let the gira spin.
    Saravá, Iansã!
    It's Xangô and Iemanjá, ye.
    Let the gira spin.
  \end{translation}
  \begin{explanation}
    \begin{description}
      \item[Aruanda] is a place in the spirit world, which varies greatly
        according to the religious stream, but which in general could be
        equated with a kind of a spiritual paradise.
      \item[Gira] is a spiritual gathering, the work, in Umbanda.
      \item[Saravá] is a salutation with the connotations of ``good luck'',
        ``blessing to the living and the dead'', ``save''.
    \end{description}
  \end{explanation}
  \yesendsongvfill
\endsong


\beginsong{Caminho da Luz}[ph={IV}]
  \beginchorus
    \lrep \[\mn{E}]O Ca|\[\mnc{A}Am]minho \[\mn{C}]da \[\mn{E}]Luz está |\[\mnc{D}Dm]den\[\mn{E}]tro \[\mn{F}]de \[\mn{D}]vo\[\mnc{E}Am]cê \rrep
    \lrep |\[C]Abra os \[Dm]olhos |\[Em]e se pode \[Am]ver \rrep
  \endchorus
  \beginchorus
    |\[\mnc{G}G]Erêre cla|\[\mnc{A}Am]rei\[\mn{E}]a! |\[\mnc{G}C]Salve a Lua |\[\mnc{A}Am]Chei\[\mn{E}]a!
    \lrep Os an|\[C]jinhos do \[Dm]Céu chegam |\[G]para brin\[Am]car \rrep
  \endchorus
  % Original image downloaded from: https://pixabay.com/vectors/road-path-black-and-white-sun-35435/
  % Image license: Public Domain
  \imagecc[3]{path_to_light_transparent_bg_PD_1280px.png}%
  \begin{translation}
    The Path of Light is in you
    \nextverse
    Open your eyes and you can see
    \nextverse
    Erê clears! Hail the Full Moon!
    \nextverse
    The angels of heaven came to play
  \end{translation}
  \begin{explanation}
    In \emph{Candomblé}, \textbf{Erê} is the intermediary between the initate
    and the Orixá. Erê resides at the exact point between the person's
    consciousness and the Orixá's unconsciousness. The word \emph{erê} means
    ``to play games'' in the \emph{Yoruba} language.
  \end{explanation}
\endsong


\scleardpage
\beginsong{Manto de Maria}[by={Pedro Gonçalves},ph={III, IV}]
  \audio[key=Em]{https://www.youtube.com/watch?v=q6W5Dbub6Qs}
  \audio[key=Em]{https://soundcloud.com/dois-sois/manto-de-maria}
  \beginverse
    |\[Em] \[\mn{E}]Todos os |dias a flo\[\mn{B}]resta \[\mn{A}]vem \[\mn{G}]cha|\[\mn{E}]mar
    Só quem a en|\[D]xerga é que vai aproxi|\[Em]mar
    Todos os |dias as lições vêm nos gui|ar
    Só quem a en|\[D]xerga é que vai aproxi|\[Em]mar
    E o Daime |\[D]leva pra transforma|\[Em]ção
    Rompendo esse |\[D]véu desse mundo de ilu|\[Em]são
    E é a |\[D]cura dessa manifesta|\[Em]ção
    De luz e a|\[D]mor, presença e pura comun|\[Em]hão
  \endverse
  \beginverse
    Ave Ma|\[Em]ria vossa proteção di|vina
    Eu concebo essa |cura e me permito ao a|mor
    Todos os |seres dessa cura inten|siva
    Eu permito total|mente a entrada da sa|bedoria
    Ave Ma|ria vossa proteção di|vina
    Eu me entrego total|mente e me permito a essa |luz
    Ave Ma|ria e |\[C]todos os \[B]guardi|\[Em]õ|'es
    Mani|\[C]festa|\[B7]ções de |\[Em]luz
  \endverse
  \beginverse
    |\[Em] Eu me per|\[D]mito totalmente a essa |\[Em]força
    Eu me per|\[D]mito porque eu quero me cu|\[Em]rar
    Eu me per|\[D]mito pra acabar com meu sof|\[Em]rer
    Porque eu sou filho a de |\[D]Deus a aprendiz do seu po|\[Em]der
    Porque a era da |\[D]luz vem chegando pra vi|\[Em]ver
    Ayahuasca é tão di|\[D]vina professora do meu |\[Em]ser
    É a força pode|\[D]rosa que me leva a renas-
    \up{1}(|\[Em]cer | | | | | | \e)
  \endverse
  \textnote{\emph{D.C. al Fine}}
  \beginverse
    |\sublyr{cer}\[C]Cura nas graças de |\[D]Ave Maria
    Ca|\[Em]minho a continui|dade do meu cora-
    |\sublyr{ção}\[C]Cura no manto de n|\[D]ossa Senhora
    Sen|\[Em]tindo a benção da |imensidão
    Do |\[C]ser, sen|\[D]tir e a|\[Em]mar | | | \e
  \endverse
  \beginchorus
    |Saia do mar, linda sereia
    |Saia do mar, vem brincar na areia
    |Saia do mar, bela sereia
    |Saia do mar, vem brincar com ela
  \endchorus
  \beginchorus
    Eu |vim eu vim cantar, se|reia do mar
  \endchorus
  \begin{translation}
    Every day the forest comes to call
    Only those who see it will approach
    Every day the lessons come to guide us
    Only those who see it will approach
    And Daime leads to transformation
    Breaking this veil of this world of illusion
    And it is the cure of this manifestation
    Of light and love, presence and pure communion
    \nextverse
    Ave Maria your divine protection
    I conceive this cure and allow myself to love
    All beings of this intensive healing
    I totally allow wisdom to enter
    Ave Maria your divine protection
    I give myself totally and allow myself to this light
    Ave Maria and all the guardians
    Light manifestations
    \nextverse
    I totally allow myself to this force
    I allow myself because I want to heal
    I allow myself to end my suffering
    Because I am a child of God, an apprentice of his power
    Because the age of light is coming to life
    Ayahuasca is such a divine teacher of my being
    It is the powerful force that leads me to be reborn
    \nextverse
    Healing in the graces of Ave Maria
    I walk the continuity of my heart
    Healing in the cloak of our Lady
    Feeling the blessing of immensity
    Of being, feeling and loving
    \nextverse
    Exit the sea, beautiful mermaid
    Exit the sea, come play in the sand
    Exit the sea, beautiful mermaid
    Exit the sea, come play with her
    \nextverse
    I came to sing, mermaid of th sea
  \end{translation}
\endsong


\beginsong{Tupinambá}[ph={IV}]
  \audio[key=Cm]{https://soundcloud.com/mantric-mambo/tupinamba}
  \beginchorus
    \[\mnc{F#}(Bm)]Seu Tu\[\mn{D}]pinam|\[\mn{C#}\mn{B}]bá
    Quando vem na al|deia
    Ele traz na |\[Em]cinta
    Uma cobra co|\[Bm]ral
  \endchorus
  \beginchorus
    \[\mn{F#}\mn{D}]{Oi---}|\[\mnc{E}Em]é uma cobra co|ral
    Oi---|\[Bm]é uma cobra co|ral
  \endchorus
  \begin{translation}
    Your Tupinambá
    When he comes to the village
    He brings the girdle
    A coral snake
    \nextverse
    Oié, a coral snake
    Oié, a coral snake
  \end{translation}
\endsong


\beginsong{Caboclo das Matas}[ph={IV}]
  \audio[key=C\shrp{}m]{https://soundcloud.com/josii-yakecan/caboclo-das-matas}
  \audio[key=Am]{https://www.youtube.com/watch?v=4Fpazy7zb1o}
  \capo{4}
  \beginverse
    \[^\mn{E}]Cab|\[\mncii{C}{A}Am]oclo \[E7] \[^\mn{E}]das m|\[\mncii{C}{A}Am]atas,
    Das cacho|eiras, das \[E7]pedras, das pe|\[Am]dreiras
    E das ondas do |\[E7]mar
  \endverse
  \notesoff
  \beginverse
    Cab|^oclas ^ guerr|^eiras,
    Mensa|geiras da ^paz e da harmo|^nia
    Saldados de Oxa|^lá | \e
  \endverse
  \noteson
  \beginchorus
    \[\mn{E}]Vêm \[\mn{F}]de A\[\mn{E}]ru|\[\mncii{D}{B}Dm]anda, \[\mn{D}]vêm, \[\mn{B}]vêm, \[\mn{D}]vê|m
    Trazendo |\[Am]força, vêm, vêm, vê|m
    Quebrando Mi|\[E7]ronga, vêm, vêm, vê|m
    Na Umbanda sara|\[Am]vá |\[\up{1}(A7)] \e
  \endchorus
  \begin{translation}
    Caboclo of the forests,
    Of the waterfalls, the rocks, the quarries
    And in the sea waves
    \nextverse
    Caboclas warriors,
    Messengers of peace and harmony
    Praise Oxalá
    \nextverse
    Come from Aruanda, come, come, come
    Bringing strength, come, come, come
    Breaking Mironga, come, come, come
    Praise in Umbanda
  \end{translation}
  \begin{explanation}
    \begin{description}
      \item[Mironga] is a term originating from the word ``milonga'' from
      Kimbundu, the language spoken by the Ambundus in Angola. It refers to
      the mystery, secret. In the Umbanda religion, it has become a term that
      designates someone who makes \emph{mironga}, or who is a
      \emph{mirongueiro}. It took on the meaning of a spell, manipulation of
      energies.
    \end{description}
  \end{explanation}
\endsong


\beginsong{Mãe Verdadeira}[by={Totó}, ph={IV}]
  \audio[key=Dm]{https://soundcloud.com/user-55157743/38-ma-e-verdadeira}
  \meter{3}{4}
  \beginchorus
    |\[(Dm)] \[\mn{D}]Minha |mãe \[\mn{F}]ver\[\mn{G}]da|\[\mnc{A}Dm]dei|ra me
    en|sina a con|tar as es|\[Am]tre|las
    Me en|sina a ser |sempre cri|\[B&]an|ça,
    | que eu não |perca a lem|\[F]bran|ça
    |Que eu seja |sempre uma |\[Dm]flor, | \e
    |que eu sempre |busque o a|\[Am]mor | \e
    |E no jar|dim da espe|\[B&]rança, | \e
    | seja |semea|\[F]dor | \e
  \endchorus
  \beginchorus
    \ind | \[\mn{F}]Olha a |linda \[\mn{D}]se|\[\mncii{F}{D}Dm]reia, | \e
    \ind | vem do |fundo do |\[B&]mar | \e
    \ind | Ela |brinca na a|\[Dm]reia, | \e
    \ind | ela |gira no |\[C]ar | \e
  \endchorus
  \begin{translation}
    My real mother teaches me to count the stars
    Teaches me to always be a child, that I don't lose the memory
    May I always be a flower, may I always seek love
    And in the garden of hope, be a sower
  \nextverse
    Look at the beautiful mermaid, who comes from the bottom of the sea
    She plays in the sand, she spins in the air
  \end{translation}
\endsong


\beginsong{Medicina Ancestral}[by={Vinícius Santos Rocha},ph={IV}]
  \audio[key=Am]{https://www.youtube.com/watch?v=AXOXUY6R9fg}
  \audio[key=Dm]{https://soundcloud.com/bastiaan-yansa/medicina-ancestral}
  \meter{3}{4}
  \beginchorus\memorize
    \lrep \[^\mn{E}]A|\[\mnc{A}Am]ho \[^\mn{C}]meus |\[\mnc{B}E]ir\[^\mn{G#}]mãos \[^\mn{B}]de |\[\mnc{A}Am]fé \rrep
    \lrep To|\[Am]mei Chacro|\[C]nita, Aya|\[F]huasca, Ya|\[E]gé \rrep
    \lrep A |\[Am]força este|\[G]lar na flo|\[F]resta cu|\[E]ra de Pa|\[Am]jé \rrep
  \endchorus
  \notesoff
  \beginchorus
    \lrep É |^medi|^cina ances|^tral \rrep
    \lrep E|^spirito |^Santo que |^vem nos cu|^rar \rrep
    \lrep E|^leva nos|^sa Consci|^ência ao |^reino do as|^tral \rrep
  \endchorus
  \beginchorus
    \lrep Es|^sência |^da luz so|^lar \rrep
    \lrep No |^planeta |^Terra vem |^pra traba|^lhar \rrep
    \lrep Com |^todos os |^seres Di|^vinos pa|^ra nos gui|^ar \rrep
  \endchorus
  \begin{translation}
    Aho, my fellow believers
    I drank Chacronita, Ayahuasca, Yagé
    The star power in the forest, healing of the shaman
    \nextverse
    It is ancestral medicine
    Holy Spirit who comes to heal us
    Raises our consciousness to the realm of the astral
    \nextverse
    Essence of sunlight
    On Earth comes to work
    With all the Divine beings to guide us
  \end{translation}
\endsong


\beginsong{Ela é um Ponto de Luz}[by={Marcos Trench},ph={IV}]
  \audio[key=C]{https://soundcloud.com/jesus-gomez-5/ela-e-um-ponto-de-luz}
  \beginchorus\memorize
    |\[\mnc{C}C] Ela é um ponto \[^\mn{E}]de |\[^\mn{G}\mn{E}]Luz | ela está na Ju|\[\mncii{D}{C}F]rema
    | É Oxum minha |\[G]Mãe | é O\[F]xum minha |\[C]Mãe
  \endchorus
  \notesoff
  \beginchorus
    |^ Ela é nossa Ra|inha | ela é Mãe das prin|^cesas
    | É Oxum minha |^Mãe | é O^xum minha |^Mãe
  \endchorus
  \beginchorus
    |^ As águas que |correm | na natu|^reza
    | É Oxum minha |^Mãe | é O^xum minha |^Mãe
  \endchorus
  \begin{translation}
    She is a light point, she is inside Jurema
    It is my Mother Oxum, it is my Mother Oxum
    \nextverse
    She is our Queen, she is the mother of pricesses
    It is my Mother Oxum, it is my Mother Oxum
    \nextverse
    The waters that are flowing in nature
    It is my Mother Oxum, it is my Mother Oxum
  \end{translation}
\endsong


\beginsong{Pantera Negra}[by={Leo do Gamarra}, key={Dm}, sks={Dm, Dm--Em}, ph={IV}]
  \audio[key=Dm]{https://soundcloud.com/leonardo-fernandes-40/pantera-negra}
  \newchords{chords_panteranegra_a}\newchords{chords_panteranegra_b}
  \mnbeginchorus\memorize[chords_panteranegra_a]
    \[^\mn{A}]Eu |\[\mnc{D}Dm]vi u\[\bm]ma pante\[^\mn{F}]ra |\[^\mn{A}]neg\[^\mn{F}]ra \[\bmc\mn{D}]no \[^\mn{B&}]as|\[\mnc{A}A]tral | \[\bm] \e
    \mnendchorus\glueverses\mnbeginchorus\memorize[chords_panteranegra_b]
    \[^\mn{A}]Veio de |\[\mnc{G}Gm]longe expul\[\bmc\mn{F}]sando to\[^\mn{D}]do |\[Dm]mal \[\bm]
    \[^\mn{A}]trazendo |\[\mnc{G}Gm]{o poder} do a\[\bmc\mn{/}]mor uni\[^\mn{/}]ver|\[\mnciii{/}{/}{/}Dm]sal \[\bm] \up{2}(| \[\bm] \e)
  \mnendchorus
  \notesoff
  \beginchorus\replay[chords_panteranegra_a]
    Che|^gou o ^cacique to|cando o seu ^mara|^cá | ^ \e
    \endchorus\glueverses\beginchorus\replay[chords_panteranegra_b]
    Veio no |^vento condu^zindo a luz do |^Céu ^
    trazendo a |^força do Ar^canjo Gabri|^el ^ \up{2}(| ^ \e)
  \endchorus
  \beginchorus\replay[chords_panteranegra_a]
    Eu |^vi o e^xercito de |Deus a ^perfi|^lar | ^ \e
    \endchorus\glueverses\beginchorus\replay[chords_panteranegra_b]
    Pra rece|^ber a luz a ^santa ora|^ção ^
    da Dou|^trina da Vir^gem da Concei|^ção ^ \up{2}(| ^ \e)
  \endchorus
  \beginchorus\replay[chords_panteranegra_a]
    Che|^gou um ^Rei todo de |branco e ^pés no |^chão | ^ \e
    \endchorus\glueverses\beginchorus\replay[chords_panteranegra_b]
    Para mos|^trar que nesse ^mundo de ilu|^são ^
    o que im|^porta é o a^mor no cora|^ção ^ \up{2}(| ^ \e)
  \endchorus
  \begin{translation}
    I saw a black panther in the astral.
    It came from afar expelling all evil,
    bringing the power of universal love.
    \nextverse
    The chief arrived playing his maraca.
    It came on the wind carrying the light of Heaven,
    bringing the strength of Archangel Gabriel.
    \nextverse
    I saw God's army standing up to attention.
    To receive the light of the holy prayer
    of the Doctrine of the Virgin of Conception.
    \nextverse
    A King arrived all in white and feet on the ground.
    To show that in this world of illusion
    what matters is the love in the heart.
  \end{translation}
  \begin{explanation}
    \begin{description}
      \item[Cacique] means `chieftain' in the extinct Taíno language of the
      Caribbean. The Taíno were the first New World peoples encountered by
      Christopher Columbus during his 1492 voyage.
    \end{description}
  \end{explanation}
\endsong


\beginsong{Ciranda para Janaína}[ph={IV}]
  \audio[key=Bm]{https://soundcloud.com/jonathancompositor/ciranda-da-jana-na}
  \beginverse
    \[\mn{F#}]O seu co\[\mn{G}]lar é \[\mn{F#}]de |\[\mnc{D}Bm]con\[\mn{B}]cha
    Seu vestido se arrasta na a|\[Em]reia
    Ela tem cheiro de |\[F#]mar
    Ela sabe cantar, ponto de se|\[Bm]reia
  \endverse
  \beginverse
    O seu colar é de |\[Bm]concha
    Seu vestido se arrasta na a|\[Em]reia
    \echo{Ela tem cheiro de |\[F#]mar
    Ela sabe cantar, ponto de se|\[Bm]reia}
  \endverse
  \beginverse
    \ind[2] \[\mn{D}]Ó Ja\[\mn{C#}]na|\[\mncii{B}{F#}Bm]ína quando estou fe\[\mn{G}]liz \[\mn{F#}]eu |\[\mnc{C#}F#]choro
    \ind[2] Ó Jana|ína deixa eu dormir no seu |\[Bm]colo
    \ind[2] \echo{Ó Jana|ína quando estou feliz eu |\[F#]choro
    \ind[2] Ó Jana|ína deixa eu dormir no seu |\[Bm]colo}
  \endverse
  \beginverse
    \ind \[\mn{B}]E é no \[\mn{C#}]seu |\[\mncii{D}{B}Em]colo que \[\mn{C#}]a\[\mn{D}]fo\[\mn{B}]go a min\[\mn{C#}]ha s|\[\mn{D}]e\[\mn{B}]de
    \ind Quis te pes|\[Bm]car, mas caí na sua r|ede
    \ind Feita de |\[F#]fios de cabelo maranh|ado
    \ind Moro no ma|\[Em]{r e hoje} \[F#7]sou seu namor|\[Bm]ado
    \ind \echo{Eu moro no ma|\[Em]{r e hoje} \[F#7]sou seu namor|\[Bm]ado}
  \endverse
  \beginverse
    \textnote{Instrumental, on 1st playthrough:}
    \ind[3] |\[Em]{ }{ }{ } |\[Bm]{ }{ }{ } |\[F#]{ }{ }{ } |\[Bm] \e
    \ind[3] |\[Em]{ }{ }{ } |\[Bm]{ }{ }{ } |\[F#]{ }{ }{ } |\[Bm] \e
  \endverse
  \textnote{\emph{D.C. al Fine}}
  \beginchorus
    \ind[2] Ó Jana|\[Bm]ína quando estou feliz eu |\[F#]choro
    \ind[2] Ó Jana|ína deixa eu dormir no seu |\[Bm]colo
  \endchorus
  \begin{translation}
    Your necklace is from shell
    Your dress drags on the sand
    She smells of the sea
    She can sing, mermaid song
    \nextverse
    O Janaína when I'm happy I cry
    O Janaína, let me sleep on your lap
    \nextverse
    And it's in her lap that I drown my thirst
    I wanted to catch you but I fell in your net
    Made from matted strands of hair
    I live at the sea and today I'm your lover
    I live at the sea and today I'm your lover
  \end{translation}
\endsong


\beginsong{Recado da Mãe Divina}[by={Chandra Lacombe},ph={IV}]
  \audio[key=D]{https://soundcloud.com/canal-chandra/04-recado-da-m-e-divina}
  \audio[key=D]{https://www.youtube.com/watch?v=rSBNDJSmECc}
  \newchords{chords_recado_da_mae_a}\newchords{chords_recado_da_mae_b}
  \mnbeginchorus\memorize[chords_recado_da_mae_a]
    \[^\mn{D}]Vem sur|\[D]gindo um novo |\[G]tempo
    Traz |\[\mnc{C#}F#7]glórias \[^\mn{D}]do \[^\mn{E}]di|\[\mnc{B}Bm]vino
    \[^\mn{D}]Mais |\[Em]puros e a|\[\mnc{E}A]tentos
    \[^\mn{F#}]Nos \[^\mn{E}]tor|\[\mnc{D}G]namos canais do \[A]in\[^\mn{C#}]fi|\[\mnc{D}D]nito
  \mnendchorus
  \mnbeginchorus\memorize[chords_recado_da_mae_b]
    \ind \[^\mn{D}]Mãe di|\[D]vina eu quero |\[\mnc{G}G]ser
    \ind Um fil|\[\mnc{F#}F#7]ho re\[^\mn{G}\mn{A}]ali|\[\mnc{F#}Bm]zado
    \ind E é pe|\[\mnc{G}Em]ran\[^\mn{F#}]te o \[^\mn{E}]seu \[^\mn{D}]po|\[\mnc{E}A]der
    \ind Que me en|\[\mnc{F#}G]trego \[^\mn{E}]pra se \[\mncii{D}{C#}A]liber|\[\mnc{D}D]tado
  \mnendchorus
  \notesoff
  \beginchorus\replay[chords_recado_da_mae_a]
    Como um |^rio que corre para o |^mar
    Corren|^tezas carregam o |^medo
    Confi|^ança para atraves|^sar
    A fron|^teira do eu ^derra|^deiro
  \endchorus
  \beginchorus\replay[chords_recado_da_mae_b]
    \ind Não há des|^culpas para se esco|^rar
    \ind Já foi |^dito a hora é |^essa
    \ind O |^tempo é de se inte|^grar
    \ind Abra|^çando o que a^inda |^resta
  \endchorus
  \beginchorus\replay[chords_recado_da_mae_a]
    Estou mo|^rrendo para o pas|^sado
    E nem an|^seio pelo o fu|^turo
    Minha co|^roa tem brilho dou|^rado
    Provo o |^néctar do a^mor ma|^du\sublyr{\up{2}\textbf{Hey-}}ro
  \endchorus
  \beginchorus\replay[chords_recado_da_mae_b]
    \ind[2] \textbf{-aa hey}|^{\textbf{aa heyaa}}, hay|^ee
    \ind[2] Hay|^ee hayee hay|^aa
    \ind[2] Hay|^ee hayee hay|^aa
    \ind[2] Hay|^ee \up{1}(hay^ee hay)|^{a\sublyr{\up{1}\textbf{Hey-}}a}
    % note that second round is a bit different
  \endchorus
  \begin{translation}
    The new time has emerged
    Bringing the glory of the Divine
    More pure and attentive
    We become channels of the infinite
    \nextverse
    Divine Mother I want to be
    A realized child
    And it is by your power
    That I commit to be liberated
    \nextverse
    Like a river that runs to the sea
    Currents carrying away fear
    With the trust to cross over
    The boundaries of myself
    \nextverse
    No longer can excuses be supported
    The moment is now
    The time of becoming complete
    Embraces that which still remains
    \nextverse
    I am dying to the past
    And I do not even yearn for the future
    My crown has a golden shine
    I taste the nectar of mature love
    \nextverse
    Heya heya heya hayaa\ldots
  \end{translation}
\endsong


\beginsong{Aliança}[by={Madr. Beth},ph={IV},tags={Santo Daime}]
  \audio[key=Am]{https://www.youtube.com/watch?v=YLKr6Tq60aM}
  \audio{https://soundcloud.com/astral-flowers-music/alianca}
  \mnbeginchorus\memorize
    \lrep \[^\mn{E}]Esta ali|\[\mnc{A}Am]an\[^\mn{B}]ça \[^\mn{C}]vei\[^\mn{B}]o \[^\mn{C}]do \[^\mn{D}]as|\[\mnc{E}E]tral
    \[^\mn{A}]Vem do as|\[\mnc{F}Dm]tral \[^\mn{D}]do \[^\mn{E}]as\[^\mn{F}]tral su\[^\mn{A}]peri|\[\mnc{E}E]or \rrep
    \lrep \[^\mn{E}]Ela é \[^\mn{A}]di|\[\mncii{F}{D}Dm]vina e \[^\mn{E}]con\[^\mn{F}]sagra as \[^\mn{A}]medi|\[\mnc{E}C]ci\[^\mn{C}]nas
    Essas flo\[^\mn{E}]res \[^\mn{D}]tão |\[\mnc{B}G]finas do nos\[^\mn{C}]so \[^\mn{B}]Pai \[^\mn{A}]Cri\[^\mn{G}]a|\[\mnc{A}Am]dor \rrep
  \mnendchorus
  \notesoff
  \beginchorus
    \lrep Evolu|^indo as quatro portas do |^mundo
    Na expan|^são das doutrinas na|^tivas \rrep
    \lrep A profe|^cia da águia e do |^condor
    Vem do Norte para o |^Sul selando com todo a|^mor \rrep
  \endchorus
  \beginchorus
    \lrep Sagrado Ta|^baco, Peyote, Santo |^Daime
    Santa Ma|^ria, São Pedro clare|^ou \rrep
    \lrep Estes |^são os Sacramentos Di|^vinos
    da nossa Mãe |^Terra, Cristo vivo Reden|^tor \rrep
  \endchorus
  \beginchorus
    \lrep O Grande E|^spírito me entrego com a|^mor
    Com a|^mor Aliança Superi|^or \rrep
    \lrep Clareai as |^trevas deste povo sofre|^dor
    Iluminando o ca|^minho do fiel segui|^dor \rrep
  \endchorus
  \begin{translation}
    This Alliance came from the astral
    It comes from the astral, from the superior astral
    She is divine and consecrates medicines
    These fine flowers of our Creator Father
    \nextverse
    Evolving the four doors of the world
    In the expansion of native doctrines
    The prophecy of the eagle and the condor
    Comes from North to South sealing with all love
    \nextverse
    Sacred Tobacco, Peyote, Santo Daime
    Santa Maria, São Pedro illuminate
    These are the divine sacraments
    of our Mother Earth, the redeeming living Christ
    \nextverse
    The Great Spirit I give myself with love
    With love Superior Alliance
    Lighten the darkness of these suffering people
    Illuminating the path of the faithful follower
  \end{translation}
\endsong


\beginsong{Caboclo na Mata}[ph={IV}]
  \audio[]{https://soundcloud.com/pachamamachurch/caboblo-na-mata}
  \beginchorus\memorize
    Eu |\[\mnc{E}Am]vi um caboclo na |mata, \[^\mn{A}]um \[^\mn{B}]in|\[\mnc{C}G]dio Hu\[^\mn{B}]ni \[^\mn{A}]Ku|\[Am]in
  \endchorus\notesoff\glueverses
  \beginchorus
    Com |^sua maraca na |mão, baten|^do com pé no |^chão
  \endchorus\glueverses
  \beginchorus
    A ma|^raca de ca|baça, toda |^cheia de pe|^drinha
  \endchorus\glueverses
  \beginchorus
    U|^ma pena é ver|melha, outra é |^bem ama|^relinha
  \endchorus\glueverses
  \beginchorus
    Vem tra|^zendo as medi|cinas la do |^alto do Jor|^dão
  \endchorus\glueverses
  \beginchorus
    Ra|^pé e Nixi |Pae, Kampun, Sa|^nanga e Tchi |^Txan
  \endchorus\glueverses
  \beginchorus
    Yube |^mana i bu|bu mana |^ibubu bu|^ta
  \endchorus\glueverses
  \beginchorus
    Eska|^watã kaya|wey, kaya|^wey ki |^ki
  \endchorus
  \begin{translation}
    I saw a caboclo in the woods, an Indian Huni Kuin
    With his gauntlet in his hand, beating with his foot on the floor
    The maraca of gourd, all full of pebbles
    One feather is red, another is very yellow
    He has brought medicines from the top of the Jordão
    Rapé and Nixi Pae, Kampun, Sananga and Tchi Txan
    Yube mana i bubu mana ibubu buta
    Eskawatã kayawey, kayawey ki ki
  \end{translation}
\endsong


\beginsong{Aya Aya Ayahuasca}[tags={Aya},ph={IV}]
  \beginchorus\memorize
    \ind \[^\mn{G}]A|\[\mnc{A}Am]ya a|ya a|\[G]yahuas|\[Am]ca
    \ind A|\[C]ya a|\[Dm]ya a|\[G]ya\[Em]huas|\[Am]ca
  \endchorus
  \notesoff
  \beginchorus
    Cha|^mei o Rei das |Flores el|^e me respon|^deu
    A|^ya a|^ya a|^ya^huas|^ca  \goto{Aya aya}
  \endchorus
  \beginchorus
    Ó |^Rei Ayahuas|ca ele |^toma Ia|^gé
    |^Ia|^gé a|^ya^huas|^ca
  \endchorus
  \beginchorus
    Ó |^Rei Ayahuas|ca ele |^vive na flo|^resta
    Ele |^é filho de |^indio e da Ra|^inha ^da Flo|^resta  \goto{Aya aya}
  \endchorus
  \beginchorus
    Vive|^mos neste plan|tas com a |^vossa prote|^ção
    Se ba|^lança mara|^cá com a|^mor no ^cora|^ção  \goto{Aya aya}
  \endchorus
  \begin{translation}
    Aya aya aya ayahuasca, aya aya aya ayahuasca
    \nextverse
    I called the King of Flowers, he answered me; Aya aya ayahuasca
    \nextverse
    Oh King Ayahuasca, he takes Iagé; Iagé ayahuasca
    \nextverse
    Oh King Ayahuasca, he lives in the forest
    He is son of indian and the Queen of the Forest
    \nextverse
    We live among the plants, with your protection
    Swinging maracá with love in the heart
  \end{translation}
  \begin{explanation}
    About the use of the word \emph{Rei}, ``King'': in many indigenous cultures of South America,
    the spirit of \emph{Banisteriopsis Caapi} is considered male, contrary to the view commonly
    held in the modern global culture.
  \end{explanation}
\endsong


\beginsong{Mamãe Maria}[by={Ale de Maria},ph={IV}]
  \audio[]{https://www.youtube.com/watch?v=GNrGJH3iI6M}
  \beginchorus
    \[\mn{F#}]He\[\mn{A}]yo |\[\mnc{B}Bm]he-| | | \[\mn{D}]yo \[\mn{B}]he\[\mn{A}]yo
    |\[D]heyo | | yo heyo |\[Bm]heyo | | \e
  \endchorus
  \beginverse\memorize
    Mamãe Ma|\[Bm]ria, em |seu jardim em |\[D]flor
    Ma|mãe abenço|\[Bm]ou, me ensi|nou a can|\[Em]tar | \e
    Sigo re|\[G]zando essa |canção em lou|\[D]vor
    Cana|rinho beija-|\[Em]flor, do jar|\[G]dim de ma|\[D]mãe | \e
  \endverse
  \notesoff
  \beginverse\replay
    Terra que |^dança, que ba|lança seu rei|^nado
    Com o |mar sempre ao seu |^lado e o |sol sempre a bril|^har | \e
    Ouça o |^canto, a|tenda o cha|^mado
    Passa|rinho encan|^tado com sau|^dade de vo|^ar | \e
  \endverse
  \beginverse
    Voar bem |\[Em]longe sobre o |mar na maré |\[D]cheia | \e
    Onde a se|\[Em]reia canta |pra mãe Ieman|\[D]já | \e
    Lá na flo|\[Em]resta, santa |verde Ama|\[D]zônia | \e
    Santa Ma|\[Em]ria é quem cla|\[G]reia o Mapi|\[D]á | \e
  \endverse
  \beginverse\memorize
    Salve Aya|\[Bm]huasca, Vô Pey|ote e São |Pedro | \e
    Salve essas |flores que ma|mãe do céu nos |\[F#m]dá | \e
    São essas |\[Em]rosas que ma|mãe do céu cul|\[D]tiva | \e
    São as est|\[Em]relas que não |\[G]param de bril|\[D]har | \e
  \endverse
  \beginverse\replay
    E assim hon|^rar o brilho |do fogo sa|grado | \e
    Do Santo |Daime e da |arca da ali|^ança | \e
    Honrar o a|^mor, a luz pre|sente na fa|^mília | \e
    Caminho ao |^céu onde se |^reza, busca e |^dança | \e
    % alternate line: Onde se reza pros que dançam
  \endverse
  \gotoiii{Heyo heyo}{Salve Ayahuasca}{E assim honrar}
  \begin{translation}
    Mother Mary in her flowering garden
    Mother blessed, taught me how to sing
    I keep praying this song in praise
    Canary hummingbird, from Mother's garden
    \nextverse
    Earth dancing and oscillating its reign
    With the sea always by its side and the sun always shining
    Listening to the song, answering the call
    Enchanted bird with longing to fly
    \nextverse
    Flying well away over the sea at full tide
    Where the mermaid sings to mother Iemanja
    There in the forest, holy green Amazon
    Santa Maria is the one who clears Mapiá
    \nextverse
    Hail Ayahuasca, fly Peyote and San Pedro
    Hail these flowers that Mother gives us
    These are the roses that Mother of the sky cultivates
    They are the stars that do not stop shining
    \nextverse
    And thus honor the glow of the sacred fire
    Of the Santo Daime and of the ark of the covenant
    Honor the love, the light present in the family
    Way to the sky where you are praying, searching and dancing
    % alternate line: Where the dancing prayers are
  \end{translation}
\endsong


\beginsong{Eu Canto nas Alturas}[by={Mestre Irineu Serra},tags={Santo Daime},ph={IV}]
  \audio[]{http://www.nossairmandade.com/hymn.php?hid=230}
  \audio[]{https://www.youtube.com/watch?v=-2GYf9ASAxA}
  \beginchorus\memorize
    \[^\mn{D}]Eu |\[\mnc{C}C]canto \[^\mn{D}]nas \[^\mn{E}]al|\[\mncii{F}{D}Dm]turas
    A minha |\[C]voz é reti|\[Dm]nida
    \lrep Porque eu |\[F]sou filho de |\[C]Deus
    E tenho a |\[Am]minha Mãe que|\[Dm]rida \rrep
  \endchorus
  \notesoff
  \beginchorus
    A minha |^Mãe que me ensi|^nou
    A minha |^Mãe que me man|^dou
    \lrep Eu |^sou filho de |^Vós
    Eu |^devo ter a|^mor \rrep
  \endchorus
  \beginchorus
    Com a|^mor tudo é ver|^dade
    Com a|^mor tudo é cer|^teza
    \lrep Eu |^vivo neste |^mundo
    Sou |^dono da ri|^queza \rrep
  \endchorus
  \beginchorus
    A minha |^Mãe é a Lua |^Cheia
    É a Es|^trela que me gu|^ia
    \lrep Estando |^bem perto de |^mim
    Junto a |^mim é prenda |^minha \rrep
  \endchorus
  \beginchorus
    A ri|^queza todos |^têm
    Mas é pre|^ciso compreen|^der
    \lrep Não |^é com fingi|^mento
    Todos |^querem mere|^cer \rrep
  \endchorus
  \begin{translation}
    I sing in the heights
    My voice is resonant
    Because I am a son of God
    And I have my dear Mother
    \nextverse
    My Mother who taught me
    My Mother who sent me
    I am Thy son
    I must have love
    \nextverse
    With love everything is truth
    With love everything is certain
    I live in this world
    I am an owner of riches
    \nextverse
    My Mother is the full Moon
    She is the Star who guides me
    Being very near me
    Close to me, She is my gift
    \nextverse
    The riches everyone has
    But it is necessary to understand
    It is not with pretense
    All want to be worthy
  \end{translation}
\endsong


\beginsong{Flor das Águas}[by={Mestre Irineu Serra},tags={water, sea, Santo Daime},ph={IV},key={Am},sks={Cm, Dm, Am}]
  \audio{https://www.nossairmandade.com/hymn/94/FlorDas\%C3\%81guas}
  \beginchorus
    |\[\mnc{E}Am]Flor \[\mn{C}]das |\[\mnc{D}Dm]águas, \[\mn{C}]da \[\mn{D}]on\[\mn{F}]de |\[\mnc{E}E]vens, para \[\mnc{D}(E7/D)]on\[\mn{E}]de |\[\mnc{C}Am]vais
    Vou \[\mn{B}]fa|\[\mn{A}]zer a mi\[\mnc{C}C]nha \[\mn{D}]lim|\[\mnc{E}E]peza, no \[\mnc{D}(E7/D)]cora|\[\mnc{C}C]ção \[\mn{B}]es\[\mnc{G#}E]tá \[\mn{B}]meu |\[\mnc{A}Am]Pai
  \endchorus
  \mnbeginchorus\memorize
    \ind \[^\mn{A}]A \[^\mn{C}]mo|\[\mnc{E}Am]rada \[\mnc{D}G]do \[^\mn{E}]meu |\[\mncii{C}{B}Am]Pai \[^\mn{A}]é \[^\mn{C}]no |\[\mnc{E}C]cora\[^\mn{D}]ção \[^\mn{E}]do |\[\mnc{D}Dm]mundo
    \ind Aonde e|xiste \[\mnc{C}C]to\[^\mn{B}]do a|\[\mnc{E}E]mor e \[\mnc{D}(E7/D)]{tem um} |\[\mnc{C}C]se\[^\mn{B}]gre\[\mnc{G#}E7]do \[^\mn{B}]pro|\[\mnc{A}Am]fundo
  \mnendchorus
  \notesoff
  \beginchorus
    \ind Este |^segre^do pro|^fundo está em |^toda humani|^dade
    \ind Se to|dos se ^conhe|^cerem ^aqui |^dentro ^da ver|^dade
  \endchorus
  \begin{translation}
    Flower of the waters, from where do you come, where do you go?
    I will do my cleansing, my Father is in my heart
    \nextverse
    The House of my Father is in the heart of the world
    Where all Love exists, and there is a profound secret
    \nextverse
    This profound secret is in all of humanity
    If everyone knew themselves here within the Truth
  \end{translation}
  \begin{explanation}
    It is said that after \emph{Flor das Águas} was presented at a Santo Daime
    concentration work, Mestre asked everyone attending: ``Where is the heart of
    the world?'' Nobody answered. He asked again and still got no response. Then
    he said: ``The heart of the world is the sea!''
  \end{explanation}
\endsong


\beginsong{Brilho do Sol \\ Paiste auringon}[by={Padr. Sebastião},tags={Sun, Santo Daime},ph={IV, V},key={D},sks={D, E}]
  \audio{https://www.nossairmandade.com/hymn/99/BrilhoDoSol}
  \mnbeginchorus\memorize
    \lrep \[^\mn{A}]Eu |\[\mnc{D}D]sou \[^\mn{F#}]bri\[^\mn{E}]lho \[^\mn{D}]do \[\bm]sol, |sou \[^\mn{F#}]bri\[^\mn{E}]lho \[^\mn{D}]da \[\mnc{F#}Bm/F#]lua \rrep
    \lrep \[^\mn{F#}]Dou |\[\mnc{G}G]bri\[^\mn{F#}]lho \[^\mn{E}]as \[^\mn{D}]es\[\mnc{F#}D]trelas, \[^\mn{D}]por\[^\mn{F#}]que |\[\mnc{E}A7]todas \[^\mn{B}]me a\[^\mn{C#}]com\[\mnc{D}D]panham \rrep
  \mnendchorus
  \notesoff
  \beginchorus
    \lrep Eu |^sou brilho do ^mar eu |vivo no ven^to \rrep
    \lrep Eu |^brilho na flo^resta porque |^ela me per^tence \rrep
  \endchorus
  \textnote{suomeksi:}
  \beginchorus
    \lrep Olen |^paiste aurin^gon, |sekä loiste ^kuun \rrep
    \lrep Va|^laisen tähdis^tön, koska se |^on mun seura^nain \rrep
  \endchorus
  \beginchorus
    \lrep Olen |^välke meren^pinnan, hen|gitän tuules^sa \rrep
    \lrep Loistan |^metsän sydä^messä, sillä se |^kuuluu minul^le \rrep
  \endchorus
  \imagecc[3]{sun_bw_transparent_bg_1280px.png}%
\endsong


\beginsong{Força do Rapé}[by={Gustavo Soslaio Mello}, tags={rapé}]
  \beginchorus\memorize
    |\[\mnc{E}Em]E a \[^\mn{B}]força do Ra|pé
    Quero |\[Am]ver quem \[B7]vai ficar de |\[Em]pé
  \endchorus
  \notesoff
  \beginchorus
    |^Santa para levan|tar
    O e|^spiri^to do alto o|^lhar
  \endchorus
  \beginchorus
    |^Quem é o eu da ques|tao
    E o |^eu ^da immensi|^dao
  \endchorus
  \beginchorus
    |^Quem desperta e faz bri|lhar
    O e|^spiri^to do alto o|^lhar
  \endchorus
  \beginchorus
    O e|\[Am]spiri\[B7]to do alto o|\[Em]lhar
  \endchorus
  \begin{translation}
    And the strength of Rapé
    I want to see who will remain standing
    \nextverse
    Saint lifts up
    The spirit of the high view
    \nextverse
    Who is the I of the matter?
    And the self of immensity?
    \nextverse
    Who awakens and makes us shine?
    The spirit of the high view
  \end{translation}
\endsong


\beginsong{Na Força da Mata Eu Tomo meu Rapé}[tags={rapé}]
  \beginchorus\memorize
    \lrep \[^\mn{A}]Na \[^\mn{B}]for|\[\mnc{C}Am]ça da m\[E]ata eu tomo o |\[Dm]meu rap\[E]é \rrep
    \lrep Para |\[Am]me limp\[C]a, para me |\[E]proteg\[Am]er \rrep
  \endchorus
  \notesoff
  \beginchorus
    \lrep Nos ol|^hos da gra^nde Á|^guia dour^ada \rrep
    \lrep Sou o |^Corvo n^egro e vou vo|^ar també^m \rrep
  \endchorus
  \beginchorus
    \lrep Eu |^vou segui^ndo nes|^te cami^nho \rrep
    \lrep Quem vai |^me guia^ndo, não me dei|^xa sozi^nho \rrep
  \endchorus
  \beginchorus
    \lrep Salve |^esta fo^rça, a força |^do rap^é \rrep
    \lrep Que tem |^como Me^stre? Caboclos |^e Pajé^s \rrep
  \endchorus
  \begin{translation}
    In the strength of the forest I take my rapé
    To cleanse me, to protect me
    \nextverse
    In the eyes of the great Golden Eagle
    I'm the black Crow and I'll fly too
    \nextverse
    I'll follow in this way
    Who's leading me, do not leave me alone
    \nextverse
    Save this force, the strength of rapé
    What do you have as Master? Caboclos and Pajés
  \end{translation}
  \begin{explanation}
    \begin{description}
      \item[caboclo:] (in Umbanda religion) the spirit of an indigenous person of the Amazon
        rainforest
      \item[pajé:] a spiritual healer in certain Amazonian indigenous tribes
    \end{description}
  \end{explanation}
\endsong


\beginsong{Tomando Rapé}[tags={rapé}]
  \meter{4}{4}
  \beginchorus\memorize
    \[^\mn{E}]Toman|\[\mnc{A}Am]do \[^\mn{C}]Ra\[^\mn{E}]pé a for|\[\mnc{F}Fmaj7]ça \[^\mn{D}]che\[^\mn{E}]go\[E]u
    \lrep Se ma|\[E]mãe Jur\[E7]ema foi que |\[E]nos mando\[Am]u \rrep
  \endchorus
  \notesoff
  \beginchorus
    Toman|^do Uni para |^se cura^r
    \lrep Peço |^força o M^estre pra nos |^ajuda^r \rrep
  \endchorus
  \beginchorus
    No mei|^o das matas tocou |^meu tambo^r
    \lrep Se meu |^pai Ox^ossi Sara|^vá chego^u \rrep
  \endchorus
  \beginchorus
    Peco |^meu Jesus que me |^deu a lu^z
    \lrep Que me |^deu firm^eza pra nos |^caminha^r \rrep
  \endchorus
  \beginchorus
    A es|^trada é longa tenho |^fé de chega^r
    \lrep Nos pes |^de Jesus no ^Céu onde e|^le est^a \rrep
  \endchorus
  \begin{translation}
    Taking Rapé the strength has arrived
    Mother Jurema herself was the one who sent us
    \nextverse
    Taking Uni to heal
    I ask the strength of the Mestre to help us
    \nextverse
    In the middle of the woods he touched my drum
    My father Oxossi Saravá himself arrived
    \nextverse
    I ask my Jesus who gave me the light
    that gave me the strength to walk
    \nextverse
    The road is long I have faith to arrive
    by Jesus' feet in heaven where he is
  \end{translation}
\endsong


\beginsong{Força do Rapé 2}[tags={rapé, (chords missing)}]
  \meter{4}{4}
  \beginverse
    |Eu vim do Terreiro |do cobra coral
    |encontro dos guerreiros | com os seres do Astral
  \endverse
  \beginverse
    |Eu sou trabalhador |estou cheio d'amor
    |eu estou animado trazendo |esta flor
  \endverse
  \beginverse
    |Limpar os aparrelhos |limpar neste salão
    |Tirar os amaguras |ajudar os meus irmãos
  \endverse
  \beginverse
    |A força da floresta |os tesouros da mamae
    |o Mestre esta mostrando |a beleza do meu pai
  \endverse
  \beginverse
    |A força de jiboja |o misterio da papé
    |trazendo medicina |com a força do Rapé
  \endverse
\endsong


\beginsong{Salve o Rapé}[tags={rapé}]
  \audio[]{https://soundcloud.com/ceu-do-infinito-amor/salve-o-rape}
  \newchords{chords_salve_o_rape_a}\newchords{chords_salve_o_rape_b}
  \beginverse\memorize[chords_salve_o_rape_a]
    \[^\mn{C}]Salve \[^\mn{D}]Tu|\[\mnc{E}Am]pã, Ju\[^\mn{C}]ra\[^\mn{D}]mi|\[^\mn{E}]dam, sal\[^\mn{F}]ve \[^\mn{E}]O|\[\mnc{D}G]xós\[^\mn{B}]si | \e
    Salve Os|\[Am]sain, Tupinam|bá, Maraxim|\[G]bé | \e
  \endverse\glueverses
  \beginchorus\memorize[chords_salve_o_rape_b]
    Salve os Ca|\[F]boclos que trou|xeram da flo|\[Am]resta | \e
    A ener|\[G]gia cura|tiva do Ra|\[Am]pé | \e
  \endchorus
  \notesoff
  \beginverse\replay[chords_salve_o_rape_a]
    Salve o Ca|^valo, o Bem-te-|vi e o Beija-|^flor | \e
    Que vem dan|^çando e nos cu|rando com a|^mor | \e
  \endverse\glueverses
  \beginchorus\replay[chords_salve_o_rape_b]
    No sopro |^santo, a for|ça do Cria|^dor | \e
    Dá-nos a |^paz e leva em|bora toda |^dor | \e
  \endchorus
  \beginverse\replay[chords_salve_o_rape_a]
    Salve Je|^sus, salve Jo|sé, salve Ma|^ria | \e
    Salve a |^fé, a espe|rança e ale|^gria | \e
  \endverse\glueverses
  \beginchorus\replay[chords_salve_o_rape_b]
    Salve Mi|^guel que vem cu|rar com sua es|^pada | \e
    Salve o po|^der da nossa |linha unifi|^cada | \e
  \endchorus
  \beginverse\replay[chords_salve_o_rape_a]
    Salve o Ra|^pé, salve o Ra|pé, salve o Ra|^pé | \e
    Santo ta|^baco com a |bênção dos pa|^jés | \e
  \endverse\glueverses
  \beginchorus\replay[chords_salve_o_rape_b]
    Que vem liv|^rar de todo |mal a minha es|^trada | \e
    Que forta|^lece no meu |peito a minha |^fé | \e
  \endchorus
  \begin{translation}
    Hail Tupã, Juramidam, hail Oxóssi
    Hail Ossain, Tupinambá, Maraximbé
    Hail the Caboclos that brought from the forest
    The healing energy of Rapé
    \nextverse
    Hail the Horse, the Great Kiskadee \emph{(a bird)} and the Hummingbird
    Who have been dancing and healing us with love
    In the holy breath, the strength of the Creator
    Give us peace and take away all pain
    \nextverse
    Hail Jesus, hail Joseph, hail Mary
    Hail faith, hope and joy
    Hail Saint Michael who comes to heal with his sword
    Hail the power of our unified line
    \nextverse
    Hail the Rapé, hail the Rapé, hail the Rapé
    Holy tobacco with the blessing of the shamans
    That comes to rid my road of all evil
    That strengthens my faith in my chest
  \end{translation}
  \begin{explanation}
    \begin{description}
      \item[Tupã:] the supreme God of all creation in the mythology of the
        \emph{Guaraní} people of south-central part of South America
      \item[Juramidam:] ``God \emph{(jura)} and his soldiers \emph{(midam)}'',
        a name for the Santo Daime work personified, as disclosed to
        \emph{Mestre Irineu Serra} in a vision
      \item[Oxóssi:] Orixá of hunting and abundance
      \item[Ossain:] Orixá of sacred leaves, medicinal and liturgical herbs ---
        knows the secrets to all of them
      \item[Tupinambá:] a spirit that personifies \emph{Oxóssi}
    \end{description}
  \end{explanation}
\endsong


%%%%%%%%%%%%%%%%%%%%%%%%%%%%%%%%%%%%%%%%%%%%%%%%%%%%%%%%%%%%%%%%%%%
%%% LATEST PRINTOUT CONTAINED THE SONGS ABOVE.                  %%%
%%%%%%%%%%%%%%%%%%%%%%%%%%%%%%%%%%%%%%%%%%%%%%%%%%%%%%%%%%%%%%%%%%%
%%% Please try to not change the song numbers above this point. %%%
%%% Add new songs only after this point.                        %%%
%%%%%%%%%%%%%%%%%%%%%%%%%%%%%%%%%%%%%%%%%%%%%%%%%%%%%%%%%%%%%%%%%%%



    % Sanskrit language bhajans and mantras from the Indian subcontinent
% ==================================================================
%
% The following sets the song number for the first song in this file.
% The number will automatically be incremented by one for each song.
% Please do not change this! Changing would make different versions of
% the songbook to have different numbers for the same songs, and it
% would totally mess up the selection booklets causing them to have
% wrong songs in them. (For the same reason, add new songs only to the
% end of each songs_ file.)
\setcounter{songnum}{400}


\beginsong{Om Purnam \\ Purnamadah}[by={traditional, Shantala \& Satyaa \& Sari},ex={from Upaniṣad, the first prayer},ph={I, II},tags={unity}]
  \beginverse
    |\[Am]\[Am/B] |\[Am]\[Am/B] |\[Am]\[Am/B] |\[Am]\[Am/B]
  \endverse
  \beginverse
    |\[\mnc{A}Am]Pur\[\mn{B}]na\[\mn{C}]ma\[\mnc{B}Am/B]dah |\[\mnc{A}Am]Pur\[\mn{B}]na\[\mn{C}]mi\[\mnc{B}Am/B]dam
    |\[Am]Purnat \[Am/B]Purnama|\[Am]dachya\[Am/B]te
    |\[Am]Purnasya \[Am/B]Purna|\[Fmaj7]mada\[G]ya
    |\[Am] Purnam|evava\[Fmaj7]shishya|\[G6]te
  \endverse
  \beginverse
    |\[Am]Om |\[Fmaj7] \[G6]
    |\[Am] |\[Fmaj7] \[G6] |\[Gadd9] |\[G(5)]
  \endverse
  \begin{feeler}
    That is the whole. This is the whole.\\
    From wholeness emerges wholeness.\\
    Wholeness coming from wholeness,\\
    wholeness still remains. \\
  \end{feeler}
  \begin{explanation}
    \emph{``\ldots contains the secret of the mystic approach towards life. This small sutra contains the 
    essence of the Upanishadic vision. Neither before nor afterwards has the vision been 
    transcended; it still remains the Everest of human consciousness. The Upanishadic vision is 
    that the universe is a totality, indivisible; it is an organic whole. The parts are not 
    separate, we are all existing in a togetherness: the trees, the mountains, the people, the 
    birds, the stars, howsoever far away they may appear --- don't be deceived by the appearance --- 
    they are all interlinked, all bridged. Even the smallest blade of grass is connected to the 
    farthest star, and it is as significant as the greatest sun. Nothing is insignificant, nothing 
    is smaller than anything else. The part represents the whole, just as the seed contains the 
    whole\ldots''} --- Osho: Philosophia Ultima
  \end{explanation}
\endsong


\beginsong{Mahamrityunjaya Mantra \\ Oṃ Tryambakaṃ}[ph={III},tags={health, liberation}]
  \beginverse% \quad on the first line is there to not extend melody notes to the next bar
    |\[\mnc{E}C]Oṃ Tryamba\[\mn{D}]ka|\[G]ṃ \[\mn{C}]Ya\[\mn{D}]jā\[\mn{E}]mah|\[\mnciii{D}{C}{A}Dm]e \quad |\[Fmaj7] \e
    |\[C] Sugandhiṃ |\[G]Puṣṭivardha|\[Dm]nam | \e
    Ur|\[F]vārukam Iva |\[C]Bandhanān Mṛ|\[G]tyor | \e
    Muk|\[Dm]ṣīya Mā 'Mrtāt \echo{Muk|ṣīya Mā 'Mṛtāt}
    Muk|\[G]ṣīya Mā 'Mrtāt \echo{Muk|ṣīya Mā 'Mṛtāt}
  \endverse
  \begin{feeler}
    We Meditate on the Three-eyed reality\\
    Which permeates and nourishes all like a fragrance.\\
    May we be liberated from death for the sake of immortality,\\
    Even as the cucumber is severed from bondage to the creeper.
  \end{feeler}
\endsong


\beginsong{Mahamrityunjaya Mantra 2 \\ Oṃ Tryambakaṃ 2}[ph={III}, tags={health, liberation}, key={Am}, sks={Cm, Am--Em}]
  \beginverse
    |\[\mnc{A}Am]Oṃ \[\mn{B}]Tr\[\mn{C}]yam\[\mn{B}]ba\[\mn{A}]ka|ṃ \[\mn{B}]Ya\[\mn{C}]jā\[\mn{A}]ma\[\mn{B}]h|\[Em]e
    Sugandhiṃ |Puṣṭivardhana|\[Dm]m
    Urvārukam |Iva Bandhanā|\[G]n
    Mṛtyor |Mukṣīya Mā 'Mṛ|\[Am]tāt | \e
  \endverse
  \begin{explanation}
    Mahamrityunjaya Mantra (maha-mrityun-jaya) is one of the more potent of the ancient Sanskrit
    mantras. Maha Mrityunjaya is a call for enlightenment and is a practice of purifying the karmas
    of the soul at a deep level. It is also said to be quite beneficial for mental, emotional, and
    physical health.
  \end{explanation}
\endsong


\beginsong{Moola Mantra}[index={Om Satchitananda Parabrahma},tags={source},ph={I, II}]
  \beginchorus
    \[\mn{E}]Om |\[\mnc{A}Am]Satchitanan\[\mn{G}]da |\[\mn{A}]Para\[\mn{B}]brah\[\mn{A}]ma
    |Purushothama Para|matma
    Sri |\[F]Bhagavati Same|\[G]tha
    Sri |\[Am]Bhagavate Nama|ha
  \endchorus
  \beginchorus
    Hari |\[Am]Om Tat |Sat, Hari |\[G]Om Tat |Sat
    Hari |\[C]Om Tat |\[G]Sat, Hari |\[Fmaj7]Om Tat |\[Am]Sat
  \endchorus
  \begin{feeler}
    Oh Divine Force, Spirit of All Creation,\\
    Highest Personality, Divine Presence,\\
    manifest in every living being.\\
    Supreme Soul manifested\\
    as the Divine Mother and\\
    as the Divine Father.\\
    I bow in deepest reverence.\\
  \end{feeler}
  \begin{explanation}
    Moola mantra evokes the living God, asking protection and freedom from all sorrow
    and suffering. It is a prayer that adores the great creator and liberator, who out of love and
    compassion manifests, to protect us, in an earthly form.  The calmness that the mantra can
    give is to be experienced, not spoken about. Here is the key with which any door to spiritual
    treasure could be opened. A tool which can be used to achieve all desires. A medicine which
    cures all ills. Just like when you call a person he comes and makes you feel his presence, the
    same manner when you chant this mantra, the supreme energy manifests everywhere around you. As
    the Universe is Omnipresent, the supreme energy can manifest anywhere and any time. It is also
    very important to know that the invocation with all humility, respect and with great necessity
    makes the presence stronger.
    \begin{description}
      \item[Om:] Calling on the highest energy of all there is. It is said 'In the beginning was the
        Supreme word and the word created every thing. That word is Om'. If you are meditating in
        silence deeply, you can hear the sound Om within. The whole creation emerged from the sound
        Om. It is the primordial sound or the Universal sound by which the whole universe vibrates.
        This divine sound has the power to create, sustain and destroy, giving life and movement to
        all that exist.
      \item[Sat:] Truth. The formless. The all penetrating existence that is formless, shapeless,
        omnipresent, attributeless, and qualityless aspect of the Universe, experienced as emptiness
        of the Universe. The body of the Universe that is static. Everything that has a form and can
        be sensed evolved out of this. So subtle that it is beyond all perceptions. It can only be
        seen when it has become gross and has taken form. We are in the Universe and the Universe is
        in us. We are the effect and Universe is the cause and the cause manifests itself as the
        effect.
      \item[Chit:] The Pure Consciousness of the Universe that is infinite, omni-present
        manifesting power of the Universe. Out of this is evolved everything that we call Dynamic
        energy or force. It can manifest in any form or shape. It is the consciousness manifesting
        as motion, as gravitation, as magnetism, etc. Also manifesting as the actions of the body,
        as thought force. The Supreme Spirit.
      \item[Ananda:] Pure bliss, love, joy and friendship nature of the Universe. When you experience
        either the Supreme Energy in this Creation (Sat) and become one with the Existence or
        experience the aspect of Pure Consciousness (Chit), you enter into a state of Divine Bliss
        and eternal happiness (Ananda).
      \item[Parabrahma:] The Supreme creator being in his Absolute aspect; beyond space and time.
        The essence of the Universe that is with and without form.
      \item[Purushothama:] The energy that incarnates as an Avatar in human form to help and guide
        mankind and relate closely to the beloved creation.  This has different meanings. Purusha
        means soul and Uthama means the supreme, the Supreme spirit. It also means the supreme
        energy of force guiding us from the highest world. Purusha also means Man, and Purushothama
        is the energy that incarnates as an Avatar to help and guide Mankind and relate closely to
        the beloved Creation.
      \item[Paramatma:] Supreme inner energy that is immanent in every creature and in all beings,
        living and non-living. Who comes to me in my heart, and becomes my inner voice whenever I
        ask. It's the force that can come to you whenever you want and wherever you want to guide
        and help you.
      \item[Sri Bhagavathi:] The divine mother, the power aspect of creation. The female aspect,
        which is characterized as the Supreme Intelligence in action, the Power (The Shakti). It is
        referred to the Mother Earth (Divine Mother) aspect of the creation.
      \item[Sametha:] Together or in communion with.
      \item[Sri Bhagavathe:] The Male aspect of the Creation, which is unchangeable and permanent.
      \item[Namaha:] Salutations or prostrations to the Universe that is Om and also has the
        qualities of Sat Chit Ananda, that is omnipresent, unchangeable and changeable at the same
        time, the supreme spirit in a human form and formless, the indweller that can guide and help
        in the feminine and masculine forms with the supreme intelligence. I thank you and
        acknowledge this presence in my life. I seek your presence and guidance all the time.
      \item[Hari om tat sat:] God is the truth. Hari is another name of Lord Vishnu.
    \end{description}
  \end{explanation}
\endsong


\beginsong{Gayatri Mantra}[index={Om Bhur Bhuvah Svaha\ldots},tags={wisdom, liberation},ph={III}]
  \beginverse
    \[Am] \[\mn{A}]Om |\[\mnc{B}G]Bhur Bhuvah \[\mn{C}]Sva|\[\mncii{B}{A}Am]ha
    Tat |\[G]Savi|\[Am]tur Va|\[G]ren|\[F]yam
    Bhar|\[G]gho Devasya |\[C]Dheemahi
    Dhi|\[G]yo Yo |\[C]Nah Pra|\[G]choda|\[Am]yat
  \endverse
  \begin{explanation}
    A prayer of praise that awakens the vital energies and gives liberation and deliverance from
    ignorance. This mantra is known to impart wisdom, understanding, and enlightenment. This is
    said to be the oldest and most powerful of mantras, being thousands of years old. It purifies
    the person chanting it as well as the listener as it creates a tangible sense of well being in
    whoever comes across it.

    We meditate on that most adorable, desirable and enchanting luster and brilliance of our
    Supreme Being, our Source Energy, our Collective Consciousness who is our creator, inspirer
    and source of eternal Joy.  May this Light inspire and guide our mind and open our hearts.
    That Divine Illumination which pervades the physical plane, astral plane and the celestial
    plane. That which is the most adorable. On that Divine Radiance we Meditate. May that
    Enlighten our Intellect and Awaken our Spiritual Wisdom.
    \brk
    \paragraph{\small Om Bhur Bhuvah Svaha:} Preamble to the main mantra; means that we invoke in our prayer
      and meditation the One who is our inspirer, our creator and the abode of supreme Joy.  It also
      means, we invoke the earthly, physical world, the world of our mind, and the world of our
      soul.
    \begin{description}
      \item[\hspace{2em} Om:] God/Brahma, Divine Illumination which pervades
      \item[\hspace{2em} Bhur:] Pranic energy
      \item[\hspace{2em} Bhuvah:] Destroyer of sufferings
      \item[\hspace{2em} Svaha:] Happiness bestowing
    \end{description}
    \paragraph{\small Tat Savitur Varenyam:} Divine Illumination, which is the Most Adorable
    \begin{description}
      \item[\hspace{2em} Tat:] THAT, denoting the Supreme Being, God or Spirit
      \item[\hspace{2em} Savitur:] The radiating source of life with the brightness of the Sun. Bright Sun/God
        (also a deva some call upon using this mantra)
      \item[\hspace{2em} Varenyam:] Most adorable, most desirable, greatest
    \end{description}
    \paragraph{\small Bhargho Devasya Dheemahi:}
    \begin{description}
    \item[\hspace{2em} Bhargho:] Luster and splendor; destroyer of misdeeds
      \item[\hspace{2em} Devasya:] Divine or Supreme God
      \item[\hspace{2em} Dheemahi:] ``We meditate upon''; knowledge imparted/understood
    \end{description}
    \paragraph{\small Dhiyo Yo Nah Prachodayat:}
    \begin{description}
      \item[\hspace{2em} Dhiyo:] Our understanding of reality, our intellect, our intention; Intelligence
      \item[\hspace{2em} Yo:] He who
      \item[\hspace{2em} Nah:] Our
      \item[\hspace{2em} Prachodayat:] May he inspire, guide; enlightenment
    \end{description}
  \end{explanation}
  \yesendsongvfill
\endsong


\beginsong{Om Asato Ma \\ Pavamana Mantra}[ex={from Bṛhadāraṇyaka Upaniṣad},tags={transcendence},ph={III}]
  \beginverse
    |\[\mnc{A}Am\mn{G}]O|\[\mnc{E}Em]om asa\[^\mn{A}]to \[^\mn{B}]ma |\[\mnc{A}Am]sat ga\[^\mn{G}]ma\[^\mn{E}]ya
    |\[Em] Tamaso ma |\[C]jyotir gama|\[G]ya
    Mrit|\[Em]yor ma amri|\[Am]tam gamaya | \e
  \endverse
  \notesoff
  \textnote{suomeksi:}
  \beginverse
    |^O|^om Johda minut |^totuuteen
    |^ Pimeydestä |^va|^loon
    Tiedot|^tomuudesta tietoi|^suuteen | \e
  \endverse
  \begin{translation}
    Lead me from illusion to reality \emph{(of eternal self)},
    from darkness \emph{(ignorance)} to light \emph{(spiritual understanding)},
    from \emph{(the world of)} death to immortality \emph{(of self-realization)}.
  \end{translation}
\endsong


\beginsong{Prabhu Aap Jago}[by={Carioca},tags={transcendence},ph={III}]
  \newchords{chords_prabhu_a}\newchords{chords_prabhu_b}
  \beginchorus\memorize[chords_prabhu_a]
    \[^\mn{C}]Pra\[^\mn{D}]bhu |\[\mnc{E}C]Aap Jago \[^\mn{D}]Pra\[^\mn{E}]bhu |\[\mnc{F}Fmaj7]Aap \[^\mn{E}]Ja\[^\mn{C}]go
    Prabhu |\[Dm]Aap Jago Para|\[G]mathma Jago
  \endchorus
  \notesoff
  \beginverse\memorize[chords_prabhu_b]
    Mere |\[Em]Sarva Jago Sar|\[Am]vatra Jago
    Prabhu |\[Dm]Aap Jago Para|\[G]mathma Jago
    Mere |\[Em]Sarva Jago Sar|\[Am]vatra Jago
    Prabhu |\[Dm]Aap Jago Para|\[G]mathma Ja|\[C]go | \e
  \endverse
  \textnote{in English:}
  \beginchorus\replay[chords_prabhu_a]
    |^Cease the cause of |^suffering
    Illumi|^nate the cause of |^joy
  \endchorus
  \beginverse\replay[chords_prabhu_b]
    |^Cease the cause of |^suffering
    Illumi|^nate the cause of |^love
    |^Cease the cause of |^suffering
    Illumi|^nate the |^cause of |^love | \e
  \endverse
  \begin{feeler}
    God awaken, God awaken in me, God awaken everywhere.\\
    May love awaken; may love awaken everywhere.
  \end{feeler}
\endsong


\beginsong{Shakti Kundalini \\ Om Mata Om Kali}[tags={Divine Mother},ph={I, IV}]
  \meter{4}{4}
  \beginchorus
    \[\mn{F}]Om |\[\mncii{E}{D}Dm]Mata \[\mn{F}]Om |\[\mn{E}\mn{D}]Kali
    |\[C]Durga Devi na|\[Dm]mo namaha
  \endchorus
  \beginchorus
    |\[Dm]Shakti kunda|\[C]lini |\[B&]Jagadambe Ma|\[A]ta
    |\[Dm]Shakti kunda|\[C]lini |\[B&]Jagadam\[A]be Ma|\[Dm]ta
  \endchorus
  \begin{feeler}
    I bow unto the Divine Mother and Her many feminine aspects: Kali, remover of delusion and
    ignorance; Divine Goddess Durga; Shakti, universal life force and consort to Shiva; and
    Kundalini, the Goddess energy that rises within us. Praise to the Mother of the World!
  \end{feeler}
\endsong


\beginsong{Jay Shri Ma \\ Ananda Ma \\ Kali Ma}[tags={Divine Mother},ph={IV}]
  \beginchorus
    \[\mn{D}]Jay \[\mn{F}]Shri |\[\mnc{A}F]Ma Kali Ka\[\mn{G}]li |\[Gm]Ma
    Jay Shri |\[Dm]Ma | \e
  \endchorus
  \beginchorus
    |\[F] Ananda Ma |\[C]Durga Devi
    |\[Gm]Jagadambe Shri |\[Dm]Ma
  \endchorus
\endsong


\beginsong{Jay Ambe}[tags={Divine Mother},ph={II, IV}]
  \beginchorus
    |\[\mnc{D}Dm]Jay Am\[\mn{A}]be |\[\mnc{G}C]Jaga\[\mn{A}]dam\[\mnc{E}Am]be
    |\[F]Mata Bha\[C]vani ki |\[Dm]Jay Ambe
  \endchorus
  \beginchorus
    |\[Dm] Durgati Nashini |\[F]Durga Jaya Jaya
    |\[C] Kala Vinashini |\[Dm]Kali Jaya Jaya
  \endchorus
  \beginchorus
    |\[C]Uma Rama Brah|\[F]mani Jaya Jaya
    |\[C]Radha \[Am]Rukamani |\[Dm]Sita Jaya Jaya
  \endchorus
\endsong


\beginsong{Saraswati}[tags={Divine Mother, learning},ph={II, III}]
  \beginverse
    \[\mn{D}]Sa\[\mn{E}]ras|\[\mncii{F}{E}Dm]wa\[\mn{D}]ti | \[\mn{D}]Ma\[\mn{E}]ha|\[\mn{F}\mn{E}]lax\[\mn{D}]mi | \e
    Durga |\[F]Devi |\[C]Nama|\[Dm]ha | \e
  \endverse
  \beginverse
    Saras|\[Gm]wati | Maha|\[F]laxmi | \e
    Durga |\[B&]Devi |\[C]Nama|\[Dm]ha | \e
  \endverse
\endsong

\beginsong{Devi Mantra \\ Sarva Mangala}[tags={Divine Mother, Shiva},ph={III, IV}]
  % \capo{3}
  \textnote{part A: Devi Mantra}
  \beginchorus
    |\[\mnc{A}Am]Sarva Mangala |Man\[\mn{C}]ga\[\mn{A}]ly\[\mnc{B}Em]e
    |\[G]Shive Sarv|\[Em]artha Sadhi\[Am]ke
    |Sharanye Tryambake |Gau\[Em]ri
    Nara|\[Am]yan\[G]i Na|\[Em]mostut\[Am]e
    Nara|\[Am]yan\[G]i Na|\[Em]mostut\[Am]e | | \e
  \endchorus
  \textnote{part B: Om Namah Shivaya}
  \beginchorus
    \ind |\[Am]Om Namah Shivaya, |\[C]Om \[G]Nama Shiva \rep{3}
  \endchorus\glueverses\beginverse
    \ind |\[Am]Om Namah Shivaya, |\[C]Om \[G]Nama Shiv|\[Am]aya \up{1}(| \e)
  \endverse
    \textnote{\emph{D.C. al Fine}}
  \beginchorus
    \ind |\[Am]Shivaya, Shivaya, |\[Em]Shivay\[Am]a \rep{3}
  \endchorus\glueverses\beginverse
    \ind |Shivaya, Shivaya
  \endverse\glueverses\beginchorus
    \ind |\[C]Om \[G]Namah Shiv|\[Am]aya
  \endchorus
  \begin{feeler}
     \textbf{Devi Mantra:}\par
     Welcome to you O Narayani; who is the positiveness in all the auspicious,
     one who is so auspicious herself and has all auspicious qualities,
     The provider of protection, the one with three eyes and a beautiful face;
     we salute you, O Narayani.
  \end{feeler}
\endsong


\beginsong{Mataji}[by={trad., Elisabet Just},tags={Divine Mother},ex={saṃskṛtam, português},ph={IV}]
  \audio[]{https://www.youtube.com/watch?v=\_UEUQZsaDZ4}
  %\capo{3}
  \beginverse
    |\[\mnc{A}Am]Ayi Ayi Ayi Ayi Di|wa\[\mn{B}]li \[\mn{A}]Hai \[\mn{G}]Ye \[\mn{A}\mn{B}\mn{C}]Ayi
    |\[G]Aise Shubhawa\[C]sar Par Hum |\[Em]Puje Maha \[Am]Lakshmi
    |\[Am]Sabhi Devi Devata |Aapa Hi Ko Puje
    |\[C]Nirmala \[E]Ma, O Maiya |\[G]Nirmala \[Am]Ma
  \endverse\glueverses
  \beginchorus
    O Chindra|wara, Wali Maha, Laksh|\[(C)]mi Mata\[Am]ji
  \endchorus
  \beginchorus
    Ô \sublyr{\up{2}(mi)}iê iê i|\sublyr{Ayi Ayi Ayi}\[Am]ê, ô \sublyr{Ayi}iê \sublyr{Di-}iê |\sublyr{wali}Xo\sublyr{Hai}roo\sublyr{Ye}dô \sublyr{A-}
    Ô \sublyr{y-}iê \sublyr{i}iê i|\sublyr{Ayi Ayi Ayi}\[Am]ê, ô \sublyr{Ayi}iê \sublyr{Di-}iê |\sublyr{wali}Xo\sublyr{Hai}roo\sublyr{Ye}dô \sublyr{A-}
    Olomi ai|\sublyr{Aise}\[G]ê \sublyr{Shubhawa}Xorô \sublyr{sar}\[C] \sublyr{Par}Ó\sublyr{Hum}manfé|\sublyr{Pu-}\[Em]é \sublyr{je}Xo\sublyr{Ma-}roo \sublyr{ha}dô\sublyr{Laksh-}\[Am]
  \endchorus\glueverses
  \beginverse
    Ô \sublyr{mi}iê iê i|\sublyr{Sabhi}\[Am]ê, \sublyrpush{Devi De}\sublyr{va-}ô \sublyr{ta}iê iê |\sublyr{Aapa}Xo\sublyr{Hi Ko}roodô\sublyr{Puje}
    Ô iê iê i|\sublyr{Nir-}\[C]ê, \sublyrpush{mala} \sublyr{Ma,}\[E] \sublyr{O}ô \sublyr{Mai-}iê \sublyr{ya}iê |\sublyr{Nir-}\[G]Xo\sublyr{mala}roodô\sublyr{Ma}\[Am]
  \endverse\glueverses
  \beginchorus
    Ô iê iê i|\sublyr{Ayi Ayi Ayi}\[Am]ê, ô \sublyr{Ayi}iê \sublyr{Di-}iê |\sublyr{wali}Xo\sublyr{Hai}roo\sublyr{Ye}dô \sublyr{A(yi)}
  \endchorus
  \beginchorus\musicnote{decelerando}
    \sublyrpush{Ô iê iê i} |\sublyr{ê,}\[Am]Om Mani Pad\sublyr{ô}me \sublyr{iê iê}Hum |\sublyr{Xoroodô} \e
  \endchorus
  \beginchorus
    \ind \sublyrpush{\up{\textbf{1}}Ô iê iê i} |\sublyr{ê,}\[Am]Om Mani Pad\sublyr{ô}me \sublyr{iê iê}Hum |\sublyr{Xo}\[G]Om \sublyr{roo}Ma\sublyr{dô}ni Pad\sublyr{ô}me \sublyr{iê iê i}Hum
    \ind |\sublyr{ê}\[F]Om Mani \[Em]Pa\sublyr{ô iê iê}dme |\sublyr{Xoroodô}\[Am]Hum
    \rep{4}
  \endchorus
  \beginchorus
    |\[Am]Om Om Mani |\[G]Padme \[Am]Hum
    \rep{4}
  \endchorus
  \beginchorus
    \ind |\[Am]Om Mani Padme Hum |\[G]Om Mani Padme Hum
    \ind |\[F]Om Mani \[Em]Padme |\[Am]Hum
  \endchorus
  \beginchorus
    Eu |\sublyr{\up{\textbf{1}}Om}\[Am]vi ma\sublyr{Ma-}mãe \sublyr{ni}O\sublyr{Pad-}xum \sublyr{me}na \sublyr{Hum}cacho|\sublyr{Om}\[G6]ei\sublyr{Mani}ra \sublyrpush{Padme Hum}sen-
    |\sublyr{Om}\[F]tada \sublyr{Ma-}na \sublyr{ni}bei\sublyr{Pad-}\[G7]ra \sublyr{me}do r|\sublyr{Hum}\[Am]io
  \endchorus
  \beginchorus
    Colhendo |\[Dm]lírio, lírio lê \[G] colhendo
    |\[C]lírio, lírio lá \[F] colhendo
    |\[B\textdegree7]lírio pra enfei\[E7]tar o seu con|\[Am]gá
    \rep{4}
  \endchorus
  \goto{Om Mani Padme Hum}
  \begin{translation}
    The day of Diwali has arrived.
    On this auspicious occasion allow us to worship Shri Mahalaxmi.
    All the Gods and Goddesses worship you,
    oh Mahalaxmi Mataji Goddess of Chindwara.
    \nextverse
    I saw Mother Oxum at the waterfall
    sitting on the river's edge
    gathering lilies, there picking lilies,
    lilies to decorate her altar.
    % Image downloaded from: https://openclipart.org/detail/185904/lily
    % Original: Drinks of the World --- James Mew and John Ashton, 1892
    % Image license: Public Domain
    \imager[5]{lily_drawing_bw_transparent_bg_PD__1025px.png}%
  \end{translation}
  \begin{explanation}
    \begin{description}
      \item[Mataji:] a Hindi term meaning ``respected mother''
      \item[Diwali:] the yearly Hindu festival of lights, when prayers are offered to
        \textbf{Lakshmi}, goddess of prosperity and fortune
      \item[Oxum:] see song \emph{Ide Were} for explanation
    \end{description}
  \end{explanation}
\endsong


\beginsong{Shiva Shambho}[tags={Shiva},ph={II, IV}]
  \audio[]{https://www.youtube.com/watch?v=YBso7TPtvJU}
  \beginchorus
    |\[\mnc{A}Am]Jaya jaya Shiva \[\mn{B}]Sham|\[\mn{C}\mn{B}\mn{A}]bho, |\[\mnc{G}G]jaya jaya \[\mn{C}]Shiva \[\mn{B}]sham|\[\mnc{A}Am]bho
  \endchorus
  \beginchorus
    |\[Am]Mahadeva sham|bho, |\[G]Mahadeva sham|\[Am]bho
  \endchorus
  \begin{explanation}
    \begin{description}
      \item[Mahadeva:] a title for Lord Shiva, meaning ``Great God``
      \item[Shambho:] ``the auspicious one``
    \end{description}
  \end{explanation}
\endsong


\beginsong{Haidakandhi}[tags={Shiva, Vishnu},ph={III}]
  \meter{4}{4}
  \beginchorus
    |\[Am] \[\mn{A}]Om Namah \[\mn{B}]Shi|\[\mnc{C}F]va\[\mn{A}]ya \[\mn{A}]Na\[\mn{A}]mah |\[\mnc{E}C]Om |\[\mnc{D}G]Haida\[\mn{C}]kan\[\mn{B}]dhi
  \endchorus
  \notesoff
  \beginchorus
    |\[Am] Hari Hari |\[F] Hari Hari |\[C] Hari Hari |\[G]Shankara
  \endchorus
  \begin{explanation}
    \begin{description}
      \item[Hari] is a name for the supreme absolute in the \emph{Vedas} (also
      in \emph{Guru Granth Sahib} and many other sacred texts of South Asia).
      Hari refers to \emph{Vishnu} who takes away all the sorrows of his
      devotees.
      \item[Shankara] is one of the names for Shiva.
    \end{description}
  \end{explanation}
\endsong


\begin{intersong} % A quote from the Upanishads about the fourth state of consciousness
  \begin{feeler}
    ``There must be a fourth state beyond the waking, dreaming and dreamless states in which
    the absolute oneness of Brahman-Atman is what should be known.''\\
    --- \emph{The Upanishads}
  \end{feeler}
  \begin{explanation}
    \begin{description}
      \item[Brahman] is the highest Universal Principle, the Ultimate Reality. It is the final
        cause of all that exists.
      \item[Atman] is the true inner self, the soul, of an individual.
      \item[The Upanishads] are ancient Sanskrit texts that contain some of the central
        philosophical concepts of Hinduism.
    \end{description}
  \end{explanation}
\end{intersong}


\beginsong{Om Namah Shivaya}[tags={Shiva},ph={IV}]
  \beginchorus\memorize % memorize chords even though in 'chorus'
    \textnote{intro:}
    |\[\mnc{A}Am]Om Namah Shi|\[\mnc{C}F]vaya; |\[\mnc{D}G]Om Na\[^\mn{C}]mah \[^\mn{B}]Shi|\[\mnc{A}Am]vaya
  \endchorus
  \notesoff
  \beginchorus
    |^ Shivaya |^Namaha; |^ Shivaya |^Namaho
  \endchorus
  \beginchorus
    |^Sham Bol ^Shankara |^Namah ^Shivaya; \replay
    |^Girija ^Shankara |^Namah ^Shivaya
  \endchorus
  \beginchorus
    |^Aruna^chala Shiva |^Namah Shi^vaya; \replay
    |^Aruna^chala Shiva |^Namah ^Shivaya
  \endchorus
  \beginchorus
    Hari |\[C]Om Namah Shi|\[G]vaya; |\[F]Om Namah Shi|\[Am]vaya
  \endchorus
  \beginchorus
    \textnote{outro:}
    |^Om Namah Shi|^vaya; |^Om Namah Shi|^vaya
  \endchorus
  % Image downloaded from: https://imgbin.com/png/VxTaSrxf/shiva-hanuman-art-ganesha-sai-baba-of-shirdi-png
  % Image license: Free for non-commercial use
  \imagecc[3]{shiva_bw_transparent_bg_760x859px.png}%
\endsong


\beginsong{Ganesha Mantra \\ Removing of obstacles Mantra }[index={Om Gam Ganapatayei Namaha},by={Prembabanda},tags={Ganesha},ph={I, IV}]
  \showmantra{Om Gam Ganapatayei Namaha}
  \begin{feeler}
    Salutations to the remover of obstacles.
  \end{feeler}
  \begin{explanation}
    This sound formula assists us in the removal of obstacles. In order for that to happen there
    is no need to know the exact nature of the hindrances. Just the awareness and recognition that
    there are obstacles and then chanting this mantra with the intention for resolve is enough.
    This mantra unifies us within. When there is oneness there are no obstacles. This mantra is
    also used for the beginning of any endeavor. Whenever we start anything anew we can bless the
    project with the energy of Ganesh through this mantra.
    \vspace{2em}
    \begin{description}
      \item[Gam:] the seed sound for Ganesh
      \item[Ganapati:] another name for Ganesh --- the Remover of Obstacles, and of Oneness/Unity
      \item[Yei:] a sound that activates shakti/energy
    \end{description}
  \end{explanation}
  \imagecb[2]{ganesha_bw_transparent_background_1280x1232.png}%
  \textnote{song: part A}
  \beginchorus\memorize % memorize chords even though in 'chorus'
    |\[\mnc{E}Em]Om \[^\mn{B}]Parvati Patayei |\[\mnc{A}Bm]Hara Ha\[^\mn{F#}]ra \[^\mn{A}]Hara Ma\[^\mn{B}]ha\[^\mn{A}]de\[^\mn{G}]va
    |\[C] Gajana\[D]nam Bu|\[Em]ta
  \endchorus
  \notesoff
  \beginchorus
    |^Ganadi Sevatam |^ Kapitha Jambu
    |^ Phalacha^ru |^Bhakshanam
  \endchorus
  \beginchorus
    |^Umasutam Shoka |^ Vinasha Karakam
    |^ Namami ^Vigneshvara |^Pada Pankajam
  \endchorus
  \textnote{part B}
  \beginchorus
    |\[Em]Om Gam Ganapata|yei Nama\[Bm]ha
    |\[Em]Om Gam Ganapata|yei Nama\[Bm]ha
    |\[G]Om Gam Ganapata|yei Nama\[Bm]ha
    |\[Em]Om Gam Ganapata|yei Nama\[Bm]ha
  \endchorus
  \begin{translation}
    O elephant---faced God, Ganesha,
    You are served by the attendants of Shiva.
    \nextverse
    And you eat forest apples and blackberries.
    \nextverse
    You are \emph{Uma}'s son, the destroyer of sorrows.
    I bow to the lotus feet of the remover of obstacles.
  \end{translation}
  \begin{explanation}
    \begin{description}
      \item[Uma:] ``light'', Lady of the Mountains, also known as \emph{Parvati}
    \end{description}
  \end{explanation}
\endsong


\beginsong{Ganesha Sharanam}[tags={Ganesha},ph={I, IV}]
  \audio[]{https://soundcloud.com/sound-of-light/61-ganesh-sharanam}
  \beginverse
    |\[\mnc{E}Em]Om Bom |Hare
    Na|'maha Shi|\[\mnlow{G}\mnlow{F#}]va\[\mnlow{E}]ya
    |\[C] Ganga Par|vati Ma
    |\[D] Ganesha |\[B7]Sharanam
  \endverse
\endsong


\beginsong{Jaya Ganesha}[tags={Ganesha},ph={I, IV}]
  \audio[]{https://soundcloud.com/sound-of-light/60-jai-ganesh}
  \beginchorus
    |\[\mnc{C}C]Jaya Ga\[\mn{D}]nesha |\[\mn{E}]jaya Ga\[\mn{F}]nesha |\[\mn{E}]jaya Ga\[\mn{F}]ne\[\mn{E}]sha |\[G]de\[\mn{D}]va
  \endchorus
  \beginchorus
    |\[F]Mata jaki |\[G]Parvati |\[Dm]pita \[G]Maha|\[C]deva
  \endchorus
  \begin{translation}
    Glory to You, O Lord Ganesha!
    \nextverse
    Born of Parvati, daughter of the Himalayas, and the great Shiva.
  \end{translation}
\endsong


\beginsong{Govinda Hari Om}[tags={Vishnu, Krishna},ph={III},key={Am},sks={Am, Gm--C\shrp{}m}]
  % in Am the notes range from A to G'
  \meter{4}{4}
  \beginverse
    |\[\mnc{A}Am]Go\[\mn{E}\mn{D}]vin|\[Dm]da | \[\mn{C}]Ha\[\mn{B}]ri |\[\mnc{A}Am]Om Hari |\[\mnc{B}E]Hari | \altchords{\id[1]{(Bm)}|Bm |Em | - |Bm |F\shrp{}}
    |\[Am]Gopa|\[Dm]la | Hari |\[Am]Om|\[G] | \altchords{|Bm |Em | - |Bm |A}
    |\[C]Sada |\[Dm]Sadhana | Ananda |\[Am]Bhavana| \altchords{|D |Em | - |Bm}
    \endverse\glueverses\beginchorus
    |\[Dm]Vishnu |\[Am]Sadhana |\[E]Hari |\[Am]Om|\altchords{|Em |Bm |F\shrp{} |Bm}
  \endchorus
\endsong


\beginsong{Guru Brahma}[by={Adi Sankaracharya},tags={teacher},ph={II, III}]
  \audio[]{https://soundcloud.com/bastiaan-yansa/guru-brahma}
  \transpose{5}
  \beginverse
    \[\mn{E}]Gur|\[C]ur Brahm|\[Am]a Gur|\[\mnc{F#}B7]ur \[\mn{E}]Vishn|\[\mnc{G}Em]u
    Gur|\[C]ur Dev|\[Am]o Mah|\[B7]eshwar|\[Em]ah
    Gu|\[C]ru Saak|\[Am]shaat Par|\[B7]a Brahm|\[Em]a
    Tas|\[C]mai S|\[Am]hri Guruv|\[B7]e Namah|\[Em]a | \e
  \endverse
  \begin{translation}
    Guru that is Brahma, Guru that is Vishnu,
    Guru that is Lord Maheshwara \emph{(Shiva)}.
    Guru that is verily the supreme reality.
    Sublime prostrations to that Guru.
  \end{translation}
  \begin{explanation}
    This \emph{shloka} (category of verse line developed from the Vedic Anuṣṭubh poetic meter)
    is by \emph{Adi Sankaracharya} (788--820), a Hindu mystic, as a part of \emph{Guru strotam},
    a sacred prayer dedicated to his spiritual guide \emph{Govinda Bhagwadpada}.
  \end{explanation}
\endsong


\beginsong{Om Shanti Om \\ Peace Mantra}[tags={peace},ph={I, II},key={Am},sks={Am, Am--D\shrp{}m}]
  % in Am the notes range from G to F'
  \mnbeginchorus
    |\[\mnc{A}Am]Om |\[\mnc{G}Em]Shan\[\mn{C}\mn{B}]ti |\[\mnc{A}Am]Om | \e \altchords{\id[1]{(Bm)}|Bm |F\shrp{}m |Bm | \e}
    \mnendchorus\glueverses\mnbeginverse
    \[\mn{A}]Om |\[\mnc{D}Dm]Shanti |\[\mnc{F}Dm/F]Shanti |\[\mnc{E}Am/E]Shan\[\mn{D}]ti\[\mn{E}]hi | \e \altchords{|Em |Em/G |Bm/F\shrp{} | \e}
  \mnendverse
  \notesoff
  \textnote{outro:}
  \beginverse
    |\[Am]Om |\[Em]Shanti |\[Am]Om | \e \altchords{|Bm | F\shrp{}m | Bm | \e}
  \endverse
  \begin{feeler}
    Peace in my heart, peace with each other, peace in the cosmos.
  \end{feeler}
\endsong


\beginsong{Sudhossi Budhossi \\ Forever Pure}[by={trad., Shimshai},ex={saṃskṛtam, english, español}, tags={liberation, transcendence},ph={III}]
  \audio[]{https://www.youtube.com/watch?v=AMH7WaiLqWk}
  \newchords{sudhossi_a}\newchords{sudhossi_b}
  \meter{3}{4}
  \beginverse\memorize[sudhossi_a]
    \[^\mn{B}]Su|\[\mnc{E}Em]dhos\[^\mn{F#}]si \[^\mn{E}]bu|\[\mnc{G}C]dhossi ni|\[\mnc{A}D]ran\[^\mn{G}]ja\[^\mn{F#}]no|\[\mnc{E}Em]si
    Sam|\[Am]sara may|\[Em]a pari|\[D]var jito|\[Em]si
  \endverse\glueverses\beginchorus\memorize[sudhossi_b]
    Sam|\[Am]sara swapa|\[Em]nam tria|\[Am]ja mohan ni|\[Em]dram
    Na|\[Am]jamna mri|\[C]tyor tat|\[D]sat swa ru|\[Em]pe
  \endchorus
  \notesoff
  \beginchorus
    \ind |\[D]Na na na\ldots |\[Em] |\[D] |\[Em]
  \endchorus
  \beginverse\replay[sudhossi_a]
    You |^are forever |^pure, you |^are forever |^true
    And the |^dream of this |^world can |^never touch |^you
  \endverse\glueverses\beginchorus\replay[sudhossi_b]
    So |^give up your at|^tachment and |^give up your con|^fusion
    And |^fly to that |^space that's be|^yond all il|^lusion
    % % Alternate last line (both been used by the author):
    %And a|^bide in the |^truth that's be|^yond all il|^lusion
  \endchorus
  \goto{Na na na}
  \beginverse\replay[sudhossi_a]
    E|^res siempre |^puro e|^res verda|^dero
    Y el |^sueño del |^mundo no |^te toca|^rá
  \endverse\glueverses\beginchorus\replay[sudhossi_b]
    De|^ja los a|^pegos de|^ja la confu|^sión
    Y vi|^ve en la ver|^dad más al|^lá de la ilu|^sión\goto{Na na na}
  \endchorus
  \begin{feeler}
    It is said that this Sanskrit mantra was originally sung every night as\\
    a lullaby by an Indian mother to her 12 children who all became sadhus.
  \end{feeler}
\endsong


\beginsong{Dhanvantre Mantra \\ Healing Mantra}[index={Om Shree Dhanvantre},tags={health},ph={III}]
  \showmantra{Om Shree Dhanvantre Namaha}
  \begin{feeler}
    Salutations to the being and power of the Celestial Healer.
  \end{feeler}
  \begin{explanation}
    \textbf{Dhanvantari} is the celestial healer. This mantra helps us find the right path to 
    healing, or directs us to the right health practitioner. In India it is also commonly chanted 
    during cooking in order for the food to be charged with healing vibrations – either to prevent 
    disease or assist in healing for those who are sick. This mantra can be chanted for any 
    situation that one would like to be healed or remedied. Good to remember and be open to the 
    path of healing not necessarily looking the way we expect it!
  \end{explanation}
\endsong


\beginsong{Om Namo Bhagavate Vasudevaya}[tags={liberation},ph={II, III}]
  \showmantra{Om Namo Bhagavate Vasudevaya}
  \begin{feeler}
    Salutations to the Indweller who is omnipresent, omnipotent, immortal and divine.
  \end{feeler}
  \begin{explanation}
    \textbf{Vasudeva} is the individual aspect of God that dwells inside of us. This mantra frees 
    our minds and spirits from negative patterns in this life. Regular and consistent practice of 
    this mantra gives us a complete spiritual freedom: it frees us from the cycle of rebirth and 
    helps us realize ourselves as a manifestation of transcendent divinity. It can also help bring 
    in an advanced spiritual soul if chanted by the mother during pregnancy.
  \end{explanation}
\endsong


\beginsong{Hari Om Shiva Om\ldots \\ Cosmic vibration Mantra}[tags={Vishnu, Shiva},ph={II}]
  \showmantra{Hari Om Shiva Om Shiva Om Hari Om}
  \begin{explanation} 
    \textbf{Hari} is another name of Lord Vishnu. Can also be translated as The Remover of ego. 
    Universal mantra of cosmic vibration.
  \end{explanation}
\endsong


\beginsong{Om Eim Saraswatyei Namaha}[tags={learning},ph={II}]
  \showmantra{Om Eim Saraswatyei Namaha}
  \begin{explanation}
    Salutations to Saraswati, the goddess of music, poetry, the arts, education, 
    learning and divine speech. Opens us towards education, learning, and the artistic world of 
    music and poetry. Whenever you find yourself moved to tears by a piece of music, or touched 
    by the words of the great poets and sages, you are in the presence of Saraswati. May we be 
    at ease while learning the wonders of the unfolding mystery of Life.
  \end{explanation}
\endsong


\beginsong{Om Namo Narayana}[tags={transcendence},ph={II}]
  \showmantra{Om Namo Narayana}
  \begin{explanation}
    I bow to the divine. Salutes the all-pervading aspect of the Great Spirit anchored 
    in our hearts and in all beings. Destroys barriers, obstacles, afflictions, and difficulties. 
    Leads to self-realisation. Traditionally chanted to assist the dying as they make their 
    transition, the mantra asks prayerfully, that we may all merge into the grace of divine light.
  \end{explanation}
\endsong


% Image to show on the empty recto page (remove if no longer empty!)
\begin{intersong}%
  \subsection*{Chakras}
  \imagecc[1]{Chakras_map_997x1132px.png}%
  \begin{description}
    \item[Sahasrāra:] ``thousand-petaled'', the crown chakra
    \item[Ājñā:] ``command'', the third eye chakra
    \item[Viśuddha:] ``especially pure'', the throat chakra
    \item[Anāhata:] ``unstruck'', the heart chakra
    \item[Maṇipūra:] ``jewel city'', the solar plexus chakra
    \item[Svādhiṣṭhāna:] ``one's own base'', the sacral chakra
    \item[Mūlādhāra:] ``root support'', the root chakra
  \end{description}
\end{intersong}


\beginsong{Rigveda: Soma}[by={translated to Finnish by Klaus Karttunen},ex={from Rigveda (book 8, hymn 48), ca 1000 BCE}]
  \chordsoff % no vertical space for non-existing chords
  \normalsize % to fit on one spread
  {\noindent\textbf{Somalle:}}\vspace{2em}
  \beginverse
    Makeasta ravinnosta olen ollut osallinen, viisaana,
    hyvin huolehtivasta, parhaan vapauden löytäjästä,
    jonka luo kokoontuvat jumalat ja ihmiset,
    medeksi he sitä nimittävät.
  \endverse
  \beginverse
    Kun olet tunkeutunut sisään, olet kuin Aditi,
    jumalallisen vihan karkottaja,
    oi mehu, Indran kumppanuudesta iloiten
    aja meidät rikkauteen kuin tottelevainen tamma valjaisiin.
  \endverse
  \beginverse
    Olemme juoneet Somaa, olemme tulleet kuolemattomiksi,
    olemme menneet valoon, olemme löytäneet jumalat.
    Mitä nyt meille tekisi vihamielisyys?
    Mitä kuolevaisen pahuus, oi kuolematon?
  \endverse
  \beginverse
    Siunaukseksi tule juotuna sydämellemme, oi mehu,
    ystävällinen ole kuin isä pojalle, oi Soma,
    ymmärtävä kuin ystävä ystävälle, laajamaineinen,
    pidennä elämäämme, Soma, elääksemme.
  \endverse
  \beginverse
    Juotuani nämä ihanat, vapauttavat mehut,
    niveliäni ne sitovat kuin vaunuja nahkahihnat,
    suojelkoot ne minua jalan murtumalta,
    pitäkööt mehut minut erossa katkeamisesta.
  \endverse
  \beginverse
    Sytytä minut kuin sytytetty tuli,
    tee kauasnäkeväksi, tee meidät paremmiksi,
    sillä silloin sinun juopumuksessasi, Soma,
    kuin rikkaana itseäni pidän --- etene menestykseksi.
  \endverse
  \beginverse
    Sinusta puserretusta, innokkain mielin
    nauttisimme kuin perintöomaisuudesta;
    kuningas Soma, pidennä elämäämme,
    kuin Aurinko keväisiä päiviä.
  \endverse
  \beginverse
    Kuningas Soma, ole meille lempeä onneksemme,
    sinun palvojiasi me olemme, tiedä se!
    Nousee kyky ja into, oi mehu,
    älä anna meitä pois vihollisen tahdosta.
  \endverse
  \beginverse
    Sillä sinä olet ruumiittemme suojelija, Soma,
    miesten valvojana olet asettunut kaikkiin jäseniin,
    jos me sinun lupauksesi rikkoisimme,
    ole meille lempeä, hyvä ystävä onneksemme, oi jumala.
  \endverse
  \beginverse
    Olisimmepa lempeän ystävämme kanssa,
    älköön hän minua juotuna vahingoittako, keltakasvoinen,
    tämä Soma, joka on asetettu meihin,
    siksi menen pyytämään Indralta elämäni pidennystä.
  \endverse
  \beginverse
    Pois ovat menneet uupumukset, sairaudet
    vavisten hävisivät, pimentäjättäret ovat pelästyneet,
    mahtava Soma on meihin noussut,
    olemme tulleet sinne, missä elämä pidentyy.
  \endverse
  \beginverse
    Mehu joka juotuna sydämiimme, oi isät,
    kuolematon kuolevaisiin on saapunut,
    sitä Somaa vuodatuksi haluamme kunnioittaa,
    olla hänen armossaan ja suopeamielisyydessään.
  \endverse
  \beginverse
    Sinä, Soma, olet yhdessä isien kanssa
    ulottanut itsesi halki taivaan ja maan,
    sellaista sinua me vuodatuksin haluamme kunnioittaa,
    oi mehu, olisimmepa rikkauksien valtiaita.
  \endverse
  \beginverse
    Puhukaa puolestamme, te suojelevat jumalat,
    älköön meitä uni kukistako, älköön tyhjä puhe,
    me Somalle joka päivä rakkaina
    kokouksessa haluamme puhua hyvine poikinemme.
  \endverse
  \beginverse
    Sinä meille, Soma, joka puolelta voiman antaja,
    sinä valon löytäjä, astu meihin, miesten valvoja,
    sinä meitä, oi mehu, apulaistesi kanssa
    suojele takaapäin ja myöskin edestäpäin.
  \endverse
  \begin{explanation}
    \begin{description}
      \item[Rigveda] on kokoelma varhaisia intialaisia uskonnollisia hymnejä. Sitä pidetään
        vanhimpana Intian pyhistä teksteistä; se lienee laadittu joskus 1700--1000 eaa.
      \item[Soma] on kasviperäinen enteogeeninen rituaalijuoma, jonka sisältämistä kasveista
        ei ole varmaa tietoa. Rigvedassa Somalle on osoitettu lukuisia hymnejä.
    \end{description}
  \end{explanation}
\endsong


%%%%%%%%%%%%%%%%%%%%%%%%%%%%%%%%%%%%%%%%%%%%%%%%%%%%%%%%%%%%%%%%%%%
%%% LATEST PRINTOUT CONTAINED THE SONGS ABOVE.                  %%%
%%%%%%%%%%%%%%%%%%%%%%%%%%%%%%%%%%%%%%%%%%%%%%%%%%%%%%%%%%%%%%%%%%%
%%% Please try to not change the song numbers above this point. %%%
%%% Add new songs only after this point.                        %%%
%%%%%%%%%%%%%%%%%%%%%%%%%%%%%%%%%%%%%%%%%%%%%%%%%%%%%%%%%%%%%%%%%%%


    % Buddhist mantras
% ================
%
% The following sets the song number for the first song in this file.
% The number will automatically be incremented by one for each song.
% Please do not change this! Changing would make different versions of
% the songbook to have different numbers for the same songs, and it
% would totally mess up the selection booklets causing them to have
% wrong songs in them. (For the same reason, add new songs only to the
% end of each songs_ file.)
\setcounter{songnum}{470}


\beginsong{Compassion Mantra \\ Mantra of Avalokiteshvara}[index={Om Mani Padme Hum},tags={compassion},ph={I, II}]
  \showmantra{Om Mani Padme Hum}
  \vspace{1em}
  \textnotefornext{in Pāḷi:}
  % move the next one up (this is a special case of two \showmantra macros):
  \vspace{-2em}
  \showmantra{Om Mani Peme Hung}
  \begin{feeler}
    OM,\\
    the jewel (method; MANI)\\
    in the lotus (wisdom; PADME)\\
    indivisible (HUM).\\\vspace{1em}
    Hail to the Jewel in the Lotus.
  \end{feeler}
  \begin{explanation}
    % From a lecture given by The Dalai Lama at the Kalmuck Mongolian Buddhist
    % Center, New Jersey:
    \textbf{The 14th Dalai Lama:} ``It is very good to recite the mantra OM
    MANI PADME HUM, but while doing it, think the meaning of the six syllables
    which is great and vast.
    \par
    The first, OM [\ldots] symbolizes the practitioner's impure body, speech,
    and mind; it also symbolizes the pure exalted body, speech, and mind of a
    Buddha. [\ldots]
    \par
    The path is indicated by the next four syllables. MANI, meaning jewel,
    symbolizes the factors of method: the altruistic intention to become
    enlightened, compassion, and love. [\ldots]
    \par
    The two syllables, PADME, meaning lotus, symbolize wisdom. Just as a lotus
    grows forth from mud but is not sullied by the faults of mud, so wisdom is
    capable of putting you in a situation of non-contradiction where as there
    would be contradiction if you did not have wisdom. [\ldots]
    \par
    Purity must be achieved by an indivisible unity of method and wisdom,
    symbolized by the final syllable HUM, which indicates indivisibility. [\ldots]
    \par
    Thus the six syllables mean that in dependence on the practice of a path
    which is an indivisible union of method and wisdom, you can transform your
    impure body, speech, and mind into the pure exalted body, speech, and mind
    of a Buddha. [\ldots]''
    % % Commented out for ..mm.. reasons
    %\par
    %\textbf{Lama Thubken Trinley:} ``These six syllables prevent rebirth into
    %the six realms of cyclic existence. It translates as 'OM the jewel in the
    %lotus HUM'. OM prevents rebirth in the God realm, MA prevents rebirth in
    %the Asura (Titan) realm, NI prevents rebirth in the Human realm, PA
    %prevents rebirth in the Animal realm, ME prevents rebirth in the Hungry
    %Ghost realm, and HUM prevents rebirth in the Hell Realm.''
  \end{explanation}
\endsong


\beginsong{Jewel in the Lotus Flower}[index={Om Mani Padme Hum},tags={compassion},ph={I, II}]
  \meter{4}{4}
  \beginverse
    \[\mn{A}]There's \[^\mn{C}]a |\[\mnc{D}Dm]jewel in \[^\mn{C}]the \[^\mn{D}]Lo\[^\mn{C}]tus |\[^\mn{D}]flower
    Unfolding |\[C]deep with\[Am]in my |\[Dm]soul
    To be a |jewel in a Lotus |flower
    Unfolding |\[C]is the \[Am]highest |\[Dm]goal
  \endverse
  \notesoff
  \beginchorus
    ^Hari |^Om Mani Padme |Hum
    Om Mani |^Om Mani ^Padme |^Hum
  \endchorus
  \imagecc[1]{om_mani_padme_hum_script_bw_transparent_bg_2000px.png}
\endsong


\beginsong{Padmasambhava Mantra \\ Vajra Guru Mantra}[index={Om Ah Hum},ph={I, II}]
  \showmantra{Om Ah Hum Vajra Guru Padme Siddhi Hum}
  \vspace{1em}
  \textnotefornext{in Pāḷi:}
  % move the next one up (this is a special case of two \showmantra macros):
  \vspace{-2em}
  \showmantra{Om Ah Hung Benza Guru Peme Siddhi Hung}
  \vspace{1em}\vfill
  \textnotefornext{song:}
  \beginchorus
    |\[\mnc{B}Em]Om A\[\mn{A}]h |\[\mn{B}]Hum |Vajra \[\mn{C}]Gu\[\mn{A}]ru |\[Asus2]Padme \[\mn{C}]Sid\[\mn{D}]dhi |\[\mnc{B}Em]Hum | \e
  \endchorus
  \begin{explanation}
    \textbf{Padmasambhava} was a historical teacher in the 8th century, who
    is regarded as the founder of the Nyingma tradition. He is said to have
    been a renowned scholar, meditator, and magician --- the `second Buddha'
    in the minds of many in Tibet.
    \begin{description}
      \item Dilgo Khyentse Rinpoche:\par
        ``It is said that the twelve syllables Om Ah Hum Vajra Guru Padme
        Siddhi Hum carry the entire blessing of the twelve types of teaching
        taught by Buddha, which are the essence of His 84000 Dharmas\ldots''
      \item Jamyang Khyentse Wangpo:\par
        ``It begins with \textbf{OM AH HUM}, which are the seed syllables of
        the three vajras (of body, speech and mind).
        \par
        \textbf{VAJRA} signifies the \emph{dharmakaya} (\emph{Truth body}
        which embodies the very principle of enlightenment and knows no limits
        or boundaries) since, like the adamantine vajra, it cannot be `cut'
        or destroyed by the elaborations of conceptual thought.
        \par
        \textbf{GURU} signifies the \emph{sambhogakaya}
        (\emph{body of mutual enjoyment} which is a body of bliss or clear
        light manifestation), which is `heavily' laden with the qualities of
        the seven aspects of union.
        \par
        \textbf{PADME} signifies the \emph{nirmanakaya} (\emph{created body}
        which manifests in time and space), the radiant awareness of the wisdom
        of discernment arising as the lotus family of enlightened speech.
        \par
        Remembering the qualities of the great Guru of Oddiyana
        (Padmasambhava), who is inseparable from these three \emph{kayas},
        pray with the devotion that is the intrinsic display of the nature of
        mind, free from the elaboration of conceptual thought.
        \par
        All the supreme and ordinary accomplishments --- \textbf{SIDDHI} ---
        are obtained through the power of this prayer, and by thinking,
        `\textbf{HUM}! May they be bestowed upon my mindstream, this very
        instant!'''
    \end{description}
  \end{explanation}
  \yesendsongvfill
\endsong


\beginsong{Om Namo Amitābhaya}[ph={I}]
  \beginchorus
    |\[\bmc\mnc{A}Am]Om na\[\mnc{E}]mo\[\bmadj{-.5ex}] Ami|\[\mnc{F}\bmc G]tā\[\mn{E}]bha\[\mn{D}]ya\[\bmadj{-.7ex}]
    |\[\bmc C]Buddhaya, \[\bmc Em7]Dharmaya, |\[\bmc Am]Sanghaya\[\bmadj{-.7ex}]
  \endchorus
  \beginchorus
    |\[\bmc Am]Om na\[\bm]mo, |\[\bmc G]Om na\[\bm]mo
    |\[\bmc F]Om na\[\bm]mo Ami|\[\bmc E]tābhaya\[\bmadj{-.7ex}]
  \endchorus
  \begin{explanation}
    \textbf{Amitābha} is the principal buddha in Pure Land Buddhism, a branch
    of East Asian Buddhism. In Vajrayana Buddhism, Amitābha is known for his
    longevity attribute, magnetising red fire element, the aggregate of
    discernment, pure perception and the deep awareness of emptiness of
    phenomena. According to these scriptures, Amitābha possesses infinite
    merits resulting from good deeds over countless past lives as a bodhisattva
    named Dharmakāra. Amitābha means ``Infinite Light'' so Amitābha is also
    called ``The Buddha of Immeasurable Life and Light''.\\
    \par
    \noindent Buddhists take refuge in the \emph{Three Jewels} or
    \emph{Triple Gem}, which are:
    \begin{description}
      \item[\hspace{2em} Buddha:] the fully enlightened one
      \item[\hspace{2em} Dharma:] the teachings (expounded by the Buddha)
      \item[\hspace{2em} Sangha:] the spiritual community (the monastic order
        of Buddhism that practices the Dharma)
    \end{description}
  \end{explanation}
\endsong


\beginsong{Perfection Mantra \\ Gate Gate}[index={Teyata Gate Gate},ph={II}]
  \showmantra{Teyata Gate Gate Paragate Para Samgate Bodhi \brk So Ha}
  \begin{feeler}
    Gone, gone, gone far beyond to the awakened state.
  \end{feeler}
  \vfill
  \textnotefornext{song:}
  \meter{6}{8}
  \beginchorus
    \[\mn{B}]Ga\[\mn{D}]te |\[\mnc{E}Em]Gate Pa\[\mn{D}]ra|\[D]gate
    Para Sam|\[Bm]gate Bodhi |\[Em]So Ha
  \endchorus
  \beginchorus
    Gate |\[G]Gate Para|\[D]gate
    Para Sam|\[Bm]gate Bodhi |\[Em]So Ha
  \endchorus
  \begin{explanation}
    The path that takes us to enlightenment comprises the six arts of
    perfection. This mantra helps us to be generous, patient,
    conscientious, diligent, focused and wise.
  \end{explanation}
\endsong


\beginsong{Tara Mantra}[index={Om Tare Tu Tare},by={traditional, Deva Premal},ph={III}]
  \showmantra{Om Tare Tu Tare Ture Mama Ah Yuh Pune Jana Putim Kuru So Ha}
  \begin{feeler}
    The liberator of suffering shines light upon me to create\\
    an abundance of merit and wisdom for long life and happiness.
  \end{feeler}
  \vfill
  \textnotefornext{song:}
  \beginchorus
    \[\mn{E}]Om |\[\mnc{A}Am]Tare Tu |\[\mnc{C}Fmaj7]Tare Tu|\[\mnc{D}G]re \[\mn{F}\mn{D}]So |\[\mnc{E}C]Ha
    Om |\[Dm]Tare Tu |\[Em]Tare Tu|\[Fmaj7]re So |\[Am]Ha
  \endchorus
  \begin{explanation}
    Long life and good health for oneself and others is sought through
    recitation of this mantra thus making one’s life and particularly the
    spiritual journey meaningful.
    \par
    \emph{Tara}, who Tibetans also call \emph{Dolma}, is commonly thought to
    be a Bodhisattva or Buddha of compassion and action, a protector who comes
    to our aid to relieve us of physical, emotional and spiritual suffering.
    \par
    Tara has 21 forms, of which two are especially popular among Tibetan
    people: \emph{White Tara}, who is associated with compassion and long life,
    and \emph{Green Tara}, who is associated with enlightened activity and
    abundance.
  \end{explanation}
\endsong


\beginsong{Medicine Buddha Mantra \\ Healing Mantra}[index={Teyata Om Bekanze Bekanze},tags={health},ph={III}]
  \showmantra{Teyata Om Bekanze Bekanze Maha Bekanze Radza Samut Gate So Ha}
  \begin{feeler}
    I invoke the healing buddha inside me by going all the way to the supreme heights\\
    to remove the pain of illness and spiritual ignorance.
  \end{feeler}
  \vfill
  \textnotefornext{song:}
  \beginchorus
    |\[\mnc{D}Dm]Teyata Om |\[\mnc{F}F]Bekanze Bekanze |\[C] \[\mn{E}]Ma\[\mn{D}]ha \[\mn{E}]Be\[\mn{D}]kan\[\mn{E}]ze
    | Radza Samut Gate |\[Dm]So Ha | \e
  \endchorus
  \begin{explanation}
    The practical purpose of spirituality is to help others deal with their
    various life issues. Sickness represents a major problem. Reciting this
    mantra may contribute to healing on many levels adding to the effectiveness
    of medical treatment and medicines.
  \end{explanation}
\endsong


\beginsong{Purification Mantra}[index={Om Benza Satto Hung},ph={I}]
  \showmantra{Om Benza Satto Hung}
  \begin{explanation}
    \ldots which is the short version of the 100 syllable Mantra:\\
    \\
    OM BENZA SATVO SA MA YA MA NU PALA YA BHENZA SATTO TENO PA TISHTHA DRIDHO
    ME BHAWA SUTOKHAYO ME BHAWA SUPOKHAYO ME BHAWA ANURAKTO ME BHAWA SARVA
    SIDDHI ME PRAYACCHA SARVA KARMA SUTSA ME TSITTAM SHREYANG KURU HUNG HA HA
    HA HA HO BHAGWAN SARVA TATHAGATA BENZA MA ME MUCCHA BHENZE BHAWA MAHA
    SAMAYASATTVA AH HUNG PHET
  \end{explanation}
  \begin{feeler}
    Buddha of Purification within me, embodying all the Buddhas, please
    protect my resolve to purify all my karmas and always bestow on me the
    ability to make my mind good, virtuous, auspicious and immeasurably loving
    with the indestructible strength of a diamond.
  \end{feeler}
  \begin{explanation}
    Even though our potential remains obscure in the darkness of negativity,
    it need not be permanent. This mantra helps transform negative karma
    created over many lifetimes.
  \end{explanation}
\endsong


\beginsong{Teacher Buddha Mantra}[index={Om Muni Muni},tags={teacher},ph={I}]
  \showmantra{Om Muni Muni Maha Muni So Ha}
  \begin{feeler}
    To the teacher, teacher, the great teacher, I pay homage.
  \end{feeler}
  \begin{explanation}
    Shakyamuni, the historical Buddha, cast as the overall teacher of the
    tradition, illustrates the point that without a good teacher in the
    beginning there can be no success in spiritual training. Reciting this
    mantra therefore helps us find a good teacher to lead us towards clarity
    of mind and ultimately discovery of our own pure consciousness which is
    the real guru.
  \end{explanation}
\endsong


\beginsong{Wisdom Mantra}[index={Om Ah Ra Pa Tsa Na},tags={wisdom},ph={II}]
  \showmantra{Om Ah Ra Pa Tsa Na Dhi Dhi Dhi\ldots}
  \begin{feeler}
    Amidst the chaos, everything is pure by nature.
  \end{feeler}
  \begin{explanation}
    The pinnacle of spiritual success is to achieve enlightenment (regardless
    of what it means or whether such a thing is possible or not). This depends
    on recognition of our potential. The mantra confirms that each of us has
    the capacity to replace ignorance with wisdom.
  \end{explanation}
\endsong


%%%%%%%%%%%%%%%%%%%%%%%%%%%%%%%%%%%%%%%%%%%%%%%%%%%%%%%%%%%%%%%%%%%
%%% LATEST PRINTOUT CONTAINED THE SONGS ABOVE.                  %%%
%%%%%%%%%%%%%%%%%%%%%%%%%%%%%%%%%%%%%%%%%%%%%%%%%%%%%%%%%%%%%%%%%%%
%%% Please try to not change the song numbers above this point. %%%
%%% Add new songs only after this point.                        %%%
%%%%%%%%%%%%%%%%%%%%%%%%%%%%%%%%%%%%%%%%%%%%%%%%%%%%%%%%%%%%%%%%%%%


    % English (mostly) language songs
% ===============================
%
% The following sets the song number for the first song in this file.
% The number will automatically be incremented by one for each song.
% Please do not change this! Changing would make different versions of
% the songbook to have different numbers for the same songs, and it
% would totally mess up the selection booklets causing them to have
% wrong songs in them. (For the same reason, add new songs only to the
% end of each songs_ file.)
\setcounter{songnum}{300}


\beginsong{Spirit}[by={Rainer Scheurenbrand}, ph={I}, key={Am}, gk={Bm, Am--C\shrp{}m}]
  \audio[key=Em]{https://rainerscheurenbrand.bandcamp.com/track/spirit}
  \audio[key=Em]{https://www.youtube.com/watch?v=70M8K3PSWzw}
  \transpose{-7} % In Em the notes go from D to D', in Am from G to G'
  \newchords{chords_spirit_a}\newchords{chords_spirit_b}
  \beginchorus\prep{4}
    |\[\mnc{G}Em]Come \[\mn{E}]back, |\[\mn{D}]spirit \[\mn{E}]come back, |\[\mn{G}]come back, \[\mn{E}]spirit \[\mn{D}]come |\[\mn{E}]back
  \endchorus
  \beginchorus\memorize[chords_spirit_a]
    |\[\mnc{A}Am]Spirit of \[^\mn{B}]the \[^\mn{G}]w|\[\mnc{E}Em]ind, the |\[\mnc{A}Am]earth and \[^\mn{B}]the \[^\mn{G}]s|\[\mnc{E}Em]ky
    |\[Bm]Spirit of the s|\[Em]ea, |\[Bm]spirit of this p|\[Em]lace
  \endchorus
  \beginchorus\memorize[chords_spirit_b]
    \ind[1]|\[\mnc{G}G]Take \[\mnc{A}G/A]my h|\[\mnc{B}Em/B]and and \[Em]bring your |\[\mnc{A}B7]strength in\[^\mn{G}]to \[^\mn{F#}]my h|\[\mnc{E}Em]eart
    \ind[1]|\[G]Take \[G/A]my h|\[Em/B]and, re\[Em]member |\[\mncii{D}{B}B7]me, \up{2}(| | ) where I come |\[Em]from
  \endchorus
  \notesoff
  \beginverse
    |\[Em]{} | | | \e
  \endverse
  \goto{Come back}
  \beginchorus\replay[chords_spirit_a]
    |^Spirit of the s|^tars, the |^sun and the m|^oon
    |^Spirit of the d|^awn, |^thunderspiritm|^an
  \endchorus
  \beginchorus\replay[chords_spirit_b]
    \ind[1]|^Take ^my h|^and and ^bring your |^light into my h|^eart
    \ind[1]|^Take ^my h|^and, re^member |^me, \up{2}(| | ) where I come |^from
  \endchorus
  \beginverse
    |\[Em]{} | | | \e \rep{4}
  \endverse
  \beginchorus\replay[chords_spirit_a]
    |^Trai nai nai nai n|^aina nai naina |^trai nai nai nai n|^aina nai nai
    |^Trai nai nai nai n|^aina nai naina |^trai nai nai nai n|^aina nai nai
  \endchorus
  \goto{Come back}
  \beginchorus\replay[chords_spirit_a]
    |^Spirit of the f|^lowers, |^animals and p|^lants
    |^Spirit of the r|^ose, |^dancing butterf|^ly
  \endchorus
  \beginchorus\replay[chords_spirit_b]
    \ind[1]|^Take ^my h|^and and ^teach my |^soul to spread its w|^ings
    \ind[1]|^Take ^my h|^and, re^member |^me, \up{2}(| | ) where I come |^from
  \endchorus
\endsong


\beginsong{Come}[by={Sudhananda, Rumi},tags={circle},ph={I}]
  \meter{6}{8}
  \beginchorus
    \lrep |\[\bmc\mnc{A}Am]Come, c\[\bmc\mnc{E}]ome, who|\[\bmc\mncii{E}{D}Fmaj7]ever \[\mn{C}]you \[\bmc\mn{A}]are
    |\[\bmc G]Wanderer, w\[\bm]orshiper, |\[\bmc C]lover of l\[\bm]eaving
    |\[\bmc Am]Come, c\[\bm]ome, who|\[\bmc Fmaj7]ever you \[\bm]are
    This |\[\bmc G]isn't a c\[\bm]aravan |\[\bmc C]{} of des\[\bmc Em7]pair \rrep
    \vspace{1em}
    And |\[\bmc Dm]it doesn't m\[\bm]atter if you've |\[\bmc E7]broken your v\[\bm]ows
    A |\[\bmc C]thousand \[\bm]times be|\[\bmc D7]fore \[\bm]and yet a|\[\bmc Fmaj7]gain
    C\[\bm]ome again |\[\bmc G]come and y\[\bm]et again \up{2}(|\[\bmc C]come \[\bm]{} )
  \endchorus
  \beginchorus
    |\[\bmc F]{} C\[\bm]ome again |\[\bmc G]come \[\bm]{} \rep{5}
  \endchorus\glueverses\beginverse
    |\[\bmc G]Come again \[\bm]
  \endverse
\endsong


\beginsong{Merry Meet \\ May the Circle Be Open}[by={Starhawk},tags={circle},ph={I, IV, V}]
  \beginverse
    \[\mn{A}]May the |\[\mnc{D}Dm]circle \[\mn{E}]be \[\mn{F}]open |\[\mncii{E}{D}C]but \[\mn{C}\mn{E}]un\[\mnc{D}Dm]bro\[\mn{A}]ken
    May the |\[Dm]peace of the Goddess be |\[C]ever in our \[Dm]hearts
  \endverse
  \beginchorus
    |\[Dm]Mer\[C]ry \[Dm]meet and |\[C]merry \[Dm]part
    And |\[F]merry \[C]meet a|\[Dm]gain
  \endchorus
\endsong


\beginsong{Earth My Body}[tags={earth, water, air, fire},ph={I, II}]
  \beginverse
    |\[\mnc{E}Em]Earth my \[\mnc{D}D]body |\[\mnc{E}Em]water my \[\mnc{B}Bm]blood
    |\[Em]Air my \[D]breath and |\[C]fire my \[D]spirit
  \endverse
  \notesoff
  \textnotefornext{en español:}
  \beginverse
    |^Tierra mi ^cuerpo |^agua mi ^sangre
    |^Aire mi ali^ento y |^fuego mi e^spíritu
  \endverse
  \textnotefornext{suomeksi:}
  \beginverse
    |^Maa on ^kehoni |^vesi on ^vereni
    |^Ilmaa hengi^tykseni |^tulta mun ^henkeni
  \endverse
\endsong


\beginsong{The River Is Flowing}[tags={Mother Earth, Sun, Moon, fire},ph={I}]
  \beginverse
    \[\mn{E}]The |\[\mnc{A}Am]river \[^\mn{B}]is \[^\mn{C}\mn{B}]flow\[^\mn{A}]ing, |\[\mnc{D}G]flowing \[^\mn{E}]and \[\mncii{C}{B}Am]grow\[^\mn{A}]ing
    The |\[Am]river is flowing |\[G]back to the \[Am]sea
    |\[Am]Mother Earth is carrying me, her |\[G]child I will \[Am]always be
    |\[Am]Mother Earth carry me |\[G]back to the \[Am]sea
  \endverse
  \notesoff
  \beginverse
    ^The |^moon she is waning, |^waxing and ^waning
    The |^moon she is waiting for |^us to be ^free
    |^Sister moon watch over me, your |^child I will ^always be
    |^Sister moon watch over me |^until we are ^free
  \endverse
  \beginverse
     ^The |^sun he is shining, |^rising and ^shining
     The |^sun he is shining to |^brighten our ^way
     |^Father sun shine over me, your |^child I will ^always be
     |^Father sun shine over me and |^brighten our ^way
  \endverse
  \beginverse
     ^The |^fire is burning, de|^stroying and ^burning
     The |^fire is burning for |^us to get ^pure
     |^Violet flame burn over me, a |^child I will ^always be
     |^Violet flame burn over me |^until I am ^pure
  \endverse
\endsong


\beginsong{Mother I Feel You}[by={Dianne Martin},tags={earth, water, air, fire},ph={I}]
  \beginchorus\memorize
    |\[\mnc{E}Em]Mother I \[^\mn{G}]feel \[^\mn{E}]you |\[\mnc{D}Bm]under \[^\mn{E}]my \[Em]feet
    |\[Em]Mother I feel your |\[D]heart \[Em]beat
  \endchorus
  \notesoff
  \beginchorus
    \ind |^Heya heya heya heya |^heya heya ^ho
    \ind |^Heya heya heya heya |^heya ^ho
  \endchorus
  \beginchorus
    |^Sister I hear you |^in the river ^song
    |^Eternal waters flowing |^on and ^on  \goto{Heya heya}
  \endchorus
  \beginchorus
    |^Father I see you |^when the eagles ^fly
    |^Light of the spirit gonna |^take us ^high  \goto{Heya heya}
  \endchorus
  \textnotefornext{suomeksi:}
  \beginchorus
    |^Maa, äitini, sun |^tunnen alla ^jalkojen
    |^Tunnen ja kuulen, |^sykkeen sun ^sydämen
  \endchorus
  \beginchorus
    \ind |^Heii-jannaa hoii-jannaa |^heii-jannaa ^hou
    \ind |^Heii-jannaa hoii-jannaa |^heii ^hou
  \endchorus
  \beginchorus
    |^Tuli, veljeni, polta |^vanhat liat ^pois
    |^jotta tilalle |^uutta ^tulla vois  \goto{Heii-jannaa}
  \endchorus
  \beginchorus
    |^Vesi, siskoni, virtaa |^joet mielee^ni
    |^Tunnen sun voiman |^vesipisa^rassakin  \goto{Heii-jannaa}
  \endchorus
  \beginchorus
    |^Ilma, isäni, tuule |^läpi ruumii^ni,
    |^jotta sieluni |^vapaa ^olisi  \goto{Heii-jannaa}
  \endchorus
\endsong


\beginsong{Guided and Protected}[tags={protection},ph={II, IV}]
  \beginverse
    \[^\mn{G#}]We are |\[\mnc{A}Am]guided and \[^\mn{B}]pro|\[^\mn{C}]tec\[^\mn{A}]ted
    Wher|\[F]ever we are | \e
    Wher|\[Dm]ever we are | \e
    Wher|\[E]ever we are | \e
  \endverse
  \notesoff
  \textnotefornext{suomeksi:}
  \beginverse
    Meitä |^opastetaan ja |suojellaan
    Missä |^olemmekaan | \e
    Missä |^olemmekaan | \e
    Missä |^olemmekaan | \e
  \endverse
\endsong


\beginsong{I Hear the Silence}[by={Annie Burns}, tags={source}, ph={II}, key={Dm}, gk={Cm, Cm--Em}]
  \transpose{5} % in Am the notes range from E to E', in Dm from A to A'
  \preferflats % for B\flt{} instead of A#
  \beginchorus
    \[\mn{E}]I |\[\mnc{A}Am]hear \[\mn{C}]the |\[\mncii{B}{G}Em]silence \[\mn{E}]cal\[\mn{G}]ling |\[\mnc{A}Am]me \altchords{\id[1]{(Am) \capo{5}}|Am |Em |Am}
    So |\[G]softly calling |\[F]me \altchords{|G |F}
    To |\[G]what I've alway|\[Am]{s been} | \e \altchords{|G |Am \e}
  \endchorus
  \beginchorus
    \[\mn{A}]I \[\mn{C}]am |\[\mnc{E}C]burning and |\[\mnc{D}G]bur\[\mn{B}]ning \[\mn{G}]in yo|\[\mnc{A}Am]{ur grace} | \e \altchords{|C |G |Am | \e}
  \endchorus\glueverses\beginchorus
    I am |\[C]burning and |\[G]burning \altchords{|C |G}
  \endchorus\glueverses\beginverse
    I am |\[C]burning and |\[G]burning in yo|\[Am]{ur grace} | \e \altchords{|C |G |Am | \e}
  \endverse
  \beginchorus
    \[\mn{C}]I \[\mn{B}]sur|\[\mnc{C}Am]ren\[\mn{A}]der |\[\mnc{G}G]{} \[\mn{C}]to \[\mn{B}]this |\[\mnc{C}Am]mys\[\mn{A}]tery |\[\mnc{G}G]{} \e \altchords{|Am |G |Am |G}
    A|\[Am]waken |\[G]{} to the |\[Am]beauty | \e \altchords{|Am |G |Am | \e}
  \endchorus
\endsong


\beginsong{I Just Close My Eyes \\ The Sound of Water}[by={Deva Madhuro},tags={peace},ph={II},key={Am},gk={Bm, Bm--D\shrp{}m}]
  \audio[key=Bm]{https://soundcloud.com/deva-premal/i-just-close-my-eyes}
  \audio[key=B\flt{}m]{https://soundcloud.com/tosha-carter/i-just-close-my-eyes}
  \newchords{chords_ijustclose_a}\newchords{chords_ijustclose_b}
  \mnbeginchorus\memorize[chords_ijustclose_a]
    |\[\mnc{E}Am]I \[^\mn{C}]just \[^\mn{A}]close \[^\mn{E}]my \[^\mn{D}]e|\[G]yes \altchords{\id[1]{(Bm)}|Bm |A}
    \[^\mn{B}]And the |\[\mnc{C}F]Earth \[^\mn{A}]is \[^\mn{F}]carried \[^\mn{C}\mn{B}]aw|\[Em]ay \altchords{|G |F\shrp{}m}
    \[^\mn{A}]by \[^\mn{G}]the \[^\mn{B}]r|\[\mncadj{0.5ex}{A}Am]iver | \e \altchords{|Bm | \e}
  \mnendchorus
  \mnbeginchorus\memorize[chords_ijustclose_b]
    |\[Fmaj7]{} \[^\mn{A}]What \[^\mn{B}]is \[^\mn{C}]left \[^\mn{D}]be\[^\mn{E}]h|ind I \[^\mn{F}\mn{D}]can't |\[Em]{} \[^\mn{E}]say |\[Em7]{} \altchords{|Gmaj7 | - |F\shrp{}m |F\shrp{}m7}
    |\[Dm7]{} \[^\mn{A}]It's just \[^\mn{C}]the \[^\mn{B}]s|\[Em]ound \[^\mn{A}]of \[^\mn{G}]w|\[\mncadj{0.7ex}{A}Am]ater | \e \altchords{|Em7 |F\shrp{}m |Bm | \e}
  \mnendchorus
  \notesoff
  \textnotefornext{suomeksi:}
  \beginchorus\replay[chords_ijustclose_a]
    |^Suljen vain silmän|^i \altchords{\id[2]{(Dm)}|Dm |C}
    ja |^maailma katoa|^a \altchords{|B\flt{} |Am}
    mukana v|^irran | \e \altchords{|Dm | \e}
  \endchorus
  \beginchorus\replay[chords_ijustclose_b]
    |^ Mitä jäljelle |jää, en |^ tiedä |^ \altchords{|B\flt{}maj7 | - |Am |Am7}
    |^ Kuulen vain v|^eden ä|^änen | \e \altchords{|Gm7 |Am |Dm | \e}
  \endchorus
\endsong


\beginsong{Spiral}[by={Denean},tags={source, path},ph={II}]
  \beginverse
    |\[\mnc{E}Em]Spiral to the |\[\mnc{G}D]cen\[^\mn{F#}]ter \[^\mn{E}]of \[^\mn{D}]the |\[\mnc{E}Em]light | \e
    Where |sacred dreams and |\[D]visions do a|\[Em]bide | \e
    A|\[G]waken to the |\[D]calling of the |\[Em]Spirit | \e
  \endverse\glueverses\beginchorus
    |Soaring with the |\[D]winged ones in the |\[Em]sky | \e
  \endchorus
  \notesoff
  \beginverse
    |^Speak to me, oh! |^Sweet Celestial |^Voice | \e
    And |journey to the |^center of my |^soul | \e
    So |^I can hear the |^songs that you are |^singing | \e
  \endverse\glueverses\beginchorus
    |As my path of |\[D]Love and Light un|\[Em]folds | \e
  \endchorus
  \imagecc[2]{spiral_decorative_transparent_bg_1280px.png}%
\endsong


\beginsong{Fly Like an Eagle}[tags={flying},ph={II, IV}]
  \beginverse
    |\[\mnc{E}Em]Fly like \[^\mn{D}]an \[^\mn{E}]eag\[^\mn{B}]le \rep{2} |\[\mnc{A}D]soaring \[^\mn{B}]so \[^\mn{A}]high \rep{2}
    |\[Em]Circling the universe \rep{2} on |\[D]wings of pure light \rep{2}
  \endverse
  \notesoff
  \beginverse
    \ind |\[Em]Hey witchi chai yo \rep{2} |\[D]witchi chai yo \rep{2}
    \ind |\[Em]Hey witchi chai yo \rep{2} |\[D]witchi chai yo \rep{2}
  \endverse
  \beginverse
    |^Fly like a butterfly \rep{2} |^flowing so low \rep{2}
    |^Circling the Mother Earth \rep{2} with |^wings of rainbow \rep{2}
  \endverse
  \goto{Hey witchi}
  \beginverse
    |^This place is holy \rep{2} |^sacred is the ground \rep{2}
    |^Forest mountain river \rep{2} |^listen to their sound \rep{2}
  \endverse
  \goto{Hey witchi}
\endsong


\beginsong{Above and Below}[by={Kailash Kokopelli},tags={source},ph={I, II, IV},key=Bm,gk={Cm, (Bm)--Cm--E\flt{}m}]
  \transpose{7} % in Em the notes range from B to C', in Bm from F# to G'
  \beginchorus
    \[\mn{B}]A|\[\mnc{E}Em]bove and \[\mn{F#}]be|\[\mn{G}]low \[\mn{B}]and |\[\mnc{A}D]all \[\mn{G}\mn{F#}]around \[\mn{G}]you |\[\mn{A}]are
  \endchorus\glueverses\beginverse
    |\[Am]{} You are the |essence of \[D]all the |beauty of \[Em]life | \e
    |\[Am]{} You are the |essence of \[D]all the |love of my \[Em]life | \e
  \endverse\glueverses\beginchorus
    Sacred |\[C]one source wit|\[D]hin \[Bm]and be|\[Em]yond | \e
  \endchorus
\endsong


\beginsong{Humble}[by={I Chelle},ph={II}]
  \beginverse
    |\[\mnc{F}Dm]Hum\[\mn{D}]ble \[^\mn{F}]your|\[\mnc{E}C]self to the sight \[^\mn{D}]of \[^\mn{E}]the |\[\mnc{F}Dm]mot\[^\mn{D}]her
    You gotta |\[Am]bend down low and
    |\[Dm]Humble your|\[C]self to the sight of the |\[Dm]mother
    You gotta |\[Am]know what she knows
  \endverse
  \notesoff
  \beginverse
    \ind And |\[F]we shall |\[C]lift each other |\[Dm]up
    \ind |\[Am]Higher and higher
    \ind |\[F]We shall |\[C]lift each other |\[Dm]up | \e
  \endverse
  \beginverse
    |^Hum^ble your|^self to the force of the |^sun
    You gotta |^bend down low and
    |^Humble your|^self to the force of the |^sun
    You gotta |^know what he shows  \goto{And we}
  \endverse
  \beginverse
    |^Hum^ble your|^self to the light of the |^moon
    You gotta |^bend down low and
    |^Humble your|^self to the light of the |^moon
    You gotta |^shine with her glow  \goto{And we}
  \endverse
  \beginverse
    |^Hum^ble your|^self to the spirit of the |^forest
    You gotta |^bend down low and
    |^Humble your|^self to the spirit of the |^forest
    Don't |^be afraid to let go  \goto{And we}
  \endverse
\endsong


\beginsong{Take Me Away}[tags={love},ph={II}]
  \beginverse
    |\[\mnc{E}Am]Take me away \[^\mn{D}]won't \[^\mn{C}]you |\[\mnc{D}G]car\[^\mn{E}]ry \[^\mn{D}]me
    |\[F]{} Let me rest in your |\[G]arms just for a |\[Am]while | \e
  \endverse
  \notesoff
  \beginverse
    |^Take me away won't you |^carry me
    |^ Let me bathe in the |^sweetness of your |^smile | \e
  \endverse
  \beginchorus
    Ama |\[C]take |\[G]me a|\[Am]way | \e
    Ama |\[C]take |\[G]me a|\[Am]way | \e
  \endchorus
\endsong


\beginsong{In the End of Days}[by={Amir Paiss},tags={Shiva},ph={II}]
  \beginverse
    \[^\mn{A}]In \[^\mn{B}]the \[^\mn{C}]end \[^\mn{D}]of |\[\mnc{E}Am]days, | \[^\mn{F}]world \[^\mn{E}]in \[^\mn{D}\mn{C}]ever |\[\mnciii{D}{C}{A}Dm]change | \e
    Clouds are hiding |\[F]sun, night is on the |\[Dm]run
    Till it turns a|\[Am]round | \e
  \endverse
  \notesoff
  \beginverse
    All will pass a|^way, | nature of all |^things | \e
    What is born to|^day, surely dies some |^day
    Now it's time to |^pray | \e
  \endverse
  \beginchorus
    \ind |\[Am]Om namah Shivaya |Om |\[Dm]Om namah Shivaya |Om
    \ind |\[F]Om namah Shivaya |\[Dm]Om namah Shivaya |\[Am]Om | \e
  \endchorus
  \beginverse
    Master of the |^change, | dances and de|^stroys | \e
    Just to leave a |^room for the new to |^bloom
    Pray before it's |^gone | \e
  \endverse
  \beginverse
    Listen to the |^wind, | carrying a|^way | \e
    Moments of the |^day, secrets of a |^ray
    Now it's time to |^pray | \e \goto{Om namah Shivaya}
  \endverse
  \beginverse
    Mirrors of a |^soul, | life is ever|^changing gate | \e
    Waving of a |^being, being lover |^ring
    Hollow and com|^plete | \e \goto{Om namah Shivaya}
  \endverse
\endsong


\beginsong{Return Again}[by={Schlomo Carlibach},ph={II}]
  \audio[]{https://www.youtube.com/watch?v=OEsMlW3mB4I}
  \meter{3}{4}
  \beginchorus
    |\[\mnc{D}Dm]Return \[\mn{B&}]a|\[\mnc{C}C]gain |\[\mnc{D}Dm]return \[\mn{B&}]a|\[\mnc{C}C]gain
    Re|\[B&]turn to the |\[C]land of your |\[Dm]soul | \e
  \endchorus
  \beginchorus
    |\[F]Return to |\[Gm]what you are
    |\[F]Return to |\[Gm]who you are
    |\[F]Return to |\[Gm]where you are
    |Born and re|\[B&]born a|\[A]gain | \e
  \endchorus
\endsong


\beginsong{Angels Singing \\ Wood Stone}[tags={angels},ph={II, IV}]
  \beginchorus\memorize
    |\[\mnc{D}Dm]Wood, stone, |\[\mnc{C}C]feather \[^\mn{E}]and \[\mnc{D}Dm]bone
    |Currents of the ocean, |\[C]guide us \[Dm]home
  \endchorus
  \notesoff
  \beginverse
    \ind |\[Dm]Angels |singing, |an\[C]gels |\[Dm]singing
  \endverse\glueverses\beginchorus
    \ind In my |\[F]soul, in my |\[C]soul, in my |\[Dm]soul | \e
  \endchorus
  \beginchorus
    |^River, sea, |^red wood ^tree
    |Howling of the wind gonna |^set us ^free
  \endchorus
  \goto{Angels singing}
  % Image downloaded from: https://imgbin.com/png/6NnwqU2z/angel-halo-light-png
  % Image license: Free for non-commercial use
  \imagecc[2]{angel_halo_transparent_bg_465x727px.png}%
\endsong


\beginsong{I Release Control}[tags={canon},ph={II, III}]
  \beginverse
    |\[\mnc{A}A]I \[\mn{E}]release \[\mn{D}]cont|\[\mn{E}]rol |\[\mn{A}]and \[\mn{E}]sur\[\mn{D}]ren\[\mn{C#}]der
    |\[\mnlow{A}]to \[\mnlow{B}]the \[\mnlow{A}]flow |\[\mnlow{C#}]of \[\mnlow{A}]love |\[\mnlow{C#}]that \[\mnlow{A}]will \[\mnlow{G#}\mnlow{A}]heal me | \e
  \endverse
  % \musicnotefornext{Second voice starts its round at the beginning of the second bar. More voices can be added.}
\endsong


\beginsong{There is So Much Magnificence}[by={Peter Makena},tags={sea, thankfulness, canon},ph={II, III, IV}]
  \beginchorus\memorize
    |\[C]{} \[^\mn{E}]There \[^\mn{F}]is |\[\mnc{G}G]so \[^\mn{E}]much \[^\mn{D}]mag|\[Am]ni\[^\mn{C}]fice|\[\mnc{B}Em]nce \[^\mn{E}]in \[^\mn{D}]the |\[\mnc{C}F]ocean |\[C]
    |\[Dm]Waves are coming in, |\[G]waves are coming in
  \endchorus
  \notesoff
  \beginchorus
    |^Ha|^le|^lu|^jah |^Ha|^le|^lu|^jah
  \endchorus
\endsong


\beginsong{Heart of the Mother}[by={Michael Stillwater},tags={source, canon}, ph={I, IV}]
  \beginchorus
    |\[\mnc{A}Am]I \[\mn{E}]am one \[\mn{D}]with \[\mn{E}]the |\[\mnc{F}Dm]heart \[\mn{D}]of \[\mn{E}]the \[\mn{F}]Mo\[\mn{D}]ther
    |\[G]I am one with the |\[E]heart of Love
    |\[Am]I am one with the |\[Dm]heart of the Father
    |\[G]I am one with the |\[E]God |\[(E7)]
  \endchorus
  \beginchorus
    |\[\mncii{C}{E}Am]A|\[\mnc{F}Dm]ve \[\mn{C}]Ma|\[\mnc{B}G\mn{C}\mn{B}\mn{A}]ri|\[\mnc{G#}E]a
    |\[Am]Kyri|\[Dm]e E|\[G]lei|\[E]son |\[(E7)]
  \endchorus
  \begin{explanation}[EN]
    \begin{description}
      \item[Ave Maria:] ``Hail Mary'' \emph{(latin)}
      \item[Kyrie Eleison:] ``Lord, have mercy'' \emph{(ancient Greek)}
    \end{description}
  \end{explanation}
\endsong


\beginsong{Spirit of the Plants}[by={Lisa Thiel},ph={II, IV}]
  \beginchorus
    \[\mn{A}]The |\[\mnc{D}Dm]spirit of the \up{*}\[\mnc{E}C]plants \altlyr[*]{wind, earth, sea\ldots}
    Has |\[B&]come to m\[Am]e
    In the |\[Dm]form of a \[C]beautiful
    |\[B&]Dancing \up{¤}\[Am]green \[Dm]woman | \e \altlyr[¤]{white, yellow, blue\ldots}
    \lrep Her |\[Dm]eyes fill \[C]me with |\[B&]pea\[Am]ce
    Her |\[Gm]dance fills \[Am]me with |\[Dm]joy \rrep
  \endchorus
\endsong


\beginsong{We Are Opening Up}[tags={opening},ph={II, IV}]
  \beginchorus\memorize   % memorize the chords even though in 'chorus'
    \[^\mn{E}]We are |\[\mnc{G}Em]ope\[^\mn{F#}]ning \[^\mn{E}]up in a |\[^\mn{G}]sweet \[^\mn{F#}]sur\[^\mn{E}]render
    To the |\[D]luminous \[Bm]love light |\[Em]of the one
  \endchorus
  \notesoff
  \beginchorus
    \ind We are |\[Em]open|ing we are |\[Bm]o\[D]pen|\[Em]ing
  \endchorus
  \beginchorus
    We are |^rising up like a |phoenix from the fire
    |^Brothers and ^sisters spread your |^wings and fly higher
  \endchorus
  \vspace{1em}
  \goto{We are opening}
\endsong


\beginsong{Fire Child}[tags={fire},ph={II}]
  \beginchorus\memorize
    \[^\mn{F}]Light |\[\mnc{E}Dm]stre\[^\mn{D}]aming \[^\mn{F}]light |\[^\mn{E}]stre\[^\mn{D}]aming
    |\[F]Making my \[C]fire child |\[Dm]glow
  \endchorus
  \notesoff
  \beginchorus
    |^Fire child dance |fire child sing
    |^Fire child ^you’ll be |^mine
  \endchorus
\endsong


\beginsong{Kali Burn It All Away}[index={Om Namo Kali},tags={fire, Divine Mother, Kali},ph={III}]
  \beginchorus\memorize
    |\[\mnc{A}Am]Om \[^\mn{E}]namo Kali Kali |\[\mnc{D}G]Om \[^\mn{F}]na\[\mnc{E}Am]mo
    |\[Dm]Om namo \[Am]Kali Kali |\[G]Om na\[Am]mo
  \endchorus
  \notesoff
  \beginchorus
    |^Oh great mother we in|^voke you in this ^space
    |^Take away the ^pain and |^fill us with your ^grace
  \endchorus
  \beginchorus
    Kali |^burn it all away |^burn it all a^way
    |^If it doesn't ^serve us then |^burn it all a^way
  \endchorus
\endsong


\beginsong{Clear Blue Sky}[by={Nick Barber},tags={liberation},ph={II, III}]
  \newchords{chords_clearbluesky_a}\newchords{chords_clearbluesky_b}\newchords{chords_clearbluesky_c}
  \beginverse\memorize[chords_clearbluesky_a]
    |\[Am]{} \[\mn{A}]All \[\mn{E}]these |\[Dm]thoughts now |rising in my |\[Am]mind
    | They pass like |\[E7]clouds | through a clear blue |\[Am]sky
    | All these |\[Dm]feelings that |come from every |\[Am]side
    | They pass like |\[E7]clouds | through a clear blue |\[Am]sky
  \endverse\glueverses\beginchorus\memorize[chords_clearbluesky_b]
    | And all these |\[Dm]visions they |pass before my |\[Am]eyes
    | Just like |\[E7]clouds | through a clear blue |\[Am]sky
  \endchorus
  \notesoff
  \beginverse\memorize[chords_clearbluesky_c]
    \ind |\[E7]{} And the |\[Am]calm of a clear blue |sky
    \ind Is where I’m |\[Em]going when I |die
    \ind There’s nowhere |\[Dm]else to go, |\[E7]{} that’s all there |\[Am]is
    \ind |\[E7]{} And the |\[Am]calm of a clear blue |sky
    \ind Is where I’m |\[Em]going when I |die
    \ind There is nothing |\[Dm]else to know; |\[E7]{} that’s all there |\[Am]is
  \endverse
  \brk
  \beginverse\replay[chords_clearbluesky_a]
    |^ And ^all ^these |^fears are |passing through my |^mind
    | Just like |^clouds | through a clear blue |^sky
    | All the |^years are |passing through my |^mind
    | Just like |^clouds | through a clear blue |^sky \replay[chords_clearbluesky_b]
    | This world of il|^lusion |passes through my |^mind
    | Just like |^clouds | through a clear blue |^sky \replay[chords_clearbluesky_b]
    | So many |^energies of |every different |^kind
    | But they pass like |^clouds | through a clear blue |^sky
  \endverse
  \beginverse\replay[chords_clearbluesky_c]
    \ind |^ And the |^calm of the clear blue |sky
    \ind Is where I |^live with open |eyes
    \ind There is nowhere |^else to go, |^ that’s all there |^is
    \ind |^ The |^calm of the clear blue |sky
    \ind Is where I |^live with open |eyes
    \ind There is nothing |^else to know, |^ that’s all there |^is
  \endverse
  \beginchorus
    |\[E7]{} That’s all there |\[Am]is, | that’s all there |\[Em]is
    | That’s all there |\[Dm]ever was, |\[E7]{} and that’s all there |\[Am]is
  \endchorus
\endsong


\beginsong{Teacher in the Forest}[by={Nick Barber},tags={Aya},ph={III, IV}]
  \meter{3}{4}
  \beginverse
    \[\mn{B}]The |\[\mnc{E}Em]teacher |\[\mnc{F#}B7]in the |\[Em]forest |\[B7]said that
    |\[Em]we should |\[D]puri|\[G]fy | \e
    So |\[Em]we are |\[B7]able |\[Em]to re|\[B7]ceive our
    |\[G]angel |\[D]wings and |\[Em]fly | \e
  \endverse
  \notesoff
  \beginverse
    ^The |^teacher |^in the |^forest |^said that
    |^we should |^seek re|^birth | \e
    So |^we can |^be a |^bridge be|^tween
    |^heaven |^and the |^earth | \e
  \endverse
  \beginverse
    \ind Ó |\[Am]daime, |daime a|\[Em]mor, | |\[B7]Daime mi|nha que|\[Em]ri|da
    \ind |\[Am]Daime, |daime |\[Em]luz, |estou con |\[B7]Vós, | | e|terna |\[Em]vi|\[B7]da |\[Em]{} |\[B7]
  \endverse
  \beginverse
    ^The |^teacher |^in the |^forest |^told a
    |^secret |^from a|^bove | \e
    That |^all this |^world of con|^fusion and |^pain is
    |^truly |^made of |^love | \e
  \endverse
  \beginverse
    ^The |^teacher |^in the |^forest |^told a
    |^secret |^in the |^night | \e
    That |^all this |^world that |^seems so |^real is
    |^truly |^made of |^light | \e  \goto{Ó daime \rep{2}}
  \endverse
  \beginverse
    ^The |^teacher |^in the |^forest |^said that
    |^we should |^learn to |^love | \e
    So |^we can be |^brothers and |^sisters to|^gether on
    |^earth as in |^heaven a|^bove | \e
  \endverse
  \beginverse
    ^The |^teacher |^in the |^forest |^said it's
    |^love that |^is the |^glue | \e
    That |^holds to|^gether the |^atoms and |^stars as
    |^well as |^me and |^you | \e  \goto{Ó daime \rep{3}}
  \endverse
  \iflyriconly\forcebrk\fi
  \begin{translation}[EN]
    Oh give me, give me love, daime my beloved.
    Daime, give me light, I am with You... eternal life.
  \end{translation}
  \begin{explanation}[EN]
    \textbf{Dai-me} is Portuguese for ``give me''. The Ayahuasca drink is also called
    \textbf{Daime} in some circles.
  \end{explanation}
\endsong


\scleardpage
\beginsong{The Heart's Song}[by={Minna Pyhälä}, ph={III}, tags={heart}, key={G}, gk={A, G\shrp{}--D}]
  \audio[key={F\shrp{m}m},pitch={432}]{https://soundcloud.com/spirit-sings-minna/a-hearts-song-minna}
  \meter{6}{8}
  \transpose{2} % in G the notes range from F# to E', in A from G# to F#
  % NOTE: setup smaller fonts etc to fit the song better, commented out for now
  \normalsize
  \renewcommand{\printchord}[1]{\sffamily\bfseries\footnotesize\color{chordcolor}#1} % to go with the above
  \renewcommand{\altchordstyle}{\scriptsize\itshape}
  \renewcommand{\clineparams}{%
    \baselineskip=8.46pt%
    \lineskiplimit=1pt%
    \lineskip=1pt%
  }
  \beginverse
    |\[\mnc{G}G]This is \[^\mn{F#}]a |\[\mnc{G}Em]story for \[^\mn{F#}]the |\[\mnc{G}C]little \[^\mn{B}]one \[^\mn{G}]with|\[\mnc{A}D]in
    Who has |\[G]lost her |\[Em]way and |\[C]wanders on her |\[D]{} own | | | \e
    May she |\[G]hear this sweet |\[Em]song, coming |\[C]from her precious |\[D]heart
    The |\[G]words she always |\[Em]longed to hear and be|\[C]lieve to be |\[D]{} true: | | | \e
  \endverse
  \beginverse
    \ind \[\mn{D}]I am |\[\mnc{E}G]he\[\mn{D}]re \[\mn{E}]with |\[\mnc{B}Em]you, there \[\mn{D}]is |\[\mncii{C}{D}C]no \[\mn{C}]need \[\mn{B}]to |\[\mnc{A}D]run \altchords{\id[1]{(G) \capo{2}}|G |Em |C |D}
    \ind For I em|\[G]brace you, I |\[Em]love you, |\[C]just as you |\[D]{} are | | | \e \altchords{|G |Em |C |D | | | \e}
  \endverse
  \notesoff
  \beginverse
    There are |^times when you |^feel |^lost and con|^fused
    When the |^world inside |^feels |^more than you can |^ bare | | | \e
    With a |^fire of |^rage burning |^fiercely in|^side
    Or |^cries of |^pain longing to |^be screamed and |^ heard | | | \e
  \endverse
  \beginverse
    \ind And |^when you have |^cried all the |^tears that you can |^cry
    \ind |^There beyond the |^mess, is a |^stillness in|^{-side} | | | \e
  \endverse
  \beginverse
    And |^here I |^am always, |^waiting patient|^ly
    To |^witness you, to be|^hold you, I am |^not afraid of |^ you | | | \e
    Will you al|^low me to |^see you |^just as you |^are
    I |^promise I |^will not de|^mand anything |^ of you | | | \e
  \endverse
  \beginverse
    \ind For |^you are |^free to |^be here of your own |^will
    \ind You may |^come and you may |^go |^just as you |^ please | | | \e
  \endverse
  \beginverse
    Let's for|^get all the |^stories, the |^people and their |^needs
    Their |^thoughts and their |^feelings are |^stories of their |^ own | | | \e
    No |^matter what |^others will |^say or may |^do
    You can |^take this sacred |^moment to |^be here with |^ me | | | \e
  \endverse
  \beginverse
    \ind For |^I am with |^you, for as |^long as it |^takes
    \ind For you to |^trust and |^speak your |^words of |^ truth | | | \e
  \endverse
  \beginverse
    For |^now, it is just |^you and me, ex|^ploring this |^moment
    This |^space so |^special, can you |^feel the peace and |^ quiet here | | | \e
    For |^here and |^now all is |^safe and |^well
    There is |^nowhere to |^go, nothing to |^do or |^ be | | | \e
  \endverse
  \beginverse
    \ind I am |^here for |^you, you can sur|^render to my em|^brace
    \ind To just |^be, to let |^go of the |^need to under|^{-stand} | | | \e
  \endverse
  \beginverse
    I |^see your |^eyes, so |^innocent and |^wild
    So |^vulnerable, and at |^times so |^apprehensive |^ too | | | \e
    You can |^give me your |^doubt, your |^fear, your loneli|^ness
    For |^you do not |^need to carry |^them a|^{-}round | | | \e
  \endverse
  \beginverse
    \ind You can |^give it all to |^me and |^Mother |^Earth
    \ind And |^feel the love of |^Father Sky |^who adores |^ you | | | \e
  \endverse
  \beginverse
    Oh the |^birds they |^sing for you and the |^trees they breathe for |^you
    The |^water she |^kisses you |^washing away all |^you do not need | | | \e
    |^Come my |^child, my |^little sweet |^one
    You are |^welcome in this |^world, so a|^waited |^ for | | | \e
  \endverse
  \beginverse
    \ind You are cre|^ated out of |^the |^love of |^life
    \ind So |^come, my |^child, come |^share your blessed |^ light | | | \e
  \endverse
  \beginverse
    I |^see you are |^learning to |^walk your path of |^truth
    To |^listen to your |^heart and to |^open up to |^ trust | | | \e
    You can |^call me any|^time, I am |^always here with |^you
    Walking |^side by side with |^you wher|^ever you |^ are | | | \e
  \endverse
  \beginverse
    \ind You are |^free you are |^free, you are |^free to |^be
    \ind A|^ll that you |^feel in |^any given |^ moment | | | \e
  \endverse
  \beginverse
    You can |^give yourself to |^me, your |^very own |^heart
    And |^from |^here, we will ex|^plore the magic |^ of this world | | | \e
    For |^beauty she a|^waits for you to |^open up your |^eyes
    To re|^ceive you in her em|^brace and |^love you all the |^ more | | | \e
  \endverse
  \beginverse
    \ind Oh |^what a precious |^gift this |^sweet melo|^dy
    \ind |^Thank you for |^listening to |^my simple |^ song | | | \e
  \endverse
  \beginverse
    Lai lai lai\ldots
  \endverse
\endsong


\begin{intersong}
  \imagecc[1]{heart_by_larva__789px.png}
\end{intersong}


\beginsong{Celestial Heart}[by={Nick Barber},ph={III},key=Am,gk={Bm, Gm--C\shrp{}m}]
  \meter{3}{4}
  \transpose{5} % in Am the notes range from A to G'
  \beginverse
    |\[\mnc{E}Em]Oh my ce|\[\mnc{F#}D]les\[^\mn{G}]ti\[^\mn{F#}]al |\[\mnc{E}Em]heart | \e \altchords{\id[1]{(Bm)}|Bm |A |Bm | \e}
    With your |\[G]love that |\[Am]takes me so |\[Em]high | \e \altchords{|D |Em |Bm | \e}
    You |\[D]teach me the |way I can |\[Bm]live | \e \altchords{|A | - |F\shrp{}m | \e}
    You |\[C]show me what |\[D]it means to |\[Em]die | \e \altchords{|G |A |Bm | \e}
  \endverse
  \notesoff
  \beginverse
    |^Oh my ce|^lestial |^love | \e
    With your |^light you |^heal my |^pain | \e
    In your |^kingdom I |feel I am |^home | \e
    And I |^don't want to |^leave you a|^gain | \e
  \endverse
  \beginverse
    |^Oh my ce|^lestial |^dream | \e
    With your |^light that |^shows me the |^way | \e
    To |^lead me back |home to the |^love | \e
    That's |^deeper than |^my words can |^say | \e
  \endverse
  \beginverse
    |^Oh my ce|^lestial |^queen | \e
    With your |^love that |^takes me so |^high | \e
    \lrep My |^heart has |opened its |^wings | \e
    With |^you I have |^learned how to |^fly | \e \rrep
  \endverse
\endsong


\beginsong{Heart's Mystery}[by={Nick Barber},tags={heart},ph={III},key={C},gk={C, G--D}]
  % in C the notes range from C to A
  \meter{3}{4}
  \beginchorus
    \ind |\[\mnc{D}Dm]When you |\[\mn{F}]let \[\mn{E}]go \[\mn{D}]of |\[\mnc{C}C]fear
    \ind The |truth will ap|\[Dm]pear
    \ind So |simple and |\[C]clear | \e
  \endchorus
  \beginverse
    |\[\mn{F}Fmaj7]{} There's a |fee\[^\mn{G}]ling \[^\mn{A}]in|\[\mnc{C}C]side
    So |deep and so |\[Fmaj7]wide, so |open and |\[C]free | \e
    |\[Fmaj7]{} When |love is |\[C]revealed
    All |beings are |\[Fmaj7]healed so |natural|\[C]ly | \e \goto{When}
  \endverse
  \beginverse
    |^ Let your |light show the |^way
    For|ever to |^stay in the |circle of |^friends | \e
    |^ Let your |heart be your |^guide
    To |lead you in|^side where |love never |^ends | \e \goto{When}
  \endverse
  \beginverse
    |^ And when |love over|^flows
    You can |only let |^go and be |swept out to |^sea | \e
    |^ This |journey will |^end
    Where it |started my |^friend: in the |heart's myste|^ry | \e \\ \goto{When}
  \endverse
\endsong


\beginsong{Temple of My Heart}[by={Kevin James Carroll},tags={heart},ph={III}]
  \beginverse
    \[\mn{A}]From \[\mn{C}]the |\[\mnc{E}Am]temple of my |heart
    up to the |\[D]highest mountain | \e
    Love em|\[C]brace all that i|\[Em7/B]s
    and all that could |\[Am]be | \e
  \endverse
  \notesoff
  \beginverse
    ^From ^the |^temple of my |soul
    down to the |^deepest ocean | \e
    Jah love f|^low like river |^
    eternally |^free | \e
  \endverse
  \beginchorus
    \ind El|\[Am]ah elah e|lah elah elah
    \ind E|\[C]lah elah e|lah elah elah
    \ind E|\[G]lah elah e|lah elah elah
    \ind E|\[Am]lah elah e|lah
  \endchorus
\endsong


\beginsong{Wake Up the Heart / Radhe Shyam}[by={Nalini Blossom. Prembabanda},tags={heart, love, Krishna, Radha},ph={III}]
  \textnotefornext{part A: Wake up the heart}
  \beginchorus\memorize
    \[\mn{A}]I |\[\mnc{D}Dm]want to sing song of |\[F]pure light
    So |\[C]love will shine thro|\[Dm]ugh me
  \endchorus
  \notesoff
  \beginverse
    ^I |^sing it with a voice |^like a pray
    To |^wake up the heart, |^wake up the heart
  \endverse
  \beginchorus
    To |\[Gm]wake up the heart, to |\[Dm]wake up the heart
    So |\[C]love will shine thro|\[Dm]ugh me
  \endchorus
  \textnotefornext{part B: Radhe Shyam}
  \beginchorus
    | \[Dm]{} Radhe Radhe Radhe |Shyam,
    | \hspace{1.5em} Govinda Radhe Ra|\[Gm]dhe
    | \hspace{2.5em} Gopa|\[C]la
    | \hspace{1.5em} Radha Ramana Gopa|\[Dm]le
  \endchorus
  \begin{explanation}[EN]
    Govinda, Shyam and Gopala are names for \textbf{Krishna}.
    Radhe and Ramana are \textbf{Radha}'s alternate names.
    \textbf{Radha Krishna} are collectively known within Hinduism as the combination of
    both the feminine as well as the masculine aspects of God.
  \end{explanation}
\endsong


\iflyriconly\forcebrk\fi
\beginsong{Ho'oponopono}[ph={III}, key={Am}, gk={Am, Gm--Bm}]
  \musicnotefornext{\tiny melody from Loreena Mckennitt's ``Tango to Evora''}
  \transpose{5} % in Am the melody ranges from A to A'
  \mnbeginchorus\memorize
    |\[\mnc{B}Em]Ho'opono\[\bmc\mn{C}]po\[^\mn{B}]no | Ho'o\[\bmc\mn{E}]pono\[^\mn{B}]pono
    |\[\mnc{A}Bm7]Ho'opono\[\bmc\mn{B}]po\[^\mn{A}]no | Ho'o\[\bmc\mn{B}]pono\[^\mn{A}]pono
    |\[\mnc{G}Em]Ho'opono\[\bmc\mn{A}]po\[^\mn{G}]no | Ho'o\[\bmc\mn{A}]pono\[^\mn{G}]pono
    |\[\mnc{F#}B]Ho'opono\[\bmc\mn{G}]po\[^\mn{F#}]no | Ho'o\[\bmc\mn{G}]pono\[^\mn{/}]po\[^\mn{/}]no \up{2}(|\[Em]{} \[\bm]{} |{ }{ } \[\bm]\e)
  \mnendchorus
  \notesoff
  \beginchorus
    |^I am ^sorry | Please for^give me
    |^I ^love you | I ^thank you
    |^I am ^sorry | Please for^give me
    |^I ^love you | I ^thank you \up{2}(|^ ^ | ^ \e)
  \endchorus
  \ifchorded\nolyrics{%
    \musicnotefornext{interlude:}
    \beginverse
      \ind |\[Am]{} \[\bm]{} |{ }{ } \[\bm]{} |\[Em]{} \[\bm]{} |{ }{ } \[\bm]{} |\[Bm]{} \[\bm]{} |{ }{ } \[\bm]{} |\[Em]{} { }{ } |{ }{ } \[\bm]{} \e{ } \rep{2}
    \endverse
  }\fi%
  \textnotefornext{suomeksi:}
  \beginchorus
    |^Olen pahoil^lani | Annathan ^anteeksi
    |^Rakastan ^sinua | Minä kii^tän sinua
    |^Olen pahoil^lani | Annathan ^anteeksi
    |^Rakastan ^sinua | Minä kii^tän sinua \up{2}(|^ ^ | ^\e)
  \endchorus
  \begin{explanation}[EN]
    Ho'oponopono practice belongs to an old Hawaiian system of knowledge and wisdom, \emph{Huna}.
    Ho'oponopono translates as ``to make rightly right''.
    The four sentences are a prayer through which one gains inner peace, harmony and moves
    from separation to unity.
    \begin{description}
     \item[I am sorry:] I apologise. I perceive that I suffer, and that connects me to my feelings.
       I no longer reject the problem, but recognise my learning task. I or my forebears have
       caused harm. I know where I stand and feel remorse.
     \item[Please forgive me:] Please forgive me for having, through myself or my forebears,
       consciously or unconsciously disturbed you and me in the course of our evolution. Please
       forgive me for having acted contrary to the divine laws of harmony and love. Please forgive
       me for having until now judged you (or the situation), and in the past disregarded our
       spiritual identity and connectedness.
     \item[I love you:] I love you and I love myself. I see and respect the divine in you. I love
       and accept the situation just as it is. I love the problem that has come to me to open my
       eyes. I love you and myself unconditionally with all our weaknesses and faults.
     \item[I thank you:] Thank you, for understanding that the miracle is already underway. I thank
       God for the transformation of my request. I give thanks, because what I have received and
       what will come to pass is what I have deserved through the law of cause and effect. I give
       thanks because, through the power of forgiveness, I am now freed from the energetic chains
       of the past. I give thanks that I may recognise and join with the Source of all Being.
    \end{description}
  \end{explanation}
  \yesendsongvfill
\endsong


\beginsong{Let the Way of the Heart}[by={Rich Stillwell},tags={heart},ph={III}]
  \beginchorus
    \[\mn{D}]Let \[\mn{E}]the |\[\mnc{F}Dm]way of the |heart, let the |\[C]way of the |heart
    Let the |\[B&]way of the |\[C]heart shine |\[Dm]through | \e
  \endchorus
  \beginverse
    |\[F]Love upon |\[C]love upon |\[Dm]love | \e
    |\[B&]{} All hearts are |\[C]beating as |\[Dm]one | \e
    |\[F]Light upon |\[C]light upon |\[Dm]light | \e
    |\[B&]{} Shining and |\[C]brightening as the |\[Dm]sun | \e
  \endverse
  \textnotefornext{outro:}
  \beginverse
    |\[B&]{} All disap|\[C]pear into |\[Dm]one | \e
  \endverse
\endsong


\beginsong{Angel Heart}[by={Nick Barber},tags={heart},ph={III}]
  \beginverse
    \[\mnc{D}G]An\[\mn{E}]gel |\[\mnc{C}C]heart | \e
    Is |\[Am]this another |\[G]ending or a |\[C]start | \e
    |\[Am]Is there any |\[G]way that they could |\[C]be \[C/B]a|\[Am]part
    |\[Am]{} In the |\[G]end | |\[F]{} we come full |\[G]circle a|\[C]gain | \e
  \endverse
  \notesoff
  \beginverse
    ^Gen^tle |^soul | \e
    You |^know each daily |^trouble takes its |^toll | \e
    But |^every silver |^lining hides a |^seam ^of |^gold
    |^ In the |^end | |^ we come full |^circle a|^gain | \e
  \endverse
  \beginverse
    ^An^gel |^heart | \e
    |^Never be a|^fraid to face the |^dark | \e
    |^If you are you'll |^never let the |^hea^ling |^start
    |^ In the |^end | |^ we come full |^circle a|^gain | \e
  \endverse
  \beginverse
    ^Gen^tle |^soul | \e
    |^Never be a|^fraid to face the |^goal | \e
    |^Don't you know the |^light you see is |^your ^own |^soul
    |^ In the |^end | |^ we come full |^circle a|^gain | \e
  \endverse
  \beginverse
    ^Spe^cial |^one | \e
    |^Set your ship to |^sail into the |^sun | \e
    And |^when you finally |^get there you have |^just ^be|^gun
    |^ In the |^end | |^ we come full |^circle a|^gain | \e
  \endverse
\endsong


\beginsong{Beloved of Mine}[by={Bachan Kaur, Suruya Devi},tags={love, you},ex={english, español},ph={III}]
  \transpose{2}
  \beginchorus\memorize
    |\[Am]{} \[^\mn{A}]Beloved \[^\mn{B}]of |\[\mnc{C}C]mine, I \[^\mn{E}]see |\[\mnc{D}G]you, \[^\mn{C}]I \[^\mn{B}]see |\[\mncii{C}{A}D]me
    In all that you |\[Am]do, beloved of |\[C]mine
    We are |\[G]one, we are the |\[D]moon
    And we are the |\[Am]sun, |\[C]{} we are the |\[G]sun |\[D]
  \endchorus
  \notesoff
  \beginchorus
    Que mi cora|^zón refleje tu |^luz de amor
    Como la |^luna refleja la |^luz del sol
    En a|^mor, siempre en a|^mor |^ |^
  \endchorus
  \beginchorus\prep{4}
    |^Sita Ram |^Sita Ram |^jay jay Ram |^Sita Ram
  \endchorus
  \begin{translation}[EN]
    May my heart reflect your light of love
    As the moon reflects sunlight
    In love, always in love
  \end{translation}
  \begin{explanation}[EN]
    \begin{description}
      \item[Sita:] an avatar of Sri Lakshmi, consort of Lord Ram(a)
      \item[Ram:] an avatar of god Vishnu, married to Sita
    \end{description}
  \end{explanation}
\endsong


\beginsong{How Could Anyone}[by={Shaina Noll},tags={you},ph={III}]
  \beginverse
    \[\mn{A}]How could |\[\mnc{D}Dm]anyone ev\[\mn{C}]er |\[G]tell you
    You are |\[C]anything \[C/B]less than |\[Am]beautiful?
    How could |\[Dm]anyone ever |\[G]tell you
    You are |\[C]less \[C/B]than |\[Am]whole?
  \endverse
  \notesoff
  \beginverse
    ^How could |^anyone fail ^to |^notice
    that your |^love is ^just a |^miracle?
    And how |^deeply we are con|^nected
    in our |^souls? | \e
  \endverse
\endsong


\beginsong{Forever Shining}[tags={love},ph={III, IV}]
  \beginverse
    \[\mn{E}]Forever |\[\mnciii{F}{E}{D}Dm]shining forever |\[Am]flowing
    |\[Dm]Guiding me to |\[C]you
    You are |\[Dm]beautiful
    You |\[Am]fill me up with |\[C]love | \e
  \endverse
\endsong


\beginsong{Call Me by My True Name}[by={Yopi Jay},ph={III}]
  \transpose{5}
  \meter{6}{8}
  \beginchorus
    My |\[\mnc{A}Am]joy is like Spring so |\[\mncii{G}{A}C]warm
    it |\[G]makes flowers bloom all |\[Am]over the Earth.
    My |\[Am]pain is like a river of |\[C]tears,
    so |\[G]vast it fills the |\[Am]four oceans
  \endchorus
  \beginchorus
    Please |\[C]call me by my |\[G]true name,
    so I can |\[Dm]hear my cries and |\[Am]laughters at once,
    |\[C]so I can |\[G]feel that my |\[Em]joy and pain are |\[Am]one.
    \vspace{1em}
    Please |\[C]call me by my |\[G]true name
    so that |\[Dm]I can wake |\[Am]up
    and the |\[C]doors of my |\[G]heart will be |\[Em]left o|\[Am]pen.
  \endchorus
  \begin{feeler}
    This song has been inspired by the poem ``Call me by my true names''\\
    by \emph{Thích Nhât Hạnh}, a Vietnamese Buddhist monk and peace activist.
  \end{feeler}
\endsong


\beginsong{Hadoway}[by={Ashleigh Forest},ph={III}]
  \meter{3}{4}
  \beginverse
    |\[\mnc{A}Am]List\[^\mn{B}]en \[^\mn{C}]now |carefully be|loved dear |one
    The |night will not |end 'til these |songs have been |sung
    %% alternate version: this song
  \endverse
  \notesoff
  \beginchorus
    \ind And |\[F]hadee hadow |\[G]wadoway
    \ind Oh |\[Em]hadee hadow |\[Am]wadoway
    \ind |\[Fmaj7]Hadoway ee |yay ee oh |\[E]way | \e
  \endchorus
  \beginverse
    |^Follow your |dreamspell the |music won't |last
    For|get all your |worries your |future your |past
  \endverse
  \goto{And hadee}
  \beginverse
    |^Free fall and |wander there's |nothing to |find
    For the |dance you are |dancing is |all in your |mind
  \endverse
  \goto{And hadee}
  \beginverse
    |^There's no be|ginning and |there is no |end
    With|in this il|lusion just |follow your |way
    %% alternate, apparently original, version:
    %|^There's no be|ginning there's |never an |ending
    %With|in this il|lusion we're |all just pre|tending
  \endverse
  \goto{And hadee}
  \beginverse
    |^I'm just here |present there's |no expla|nation
    And the |dream that I'm |dreaming is |all my cre|ation
    %% alternate, apparently original, version:
    %|^I'm just a |story there's |no expla|nation
    %And the |dream that I'm |dreaming is |all my cre|ation
  \endverse
  \goto{And hadee}
  \beginverse
    |^Infinite |one song re|member what's |true
    There is |nobody |here no |me and no |you
  \endverse
  \goto{And hadee}
\endsong


\beginsong{Music of Silence}[tags={love},ph={III}]
  \beginchorus
    |\[\mnc{D}G]Music \[\mn{B}]of |\[\mnc{C}Am]silence, |\[G]music beyond |\[Am]words
    |\[Dm]Children of the O|\[Em]cean, |\[Am]that's what we are | \e
  \endchorus
  \beginchorus
    |\[Dm]Love is the most shining |\[G]star
    In the |\[C]inner \[C/B]sky of your |\[Am]being
    |\[Dm]Love is the most shining |\[Em]star
    |\[Am]inside you |\[\up{1}(A7)]\e
  \endchorus
\endsong


\begin{intersong}%
  % A fractal image
  \imagecc[1]{Prokofiev_-_Golden_Ratio_fractal_bw_transparent_bg_1543x910px.png}%
  {\scriptsize Fractal pattern with Hausdorff dimension \(\frac{log \varphi}{log \sqrt[\varphi]{\varphi}} = \varphi \approx 1.618034 \),
  the Golden Ratio. The construction is similar to Heighway's dragon, except for the similarity
  ratios. It is generated from an IFS consisting of two similarities of ratios: \(r\) and \(r^2\),
  with \(r=\frac{1}{\varphi^{\left(\frac{1}{\varphi}\right)}}\). By: Prokofiev.}
\end{intersong}


\beginsong{All Related}[by={Nessi Gomes},ex={english, español, lakȟótiyapi},tags={love},ph={III, IV}]
  \audio[]{https://soundcloud.com/nessigomes/all-related-album-version}
  % Note: original is in C#m
  \newchords{chords_allrelated_a}\newchords{chords_allrelated_b}
  %\capo{4} % hmm, this made the page too filled and created problems with the previous song's image placement; TODO: fix
  \meter{3}{4}
  \beginverse\memorize[chords_allrelated_a]
    % note that last note is supposed to be a bit behind, so UL instead of ULL
    \[^\mn{E}]Wakȟáŋ |\[\mnc{A}Am]Tȟáŋ|ka | | Gran E|\[\mnc{G}Em]spíri|tu | | \e
    Agra|\[F]dez|co | | Pacha|\[Em]ma|ma | | \e
    Wakȟáŋ |\[Am]Tȟáŋ|ka | | Tunka|\[Em]shi|la | | \e
    Agra|\[F]dez|co | | Pacha|\[Em]ma|ma | | \e
    En el |\[F]cie|lo, | con los |pájaros que |\[Em]vue|lan | \e
    Las |oraci|\[E]o|nes | \e
  \endverse
  \notesoff
  \beginverse\memorize[chords_allrelated_b]
    \ind |\[^E]Y a|\[Am]mo|\[Em]r mis|\[G]terio de la |\[F]luna
    \ind A|\[Am]mo|\[Em]r mi |\[G]vida preci|\[F]osa
    \ind A|\[Am]mo|\[Em]r |\[G]dame dame |\[F]fuerza
    \ind A|\[Am]mo|\[Em]r con|\[G]fía medi|\[F]cina \up{1}(| | | | \e)
  \endverse
  \beginverse\replay[chords_allrelated_a]
    I crossed |^over | | | anxious |^spirit | | | \e
    Lost my |^knowing | | | confused |^senses | | | \e
    Ancient |^voice|s | | spill their |^secre|ts | | \e
    Taking |^foot|steps | | with our |^mo|ther | | \e
    All re|^la|ted | | broken|^hear|ted | | \e
    No more |\[E]dark|ness | \e
  \endverse
  \beginverse\replay[chords_allrelated_b]
    \ind \up{1}(|With this) |^lo|^ve |^we are all re|^lated
    \ind In |^lo|^ve |^ give me |^strength
    \ind In this |^lo|^ve |^we are all re|^lated
    \ind In |^lo|^ve |^ give me |^strength \replay[chords_allrelated_b]
    \ind In this |\sublyr{lo-}^Nan nana nana |\sublyr{ve}^nanna |^nan nana nana |^Yage
    \ind |^Nan nana nana |^nanna |^nan nana nana |^Yage
    \ind |^Nan nana nana |^nanna |^nan nana nana |^Yage
    \ind |^Nan nana nana |^nanna |^we are all re|^lated
  \endverse\glueverses\beginchorus\replay[chords_allrelated_b]
    \ind |^ |^ |^ |^ \rep{4}
  \endchorus\glueverses\beginverse
    \ind | | | | | \e
  \endverse
  \beginchorus
    |\[C]Cura cura |cura cura |\[B&]cura\ldots | |\[Am]mi | |\[Em]{} | \e \rep{4}
  \endchorus\glueverses\beginverse
    | | medi|\[E]ci|na | \e
  \endverse
  \goto{Y amor}
  \goto{With this love}
  \begin{translation}[EN]
    Wakȟáŋ Tȟáŋka, Great Spirit, I thank you, Pachamama
    Wakȟáŋ Tȟáŋka, Tunkashila, I thank you, Pachamama
    In the sky, with the birds that fly the prayers
    \nextverse
    And love, mystery of the moon; love, my precious life
    Love, give me give me strength; love, trust the medicine
  \end{translation}
  \begin{explanation}[EN]
    \begin{description}
      \item[Wakȟáŋ Tȟáŋka] is the term for the sacred or the divine, the Great Mystery, in Lakota
        spi\-ri\-tuality, which is non-monotheistic. It can be interpreted as the power or the
        sacredness that resides in everything. It is often translated as ``Great Spirit''.
      \item[Tunkashila] is the word for ``grandfather'' in the lakȟótiyapi language. It represents
        the wise grandfather spirit or the male aspect of creation.
    \end{description}
  \end{explanation}
\endsong


\beginsong{More Strength}[by={Second Nature Sounds},ph={III}]
  \beginverse
    \ind \[\mn{E}]Give me just a |\[\mncii{F}{D}Dm]little more strength, oh |\[Am]Jah
    \ind To deal with the |\[Dm]pressures of today, oh |\[Am]Jah
    \ind There's mountains to |\[Gm]climb |\[Am]{} \e
    \ind Help me just to |\[Dm]get over |\[Am]{} \e
    \vspace{0.5em}
    \ind Yes, I need a |\[Dm]little more strength, oh |\[Am]Jah
    \ind It's not an easy |\[Dm]road, heavy load, oh |\[Am]Jah
    \ind Been traveling for |\[Gm]miles |\[Am]{} \e
    \ind Help me just to |\[Dm]get over |\[Am]{} \e
  \endverse
  \beginverse\memorize
    |\[Dm]{} Oh Jah, I'm feeling so we|\[Am]ary, please let me through
    |\[Gm]{} Because my load is so he|\[Am]avy
    But I'm still wi|\[Dm]lling to travel the mo|\[Am]untain so high
    So I'm |\[Gm]asking you father please he|\[Am]lp me, I'm begging you
    \vspace{0.5em}
    |\[Dm]{} I know it wouldn't be e|\[Am]asy, this rocky road
    |\[Gm]{} It sometimes gets kind of whe|\[Am]ezy with my heavy load
    |\[Dm]Gotta move on, but my jo|\[Am]urney is long (yeah)
    |\[Gm]{} Jah please, guide and prot|\[Am]ect me
  \endverse
  \goto{Give me just a little more strength}
  \beginverse
    |^ Oh yes some people try st|^op me from going through
    |^ Yes they've been trying to bl|^ock me
    But it's as go|^od over evil and str|^ength over weakness
    |^ Thank you a lot for your me|^rcies, your love so true
    \vspace{0.5em}
    |^ Sometimes my journey seems lo|^nger, that's when I know
    |^ Oh Lord I need to be str|^onger
    Give me the str|^ength to go on, got your wo|^rks to perform
    Yes I can |^make it oh Jah, if you he|^lp me
  \endverse
  \goto{Give me just a little more strength}
  %\musicnotefornext{interlude}
  \beginverse
    |^ Oh Jah I pray: |^help me face another day
    |^ So much to say to |^all the people along the way
    |^It's not okay to |^see how much they got astray
    |^Pray, to Jah for guidance, |^don't you disregard
    \vspace{0.5em}
    |^ Sometimes it's hard, but |^Lord I've got to keep the faith
    |^ It's my reward, by |^doing good it will be great
    |^Moving on, it's |^not an easy road, I say:
    |^``Hey, brother man, we've |^got to keep the faith''
  \endverse
  \goto{(Give me just a) little more strength}
\endsong


\beginsong{I Sense Your Presence}[by={Shimshai},tags={source, love},ph={III}]
  \transpose{7} % could alternatively use \capo{7} for higher notes
  \beginchorus
    \ind |\[\mnc{E}Am]Om Shabbat Sha|\[\mnc{F}Dm]lom, |\[Em]{} holy way of the most |\[Am]high
    \ind |Om Shabbat Sha|\[Dm]lom, \[Em]I sense your |\[Am]presence
    \ind |\[Em]{} I sense your |\[Am]presence
  \endchorus
  \notesoff
  \beginverse
    |\[Am]{} And I |\[Dm]am the light
    |\[Em]{} with|\[Am]in your soul
    | In the essence of |\[Dm]truth and right
    |\[Em]{} love makes the |\[Am]circle whole
    | And near we |\[Dm]stand in line
    |\[Em]{} waiting for some |\[Am]sacred sign
    | But to find the |\[Dm]balance is the
    |\[Em]purpose of this |\[Am]time,
    | to restore the |\[Dm]balance of the
    |\[Em]universal |\[Am]mind
    \lrep |\[Em]{} I sense your |\[Am]presence \rrep
  \endverse
  \beginverse
    |^ And in the |^presence of
    |^ my lord of |^light and love
    | Every|^thing I see
    in|^spiring to be |^free
    | And when I |^call to thee
    |^ come on |^bended knee
    | Surrender |^to the all
    per|^vading light and |^love,
    | reflection |^of the ones
    sur|^rounding me with |^love
    \lrep |\[Em]{} I sense your |\[Am]presence \rrep
  \endverse
  \beginverse
    |^ Within and with|^out, |^ above and be|^low you
    | East, west, north and |^south, ^I sense your |^presence | \e
    | Without and with|^in, |^ below and a|^bove you
    | East, west, north and |^south, ^I sense your |^presence
    |\[Em]{} I sense your |\[Am]presence
  \endverse
  \goto{Om Shabbat Shalom}
  \begin{explanation}[EN]
    \textbf{Shabbat Shalom} is a Hebrew greeting used around Shabbath (weekly day of rest),
    wishing peace and completion to arise from observing the Sabbath.
  \end{explanation}
\endsong


\beginsong{Amazing Grace}[by={John Newton},ph={III}]
  \meter{3}{4}
  \beginverse
    \[^\mn{A}]A|\[\mnc{D}D]mazing |Grace! how |\[G]sweet the |\[D]sound
    that |saved a |wretch like |\[A7]me!
    I |\[D]once was |lost, but |\[G]now am |\[D]found,
    was |blind, but |\[A7]now I |\[D]see.
  \endverse
  \notesoff
  \beginverse
    'Twas |^grace that |taught my |^heart to |^fear,
    and |grace my |fears re|^lieved;
    how |^precious |did that |^grace ap|^pear,
    the |hour I |^first be|^lieved!
  \endverse
  \beginverse
    Thro' |^many |dangers, |^toils and |^snares
    I |have al|ready |^come;
    'tis |^grace hath |brought me |^safe thus |^far,
    and |grace will |^lead me |^home.
  \endverse
  \beginverse
    When |^we've been |there ten |^thousand |^years,
    bright |shining |as the |^sun,
    we've |^no less |days to |^sing God's |^praise
    than |when we |^first be|^gun.
  \endverse
  \begin{feeler}
    \emph{Aldous Huxley}, in his famous essay \emph{The Doors of Perception} (1954), called
    the mescaline experience a ``gratuitous grace'', a term coined from
    \emph{Thomas Aquinas' Summa Theologiae}.
  \end{feeler}
\endsong


\beginsong{Kumbaya}[ph={IV}]
  \beginverse
    |\[\mnc{C}C]Kum\[^\mn{E}]ba\[^\mn{G}]ya my |Lord |\[F]kumba\[C]ya | \e
    |\[C]Kumbaya my |Lord |\[F]kumba\[G]ya | \e
    |\[C]Kumbaya my |Lord |\[F]kumba\[C]ya | \e
    |\[F]Oh \[C]Lord |  |\[G]kumba\[C]ya | \e
  \endverse
  \notesoff
  \beginverse
    |^Someone's \up{*}singing my |Lord |^kumba^ya | \e
    |^Someone's \up{*}singing my |Lord |^kumba^ya | \e
    |^Someone's \up{*}singing my |Lord |^kumba^ya | \e
    |^Oh ^Lord |  |^kumba^ya |
  \endverse
  \altlyr{Crying, praying\ldots}
\endsong


\beginsong{I Release}[tags={opening, love},ph={III, IV}]
  \beginverse
    \[\mn{A}]I re|\[\mnc{E}Em]lease and I let |go \rep{2}
    I let the |\[D]spirit run my |\[Em]life \rep{2}
    And my |\[Em]arms are open |wide \rep{2}
    And I'm |\[D]only here for |\[Em]love \rep{2}
    No more |\[Em]struggles no more |strife \rep{2}
    With my |\[D]faith I see the |\[Em]light \rep{2}
  \endverse
  \beginchorus
    Yes I am |\[Em]free by the |spirit \rep{2}
    And I'm |\[D]only here for |\[Em]love \rep{2}
  \endchorus
\endsong


\begin{intersong}
  \begin{feeler}
    ``True confession consists of telling our deed in such a way that our soul is changed in the telling it.'' --- \emph{Maude Petre} (1863--1942)
  \end{feeler}
  \vfill
\end{intersong}


\beginsong{Angels of Healing}[by={Lisa Thiel},tags={angels},ph={III}]
  \transpose{-2}
  \meter{3}{4}
  \beginchorus\memorize
    |\[\mnc{E}Em]Bles\[^\mn{F#}]sing |\[\mnc{G}G]angels come |\[D]be with |\[Em]me
    |\[Em]Heal my |\[D]spirit, |\[Em]mind and bo|dy
  \endchorus
  \notesoff
  \beginchorus
    |^Blessing |^angels of |^green and |^gold
    |^Heal my |^heart and |^heal my |soul
  \endchorus
  \beginchorus
    |^Blessing |^angels of |^violet and |^blue
    |^Open my |^eyes to the |^vision of |truth
  \endchorus
\endsong


\beginsong{I Am the Light of the Soul}[ititle={Bliss},ph={III}]
  \meter{6}{8}
  \beginchorus
    |\[\mnc{A}Am]I \[\mn{E}]am the |\[C]light of the soul I am |\[G]bountiful
    I am |\[F]beautiful I am |\[Dm]bliss I am I |\[E]am
  \endchorus
  \beginchorus
    |\[Am]I am the light of the |\[G]soul I am |\[Dm]bountiful
    I am |\[F]beautiful I am |\[Dm]bliss I am I |\[E]am
  \endchorus
  \begin{feeler}
    ``\ldots he that beholds must be akin to that which he beholds, and must,
    before he comes to this vision, be transformed into its likeness.
    Never could the eye have looked upon the sun had it not become sun-like,
    and never can the soul see Beauty unless she has become beautiful.''
    --- \emph{Plotinus} (c. 240--270) in \emph{Enneads}
  \end{feeler}
\endsong


\beginsong{Holy}[by={Netanel Goldberg},ph={III},key={C},gk={C, C--D}]
  \beginverse
    |\[\mnc{C}C]Fly \[\bm] like a r|\[\mncadj{.5ex}{B}Em]iver\[\bm], { } |\[\mnc{C}C]flow \[\bm] \[^\mn{E}]with the o|\[\mncadj{.5ex}{B}Em]{-}cean \[\bm]
    |\[Dm]Fly \[\bm] on the w|\[C]ind \[\bm] tha|\[G]{t blows} \[\bm] through the w|\[F]inter \[\bm]
    |\[C]Dream \[\bm] abo|\[Em]ut love\[\bm], bel|\[C]ieve \[\bm] in y|\[Em]our dreams \[\bm]
    |\[Dm]Live \[\bm] in the o|\[C]{-}cean \[\bm] o|\[G]{f love} \[\bm]
    |Close your ey\[\bm]es and f|\[G7]eel the w\[\bm]ind that is b|\[C]lowing \[\bm]
    |\[Em]Open up your h\[\bm]ands and s|\[G]ing \[\bm]
  \endverse
  \beginchorus
    \ind |^I ^ am |^holy^, { } |^I ^ am |^holy ^
    \ind |^I am ^here t|^o live^ thi|^{s life}^ \up{2}(|{ }{ } \[\bm] \e)
  \endchorus
  \notesoff
  \beginverse
    |^Dance ^ in not k|^nowing^, |^know ^ your perfec|^{t power} ^
    |^Dance ^ like a l|^io^n |^{-} ^ in the wi|^ld ^
    |^Laugh ^ like a chi|^ld, ^ s|^ing ^ in ful|^{l presence}^
    |^Sing ^ with the l|^ions ^ in the w|^ild ^
    |Close your ey^es and f|^eel the w^ind that is b|^lowing ^
    |^Open up your h^eart and s|^ing ^
  \endverse
  \beginchorus
    \ind |^I ^ am |^holy^, { } |^I ^ am |^holy ^
    \ind |^I am ^here t|^o live^ thi|^{s life}^ \up{2}(|{ }{ } \[\bm] \e)
  \endchorus
  \textnotefornext{outro:}\vspace{-1ex}
  \beginverse\chordsoff
    \ind |^I ^ am |^open ^ to |^see ^ flowers |^blooming^
    \ind |^I ^ am here t|^o live^ thi|^{s life}^ | \e
  \endverse
\endsong


\beginsong{Astral Bliss}[ititle={Divine Star},by={Giita Rani},ph={III}]
  \audio[]{https://www.youtube.com/watch?v=ootDkAMN4lk}
  \meter{3}{4}
  \beginverse
    \[^\mn{B}\mn{G}]I |\[\mnc{F#}Em]saw \[^\mn{E}]the \[^\mn{B}]as\[^\mn{D}]tral |\[^\mn{B}]bliss, lighte|ning upon all |\[C]beings
    Bet|ween good and e|\[D]-vil, he|re Jesus |\[G]lived
    Now |\[Em]here we |are in |\[D]front of this |\[Em]cross
    It's |\[Am]firmament of |light down |\[D]here guiding |\[Em]us | | | \e
  \endverse
  \notesoff
  \beginverse
    Oh |^my under|standing, kind of |flower of |^love
    For|give bad int|^ention, as a |human still I |^walk
    I'm |^neither good nor |bad, I |^am what I |^am
    Di|^vine shining |star of \up{*}e|^ternal self I |^am | | | \e \altlyr{|love I am a |part}
  \endverse
  \beginverse
    Di|^vine shining |star, soft |hand carrying |^us
    |Through the univ|^erse, he|aling the |^Earth
    Di|^vine shining |star, white |^light guiding |^us
  \endverse\glueverses\beginchorus
    Om |\[Am]Bhur Bhuwaswa|ha en|\[D]lightening the |\[Em]Earth | | | \e
  \endchorus
\endsong


\beginsong{Universal Lover \\I'm in You and You're in Me}[by={Fantuzzi},tags={source},key={Dm},gk={Dm,Am--Em},ph={III, IV}]
  \audio[key=Em]{https://www.youtube.com/watch?v=isBLJ-G86FY}
  \audio[key=Em]{https://soundcloud.com/kat-dancer/10-universal-lover}
  \transpose{-2} % in Em the notes range from D to B, in transposed Dm from C to A
  \beginverse
    \[^\mn{E}]You are \[^\mn{F#}]my |\[\mnc{G}Em]mo\[^\mn{E}]ther, \[\bm]{} you are \[^\mn{F#}]my |\[^\mn{G}]fa\[^\mn{E}]ther \[\bm]{}
    You are my |\[D]lover and you \[\bm]are my |\[Em]friend \[\bm]{}
    You're the be|ginning, \[\bm]{} you are the |center \[\bm]{}
    And you |\[D]are be\[\bm]yond the |\[Em]end \[\bm]{}
  \endverse
  \notesoff
  \beginverse
    You are the |^colors ^ of the |rainbow ^ \altchords{\id[1]{(Am)}|Am | \e}
    You are the |^pure white ^light in |^me ^ \altchords{|G |Am}
    You are the |river, ^ you are the |mountain ^ \altchords{| - | \e}
    You are the |^sky, you ^are the |^sea ^ \altchords{|G |Am}
  \endverse
  \beginverse\noteson
    \ind \[\mn{E}]And |\[\mnc{B}Em]I \[\bmc\mn{A}]love \[\mn{G}]you |\[\mnciii{A}{G}{F#}D]so \[\bm]{} '\[\mn{G}]cause |\[\mn{A}]you \[\bmc\mn{G}]help \[\mn{F#}]me |\[\mnc{E}Em]see \[\bm]{} \altchords{|Am |G | - |Am}
    \ind To |see \[\bm]you in |\[D]all \[\bm]{} is to |see \[\bm]you in |\[Em]me \[\bm]{} \altchords{| - |G | - |Am}
    \ind \lrep 'Cause |\[D]I'm in you and \[\bm]you're in |\[Em]me \[\bm]{} \rrep\rep{3} \altchords{|G |Am}
  \endverse
  \beginverse
    I want to |^touch you, ^ I want to |feel you ^
    I want to |^be right ^by your |^side ^
    I want to |know you, ^ I want to |love you ^
    I want to |^serve you ^all the |^time
  \endverse
  \beginverse
    \[\bm]Profets and re|^ligions ^ are |many ^ \altchords{\id[2]{(Em)}|Em | \e}
    But |^God we ^know is |^one ^ \altchords{|D |Em}
    Teachers and |teachings ^ are |many ^ \altchords{| - | \e}
    But the |^truth we ^know is |^one ^ \altchords{|D |Em}
  \endverse
  \beginverse
    \ind And |\[Em]I \[\bm]love you |\[D]so \[\bm]{} 'cause |you \[\bm]help me |\[Em]see \[\bm]{} \altchords{|Em |D | - |Em}
    \ind To |see \[\bm]you in |\[D]all \[\bm]{} is to |see \[\bm]you in |\[Em]me \[\bm]{} \altchords{| - |D | - |Em}
    \ind \lrep 'Cause |\[D]I'm in you and \[\bm]you're in |\[Em]me \[\bm]{} \rrep\rep{3} \altchords{|D |Em}
  \endverse
  \begin{feeler}
    ``All reality is one in substance, one in cause, on in origin\ldots and every particle of
    reality is composed inseparably of the physical and the psychic. The object of philosophy,
    therefore, is to preserve unity in diversity, mind in matter and matter in mind\ldots
    To rise to that highest knowledge of the universal unity is the intellectual equivalent
    of the love of God.'' --- \emph{Giordano Bruno} (1548--1600)
  \end{feeler}
\endsong


\beginsong{Through the Looking Glass}[by={Zeger Vandenbussche}, ph={III, IV}, key={Am}, gk={Cm, Am--D\shrp{}m}]
  \beginverse
    \[^\mn{A}]We're |\[Am]look\[^\mn{E}]ing through an |\[G]o\[^\mn{D}]pen\[^\mn{C}]ing \[^\mn{D}]in |\[\mnc{E}C]may\[^\mn{D}]a's \[^\mn{E}]veil
    in|\[Em]to the real be|\[Am]yond
    We |\[G]open up our |\[C]eyes to the |\[Em]splendour all a|\[Am]round
    There's |\[G]nothing real but |\[C]love, all
    |\[Em]else is without |\[Am]ground | \e
  \endverse
  \notesoff
  \beginverse
    We |^rise above the |^battle and we |^see it was
    |^just a game we |^played
    We |^leave our toys be|^hind and |^listen to the |^call
    We |^ring our words like |^bells and
    |^sing our way back |^home | \e
  \endverse
  \beginverse
    We're |^falling, falling, |^falling through the |^looking glass
    to the |^other side of |^fear
    And |^when we're here we |^see the |^veil has just one |^side
    Il|^lusions melt a|^way we
    |^leave it all be|^hind | \e
  \endverse
  \beginverse
    We're |^falling, falling, |^falling through the |^looking glass
    in|^to each others |^arms
    At |^last we meet for |^real in the |^light of the inner |^Sun
    |^shining through the |^veil and
    |^lightening up our |^eyes | \e
  \endverse
  \begin{explanation}[EN]
    \begin{description}
     \item[maya:] the illusion of the reality of sensory experience and of the experienced
       qualities and attributes of oneself (in Vedic philosphy)
    \end{description}
  \end{explanation}
  \yesendsongvfill
\endsong


\beginsong{Wake Up}[by={Mirabai},tags={love, source},ph={III, IV}]
  \beginchorus
    |\[\mnc{G}G]Wake \[\mn{B}]up, |\[D]rise up, sweet |\[Em]{} fami|ly
    It's a |\[C]{} time for the |Lord, and re|\[D]member love is |here
    |\[G]Love, |\[D]love is |\[Em]{} all you |see
    If you |\[C]{} wake up and |\[D]rise up right a|\[G]way | \e
  \endchorus
  \beginchorus
    The Lord has |\[D]blessed you | in so many |\[Em]ways | \e
  \endchorus\glueverses
  \beginverse
    So |\[Am]rise up right |\[Em]now and sing this |\[D]prayer | \e
  \endverse
\endsong


\beginsong{Pacha Mama}[by={Ronny Hickel},tags={flying},ph={III, IV}]
  \beginverse
    \[\mn{E}]I wanna be |\[\mnc{B}Em]free, so |\[\mncii{A}{F#}D]free
    Like a |\[Am]flower and a |\[Em]bee
    Like a |\[Em]bird in the |\[Bm]tree
    Like a |\[Am]dolphin in the |\[Em]sea
  \endverse
  \notesoff
  \beginverse
    ^I wanna fly |^high, so |^high
    Like an |^eagle in the |^sky
  \endverse\glueverses\beginchorus
    And |\[Em]when my time has |\[Bm]come
    I'm gonna |\[Am]lay down and |\[Em]fly
  \endchorus
  \beginchorus
    \ind Pacha |\[Em]Mama | I'm coming |\[Bm]home | \e
    \ind To the |\[Am]place |\[Bm]{} where I be|\[Em]long | \e
  \endchorus
  \beginverse
    ^I wanna be |^free, be |^me
    Be the |^being that I |^see
    Not to |^rise and not to |^fall
    Being |^one and loving |^all
  \endverse
  \beginverse
    ^There's no |^high there's no |^low
    There's |^nowhere else to |^go
  \endverse\glueverses\beginchorus
    Just in|\[Em]side a little |\[Bm]star
    Telling |\[Am]me: be as you |\[Em]are \goto{Pacha Mama}
  \endchorus
\endsong


\beginsong{Every Part of the Earth}[tags={Mother Earth},ph={III}]
  \beginchorus
    |\[\mnc{D}Dm]Every part of the |\[\mnc{C}C]Earth
    Is sacred to my |\[Dm]people | \e
    |\[Dm]We are part of the |\[C]Earth
    And she is part of |\[Dm]us | \e
  \endchorus
  \beginchorus
    \ind |\[Dm]All beings share the |\[C]same breath
    \ind |\[Am]All beings share the |\[Dm]same breath
  \endchorus
  \beginchorus
    If |\[Dm]beasts were gone we would |\[C]die
    Of a |\[Dm]great loneliness of |spirit
  \endchorus
  \beginchorus
    \ind |\[Dm]All things are co|\[C]nnected
    \ind |\[Am]All things are co|\[Dm]nnected
  \endchorus
  \beginchorus
    |\[Dm]This we know that the |\[C]Earth
    Does not belong to |\[Dm]us; |we belong to the Earth!
  \endchorus
  \beginchorus
    \ind |\[Dm]Our God is the |\[C]same God
    \ind |\[Am]Our God is the |\[Dm]same God
  \endchorus
\endsong


\beginsong{Starwalker}[by={Buffy Sainte-Marie},ph={IV}]
  \beginverse
    |\[Am]{} \[^\mn{C}]Heya hey hey |\[^\mn{A}]heya hey hey
    Heya |\[F]hey hey \[C]hey heya |\[Em]heya hey heya hey
    |\[Am]Heya hey yoh\ldots | |\[F]{}  \[C]{} Ay |\[Em]heya heyo heya
  \endverse
  \beginverse
    |^ Starwalker, he's a |friend of mine
    |^ You've ^seen him |^looking fine
    He's a |^ straight talker, he's a | Starwalker
    Don't |^drink no ^wine Ay |^heya heyo heya
  \endverse
  \beginverse
    |^ Wolf Rider she's a |friend of yours
    |^ You've ^seen her |^opening doors
    She's a |^history turner, she's a |sweetgrass burner
    And a |^ dog ^soldier Ay |^heya heyo heya
  \endverse
  \beginverse
    |^ Holy light, |guard the night
    |^ Pray ^up your |^medicine song
    Oh, |^ straight dealer you're a | spirit healer
    Keep |^going o^n Ay |^heya heyo heya
  \endverse
  \beginverse
    |^ Lightning Woman, |Thunderchild
    |^ Star ^soldiers |^one and all oh
    |^Sisters, Brothers |all together
    |^ Aim ^straight, |^ stand tall
  \endverse
  \goto{Heya hey hey}
  \goto{Starwalker}
  \goto{Heya hey hey}
  % Image by: Sandra Lange, scanned by: larva
  % Image license: added here with permission
  \imagecc[2]{mantobird_by_Sandra_Lange_2013_bw_withperm_1195x1158px.png}%
  \begin{explanation}[EN]
    \begin{center}
      {\tiny Image by: Sandra Lange}%
    \end{center}
  \end{explanation}
\endsong


\beginsong{Reconnected}[by={Murray Kyle},tags={wisdom, Mother Earth},ph={IV}]
  \meter{6}{8}
  \beginverse
    A |\[\mnc{A}Am]fresh new \[\mn{C}]breeze is |\[G/B]growing
    The |\[F]winds of change are |\[G/B]blowing
    |\[Am]Every moment's a |\[G/B]miracle of |\[D]life | \e
  \endverse
  \notesoff
  \beginverse
    The |^ancient ^wisdom |^has returned
    And |^walking on this |^land we've learned
    That |^we are the |^ones we've been waiting |^ for | \e
  \endverse
  \beginverse
    We are the |^elders, ^we are the |^children
    We carry |^with us all that has |^ever been
    We are the |^vision, |^circling once |^again | \e
  \endverse
  \beginverse
    And it is |^timeless, ^it is |^blossoming
    in the |^lives we live and the |^songs we sing
    \lrep We are |^reconnected to the \brk|^spirit of the |^ Earth | \e \rrep\rep{3}
  \endverse
\endsong


\begin{intersong}
  \begin{feeler}
    ``Meaning makes all things bearable'' --- \emph{Carl Jung} (1875--1961)
  \end{feeler}
  \vfill
\end{intersong}


\beginsong{Strong Wind \\ Pure Earth \\ Pyhä maa}[tags={earth, air, water, fire}, ph={IV}, key={Em}, gk={Em, Cm--Em}]
  \beginchorus\memorize
    \[^\mn{G}\mn{E}]Strong |\[Em]wind, deep |\[\mncii{D}{B}Bm]water, \[^\mn{G}\mn{E}]tall |\[Em]trees, warm |\[\mncii{D}{B}Bm]fire \altchords{\id[1]{(Dm)}|Dm |Am |Dm |Am}
    I can |\[C]feel it in my |\[D]body, I can |\[C]feel it \[Bm]{in my} |\[Em]soul \altchords{|B\flt{} |C |B\flt{} Am |Dm}
  \endchorus
  \beginchorus
    \lrep Heya |\[Em]heya heya \[\bm]heya heya |\[Bm]heya heya \[\bm]ho \rrep \altchords{|Dm |Am }
    Heya |\[C]he\[\bmadj{-.5ex}]ya heya |\[D]heya \[\bm]{} heya |\[C]heya \[D]heya |\[Em]ho \[\bm]{} \altchords{|B\flt{} |C |B\flt{} C |Dm}
  \endchorus
  \notesoff
  \textnotefornext{suomeksi:}
  \beginchorus % Finnish by S
    Pyhä |^maa, syvä |^vesi, lämmin |^tuli, vahva |^tuuli
    Ne voin |^tuntea ihol|^lani, ne voin |^tuntea ^sielus|^sain
  \endchorus
  \goto{Heya}
  %% alt. Finnish
  %\beginchorus
  %  ^Puh^das |^maa, lämmin |^tuli, kova |^tuuli, syvä |^vesi
  %  Mä vain |^tunnen ihol|^lani, ne mä |^tunnen ^sisäs|^säin
  %\endchorus
  \vfill% move alt. lyrics down
  \beginverse
    \small
    \textnotefornext{Alternate lyrics:}
    Pure earth, warm fire, strong wind, deep water
    I can feel them in my body, I can feel them in my soul
  \endverse
  \noendsongvfill
\endsong


\beginsong{One Planet \\ Our Planet Is Turning \\ Winter Solstice Chant}[tags={Mother Earth},ph={IV}]
  \beginverse
    |\[Dm]{} \[\mn{D}]One planet \[\mn{E}]is |\[\mn{F}]turning | circle on her
    |\[F]path a\[C]round the |\[Dm]sun
    Earth mother is |\[Dm]calling
    |\[Dm]{} her \[C]children |\[Dm]home
  \endverse
  \notesoff
  \beginverse
    |^ ^The light is ^re|^turning | although it
    |^is the ^darkest |^hour
    No one can |^hold
    |^ ^back the |^dawn
  \endverse
  \beginverse
    |^ ^Let's keep ^it |^burning, | let's keep the
    |^flame of the ^hope a|^live
    Make safe our |^journey
    |^ ^through the |^storm
  \endverse
\endsong


\beginsong{Three Little Birds}[by={Bob Marley},tags={morning},ph={IV}]
  \ifchorded%
    \beginverse
      \musicnotefornext{intro:}*
      {\nolyrics%
        |\[A] |  |  | \e
      }%
    \endverse
  \fi%
  \beginverse
    \ind Don't |\[\mnc{B}A]wor\[\mn{A}]ry, about a th|ing
    \ind 'Cause |\[D]every little thing, gonna be alr|\[A]ight
    \ind Singin' don't |worry, about a th|ing
    \ind 'Cause |\[D]every little thing, gonna \[(G)]be \[D]alr|\[A]ight
  \endverse
  \beginverse
    Rise up this |\[A]mornin'
    Smile with the |\[E]rising sun
    Three little |\[A]birds perch by my |\[D]doorstep
    Singin' |\[A]sweet songs
    of melodys |\[E]pure and true
    sayin', |\[D]{} this my message to you-|\[A]oo-oo
  \endverse
  \vspace{1em}
  \goto{Don't worry}
  \goto{Rise up this morning}
  \vspace{1em}
  \musicnotefornext{outro, repeat to fade out:} \goto{Don't worry}
\endsong


\beginsong{We Are One in Harmony}[by={Michael Stillwater},tags={circle},ph={IV}]
  \beginchorus\memorize
    |\[\mnc{B}G]We \[^\mn{A}]are \[^\mn{G}]one \[^\mn{B}]in |\[\mnc{A}F]har\[^\mn{G}]mo\[^\mn{F}]ny |\[\mnc{G}C]singing in cele|\[\mnc{A}G]bra\[^\mn{B}]tion
    |We are one in |\[F]harmony |\[C]singing in |\[G]love
  \endchorus
  \notesoff
  \beginverse
    We are |^o|^ne |^singing in cele|^bration
    We are |o|^ne |^singing in |^love
  \endverse
\endsong


\beginsong{We Are Circling}[tags={circle},ph={IV}]
  \beginverse
    |\[\mnc{D}Dm]We are \[\mnc{C}C]circl\[Dm]ing, |circling to\[C]ge\[Dm]ther
    |\[Dm]We are \[F]singing, |\[Am]singing our \[Dm]heart song
    |\[Dm]This is \[C]fami\[Dm]ly, |this is \[C]uni\[Dm]ty
    |\[Dm]This is cele\[C]bra\[Dm]tion, |\[Am]this is \[Dm]sacred | \e
  \endverse
\endsong


\beginsong{Jay Ma}[tags={Divine Mother},ph={IV}]
  \beginchorus\memorize
    \[^\mn{E}]When \[^\mn{G}]the |\[\mnc{B}Em]fire in my s|oul
    Burns a |\[C]longing for the g|\[Em]oal
    Then I |\[C]know in my |\[D]heart it is |\[Em]you | \e
  \endchorus
  \notesoff
  \beginchorus
    \ind Jay |^Ma jay |Ma jay |^Ma jay |^Ma
    \ind Jay |^Ma jay |^Ma jay jay |^Ma | \e
  \endchorus
  \beginchorus
    When the |^truth is rev|ealed
    All the |^sorrow will be h|^ealed
    And I |^know in my |^heart it is |^you | \e \goto{Jay Ma}
  \endchorus
  \beginchorus
    In the |^still of the n|ight
    From the |^darkness comes the l|^ight
    And I |^know in my |^heart it is |^you | \e \goto{Jay Ma}
  \endchorus
  \begin{explanation}[EN]
    \textbf{Jay Ma} could be translated as ``Praise the Mother (of the World)''
  \end{explanation}
\endsong


\beginsong{May the Love \\ Loka Samasta Sukino Bhavantu}[by={Peter Makena},tags={love, source},ph={IV, V}]
  \beginverse
    \[\mn{E}]May the |\[Am]love \[^\mn{D}]we \[^\mn{C}]share \[^\mn{B}]here |\[^\mn{A}]spread \[^\mn{B}]its \[^\mn{C}]wings
    |Fly across the |earth
    And sing a |\[G]song to every |\[Em]soul
    That is a|\[Am]live | \e
  \endverse
  \notesoff
  \beginverse
    ^May the |^blessing of the |Universe
    % alternate line, original(?): ^May the |^blessing of your |grace my Lord
    |Shine on every |one
    And may we |^all see the light with|^in
    The light with|^in | \e
  \endverse
  \beginchorus
    \ind Lo|\[G]ka samasta |\[Em]sukino bha|\[Am]vantu | \e
    \ind May |\[G]all the beings of |\[Em]all the worlds be |\[Am]happy | \e
  \endchorus
  \beginverse
    Halle|\[Am]luya Hal|leluya
    Halle|\[G]luya Hal|\[Em]leluya
    Halle|\[Am]luya | \e
  \endverse
  \beginverse
    \emph{Salam aleycum\ldots Shalom aleichem\ldots Hare Krishna\ldots}
    \emph{Jah Rastafari\ldots Om Mani Padme Hum\ldots Pachamama\ldots}
    \emph{(\ldots)}
  \endverse
\endsong


\beginsong{Every Little Cell}[by={Clemens Kuby},tags={happiness},ph={IV, V}]
  \beginchorus
    |\[\mnc{E}C]Every \[\mn{D}]little \[\mn{C}]cell in my |body is happy
    |\[C]Every little cell in my |\[G]body is \[C]well
  \endchorus
  \beginchorus
    |\[C]I'm so glad e|very little cell
    |\[C]In my body is |\[G]happy and \[C]well
  \endchorus
  % Image downloaded from: https://freesvg.org/animal-cell-vector-illustration
  % Image license: Public Domain (CC0)
  % Slightly edited by: larva
  \imagecc[2]{Animal_Cell_ed_by_larva__transbg_CC0_991x733px.png}%
\endsong


\beginsong{Don't Worry Be Happy}[by={Bobby McFerrin},ph={IV, V}]
  \audio[key={B}]{https://www.youtube.com/watch?v=d-diB65scQU}
  \beginchorus
    \musicnotefornext{whistle}
    \ind |\[\mnc{G}G]\[\mn{E}\mn{D}] | |\[Am]{} | \e
    % \ind |\[\mnc{G}G]\[\mn{E}\mn{D}]|\[\mn{E}\mn{A}\mn{B}] \[\mn{D}\mn{B}\mn{A}\mn{G}]|\[\mncii{A}{B}Am]\[\mn{A}] |\hspace{1em}\[\mn{D}\mn{B}\mn{A}\mn{G}]
    \ind |\[C]{} | |\[G]{} | \e
  \endchorus
  \beginverse
    |\[G]{} \[^\mn{B}]Here's \[^\mn{A}]a \[^\mn{B}]litt\[^\mn{A}]le |\[^\mn{B}]song I \[^\mn{D}]wrote
    You |\[Am]might want to sing it no|te for note
    Don't |\[C]worry, | be |\[G]happy | \e
  \endverse\notesoff\glueverses\beginverse
    |^ In ev'ry life we ha|ve some trouble
    |^ But when you worry you ma|ke it double
    Don't |^worry, | be |^happy \sublyr{Don't}\hspace{1em} |\sublyr{worry, be happy now} \e
  \endverse
  \noteson
  \beginchorus
    \ind |\[\mnc{G}G]Oo\[\mn{E}\mn{D}]{o\ldots}|\[\mn{E}\mn{A}\mn{B}]{\ldots} \[\mn{D}\mn{B}\mn{A}\mn{G}]{Ooo\ldots}|\[\mnciii{A}{B}{A}Am]{\ldots} \hspace{.5em} \sublyr{Don't}\hspace{1em} |\sublyr{worry}\hspace{1.5em} \[\mn{D}\mn{B}\mn{A}\mn{G}]{Ooo\ldots}
    \ind |\[C]{\ldots} \sublyr{Be}\hspace{.5em} |\sublyr{happy}\hspace{1.5em} Ooo\ldots|\[G]{\ldots} \sublyr{Don't}\hspace{1em} |\sublyr{worry, be happy} \e
  \endchorus
  \notesoff
  \beginverse
    |^ Ain't got no place to la|y your head
    |^Somebody came and to|ok your bed
    Don't |^worry, | be |^happy | \e
  \endverse\glueverses\beginverse
    The |^landlord say your re|nt is late
    |^ He may have to l|itigate
    Don't |^worry, | be |^happy \sublyr{Look at}\hspace{1.5em} |\sublyr{me, I'm happy} \e
  \endverse
  \goto{Ooo}
  \beginverse
    |^ Ain't got no cash, ain't go|t no style
    |^Ain't got no gal to ma|ke you smile
    Don't |^worry, | be |^happy | \e
  \endverse\glueverses\beginverse
    'Cause |^when you worry your fa|ce will frown
    |^ And that will bring ev'ryb|ody down
    Don't |^worry, | be |^happy \sublyr{Don't}\hspace{1em} |\sublyr{worry, be happy now} \e
  \endverse
  \goto{Ooo}
\endsong


\beginsong{Dear Friends}[by={Gabriel Meir Harevy},tags={love, peace},ph={III, V}]
  \beginverse
    |\[\mnc{E}Em]Dear \[\mnc{D}D]friends |\[Em]dear \[Bm]friends
    |\[Em]Let me \[D]tell you |\[C]how I \[Bm]feel
    |\[Em]You have \[D]given me |\[C]so much \[Bm]pleasure
    |\[C]I \[Bm]love |\[Em]you
  \endverse
  \notesoff
  \beginverse
    |^Love ^love |^love ^love
    |^The mes^sage |^is ^love
    |^Love your ^neighbor |^as thy^self
    |^Love ^love |^love
  \endverse
  \beginverse
    |^Peace ^peace |^peace ^peace
    |^Wars have ^been, |^wars must ^cease
    |^West ^east |^north and ^south
    |^Peace ^peace |^peace
  \endverse
  \beginverse
    A|^mor a^mor a|^mor a^mor
    |^El men^saje |^es a^mor
    |^Ama tu pro^jimo |^como a ti ^mismo
    |^El es ^tu a|^mor
  \endverse
  \textnotefornext{suomeksi:}
  \beginverse
    |^Rakkaat ^ystävät, |^rakkaat ^ystävät
    |^Antakaa kun ^kerron |^mitä mä ^tunnen
    |^Ootte ^antaneet |^mulle niin ^paljon
    |^Ra^kastan |^teitä
  \endverse
\endsong


\beginsong{Thank You}[tags={thankfulness},ph={V}]
  \beginchorus
    |\[\mnc{D}Dm]Thank you \[\mn{F}]for \[\mn{G}]the *\[\mn{A}]day, Lord
    |\[C]Thank you for the \[Dm]*day
  \endchorus
  \beginchorus
    It's |\[Dm]healing it's healing it's |\[C]healing m\[Dm]e
  \endchorus
  \textnotefornext{suomeksi:}
  \beginchorus
    |\[Dm]Kiitos kiitos *päivästä
    |\[C]Kiitos *päiväs\[Dm]tä
  \endchorus
  \beginchorus
    Se |\[Dm]parantaa ja antaa |\[C]meille voima\[Dm]a
  \endchorus
  \altlyr{Friends, circle, \ldots}
  \begin{feeler}
    ``Gratefulness is heaven itself.'' --- \emph{William Blake} (1757--1827)
  \end{feeler}
\endsong


%%%%%%%%%%%%%%%%%%%%%%%%%%%%%%%%%%%%%%%%%%%%%%%%%%%%%%%%%%%%%%%%%%%
%%% LATEST PRINTOUT CONTAINED THE SONGS ABOVE.                  %%%
%%%%%%%%%%%%%%%%%%%%%%%%%%%%%%%%%%%%%%%%%%%%%%%%%%%%%%%%%%%%%%%%%%%
%%% Please try to not change the song numbers above this point. %%%
%%% Add new songs only after this point.                        %%%
%%%%%%%%%%%%%%%%%%%%%%%%%%%%%%%%%%%%%%%%%%%%%%%%%%%%%%%%%%%%%%%%%%%


\beginsong{Healing Song}[by={Kailash Kokopelli},tags={health}, ph={III}, key={G}, gk={G, G--D}]
  \mnbeginchorus
    \[\mn{G}]My \[\bm]Body \[\mn{A}]is |\[\mnc{B}G]hea\[\mn{A}]ling it|\[\mn{G}]self, my \[\mn{A}]Body is |\[\mn{G}]strong | \e
    \mnendchorus\glueverses\mnbeginverse
    |\[\mnc{C}C]Healing, |\[\mn{E}]my Body \[\mn{D}]is |\[\mnc{B}G]heali\[\mn{A}\mn{G}\mn{B}]ng | \e
    \[\mn{G}]My Body \[\mn{A}]is |\[\mnc{B}]hea\[\mn{A}]ling it|\[\mn{G}]self, my \[\mn{A}]Body is |\[\mn{G}]strong | \e
  \mnendverse
  \beginchorus
    \textbf{My Spirit} is |\[G]healing my |body, \textbf{my Spirit} is |strong | \e
    \endchorus\glueverses\beginverse
    |\[C]Healing, |\textbf{my Spirit} is |\[G]healing | \e
    \textbf{My Spirit} is |healing my |body, \textbf{my Spirit} is |strong | \e
  \endverse
  \vspace{1ex}
  \beginverse\chordsoff\bfseries
    My Soul\ldots
    My Love\ldots
    The Sun\ldots
    The Moon\ldots
    The Stars\ldots
    The Earth\ldots
    The Water\ldots
    The Air\ldots
    The Fire\ldots
    You\ldots
    I\ldots
    This song\ldots
  \endverse
  \showlilypondfalse % to save space
  % Present the melody on a staff using Lilypond
  \begin{lilywrap}\begin{lilypond}[]
    % transcribed by larva, latest update on 2023-06
    \include "tex/lp-include-head.ly"
    theMelody = \relative g' {
      \set melismaBusyProperties = #'() \slurDashed
      \key g \major \time 4/4 \partial 2.
      \repeat volta 2 {
        g4 g8( g) a4 | b2 a4 a4 | g8( g8) g4 a8( a8) a4 | g1~ | g4
      } \break
      r2. | c2 c2~ | c4 c4 c8( c8) c4 | b2 \once\slurSolid b8( a g b8~) | b4
      g4 g8( g) a4 | b2 a4 a4 | g8( g8) g4 a8( a8) a4 | g1~ | g4 \bar "|."
    }
    theLyricsOne = \lyricmode {
      \set stanza = "1."
      %\repeat volta 2 { % skip repeat here as it creates a problem with some lp versions
        My Bo -- dy is | hea -- ling it -- | self, __ _
        my Bo -- dy is | strong. | _
      %}
      | Hea -- ling, | _
      my Bo -- dy is | hea -- ling; _ _ _ | _
      My Bo -- dy is | hea -- ling it -- | self, __ _
      my Bo -- dy is | strong. | _
    }
    theChords = \chordmode {
      \repeat volta 2 {
        s2. | g1 | g | g | g4
      }
      s2. | c1 | c1 | g1 | g | g | g | g4
    }
    %\layout { #(layout-set-staff-size 14) } % for better fit
    \include "tex/lp-include-tail-notab.ly"
  \end{lilypond}\end{lilywrap}
\endsong


    % Songs in other languages
% ========================
%
% The following sets the song number for the first song in this file.
% The number will automatically be incremented by one for each song.
% Please do not change this! Changing would make different versions of
% the songbook to have different numbers for the same songs, and it
% would totally mess up the selection booklets causing them to have
% wrong songs in them. (For the same reason, add new songs only to the
% end of each songs_ file.)
\setcounter{songnum}{500}


\beginsong{Ancient Aramaic Prayer}[ph={I}]
  % NOTE: try to align the translation with the prayer with \vskips. Check this
  % after style changes etc.
  \chordsoff % there are no chords
  \beginverse\justifycenter
    \vspace{3em}
    Abwûn d'bwaschmâja
    \vspace{1em}
    Nethkâdasch schmach
    \vspace{1em}
    Têtê malkuthach.
    \vspace{1em}
    Nehwê tzevjânach aikâna d'bwaschmâja af b'arha.
    \vspace{1em}
    Hawvlân lachma d'sûnkanân jaomâna.
    \vspace{1em}
    Waschboklân chaubên wachtahên aikâna\\
    daf chnân schwoken l'chaijabên.
    \vspace{1em}
    Wela tachlân l'nesjuna
    \vspace{1em}
    ela patzân min bischa.
    \vspace{1em}
    Metol dilachie malkutha wahaila wateschbuchta l'ahlâm almîn.
    \vspace{1em}
    Amên.
    \vspace{5em}
    \imagec[4]{ancient_aramaic_symbol_bw_transparent_bg_184x225px.png}
  \endverse
  \brk % to suggest putting a page break here
  \begin{translation}[EN]\justifycenter
    \vspace{-2em} % to align
    Oh Thou, from whom the breath of life comes,
    who fills all realms of sound, light and vibration.
    \vspace{1em} % to align
    May Your light be experienced in my utmost holiest.
    \vspace{1.5em} % to align
    Your Heavenly Domain approaches.
    \vspace{0.5em} % to align
    Let Your will come true --- in the universe \emph{(all that vibrates)}
    just as on Earth \emph{(that is material and dense)}.
    \vspace{0.5em} % to align
    Give us wisdom \emph{(understanding, assistance)}
    for our daily need.
    \vspace{0.5em} % to align
    Detach the fetters of faults that bind us \emph{(karma)},
    like we let go the guilt of others.
    \vspace{1em} % to align
    Let us not be lost in superficial things
    \emph{(materialism, common temptations)},
    \vspace{0.5em} % to align
    but let us be freed from that what keeps us off from
    our true purpose.
    \vspace{0.5em} % to align
    From You comes the all-working will, the lively strength to act,
    the song that beautifies all and renews itself from age to age.
    \vspace{1em} % to align
    Sealed in trust, faith and truth.
    \emph{(I confirm with my entire being.)}
  \end{translation}
  \begin{explanation}[EN]
    The symbol has been used by ancient Near Eastern scribes to indicate that
    the writing is of a sacred nature.
    \begin{description}
      \item[upper dot:] God (mind)
      \item[left dot:] Son (wisdom)
      \item[right dot:] Spirit (life)
      \item[bottom dot:] One Universal God
    \end{description}
  \end{explanation}
\endsong


\beginsong{\texorpdfstring{Wild Gazelle --- \textpersian{اهوی وحشی}}{Wild Gazelle}}
    [by={\texorpdfstring{Faramarz Aslani --- \textpersian{فرامرز اصلانی}, Hafez --- \textpersian{حافظ}}{Faramarz Aslani, Hafez}},
     ex={persian},
     key={Am},
     gk={Am, Gm--C\shrp{}m}]
  \audio[key=Em]{https://soundcloud.com/armin-poyamanesh/fymww4dufmfn}
  \meter{12}{8} %
  \capo{3}
  %TODO: needs refining!
  \beginverse
    \[\mn{B}]Äl|\[\mncii{D}{B}Em]{a e}|-\[\mn{A}]i a\[\mn{B}]huy|\[\mncii{C}{A}Am]e vähshi | - \[B7]{} \[\mn{G}]ko\[\mn{A}]ja|\[\mnc{B}Em]ii \hfill{\scriptsize\textpersian{الا ای آهوی وحشی کجائی}}
    | |\[Cmaj7]{} |\[B7]{} \e
    Mära |\[Em]ba tost | chä|\[Am]{-ndin} | - \[B7]{} ashena|\[Em]ii \hfill{\scriptsize\textpersian{مـرا با تـُست چندین آشـنائی}}
    | |\[Cmaj7]{} |\[B7]{} \e
  \endverse
  \beginverse
    \[\mn{A}]Do tän|\[Am]hao do särgar\[\mn{G}]dan \[\mn{A}]do |\[\mnc{B}Em]bi\[\mn{G}]käs \hfill{\scriptsize\textpersian{دو تنها و دو سرگردان دو بیکس}}
    Dädo |\[C]damät kämin az pisho a|\[Am]{z päs} \hfill{\scriptsize\textpersian{دد و دامت کمین از پیش و از پس}}
    Bi|\[Am]a ta hale yek digär |\[Em]bedanim \hfill{\scriptsize\textpersian{بـیا تا حـال یکدیگر بدانیم}}
    Mor|\[C]ade häm bejuiim ar tä|\[Am]vanim \hfill{\scriptsize\textpersian{مـراد هـم بجوییم اَر توانیم}}
  \endverse
  \beginverse
    \[\mn{F#}]Ke mi|\[B7]binäm ke in däshte mosh|\[\mnc{A}Em]ä\[\mn{G}]väsh \hfill{\scriptsize\textpersian{که می‌بینم که این دشت مشوش}}
    chär|\[Am]agahi näda\[B7]räd khorrä|\[Em]mo khäsh |\[B7]{} |\[Em]{} \e \hfill{\scriptsize\textpersian{چراگاهی ندارد خرم و خـَوش}}
  \endverse
  \beginverse
    \ind \[\mn{B}]Ke kha|\[Em]häd shod beguiid \[\mn{A}]ey \[\mn{B}]räf|\[\mncii{C}{A}Am]ighan \hfill{\scriptsize\textpersian{که خواهد شد بگویید ای رفیقان}}
    \ind Räfig|\[C]he bikäsan yare ghär|\[Am]iban \hfill{\scriptsize\textpersian{رفیق بیکسان یار غریبان}}
  \endverse
  \beginverse
    \ind \[\mn{A}]Mägär |\[B7]Khezre mobaräk pei |\[\mnc{B}Em]dä\[\mn{A}]ra\[\mn{G}]yd \hfill{\scriptsize\textpersian{مگر خضر مبارک پی درآید}}
    \ind ze io|\[C]mne hemmätäsh kari gosha|\[Am]{-yäd}, \[B7]{} \hfill{\scriptsize\textpersian{ز یـُمن همتش کاری گشاید}}
    \ind gosha|\[Em]{-yäd}
  \endverse
  \beginverse
    |\[Em]{} | | |\[Am]{} | |\[Em]{} |\[B7]{} \e
    |\[Cmaj7]{} |\[Am]{} |\[B7]{} |\[Em]{} | \e
  \endverse
  \beginverse
    Cho an |\[Em]särve | räva|\[Am]{-n} shod | - \[B7]{} karevan|\[Em]i \hfill{\scriptsize\textpersian{چوآن سرو روان شد کاروانی}}
    | |\[Cmaj7]{} |\[B7]{} \e
    zesh|\[Em]akhe särv | - m|\[Am]i kon | - \[B7]{} sayeb|\[Em]ani \hfill{\scriptsize\textpersian{ز شاخ سرو می‌کن سایه بانی}}
    | |\[Cmaj7]{} |\[B7]{} \e
  \endverse
  \beginverse
    Läbe s|\[Am]är cheshmeii o tärfe |\[Em]juii \hfill{\scriptsize\textpersian{لب سر چشمه‌ای و طرفِ جوئی}}
    näme |\[C]äshki o ba khod gofte|\[Am]guii \hfill{\scriptsize\textpersian{نم اشکی و با خود گفت و گوئی}}
    Be ya|\[Am]{-de} räftegan o dust|\[Em]daran \hfill{\scriptsize\textpersian{به یاد رفتگان و دوستداران}}
    mov|\[C]afegh gärd ba äbre bä|\[Am]haran \hfill{\scriptsize\textpersian{موافق گرد با ابر بهاران}}
  \endverse
  \beginverse
    Chon|\[B7]alan ayädät abe rä|\[Em]van pish \hfill{\scriptsize\textpersian{چو نالان آیدت آب روان پیش}}
    mäd|\[Am]äd bäkhshäsh ze \[B7]abe dide|\[Em]ye khish \hfill{\scriptsize\textpersian{مدد بخشش ز آب دیدۀ خویش}}
    |\[B7]{} |\[Em]{} \e
  \endverse
  \beginverse
    \ind \[\mn{B}]Näkärd |\[Em]an hamdame dir\[\mn{A}]in \[\mn{B}]mo|\[\mnc{C}Am]da\[\mn{A}]ra \hfill{\scriptsize\textpersian{نکرد آن همدم دیرین مدارا}}
    \ind mosä|lmanan, mosä\[B7]lmanan khod|\[Em]ara \hfill{\scriptsize\textpersian{مسلمانان مسلمانان خدا را}}
  \endverse
  \beginverse
    \ind \[\mn{G}]Mägä|\[C]r Khezre mobaräk pey \[\mn{A}]tä|\[\mnc{B}Am]va\[\mn{A}]näd \[B7]{} \hfill{\scriptsize\textpersian{مگر خضر مبارک ‌پی تواند}}
    \ind Ke in |\[Am]tänha be an \[B7]tänha res|\[Em]anäd \hfill{\scriptsize\textpersian{که این تنها به آن تنها رساند}}
  \endverse
  \beginchorus
    |\[Cmaj7]{} |\[Am]{} \[B7]{} |\[Am]{} \[B7]{} |\[Em]
  \endchorus
  \begin{translation}[EN]
    O wild gazelle, where can I find you?
    We have lots in common, me and you!
    \nextverse
    We are both alone, wanderers, we both have no one
    In behind and before, beasts and traps ambush [begun]
    \nextverse
    Let's ask each other, ``how do you feel?''
    Grant each other's wishes, if we can deal
    \nextverse
    Alas! As far as I can see, this troubled meadow
    Has no place to graze happily, free from sorrow
    \nextverse
    O friends! Let me know who [on earth] will be the one
    to help outcasts, to company the ones having no one
    \nextverse
    May green feet \emph{Khizr} come [strolling]
    And his mighty will get things rolling
    \nextverse
    As that fleeting tall as cedar [beauty] did get going
    Let's shelter [from longing] by the cedar tree's twig
    \nextverse
    Let's sit by a spring fountain, next to a brook
    Let's have moist in eyes, to myself let me talk
    \nextverse
    Let's recall memories of the bygone [people], of the friendlier
    And the spring cloud will accompany us in tear
    \nextverse
    As the running wailing water gets close by
    Give it help by the [running] water of your eye
    \nextverse
    She didn't get along with me, that old fond friend!
    [Help,] O people of faith! Help! For the love of God!
    \nextverse
    No one but the green feet Khizr can be the one
    To unite this lonely one with the other lonely one
  \end{translation}
  \begin{explanation}[EN]
    Lyrics for this song are based on a poem by the famous poet
    \textpersian{حافظ} (Hafez).\par\vspace{2ex}
    \textbf{Khizr} is a figure described, but not mentioned by name, in the
    Quran as a righteous servant of God possessing great wisdom or mystic
    knowledge. The miracle of Khizr was, that if he did sit on a dry
    land, green vegetation would appear on the ground beneath him and that
    land would become green. The name of the prophet means ``green'' in
    Arabic.
    \vfill
    \begin{center}
      \tiny Original English translation by BlueBird under Creative Commons
      License (\href{https://creativecommons.org/licenses/by/4.0/}{CC BY 4.0}).
    \end{center}
  \end{explanation}
\endsong


\beginsong{Beautiful Names of God}[tags={source},ex={arabic},ph={I}]
  \meter{3}{4}
  \beginverse
    \[\bmc \mnc{B}]Bis\[\mn{A}]mil|\[\bmc Am]lah, \[\bm]{} \[\bm\mn{C}]Al|\[\bmc C]lāh, \[\bm]{} \[\bm]Raḥ|\[\bmc\mnc{B}G]mān, \[\bm]{} \[\bm\mn{G}\mn{A}]Ra|\[\bmc Am]ḥīm \[\bm]
    \[\bm]Mā|\[\bm]lik, \[\bm]{} \[\bm]Qud|\[\bmc Dm]dūs, \[\bm]{} \[\bm]Sa|\[\bmc C]laām, \[\bm]Mu’\[\bm]min, |\[\bmc E]Muhay\[\bm]min \[\bm]{} | \[\bm]{} \[\bm]{} \e
    \[\bm]A|\[\bmc Am]zīz, \[\bm]{} \[\bm]Jab|\[\bmc E]bār, \[\bm]{} \[\bm]Muta |\[\bmc C]kab\[\bmc Dm]bir, \[\bm]Khā|\[\bmc E]liq \[\bm]{} \[\bm]{} | \[\bm]{} \[\bm]{} \e
%    % Original(?), rhythmically stranger version below:
%    \[\bmc \mnc{B}]Bis\[\mn{A}]mil|\\bmc [Am]lah, \[\bm]{} \[\bm]Al|\[\bmc C]lāh, \[\bm]{} \[\bm]Raḥ|\[\bmc G]mān, \[\bm]{} \[\bm]Ra|\[\bmc Am]ḥīm \[\bm]
%    \[\bm]Mā|\[\bm]lik, \[\bm]{} \[\bm]Qud|\[\bmc Dm]dūs, \[\bm]{} \[\bm]Sa|\[\bmc C]laām, \[\bm]Mu’\[\bm]min, |\[\bmc E]Muhay\[\bm]min \[\bm]
%    |\[\bm]{} \[\bm]A\[\bmc Am]zīz, |\[\bm]{} \[\bm]Jab\[\bmc E]bār, |\[\bm]{} \[\bm]Muta \[\bmc C]kab|\[\bmc Dm]bir, \[\bm]Khā\[\bmc E]liq | \[\bm]{} \[\bm]{} \e
  \endverse
  \begin{translation}[EN]
    \emph{In Qur'an:} Begin in the name of God, the One, Compassion, Mercy;
    Sovereign, Holy, Peace, Guarantor, Guardian; 
    Allmighty, Powerful, Tremendous, Creator
  \end{translation}
\endsong


\beginsong{Ishq Allāh\\Love, Lover and Beloved}[by={James Burgess},tags={source, love},ex={arabic, english},ph={IV}]
  \beginchorus
    \ind |\[\mnc{B}Bm]Ishq \[\mn{F#}]Allāh ma'|\[\mn{B}]būd \[\mn{F#}]Allāh
    \ind Ishq Al|lāh ma'\[A]būd Al|\[Bm]lāh
  \endchorus
  \beginverse
    |\[A]God is Love, |\[Bm]Lover and Beloved
    |\[A] Love, Lover and Be|\[Bm]loved
    |\[A]I am Love, |\[Bm]Lover and Beloved
    |\[A] Love, Lover and Be|\[Bm]loved
  \endverse
  \begin{explanation}[EN]
    \begin{description}
      \item[Ishq Allāh ma'būd Allāh] translates literally to ``love God adored God''
        which can be interpreted as ``God is Love and God is the Beloved'' --- and more poetically
        as ``God is Love, Lover and Beloved''.
    \end{description}
  \end{explanation}
\endsong


\beginsong{Mash Allāh}[tags={you, source},ex={arabic, english},ph={IV}]
  \beginchorus
    \[\mn{E}]Through \[\mn{F#}]your |\[\mnc{G}Em]eyes shines the light
    Mash Al|lāh mash Allāh
    |\[D]Wonder of \[B7]God in |\[Em]You
  \endchorus
  \beginverse
    |\[G]Mash Al|\[Am]lāh mash Allāh
    |\[D7]Mash Al|\[Em]lāh mash Allāh
    |\[G]Mash Al|\[Am]lāh mash Allāh
    |\[B7]Wonder of God in |\[Em]You
    |\[B7]Wonder of God in |\[Em]You
  \endverse
  \begin{explanation}[EN]
    \textbf{Mash Allāh} is Arabic and means ``as God willed it''. It is used to express thankfulness,
    appreciation or joy for what was just mentioned.
  \end{explanation}
\endsong


\beginsong{Shalom Aleichem}[ph={I, II}]
  \audio[key={Dm}]{https://www.youtube.com/watch?v=913jZFL1bdE}
  \beginverse
    |\[\mnc{A}Dm]Sha\[\mn{F}]lom \[\mn{E}]a\[\mn{D}]leichem |\[\mnc{C#}A7]mal'\[\mn{D}]a\[\mn{E}\mn{D}]chei \[\mn{C#}]has\[\mn{B&}]ha\[\mn{A}]lom
    |\[B&]Mal'achei e|\[A7]lyon
    |\[Dm]Mimelech |\[A7]malchei hamelachim
    Ha-|\[Gm]kadosh baruch |\[A7]hu
  \endverse
  \beginverse
    |\[\mnc{F}F]{\up{*}(Bo'a}\[\mn{A}]chem) l'shalom |\[\mnc{G}C]mal'\[\mn{F}]a\[\mn{E}]chei \[\mn{F}]has\[\mn{G}\mn{A}]ha\[\mn{G}]lom
    |\[B&]Mal'achei e|\[A7]lyon
    |\[Gm]Mimelech |\[A7]malchei ha\[Dm]melachim
    Ha-|\[B&]kadosh \[A7]baruch |\[Dm]hu
  \endverse
  \altlyr{Barchuni, Tseitchem}
  \begin{translation}[EN]
    Peace be unto you, ye ministeri​​​​ng Angels, Angels of the
    most High, ye that come from the Supreme King of Kings,
    the Holy One, blessed be He.
    \nextverse
    May your coming be in peace, ye ministeri​​​​ng Angels\ldots
    \nextverse
    Bless​ me with peace, ye ministeri​​​​ng Angels\ldots
    \nextverse
    Go ye forth in peace, ye ministeri​​​​ng Angels\ldots
  \end{translation}
\endsong


\beginsong{Lecha Eli}[by={Rabbi Avraham Iebn Ezra, Yair Gadassi},ex={hebrew},tags={source},ph={II}]
  \audio[]{https://soundcloud.com/shimonlevtahor/vimala-zohar-gad-lecha-eli}
  \beginverse
    |\[Am]{} \[^\mn{A}]Lecha \[^\mn{E}]E|li | \[^\mn{A}]teshu\[^\mn{E}]ka|\[G\mn{D}\mn{C}\mn{D}]ti
    | Becha chesh|\[Dm]ki |\[Em]{} ve'ahava|\[Am]ti | \e
    |\[Am]{} Lecha li|bi | vechilyo|\[G]tai
    | Lecha ru|\[Dm]chi |\[Em]{} venishma|\[Am]ti | \e
  \endverse
  \beginchorus
    \ind |\[Dm]{} Hashive|ni va'ashu|\[G]va \up{2}(| | \e)
    \ind | Vetirtzeh |\[Dm]{} |\[Em]et teshuva|\[Am]ti | \e
  \endchorus
  \notesoff
  \beginverse
    |^ Lecha ya|dai | lecha rag|^lai
    | Umimach |^hee |^ techuna|^ti | \e
    |^ Lecha atz|mi | lecha da|^mi
    | Ve'ori |^im |^ geviya|^ti | \e  \goto{Hashiveni}
  \endverse
  \beginchorus
    |\[Am]Oh |ho oh ho ho ho |\[G]ho | \e
    |\[Dm]Oh |\[Em]ho oh ho ho ho |\[Am]ho | \e
  \endchorus
  \beginverse
    |^ Lecha ez'|ak | becha ed|^bak
    | Adei shu|^vi |^ le'adma|^ti | \e
    |^ Lecha a|ni | be'odi |^chai
    | Ve'af ki |^a- |^ charei mo|^ti | \e  \goto{Hashiveni}
  \endverse
  \begin{translation}[EN]
    For You my God is my passion, in You is my desire and my love
    Yours are my heart and my organs, tours are my spirit and my soul
    \nextverse
    \ind Bring me back to You and I will return
    \ind And You shall want my repentance
    \nextverse
    Yours are my hands and legs, and from You is my character
    Yours are my bones and my blood, and my skin and my body
    \nextverse
    Oh ho oh ho ho ho ho
    \nextverse
    To You I will call and to You I will cling, until I return to my land
    I give myself to You whilst I still live, and even after I die
  \end{translation}
\endsong


\beginsong{Lev Tahor}[ph={II, III}, key={Am}, gk={Bm, Am--Em}]
  \audio[key=Cm]{https://soundcloud.com/bettinamaureenji/lev-tahor}
  % in Am the notes range from G to C' (or E' if going the high route :))
  \beginverse
    \[\mn{A}]Lev \[\mn{B}]ta|\[\mnc{C}C]ho|\[\mnc{B}G]r b'ra \[\mn{G}]li \[\mn{B}]E\[\mn{C}]lo|\[\mnc{A}Am]him; | \e \altchords{\id[1]{(Bm)}|D |A |Bm | \e}
    v'ruach na|\[C]cho|\[G]n chadesh b'ki|\[Am]rbi. | \e \altchords{|D |A |Bm | \e}
    \vspace{.5em}
    Al tashli|\[F]cheni|\[G]{} mil'fa|\[Am]necha; | \e \altchords{|G |A |Bm | \e}
    v'ruach kodshe|\[C]cha|\[G]{} al tikach mi|\[Am]meni. | \e \altchords{|D |A |Bm | \e}
  \endverse
  \imagerc[3]{lev_tahor_hebrew_script_bw_transparent_bg_1477x300px.png}%
  \begin{translation}[EN]
    Create me a clean heart, O God;
    and renew a stedfast spirit within me.
    \nextverse
    Cast me not away from Thy presence;​
    and take not Thy holy spirit from me.
  \end{translation}
\endsong


\beginsong{Asse Wana Hey Wana \\ Hey Niketi}[ex={hopílavayi, english},tags={heart, circle},ph={III, IV}]
  \audio[]{https://www.youtube.com/watch?v=WsEoKmEGe18}
  \beginchorus
    |\[\mnc{G}Em]Asse \[\mn{F#}]wa\[\mn{E}]na |\[\mnc{A}Am]hey wana |\[\mnc{F#}D]asse \[\mn{E}]wa\[\mn{D}]na |\[\mnc{E}Em]hey wana
  \endchorus
  \beginchorus
    \ind |\[\mnc{B}Em]Hey ni\[\mn{A}]ke\[\mn{G}]ti |\[\mnc{A}D]hey wana |\[Bm]hey ni\[\mn{G}]ke\[\mn{F#}]ti |\[\mnc{G}Em]hey \[\mn{E}]wana
  \endchorus
  \notesoff
  \beginchorus
    |\[Em]Hey sister |\[Am]we are one, |\[D]hey brother |\[Em]we are one
  \endchorus
  \beginchorus
    \ind |\[Em]No matter |\[D]where we're going to |\[Bm]no matter |\[Em]where we're coming from
  \endchorus
  \begin{explanation}[EN]
    \begin{description}
     \item[Wana] is a Hopi word for ``heart''. We are all connected in our hearts.
    \end{description}
  \end{explanation}
\endsong


\beginsong{Weha Ehayo}[by={Lakota},ex={lakȟótiyapi, español, english},ph={III}]
  \beginverse % 25 beats in this verse
    \ind \[\bmc\mnc{D}D]Weha eh\[\bmadj{-.5ex}]ay\[\bmadj{-.5ex}]o \[\bmc\mnc{C#}A]weha eh\[\bmadj{-.5ex}]ay\[\bmadj{-.5ex}]o
    \ind W\[\bmadj{-.5ex}]eha e\[\bmc C]hay\[\bmadj{-.5ex}]o \[\bmc G]weha eh\[\bmadj{-.5ex}]ay\[\bmadj{-.5ex}]o
    \ind W\[\bmadj{-.5ex}]eha e\[\bmc C]hay\[\bmadj{-.5ex}]o \[\bmc G]weha \[\bmadj{-.5ex}]eha\[\bmc D]yo! \[\bm]{} \[\bm]{} \[\bm]{} \[\bm]{} \[\bm]{} \[\bm]{} \[\bm]
  \endverse
  \beginverse\memorize % 36 beats in this verse
    \[\bmc D]Gran Esp\[\bmadj{-.5ex}]írit\[\bmadj{-.5ex}]u \[\bmc A]yo voy \[\bm]a ped\[\bmadj{-.5ex}]ir, ó\[\bmadj{-.5ex}]yem\[\bmadj{-.5ex}]e \[\bm]{} \[\bm]
    A\[\bmadj{-.5ex}]l uni\[\bmc C]vers\[\bmadj{-.5ex}]o \[\bmc G]yo voy \[\bm]a ped\[\bmadj{-.5ex}]ir, ó\[\bmadj{-.5ex}]yem\[\bmadj{-.5ex}]e \[\bm]{} \[\bm]
    Par\[\bmadj{-.5ex}]a mi \[\bmc C]puebl\[\bmadj{-.5ex}]o \[\bmc G]que sobrev\[\bmadj{-.5ex}]iv\[\bmadj{-.5ex}]a
    y\[\bmadj{-.5ex}]o he d\[\bmadj{-.5ex}]icho \[\bmc D]hey! \[\bm]{} \[\bm]{} \[\bm]{} \[\bm]{} \[\bm]{} \[\bm]{} \[\bm]{} \goto{Weha ehayo}
  \endverse
  \beginverse
    ^Pacham^am^a ^yo voy ^a ped^ir, ó^yem^e ^ ^
    ^a Wira^coch^a ^yo voy ^a ped^ir, ó^yem^e ^ ^
    par^a mi ^puebl^o ^que siempre v^iv^a
    y^o he d^icho ^hey! \[\bm]{} \[\bm]{} \[\bm]{} \[\bm]{} \[\bm]{} \[\bm]{} \[\bm]{} \goto{Weha ehayo}
  \endverse
  \beginverse
    ^Great Sp^ir^it ^I am ^going to p^lead, h^ear my c^all ^ ^
    T^o the ^univ^erse ^I am ^going to p^lead, h^ear my c^all ^ ^
    F^or the sur^viv^al ^of our p^eop^le
    ^I am s^aying ^hey! \[\bm]{} \[\bm]{} \[\bm]{} \[\bm]{} \[\bm]{} \[\bm]{} \[\bm]{} \goto{Weha ehayo}
  \endverse
  \begin{explanation}[EN]
    \begin{description}
     \item[Wiracocha] is the great creator deity in the pre-Inca and Inca mythology in the Andes.
    \end{description}
  \end{explanation}
\endsong


\beginsong{He yama yo}[by={Lakota, version by Curawaka}, ex={lakȟótiyapi}, tags={thankfulness}, ph={III, IV, V}, key={Dm}, gk={Cm, Cm--Em}]
  \audio[key=Dm]{https://www.youtube.com/watch?v=SofBQbI38PQ}
  % In Dm the notes go from A to A'
  \beginchorus
    \[\mn{A}]He |\[\mnc{D}Dm]yama yo, wa\[\mn{E}]na |\[\mn{F}]he\[\mn{E}]ne \[\mn{D}]yo \altchords{\id[1]{(Am)}|Am | \e}
    He |\[A7]yama yo, | wana hene y|\[Dm]o | \e \altchords{|E7 | - |Am | \e}
  \endchorus
  \beginverse
    \[\mn{A}]Wa|\[\mnc{A}Dm]hee | \[\mn{G}]ya \[\mn{F}]ya |\[\mnc{G}C]na, | he \[\mn{F}]he \[\mn{E}]he |\[\mnc{F}Dm]ho | \e \altchords{|Am | - |G | - |Am | \e}
    Wa|\[A7]hee | he |\[B&]he he he he h|o, wa|\[Dm]hee | \e \altchords{|E7 | - |F | - |Am | \e}
  \endverse
\endsong


\beginsong{Mahk Jchi}[by={Pura Fé}, ph={IV, V}, ex={tutelo-saponi}, tags={canon}, key={Am}, gk={Am, Am--Bm}]
  \audio[]{http://thebirdsings.com/mahkjchi/}
  \audio[]{https://www.youtube.com/v=bOn4vIybDU8}
  \meter{3}{4}
  \transpose {-5} % Transpose to Am, where notes go from G to A'
  \beginchorus
    \[\mn{D}]Mahk |\[\mnc{D}Dm]Jchi tahm |boo-\[\mn{A}]ee
    yahm |\[G]pi-gih-dee | \e
    Mahk |\[F]Jchi tahm |\[C]boo-ee
    kahn |\[Dm]speh-wah eh-|bi
  \endchorus
  \beginverse
    Mahm-pi |\[F]wah |ho-ka yi|\[G]i nonk | \e
    tah |\[F]hond tah-|\[C]ni kih-|\[G]yee tai-yee | \e
    Ghee weh |\[F]meh |yee-tai-|\[G]yee | \e
  \endverse\glueverses\beginchorus
    Nan-ka |\[F]yaht yah |\[C]moo-ni-yeh |\[Dm]wah-jhi-seh | \e \up{2}(| | \e)
  \endchorus
  \begin{translation}[EN]
    Our hearts are full and our minds are good.
    Our ancestors come and give us strength.
    Stand tall, sing, dance and never forget who you are
    Or where you come from.
  \end{translation}
\endsong


\beginsong{Marirí}[ex={quechua},tags={protection},ph={I, II}]
  \beginchorus\meter{5}{4}
    \up{*}\[\bmc\mnc{E}]Lu\[\bm]pu|\[\bmc\mnc{G}C]nita \[\bm]supay \[\bm]callam\[\bm]puntay \[\bm]man|\[\bmc\mnc{A}Fmaj7]tay \[\bmc\mnc{G}]mar\[\bmc\mnc{E}C]í
  \endchorus\glueverses
  \beginchorus
    \up{*}\[\bm]Lu\[\bm]pu|\[\bmc Am]nita \[\bm]supay \[\bm]callam\[\bm]puntay \[\bm]man|\[\bmc Dm]tay \[\bm]mar\[\bmc Am]í
  \endchorus
  \beginchorus\meter{4}{4}
    \[\bm]{} \[\bmc\mnc{E}]Mari|\[\bmc C]rí \[\bm]mari\[\bmc Fmaj7]rí \[\bm]mari|\[\bmc Am]rí \[\bm]\rep{3}
  \endchorus\glueverses
  \beginverse
    \[\bm]{} \[\bm]{} |\[\bmc C]{} \[\bm]{} \[\bmc Fmaj7]{} \[\bm]{} |\[\bmc Am]{} \[\bm]{} \e
  \endverse\chordsoff
  \altlyr{Irapay, Ashpasuri, Bobinsana, Chiringa, Ayahuasca, Chacrunera,
    Chaliponga, Huanilla, Catahua, Chihuahuaco, Tomapende, Aguaje, Palmicha, Wicungo, Remocaspi,
    Huachumita, Iboga, Peyotito, Hongocitos, Sapotito, Boa Boa, Otorongo, Urcututo, Yanguntoro,
    Aguilita, Condorcito, Lucero, Quillaruna, Chulla Chaqui, Wiracocha, Pachamama, Taita Inti\ldots
  }
  \begin{translation}[EN]
    \up{*}\textit{Lupunita}, with the tip of my tongue I call on your power.
  \end{translation}
  \begin{explanation}[EN]
    \begin{description}
      \item[Marirí] is a powerful protecting and healing spirit that lives in \textit{yachay},
        the phlegm that contains the essence of a curandera's power. The spirit can be passed
        to another by a curandero, who regurgitates it, or by another nature spirit. It is
        nurtured by tobacco smoke. Marirí is very important in the practices of curanderos of
        the Upper Amazon.
      \item[Lupunita:] wolf
      \item[Irapay:] plant spirits
      \item[Ashpasuri:] animal spirits
      \item[Bobinsana \ldots  Sapotito:] plants
      \item[Boa Boa \ldots  Condorcito:] animals
      \item[Lucero:] stars
      \item[Quillaruna:] Moon
      \item[Chulla Chaqui \ldots Taita Inti:] deities
    \end{description}
  \end{explanation}
\endsong


\beginsong{Ayahuasca Takimuyki}[by={Don José Campos},ex={quechua, español},tags={Aya},ex={quechua, español},ph={II},key={Bm},gk={Cm, (Bm)--Cm--C\shrp{}m}]
  \ifchorded\else% Do not show measure bars (with meters) in lyrics-only book
    \measuresoff%
  \fi
  % in Am the notes range from E to G'
  % in Bm the notes range from F# to A'
  \transpose{2}
  \beginchorus\memorize
    \meter{3}{4}|\[\bmc\mncii{E}{C}C]Ayaw\[\bm]aska \[\bm]urqu\meter{2}{4}|\[\bmc\mncii{D}{C}F]man\[\bm]ta \meter{3}{4}|\[\bmc\mnc{A}Am]taki \[\bm]taki\[\bmc\mnc{C}]muy\meter{1}{4}|\[\bmc\mnc{E}Em]ki \altchords{\id[1]{(Am)}|C . . |F . |Am . . |Em}
  \endchorus
  \mnbeginchorus\prep{4}
    \ind \meter{3}{4}|\[\bmc\mnc{E}Em]Chuyay \[\bm]chuyay \[\bmc\mn{G}]ham\[\mn{E}]pi\meter{2}{4}|\[\bmc\mncii{D}{C}Am]kuy\[\bm]niy\altchords{|Em . . |Am . }
    \ind \meter{3}{4}|\[\bmc\mnc{A}F]Mişki \[\bm]ñuñu \[\bm]kur\[\mn{G}]ku\meter{2}{4}|\[\bmc\mnc{A}Am]chay\[\bm]paq \up{4}(\meter{3}{4}|\[\bm]{} \[\bm]{} \[\bm]{} ) \altchords{|F . . |Am . \up{4}(| . . .)}
  \mnendchorus
  \notesoff
  \beginchorus
    \meter{3}{4}|^Ayah^uasca ^curan\meter{2}{4}|^de^ra \meter{3}{4}|^taki ^taki^muy\meter{1}{4}|^ki
  \endchorus
  \goto{Chuyay chuyay}
  \beginchorus
    \meter{3}{4}|^Ayah^uasca ^luce\meter{3}{4}|^rito ^man\[\bm]ta \meter{3}{4}|^taki ^taki^muy\meter{1}{4}|^ki
  \endchorus
  \goto{Chuyay chuyay}
  \beginchorus
    \meter{3}{4}|^Ayah^uasca ^chacru\meter{2}{4}|^ne^ra \meter{3}{4}|^taki ^taki^muy\meter{1}{4}|^ki
  \endchorus
  \goto{Chuyay chuyay}
  \beginchorus
    \meter{3}{4}|^Ayah^uasca ^pintu\meter{2}{4}|^re^ra \meter{3}{4}|^taki ^taki^muy\meter{1}{4}|^ki
  \endchorus
  \goto{Chuyay chuyay}
  \begin{translation}[EN]
    Ayahuasca of the mountain \emph{\small(healer, from stars, and Chacruna, vision painter)}
    Singing, singing, I come to you
    \nextverse
    Cleanse, cleanse, our medicine
    Sweet milk for my little body
  \end{translation}
  \vfill%
  \musicnotefornext{alternate, simplified, rhythm: make any/every measure 3/4}
  \noendsongvfill
  % NOTE: Our version has the first verse in quechua, and the rest in quechua/spanish.
  % Adapted lyrics in 100% quechua, without spanishisms:
  %   Ayawaska urqumanta takitakimuyki,
  %   chuyay chuyay hanpikuyniy
  %   mişki ñuñu kurkuchaypaq.
  %   Ayawaska hanpiqniyku...
  %   Ayawaska chaska quyllurmanta...
  %   Ayawaska chaqrunallay...
  %   Ayawaska chaqru panqa...
  %   Ayawaska llimpiqchallay...
  % Adapted lyrics in 100% spanish:
  %   Ayahuasca de la montaña, te vengo cantando y cantando,
  %   limpia limpia medicina nuestra
  %   dulce leche para mi cuerpecito.
  %   Ayahuasca curandera...
  %   Ayahuasca de lucerito...
  %   Ayahuasca chacrunera...
  %   Ayahuasca hoja de chacru...
  %   Ayahuasca pinturera...
\endsong


\beginsong{Paparuy}[by={Don Aquilino Chujandama},ph={II},key={Em (2 oct)},gk={Bm, Am--C\shrp{}m}]
  \audio[key=Dm]{https://soundcloud.com/ayahuapu/paparuy-en-vivo}
  \audio[key=Em]{https://www.youtube.com/watch?v=iZljniFESak}
  % in Dm the notes range from C to C'
  % in Em the notes range from D to D'
  % in Am the notes range from G to G'
  % in Bm the notes range from A to A'
  \transpose{2} % to Em for singing in two octaves
  \mnbeginverse
    \[^\mn{D}]Papa|\[\mnc{A}Dm]ruy pa\[^\mn{G}\mn{A}]pa|ruy pa\[^\mn{C}]pa|\[\mnc{G}Gm]ruy pa\[^\mn{F}\mn{G}]pa|ruy \altchords{\id[1]{(Bm)}|Bm | - |Em | \e}
    \[^\mn{D}]Papa|\[\mnc{F}Dm]ruy \[^\mn{D}]pa\[^\mn{C}\mn{D}]pa|ruy papa|\[\mnc{E}C]ruy \[^\mn{F}]pa\[^\mn{E}\mn{D}]pa|\[Dm]ruy | \e \altchords{|Bm | - |A |Bm | \e}
  \mnendverse
  \notesoff
  \beginverse
    Wayru|^ruy wayru|ruy wayru|^ruy wayru|ruy
    Wayru|^ruy wayru|ruy wayru|^ruy wayru|^ruy | \e
  \endverse
  \beginverse
    Oto|^rongo wa|waí oto|^rongo wa|waí
    Shinapuri |^kungi wa|waí iwa|^waí kawa|^waí | \e
  \endverse
  \beginverse
    Side|^rachi wa|waí Side|^rachi wa|waí
    Shinapuri |^kungi wa|waí iwa|^waí kawa|^waí | \e
  \endverse
  \beginverse
    Churi|^chiyu wa|waí churi|^chiyu wa|waí
    Shinapuri |^kungi wa|waí iwa|^waí kawa|^waí | \e
  \endverse
  \beginverse
    Tibi|^rungi wa|waí tibi|^rungi waw|aí
    Shinapuri |^kungi wa|waí iwa|^waí kawa|^waí | \e
  \endverse
  \begin{translation}[EN]
    Bird, bird\ldots
    \nextverse
    Tree, tree\ldots
    \nextverse
    Spirit of \emph{jaguar/deer/\ldots}, take me with you.
    Take me while you roam and make me part of your kingdom.
    % As the jaguar/deer/... raises it's little one; so that it walks; what are you looking?
  \end{translation}
  \begin{explanation}[EN]
    \textbf{Paparuy}: bird; \textbf{otorongo}: jaguar; \textbf{wayruruy}: tree; \textbf{churichiuy}: deer
  \end{explanation}
\endsong


\beginsong{Vem, Mãe Natureza}[by={trad., version by Curawaka}, ex={hancha kuin, português}, ph={II, III}, key={Am}, gk={Am, Gm--Bm}]
  \audio[key=Am]{https://soundcloud.com/curawaka_official/vem-mae-natureza}
  \audio[key=Am]{https://www.youtube.com/watch?v=pwAsfZ7feTU}
  \audio[key=Am]{https://soundcloud.com/yakuruna/mae-natureza}
  \mnbeginchorus\memorize
    |\[\mnc{A}Am]Vem, \[\bmc\mn{E}]vem, mãe \[^\mn{A}]natu|re\[\bm]za; |\[\mnc{G}G]vem, \[\bmc\mn{D}]vem, mãe \[^\mn{G}]natu|re\[\bm]za\[^\mn{F}]
    |\[\mnc{E}Am]Vem, \[\bm]vem, mãe natu|\[\mncadj{1ex}{D}E7]{re}za; \[\mnc{E}E]vem nos ensi|\[\mnc{A}Am]nar \[\bm] |{ } { } \[\bm] \e
  \mnendchorus
  \notesoff
  \beginchorus
    |^Vem, ^vem, tete pã|wã; ^ |^vem, ^vem, tete pã|wã ^
    |^Vem, ^vem, tete pã|^wã; ^vem a nos cu|^rar ^ |{ } { } ^ \e
  \endchorus
  \beginchorus
    |^Vem, ^vem, yube she|nu; ^ |^vem, ^vem, yube she|nu ^
    |^Vem, ^vem, yube she|^nu; yube ^shenu kene|^ya ^ |{ } { } ^ \e
  \endchorus
  \mnbeginchorus\noteson\memorize
    \ind \[^\mn{A}]Kene|\[Am]ya \[^\mn{E}]hãwã \[\mnc{C}\bm]beru \[^\mn{A}]kene|ya; \[\bm] \[^\mn{G}]kene|\[G]ya \[^\mn{D}]hãwã \[\mnc{B}\bm]beru \[^\mn{G}]kene|ya \[\bm]
    \ind \[^\mn{D}]Kene|\[\mnc{E}Am]ya hãwã \[\bm]beru kene|\[\mncadj{1ex}{D}E7]ya; Yube \[\mnc{E}E]beru kene|\[\mnc{A}Am]ya \[\bm] |{ } { }\[\bm] \e
  \mnendchorus
  \beginchorus
    \ind Maï|^{te hãwã} ^shaka maï|te; ^ maï|^te hãwã ^shaka maï|te ^
    \ind Maï|^{te hãwã} ^shaka maï|^te; yube ^shaka maï|^te ^ |{ } { } ^ \e
  \endchorus
  \showlilypondfalse % to save space
  \begin{lilywrap}\begin{lilypond}[]
    % transcribed by larva, latest update on 2024-01
    \include "tex/lp-include-head.ly"
    theMelody = \relative a'' {
      \key a \minor \slurDashed
      \set melismaBusyProperties = #'()
      \tempo 4 = 94
      \time 4/4 \partial 4
      \repeat volta 6 {
        r4 \mark \markup {"6x"} | a2 e8 e8 a8 a8 | a2( a2)
        | g2 d8 d8 g8 g8 | g2( g4)( f4)
        | e2 e8 e8 e8 e8 | \once\slurSolid e8( d8)( d8)(d8) e8 e8 e8 e8
        | a,1~ | a2.
      }
      \repeat volta 4 {
        a'8 \mark \markup {"4x"} a8 | a4 e8 e8 c8 c8 a'8 a8 | a2.
        g8 g8 | g4 d8 d8 b8 b8 g'8 g8 | g2.
        d8 d8 | e4 e8 e8 e8 e8 e8 e8 | \once\slurSolid e8( d8) d8 d8 e8 e8 e8 e8
        | a,1~ | a2.
      }
      \fine
    }
    theLyricsOne = \lyricmode {
      \set stanza = "1."
      \repeat volta 2 {
        | Vem, vem, mãe na -- tu -- | re -- za;
        | Vem, vem, mãe na -- tu -- | re -- za; _
        | Vem, vem, mãe na -- tu -- | re -- _ za; __ _
        Vem nos en -- si -- | nar. | _
      }
      \set stanza = "4."
      \repeat volta 2 {
        Ke -- ne -- | ya hã -- wã be -- ru ke -- ne -- | ya;
        Ke -- ne -- | ya hã -- wã be -- ru ke -- ne -- | ya;
        Ke -- ne -- | ya hã -- wã be -- ru ke -- ne -- | ya; _
        Yu -- be be -- ru ke -- ne -- | ya. | _
      }
    }
    theLyricsTwo = \lyricmode {
      \set stanza = "2."
      \repeat volta 2 {
        | Vem, vem, te -- te pã -- | wã; __ _
        | Vem, vem, te -- te pã -- | wã; __ _ _
        | Vem, vem, te -- te pã -- | wã; __ _ _ _
        Vem a nos cu -- | rar. | _
      }
      \set stanza = "5."
      \repeat volta 2 {
        Ma -- ï -- | te hã -- wã sha -- ka ma -- ï -- | te;
        Ma -- ï -- | te hã -- wã sha -- ka ma -- ï -- | te;
        Ma -- ï -- | te hã -- wã sha -- ka ma -- ï -- | te; _
        Yu -- be sha -- ka ma -- ï -- | te. | _
      }
    }
    theLyricsThree = \lyricmode {
      \set stanza = "3."
      \repeat volta 2 {
        | Vem, vem, yu -- be she -- | nu; __ _
        | Vem, vem, yu -- be she -- | nu; __ _ _
        | Vem, vem, yu -- be she -- | nu; _
        Yu -- be she -- nu ke -- ne -- | ya. | _
      }
    }
    theChords = \chordmode {
      \repeat volta 6 {
        s4 | a1:m | a:m
        | g | g
        | a:m | e2:7 e2
        | a1:m | a2.:m
      }
      \repeat volta 4 {
        s4 | a1:m | a:m
        | g | g
        | a:m | e2:7 e2
        | a1:m | a2.:m
      }
    }
%     \layout { #(layout-set-staff-size 15) } % for better fit
   \include "tex/lp-include-tail-notab.ly"
  \end{lilypond}\end{lilywrap}
  \begin{translation}[EN]
    Come, come, Mother Nature \rep{3}
    Come, teach us
    \nextverse
    Come, come, Plant Spirits \rep{3}
    Come to heal us
    \nextverse
    Come, come, Spirits of the Forest \rep{3}
    Spirits of the Forest, hear our plea
    \nextverse
    Our Earth, our mother, hear our plea \rep{3}
    Spirits of the Earth, hear our plea
    \nextverse
    Wind, our mother, hear our plea \rep{3}
    Spirits of the wind, hear our plea
  \end{translation}
\endsong


\beginsong{Eskawatã Kayawey}[by={pajé Agostinho Inkamuru}, ex={hancha kuin}, key={Am}, gk={Am, Am--C\shrp{}m}, ph={III, IV}]
  \newchords{chords_eskawata_a}\newchords{chords_eskawata_b}
  \transpose{7}
  \mnbeginchorus\memorize[chords_eskawata_a]
    \[\bmc\mnc{A}]Nukun |\[Dm]mã\[^\mn{C}]nã \[\bmc\mn{A}]Yuxi|bu bu \[^\mn{G}]be\[\mnc{F}F]tã \altchords{\id[1]{(Bm)}|Bm -  | - D}
    |Es\[^\mn{G}]ka\[\bmcadj{1.4ex}\mn{A}]watã \[^\mn{C}]ka\[^\mn{A}]ya|\[\mnc{G}Am7]wey \[\bmc\mn{E}]ki|\[\mnc{D}Dm]ki \[\bm] |{ }{ }\[\bm]{ } | \e \altchords{ | - |F\shrp{}m7 |Bm | - | \e}
    \mnendchorus\glueverses\mnbeginchorus\memorize[chords_eskawata_b]
    \[^\mn{D}]Eska\[\bm]watã ka|\[\mnc{C}F]ya, \[\bmc\mn{F}]ka\[^\mn{G}]ya|\[^\mn{A}]wey \[\bmc\mn{F}]ka\[^\mn{G}]ya|\[\mnc{A}Am]wey,\[\bm] | \e \altchords{ - |D | - |F\shrp{}m | \e}
    \[^\mn{C}]ka\[^\mn{A}]ya\[\mnc{G}Am7]wey \[^\mn{A}]ki|\[\mnc{D}Dm]ki \[\bm] |{ }{ }\[\bm]{ } | \e \altchords{ F\shrp{}m7 |Bm | - | \e}
  \mnendchorus
  \notesoff
  \beginchorus\replay[chords_eskawata_a]
    ^Nukun |^{\textbf{niwe}} ^Yuxi|bu bu be^tã
    |Eska^watã kaya|^wey ^ki|^ki ^ |{ }{ }^{ } | \e
    \endchorus\glueverses\beginchorus\replay[chords_eskawata_b]
    Eska^watã ka|^ya, ^kaya|wey ^kaya|^wey,^ | \e
    kaya^wey ki|^ki ^ |{ }{ }^{ } | \e
  \endchorus
  \vspace{1.5ex}
  \beginverse\chordsoff
      Nukun |\textbf{shina}\ldots
      Nukun |\textbf{kãna}\ldots
      Nukun |\textbf{bari}\ldots
      Nukun |\textbf{ushe}\ldots
      Nukun |\textbf{vixi}\ldots
      Nukun |\textbf{yamã}\ldots
      Nukun |\textbf{shava}\ldots
      Nukun |\textbf{yura}\ldots
      Nukun |\textbf{muká}\ldots
  \endverse
  % Image by: larva (based on a Huni Kuin symbol)
  % Image license: CC BY-NC-SA 4.0, with special permission in Unilaiva Songbook
  \hfill\makebox[0pt][r]{\raisebox{0pt}[0pt][0pt]{%
    \imager[4]{Huni_Kuin_symbol_624x624_flattened_by_larva.png}%
  }}%
  \begin{explanation}[EN]
    This song from the \textit{Huni Kuin (Kaxinawá)} tribe calls the spirits of nature, the elements, the sun, the moon, the stars, and \textit{Yuxibu} to bring transformation.
    \begin{description}
      \item[Eskawatã kayawey:] transformation
      \item[Yuxibu:] the creator (energy)
    \end{description}
  \end{explanation}
  % The version above is the Curawaka's version.
  \audio[]{https://www.youtube.com/watch?v=DU64jmOPL5k}
  % Lyrics for another, more original(?) version, go something like this:
  %   Nukun mana Yoxibu, yubã mana Yoxibu, mana Yoxibu bãta,
  %     eskawata kayawê Eskawata kayawá (2x)
  %   Nukun shina Yoxibu, yubã shina Yoxibu, shina Yoxibu bãta,
  %     eskawata kayawê Eskawata kayawá (2x)
  %   Nukun kãna Yoxibu, yubã kãna Yoxibu, kãna Yoxibu bãta,
  %     eskawata kayawê Eskawata kayawá (2x)
  %   Nukun bari Yoxibu, yubã bari Yoxibu, bari Yoxibu bãta,
  %     eskawata kayawê Eskawata kayawá
  %   Nukun ushe Yoxibu, yubã ushe Yoxibu, ushe Yoxibu bãta,
  %     eskawata kayawê Eskawata kayawá
  %   Nukun vixi Yoxibu, yubã vixi Yoxibu, vixi Yoxibu bãta,
  %     eskawata kayawê Eskawata kayawá
  %   Nukun yamã Yoxibu, yubã yamã Yoxibu, yamã Yoxibu bãta,
  %     eskawata kayawê Eskawata kayawá
  %   Nukun shava Yoxibu, yubã shava Yoxibu, shava Yoxibu bãta,
  %     eskawata kayawê Eskawata kayawá
  %   Nukun yura Yoxibu, yubã yura Yoxibu, yura Yoxibu bãta,
  %     eskawata kayawê Eskawata kayawá
  %   Nukun muká Yoxibu, yubã muká Yoxibu, muká Yoxibu bãta,
  %     eskawata kayawê Eskawata kayawá
  \audio[]{https://www.youtube.com/watch?v=1xRclkh6kUg}
\endsong

\sclearpage % to put on the same page with the next song (yawanawa, too)
\beginsong{Pahuene}[ex={yawanawa},ph={II, III}, key={C}, gk={C, B\flt{}--E}]
  \audio[]{https://www.youtube.com/watch?v=mjGxJkavh0k}
  \beginchorus
    |\[\mnc{E}C]Cura y no\[\mn{D}]ron|\[\mn{E}]dé |cura y no\[\mn{G}]ron|\[\mn{E}]dé
    |\[\mnc{D}Dm]Cura y \[\mn{E}]to\[\mn{D}]to|\[\mnc{C}C]to |\[\mnc{D}Dm]cura y \[\mn{E}]to\[\mn{D}]to|\[\mnc{C}C]to
  \endchorus
  \beginchorus
    |\[\mnc{C}C]Icama\[\mn{D}]hi |\[\mn{E}]pa\[\mn{D}]hue\[\mnc{C}Am]ne \[\mn{A}]i|\[\mnc{C}C]pa\[\mn{D}]hue\[\mn{C}]ne \up{1}(| \e)
  \endchorus\glueverses
  \beginchorus
    |\[\mnc{D}Dm]Pahue |\[\mnc{E}C]pa\[\mn{D}]hue\[\mnc{C}Am]ne \[\mn{A}]i|\[\mnc{C}C]pa\[\mn{D}]hue\[\mn{C}]ne \up{2}(| \e)
  \endchorus
\endsong


\beginsong{Te Nande}[ex={yawanawa},ph={III, IV}, key={Em}, gk={Em, Cm--G\shrp{}m}]
  \audio[]{https://www.youtube.com/watch?v=MhXOAJ1UGxw}
  \audio[]{https://www.youtube.com/watch?v=89hzAlPXJAQ}
  \mnbeginchorus
    \[\mn{B}]Ta|\[\mnc{E}Em]raoua\[\bm]ca um|bari \[\bm]ka\[\mn{D}]ran|\[\mn{B}]ê \[\bm] | \[\bm] \e
    Mo|\[\mn{E}]roua\[\bm]ne um|bari \[\bmc\mn{F#}]sa\[\mn{D}]ran|\[\mnc{C}C]ê \[\bm] | \[\bm] \e
    \[\mn{D}]Ka|\[D]ho an\[\bm]ze um|\[\mau{G}]ba\[\mauiic{F#}{A}]ri \[\bmc\mau{E}]te \[\mauiic{D}{B}]nan|\[\mauc{E}Em]de \[\bmc\mau{G}\mau{F#}] |\[\mau{E}]\[\bm]{} \up{2}(| \[\bm] |) \[\bm]\e
  \mnendchorus
  \mnbeginchorus
    \ind |\[\mnc{E}Em]Te \[\bm] nan|de te \[\bm]nan|\[\mnc{G}C]de \[\bm] | \[\bm] \e
    \ind |\[\mnc{F#}D]Te \[\bm] nan|de \[\mn{D}]te \[\bmc\mncadj{1.5ex}{E}]nan|\[Em]de \[\bmc\ma{G}\ma{F#}]{} |\[\ma{E}]\[\bm]{} \up{2}(| \[\bm] |) \e
  \mnendchorus
  \mnbeginchorus
    \[\bmc\mn{B}]Ka\[\mn{E}]pa |\[Em]xô mi\[\bm]na \[\mn{D}]ha|r\[\mn{B}]o \[\bm] | \e
    \[\bm]Ke\[\mn{E}]ho |an\[\mn{F#}]de \[\bm]te \[\mn{D}]ne|\[\mnc{E}C]dê \[\bm\mn{D}\mn{C}]{} |\[\mn{D}\mn{C}]{} \e
    |\[\mnc{F#}D]Yo \[\bm] rain|de \[\mn{D}]yo \[\bm]ra\[\mn{E}]in|\[Em]dê \[\bmc\mn{G}\mn{F#}]{} |\[\mn{E}]\[\bm]{} \up{2}(| \[\bm] |) \e
  \mnendchorus
  \goto{Te nande}
\endsong


\beginsong{Wacomaia}[ex={yawanawa},ph={IV}]
  \audio[]{https://www.youtube.com/watch?v=27OtT76yyAg}
  \audio[]{https://www.youtube.com/watch?v=7RzhsVTOcEk}
  % TODO: check different versions. Most (original?) seem to have more 'wacomaia' words
  %       in the first and last verses of the song, perhaps different chords, too...
  \beginchorus
    \[\mnc{G}G]Waco\meter{4}{4}|maia, Wa\[\mn{A}]co|\[\mnc{B}Am]mai\[\mn{A}]a, |\[\mn{G}]Wa\[\mn{A}]co\[\mn{B}]mai|\[\mn{A}]a,
    Waco|\[C]maia, |Wacomai\meter{2}{4}|\[Am]a, \meter{4}{4}|\[G]heé! | | \e
  \endchorus
  \beginchorus
    Wa|comai|\[Am]a, tone |pinda|\[C]ke pinda |kanarô |\[G]ho | | \e
  \endchorus
  \beginchorus
    To|zake hô pa|\[Am]rá tone |pinda|\[C]ke, pinda |kanarô |\[G]ho | | \e
  \endchorus
  \beginchorus
    Ahe, e|he, ya he, |\[Am]waya, |Wacomai|a,
    Waco|\[C]maia, |Wacomai\meter{2}{4}|\[Am]a, \meter{4}{4}|\[G]heé! | | \e
  \endchorus
  \begin{explanation}[EN]
    \begin{description}
      \item[Wacomaia] means ``joy, happiness''
    \end{description}
  \end{explanation}
\endsong


\begin{intersong}
  % Kênas are the Yawanawa peoples' sacred visions in intricate body and
  % textile paintings. This is only a reproduction by larva.
  \imagecc[0]{Yawanawa_Kena_17x_3079x461px.png}
\end{intersong}


\beginsong{Kanô Kanô}[ex={yawanawa},ph={IV}]
  \audio[]{https://www.youtube.com/watch?v=0Po\_Q1Ft9aY}
  \audio[]{https://www.youtube.com/watch?v=Vrj9wXKuyDk}
  \audio[]{https://soundcloud.com/iranviene-edderv/yawanawa-kano-kano}
  \beginchorus\memorize
   \[^G]{} \[^\mn{G}]Ka|\[^\mn{A}]nô \[^\mn{G}]Ka|\[\mnc{A}Am]nô, | | | |\[C]{} \[^\mn{G}]Ka|\[^\mn{A}]nô \[^\mn{G}]Ka|\[G]nô | | \e
  \endchorus
  \notesoff
  \beginchorus
    He ka|nore kano|^rê, He Ka|nore Kano|rê
    He ka|nore kano|^rê, He Ka|nore Kano|^rê | | \e
  \endchorus
  \beginverse
    Mara|cá inakai|^nã Mara|cá inakai|nã
    Mara|cá inakai|^nã Mara|cá inakai|^nã | | \e
  \endverse
  \beginverse
    Mara|cá ioio|^iô Mara|cá ioio|iô
    Mara|cá ioio|^iô Mara|cá ioio|^iô | | \e
  \endverse
  \beginverse
    Io|iô ioio |^iô, Io|iô ioio |iô
    Io|iô ioio |^iô, Io|iô ioio |^iô | | \e
  \endverse
  \beginverse
    Para|ra iranoi |^rã, Para|ra iranoi |rã
    Para|ra iranoi |^rã, Para|ra iranoi |^rã | | \e
  \endverse
  \beginverse
    Ra|no rano i|^rã, Ra|no rano i|rã
    Ra|no rano i|^rã, Ra|no rano i|^rã | | \e
  \endverse
  \beginverse
    Para|ra ioiô |^iô, Para|ra ioiô |iô
    Para|ra ioiô |^iô, Para|ra ioiô |^iô | | \e
  \endverse
\endsong


\beginsong{Emamaa \\ Moder Jord}[by={trad., Tane Mahuta},ex={eesti, suomi, svenska; based on a Swedish folk song},tags={Mother Earth},ph={III}]
  % TODO: check song origin, find out the full swedish lyrics,
  %       and perhaps better finnish ones as well; ting?
  \textnotefornext{eesti keeles:}
  \beginchorus\memorize
    \lrep \[^\mn{A}]Ema |\[\mnc{D}Dm]Maa, ema \[C]Maa, Maa|\[Dm]e\[Am]ma \rrep
    Su |\[F]sees elab \[C]ürgne |\[Dm]vägi mis |toob mul \[Am]lohu|\[Dm]tust
  \endchorus
  \notesoff
  \beginchorus
    |\[Dm]Kummar\[Am]dan su |\[Dm]ees kallis \[Am]ema
    Et |\[F]südames \[C]mind ikka |\[Dm]kannad
  \endchorus
  \textnotefornext{suomeksi:}
  \beginchorus
    \lrep Äiti |^Maa, äiti ^Maa, Maa|^äi^ti \rrep
    Sun |^sisälläs on ^väkevä |^voima, se |minua ^lohdut|^taa
  \endchorus
  \beginchorus
    |\[Dm]Kumar\[Am]ran edes|\[Dm]säsi rakas \[Am]äiti
    Sua |\[F]sydämes\[C]säin aina |\[Dm]kannan
  \endchorus
  \textnotefornext{på svenska:}
  \beginchorus
    \lrep Moder |^Jord, moder ^Jord, Jord|^mo^der \rrep
    Du |^glöder så ^varmt i ditt |^inre, din |hetta ^ger mig |^lust
  \endchorus
\endsong


\beginsong{O la Mama \\ Ancient Mother}[tags={Divine Mother, Mother Earth},by={trad. African}, ex={some african language, english}, key={Bm}, gk={Am--Em}, ph={I, II}]
  \transpose{7} % from Em to Bm, where the notes go from A to F#
  \mnbeginchorus\memorize
    \[^\mn{E}]O \[^\mn{B}]la |\[\mnc{A}Am]Mama, |\[D]{} wa ha \[^\mn{B}]su |\[\mncii{G}{F#}Em]ko\[^\mn{E}]la |\[Bm11/D] \altchords{\id[1]{(Em)}|Am |D |Em |Bm11/D}
    O \[^\mn{F#}]la |\[\mnc{G}C]Mama, |\[D]{} \[^\mn{A}]wa \[^\mn{F#}]ha \[^\mn{D}\mn{E}]su |\[Em]wam | \e \altchords{|C |D |Em | \e}
    \vspace{1em}
    O la |\[Am]Mama, |\[D]{} kow wey ha |\[Em]ha ha ha |\[Bm11/D] \altchords{|Am |D |Em |Bm11/D}
    O la |\[C]Mama, |\[D]{} ta te ka|\[Em]yee | \e \altchords{|C |D |Em | \e}
  \mnendchorus
  \notesoff
  \textnotefornext{in English:}
  \beginchorus
    Ancient |^Mother, |^ I hear you |^calling |^
    Ancient |^Mother, |^ I hear your |^song | \e
    \vspace{1em}
    Ancient |^Mother, |^ I hear your |^laughter |^
    Ancient |^Mother, |^ I taste your |^tears | \e
  \endchorus
  % This second verse is a more unknown addon by somebody:
  \beginchorus
    Ancient |^Mother, |^ I feel you |^calling |^
    Ancient |^Mother, |^ I sing your |^song | \e
    \vspace{1em}
    Ancient |^Mother, |^ I share your |^laughter |^
    Ancient |^Mother, |^ I dry your |^tears | \e
  \endchorus
\endsong


\beginsong{Kothbiro}[by={Ayub Ogada},key={Am},gk={Am, Am--Em},ex={dholuo}, ph={III}]
  \audio[key=Am]{https://www.youtube.com/watch?v=b0Jwf-Y1uww}
  \audio[key=Am]{https://soundcloud.com/stampthewax/ayub-ogada-kothbiro}
  \meter{3}{4}
  \mnbeginverse
    |\sublyr{{ }{ }{ }Yaye nyithindogi}\[\mnc{A}Am]Haah{ }{ }{ } |\[\mn{C}]ha\[\mn{B}]ye \[\mn{G}]ha\[\mn{A}]ye|' \sublyr{un koro un u-} \e \altchords{\id[1]{(Dm)}|Dm | - | \e}
    |\sublyr{timoru nade?}\[\mn{C}]ha\[\mn{B}]ye \[\mn{G}]haye|\[Em]{'} \sublyrpush{Koth biro Ke}|\sublyr{luru dhok e}\[\mn{C}]ha\[\mn{B}]ye \sublyr{dala}\[\mn{G}]ha\[\mn{A}]ye|\[Am]{'}{ }{ } | \e \altchords{| - |Am | - |Dm | \e}
    |\sublyr{{ }{ }{ }To yaye}\[\mnc{A}Am]Haah{ }\sublyr{ny}{ }{ }|\sublyr{ithindogi}\[\mn{C}]ha\[\mn{B}]ye \[\mn{G}]ha\[\mn{A}]ye|' \sublyr{un koro un u-} \e
    |\sublyr{timoru nade?}\[\mn{C}]ha\[\mn{B}]ye \[\mn{G}]haye|\[Em]{'} \sublyrpush{Koth biro Ke}|\sublyr{luru dhok e}\[\mn{C}]ha\[\mn{B}]ye \sublyr{dala}\[\mn{G}]ha\[\mn{A}]ye|\[Am]{'}{ }{ } | \e
  \mnendverse
  \mnbeginverse
    |\[\mnc{A}Am]Ooh mam' |\sublyr{Yaye nyithindogi}\[\mn{E}]u\[\mn{D}]win\[\mn{C}]ja|' \sublyr{un koro un u-} \e
    |\sublyr{timoru nade?}Koth \[\mn{E}\mn{D}]bi\[\mn{B}]ro|\[Em]{'} \sublyrpush{Koth biro} \sublyr{Ke-}\[\mn{E}]Ke|\sublyr{lu-}lu\sublyr{ru}\[\mn{D}]ru \sublyr{dhok}dhok \sublyr{e}\[\mn{B}]e \sublyr{da-}\[\mn{C}]da\sublyr{la}la|\[Am]{'}{ }{ } | \e
    |\sublyr{{ }{ }{ }To yaye nyithindogi}\[\mnc{A}Am]Ooh mam' |\[\mn{E}]u\[\mn{D}]win\sublyr{{ }{ }un}\[\mn{C}]ja|\sublyr{koro un utimo-}' \e
    |\sublyr{ru nade?}Koth \[\mn{E}\mn{D}]bi\[\mn{B}]ro|\[Em]{'} \sublyrpush{Koth biro} \sublyr{Ke-}\[\mn{E}]Ke|\sublyr{lu-}lu\sublyr{ru}\[\mn{D}]ru \sublyr{dhok}dhok \sublyr{e}\[\mn{B}]e \sublyr{da-}\[\mn{C}]da|\sublyr{{ } { } la}\[Am]{-la} | \e
  \mnendverse
  \ifshowlilypond\vspace{-1ex}\fi%
  \begin{lilywrap}\begin{lilypond}[]
    % transcribed by larva, latest update on 2024-02
    \include "tex/lp-include-head.ly"
    % \header {
    %   title = "Kothbiro"
    %   composer = "Ayub Ogada"
    % }
    theMelody = \relative a' {
      % A B A B A (instru) A A(lyr) A B(lyr) A
      \key a \minor \slurSolid
      \set melismaBusyProperties = #'()
      \tempo 4 = 108
      \time 3/4
      % intro
      \repeat volta 2 { | a8 e'4 a,8 e'4 | a,8 e'8 d8 b4 r8 }
      | a8 b4 a8 b4 | a8 b8 d8 b4 r8
      | a8 e'4 a,8 e'4 | a,8 e'8 d8 b4 r8
      \repeat segno 2 {
        % Part A: Haah...
        % (Yaye nyithindogi un koro un utimoru nade? Koth biro keluru dhok e dala)
        \repeat volta 2 {
          | a2. \sectionmark "A"
          | c4 b4 g8 a8~ | a2 r4
          | c4 b4 g8 g8~ | g2 r4
          | c4 b4 g8 a8~ | a2 r4 | r2.
        }
        % Part B: Ooh mam'...
        % (Yaye nyithindogi un koro un utimoru nade? Koth biro keluru dhok e dala)
        \repeat volta 2 {
          a4. \sectionmark "B" a4. | e'4 d4 c4~ | c2 r4
          | c4 e8( d8) b4~ | b2 r8 e8
          | e8 d8 d8 b8
        } \alternative {
          { c8 c8~ | c2 r4 | r2. }
          { c4~ | c8 c4. r4 | r2. }
        }
      }
      \fine
    }
    theLyricsOne = \lyricmode {
      % intro
      \repeat volta 2 { \skip 1 \skip 1 \skip 1 \skip 1 \skip 1 \skip 1 \skip 1 \skip 1 }
      \skip 1 \skip 1 \skip 1 \skip 1 \skip 1 \skip 1 \skip 1 \skip 1
      \skip 1 \skip 1 \skip 1 \skip 1 \skip 1 \skip 1 \skip 1 \skip 1
      \repeat segno 2 {
        % Part A
        \repeat volta 2 {
          | Haah | ha -- ye ha -- ye | _
          | ha -- ye ha -- ye | _
          | ha -- ye ha -- ye | _ |
        }
        % Part B
        \repeat volta 2 {
          | Ooh mam' | u -- win -- ja | _
          | Koth bi -- _ ro | _
          Ke -- | lu -- ru dhok e
        } \alternative {
          { da -- la | _ }
          { da | _ -- la | }
        }
      }
    }
    theChordsCycle = \chordmode {
      | a2.:m
      | a:m | a:m
      | a:m | e:m
      | e:m | a:m | a:m
    }
    theChords = \chordmode {
      % intro
      \repeat volta 2 { | s2. | s2. }
      | s2.
      | s2. | s2. | s2.
      % part A
      \repeat volta 2 {
        \theChordsCycle
      }
      % part B
      \repeat volta 2 {
        | a2.:m
        | a:m | a:m
        | a:m | e:m
        | e2:m
      } \alternative {
        { s4 | a2.:m | a:m }
        { s4 | a2.:m | a:m }
      }
    }
   \layout { #(layout-set-staff-size 12) } % for better fit
   \include "tex/lp-include-tail-notab.ly"
  \end{lilypond}\end{lilywrap}
  \musicnote{The original order: A B A B A [instrumental] A A(+sublyrics) A B(+sublyrics) A}
  \ifshowlilypond\translationoff\fi % to fit whole song on one spread
  \begin{translation}[EN]
    \sublyr{Oh these children,}Oh mother, \sublyr{what are you doing?}if you can hear me
    \nextverse
    \sublyr{The rain is coming,}The rain is coming, \sublyr{bring the cattle home}bring the cattle home
  \end{translation}
\endsong


\beginsong{Ide Were}[by={traditional},tags={water},ex={yoruba},ph={III, IV},key={Am},gk={Gm, Gm--Bm}]
  \audio[key=Gm]{https://www.youtube.com/watch?v=YUeIFRkthb8}
  \transpose{5}%\preferflats
  \mnbeginverse
    \[\mn{B}]Ide |\[\mnc{E}Em]were were \[\mn{D}]nita \[\mn{B}]Osh|\[\mnc{A}D]un \altchords{\id[1]{(Em) \capo{5}}|Em |D}
    \[\mn{G}]I\[\mn{A}]de |\[\mnc{B}Gmaj7]were \[\mn{D}]we\[\mn{B}]re | \e \altchords{|Gmaj7 | \e}
    Ide |\[\mnc{E}Em]were were \[\mn{D}]nita \[\mn{B}]Osh|\[\mnc{A}D]un \altchords{|Em |D}
    \[\mn{G}]I\[\mn{A}]de |\[\mnc{B}G]were \[\mn{D}]we\[\mn{B}]re \[\mn{G}]nita \[\mn{E}]ya | \e \altchords{|G | \e}
    \[\mn{G}]Ocha kini|\[\mnc{E}C]ba ni\[\mn{G}]ta \[\mn{E}]Osh|\[\mnc{A}D]un \altchords{|C |D}
    \[\mn{B}]Chek|\[G]e \[\mn{G}]cheke \[\mn{E}]chek|\[C]e ni\[\mn{G}\mn{E}]ta |\[\mnc{A}D]ya \altchords{|G |C |D}
    \[\mn{G}]I\[\mn{A}]de |\[\mnc{B}Bm]were \[\mn{D}]we\[\mn{B}]re |\[B7]{} \e \altchords{|Bm |B7}
  \mnendverse
  \begin{explanation}[EN]
    This Yoruba chant is dedicated to \textbf{Oshun}, the Goddess of Love,
    happiness and prosperity. She brings to us all the good things of life,
    and is defender of the poor and the mother of all orphans; Goddess
    Oshun brings to them their needs in this life.
    \par
    Oshun (also known as Ochun or Oxum in Latin America) is an orixá, a highly
    benevolent spirit or deity that reflects one of the expressions of God in
    the Ifa and Yoruba religions (Nigeria).
    \par
    Thought to be the most beautiful of the female Orixás. No one can resist
    her charming laugh, her graceful dancing, and her lips that taste like
    honey. She has a lush womanly figure with full hips, which suggest
    fertility and eroticism.
    \par
    She exhibits all of the attributes connected with fresh flowing water:
    lively, sparking, refreshing, vivacious. She is the \underline{Goddess
    of sweet water} and can be discovered where there is fresh water, at
    rivers, ponds, lakes, and particularly waterfalls.
    \par
    She is also a healer of the sick. Teacher, who taught the Yoruba culture,
    agriculture and mysticism. The art of divination using cowrie shells. The
    bringer of song, music and dance, healing chants and meditations taught
    to her by her father Obatala, the first of the created Orishi.
    \par
    This chant speaks of a necklace as a symbol of initiation into love.
    \par
    According to the Yoruba elders, Oshun [also Osun, Oxum]{} is the ``unseen
    mother present at every gathering'', because Oshun is the Yoruba
    understanding of the cosmological forces of water, moisture, and
    attraction. Therefore she is omnipresent and omnipotent. Her power is
    represented in another Yoruba scripture which reminds us that ``no one is
    an enemy to water'' and therefore everyone has need of and should respect
    and revere Oshun, as well as her followers.
    \par
    Oshun is the force of harmony. Harmony we see as beauty, feel as love,
    and experience as ecstasy. Oshun according to the ancients was the only
    female Irunmole amongst the original 16 sent from the spirit realm to
    create the world. As such, she is revered as ``Yeye'' --- the sweet mother
    of us all. When the male Irunmole attempted to subjugate Oshun due to
    her femaleness she removed her divine energy, called ase by the Yoruba,
    from the project of creating the world and all subsequent efforts at
    creation were in vain. It was not until visiting with the Supreme Being,
    Olodumare, and begging Oshun pardon under the advice of Olodumare that
    the world could continue to be created. But not before Oshun had given
    birth to a son. This son became Elegba, the great conduit of ase in the
    Universe and also the eternal and infernal trickster.
    \par
    Oshun is known as Iyalode, the ``(explicitly female) chief of the realm.''
    She is also known as Laketi, she who has ears, because of how quickly
    and effectively she answers prayers. When she possesses her followers,
    she dances, flirts and then weeps --- because no one can love her enough
    and the world is not as beautiful as she knows it could be.
  \end{explanation}
  \imagecc[1]{Oshun_ed_by_larva_aa_1290x1613px.png}%
\endsong


\beginsong{O Mileko}[ex={swahili},ph={I, IV}]
  \meter{2}{4}
  \beginverse
    \[\mn{C}]Aka|túm\[\mn{A}]bale \[\mn{G}]aka|túm\[\mn{E}]bale \emph{\[\mn{C}]Be|\[\mn{E}]lele \[\mn{C}]be|\[\mn{E}]lele}
    Be|le tzimi be|le tzimi \emph{Tzi|mimi tzi|mimi}
  \endverse
  \beginverse
    Atzi|mi tzaya atzi|mi tzaya \emph{Tza|yaya tza|yaya}
    Tza|ya butu tza|ya butu \emph{Bu|tutu bu|tutu}
  \endverse
  \beginverse
    Abu|tu gnda abu|tu gnda \emph{Gn|danga gn|danga}
    Kun|da leli kun|da leli \emph{Ale|lila ale|lila}
  \endverse
  \beginverse
    Ale|li manga ale|li manga \emph{Man|ganga man|ganga}
    \up{*}\emph{o man|ganga man|ganga o} | | \e
  \endverse
    \altlyr{\emph{slow down}}
  \beginchorus
    |O | mi|leko, | |o | mi|leko | \e
  \endchorus
\endsong


%%%%%%%%%%%%%%%%%%%%%%%%%%%%%%%%%%%%%%%%%%%%%%%%%%%%%%%%%%%%%%%%%%%
%%% LATEST PRINTOUT CONTAINED THE SONGS ABOVE.                  %%%
%%%%%%%%%%%%%%%%%%%%%%%%%%%%%%%%%%%%%%%%%%%%%%%%%%%%%%%%%%%%%%%%%%%
%%% Please try to not change the song numbers above this point. %%%
%%% Add new songs only after this point.                        %%%
%%%%%%%%%%%%%%%%%%%%%%%%%%%%%%%%%%%%%%%%%%%%%%%%%%%%%%%%%%%%%%%%%%%


\iflyriconly\else\scleardpage\fi
\beginsong{Ave Maria}[by={Franz Schubert}, ex={latina, suomi}, ph={III}, key={C}, gk={D, C--E}]
  \meter{4}{4}
  \iflyriconly\else
    \musicnotefornext{intro:}*
    \beginverse
      |\[C] \[\bm]{ } |{ }{ } \[C7]{} |\[F]{} \[Fm]{} |\[C]{} \[\bm]{}{} \e
    \endverse
  \fi
  \mnbeginverse\memorize
    |\[\mnc{C}C]A{ } \[Am]-\[^\mn{B}]ve \[^\mn{C}]Ma|\[\mnc{E}C]ri \[\mncadj{1.5ex}{D}G7]-|\[\mnc{C}Am]a, \[\bm]{ }{ } |\[\mnc{D}Dm7]gra\[\mncii{C}{B}G7]tia \[^\mn{A}\mnc{B}]ple|\[\mnc{C}C]na! \[\bm]
    \[^\mn{E}]Ma|ri\[^\mn{D}\mn{C}]a, \[\mncii{B}{A}Am]gra\[^\mn{E}\mn{F#}]tia |\[\mnc{E}Am6]ple\[\mnc{D#}B7]na, \[^\mn{B}]Ma|\[\mnc{D}Dm]ri\[^\mn{C}]a, \[\bmc\mn{B}\mn{D}]gra\[^\mnc{E}]ti\[^\mn{F}\mn{D}]a \[^\mnc{B}]ple|\[\mnc{C}Am]na. \[Am6]
    \[^\mn{E}\mn{D}]A|\[G]ve\[^\mn{B}], \[\mncii{A}{C#}A7]A\[^\mn{E}]ve \[^\mn{G}\mn{E}]Do\[^\mn{C#}]mi|\[\mnc{D}G]nus! \[\mncadj{1ex}{A}D7]{\hspace{.75ex}}Do\[^\mn{B}]mi\[^\mn{C}\mn{B}]nus \[^\mn{A}]te|\[\mnc{G}G]cum. \[\bm]
    Bene|\[\mnc{D}G7]dicta \[\bm]tu \[^\mn{C#}]in \[^\mn{D}]mu\[^\mn{E}\mnc{D}]lie|\[\mncadj{.75ex}{E}C]{-ri}\[^\mn{C}]bus\[\bm], et |\[\mnc{D}G7]bene\[\bmc\mn{D}\mn{C#}\mn{D}\mn{F}\mn{E}\mnc{D}]dic|\[\mnc{C}Am]tus, \[\bm]
    Et |\[\mnc{D}G]bene\[\mnc{E}E]dictus fr\[^\mn{D}]uc\[^\mn{E}]tus |\[\mnc{G}Dm]ven\[^\mn{F}]tris,\[\bm] \[^\mn{A}]ventris |\[\mncii{E}{D}Dm7]tui, \[\mncii{C}{B}A\textdegree7\mn{C}\mn{E&}\mn{D}\mnc{C}]Je|\[\mnc{D}G]sus. \[G7]
    |\[\mnc{C}C]A{ } \[Am]-\[^\mn{B}]ve \[^\mn{C}]Ma|\[\mnc{E}C]ri \[\mncadj{1.5ex}{D}G7]-|\[\mnc{C}C]a! \[\bm]{ }{ } |{ }{ } \[C7]{} |\[F]{} \[Fm]{} |\[C]{} \[\bm]{}{} \e
  \mnendverse
  \notesoff
  \beginverse
    |^A{ } ^-ve Ma|^ri ^-|^a, ^ |^ma^{\hspace{1.5ex}ter} |^Dei! ^
    O|ra pro ^nobis pecca|^tori^bus, o|^ra, o^ra pro no|^bis; ^
    O|^ra, o^ra pro no|^bis ^{\hspace{.75ex}}peccatori|^bus, ^
    Nunc |^et in ^hora m|^ortis, ^ in |^hora ^mortis no|^strae, ^
    In |^hora ^mortis, mortis |^nostrae, ^ in |^hora ^mortis no|^strae. ^
    |^A{ } ^-ve Ma|^ri ^-|^a! ^{ }{ } |{ }{ } ^{} |^{} ^{} |^{ }{ } ^{} \e
  \endverse
  \ifchorded\hfill%
    \gtab{Am6}{X02212:002314} \gtab{A\textdegree7}{X01212:001324}%
  \fi
  \brk
  \textnotefornext{suomeksi:}
  \beginverse
    |^A{ } ^-ve Ma|^ri ^-|^a, ^ |^lem^{\hspace{1.5ex}pe}|^hin! ^
    Sun |luokses ^taivaaseen |^saak^ka
    nyt |^pyyntö ^nousee pala|^vin: ^
    pois |^sielun ^pyyhi synti|^taak^-|^ka. ^
    Niin |^huole^tonna uinahd|^amme, ^
    vaikk' |^syylli^nenkin tunto |^ois. ^
    Oi |^äiti, ^katso tuski|^amme, ^
    sä |^ällös ^käänny meistä |^pois. ^
    |^A{ } ^-ve Ma|^ri ^-|^a! ^{ }{ } |{ }{ } ^{} |^{} ^{} |^{ }{ } ^{} \e
  \endverse
  \beginverse
    |^A{ } ^-ve Ma|^ri ^-|^a, ^ |^tah^{\hspace{1.5ex}ra}|^ton! ^
    Jos |suoje^luksessasi |^ai^na
    me |^käymme ^täällä lepo|^hon, ^
    niin |^miel^tämme ei murhe |^pai^-|^na. ^
    Sun |^hymyil^lessäs ruusut a|^ukee ^
    maan |^päällä ^kylmän, kolkon |^tään. ^
    Oi |^äiti, ^lapses tuskaan |^raukee, ^
    sun |^armos ^huomaan yksin |^jään. ^
    |^A{ } ^-ve Ma|^ri ^-|^a! ^{ }{ } |{ }{ } ^{} |^{} ^{} |^{ }{ } ^{} \e
  \endverse
  \begin{translation}[EN]\iflyriconly\else\tiny\fi
    Hail Mary, full of grace! Mary, full of grace, Mary, full of grace.
    Hail, Hail, the Lord! The Lord is with thee.
    Blessed art thou among women, and blessed,
    And blessed is the fruit of thy womb, thy womb, Jesus. Hail Mary!
    \nextverse
    Hail Mary, Mother of God! Pray for us sinners, pray, pray for us;
    Pray, pray for us sinners,
    Now and at the hour of our death, the hour of our death,
    In the hour of death, our death, the hour of our death. Hail Mary!
  \end{translation}
  \begin{lilywrap}\begin{lilypond}[]
    %% transcribed by larva, latest update on 2024-10
    % \header {
    %   title = "Ave Maria"
    %   composer = "Franz Schubert"
    %   poet = "[these lyrics are from the Ave Maria prayer, original lyrics are different]"
    % }
    %% NOTE:
    %%  - the original is written in double time compared to this
    %%  - the accompaniment plays arpeggios as sextuplets
    \include "tex/lp-include-head.ly"
    theMelody = \relative c'' {
      \mark \markup \with-color #darkgreen \italic "Accompany with chords arpeggiated as sextuplets."
      \tempo 4 = 60
      \key a \minor \time 4/4
      \set melismaBusyProperties = #'() \slurSolid
      \sectionLabel "intro"
      | r1 | r1
      | r1 | r1
      \sectionLabel "verse"
      \repeat volta 2 {
        | c2. b8 c | e2..( d8)
        | c2 r2 | \once\slurDown d2( \grace e16 \grace d16) c8 b a( b)
        | c2 r4 e4 | e4. d16( c16) \once\slurDashed b8( a) e' fis
        | \once\slurDashed e4.( e8) dis4. b8 | d4. c8 \tuplet 3/2 { b8( d) e } \tuplet 3/2 { f( d) b }
        | c2. e8( d) | \once\slurDashed d4.( b8) \tuplet 3/2 { a8( cis) e } \tuplet 3/2 { g( e) cis }
        | d2~ \tuplet 3/2 { d8 a b } \tuplet 3/2 { \once\slurDown c( \grace d16 \grace c16 b8) a } | g2 r4 \once\slurDashed g8( g)
        | d'4. d8 \slurDashed d8.( cis16) d8( e16) d16~ | d8( e16) \slurSolid c4 r16 r4 c4
        | d4. d8 \tuplet 3/2 { d8( cis d } \tuplet 3/2 { f e d) } | c2 r4 c4
        | d4. d8 e8. e16 \tuplet 3/2 { e8( d) e } | g4 f r4 \once\slurDashed a,8( a)
        | e'4 d4 \tuplet 3/2 { c8( b c } \tuplet 3/2 { es8 d c) } | d2. r4
        | c2. b8 c | e2..( d8)
        | c2 r2 | r1
        | r1 | r1
      }
      \sectionLabel "outro"
      | r1 | r1 | r1
      \fine
    }
    theLyricsOne = \lyricmode {
      \set stanza = "1."
      | A -- ve Ma -- | ri _ -- | a, | gra -- ti --  a ple _ -- | na!
      Ma -- | ri -- a, __ _ gra _ -- ti -- a | ple _ -- na,
      Ma -- | ri -- a, gra _ -- ti -- a __ _ ple -- | na,
      A _ -- | ve, __ _ A _ -- ve Do _ -- mi -- | nus! __ _
      Do -- mi -- nus __ _ te -- | cum.
      Be -- ne -- | dic -- ta tu in mu -- li -- e | _ -- ri -- bus,
      et | be -- ne -- dic _ _ _ _ _ -- | tus,
      et | be -- ne -- dic -- tus fruc _ -- tus | ven -- tris,
      ven -- tris | tu -- i, Je _ _ _ _ _ -- | sus.
      | A -- ve Ma | ri _ -- | a! | | |
    }
    theLyricsTwo = \lyricmode {
      \set stanza = "2."
      | A -- ve Ma -- | ri _ -- | a, | ma _ _ -- ter __ _ | Dei!
      O -- | ra pro __ _ no -- bis pec -- ca -- | to -- ri -- bus,
      o -- | ra, o -- ra __ _ pro no _ _ -- | bis;
      o _ -- | ra, o -- ra __ _ pro no _ _ -- | bis __ _
      pec -- ca -- to _ -- ri -- | bus,
      nunc __ _ | et in ho _ -- ra __ _ mor | _ _ -- tis,
      in | ho -- ra mor _ -- tis no _ _ -- | strae,
      in | ho -- ra mor -- tis, mor _ -- tis | no -- strae,
      in __ _ | ho -- ra mor _ -- tis no _ _ -- | strae.
      | A -- ve Ma -- | ri _ -- | a! | | |
    }
    theLyricsThree = \lyricmode {
      \set stanza = "FI-1."
      | A -- ve Ma -- | ri _ -- | a, | lem _ _ -- pe _ -- | hin!
      Sun | luok -- ses __ _ tai _ -- vaa -- seen | saak _ -- ka
      nyt | pyyn -- tö nou _ -- see pa _ -- la -- | vin:
      pois _ | sie -- lun pyy _ -- hi syn _ -- ti -- | taak _ _ _ _ _ _ -- | ka.
      Niin __ _ | huo -- le -- ton -- na ui -- nah -- dam | _ _ -- me,
      vaikk' | syyl -- li -- nen _ -- kin tun _ -- to | ois.
      Oi | äi -- ti, kat -- so tus _ -- ki -- | am -- me,
      sä __ _ | äl -- lös kään _ -- ny meis _ -- tä | pois.
      | A -- ve Ma | ri _ -- | a! | | |
    }
    theLyricsFour = \lyricmode {
      \set stanza = "FI-2."
      | A -- ve Ma -- | ri _ -- | a, | tah _ _ -- ra _ -- | ton!
      Jos | suo -- je _ -- luk -- ses -- sa -- si | ai _ -- na
      me | käym -- me tääl _ -- lä le _ -- po -- | hon,
      niin _ | mi -- el -- täm _ -- me ei mur -- he | pai _ _ _ _ _ _ -- | na.
      Sun __ _ | hy -- myil -- les -- säs ruu -- sut au | _ _ -- kee
      maan | pääl -- lä kyl _ -- män, kol _ -- kon | tään.
      Oi | äi -- ti, lap -- ses tus _ -- kaan | rau -- kee,
      sun __ _ | ar -- mos huo _ -- maan yk _ -- sin | jään.
      | A -- ve Ma | ri _ -- | a! | | |
    }
    theChords = \chordmode {
      | c2 c2 | c2 c2:7
      | f2/c f:m/c | c1
      \repeat volta 2 {
        | c2 a:m6 | c/g g:7
        | a1:m | d2:m7 g:7
        | c1 | c2 a:m/c
        | a:m6/b b:7 | d1:m
        | a2:m a:m6 | g a:7/e
        | g/d d:7 | g1
        | g1:7 | c/g
        | g:7 | a:m
        | g2 e | d1:m
        | d2:m7 a:dim7 | g g:7
        | c a:m6 | c/g g:7
        | c1 | c2 c2:7
        | f2/c f:m/c | c1
      }
      | c1 | c1 | r1
    }
    \layout { #(layout-set-staff-size 15) } % for better fit
    \include "tex/lp-include-tail-notab.ly"
  \end{lilypond}\end{lilywrap}
  \begin{explanation}[EN]\iflyriconly\else\tiny\fi
    These lyrics are the Latin \emph{Ave Maria} prayer adapted to this melody.
    The original lyrics were from Walter Scott's narrative poem
    \emph{The Lady of the Lake}, as translated to German by Adam Storck.
    In the poem, the Ave Maria prayer is only mentioned.
  \end{explanation}
\endsong

    % Finnish songs
% =============
%
% The following sets the song number for the first song in this file.
% The number will automatically be incremented by one for each song.
% Please do not change this! Changing would make different versions of
% the songbook to have different numbers for the same songs, and it
% would totally mess up the selection booklets causing them to have
% wrong songs in them. (For the same reason, add new songs only to the
% end of each songs_ file.)
\setcounter{songnum}{600}


\beginsong{Kalevala-sävelmä}[key={Am}, gk={Dm, Gm--Em}]
  \meter{2}{4}
  \beginverse
    |\[\mnc{A}Am]Vaka |\[\mnc{B}Em/B]vanha |\[\mnc{C}C]Väi\[\mn{E}]nä|\[\mnc{B}Em/B]möi|\[Em]nen,
    |\[\mnc{c}Am]tie\[\mn{A}]tä|\[\mnc{D}Dm7]jä \[\mn{C}]i|\[\mnc{B}Em]än \[\mn{C}]i|\[\mnc{A}Am]kui|nen
  \endverse
  % Present the melody on a staff using Lilypond
  \begin{lilywrap}\begin{lilypond}[]
    % transcribed by larva, latest update on 2023-07
    \include "tex/lp-include-head.ly"
    theMelody = \relative c'' {
      \key a \minor \time 2/4
      \set melismaBusyProperties = #'() \slurDashed
      | a4 a | b b | c e | b2 | b2
      | c4 a | d c | b c | a2 a2 \bar "|."
    }
    theLyricsOne = \lyricmode {
      | Va -- ka | van -- ha | Väi -- nä -- | möi -- | nen,
      | tie -- tä -- | jä i -- | än i -- | kui -- | nen
    }
    theChords = \chordmode {
     | a2:m | e:m/b | c | e:m/b | e:m
     | a:m | d:m7 | e:m | a:m | a:m
    }
    %\layout { #(layout-set-staff-size 15) } % for better fit
    \include "tex/lp-include-tail-notab.ly"
  \end{lilypond}\end{lilywrap}
  %% Commented out for space and boringness reasons
  % \begin{lilywrap}\textnote{Haikea versio:}\begin{lilypond}[]
  %   % transcribed by larva, latest update on 2023-07
  %   \include "tex/lp-include-head.ly"
  %   theMelody = \relative c'' {
  %     \key a \minor \time 2/4
  %     \set melismaBusyProperties = #'() \slurDashed
  %     | d4 d | d a | d c | b2 | b2
  %     | d4 d| d c | b c | a2 a2 \bar "|."
  %   }
  %   theLyricsOne = \lyricmode {
  %     | Va -- ka | van -- ha | Väi -- nä -- | möi -- | nen,
  %     | tie -- tä -- | jä i -- | än i -- | kui -- | nen
  %   }
  %   theChords = \chordmode {
  %     | d2:m | d:m | d:m7 | e:m/b | e:m
  %     | d:m | d:m7 | e:m | a:m | a:m
  %   }
  %   %\layout { #(layout-set-staff-size 15) } % for better fit
  %   \include "tex/lp-include-tail.ly"
  % \end{lilypond}\end{lilywrap}
  \yesendsongvfill% to balance vspace before and after lilypond, as there is no other content
\endsong


\beginsong{Juurilaulu \\ Kuulumme piiriin}[tags={piiri}, ph={I, V}, key={Am}, gk={Bm, Am--G\shrp{}m}]
  % Present the melody on a staff using Lilypond
  \begin{lilywrap}\begin{lilypond}[] \include "tex/lp-include-head.ly"
    theMelody = \relative a' {
      \key a \minor \time 4/4 \partial 2
      \repeat volta 2 {
        a4 a8 a8 | g8( a8) a2.~
        | a2 a4 a8 b | c( a) a2.~
        | a2 a4 a8 b | c c~ c2.~
        | c2 c8 b4 g8 | g4 a2.~ | a2
      }
    }
    theLyricsOne = \lyricmode {
      Kuu -- lum -- me pii -- riin __
      Kuu -- lum -- me pii -- riin __
      Ai -- ko -- jen ta -- kaa __
      Ta -- kai -- sin kier -- toon __
    }
    theChords = \chordmode {
      s2 | a1:m | a:m | a:m | a:m | c | c2 g2 | a1:m | a:m
    }
    \include "tex/lp-include-tail.ly"
  \end{lilypond}\end{lilywrap}
  \yesendsongvfill% to balance vspace before and after lilypond, as there is no other content
\endsong


\beginsong{Lampaanpolska \\ Kekrilaulu \\ Yksi kaksi kolme neljä}[ph={III}, key={Am}, gk={Am, G\shrp{}m--Em}]
  \meter{3}{4}
  \beginverse
    |\[\mnc{A}Am]Yksi kaksi kolme |\[\mnc{B}E]neljä, |\[\mnc{C}Am]anna \[\mn{A}]i\[\mnc{G#}E]lon |\[\mnc{A}Am]olla.
    Ja |\[Am]kun suru |\[E]tulee, |\[Am]anna hä\[E]nen |\[Am]mennä.
  \endverse
  \beginverse
    |\[Am/E]Paarmat ne |\[Dm]laulaa, |\[C]neljä hiirtä |\[E]hyppelee.
    |\[Am]Kissi lyöpi |\[E]trummun päälle ja |\[Am]koko maa\[E]ilma |\[Am]pauhaa.
  \endverse
  % Image downloaded from: https://www.maxpixel.net/Musical-Instruments-Drum-Music-Jazz-Cat-1287910
  \imagecc[2]{cat_drumming_ed_by_larva__aa_transbg_CC0_1264x1788px.png}
\endsong


\beginsong{Laulu oravasta}[by={Otto Kotilainen, Aleksis Kivi}, key={Dm}, gk={Dm, Dm--Em}]
  % NOTE: this is based on the version by Aapo Similä
  \transpose{5} % transpose to Dm: then lowest note is G, the highest A
  \beginverse
    \musicnote{intro:}
    \up{*}\meter{3}{4}|\[C] \[F] \[Em]
  \endverse
  \beginverse\memorize
    \meter{3}{4}|\[\mnc{E}Am]Make\[^\mn{A}]as\[^\mn{C}]ti |\[\mncii{B}{A}Em]ora\[^\mn{G}]vai\[^\mn{E}]nen
    \meter{5}{4}|\[F]Makaa sammalhuonees\[G]sansa;
    \meter{3}{4}|\[Em]Sinnepä ei |\[C]Hallin hammas
    \meter{5}{4}|\[F]Eikä metsä\[Em]miehen \[Am]ansa
    \up{*}\meter{3}{4}|\[C]Ehtineet \[F]milloin\[Em]kaan, |\[C]ei \[F]milloin\[Em]kaan
  \endverse
  \notesoff
  \beginverse
    \meter{3}{4}|^Kammiostaan |^korkeasta
    \meter{5}{4}|^Katselee hän mailman ^piirii,
    \meter{3}{4}|^Taisteloa |^allans' monta;
    \meter{5}{4}|^Havuoksan ^rauhan^viiri
    \up{*}\meter{3}{4}|^Päällänsä ^liepoit^taa. |^ ^ ^
  \endverse
  \beginverse
    \meter{3}{4}|^Mikä elo |^onnellinen
    \meter{5}{4}|^Keinuvassa kehto^linnass'!
    \meter{3}{4}|^Siellä kiikkuu |^oravainen
    \meter{5}{4}|^Armaan kuusen ^äitin^rinnass':
    \up{*}\meter{3}{4}|^Metsolan ^kantele ^soi! |^ ^ ^
  \endverse
  \beginverse
    \meter{3}{4}|^Siellä torkkuu |^heiluhäntä
    \meter{5}{4}|^Akkunalla pienoi^sella,
    \meter{3}{4}|^Linnut laulain |^taivaan alla
    \meter{5}{4}|^Saattaa hänen ^ilta^sella
    \up{*}\meter{3}{4}|^Unien ^Kulta^laan. |^ ^ ^
  \endverse
  \musicnote{\up{*}grave}
\endsong


\beginsong{Taivas on sininen ja valkoinen}[by={trad.}, ph={II}, key={Am}, gk={Am, G\shrp{}m--Am}]
  \meter{4}{4}
  % declare new (global) named chord-replay registers:
  \newchords{chords_taivas_a}\newchords{chords_taivas_b}
  \mnbeginchorus\memorize[chords_taivas_a] % memorize chords into a named register
    |\[\mnc{A}Am]Taivas \[^\mn{C}]on \[\bmc\mn{E}]sininen \[^\mn{D}]ja |\[\mncii{C}{B}E7]val\[^\mn{A}\mn{G#}]koi\[\mnc{A}Am]nen \[^\mn{C}\mn{D}]ja
    |\[^\mn{E}]tähtö\[\mnc{F}Dm]si\[^\mn{E}\mn{D}]ä|\[\mnc{E}C]täyn\[E7]nä
    \mnendchorus\glueverses\mnbeginchorus\memorize[chords_taivas_b] % memorize chords into a named register
    |\[\mnc{E}Am]Niin on \[\mnc{B}Bm7&5]nuo\[^\mn{A}]ri |\[\mncii{G}{F}G7]sy\[^\mn{E}\mn{D}]dä\[\mnc{E}C]me\[^\mn{A}\mn{B}]ni
    |\[\mncii{C}{D}Am]a\[^\mn{E}\mn{D}]ja\[\mnc{C}E7]tuk\[^\mn{B}]sia |\[\mnc{A}Am]täyn\[\bm]nä
  \mnendchorus
  \notesoff
  \beginchorus\replay[chords_taivas_a] % replay chords from a named register
    |^Enkä mä ^muille |^ilmoi^{ta mun}
    |sydän^suru|^ja^ni
    \endchorus\glueverses\beginchorus\replay[chords_taivas_b] % replay chords from a named register
    |^Synkkä ^metsä ja |^kirkas ^taivas ne
    |^tuntee mun ^huoli|^a^ni
  \endchorus
\endsong


\beginsong{Haltin häät}[by={Hannu Seppänen, Arto Alaspää}]
  \ifchorded\baselineadj=-.2ex\fi % to fit on one page, TODO: could this be the default?
  \beginverse\memorize
    Kun |\[\mnc{C}C]ihmiskunnan aamu \[^\mn{D}]vas\[^\mn{E}]ta |\[Em]alkoi sarastaa
    Ja |\[F]Lappi oli jättiläisten |\[G]maana
    |\[C]Kaunis Malla-neito alkoi |\[Em]häitään valmistaa
    |\[F]Sulhasenaan nuori uljas |\[G]Saana
  \endverse
  \notesoff
  \beginverse
    |^Kaikkialta kansaa saapui |^Haltiin juhlimaan
    Ja |^kirkonkellot häitä alkoi |^soittaa
    |^Silloin astui kirkkoon tumma |^Pältsa Ruotsinmaan
    Hän |^vaimokseen myös Mallan tahtoi |^voittaa
  \endverse
  \beginverse
    \ind Hän |\[Am]aikoi estää häät ja kutsui |\[Em]velhot avukseen
    \ind Ja |\[Am]pian saikin juhlakansa |\[Em]kuulla kauhukseen
    \ind Kun |\[F]pohjoisesta vyöryi |\[C/G]jää ja yltyi |\[G]tuuli
  \endverse
  \beginverse
    |^Kirkkokansa pakeni ja |^Mallaa sylissään
    Myös |^Saana alkoi juosten turvaan |^kantaa
    He |^kauas eivät ehtineet kun |^jäivät alle jään
    Ja |^jähmettyivät Kilpisjärven |^rantaan
  \endverse
  \beginverse
    \ind On |\[Am]aikakaudet tuntureiksi |\[Em]heidät muuttaneet
    \ind Ja |\[Am]Kilpisjärven kasvattaneet |\[Em]Mallan kyyneleet
    \ind Kun |\[F]jäinen pohjoistuuli |\[C/G]soi myös itkee |\[G]Saana
  \endverse
\endsong


\beginsong{Finlandia-hymni}[by={Jean Sibelius, Veikko Koskenniemi}]
  % TODO: better chords?
  \beginverse
     \[\mnc{F#}D]Oi \[\mnc{E}A]Suo\[\mnc{F#}D]mi, |\[\mnc{G}G]kat\[^\mn{F#}]so, |\[\mnc{E}A]si\[^\mn{F#}]nun \[\mnc{D}G]päi\[^\mn{E}]väs' |\[A]koit\[\mnc{F#}D]taa,
    |\[D] yön \[A]uh\[D]ka |\[G]karkoi|\[A]tettu \[G]on \[A]jo |\[D]pois,
    |\[D] ja \[A/C#]aamun |\[Bm]kiuru |\[D/F#]kirkkaudessa |\[A]soit\[Em]taa
    |\[Em] kuin \[D/F#]itse |\[G]taiva|\[D]han kan\[A]si |\[F#]sois.
    |\[D] Yön \[A/C#]vallat |\[Bm]aamun |\[D/F#]valkeus \[A]jo |voit\[Em]taa,
    |\[Em] sun \[D/F#]päiväs |\[G]koittaa, |\[A7]oi syn\[D]nyin|maa! | \e
  \endverse
  \notesoff
  \beginverse
    ^Oi ^nou^se, |^Suomi, |^nosta ^korke|^al^le
    |^ pääs' ^sep^pe|^löimä |^suurten ^muis^to|^jen,
    |^ oi ^nouse, |^Suomi, |^näytit maail|^mal^le
    |^ sa ^että |^karkoi|^tit or^juu|^den
    |^ ja ^ettet |^taipu|^nut sa sor^ron |al^le,
    |^ on ^aamus |^alka|^nut, syn^nyin|maa! | \e
  \endverse
  % \begin{translation} % comment out: takes too much space
  %   Finland, behold, your day has now come dawning;
  %   Banished is night, its menace gone with light,
  %   Larks' song again in morning-brightness ringing,
  %   Filling the air to heaven's great height,
  %   And morning's glow, night's darkness overcoming;
  %   Your day is come, o my native land.
  %   \nextverse
  %   O Finland, rise, stand proud, the future facing,
  %   Your valiant deeds recalling, once again;
  %   O Finland rise, in the world's sight erasing
  %   From your fair brows vile slavery's stain.
  %   You were not broken by oppressors ruling;
  %   Your morning's come, o my native land.
  % \end{translation}
\endsong


\beginsong{Päivänsäde ja menninkäinen}[by={Reino Helismaa},ph={IV},key={C},gk={C, (B)--(C)}]
  \beginverse
    |\[\mnc{C}Am]Aurin\[^\mn{B}]ko \[^\mn{A}]kun |\[\mnc{D}Dm]päätti \[^\mn{C}]ret\[^\mn{B}]ken, |\[\mnc{C}Am]siskois\[^\mn{B}]taan \[^\mn{A}]jäi |\[\mnc{D}Dm]jälkeen \[^\mn{C}]het\[^\mn{B}]ken
    |\[Am]päivänsäde |\[E7]viimei|\[Am]nen. | \e
    |\[Dm]Hämärä jo |\[Am]metsään hiipi, |\[Dm]päivänsäde |\[Am]kultasiipi
    |\[D]aikoi juuri |\[D7]lentää eestä |\[G]sen, | \e
    kun |\[C]menninkäisen |\[Am]pienen näki |\[Dm]vastaan tule|\[G7]van;
    se |\[C]juuri oli |\[D7]noussut luolas|\[G]taan. | \e
    Kas |\[C]menninkäinen |\[C7]ennen päivän |\[F]laskua ei |\[F#\textdegree7]voi | \e
    mil|\[C]loinkaan elää |\[Dm7]pääl\[G7]lä |\[C]maan. | \e
  \endverse
  \notesoff
  \beginverse
    |^Katselivat |^toisiansa; |^menninkäinen |^rinnassansa
    |^tunsi kummaa |^leiskun|^taa. | \e
    |^Sanoi: poltat |^silmiäni, |^mut' en ole |^eläissäni
    |^nähnyt mitään |^yhtä iha|^naa! | \e
    Ei |^haittaa vaikka |^loisteesi mun |^sokeaksi |^saa;
    on |^pimeässä |^helppo taival|^taa. | \e
    Jää |^luokseni, niin |^kotiluolaan |^näytän sulle |^tien | \e
    ja |^sinut armaak|^se^ni |^vien! | \e
  \endverse
  \beginverse
    |^Säde vastas: |^peikko kulta, |^pimeys vie |^hengen multa,
    |^enkä toivo |^kuole|^maa. | \e
    |^Pois mun täytyy |^heti mennä, |^ellen kohta |^valoon lennä,
    |^niin en hetke|^äkään elää |^saa. | \e
    Niin |^lähti kaunis |^päivänsäde, |^mutta vielä|^kin,
    kun |^menninkäinen |^yksin tallus|^taa, | \e
    hän |^miettii, miksi |^toinen täällä |^valon lapsi |^on, | \e
    ja |^toinen yötä |^ra^kas|^taa. | \e
  \endverse
\endsong


% Force this song on its own page to not have an empty page after
% the song after this one. Remove this when chords are added or the
% songs are rearranged.
\sclearpage
\beginsong{Viatonten valssi}[by={Einojuhani Rautavaara, Eila Kivikk'aho},tags={(chords missing)}]
  \chordsoff % do not show empty line for non-existing chords
  \beginverse
    Kun kesäinen yö oli kirkkain ja tyyninä valvoivat veet
    ja helisi soittimet sirkkain kuin viulut ja kanteleet.
    Viisi pientä piru parkaa aivan ujoa ja arkaa
    sievin kumarruksin tohti käydä enkeleitä kohti.
  \endverse
  \beginverse
    Univormunsa karvaiset heitti, he sarvet ja saparovyön,
    oli lanteilla vain lukinseitti ja helisi harput yön.
    Enkelitkin sulkapaidan jätti tuonne, taakse aidan.
    Siellä häntä, siellä siipi toisiansa tervehtiipi.
  \endverse
  \beginverse
    Ja niinhän he, nostaen jalkaa, niin nätisti tanssia alkaa
    yli kallion kasteisen. Ja se yö oli onnellinen.
    Missäs sika, --- jos ei kerää kärsäänsä se yhtäperää ---
    siivet karvat, ynnä muuta, vielä maiskutellen suuta.
  \endverse
  \beginverse
    Sill' aikaa enkelit tanssi niin ujosti varpaillaan
    Vain pukuna pikkuinen kranssi, viis pirua toverinaan.
    Oi, pienoiset, ettehän arvaa, moni vaihtaa nahkaa ja karvaa.
    Mut harppua sirkat lyö, yhä kun on kesäyö.
  \endverse
  \beginverse
    Kerran tuli Aamunkoitto. Loppui tanssi, loppui soitto.
    Pirut, niinkuin enkelitkin, tunnusmerkkejänsä itki.  
  \endverse
\endsong


\beginsong{Täss' on nainen}[by={Hedningarna},ph={II},key={Am},gk={Am, Am--Em}]
    % TODO: chords (note: melody is good for Kalevala-type stuff)
  \meter{5}{8}
  \mnbeginchorus\memorize
    \lrep \[^\mn{A}]Täss' \[^\mn{G}]on |\[\mnc{A}Am]nainen tuu\[^\mn{G}]len |\[^\mn{A}]tuoma
    \[\mnc{B&}Amadd9-]Tuulen |\[\mnc{C}C]tuoma \[\mnc{B&}C7]ve'en \[^\mn{C}]ve|\[\mnc{A}Am]tämä \rrep
    \lrep \[^\mn{E}]Me\[^\mn{D}]ren |\[\mncii{E}{D}Am/E]aal\[^\mn{E}]to\[\mnc{D}(B&)]jen \[^\mn{E}]a|\[\mnc{A}Am]jama
    \[\mnc{B&}Amadd9-]Meren |\[\mnc{C}C]tyrskyn \[\mnc{B&}C7]työn\[^\mn{C}]te|\[\mnc{A}Am]lemä \rrep
  \mnendchorus
  \notesoff
  \beginchorus\replay
    \lrep Kuin mie |^käynen laula|mahan
    ^Laulan |^mie me^ret me|^siksi \rrep
    \lrep Suoloik|^si me^ren so|^merot
    ^Meren |^hiekat ^herne|^hiksi \rrep
  \endchorus
  \beginchorus\replay
    \lrep Yhen |^vyöni vyötän|nällä
    ^Yhen |^paita^ni pa|^nolla \rrep
    \lrep Solke|^ni so^litta|^malla
    ^Polki|^meni ^paina|^malla \rrep
  \endchorus
  \beginverse\replay
    \lrep Nouse |^luontoni lo|vesta
    ^Synty|^ni sy^västä |^maasta \rrep
    \lrep Synty|^ni sy^västä |^maasta
    ^Haavan |^alta ^halti|^ainen! \rrep
  \endverse
\endsong


\beginsong{Sadelaulu}[by={Sanna Kurki-Suonio},tags={vesi},ph={II},key={\~Am}, gk={\~Cm, \~Gm--\~C\shrp{}m}]
  \audio[key=Dm]{https://soundcloud.com/sannakurki-suonio/sadelaulu}
  % in ~Dm, the notes range from D to C'
  % in transposed ~Am, the notes range from A to G'
  \transpose{-5}
  \beginverse
    |\[\mnc{D}Dm]Sa\[^\mn{A}]de syöksy\[^\mn{E}]y|\[\mnc{F}F]vi \[^\mn{A}]sy\[\mncii{G}{F}C]li\[^\mn{E}]hin |\[\mnc{D}Dm]pi\[^\mn{A}]saraise\[^\mn{E}]t |\[\mnc{F}F]pai\[^\mn{A}]an \[\mncii{G}{F}C]pääl\[^\mn{E}]le
    |\[Dm]Vesi vihmo|\[F]en ve\[C]tävi |\[Dm]kaiken alleen |\[F]kaste\[C]levi
    |\[Dm]Minä vain sa|\[F]teessa \[C]seison |\[Dm]satehessa |\[F]suloi\[C]sessa
  \endverse
  \notesoff
  \beginverse
    |^Oi kaalinna |^ti-moo-^jaa |^Oi maalinna |^ti-moo-^ja-a-aa
  \endverse
  \beginverse
    |^Vesi mulle |^voiman ^tuopi |^voiman vahvan |^ja vä^kevän
    |^Pyyhkii pois pö|^lyiset ^mietteet |^ajatukset |^auvot^taapi
  \endverse
  \beginchorus\noteson
    \ind |\[\mnc{A}Am]Aaa-\[\mn{B}]aa |\[\mnc{G#}E]Aaa-\[\mn{E}]aa |\[\mnc{A}Am]Aaa-\[\mn{G}]aa-\[\mn{A}]aa-\[\mn{B}]aa |\[\mncii{C}{B}E]Aai-\[\mn{A}\mn{B}]aai-\[\mn{G#}]aaa-\[\mn{E}]aaa
  \endchorus
  \beginverse
    |^Ukko heittävi |^vasa^moitaan |^säästele ei |^sala^moitaan
    |^Minä vain sa|^teessa ^seison |^satehessa |^suloi^sessa
  \endverse
  \beginchorus
    |^Oi kaalinna |^ti-moo-^jaa |^Oi maalinna |^ti-moo-^ja-a-aa
  \endchorus
  \beginchorus
    \ind |\[Am]Aaa-aa |\[E]Aaa-aa |\[Am]Aaa-aa-aa-aa |\[E]Aai-aai-aaa-aaa \rep{4}
  \endchorus
  \beginverse
    |^Puhdista ve|^si puh^dista |^ajatukse|^ni kir^kasta
    |^Syän surusta |^sulat^tele |^tuskan tunteet |^tunnol^tani
    |^Aatteet alhaiset |^aivois^tani |^puhdista pi|^sara ^pieni
  \endverse
  \beginchorus
    \ind |\[Am]Aaa-aa |\[E]Aaa-aa |\[Am]Aaa-aa-aa-aa |\[E]Aai-aai-aaa-aaa
  \endchorus
  \beginverse
    |^Virtaa vesi, |^vihmo ^vesi |^voimaa tuot sä |^mulle ^vesi
    |^Virtaa vesi, |^vihmo ^vesi |^voimaa tuot sä |^mulle ^vesi
  \endverse
  \beginchorus
    |^Oi kaalinna |^ti-moo-^jaa |^Oi maalinna |^ti-moo-^ja-a-aa \rep{5}
  \endchorus % after this there are four measures with just chord C
  % Image downloaded from: https://imgbin.com/png/psfPETyB/water-png
  % Image license: Free for non-commercial use
  \imagecc[4]{water_transparent_bg_254x784px.png}
\endsong


\beginsong{Sisältäni portin löysin}[by={Pekka Streng},ph={III, IV},key={D},gk={D, C--D}]
  % in D the notes range from A to B'
  \beginverse\memorize
    |\[A] \[^\mn{A}]Sisäl|täni \[^\mn{B}]por\[^\mn{A}]tin |\[D]löy\[^\mn{F#}]sin | \e
    |\[A] melkein |huomaamattom|\[D]an. | \e
    |\[G] Kun sen |läpi hiljaa |\[D]nousen, | \e
    |\[G] näen |toisin \[A] maail|\[D]man. | \e
  \endverse
  \notesoff
  \beginverse
    |^ Värit k|auniit vasta h|^uomaan, | \e
    |^ kuulen |äänet kirkkaam|^mat. | \e
    |^ Jätän |soinnuttomat lu|^olat, | \e
    |^ jätän |varjot ^ hoippuv|^at. | \e
  \endverse
  \noteson
  \beginverse
    \ind |\[\mnc{A}A]Aaa\ldots |\[\mn{E}] |\[\mnc{F#}D] |\[\mn{D}] \[\mn{A}] |\[\mnc{C#}A] | |\[D]\[\mn{D}] |\[\mn{F#}] \[\mn{A}]
    \ind |\[A]Aaa\ldots | |\[D] | |\[A] | |\[D] | \e
  \endverse
  \notesoff
  \beginverse
    |^ Jokin |säteilee ja loi|^staa | \e
    |^ alta k|uoren synkänk|^in. | \e
    |^ Kun sen |huomaa kevyem|^min | \e
    |^ ajatuk|set ^ liikkuv|^at. | \e
  \endverse
  \beginverse
    |^ Meidän |värit ylös vi|^rtaa | \e
    |^ ja |yhteen sulaut|^uu. | \e
    |^ Kaikki to|istaan kosket|^taa, | \e
    |^ kaikki a|amuun ^ kurkot|^tuu. | \e \goto{Aaa}
  \endverse
\endsong


\beginsong{Nouse luontoni lovesta}[by={Antti Tuonela},ph={I}]
  \beginverse
    |\[\mnc{A}Dm]Nouse \[\mn{G}]luon\[\mn{F}]toni |\[\mn{G}]lo\[\mn{D}]vesta \echo{|Nouse luontoni |lovesta}
    |Syntyni syvästä |maasta \echo{|Syntyni syvästä |maasta}
    |Nouse niin kuin |nousit ennen \echo{|Nouse niin kuin |nousit ennen}
    |Minun nostate|llessani \echo{|Minun nostate|llessani}
  \endverse
  \beginverse
    Nosta |\[Dm]Ukon voima |taivahas\sublyr{Nosta}ta \echo{|Ukon voima |taivahasta}
    |Maasta Maan E|moisen voima \echo{|Maasta Maan E|moisen voima}
    |Nouse niin kuin |nousit ennen \echo{|Nouse niin kuin |nousit ennen}
    |Minun nostate|llessani \echo{|Minun nostate|llessani}
  \endverse
  \beginchorus
    \ind |\[\mnc{G}E&]Tuekseni turvakseni |väekseni \[\mn{F}]voi\[\mn{E&}]makse|\[\mnc{D}Dm]ni | \e
    \ind |\[E&]Tuekseni turvakseni |väekseni voimakse|\[Dm]ni | \e
  \endchorus
  \textnote{\emph{D.C. al Fine}}
  \beginchorus
    |\[Dm]Nouse luontoni |lovesta \echo{|Nouse luontoni |lovesta}
    |\[Dm]Nouse luontoni |lovesta \echo{|Nouse luontoni |lovesta}
  \endchorus
\endsong


\beginsong{Äidin laulu}[index={Laulan sinulle lapsoseni},by={Marika Salo},tags={Äiti Maa},ph={III}]
  \meter{4}{4}
  \beginverse
    |\[\mnc{A}Am]Laulan \[\mnc{B}G]sinulle |\[Am]lapsoseni |\[Em]laulan sinulle |\[Am]laulun
  \endverse\glueverses
  \beginchorus
    |\[Am]Kuule \[G]minua |\[Am]lapsoseni, kun |\[Em]äitisi laulaa |\[Am]sulle
  \endchorus
  \notesoff
  \beginverse
    |^Missä ^ikinä |^kuljetkin |^siellä olen |^aina
  \endverse\glueverses
  \beginchorus
    |^Olen ^jalkojes |^alla |^metsän puissa ja |^tuulessa
  \endchorus
  \beginverse
    |^Vuoret on ^syntyneet |^kupeistani |^laaksot rintojen |^välistä
  \endverse\glueverses
  \beginchorus
    |^Meret ja ^joet |^kohdustani |^veri on värjännyt |^maan
  \endchorus
  \beginverse
    |^Hyvä sun on ^täällä |^kulkea |^maan ja taivaan |^väliä
  \endverse\glueverses
  \beginchorus
    |^Äitisi ^silittää |^varpaitasi ja |^isäs silittää |^päätä
  \endchorus
  \beginverse
    Ja |^jos sattuis ^lapseni |^käymään niin, että |^ilmaan tipah|^taisit
  \endverse\glueverses
  \beginchorus
    Niin |^älä sinä ^lapseni |^huolta kanna |^isäs ottaa |^kopin
  \endchorus
  \beginverse
    |^Ei ole ^harha-|^askelia |^ei ole |^virheitä
  \endverse\glueverses
  \beginchorus
    |^Kauneutta ^kohti |^kuljet vain täällä |^ikuisessa |^sylissä
  \endchorus
  \beginverse
    Ja |^vielä ^kerron |^sinulle |^kuuntele vielä |^hetki
  \endverse\glueverses
  \beginchorus
    |^Aina oot ^ollut |^toivottu ja |^tänne terve|^tullut
  \endchorus
\endsong


\beginsong{Olkoon niin}[by={Laura Iso-Metsälä},ph={III}]
  \audio[]{https://soundcloud.com/arulai/olkoon-niin-live-in-temppeli}
  \beginverse
    |\[\mnc{B}Bm]Ol\[\mn{C#}]koon |\[\mn{D}]niin, olkoon |\[A]ni\[\mn{C#}]in | \e
    |\[Bm]Olkoon |niin, olkoon |\[A]niin | \e
    |\[Bm]Olkoon |niin, olkoon |\[A]niin | \e
    Että olet |\[Bm]terve, tur|vassa ja |\[A]vapaa | \e
  \endverse
  \beginverse
    Että olet |\[D]onnellin|en ja |\[F#m]rauhallin|en
    Olet |\[A]turvas|sa nyt ja |\[Bm]aina | \e
    Että olet |\[D]onnellin|en ja |\[F#m]rauhallin|en
    Olet |\[A]ter|ve ja |\[Bm]vapaa | \e
  \endverse
  \beginchorus
    \ind Ahee a|\[Bm]hoo, hee a|hoo, hee a|\[A]hoo | \e
    \rep{4}
  \endchorus
  \begin{feeler}
    May all beings be peaceful.\\
    May all beings be happy.\\
    May all beings be safe.\\
    May all beings awaken to\\
    the light of their true nature.\\
    May all beings be free.
  \end{feeler}
\endsong


\beginsong{Koti mun luona}[by={Malla Maanpiiri}, ph={III, IV}, key={G}, gk={G, G--B}]
  \mnbeginchorus\memorize
    |\[\mnc{C}C]Sulla sulla |\[\mnc{G}G]siskokulta on
    |\[\mnc{C}Am]aina koti mun |\[\mnc{G}Em/G]luona | \e
  \mnendchorus
  \notesoff
  \beginchorus\replay
    \ind Me |^ollaan täällä |^näyttämässä,
    \ind |^mitä rakkaus |^on | \e
  \endchorus
  \beginchorus\replay
    |^Sulla sulla |^velikulta on
    |^aina koti mun |^luona | \e
  \endchorus
  \goto{Me ollaan täällä}
  \beginchorus\replay
    |^Sulla sulla |^lapsikulta on
    |^aina koti mun |^luona | \e
  \endchorus
  \goto{Me ollaan täällä}
\endsong


\beginsong{Siunattu voima}[by={Lotta Maija}, ph={III}, key={Am}, gk={Am, Am--Em}]
  \audio[key=Am]{https://www.youtube.com/watch?v=PHNaFQiBNuU}
  \mnbeginverse
    |\[\mncii{A}{B}Am]Illan \[^\mn{A}]suus\[^\mn{G}]sa |\[^\mn{A}]va\[^\mn{B}\mnc{C}]lon \[^\mn{B}]voi\[^\mn{G}]ma, |\emph{\[^\mn{A}\mn{B}\mn{C}]siunat\[^\mn{D}]tu |\[^\mn{E}]voima} \altchords{\id[1]{(Bm)}|Bm | - | - | \e }
    |\[\mnc{D}G]Las\[^\mn{C}]ku \[^\mn{B}]au\[^\mn{A}]rin|koi\[^\mn{B}]sen \[^\mn{A}\mn{G}]illan, |\emph{\[\mnciii{A}{B}{A}\emph{Am}]siunat\[^\mn{G}]tu |v\[^\mn{A}]oima} \altchords{|A | - |Bm | \e}
    |Aurinkoinen |lehvän leuto, |\emph{siunattu |voima} \altchords{|Bm | - | - | \e }
    |\[G]Kesän henki |koivun hento, |\emph{\[\emph{Am}]siunattu |voima} \altchords{|A | - |Bm | \e}
  \mnendverse
  \notesoff
  \beginverse
    |^Vetten päälle |valon loiste, |\emph{siunattu |voima}
    |^Valon loiste |sielun kirkkaus, |\emph{^siunattu |voima}
    |Kirkastelee |kimmellellen, |\emph{siunattu |voima}
    |^Valon kanssa |värähdellen, |\emph{^siunattu |voima}
  \endverse
  \beginverse
    |^Syömeen paistaa |herätellen, |\emph{siunattu |voima}
    |^Herätellen |sytytellen, |\emph{^siunattu |voima}
    |Ikiajan |valon voima, |\emph{siunattu |voima}
    |^Kipunoita |kaukaa tuolta, |\emph{^siunattu |voima}
  \endverse
  \beginverse
    |^Lämmön synnyn |loiste ompi, |\emph{siunattu |voima}
    |^Ajan takaa |ikuisempi, |\emph{^siunattu |voima}
  \endverse
\endsong


\beginsong{Kiitos elämälle}[by={Tiia Ilomäki},tags={kiitollisuus},ph={V}]
  \beginverse
    |\[Am] \[\mn{E}]Tämä on |lau\[\mn{A}]lu e\[\mn{C}]lä|\[Em]mäl\[\mn{B}]le, | \e
    |\[Am] sen valoil|le ja |\[Em]varjoil|le.
    |\[Am] Tämä on |laulu tun|\[Em]teille, | \e
    |\[Am] joskus niin |helvetin |\[Em]tukahdutta|ville.
    |\[Am] | | | \e
    |\[Am] Tämä on |laulu tans|\[Em]sille, | \e
    |\[Am] liik|keelle niin |\[Em]kauniil|le.
    |\[Am] Tämä on |laulu nau|\[Em]rulle, | \e
    |\[Am] het|kille |\[Em]yhtei|sille.
    |\[Am] Tämä on |laulu ihmi|\[Em]sille, | \e
    |\[Am] rak|kaudelle |\[Em]jaetul|le.
    |\[Am] |\[C] |\[Am] |\[C] \e
    |\[Em] | | | \e
  \endverse
  \beginchorus
    \[\mn{B}]Mä laulan |\[G]kii\[\mn{E}]tos | \[\mn{G}]elä|\[\mnc{F#}Em]mäl\[\mn{E}]le | \e
    Mä laulan |\[G]kiitos | tun|\[Em]teille | \e
    Mä laulan |\[G]kiitos | tans|\[Em]sille | \e
    Mä laulan |\[G]kiitos | nau|\[Em]rulle | \e
    Mä laulan |\[G]kiitos | ihmi|\[Em]sille | \e
    |\[C] Olemme |tulleet tänne | luomaan |uutta
    |\[Em] maail|maa | | \e
    \up{*}\echo{maail|\[G]maa, | | | \e
    maail|\[Em]maa, | | | \e
    maail|\[G]maa, | | | \e
    maail|\[Em]maa | | | \e} \altlyr[*]{Vocalize on 2nd repeat only}
  \endchorus
  \beginverse
    |\[Am] Tämä on |laulu elä|\[Em]mälle, | \e
    |\[Am] sen valoil|le ja |\[Em]varjoil|le.
  \endverse
\endsong


\beginsong{Nyt on laulut laulettu}[ex={based on a Hungarian folk song},ph={V},key={Am},gk={Am, Gm--D\shrp{}m}]
  \audio{https://www.youtube.com/watch?v=voIa8mZ3UDc}
  \mnbeginverse
    |\[\mnc{C}Am]Nyt on \[\bmc\mn{B}]laulut |\[^\mn{A}]laulet\[\bm]tu ja |\[\mnc{E}C]lähdet\[\mnc{D}G]tävä |\[\mnc{C}C]on. \[\bm]
    |\[^\mn{E}]Hei \[\mnc{F}Fmaj7]vaan, |\[\mnc{E}C]hei\[^\mn{D}]pä \[\bmc\mn{C}]hei \[^\mn{A}]ja |\[^\mn{C}]lähdet\[\mnc{B}E]tävä |\[\mnc{A}Am]on. \[\bm]
  \mnendverse
  \notesoff
  \beginverse
    |^Toisen ^kerran |tava^tessa |^laulut ^uudet |^on. ^
    |Hei ^vaan, |^heipä ^hei ja |laulut ^uudet |^on. ^
  \endverse
%   % Lilypond notation commented out to save space for this simple song
%   \begin{lilywrap}\begin{lilypond}[] \include "tex/lp-include-head.ly"
%     theMelody = \relative c'' {
%       \key a \minor \time 4/4
%       % https://kansalliskirjasto.finna.fi/Record/fikka.4746565
%       % Unkarilainen kansansävelmä
%       % Suom. sanat: Liisa Tenkku
%       \repeat volta 2 {
%         | c4 c b b | a a a a | e' e d d | c1
%         | e2 f2 | e4. d8 c4 a | c c b b | a1
%       }
%     }
%     theLyricsOne = \lyricmode {
%       \set stanza = "1."
%       | Nyt on lau -- lut | lau -- let -- tu ja | läh -- det -- tä -- vä | on.
%       | Hei vaan, | hei -- pä hei ja | läh -- det -- tä -- vä | on.
%     }
%     theLyricsTwo = \lyricmode {
%       \set stanza = "2."
%       | Toi -- sen ker -- ran | ta -- va -- tes -- sa | lau -- lut uu -- det | on.
%       | Hei vaan, | hei -- pä hei ja | lau -- lut uu -- det | on.
%     }
%     theChords = \chordmode {
%       \repeat volta 2 {
%         | a1:m | a:m | c2 g2 | c1
%         | c2 f2:maj7 | c1 | c2 e2 | a1:m
%       }
%     }
%     \include "tex/lp-include-tail.ly"
%   \end{lilypond}\end{lilywrap}
\endsong


\beginsong{Tupakkarulla}[by={trad.},tags={uni}]
  \meter{2}{4}
  \beginverse
    |\[\mnc{D}Dm]Tuu |\[^\mn{A}]tuu |tupakka|\[A]rul|la \brk|\[Gm]mistäs |\[Dm]tiesit |\[A7]tänne |\[Dm]tul|la?
    |\[Dm]Tulin |pitkin |\[Gm7]Turun |\[A]tie|tä, \brk|\[A7]hämä|\[Dm]läisten |\[A7]härkä|\[Dm]tie|tä.
  \endverse
  \notesoff
  \beginverse
    |^Mistäs |tunsit |meidän |^por|tin? \brk|^Siitä |^tunsin |^uuden |^por|tin:
    |^haka |alla, |^pyörä |^pääl|lä \brk|^karhun |^talja |^portin |^pääl|lä
  \endverse
  \beginverse
    |^Uni |kysyi |uunin |^pääl|tä, \brk|^unen |^poika |^porstu|^as|ta:
    |^Onko |lasta |^kätky|^es|sä, \brk|^pientä |^peittei|^den si|^säs|sä?
  \endverse
  \beginverse
    |^Tuoppa |unta |tuokko|^ses|sa, \brk|^kanna |^vaski |^vakka|^ses|sa,
    |^sillä |silmät |^sive|^le,| \brk|^näky|^miset |^näppä|^e|le.
  \endverse
  \beginverse
    |^Nuku |nuku |nurmi|^lin|tu, \brk|^väsy |^väsy |^västä|^räk|ki,
    |^nuku |kun mi|^nä nu|^ku|tan, \brk|^väsy |^kun mi|^nä vä|^sy|tän.
  \endverse
  % Present the melody on a staff using Lilypond
  \begin{lilywrap}\begin{lilypond}[] \include "tex/lp-include-head.ly"
    {\key d \minor \time 2/4
      d'2 | a'2 | a'8 a'4 f'8 | e'2 | e'2
      g'4 g'4 | a'4 f'4 | e'4 f'4 | d'2 | d'2
      f'4 d'4 | e'4 f'4 | g'4 f'4 | e'2 | e'2
      a'4. g'8 | f'4 f'4 | e'4 f'4 | d'2 | d'2 \bar "|."
    }\addlyrics{
      Tuu tuu tu -- pak -- ka -- rul -- la,
      mis -- täs tie -- sit tän -- ne tul -- la?
      Tu -- lin pit -- kin Tu -- run tie -- tä,
      hä -- mä -- läis -- ten här -- kä -- tie -- tä.
    }
  \end{lilypond}\end{lilywrap}
  % Nicely align music notation on both songs of this spread to the bottom.
  % for symmetry. (This is actually the default if Lilypond block is last,
  % but in if it changes, it doesn't have to change here.)
  \noendsongvfill
\endsong


\beginsong{Leppäkerttu}[by={trad.},tags={uni},ph={V}]
  \meter{4}{4}
  \beginverse
    |\[\mnc{D}Dm]Lennä, \[^\mn{A}]lennä |\[A]leppäkerttu, |\[Gm]ison \[Dm]kiven |\[A7]juu\[Dm]reen.
    |\[Dm]Lennä leikki|\[Gm7]kedon \[A]kautta |\[A7]unipuuhun |\[Dm]suureen.
  \endverse
  \notesoff
  \beginverse
    |^Kulta-kulta|^lehden alla |^äiti ^puuron |^keit^tää.
    |^Unituutu |^leppä^kertun |^lämpimästi |^peittää.
  \endverse
  \beginverse
    |^Laula, laula, |^unilintu, |^tuoksu, ^tuomen|^tert^tu.
    |^Nuku, puna|^paitu^lainen, |^pikku leppä|^kerttu.
  \endverse
  % Present the melody on a staff using Lilypond
  \begin{lilywrap}
    \imagerb[5]{leppakerttu_transparent_bg_353x279px.png}
    \begin{lilypond}[] \include "tex/lp-include-head.ly"
      {\key d \minor \time 4/4
        d'4 d'4 a'4 a'4 | a'4 a'4 e'4 e'4
        g'4 g'4 f'4 f'4 | e'2 d'2
        f'4 d'4 e'4 f'4 | g'4 f'4 e'4 e'4
        a'4 g'4 f'4 e'4 | d'2 d'2 \bar "|."
      }\addlyrics{
        Len -- nä, len -- nä lep -- pä -- kert -- tu,
        i -- son ki -- ven juu -- reen.
        Len -- nä leik -- ki -- ke -- don kaut -- ta
        u -- ni -- puu -- hun suu -- reen.
      }
    \end{lilypond}
  \end{lilywrap}
  % Nicely align music notation on both songs of this spread to the bottom.
  % for symmetry. (This is actually the default if Lilypond block is last,
  % but in if it changes, it doesn't have to change here.)
  \noendsongvfill
\endsong


\beginsong{Pieni tytön tylleröinen}[tags={uni},key={Dm},gk={Dm, Cm--D\shrp{}m}]
  \audio[key=Dm]{https://www.youtube.com/watch?v=iFEi8XTWSRM}
  \mnbeginverse
    |\[\mnc{D}Dm]Pie\[^\mn{A}]ni \up{*}ty\[^\mn{B&}]tön |\[\mnc{A}Gm]tyl\[^\mn{G}]leröi\[^\mn{E}]nen |\[\mnc{F}Dm]tietä pit\[^\mn{E}]kin |\[^\mn{F}]kul\[^\mn{A}]ki.
    |\[\mnc{F}B&]Saapui sinne |\[Gm]Nuk\[^\mn{E}]ku-\[E7]Mat\[^\mn{D}]ti, |\[\mnc{A}A7]silmät \[^\mn{C#}]pienet |\[\mnc{D}Dm]sulki.
  \mnendverse
  \notesoff
  \beginverse
    |^Kasvoi kuusi |^kukkalatva, |^käki siinä |kukkui.
    |^Mutta \up{*}tytön |^tylle^röinen |^nurmikolla |^nukkui.
  \endverse
  \beginverse
    |^Pieni \up{*}tytön |^tylleröinen |^sievää unta |näki
    |^että hänen |^ympä^rilleen |^tuli metsän |^väki.
  \endverse
  \beginverse
    |^Tapio ja |^Tellervo ja |^Sinipiika |pieni,
    |^Mustikka ja |^Mansik^ka ja |^suuri metsän |^sieni.
  \endverse
  \beginverse
    |^Sipsutteli |^Sinipiika |^pienen \up{*}tytön |luokse;
    |^otti kiinni |^kädes^tä, |^hyppeli ja |^juoksi.
  \endverse
  \beginverse
    |^Eipä \up{*}tytön |^tylleröinen |^ollut mitään |vailla.
    |^Hauska oli |^oles^kella |^Nukku-Matin |^mailla.
  \endverse
  \altlyr{pojan (palleroinen)}
  \imagecc[4]{sleeping_baby_bw_transparent_bg_1280px.png}%
\endsong


\beginsong{Tuuin yössä muukalaista}[by={Neilikka},tags={uni},key={Bm},gk={Cm, Cm--D\shrp{}m}]
  \audio[key=Cm]{https://www.youtube.com/watch?v=OR7YYnrGvPg}
  \meter{3}{4}
  \transpose{2} % in Am the notes range from E to F', in Dm from A to Bb', in Bm from F# to G'
  \mnbeginverse
    |\[\mnc{E}Am]Tuu\[^\mn{C}]in \[^\mn{E}]yössä |\[Dm]muu\[^\mn{D}]ka\[^\mn{A}]laista, |\[\mnc{C}Am]tum\[^\mn{A}]ma\[^\mn{C}]sil\[^\mn{E}]mää |\[\mnc{B}E]lasta \altchords{\id[1]{(Am)}|Am |Dm |Am |E}
    |\[\mnc{D}Dm]Äsken \[^\mn{E}]tän\[^\mn{F}]ne |\[\mnc{E}Am]kut\[^\mn{C}]su\[^\mn{E}]maani, |\[\mnc{E}E]maasta \[^\mn{C}]i\[^\mn{B}]ha|\[Am]nas\[^\mn{A}]ta \altchords{|Dm |Am |E |Am}
  \mnendverse
  \notesoff
  \beginverse
    |^Unenrihmat |^sinne sitoo |^sen ken tuli |^vasta
    |^Nuku paluun |^sinne laulan |^tuulen suhi|^nasta
  \endverse
  \beginverse
    |^Suvisirkan |^soittelosta, |^hämärästä |^illan
    |^Laulan pilven, |^taivaan mieltä, |^laulan seitti|^sillan
  \endverse
  \beginverse
    |^Lapsen käydä |^univarpain |^kohti onnen |^maata \altchords{\id[2]{(Dm)}|Dm |Gm |Dm |A}
    |^Sinne äitis |^murhemieli |^seurata ei |^saata \altchords{|Gm |Dm |A |Dm}
  \endverse
  \beginverse
    \musicnote{interlude:}
    |^ |^ |^ |^
    |^ |^ |^ |^
  \endverse
  \beginverse
    |^Siel ei tunnu |^talven tuskat, |^siel on aina |^kesä
    |^Siellä metsän |^joka puussa |^ilolla on |^pesä
  \endverse
  \beginverse
    |^Polut syliin |^satumetsän |^houkuttaa ja |^hukkuu
    |^Siellä lapsi |^itkutonna |^niittyvillaan |^nukkuu
  \endverse
\endsong


\beginsong{Suojelusenkeli}[by={P. J. Hannikainen},tags={suojelus}]
  \meter{3}{8}
  \beginverse
    \[^\mn{B}]Maan |\[\mnc{E}Em]korvessa |kulkevi |\[B]lapsosen |\[Em]tie.
    \[B]Hänt' |\[Em]ihana |\[G]enkeli |\[D]kotihin |\[G]vie.
    Niin |\[Em]pitkä \[C]on |\[D]matka, \[B]ei |\[Em]kotia |\[B]näy, | \e
    vaan |\[Em]ihana |\[C]enke\[B]li |\[Em]vieres\[Am]sä |\[B]käy,
    vaan |\[Em]ihana |\[Am]enkeli |\[Em/B]vieres\[B7]sä |\[Em]käy.
  \endverse
  \notesoff
  \beginverse
    On |^pimeä |korpi ja |^kivinen |^tie,
    ^ja |^usein se |^käytävä |^liukaskin |^lie.
    Oi, |^pian^han |^lapso^nen |^langeta |^vois, | \e
    jos |^käsi ei |^enke^lin |^kädes^sä |^ois,
    jos |^käsi ei |^enkelin |^kädes^sä |^ois.
  \endverse
  \beginverse
    % original: Ja syntikin mustia verkkoja vaan
    Ja |^mielikin |mustia |^verkkoja |^vaan
    ^on |^laajalle |^laskenut |^korpehen |^maan.
    Niin |^pian^han |^niihin^kin |^tarttua |^vois, | \e
    jos |^käsi ei |^enke^lin |^kädes^sä |^ois,
    jos |^käsi ei |^enkelin |^kädes^sä |^ois.
  \endverse
  \beginverse
    Maan |^korvessa |kulkevi |^lapsosen |^tie.
    ^Hänt' |^ihana |^enkeli |^kotihin |^vie.
    Oi, |^laps' et^hän |^milloin^kaan |^ottaa sä |^vois | \e
    sä |^kättäsi |^enke^lin |^kädes^tä |^pois.
    sä |^kättäsi |^enkelin |^kädes^tä |^pois.
  \endverse
\endsong


\beginsong{En etsi valtaa loistoa}[by={Jean Sibelius, Sakari Topelius}]
  \meter{4}{4}
  \beginverse
    \[^\mn{F#}]En |\[D]etsi \[^\mn{G}]val\[^\mn{F#}]taa, |\[Em]loisto\[A7]a, en |\[D]kaipaa \[A7]kul\[D]taa|\[A]kaan,
    mä |\[Em]pyydän \[A7]taivaan |\[D]valoa ja |rauhaa \[Gm]päälle |\[D]maan.
    Se |\[Em]joulu suo, mi |\[F#\textdegree7]onnen tuo ja |\[B7]mielet nostaa |\[Em]Luojan luo.
    Ei |\[A7]val\[D]taa \[A7]ei\[D]kä |\[Em]kultaa\[A7]kaan, vaan |\[D]rauhaa \[A7]päälle |\[D]maan.
  \endverse
  \notesoff
  \beginverse
    Suo |^mulle maja |^rauhai^sa ja |^lasten ^jou^lu|^puu,
    Ju|^malan ^sanan |^valoa, joss' |sieluin ^kirkas|^tuu.
    Tuo |^kotihin, nyt |^pieneenkin, nyt |^joulujuhla |^suloisin,
    Ju|^ma^lan ^sa^nan |^valo^a ja |^mieltä ^jalo|^a.
  \endverse
  \beginverse
    Luo |^köyhän niinkuin |^rikka^han saa, |^joulu ^i^ha|^na!
    Pi|^mey^tehen |^maailman tuo |taivaan ^valo|^a!
    Sua |^halajan, sua |^odotan, sä |^Herra maan ja |^taivahan.
    Nyt |^köy^hän ^niin^kuin |^rikkaan ^luo su|^loinen ^joulus |^tuo!
  \endverse
\endsong


\beginsong{Kosketa minua henki}[by={Ilkka Kuusisto 1979},ex={Virsi 125},ph={III},key={G},gk={G, E\flt{}--G}]
  % in Bb the notes range from D to D, in transposed G they range from B to B
  \transpose{-3} % to G (3)
  \meter{3}{4}
  \mnbeginverse
    |\[\mnc{D}B&]Kosketa |\[\mnc{F}Dm]minua, |\[\mnc{G}Gm]Hen|\[^\mn{D}]ki, |\[\mnc{E&}Cm7]kos\[^\mn{F}]ke\[^\mn{G}]ta |\[\mnc{A}Dm]kirk\[^\mn{F}]ka|\[\mnc{G}E&]us! |\[G7/D]
    |\[\mnc{G}Cm]An\[^\mn{B&}]na |\[\mncii{D}{C}F7]e\[^\mn{B&}]lä|\[\mncii{A}{F}B&maj7]mäl|\[\mnc{D}G]le |\[\mnc{G}Cm7]suun\[^\mn{F}]ta \[^\mn{G}]ja |\[\mnc{F}F7]tar\[^\mn{E&}]koi|\[\mnc{D}B&]tus. | \e
  \mnendverse
  \notesoff
  \beginverse
    |^Kosketa, |^Jumalan |^Hen|ki, |^syvälle |^sydä|^meen. |^
    |^Sinne |^paina |^hil|^jaa |^luottamus |^rakkau|^teen. | \e
  \endverse
  \beginverse
    |^Rohkaise |^minua, |^Hen|ki, |^murenna |^pelko|^ni. |^
    |^Tässä |^maail|^mas|^sa |^osoita |^paikka|^ni. | \e
  \endverse
  \beginverse
    |^Valaise, |^Jumalan |^Hen|ki, |^silmäni |^aukai|^se, |^
    |^että |^voisin |^ol|^la |^ystävä |^toisil|^le. | \e
  \endverse
  \beginverse
    |^Kosketa |^minua, |^Hen|ki! |^Herätä |^kiittä|^mään, |^
    |^sinun |^lähel|^lä|^si |^armosta |^elä|^mään. | \e
  \endverse
  \beginverse % for alt key (F) chords only
    \altchords{\id[1]{(F)}|F |Am |Dm | - |Gm7 |Am |B\flt{} |D7/A}
    \altchords{|Gm |C7 |Fmaj7 |D |Gm7 |C7 |F | \e}
  \endverse
\endsong


\beginsong{Mörri-Möykky}[by={Marjatta Pokela},ph={IV}]
  \newchords{chords_morrimoykky_a}\newchords{chords_morrimoykky_b}
  \beginverse\memorize[chords_morrimoykky_a]
    |\[\mnc{B}Em]Kor\[^\mn{A}]pi\[^\mn{G}]kuusen |\[\mnc{F#}Am]kannon \[\mnc{E}Em]alla on |\[\mnc{F#}B7]Mörri-\[^\mn{B}]Möy\[^\mn{D#}]kyn |\[\mnc{E}Em]kolo
  \endverse\glueverses\beginchorus\memorize[chords_morrimoykky_b]
    |\[Am]Siellä on koti ja |\[Em]siellä on peti
    ja |\[B7]peikolla pehmoinen |\[\up{1}Em\up{2}(E)]olo
  \endchorus
  \notesoff
  \beginverse
    \ind |\[E]Tiu tau tiu tau |tili tali tittan
    \ind |Sirkat soittaa |\[B]salolla
  \endverse\glueverses\beginchorus
    \ind |\[A]Pikkuiset peikot ne |\[E]piilossa pysyy
    \ind |\[B7]kirkkaalla päivän |\[E]valolla
  \endchorus\glueverses\beginverse
    \ind |\[Am] |\[Em] |\[B7] |\[Em]
  \endverse
  \beginverse\replay[chords_morrimoykky_a]
    |^Syksyn tullen |^sieniä ^kasvaa |^karhunkanka|^halla
  \endverse\glueverses\beginchorus\replay[chords_morrimoykky_b]
    |^Mörri-Möykky se |^sateessa istuu
    |^kärpässienen |^alla \goto{Tiu tau tittan}
  \endchorus
  \beginverse\replay[chords_morrimoykky_a]
    |^Ottaisin minä |^Mörri-^Möykyn, |^jos vain kiinni |^saisin
  \endverse\glueverses\beginchorus\replay[chords_morrimoykky_b]
    |^Pieneen koriin |^pistäisin ja
    |^kotiin kuljet|^taisin \goto{Tiu tau tittan}
  \endchorus
  \beginverse\replay[chords_morrimoykky_a]
    Vaan |^eipä taida |^meidän ^äiti |^peikkolasta |^ottaa
  \endverse\glueverses\beginchorus\replay[chords_morrimoykky_b]
    |^Eikä se edes |^usko, että
    |^Mörri-Möykky on |^totta \goto{Tiu tau tittan}
  \endchorus
\endsong

%%%%%%%%%%%%%%%%%%%%%%%%%%%%%%%%%%%%%%%%%%%%%%%%%%%%%%%%%%%%%%%%%%%
%%% LATEST PRINTOUT CONTAINED THE SONGS ABOVE.                  %%%
%%%%%%%%%%%%%%%%%%%%%%%%%%%%%%%%%%%%%%%%%%%%%%%%%%%%%%%%%%%%%%%%%%%
%%% Please try to not change the song numbers above this point. %%%
%%% Add new songs only after this point.                        %%%
%%%%%%%%%%%%%%%%%%%%%%%%%%%%%%%%%%%%%%%%%%%%%%%%%%%%%%%%%%%%%%%%%%%


\beginsong{Rakastan sinua elämä}[by={Eduard Komanovski, Konstantin Vanshenkin}, key={Am}, gk={Am, (Am)--(Cm)}]
  % Finnish words by: Pauli Salonen
  \newchords{chords_rakastan_elamaa_a}\newchords{chords_rakastan_elamaa_b}
  \mnbeginverse\memorize[chords_rakastan_elamaa_a]
    \[^\mn{E}]Päättyy |\[\mnc{A}Am]yö, \[\bm] \[^\mn{C}]aa\[^\mn{A}]mu |\[\mnc{F}Dm]saa,\[\bm] uusi |\[\mnc{E}E7]päi\[^\mn{D}]vä \[^\mn{C}]kun \[\bmc\mn{B}]kirk\[^\mn{A}]kaa\[^\mn{B}]na |\[\mnc{C}Am]lois\[^\mn{A}]taa.\[\bm]
    \[^\mn{E}]Sulle |\[^\mn{A}]oi, \[\bm] \[^\mn{C}]ko\[^\mn{A}]ti|\[\mnc{A}F]maa, \[\bm] sävel |\[\mnc{G}G7]kau\[^\mn{F}]ne\[^\mn{E}]hin \[\bmc\mn{D}]tuu\[^\mn{C}]les\[^\mn{D}]sa |\[\mnc{F}C]soit\[^\mn{E}]taa.\[\bm]
    \mnendverse\glueverses\mnbeginchorus\memorize[chords_rakastan_elamaa_b]
    \[^\mn{E}]Rakas|\[\mnc{B}E7]tan\[\bm] \[^\mn{A}]e\[^\mn{G#}]lä|\[\mnc{A}F]mää, \[Am\mn{E}]{{ }{ }{ }{ }{ }{ }jo}\[^\mn{F}]ka |\[\mnc{G}Gm6]uute\[^\mn{A}]na \[\mnc{G}A7]aa\[^\mn{F}]mus\[^\mn{E}]sa |\[\mnc{A}Dm]au\[^\mn{D}]kee.\[\bm]
    \[^\mn{E}]Ra\[^\mn{D}]kas|\[\mnc{F}Dm6]tan \[E7\mn{E}]{ }{ }{ }{ }{e}\[^\mn{D}]lä|\[\mnc{C}Am]mää, \[D7\mn{D}]{{ }{ }{ }{ }{ }jo}\[^\mn{B}]ka |\[\mnc{E}Am]uu\[^\mn{C}]pu\[^\mn{A}]en \[\mnc{E}E7]il\[^\mn{C}]las\[^\mn{B}]sa |\[\mnc{C}Am]rau\[^\mn{A}]kee.\[\bm]
  \mnendchorus
  \notesoff
  \beginverse\replay[chords_rakastan_elamaa_a]
    Kirkka|^hin ^ päivä |^ei, ^ aina |^parhainta ^loistetta |^suone.^
    Unel|man ^ usein |^vei, ^ eikä |^ystävä ^lohtua |^tuone.^
    \endverse\glueverses\beginchorus\replay[chords_rakastan_elamaa_b]
    Rakas|^tan ^ elä|^mää, ^{{ }{ }{ }{ }{ }jo}ka |^kyynelten ^helminä |^hohtaa.^
    Rakas|^tan ^ elä|^mää, ^{{ }{ }{ }{ }jo}ka |^myrskyihin ^tietäni |^johtaa.^
  \endchorus
  \beginverse\replay[chords_rakastan_elamaa_a]
    Jäänyt |^on ^ päivän |^työ, ^ ilta |^varjoja ^tielleni |^siirtää.^
    Kaupun|gin ^ sydän |^lyö, ^ valot |^laineille ^siltoja |^piirtää.^
    \endverse\glueverses\beginverse\replay[chords_rakastan_elamaa_b]
    Rakas|^tan ^ elä|^mää, ^{{ }{ }{ }{ }{ }jo}ka |^nuoruuden ^haaveita |^kantaa.^
    Rakas|^tan ^ elä|^mää, ^{{ }{ }{ }{ }jo}ka |^muistojen ^hetkiä |^antaa.^ \replay[chords_rakastan_elamaa_b]
    Rakas|^tan ^ elä|^mää, ^{{ }{ }{ }{ }{ }sil}le |^lempeni ^tahdon mä |^antaa.^
    Rakas|^tan ^ elä|^mää, ^{{ }{ }{ }{ }jo}ka |^muistojen ^hetkiä |^kantaa.^
  \endverse
%   %% Lilypond commented out for space saving reasons in the main songbook
%   \begin{lilywrap}\begin{lilypond}[]
%     % transcribed by larva, latest update on 2023-06
%     \include "tex/lp-include-head.ly"
%     theMelody = \relative c' {
%       \key a \minor \time 4/4 \partial 4
%       \set melismaBusyProperties = #'() \slurDashed
%       e8. \sectionmark "A" e16
%       | a2. c8. a16 | f'2. 8. 16 | e4 d8. c16 b4 a8. b16 | c4 a4 r4
%       e8. e16
%       | a2. c8. a16 | a'2. 8. 16 | g4 f8. e16 d4 c8. d16 | f4 e4 r4 \break
%       \repeat volta 2 {
%         e8 \sectionmark "B" e8
%         | b'2. a8 gis8 | a2. e8 f8 | g4 8. a16 g4 f8. e16 | a4 d,4 r4
%         e8 d8
%         | f2. e8 d8 | c2. d8 b8
%         | e4 c8 a8 e4 c'8. b16 | c4 a4 r4
%       }
%     }
%     theLyricsOne = \lyricmode {
%       \set stanza = "1.A"
%       Päät -- tyy | yö, aa -- mu | saa, uu -- si | päi -- vä kun kirk -- kaa -- na | lois -- taa.
%       Sul -- le | oi, ko -- ti -- | maa, sä -- vel | kau -- ne -- hin tuu -- les -- sa | soit -- taa.
%       \repeat volta 2 {
%         \set stanza = "1.B"
%         Ra -- kas -- | tan e -- lä -- | mää, jo -- ka | uu -- te -- na aa -- mus -- sa | au -- kee.
%         Ra -- kas -- | tan e -- lä -- | mää, jo -- ka | uu -- pu -- en il -- las -- sa | rau -- kee.
%       }
%     }
%     theLyricsTwo = \lyricmode {
%       \set stanza = "2.A"
%       Kirk -- ka -- | hin päi -- vä | ei, ai -- na | par -- hain -- ta lois -- tet -- ta | suo -- ne.
%       U -- nel -- | man u -- sein | vei, ei -- kä | ys -- tä -- vä loh -- tu -- a | tuo -- ne.
%       \repeat volta 2 {
%         \set stanza = "2.B"
%         Ra -- kas -- | tan e -- lä -- | mää, jo -- ka | kyy -- nel -- ten hel -- mi -- nä | hoh -- taa.
%         Ra -- kas -- | tan e -- lä -- | mää, jo -- ka | myrs -- kyi -- hin tie -- tä -- ni | joh -- taa.
%       }
%     }
%     theLyricsThree = \lyricmode {
%       \set stanza = "3.A"
%       Jää -- nyt | on päi -- vän | työ, il -- ta | var -- jo -- ja tiel -- le -- ni | siir -- tää.
%       Kau -- pun -- | gin sy -- dän | lyö, va -- lot | lai -- neil -- le sil -- to -- ja | piir -- tää.
%       \repeat volta 2 {
%         <<
%           {
%             \set stanza = "3.B:i"
%             Ra -- kas -- | tan e -- lä -- | mää, jo -- ka | nuo -- ruu -- den haa -- vei -- ta | kan -- taa.
%             Ra -- kas -- | tan e -- lä -- | mää, jo -- ka | muis -- to -- jen het -- ki -- ä | an -- taa.
%           }
%           \new Lyrics { \set associatedVoice = "theMelody"
%             \set stanza = "3.B:ii"
%             \override LyricText.color = #grey Ra -- kas -- | tan e -- lä -- | mää, \override LyricText.color = #black sil -- le | lem -- pe -- ni tah -- don mä | an -- taa.
%             \override LyricText.color = #grey Ra -- kas -- | tan e -- lä -- | mää, jo -- ka | muis -- to -- jen het -- ki -- ä \override LyricText.color = #black | kan -- taa.
%           }
%         >>
%       }
%     }
%     theChords = \chordmode {
%       s4
%       | a1:m | d:m | e:7 | a:m
%       | a:m | f | g:7 | c2.
%       \repeat volta 2 {
%         s4
%         | e1:7 | f2 a2:m | g2:m6 a2:7 | d1:m
%         | d2:m6 e2:7 | a2:m7 d2:7 | a2:m e2:7 | a2.:m
%       }
%     }
%     %\layout { #(layout-set-staff-size 15) } % for better fit
%     \include "tex/lp-include-tail-lyricsbelow.ly"
%   \end{lilypond}\end{lilywrap}
\endsong

\beginsong{Tuuditan tulisoroista}[by={trad.}, key={Em}, gk={Gm, Gm--Bm}, tags={uni}]
  \audio[key=B\flt{}m]{https://www.youtube.com/watch?v=mpHEc8SjfZc}
  \newchords{chords_tulisoroinen_a}\newchords{chords_tulisoroinen_b}
  \capo{3}
  \mnbeginverse\memorize[chords_tulisoroinen_a]
    |\[\mnc{B}Em]Tuuditan \[^\mn{A}]tu|\[^\mn{G}]li\[^\mn{B}]so\[^\mn{E}]rois\[^\mn{F#}]ta, |\[^\mn{G}]kipunaista \[\mnc{A}Am]kiikut|\[\mnc{B}Em]telen \altchords{\id[1]{Am}|Am | - | - Dm |Am}
    \mnendverse\glueverses\mnbeginchorus\memorize[chords_tulisoroinen_b]
    |\[\mnc{B}Em]Vaa\[^\mn{E}]lin \[^\mn{B}]pien\[^\mn{A}]tä |\[^\mn{G}]val\[^\mn{B}]ke\[^\mn{E}]ais\[^\mn{F#}]ta, |\[\mnc{G}G]luojan lasta \[\mnc{F#}Bm]liikut|\[\mnc{E}Em]telen \altchords{|Am | - |C Em |Am}
  \mnendchorus
  \notesoff
  \beginverse\replay[chords_tulisoroinen_a]
    |^Taivahast' on |lahja tullut, |taivahan tu^len ki|^soista
    \endverse\glueverses\beginchorus\replay[chords_tulisoroinen_b]
    |^Luojan lemmen |leikinnöistä, |^ei vahingon ^valke|^oista
  \endchorus
  \beginverse\replay[chords_tulisoroinen_a]
    |^Tuudin lasta |maan valoksi, |en vahingon ^valke|^aksi
    \endverse\glueverses\beginchorus\replay[chords_tulisoroinen_b]
    |^Tuudin toivo|jen tuleksi, |^työn jaloisen ^jatka|^jaksi
  \endchorus
  \beginverse\replay[chords_tulisoroinen_a]
    |^Senpä tuudin |tuikkeheksi, |tähdeksi tä^hän ta|^lohon
    \endverse\glueverses\beginchorus\replay[chords_tulisoroinen_b]
    |^Maan hyväksi |maineheksi, |^suurionnis^ten i|^lohon
  \endchorus
  \begin{lilywrap}\begin{lilypond}[]
    % transcribed by larva, latest update on 2023-07
    \include "tex/lp-include-head.ly"
    theMelody = \relative b' {
      \key e \minor \time 4/4
      \set melismaBusyProperties = #'() \slurDashed
      | b4 b b a | g b e,4. fis8 | g8 g g g a4 a | b2 b2
      \repeat volta 2 {
        | b4 e b4. a8 | g4 b e,4. fis8 | g8 g g g fis4 fis | e2 e2
      }
    }
    theLyricsOne = \lyricmode {
      \set stanza = "1."
      | Tuu -- di -- tan tu -- | li -- so -- rois -- ta, | ki -- pu -- nais -- ta kii -- kut -- | te -- len;
      \repeat volta 2 {
        | Vaa -- lin pien -- tä | val -- ke -- ais -- ta, | luo -- jan las -- ta lii -- kut -- | te -- len.
      }
    }
    theLyricsTwo = \lyricmode {
      \set stanza = "2."
      | Tai -- va -- hast' on | lah -- ja tul -- lut, | tai -- va -- han tu -- len ki -- | sois -- ta;
      \repeat volta 2 {
        | Luo -- jan lem -- men | lei -- kin -- nöis -- tä, | ei va -- hin -- gon val -- ke -- | ois -- ta.
      }
    }
    theLyricsThree = \lyricmode {
      \set stanza = "3."
      | Tuu -- din las -- ta | maan va -- lok -- si, | en va -- hin -- gon val -- ke -- | ak -- si;
      \repeat volta 2 {
        | Tuu -- din toi -- vo -- | jen tu -- lek -- si, | työn ja -- loi -- sen jat -- ka -- | jak -- si.
      }
    }
    theLyricsFour = \lyricmode {
      \set stanza = "4."
      | Sen -- pä tuu -- din | tuik -- ke -- hek -- si, | täh -- dek -- si tä -- hän ta -- | lo -- hon;
      \repeat volta 2 {
        | Maan hy -- väk -- si | mai -- ne -- hek -- si, | suu -- ri -- on -- nis -- ten i -- | lo -- hon.
      }
    }
    theChords = \chordmode {
      | e1:m | e:m | e2:m a2:m | e1:m
      \repeat volta 2 {
        | e1:m | e:m | g2 b2:m | e1:m
      }
    }
    %\layout { #(layout-set-staff-size 15) } % for better fit
    \include "tex/lp-include-tail-notab.ly"
  \end{lilypond}\end{lilywrap}
\endsong


\beginsong{Lemminkäisen äiti}[index={Tuima on tuuli}by={Oskar Merikanto, Eino Leino}, key={Am}, gk={Cm, Cm--Em}]
  \capo{3}
  \mnbeginverse
    |\[\mnc{E}Am]Tuima \[^\mn{A}]on \[\bmc\mn{C}]tuuli ja |\[\mnc{B}E]pimeä \[^\mn{A}]on \[\bmc\mn{B}]tai\[^\mn{E}]vo, \altchords{\id[1]{(Cm)}|Cm |G}
    |\[\mnc{A}Am]suuri \[^\mn{C}]on \[\mnc{D}G]ula\[^\mn{B}]pal\[^\mn{D}]la |\[\mnc{E}E]aalto\[^\mn{D}]jen \[\mnc{E}E7]rai\[^\mn{D}]vo, \altchords{|Cm B\flt{} |G G7}
    |\[\mnc{C}C]lahti on \[\mnc{B}G]tyy\[^\mn{G}]ni \[^\mn{B}]ja |\[\mnc{A}Am]selkeä \[\bmc\mn{E}]vaan. \altchords{|E\flt{} B\flt{} |Cm}
    |\[\mnc{E}C]Kus\[^\mn{C}]sa mun \[\mnc{B}E7]kot\[^\mn{E}]ka\[^\mn{G#}]ni |\[\mnc{A}Am]kulkee\[\bm]kaan? \altchords{|E\flt{} G7 |Cm}
  \mnendverse
  \notesoff
  \beginverse
    |^Joudu jo ^kotihin ja |^lentosi ^heitä!
    |^Taikka jo ^ajeletkin |^aaltojen ^teitä,
    |^poikani ^pieni ja |^hentoi^nen.
    |^Lahti on ^tyyni ja |^rauhai^nen.
  \endverse
  \beginverse
    |^Ulkona ^ulapalla |^myrskyt ne ^pauhaa, \altchords{\id[2]{(Dm)}|Dm |A}
    |^täällä on ^lämmintä ja |^täällä on ^lauhaa, \altchords{|Dm C |A A7}
    |^lahti on ^tyyni ja |^selkeä ^vaan. \altchords{|F C |Dm }
    |^Laske jo ^lahtesi |^valka^maan! \altchords{|F A7 |Dm}
  \endverse
  \begin{lilywrap}\begin{lilypond}[]
    % transcribed by larva, latest update on 2023-07
    \include "tex/lp-include-head.ly"
    theMelody = \relative e' {
      \key a \minor \time 4/4
      \set melismaBusyProperties = #'() \slurDashed
      | e4 e8 a c8( c) c c | b( b) b a b4 e,4
      | a4 a8 c d d b d | e4 e8 d e4 d4
      | c4 c8 c b4 g8 b | a4 a8( a) e2
      | e'4 c8 c b4 e,8 gis | a4 a a2 \bar "|."
    }
    theLyricsOne = \lyricmode {
      \set stanza = "1."
      | Tui -- ma on tuu -- _ li ja | pi -- me -- ä on tai -- vo,
      | suu -- ri on u -- la -- pal -- la | aal -- to -- jen rai -- vo,
      | lah -- ti on tyy -- ni ja | sel -- ke -- ä vaan.
      | Kus -- sa mun kot -- ka -- ni | kul -- kee -- kaan?
    }
    theLyricsTwo = \lyricmode {
      \set stanza = "2."
      | Jou -- du jo ko -- ti -- hin ja | len -- _ to -- si hei -- tä!
      | Taik -- ka jo a -- je -- let -- kin | aal -- to -- jen tei -- tä,
      | poi -- ka -- ni pie -- ni ja | hen -- toi -- _ nen.
      | Lah -- ti on tyy -- ni ja | rau -- hai -- nen.
    }
    theLyricsThree = \lyricmode {
      \set stanza = "3."
      | Ul -- ko -- na u -- la -- pal -- la | myrs -- _ kyt ne pau -- haa,
      | tääl -- lä on läm -- min -- tä ja | tääl -- lä on lau -- haa,
      | lah -- ti on tyy -- ni ja | sel -- ke -- ä vaan.
      | Las -- ke jo lah -- te -- si | val -- ka -- maan!
    }
    theChords = \chordmode {
      | a2:m a2:m | e e
      | a:m g | e e:7
      | c g | a:m a:m
      | c e:7 | a:m a:m
    }
    %\layout { #(layout-set-staff-size 13) } % for better fit
    \include "tex/lp-include-tail-notab-nolyrics.ly"
  \end{lilypond}\end{lilywrap}
\endsong


\beginsong{Katselin taivahan tähtiä}[by={trad.}, key={Em}, gk={Gm, Gm--Am}]
  \audio[key=Gm]{https://www.youtube.com/watch?v=efchOgWBFtI}
  \capo{3}
  \newchords{chords_katselintaivahan_a}\newchords{chords_katselintaivahan_b}
  \mnbeginchorus\memorize[chords_katselintaivahan_a]
    |\[\mnc{B}Em]Kat\[^\mn{E}]selin \[\bm]tai\[^\mn{D}]va\[^\mn{E}]han |\[\mnc{F#}D]tähti\[\bmc\mn{D}]ä, ja |\[\mnc{C}Am]tuli mi\[^\mn{B}]nun \[\bmc\mn{C}]i\[^\mn{A}]kä|\[\mnc{B}B]vä\[\bm]ni
    \mnendchorus\glueverses\mnbeginchorus\memorize[chords_katselintaivahan_b]
    \[^\mn{B}]Kun |\[\mnc{C}Am]heit' o\[^\mn{B}]li \[\bmc\mn{A}]aina |\[\mnc{B}Em]kaksi \[^\mn{A}]ja \[\bmc\mn{G}]kaksi, \[^\mn{F#}]ja |\[\mnc{A}Am]minä o\[^\mn{F#}]lin \[\mnc{B}B7]yk\[^\mn{F#}]si|\[\mnc{E}Em]nä\[\bm]ni
  \mnendchorus
  \notesoff
  \beginchorus\replay[chords_katselintaivahan_a]
    |^Oi jos ^saisin sen |^omak^seni, jok' on |^keskellä ^sydän|^tä^ni
    \endchorus\glueverses\beginchorus\replay[chords_katselintaivahan_b]
    En |^surisi mä ^köyhää |^koti^ani, enkä |^orjan ^päivi|^ä^ni
  \endchorus
  \beginchorus\replay[chords_katselintaivahan_a]
    |^Niin minä ^olen |^yksi^näni, kuin |^kataja ^muista |^puis^ta
    \endchorus\glueverses\beginchorus\replay[chords_katselintaivahan_b]
    |^Ei ole ^uutta |^ei ole ^vanhaa, |^ei ole ^omi|^tuis^ta
  \endchorus
  \begin{lilywrap}\begin{lilypond}[]
    % transcribed by larva, latest update on 2023-07
    \include "tex/lp-include-head.ly"
    theMelody = \relative b' {
      \key e \minor \time 4/4 \partial 8
      \set melismaBusyProperties = #'() \slurDashed
      \repeat volta 2 {
        r8 | b4 e8( e8) e4 d8( e8) | fis4 fis d8( d8)
        d8( d8) | c8( c) c( b) c4 a | b2 b4.
      }
      \repeat volta 2 {
        \parenthesize b8 | c8( c8) c8 b a4 a | b4 b8( a8) g8( g8)
        g8( fis8) | a8( a) a( fis) b4 fis4 | e2 e4.
      }
    }
    theLyricsOne = \lyricmode {
      \set stanza = "1.A"
      \repeat volta 2 {
        | Kat -- se -- lin tai -- va -- han | täh -- ti -- ä __ _
        ja __ _ | tu -- li mi -- nun i -- kä -- | vä -- ni.
      }
      \set stanza = "1.B"
      \repeat volta 2 {
        Kun | heit' __ _ o -- li ai -- na | kak -- si ja kak -- _ si
        ja | mi -- nä o -- lin yk -- si -- | nä -- ni.
      }
    }
    theLyricsTwo = \lyricmode {
      \set stanza = "2.A"
      \repeat volta 2 {
        | Oi jos __ _ sai -- sin sen | o -- mak -- se -- ni
        jok' on | kes -- _ kel -- lä sy -- dän -- | tä -- ni.
      }
      \set stanza = "2.B"
      \repeat volta 2 {
        En | su -- ri -- si mä köy -- hää | ko -- ti -- _ a -- ni
        en -- kä | or -- _ jan __ _ päi -- vi -- | ä -- ni.
      }
    }
    theLyricsThree = \lyricmode {
      \set stanza = "3.A"
      \repeat volta 2 {
        | Niin mi -- nä o -- len __ _ | yk -- si -- nä -- ni
        kuin __ _ | ka -- ta -- ja __ _ muis -- ta | puis -- ta.
      }
      \set stanza = "3.B"
      \repeat volta 2 {
        \skip 1 | Ei __ _ o -- le uut -- ta | ei o -- le van -- _ haa __ _
        | Ei __ _ o -- le o -- mi -- | tuis -- ta.
      }
    }
    theChords = \chordmode {
      \repeat volta 2 {
        s8 | e1:m | d | a:m | b,2.~8
      }
      \repeat volta 2 {
        s8 | a1:m | e:m | a2:m b,:7 | e2.:m~8
      }
    }
    %\layout { #(layout-set-staff-size 15) } % for better fit
    \include "tex/lp-include-tail-notab.ly"
  \end{lilypond}\end{lilywrap}
\endsong


\beginsong{Kettu itki poikiansa}[by={trad.}, key={Am}, gk={Am, Gm--Em}]
  \audio[key=Dm]{https://www.youtube.com/watch?v=rWARJsJewC8}
  \meter{5}{4}
  \mnbeginverse
    |\[\mnc{A}Am]Ket\[^\mn{B}]tu \[^\mn{C}]it\[^\mn{D}]ki \[^\mn{E}]poi\[^\mn{C}]ki\[\mnc{B}E]ansa, |\[E7]ki\[^\mn{D}]ven \[^\mn{C}]päällä \[^\mn{B}]kyykyl\[\mnc{E}Am]länsä.
    |\[^\mn{A}]Ka\[^\mn{B}]hen \[^\mn{C}]kau\[^\mn{D}]pun\[^\mn{E}]gin \[^\mn{C}]vä\[\mnc{B}E]lillä, |\[E7]ka\[^\mn{D}]hen \[^\mn{C}]kirkon \[^\mn{B}]kuulu\[\mnc{A}Am]villa.
  \mnendverse
  \notesoff
  \beginverse
    |^Kun ei poikani me^nisi, |^pitkähännät piipot^taisi.
    |Noille vieraille pi^hoille, |^kaupunkien karta^noille.
  \endverse
  \beginverse
    |^Siellä nuoli noppa^jaisi, |^tinapalli paisko^aisi. \altchords{\id[1]{(Dm)}|Dm A |A7 Dm}
    |Siellä surma suin pi^täisi, |^kova kuolo kohta^jaisi. \altchords{| - A |A7 Dm}
  \endverse
  % Original image downloaded from: https://cdn.pixabay.com/photo/2014/04/03/11/48/fox-312203_960_720.png
  % Image license: Public Domain
  % Edited by: larva
  \begin{lilywrap}\imagel[4]{fox_drawing_ed_by_larva__transbg_CC0_900x1200.png}\par\begin{lilypond}[]
    % transcribed by larva, latest update on 2023-07
    \include "tex/lp-include-head.ly"
    theMelody = \relative a' {
      \key a \minor \time 5/4
      \set melismaBusyProperties = #'() \slurDashed
      | a8. b16 c8 d e c b4 b | b8 d c c b b e4 e
      | a,8. b16 c8 d e c b4 b | b8 d c c b b a4 a \bar "|."
    }
    theLyricsOne = \lyricmode {
      \set stanza = "1."
      | Ket -- tu it -- ki poi -- ki -- an -- sa,
      | ki -- ven pääl -- lä kyy -- kyl -- län -- sä.
      | Ka -- hen kau -- pun -- gin vä -- lil -- lä,
      | ka -- hen kir -- kon kuu -- lu -- vil -- la.
    }
    theLyricsTwo = \lyricmode {
      \set stanza = "2."
      | Kun ei poi -- ka -- ni me -- ni -- si,
      | pit -- kä -- hän -- nät pii -- pot -- tai -- si.
      | Noil -- le vie -- rail -- le pi -- hoil -- le,
      | kau -- pun -- ki -- en kar -- ta -- noil -- le.
    }
    theLyricsThree = \lyricmode {
      \set stanza = "3."
      | Siel -- lä nuo -- li nop -- pa -- jai -- si,
      | ti -- na -- pal -- li pais -- ko -- ai -- si.
      | Siel -- lä sur -- ma suin pi -- täi -- si,
      | ko -- va kuo -- lo koh -- ta -- jai -- si.
    }
    theChords = \chordmode {
      | a2.:m e2 | e2.:7 a2:m
      | a2.:m e2 | e2.:7 a2:m
    }
    %\layout { #(layout-set-staff-size 15) } % for better fit
    \include "tex/lp-include-tail-notab.ly"
  \end{lilypond}\end{lilywrap}
  \noendsongvfill % align lp to the bottom
\endsong


% \beginsong{O Kriste, kunnian kuningas}
%   \audio[key=Am]{https://www.youtube.com/watch?v=P_elenZRgAA}
%   \newchords{chords_okriste_a}\newchords{chords_okriste_b}
%   \meter{3}{4}
%   \beginverse\memorize[chords_okriste_a]
%     O |\[Am]Kriste, |\[C]kunnian |\[Dm]Kunin|\[Am]gas,
%     Ja |\[Dm]lunas|\[Am]taja |\[Em]laupi|\[Am]as!
%     Kuu|\[Am]le si|\[C]nua |\[Dm]rukoi|\[Am]len,
%     Ve|\[Dm]relläs’ |\[Am]ostet|\[Em]tu o|\[Am]len.
%   \endverse
%   \beginverse\memorize[chords_okriste_b]
%     \ind Suu|\[C]ret syn|\[F]tin’ kyl|\[G]läs tien|\[C]net,
%     \ind Joi|\[Dm]ta vas|\[Am]taas teh|\[Em]nyt lie|\[Am]nen,
%     \ind Koht’ |\[Am]äitin’ |\[C]kohdust’ |\[Dm]tultu|\[Am]an’,
%     \ind Ah |\[Dm]armahda |\[Am]päällen’, |\[Em]Juma|\[Am]la!
%   \endverse
%   \beginverse\replay[chords_okriste_a]
%     |^Muista, |^Herra, va|^las pääl|^len,
%     jon|^ka vah|^vast’ van|^noit meil|^len:
%     Et|^tes suo |^syntist’ |^hukku|^van,
%     vaan |^katu|^van ja |^elä|^vän.
%   \endverse
%   \beginverse\replay[chords_okriste_b]
%     \ind Tun|^nustan, |^synti|^nen o|^len,
%     \ind Huo|^kaan, ka|^dun ty|^kös tu|^len,
%     \ind Ja |^anteeks’ |^anta|^vas toi|^von,
%     \ind Mi|^tä vas|^taas ri|^koin vai|^voin.
%   \endverse
%   \beginverse\replay[chords_okriste_a]
%     Ku|^ningas |^kaikki|^valti|^as,
%     O |^Jesu |^aina |^laupi|^as!
%     Kuu|^le mi|^nua |^vaivais|^ta!
%     Ru|^koilen |^sinua |^hartaas|^ta,
%   \endverse
%   \beginverse\replay[chords_okriste_b]
%     \ind Ken |^paits’ |^sua minu|^a kuu|^lee?
%     \ind Ken |^apuun |^paits’ |^sinua tu|^lee?
%     \ind Jos et |^kuule, |^auta |^minu|^a,
%     \ind Ei |^muilta |^ole |^apu|^a.
%   \endverse
%   \beginverse\replay[chords_okriste_a]
%     Ol|^koon si|^nun, Je|^su kii|^tos!
%     Ai|^na ja |^ijät’ |^ylis|^tys,
%     Kuin |^kaikkein |^päälle |^armah|^dat,
%     Jotk’ |^tykös |^hartaast’ |^huuta|^vat.
%   \endverse
%   \beginverse\replay[chords_okriste_b]
%     \ind Se |^sama |^kunnia |^Isäl|^le,
%     \ind Ja |^ynnä |^Pyhäll’ |^Hengel|^le,
%     \ind Yh|^dell’ kol|^minai|^sell’ Her|^rall’,
%     \ind I|^jäisest’ |^aina |^hallitse|^vall’.
%   \endverse
%   \beginverse\replay[chords_okriste_a]
%     Ó |^Cristo, |^Rei da |^Glóri|^a
%     e |^misericordi|^oso |^Reden|^tor!
%     Es|^cute-me |^por |^fa|^vor
%     Fui |^comprado |^com seu |^san|^gue.
%   \endverse
%   \beginverse\replay[chords_okriste_b]
%     \ind Vo|^cê con|^hece meus |^grandes peca|^dos,
%     \ind que eu |^come|^ti con|^tra vo|^cê
%     \ind des|^de que |^saí do ú|^tero da minha |^mãe.
%     \ind Ah, |^tem misericór|^dia de |^mim, ó De|^us!
%   \endverse
%   \beginverse\replay[chords_okriste_a]
%     Lem|^bre-se, |^ó |^Se|^nhor,
%     o |^pacto |^que você |^fez por |^nós:
%     Que |^você não |^vai deixar um |^pecador mo|^rrer,
%     mas |^se ele se a|^rrepender, e|^le vive|^rá.
%   \endverse
%   \beginverse\replay[chords_okriste_b]
%     \ind Eu |^con|^fesso que sou |^um peca|^dor
%     \ind e |^eu suspiro, eu |^me arrependo |^e vou para vo|^cê
%     \ind pa|^ra buscar |^o seu |^perdã|^o,
%     \ind con|^tra o qual |^eu trans|^gre|^di.
%   \endverse
%   \beginverse\replay[chords_okriste_a]
%     Ó Rei Todo-Poderoso,
%     Jesus sempre misericordioso!
%     Ouça este pobre homem!
%     Eu te imploro com fervor,
%   \endverse
%   \beginverse\replay[chords_okriste_b]
%     \ind Quem mais pode me ouvir?
%     \ind Quem mais pode me ajudar?
%     \ind Se você não me ouvir e me ajudar,
%     \ind Ninguém mais pode me ajudar.
%   \endverse
%   \beginverse\replay[chords_okriste_a]
%     Louvado seja você, Jesus!
%     Louvado seja para todo o sempre
%     porque você é misericordioso
%     para todos aqueles que clamam por você.
%   \endverse
%   \beginverse\replay[chords_okriste_b]
%     \ind E glória ao Pai
%     \ind e ao Espírito Santo,
%     \ind para o Deus triuno
%     \ind que reina para sempre.
%   \endverse
% \endsong

% % Commented out, as not needed for page fill currently
% \begin{intersong}
%   % Original image downloaded from: https://commons.wikimedia.org/wiki/File:Harmonic_series_to_32.png
%   % Image license: Public Domain
%   % Edited by: larva
%   \imagecc[0]{harmonic_series_to_32_PD__1641x1299px.png}
%   % Harmonic series (sum):
%   \begin{center}%
%     $$\sum_{n=1}^{\infty} \frac{1}{n}$$
%     \vfill
%   \end{center}
% \end{intersong}

    % Scribblings by larva (perhaps to be removed later)
% ==================================================
%
% The following sets the song number for the first song in this file.
% The number will automatically be incremented by one for each song.
% Please do not change this! Changing would make different versions of
% the songbook to have different numbers for the same songs, and it
% would totally mess up the selection booklets causing them to have
% wrong songs in them. (For the same reason, add new songs only to the
% end of each songs_ file.)
\setcounter{songnum}{670}


\beginsong{Puhdista}[by={larva}, tags={suitsutus, suojelus}, ph={I}, key={Bm}, sks={Bm, Bm--Cm}]
  \transpose{2}
  \beginchorus
    |\[\mnc{A}Am]Happi yhtyy |\[\mnc{C}Dm7]hii\[\mn{A}]leen, |\[\mnc{C}C]lämpö nousee | \e \altchords{\id[1]{(Am)}|Am |Dm7 |C | \e}
    |\[Am]Henki koskee |\[Dm7]ainetta, |\[F]tietoisuus \[G]koho|\[Am]aa \altchords{|Am |Dm7 |F G |Am}
  \endchorus
  \beginverse
    \ind |\[Am]Puhdis|\[Em]ta, |\[G]puhdis|\[Am]ta \altchords{|Am |Em |G |Am}
    \endverse\glueverses\beginchorus
    \ind |\[Dm]Savuna ilmaan, |\[Am]uhrilahja \altchords{|Dm |Am}
    \ind |\[G]Puhdis|\[Am]ta \altchords{|G |Am}
  \endchorus
  \beginverse
    \ind[2] |\[\mnc{G}Em]Kut\[\mn{E}]sun suojelusta, |\[\mn{B}]kut\[\mn{E}]sun suojelusta \altchords{|Em | \e}
    \ind[2] |\[Am]Paikka on pyhä | \e \altchords{|Am | \e}
    \ind[2] |\[Em]Kutsun suojelusta, |kutsun suojelusta \altchords{|Em | \e}
    \ind[2] |\[Am]Aika on pyhä | \e \altchords{|Am | \e}
  \endverse
  \beginverse
    \ind |\[\mnc{A}Am]Puh\[\mn{C}]dis|\[\mnc{E}Em]ta, |\[\mnc{G}G]puh\[\mn{A}]dis|\[\mnc{E}Am]ta \altchords{|Am |Em |G |Am}
    \endverse\glueverses\beginchorus
    \ind |\[Dm]Savuna ilmaan, |\[Am]uhrilahja \altchords{|Dm |Am}
    \ind |\[G]Puhdis|\[Am]ta \altchords{|G |Am}
  \endchorus
  \textnote{\emph{D.C. al Fine}}
  \beginchorus
    |\[\mncii{D}{C}G]Puh\[\mn{B}]dis|\[\mnc{A}Am]ta \altchords{|G |Am}
  \endchorus\glueverses\beginchorus
    |\[G]Pu- | -uhdis|\[Am]ta \altchords{|G | - |Am}
  \endchorus
\endsong


\beginsong{Matkustan}[by={larva},tags={sydän},ph={II}]
  \beginverse
    |\[\mnc{A}Am]Mat\[^\mn{C}]kus|\[Em]tan, |\[Am]matkus|\[Em]tan
    |\[Am]Mieleni |\[Em]sisään, |\[Dm]syvemmäl|\[Am]le
    |\[Am]Matkus|\[Em]tan, |\[Am]matkus|\[Em]tan
    |\[Am]Ajatusten |\[Em]taakse, |\[Dm]yti|\[Am]meen
  \endverse
  \beginverse
    |\[Am]Kaikenlaista |\[Dm]tulee vastaan;
    |\[C]mikä siitä \[Em]on tärke|\[Am]ää?
  \endverse
  \beginchorus
    \lrep |\[Am/E]Sydämeen voi |\[G]luottaa \rrep
    |\[Em]Siellä se \up{1}sisällä \up{2}(syvällä) |\[Am]on
  \endchorus
\endsong


\begin{intersong}
  \begin{feeler}
    ``An old alchemist gave the following consolation to one of his disciples: `No matter how
    isolated you are and how lonely you feel, if you do your work truly and conscientiously,
    unknown friends will come and seek you.'''\\
    --- \emph{Carl Jung} (1875--1961)
  \end{feeler}
  %\vfill
\end{intersong}


\beginsong{Avaruus aukeaa sisältä}[by={larva}, tags={sydän, avaruus}, ph={II, III}, key={Am}, sks={Am}]
  \transpose{5}
  \beginchorus
    \lrep |\[\mnc{E}Em]Joskus luulen \[\bm]olevani |\[\mnc{F#}D]jotain mitä \sublyr{\up{2}en}\up{1}\[\bmc\mn{G}]min' \sublyr{vaan}\[\mn{F#}]en |\[\mnc{E}Em]oo \[\bm]|{ } \[\bm]\e \rrep
    |\[Am]Kun sen \[\bm]huomaan niin |\[C]suuntaan ta\[\bmc Bm]kaisin ko|\[Em]tiin: \[\bm] |{ } \[\bm]\e
    |\[D]Sy-\[\bm]ydä|\[Em]meen, \[\bm] |\[D]ke-\[\bm]eskel|\[Em]le \[\bm]
    |\[D]Sy-\[\bm]y-|y-\[\bm]ydä|\[Em]meen \[\bm] |{ } \[\bm]\e
    |\[D]Ke-\[\bm]e-|e-\[\bm]eskel|\[Em]le \[\bm] |{ } \[\bm]\e
    \lrep |\[Am] \[\bmc C] |\[Bm] \[\bm] |\[Em] \[\bm] |{ } \[\bm]\e \rrep
  \endchorus
  \beginchorus
    \lrep |\[\mnc{G}G]A-\[\bmc\mn{B}]va|\[\mnc{C}C]ru-\[\bm]u-|\[\mnc{B}Em]uus \[\bm] |{ } \[\bm]\e \rrep
    |\[G]A-\[\bm]va|\[C]ruus \[\bm]auke|\[Am]aa \[\bm]keskel|\[Em]tä \[\bm]
    |\[D]Sy-\[\bm]yväl|\[Em]tä |\[D]si-\[\bm]isäl|\[Em]tä \[\bm]
    |\[D]Sy-\[\bm]y-|y-\[\bm]yväl|\[Em]tä \[\bm] |{ } \[\bm]\e
    |\[D]Si-\[\bm]i-|i-\[\bm]isäl|\[Em]tä \[\bm] |{ } \[\bm]\e
    \lrep |\[Am] \[\bmc C] |\[Bm] \[\bm] |\[Em] \[\bm] |{ } \[\bm]\e \rrep
  \endchorus
\endsong


\beginsong{Ajan takaa}[by={larva},tags={rakkaus, lähde}]
  \beginchorus
    \lrep |\[\bmc\mnc{E}Am]A-\[\bm] \[\mnc{D}Am7]jan |\[\bmc Em]takaa,\[\bm] |\[\bmc Dm]totu\[\bm]uden |\[\bmc Am]luota\[\bm] \rrep
    \lrep |\[\bmc Em]To-\[\bm] |\[\bm] \[\bm]otuu|\[\bmc Am]den \[\bm]aijjai|\[\bm]jajajaja\[\bm]jajajaja \rrep
  \endchorus
  \beginchorus
    \lrep |\[\bmc Am]A-\[\bm] \[Am7]jan |\[\bmc Em]takaa,\[\bm] |\[\bmc Dm]totu\[\bm]uden |\[\bmc Am]luota\[\bm] \rrep
    \lrep |\[\bmc Dm]Jospa saisin \[\bm]sieltä |\[\bmc G]mukaani \[\bm]palan |\[\bmc Am]rakkaut\[\bm]ta |\[\bm]\[\bm]\e
    |\[\bmc Em]Sitä kylväi\[\bm]sin |\[\bmc G] \[\bm]maail|\[\bmc Am]maan\[\bm] |\[\bm]\[\bm]\e \rrep
  \endchorus
  \imagecc[3]{bufo_alvarius_bw_transparent_bg_300x240px.png}
\endsong


\beginsong{Hetki}[by={larva},tags={kiitollisuus},ph={III, IV}]
  \beginchorus\memorize
    |\[\mnc{D}Dm]Mie\[^\mn{A}]li |\[Gm]kuljettaa |\[Am]mennee\[Am7]seen ja |\[Dm]tulevaan
    |\[Dm]Hetki |\[Gm]katoaa |\[Am]aja\[C]tusten |\[Dm]mukana
  \endchorus
  \notesoff
  \beginchorus
    |\[Dm]Oi kuinka \[C]kaipaan |\[Dm]niin
    |\[B&]Sitä mitä \[C]en osaa |\[Dm]saavuttaa
  \endchorus
  \beginchorus
    \up{1}(mutta) |\[C#]Kiitos kiitos kiitos kiitos |kiitos tästä hetkes|\[Dm]tä | \e
  \endchorus
  \beginchorus
    Se |^opettaa |^elämään |^elämää ^ |^tässä vaan
    |^Opettaa |^elämään |^e-^elä|^mää!
  \endchorus
\endsong


\beginsong{Minne olenkaan matkalla}[by={larva},ph={III}]
  \beginchorus
    |\[\mnc{A}Am]Minne mä \[\mn{E}]olenkaan |\[F]matkalla,
    |\[C]tiedä sitä |\[Em]en
  \endchorus
  \beginverse
    Mutta |\[C]tahdon |\[G]luottaa,
    sillä |\[Em]kaikkeus ihme |\[Am]on
  \endverse
  \beginchorus
    pada-\[Dm7]diida-diida-|Diida-diida-diida-diida-|Daida-
    \[Em]dam-pada-|Dam padadam-|Padam-\[Am]dam | \e \up{2}(| | \e)
  \endchorus
  \beginverse
    \ind |\[Dm]Sisältäni | löydän |\[Am]surua, | \e
    \ind |\[Dm]paljon | on myös |\[Am]iloa | \e
    \ind \lrep |\[C]Elämä |\[G]kaunis |\[Am]on | \e \rrep
  \endverse
  \beginverse
    \ind |\[Dm]Ympärilläni | kohtaan |\[Am]pelkoa, | \e
    \ind |\[Dm]kaikkialla | kuitenkin |\[Am]rakkautta | \e
    \ind \lrep |\[C]Elämä |\[G]kaunis |\[Am]on | \e \rrep
  \endverse
\endsong


\beginsong{Lämpö, löyly}[by={larva},tags={sauna}]
  \beginchorus\memorize
    |\[Am] \[^\mn{A}]Lämpö, |\[^\mn{E}]löyly, \[G]iha|\[C]nainen
    |\[Dm] Poista |\[E]kiire aina|\[Am]hinen
  \endchorus
  \beginchorus
    |^ Täytä s|ielu, t^yhjää |^mieli
    |^ Anna |^ajatusten |^sulaa
  \endchorus
  \beginchorus
    |^ Tuli, |vesi, ^vasta|^jaiset
    |^ Yhes |^löylyn tänne |^tuovat
  \endchorus
  \beginchorus
    \ind |\[F] Ilon, |\[E]rauhan, meille |\[Am]suovat
  \endchorus
\endsong

\nextcol % Jump to the next page; can be removed when there wouldn't be an empty page
\beginsong{Kyynikolle pelastus}[by={larva}]
  \beginchorus
    |\[\mnc{C}C]Mikä tääll' \[\mn{B}]on |\[F]aitoo?
    Ku |\[C]rakkauskin vaatii |\[G]taitoo.
  \endchorus
  \beginchorus
    |\[C]Anna mulle |\[F]jotakin pientä:
    |\[G]lientä tai vaikkapa |\[C]sientä.
  \endchorus
  \noendsongvfill% % because of the intersong below
\endsong


\begin{intersong} % Fibonacci sequence
  \vfill
  \begin{math}
    % \mathbb comes from 'amssymb' package (the rest doesn't require any pkgs)
    (F_{n})_{n\in\mathbb{N}} = \left\lbrace
      \begin{array}{l}
        F_0 = 0\\
        F_1 = 1\\
        F_n = F_{n-1} + F_{n-2}, n>1
      \end{array}
    \right\rbrace = (0, 1, 1, 2, 3, 5, 8, 13, 21, 34, \ldots)
  \end{math}
  \vfill
  %\vfill
  %\footnotesize
  %\begin{center}
  %  1, 1, 2, 3, 5, 8, 13, 21, 34, 55, 89, 144, 233, 377, 610, 987, 1597, \ldots
  %\end{center}
\end{intersong}


%%%%%%%%%%%%%%%%%%%%%%%%%%%%%%%%%%%%%%%%%%%%%%%%%%%%%%%%%%%%%%%%%%%
%%% LATEST PRINTOUT CONTAINED THE SONGS ABOVE.                  %%%
%%%%%%%%%%%%%%%%%%%%%%%%%%%%%%%%%%%%%%%%%%%%%%%%%%%%%%%%%%%%%%%%%%%
%%% Please try to not change the song numbers above this point. %%%
%%% Add new songs only after this point.                        %%%
%%%%%%%%%%%%%%%%%%%%%%%%%%%%%%%%%%%%%%%%%%%%%%%%%%%%%%%%%%%%%%%%%%%



    \chordsoff % songs: do not show (empty line for non-existing) chords
    \renewcommand{\lyricfont}{\small} % songs: use smaller font
    \songcolumns{2} % songs: two columns per page
    \songpos{1} % songs: avoid ONLY page-turns within songs
    % songs: make penalty for breaking column/page at any line of lyrics to be the same:
    % (The default for \interlinepenalty is 1000, and for all the others 200.)
    \interlinepenalty=200 %
    % Finnish spells and poems
% ========================
%
% The following sets the song number for the first song in this file.
% The number will automatically be incremented by one for each song.
% Please do not change this! Changing would make different versions of
% the songbook to have different numbers for the same songs, and it
% would totally mess up the selection booklets causing them to have
% wrong songs in them. (For the same reason, add new songs only to the
% end of each songs_ file.)
\setcounter{songnum}{700}


\beginsong{\texorpdfstring{Maailmansynty\-runo}{Maailmansyntyruno}}[by={Kalevala: 1. runo}]
  \beginverse
    Mieleni minun tekevi,
    aivoni ajattelevi
    lähteäni laulamahan,
    saa'ani sanelemahan,
    sukuvirttä suoltamahan,
    lajivirttä laulamahan.
    Sanat suussani sulavat,
    puhe'et putoelevat,
    kielelleni kerkiävät,
    hampahilleni hajoovat.
    Veli kulta, veikkoseni,
    kaunis kasvinkumppalini!
    Lähe nyt kanssa laulamahan,
    saa kera sanelemahan
    yhtehen yhyttyämme,
    kahta'alta käytyämme!
    Harvoin yhtehen yhymme,
    saamme toinen toisihimme
    näillä raukoilla rajoilla,
    poloisilla Pohjan mailla.
  \endverse
  \beginverse
    Lyökämme käsi kätehen,
    sormet sormien lomahan,
    lauloaksemme hyviä,
    parahia pannaksemme,
    kuulla noien kultaisien,
    tietä mielitehtoisien,
    nuorisossa nousevassa,
    kansassa kasuavassa:
    noita saamia sanoja,
    virsiä virittämiä
    vyöltä vanhan Väinämöisen,
    alta ahjon Ilmarisen,
    päästä kalvan Kaukomielen,
    Joukahaisen jousen tiestä,
    Pohjan peltojen periltä,
    Kalevalan kankahilta.
    Niit' ennen isoni lauloi
    kirvesvartta vuollessansa;
    niitä äitini opetti
    väätessänsä värttinätä,
    minun lasna lattialla
    eessä polven pyöriessä,
    maitopartana pahaisna,
    piimäsuuna pikkaraisna.
    Sampo ei puuttunut sanoja
    eikä Louhi luottehia:
    vanheni sanoihin sampo,
    katoi Louhi luottehisin,
    virsihin Vipunen kuoli,
    Lemminkäinen leikkilöihin.
  \endverse
  \beginverse
    Viel' on muitaki sanoja,
    ongelmoita oppimia:
    tieohesta tempomia,
    kanervoista katkomia,
    risukoista riipomia,
    vesoista vetelemiä,
    päästä heinän hieromia,
    raitiolta ratkomia,
    paimenessa käyessäni,
    lasna karjanlaitumilla,
  \endverse
  \beginverse
    metisillä mättähillä,
    kultaisilla kunnahilla,
    mustan Muurikin jälessä,
    Kimmon kirjavan keralla.
    Vilu mulle virttä virkkoi,
    sae saatteli runoja.
    Virttä toista tuulet toivat,
    meren aaltoset ajoivat.
    Linnut liitteli sanoja,
    puien latvat lausehia.
    Ne minä kerälle käärin,
    sovittelin sommelolle.
    Kerän pistin kelkkahani,
    sommelon rekoseheni;
    ve'in kelkalla kotihin,
    rekosella riihen luoksi;
    panin aitan parven päähän
    vaskisehen vakkasehen.
    Viikon on virteni vilussa,
    kauan kaihossa sijaisnut.
  \endverse
  \beginverse
    Veänkö vilusta virret,
    lapan laulut pakkasesta,
    tuon tupahan vakkaseni,
    rasian rahin nenähän,
    alle kuulun kurkihirren,
    alle kaunihin katoksen,
    aukaisen sanaisen arkun,
    virsilippahan viritän,
    kerittelen pään kerältä,
    suorin solmun sommelolta?
  \endverse
  \beginverse
    Niin laulan hyvänki virren,
    kaunihinki kalkuttelen
    ruoalta rukihiselta,
    oluelta ohraiselta.
    Kun ei tuotane olutta,
    tarittane taarivettä,
    laulan suulta laihemmalta,
    vetoselta vierettelen
    tämän iltamme iloksi,
    päivän kuulun kunniaksi,
    vaiko huomenen huviksi,
    uuen aamun alkeheksi.
    * * *
  \endverse
  \beginverse
    Noin kuulin saneltavaksi,
    tiesin virttä tehtäväksi:
    yksin meillä yöt tulevat,
    yksin päivät valkeavat;
    yksin syntyi Väinämöinen,
    ilmestyi ikirunoja
    kapehesta kantajasta,
    Ilmattaresta emosta.
  \endverse
  \beginverse
    Olipa impi, ilman tyttö,
    kave luonnotar korea.
    Piti viikoista pyhyyttä,
    iän kaiken impeyttä
    ilman pitkillä pihoilla,
    tasaisilla tanterilla.
    Ikävystyi aikojansa,
    ouostui elämätänsä,
    aina yksin ollessansa,
    impenä eläessänsä
  \endverse
  \beginverse
    ilman pitkillä pihoilla,
    avaroilla autioilla.
    Jop' on astuiksen alemma,
    laskeusi lainehille,
    meren selvälle selälle,
    ulapalle aukealle.
    Tuli suuri tuulen puuska,
    iästä vihainen ilma;
    meren kuohuille kohotti,
    lainehille laikahutti.
  \endverse
  \beginverse
    Tuuli neittä tuuitteli,
    aalto impeä ajeli
    ympäri selän sinisen,
    lakkipäien lainehien:
    tuuli tuuli kohtuiseksi,
    meri paksuksi panevi.
    Kantoi kohtua kovoa,
    vatsantäyttä vaikeata
    vuotta seitsemän satoa,
    yheksän yrön ikeä;
  \endverse
  \beginverse
    eikä synny syntyminen,
    luovu luomatoin sikiö.
    Vieri impi veen emona.
    Uipi iät, uipi lännet,
    uipi luotehet, etelät,
    uipi kaikki ilman rannat
    tuskissa tulisen synnyn,
    vatsanvaivoissa kovissa;
    eikä synny syntyminen,
    luovu luomatoin sikiö.
  \endverse
  \beginverse
    Itkeä hyryttelevi;
    sanan virkkoi, noin nimesi:
    ``Voi poloinen, päiviäni,
    lapsi kurja, kulkuani!
    Jo olen joutunut johonki:
    iäkseni ilman alle,
    tuulen tuuiteltavaksi,
    aaltojen ajeltavaksi
    näillä väljillä vesillä,
    lake'illa lainehilla!
  \endverse
  \beginverse
    Parempi olisi ollut
    ilman impenä eleä,
    kuin on nyt tätä nykyä
    vierähellä veen emona:
    vilu tääll' on ollakseni,
    vaiva värjätelläkseni,
    aalloissa asuakseni,
    veessä vierielläkseni.
    Oi Ukko, ylijumala,
    ilman kaiken kannattaja!
  \endverse
  \beginverse
    Tule tänne tarvittaissa,
    käy tänne kutsuttaessa!
    Päästä piika pintehestä,
    vaimo vatsanvääntehestä!
    Käy pian, välehen jou'u,
    välehemmin tarvitahan!''
  \endverse
  \beginverse
    Kului aikoa vähäisen,
    pirahteli pikkaraisen.
    Tuli sotka, suora lintu;
    lenteä lekuttelevi
  \endverse
  \beginverse
    etsien pesän sijoa,
    asuinmaata arvaellen.
    Lenti iät, lenti lännet,
    lenti luotehet, etelät.
    Ei löyä tiloa tuota,
    paikkoa pahintakana,
    kuhun laatisi pesänsä,
    ottaisi olosijansa.
  \endverse
  \beginverse
    Liitelevi, laatelevi;
    arvelee, ajattelevi:
  \endverse
  \beginverse
    ``Teenkö tuulehen tupani,
    aalloillen asuinsijani?
    Tuuli kaatavi tupasen,
    aalto vie asuinsijani.''
    Niin silloin ve'en emonen,
    veen emonen, ilman impi,
    nosti polvea merestä,
    lapaluuta lainehesta
    sotkalle pesän sijaksi,
    asuinmaaksi armahaksi.
  \endverse
  \beginverse
    Tuo sotka, sorea lintu,
    liiteleikse, laateleikse.
    Keksi polven veen emosen
    sinerväisellä selällä;
    luuli heinämättähäksi,
    tuoreheksi turpeheksi.
    Lentelevi, liitelevi,
    päähän polven laskeuvi.
    Siihen laativi pesänsä,
    muni kultaiset munansa:
  \endverse
  \beginverse
    kuusi kultaista munoa,
    rautamunan seitsemännen.
  \endverse
  \beginverse
    Alkoi hautoa munia,
    päätä polven lämmitellä.
    Hautoi päivän, hautoi toisen,
    hautoi kohta kolmannenki.
    Jopa tuosta veen emonen,
    veen emonen, ilman impi,
    tuntevi tulistuvaksi,
    hipiänsä hiiltyväksi;
  \endverse
  \beginverse
    luuli polvensa palavan,
    kaikki suonensa sulavan.
    Vavahutti polveansa,
    järkytti jäseniänsä:
    munat vierähti vetehen,
    meren aaltohon ajaikse;
    karskahti munat muruiksi,
    katkieli kappaleiksi.
  \endverse
  \beginverse
    Ei munat mutahan joua,
    siepalehet veen sekahan.
  \endverse
  \beginverse
    Muuttuivat murut hyviksi,
    kappalehet kaunoisiksi:
    munasen alainen puoli
    alaiseksi maaemäksi,
    munasen yläinen puoli
    yläiseksi taivahaksi;
    yläpuoli ruskeaista
    päivöseksi paistamahan,
    yläpuoli valkeaista,
    se kuuksi kumottamahan;
    mi munassa kirjavaista,
    ne tähiksi taivahalle,
    mi munassa mustukaista,
    nepä ilman pilvilöiksi.
  \endverse
  \beginverse
    Ajat eellehen menevät,
    vuoet tuota tuonnemmaksi
    uuen päivän paistaessa,
    uuen kuun kumottaessa.
    Aina uipi veen emonen,
    veen emonen, ilman impi,
  \endverse
  \beginverse
    noilla vienoilla vesillä,
    utuisilla lainehilla,
    eessänsä vesi vetelä,
    takanansa taivas selvä.
    Jo vuonna yheksäntenä,
    kymmenentenä kesänä
    nosti päätänsä merestä,
    kohottavi kokkoansa.
    Alkoi luoa luomiansa,
    saautella saamiansa
  \endverse
  \beginverse
    selvällä meren selällä,
    ulapalla aukealla.
    Kussa kättä käännähytti,
    siihen niemet siivoeli;
    kussa pohjasi jalalla,
    kalahauat kaivaeli;
    kussa ilman kuplistihe,
    siihen syöverit syventi.
  \endverse
  \beginverse
    Kylin maahan kääntelihe:
    siihen sai sileät rannat;
  \endverse
  \beginverse
    jaloin maahan kääntelihe:
    siihen loi lohiapajat;
    pä'in päätyi maata vasten:
    siihen laitteli lahelmat.
    Ui siitä ulomma maasta,
    seisattelihe selälle:
    luopi luotoja merehen,
    kasvatti salakaria
    laivan laskemasijaksi,
    merimiesten pään menoksi.
  \endverse
  \beginverse
    Jo oli saaret siivottuna,
    luotu luotoset merehen,
    ilman pielet pistettynä,
    maat ja manteret sanottu,
    kirjattu kivihin kirjat,
    veetty viivat kallioihin.
    Viel' ei synny Väinämöinen,
    ilmau ikirunoja.
    Vaka vanha Väinämöinen
    kulki äitinsä kohussa
  \endverse
  \beginverse
    kolmekymmentä keseä,
    yhen verran talviaki,
    noilla vienoilla vesillä,
    utuisilla lainehilla.
    Arvelee, ajattelevi,
    miten olla, kuin eleä
    pimeässä piilossansa,
    asunnossa ahtahassa,
    kuss' ei konsa kuuta nähnyt
    eikä päiveä havainnut.
  \endverse
  \beginverse
    Sanovi sanalla tuolla,
    lausui tuolla lausehella:
    ``Kuu, keritä, päivyt, päästä,
    otava, yhä opeta
    miestä ouoilta ovilta,
    veräjiltä vierahilta,
    näiltä pieniltä pesiltä,
    asunnoilta ahtahilta!
    Saata maalle matkamiestä,
    ilmoillen inehmon lasta,
    kuuta taivon katsomahan,
    päiveä ihoamahan,
    otavaista oppimahan,
    tähtiä tähyämähän!''
    Kun ei kuu kerittänynnä
    eikä päivyt päästänynnä,
    ouosteli aikojansa,
    tuskastui elämätänsä:
    liikahutti linnan portin
    sormella nimettömällä,
  \endverse
  \beginverse
    lukon luisen luikahutti
    vasemmalla varpahalla;
    tuli kynsin kynnykseltä,
    polvin porstuan ovelta.
    Siitä suistui suin merehen,
    käsin kääntyi lainehesen;
    jääpi mies meren varahan,
    uros aaltojen sekahan.
  \endverse
  \beginverse
    Virui siellä viisi vuotta,
    sekä viisi jotta kuusi,
  \endverse
  \beginverse
    vuotta seitsemän, kaheksan.
    Seisottui selälle viimein,
    niemelle nimettömälle,
    manterelle puuttomalle.
    Polvin maasta ponnistihe,
    käsivarsin käännältihe.
    Nousi kuuta katsomahan,
    päiveä ihoamahan,
    otavaista oppimahan,
    tähtiä tähyämähän.
  \endverse
  \beginverse
    Se oli synty Väinämöisen,
    rotu rohkean runojan
    kapehesta kantajasta,
    Ilmattaresta emosta.
  \endverse
\endsong


\beginsong{Aamulla}[tags={aamu, Aurinko}]
  \beginverse
    Terve kasvos näyttämästä,
    Päivä kulta koittamasta,
    Aurinko ylenemästä!
    Pääsit ylös altoin alta
    Yli männistön ylenit,
    Nousit kullaisna käkenä,
    Hopeaisna kyyhkyläisnä
    Tasaiselle taivahalle,
    Elollesi entiselle,
    Matkoillesi muinaisille.
  \endverse
  \beginverse
    Nouse aina aikoinasi
    Perästä tämänki päivän,
    Tuo meille tuliaisiksi
    Anna täyttä terveyttä,
    Siirrä saama saatavihin,
    Pyytö päähän peukalomme,
    Onni onkemme nenähän;
    Käy kaaresi kaunihisti,
    Päätä päivän matkuesi,
    Pääse illalla ilohon!
  \endverse
\endsong


\beginsong{Tuulen sanat}[tags={tuuli}]
  \beginverse
    Terve kuu, terve päivä,
    Terve ilma, terve tuulet,
    Pohjois- ja etelätuuli,
    Itätuuli, länsituuli
    Lapintuuli, luoetuuli
    Suvituuli, lounaistuuli,
    Päivän nousu- ja laskutuuli
    Ja kaikki väliset tuulet!
    Lepy tuuli leppeäksi
    Lauhu ilma lauhkeaksi
    Kuu kirkas kumottamahan,
    Päivä lämmin paistamahan;
    Sivu tuulet tuulekohot,
    Sivu saakohot satehet,
    Kohti kuut kumottakohot,
    Kohti päivät paistakohot!
  \endverse
\endsong


\beginsong{Löylyn sanat: terve löyly}[tags={sauna}]
  \beginverse
    Terve löyly, terve lämmin
    terve henkäys kiukainen,
    kylpy lämpimäin kivisten,
    hiki vanhan Väinämöisen.
    Löylystä vihannan vihdan,
    tervan voimasta terveiden.
  \endverse
  \beginverse
    Löyly kiukahan kivestä,
    löyly saunan sammalista.
    Tervehyttä tekemähän,
    rauhoa rakentamahan,
    kipehille voitehiksi,
    pahoille parantehiksi.
  \endverse
\endsong


\beginsong{Löylyn sanat: tule löylyhyn}[tags={sauna}]
  \musicnote{Melodia: Kalevala-sävelmä tai esim. Hedingarna: Täss' on nainen}
  \beginverse
    Tule löylyhyn, Jumala,
    Iso ilman, lämpimähän,
    Terveyttä tekemähän,
    Rauhoa rakentamahan
  \endverse
  \beginverse
    Lyötä maahan liika löyly
    Paha löyly pois lähetä
    Ettei polta tyttöjäsi
    Turmele tekemiäsi
  \endverse
  \beginverse
    Minkä vettä viskaelen
    Noille kuumille kivillen
    Se medeksi muuttukohon
    Simaksi sirahtakohon
  \endverse
  \beginverse
    Juoskohon joki metinen
    Simalampi laikkukohon
    Läpi kiukahan kivisen
    Läpi saunan sammalisen!
  \endverse
\endsong


\beginsong{Ihmisen synty}[]
  \beginverse
    Ihminen ihala ilme,
    Sukukunnan suuri luomus,
    Tehty on mullan kakkarasta,
    Mullan kaakusta rakettu,
    (Sille Herra hengen antoi,
    Luoja suustahan sukesi.)
  \endverse
\endsong


\beginsong{Karhun synty}[]
  \beginverse
    Otsoseni, ainoiseni,
    Mesikämmen kaunoiseni,
    Kyllä mä sukusi tieän,
    Miss' oot otso syntynynnä,
    Saatuna sinisaparo,
    Jalka kyntinen kyhätty:
    Tuoll' oot otso syntynynnä
    Ylähällä taivosessa,
    Kuun kukuilla, päällä päivän,
    Seitsentähtien selällä,
    Ilman impien tykönä,
    Luona luonnon tyttärien.
  \endverse
  \beginverse
    Tuli läikkyi taivahasta,
    Ilma kääntyi kehrän päällä,
    Otsoa suettaessa,
    Mesikkiä luotaessa.
    Sieltä maahan laskettihin
    Vierehen metisen viian,
    Hongattaren huolitella,
    Tuomettaren tuu'itella,
    Juurella nyrynärehen,
    Alla haavan haaralatvan,
    Metsän linnan liepehellä,
    Korven kultaisen kotona.
  \endverse
  \beginverse
    Siitä otso ristittihin,
    Karvahalli kastettihin,
    Metisellä mättähällä,
    Sarajoen salmen suulla,
    Pohjan tyttären sylissä.
    Siinä se valansa vannoi
    Pohjan eukon polven päässä,
    Essä julkisen Jumalan,
    Alla parran autuahan,
    Tehä ei syytä syyttömälle,
    Vikoa viattomalle,
    Käyä kesät kaunihisti,
    Soreasti sorkutella,
    Elellä ajat iloiset
    Suon selillä, maan navoilla,
    Kilokangasten perillä;
    Käyä kengättä kesällä,
    Sykysyllä syylingittä,
    Asua ajat pahemmat,
    Talvikylmät kyhmästellä,
    Tammisen tuvan sisässä,
    Havulinna liepehellä,
    Kengällä komean kuusen,
    Katajikon kainalossa.
  \endverse
\endsong


\beginsong{Kiven synty}[]
  \beginverse
    Ken kiven kiveksi tiesi,
    Kun oli otraisna jyvänä,
    Nousi maasta mansikkana,
    Puun juuresta puolukkana,
    Taikka häilyi hattarassa,
    Piili pilvien sisässä,
    Tuli maahan taivahasta,
    Putosi punakeränä,
    Kaaloi kakraisna kapuna,
    Vieri vehnäisnä mykynä,
    Läpi pilvipatsahien,
    Puhki kaarien punaisten,
    Hullu huutavi kiveksi,
    Maan munaksi mainitsevi.
  \endverse
\endsong


\beginsong{Noidan synty}[]
  \beginverse
    Kyllä tieän noian synnyn,
    Sekä alun arpojia:
    Tuoll' on noita syntynynnä,
    Tuolla alku arpojien,
    Pohjan penkeren takana,
    Lapin maassa laakeassa;
    Siell' on noita syntynynnä,
    Siellä arpoja sikesi,
    Hakoisella vuotehella,
    Kivisellä pääalalla.
  \endverse
\endsong


\beginsong{Puiden synty}[]
  \beginverse
    Sampsa poika Pellervoinen
    Kesät kentällä makasi
    Keskellä jyväketoa,
    Jyväparkan parmahalla;
    Otti kuusia jyviä,
    Seitsemiä siemeniä,
    Yhen nää'än nahkasehen,
    Koipehen kesäoravan,
    Läksi maita kylvämähän,
    Toukoja tihittämähän.
  \endverse
  \beginverse
    Kylvi maita kyyhätteli,
    Kylvi maita, kylvi soita,
    Kylvi auhtoja ahoja,
    Panettavi paasikoita.
    Kylvi kummut kuusikoiksi,
    Mäet kylvi männiköiksi,
    Kankahat kanervikoiksi,
    Notkont nuoriksi vesoiksi.
    Noromaille koivut kylvi,
    Lepät maille leyhkeille,
    Kylvi tuomet tuorehille,
    Pihlajat pyhille maille,
    Pajut maille paisuville,
    Raiat nurmien rajoille,
    Katajat karuille maille,
    Tammet virran vierimaille.
  \endverse
  \beginverse
    Läksi puut ylenemähän,
    Vesat nuoret nousemahan,
    Tuuliaisen tuu'ittaissa,
    Ahavaisen liekuttaissa,
    Kasvoi kuuset kukkalatvat,
    Lautui lakkapäät petäjät,
    Nousi koivuset noroilla,
    Lepät mailla leyhkeillä,
    Tuomet mailla tuorehilla,
    Pihlajat pyhillä mailla,
    Pajut mailla paisuvilla,
    Raiat mailla raikkahilla,
    Katajat karuilla mailla,
    Tammet virran vieremillä.
  \endverse
\endsong


\beginsong{Tammen synty}[]
  \beginverse
    Oli ennen neljä neittä,
    Kolme kuulua tytärtä,
    Sininurmen niitännässä,
    Korttehen kokoannassa,
    Nenässä utuisen niemen,
    Päässä saaren terhenisen.
    Niitit päivän, niitit toisen,
    Niitit kohta kolmannenki,
    Minkä niitit, sen haravoit,
    Kaikki karhille vetelit,
    Laitit heinät lallosille,
    Sataisille saprasille,
    Siitä suovahan kokosit,
    Saatoit sankapieleksihin.
  \endverse
  \beginverse
    Jo oli nurmi niitettynä,
    Heinät luotu pielin pystyin,
    Tuli Turjan lappalainen,
    Nimeltä tulinen Tursas,
    Tunki heinäset tulehen,
    Paiskasi panun väkehen.
  \endverse
  \beginverse
    Tuli tuhkia vähäinen,
    Kypeniä pikkarainen,
    Tytöt tuossa arvelevat,
    Neiet neuvoa pitävät,
    Kunne tuhkat koottanehen,
    Poron pohjat pantanehen:
    ``Noistapa puuttuvi poroa,
    Lipeätä liuvahtavi,
    Pestä päätä Päivän poian,
    Silmiä hyvän urohon''.
  \endverse
  \beginverse
    Tuli tuuli tunturista,
    Kova ilma koillisesta,
    Tuonne tuuli tuhkat kantoi,
    Porot koillinen kokosi,
    Nenästä utuisen niemen,
    Päästä saaren terhenisen,
    Korvalle tulisen kosken,
    Pyhän virran vieremille.
    Tuuli tuopi tammen terhon,
    Kantoi maalta kaukaiselta
    Korvalle tulisen kosken,
    Pyhän virran vieremille,
    Heitti paikalle hyvälle,
    Maan lihavan liepehelle.
    Nousi tuosta nuori taimi,
    Vesa verraton vetihe,
    Siitä kasvoi kaunis tammi,
    Yleni rutimon raita,
    Latva täytti taivahille,
    Oksat ilmoille olotti.
  \endverse
\endsong


\beginsong{Tulen synty}[tags={tuli}]
  \beginverse
    Ei tuli syviltä synny,
    Eikä kasva karkealta,
    Tuli syntyi taivosessa,
    Seitsentähtyen selällä,
    Siell' on tulta tuu'iteltu,
    Valkeaista vaapoteltu,
    Kultaisessa kursikossa,
    Kultakunnahan kukulla.
  \endverse
  \beginverse
    Kasi kaunis, neito nuori,
    Tulityttö taivahinen,
    Tuopa tulta tuu'ittavi,
    Vaapottavi valkeata,
    Tuolla taivahan navoilla,
    Yllä taivahan yheksän,
    Hopeaiset nuorat notkui,
    Koukku kultainen kulisi,
    Neien tulta tuu'ittaissa,
    Vaapottaissa valkeaista.
  \endverse
  \beginverse
    Putosi tuli punainen,
    Kirposi kipuna yksi,
    Kultaisesta kursikosta,
    Hopeaisesta sulusta,
    Ilmalta yheksänneltä,
    Kaheksannen kannen päältä,
    Läpi taivahan tasaisen,
    Halki tuon ihalan ilman,
    Läpi ramppalan ovista,
    Läpi lapsen vuotehesta;
    Paloi polvet poikuelta,
    Paloi paarmahat emolta.
  \endverse
  \beginverse
    Se lapsi meni manalle,
    Katopoika tuonelahan,
    Kun oli luotu kuolemahan,
    Katsottu katoamahan,
    Tuskissa tulen punaisen,
    Angervoisen ailuissa;
    Märäten meni manalle,
    Torkahellen tuonelahan,
    Tuonen tyttöjen torua,
    Manan lasten lausuella.
  \endverse
  \beginverse
    Emopa ei manalle mennyt;
    Akka oli viisas villikerta,
    Se tunsi tulen lumoa,
    Valkeaisen vaivutella,
    Läpi pienen neulan silmän,
    Halki kirvehen hamaran,
    Puhki kuuman tuuran putken,
    Kerivi tulen kerälle,
    Suorittavi sommelolle,
    Kierähyttävi keräsen
    Pitkin pellon pientaretta,
    Läpi maan, läpi manuen,
    Työnti Tuonelan jokehen,
    Manalan syväntehesen.
  \endverse
\endsong

%% % Tulen synty, toinen versio
%%\beginsong{Tulen synty B}[tags={tuli}]
%%  \beginverse
%%    Tulta iski ilman Ukko,
%%    Valahutti valkeata,
%%    Miekalla tuliterällä
%%    Säilällä säkenevällä,
%%    Ylisessä taivosessa,
%%    Tähtitarhojen takana.
%%  \endverse
%%  \beginverse
%%    Saipa tulta iskemällä,
%%    Kätkevi tulikipunan
%%    Kultaisehen kukkarohon,
%%    Hopeaisehen kehä'än,
%%    Antoi neien tuu'itella,
%%    Ilman immen vaapotella.
%%  \endverse
%%  \beginverse
%%    Neiti pitkän pilven päällä,
%%    Impi ilman partahalla,
%%    Tuota tulta tuu'ittavi,
%%    Valkeaista vaapottavi,
%%    Kultaisessa kätkyessä,
%%    Hihnoissa hopeisissa;
%%    Hopeiset hihnat helkkyi,
%%    Kätkyt kultainen kulisi,
%%    Pilvet liikkui, taivot naukui,
%%    Taivon kannet kallistihe,
%%    Tulta tuu'iteltaessa,
%%    Valkeata vaapottaissa.
%%  \endverse
%%  \beginverse
%%    Impi tulta tuu'itteli,
%%    Valkeaista vaapotteli,
%%    Tulta sormin suoritteli,
%%    Käsin vaali valkeaista,
%%    Tuli tuhmalta putosi,
%%    Valkea varattomalta,
%%    Kätösistä käänteliän,
%%    Sormilta somittelian.
%%  \endverse
%%  \beginverse
%%    Kirposi tulikipuna,
%%    Suikahti punasoronen,
%%    Läpi läikkyi taivosista,
%%    Puhki pilvistä putosi.
%%    Päältä taivahan yheksän,
%%    Halku kuuen kirjokannen.
%%  \endverse
%%  \beginverse
%%    Tuikahti tulikipuna,
%%    Putosi punasoronen,
%%    Luojan luomilta tiloilta,
%%    Ukon ilman iskemiltä,
%%    Puhki reppänän retuisen,
%%    Kautta kuivan kurkihirren,
%%    Tuurin uutehen tupahan,
%%    Palvosen laettomahan;
%%    Sitten sinne tultuansa
%%    Tuurin uutehen tupahan,
%%    Panihe pahoille töille,
%%    Löihe töille törkeille:
%%    Riipi rinnat tyttäriltä,
%%    Käsivarret neitosilta,
%%    Turmeli pojilta polvet,
%%    Isännältä parran poltti.
%%  \endverse
%%  \beginverse
%%    Äiti lastansa imetti
%%    Kätkyessä vaivaisessa
%%    Alla reppänän retuisen;
%%    Siihen tultua tulonen
%%    Poltti lapsen kätkyestä,
%%    Puhki paarmahat emolta,
%%    Meni siitä mennessänsä,
%%    Vieri vieriellessänsä,
%%    Ensin poltti paljon maita,
%%    Paljon maita, paljon soita,
%%    Poltti auhtoja ahoja,
%%    Sekä korpia kovasti,
%%    Viimein vieprahti vetehen,
%%    Aaltoihin Aluejärven.
%%  \endverse
%%  \beginverse
%%    Tuosta tuo Aluejärvi
%%    Oli syttyä tulehen,
%%    Säkehinä säihkyellä,
%%    Tuon tuiman tulen käsissä,
%%    Ärtyi päälle äyrästensä,
%%    Kuohui päälle korpikuusten,
%%    Kuohui kuiville kalansa,
%%    Arinoille ahvenensa.
%%  \endverse
%%  \beginverse
%%    Viel' ei viihtynyt tulonen,
%%    Aalloista Aluejärven,
%%    Karkasi katajikkohon,
%%    Niin paloi katajakangas,
%%    Kohahutti kuusikkohon,
%%    Poltti kuusikon komean,
%%    Vieri vieläki etemmä,
%%    Poltti puolen Pohjanmaata,
%%    Sakaran Savon rajoa,
%%    Kappalehen Karjalata.
%%  \endverse
%%  \beginverse
%%    Kävi siitä kätkösehen,
%%    Pillojansa piilemähän,
%%    Heittihe lepeämähän
%%    Kahen kannon juuren alle,
%%    Lahokannon kainalohon,
%%    Leppäpökkelön povehen,
%%    Sieltä tuotihin tupihin,
%%    Honkaisihin huonehisin,
%%    Päivällä käsin pi'ellä
%%    Kivisessä kiukahassa,
%%    Yöllä lie'essä levätä
%%    Hiilisessä hinkalossa.
%%  \endverse
%%\endsong


\beginsong{Veden synty}[tags={vesi}]
  \beginverse
    Tiettävä on vetosen synty,
    Kanssa kastehen sijentö:
    Vesi on tullut taivosesta,
    Pilvistä pisarehina,
    Siitä vuoressa sikesi,
    Kasvoi kallion lomassa.
    Vesiviitta Vaitan poika,
    Suoviitta Kalevan poika,
    Veen kaivoi kalliosta,
    Veen vuoresta valutti,
    Kepillänsä kultaisella,
    Sauvallansa vaskisella.
  \endverse
  \beginverse
    Vuoresta valuttuansa,
    Kalliosta saatuansa,
    Vesi heilui hettehenä,
    Kulki pieninä puroina,
    Siitä suureksi sukeni,
    Sai jokena juoksemahan,
    Virtana vipajamahan,
    Koskena kohajamahan,
    Tuonne suurehen merehen,
    Alaisehen aukehesen.
  \endverse
\endsong


\beginsong{Raudan synty}[]
  \beginverse
    Itse ilmoinen Jumala,
    Tuo Ukko, ylinen Luoja,
    Hieroi kahta kämmentänsä
    Vasemmassa polven päässä,
    Siitä syntyi neittä kolme,
    Koko kolme Luonnotarta,
    Rauan ruostehen emoiksi,
    Suu sinervän siittäjiksi.
  \endverse
  \beginverse
    Neiet käyä notkutteli,
    Astui immet ilman äärtä,
    Utarilla uhkuvilla,
    Nännillä pakottavilla,
    Lypsit maalle maitojansa,
    Uhkutit utariansa,
    Lypsit maille, lypsit soille,
    Lypsit vienoille vesille.
    Yksi lypsi mustan maion,
    Vanhimpainen neitoksia,
    Toinen puikutti punaisen,
    Keskimmäinen neitosia,
    Kolmas valkean valutti,
    Nuorimpainen neitosia.
    Ku on lypsi mustan maion,
    Siitä syntyi melto rauta,
    Ku on puikutti punaisen,
    Siit' on saatu rääkyrauta,
    Ku on valkean valutti,
    Siit' on tehtynä teräkset.
  \endverse
  \beginverse
    Oli aikoa vähäisen,
    Rauta tahtovi tavata
    Vanhempata veljeänsä,
    Käyä tulta tuntemassa.
    Tuli tuhmaksi repesi,
    Kovin kasvoi kauheaksi,
    Poltti soita, poltti maita,
    Poltti korpia kovia,
    Oli polttoa poloisen
    Rauta raukan veikkosensa;
    Rauta pääsevi pakohon,
    Pakohon ja piilemähän
    Pimeähän Pohjolahan,
    Lapin laajalle perälle,
    Suurimmalle suon selälle,
    Tuiman tunturin laelle,
    Jossa joutsenet munivat,
    Hanhi poiat hautelevi.
  \endverse
  \beginverse
    Rauta suossa soikottavi,
    Vetelässä vellottavi,
    Piili vuoen, piili toisen,
    Piili kohta kolmannenki,
    Ei toki pakohon pääsnyt
    Tulen tuimista käsistä,
    Piti käyä toisen kerran,
    Lähteä tulen tuville,
    Astalaksi tehtäessä,
    Miekaksi taottaessa.
  \endverse
  \beginverse
    Susi juoksi suota myöten,
    Karhu kangasta samosi,
    Suo nousi suen jaloissa,
    Kangas karhun kämmenissä,
    Kasvoi rautaiset karangot,
    Teräksiset tierottimet,
    Suen sorkkien sijoille,
    Karhun kannan kaivamille.
  \endverse
  \beginverse
    Tuop' on seppo Ilmarinen,
    Taki taitava takoja,
    Oli teitensä käviä,
    Matkojensa mitteliä,
    Joutuvi suen jälille,
    Karhun kantapään sijoille.
    Näki rautaiset orahat,
    Teräksiset tierottimet,
    Suen suurilla jälillä,
    Karhun kannan kääntämillä,
    Sanovi sanalla tuolla:
    ``Voi sinua rauta raukka,
    Kun olet kurjassa tilassa,
    Alahaisessa asussa,
    Suolla sorkissa sutosen,
    Aina karhun askelissa',
    Kasvaisitko kaunihiksi,
    Koreaksi korkenisit,
    jos sun suosta suorittaisin,
    Sekä saattaisin pajahan,
    Tunkisin tulisijahan,
    Ahjohon asettelisin?''
  \endverse
  \beginverse
    Rauta raukka säpsähtihe,
    säpsähtihe, säikähtihe,
    Kun kuuli tulen sanomat,
    Tulen tuiman maininnaiset.
  \endverse
  \beginverse
    Sanoi seppo Ilmarinen:
    ``Et sä synny rauta raukka,
    Ei sinun suku sukeu,
    Eikä kasva heimokunta,
    Ilman tuimatta tuletta,
    Ilman viemättä pajahan,
    Ahjohon asettamatta,
    Lietsimellä lietsomatta;
    Vaan ellös sitä varatko,
    Ellös olko milläskänä,
    Tuli ei polta tuttuansa,
    Herjaele heimoansa;
    Kun tulet tulen tuville,
    Hiilisehen hinkalohon,
    Siellä kasvat kaunihiksi,
    Ylenet ylen ehoksi,
    Miesten miekoiksi hyviksi,
    Naisten nauhan päättimiksi.''
  \endverse
  \beginverse
    Senpä päivyen perästä
    Rauta suosta sotkettihin,
    Vetelästä vellottihin,
    Saatihin saven seasta;
    Itse seppo suossa seisoin
    Polvin mustassa murassa
    Rauan suosta raaettaissa,
    Maan murasta muokattaissa,
    Otti rautaiset orahat,
    Teräksiset tierottimet,
    Suen suurista jälistä,
    Karhun kantapään tiloista.
  \endverse
  \beginverse
    Se on seppo Ilmarinen
    Siihen painoi palkehensa,
    Siihen ahjonsa asetti,
    Suurille suen jälille,
    Karhun kannan hiertimille;
    Rauan tunkevi tulehen,
    Lietsoi yön levähtämättä,
    Päivän umpehuttamatta,
    Lietsoi päivän, lietsoi toisen,
    Lietsoi kohta kolmannenki,
    Rauta vellinä venyvi,
    Kuonana kohaelevi,
    Venyi vehnäisnä tahasna,
    Rukihisna taikinana,
    Sepon suurissa tulissa,
    Ilmi valkean väessä.
  \endverse
  \beginverse
    Siitä seppo Ilmarinen
    Katsoi ahjonsa alusta,
    Mitä ahjo antanevi,
    Palkehensa painanevi.
    Ensin saapi rääkyrauan,
    Sitten kuonaisen kuletti,
    Siitä valkean valutti
    Alisesta lietsimestä.
  \endverse
  \beginverse
    Siinä huuti rauta raukka:
    ``Oi sie seppo Ilmarinen
    Ota pois minua täältä
    Tuskista tulen vihaisen!''
  \endverse
  \beginverse
    Sanoi seppo Ilmarinen:
    ``Jos otan sinun tulesta,
    Ehkä kasvat kauheaksi,
    Kovin raivoksi rupeat,
    Vielä veistät veljeäsi,
    Lastuat emosi lasta.''
  \endverse
  \beginverse
    Siinä vannoi rauta raukka,
    Vannoi vaikean valansa,
    Ahjossa, alaisimella,
    Vasaroilla valkkamilla:
    ``En mä liikuta lihoa,
    Enkä verta vierehytä;
    On mun puuta purrakseni,
    Haukatakseni hakoa,
    Närettä näpätäkseni,
    Kiven syäntä syöäkseni,
    Ett' en veistä veljeäni,
    Lastua emoni lasta.
    Parempi on ollakseni,
    Ehompi eleäkseni,
    Kulkialla kumppalina,
    Käyvällä käsiasenna,
    Kuin sukua suin piellä,
    Heimoani herjaella.''
  \endverse
  \beginverse
    Silloin seppo Ilmarinen,
    Takoja ijän ikuinen,
    Rauan tempasi tulesta,
    Asetti alaisimelle,
    Rakentoa raukeaksi,
    Takoa teräkaluiksi,
    Keihä'iksi, kirvehiksi,
    Kaikenlaisiksi kaluiksi.
    Takoa taputtelevi,
    Lyöä helkähyttelevi,
    Vaan ei kiehu rauan kieli,
    Ei sukeu suu teräksen,
    Rauta ei kasva karkeaksi,
    Terä rauan tenhosaksi.
  \endverse
  \beginverse
    Siitä seppo Ilmarinen
    Arvellen ajatelevi,
    Mitä tuohon tuotanehen
    Ja kuta ve'ettänehen
    Teräksen tekomujuiksi,
    Rauan karkaisuvesiksi.
    Laati pikkuisen poroa,
    Lipeäistä liuvotteli,
    Siitä koitti kielellänsä,
    Hyvin maistoi mielellänsä,
    Itse tuon sanoiksi virkki:
    ``Ei nämät hyvät minulle
    Teräksen tekovesiksi,
    Rautojen rakento-maiksi.''
  \endverse
  \beginverse
    Mehiläinen maasta nousi
    Sinisiipi mättähästä,
    Lentelevi, liitelevi,
    Ympäri sepon pajoa;
    Senp' on seppo Ilmarinen
    Käski käyä metsolassa,
    Tuoa mettä metsolasta,
    Simoa simasalosta,
    Teräksille tehtäville,
    Rauoille rakettaville.
  \endverse
  \beginverse
    Herhiläionen Hiien lintu,
    Hiien lintu, Lemmon katti,
    Lensi ympäri pajoa
    Kipujansa kaupotellen,
    Lentelevi, kuuntelevi
    Sepon selviä sanoja
    teräksistä tehtävistä,
    Rauoista rakettavista.
  \endverse
  \beginverse
    Tuo oli siiviltä sivakka,
    Kynäluilta luikkahampi,
    Tuop' on ennätti e'ellä,
    Nouti Hiien hirmuloita,
    Kantoi käärmehen kähyjä,
    Maon mustia mujuja,
    Kusiaisen kutkelmoita,
    Sammakon salavihoja,
    Teräksen tekomujuihin,
    Rauan karkaisuvetehen.
  \endverse
  \beginverse
    Itse seppo Ilmarinen,
    Takoja alinomainen,
    Luulevi, ajattelevi,
    Mehiläisen tulleheksi,
    Tuon on mettä tuoneheksi,
    Kantaneheksi simoa,
    Sanan virkkoi, noin nimesi:
    ``Kas nämät hyvät minulle
    Teräksen tekovesiksi,
    Rautojen rakennusmaiksi.''
    Siihen kasti rauta raukan,
    Siihen tempaisi teräksen,
    Pois tulesta tuotaessa,
    Ahjosta otettaessa;
    Siitä sai teräs pahaksi,
    Rauta raivoksi rupesi,
    Veisti raukka veljeänsä,
    Sukuansa suin piteli,
    Veren laski vuotamahan,
    Hurmehen hurajamahan.
  \endverse
\endsong


\beginsong{Varjele vakainen luoja}[by={Kalevala: 43. runo}]
  \beginverse
    Anna Luoja, suo Jumala
    Anna onni ollaksemme.
    Hyvin ain’ eleäksemme,
    Kunnialla kuollaksemme.
    Suloisessa Suomenmaassa
    Kaunihissa Karjalassa!
  \endverse
  \beginverse
    Varjele, vakainen Luoja
    Kaitse, kaunoinen Jumala,
    Ole puolla poikiesi,
    Aina lastesi apuna,
    Aina yöllisnä tukena,
    Päivällisnä vartiana.
  \endverse
\endsong


% \sclearpage
% \beginsong{Höyhensaaret}[by={Eino Leino}]
%   \beginverse
%     Mitä siitä jos nuorna ma murrunkin
%     tai taitun ma talvisäihin,
%     moni murtunut onpi jo ennemmin
%     ja jäätynyt elämän jäihin.
%     Kuka vanhana vaappua tahtoiskaan?
%     Ikinuori on nuoruus laulujen vaan
%     ja kerkät lemmen ja keväimen,
%     ilot sammuvi ihmisten.
%   \endverse
%   \beginverse
%     Mitä siitä jos en minä sammukaan
%     kuin rauhainen, riutuva liesi,
%     jos sammun kuin sammuvat tähdet vaan
%     ja vaipuvi merillä miesi.
%     Kas, laulaja tähtiä laulelee
%     ja hän meriä suuria seilailee
%     ja hukkuvi hyrskyhyn, ennen kuin
%     käy purjehin reivatuin.
%   \endverse
%   \beginverse
%     Mitä siitä jos en minä saanutkaan,
%     mitä toivoin ma elämältä,
%     kun sain minä toivehet suuret vaan
%     ja kaihojen kantelen hältä.
%     Ja vaikka ma laps olen pieni vain,
%     niin jumalten riemut ma juoda sain
%     ja juoda ne täysin siemauksin ---
%     niin riemut kuin murheetkin.
%   \endverse
%   \beginverse
%     Ja vaikka ma laps olen syksyn vaan
%     ja istuja pitkän illan,
%     sain soittaa ma kielillä kukkivan maan
%     ja hieprukan hivuksilla.
%     Niin mustat, niin mustat ne olivat;
%     ja suurina surut ne tulivat,
%     mut kaikuos riemu nyt kantelen
%     vielä kertasi viimeisen!
%   \endverse
%   \beginverse
%     Oi, kantelo pitkien kaihojen,
%     sinä aarteeni omani, ainoo!
%     Me kaksi, me kuulumme yhtehen,
%     jos kuin mua kohtalot vainoo.
%     Me kuljemme kylästä kylähän näin,
%     ohi kylien koirien räkyttäväin,
%     ja keskellä raition raakuuden
%     sävel soipa on keväimen.
%   \endverse
%   \beginverse
%     Me kaksi, me tulemme metsästä
%     ja me metsien ilmaa tuomme,
%     me laulamme nuoresta lemmestä
%     ja lempemme kuvan me luomme,
%     me luomme sen maailman tomusta niin
%     kuin Luoja loi ihmisen Eedeniin
%     ja korvesta kohoitamme me sen
%     kun vaskisen käärmehen.
%   \endverse
%   \beginverse
%     Te ystävät, joiden rinnassa kyyt
%     yön-pitkät pistää ja kalvaa,
%     te, joita jäytävi sydämen syyt
%     ja elämä harmaja halvaa,
%     oi, helise heille mun kantelein,
%     oi, helise onnea haavehein
%     ja unta silmihin unettomiin
%     mun silmäni suljit sa niin.
%     Kas, ylläpä mustien murheiden
%     on kaunihit taivaankaaret
%     ja kaukana keskellä aaltojen
%     on haaveiden höyhensaaret
%     ja ken sinne lapsosen kaarnalla käy,
%     ei sille ne aavehet yölliset näy,
%     vaan rinnoin hän uinuvi rauhaisin
%     kuin äitinsä helmoihin.
%   \endverse
%   \beginverse
%     Mitä siitä jos valhetta onkin ne vaan
%     ja kestä ei päivän terää!
%     Me uinumme siksi kuin valveutaan
%     ja vaivat ne jällehen herää.
%     Moni nukkui nuorihin toiveisiin
%     ja heräsi hapsihin hopeisiin;
%     hän katsahti ympäri kummissaan
%     ja --- uinahti uudestaan.
%     Miks ihmiset tahtoa, taistella
%     ja koittaa korkealle?
%     Me olemme kaikki vain lapsia
%     ja murrumme murheen alle.
%     Miks emme me kaikki vois uinahtaa
%     ja hyviä olla ja hymytä vaan
%     ja katsoa katsehin kirkkahin
%     vain sielumme syvyyksiin?
%   \endverse
%   \beginverse
%     Oi, unessa murheet ne unhottuu
%     ja rauhaton rauhan saapi,
%     oi, unessa vankikin vapautuu,
%     sen kahlehet katkeaapi,
%     ja köyhä on rikas kuin kuningas maan
%     ja kevyt on valtikka kuninkaan
%     ja kaikki, kaikki on veljiä vaan ---
%     oi, onnea unelmain!
%   \endverse
%   \beginverse
%     Oi, onnea uinua uudelleen
%     ne lapsuen päivät lauhat
%     ja itkeä jällehen yksikseen
%     ne riemut ja rinnan rauhat;
%     taas uskoa, että on lapsi vaan
%     ja että voi alkaa uudestaan
%     ja uskoa uusihin toiveisiin
%     sekä vanhoihin ystäviin!
%   \endverse
%   \beginverse
%     Taas uskoa riemuhun, keväimeen
%     ja lippuhun pilvien linnan
%     ja uskoa lempehen puhtaaseen
%     taas kahden puhtahan rinnan,
%     taas uskoa itsensä rikkahaks
%     ja maailman suureks ja avaraks ---
%     voi, kuinka se sentään on ihanaa,
%     kun sen nuorena uskoa saa!
%   \endverse
%   \beginverse
%     Voi, kuinka se sille on ihanaa,
%     joka kaiken sen kadotti kerran,
%     joka häkistä katseli maailmaa
%     ja näki vain vaaksan verran,
%     joka etsi kauneutta, elämää,
%     ja näki vain markkinavilinää,
%     ja näki räyhäävän raakuuden, tyhmyyden ---
%     niit' aikoja unhota en.
%   \endverse
%   \beginverse
%     Kun muistelen, kuinka ma kerjännyt
%     olen koirana lempeä täällä,
%     miten rikasten portailla pyydellyt
%     olen tuiskulla, tuulissäällä,
%     vain lämpöä hiukkasen, hiukkasen vain
%     ja kun minä muistelen, mitä mä sain
%     ja mitä mä nielin ja vaikenin
%     ja mitä mä ajattelin!
%   \endverse
%   \beginverse
%     Miten olen minä kulkenut, uskonut,
%     ett'eivät ne unhoitukaan!
%     Ja sentään ne olen minä unhoittanut
%     kuin unhoittaa voi kukaan.
%     Ja sentään se nousi, niin kohtalot kaas,
%     ja sentään ma seppona seison taas
%     ja taivahan kansia taon ja lyön ---
%     oi, onnea tähtisen yön!
%   \endverse
%   \beginverse
%     Ne saapuvat, saapuvat uudestaan
%     mun onneni orhit valkeet,
%     ne painavat vanhalla voimallaan
%     mun rintani jättipalkeet.
%     Ja kirkas on taivas ja kukkii maa
%     ja säkenet suustani suitsuaa
%     ja ääneni on kuni ukkosen ---
%     oi, onnea unelmien!
%   \endverse
%   \beginverse
%     Mitä siitä jos haaveeni verkot vaan
%     on verkkoja hämähäkin!
%     Mitä siitä jos omieni viittova vaan
%     on laulua laineiden näkin!
%     Moni nukkui nuorihin toiveisiin
%     ja heräsi hapsihin hopeisiin
%     tai herännyt täällä ei milloinkaan.
%     Missä? Milloin? Helmassa maan.
%     Minä tahdon riemuja keväimen
%     ja onnesta osani kerta!
%     Olen imenyt rintoja totuuden,
%     mut niistä vaan tuli verta.
%     Siis, tulkaa te utaret unelmien,
%     minä vaivun riemunne rinnoillen
%     ja uskon päivähän, aurinkohon.
%     Unen maito on loppumaton.
%   \endverse
%   \beginverse
%     Oi, kauniisti mulle te kaartukaa,
%     mun syömeni sateenkaaret!
%     Mua hiljaa, hiljaa tuudittakaa,
%     te haaveiden höyhensaaret!
%     Mua katsokaa: olen lapsi vaan,
%     olen riisunut päältäni riemut maan
%     ja pyytehet kullan ja kunnian.
%     Uni onni on laulajan.
%   \endverse
%   \beginverse
%     Minä tahdon vain uinua yksikseen.
%     En tahtois vielä mä kuolla.
%     Mut kuulkaa, jo äitini huhuilee
%     Tuonen aaltojen tuolla puolla.
%     Oi, odota hetkinen, äityein!
%     En viel' olis valmis ma matkallein,
%     mun syömeni on niin syyllinen.
%     Suo että mä pesen sen.
%   \endverse
%   \beginverse
%     Suo että mä ensin huuhdon vaan
%     nämä synkeät, huonot aatteet,
%     suo että mä päälleni ensin saan
%     ne puhtahat, valkeat vaatteet,
%     jotk' ompeli onneni impynen,
%     hän, hämärän impeni ihmeellinen,
%     min kuvaa kannan ma sydämessäin
%     siit' asti kuin hänet mä näin.
%   \endverse
%   \beginverse
%     Me tulemme, äitini armahain!
%     Oi katso, meitä on kaksi!
%     Oi katso, mik' on mulla rinnassain!
%     Niin oisitko rikkahaksi
%     sinä uskonut koskaan kuopustas?
%     Ja katso, me pyydämme siunaustas,
%     sun poikasi synkeä, syyllinen,
%     ja mun impeni puhtoinen.
%   \endverse
%   \beginverse
%     Katso, kuin hän on kaunis ja valkoinen
%     ja muistuttaa niin sua!
%     Hän on niin hellä ja herttainen,
%     vaikk'ei hän lemmi mua,
%     Elä kysele hältä, miks tänne mun toi,
%     mut usko, se niin oli parhain, oi!
%     Ja usko, nyt ett' olen onnellinen
%     kuin aikoina lapsuuden.
%   \endverse
%   \beginverse
%     Elä kysele multa sa laaksoista maan!
%     Ei olleet ne luodut mulle.
%     Mut jos sinun silmäsi tutkii vaan,
%     voin laulaa ma laulun sulle
%     kuin lauloin ma lapsen aikoihin ---
%     kas, lauluna sujuu se paremmin
%     ja kyynelet kuuluvat kantelehen.
%     Niitä muuten ma ilmoita en.
%   \endverse
% \endsong


% \beginsong{Hymyilevä Apollo}[by={Eino Leino}]
%   \beginverse
%     Näin lauloin ma kuolleelle äidillein
%     ja äiti mun ymmärsi heti.
%     Hän painoi suukkosen otsallein
%     ja sylihinsä mun veti:
%     ``Ken uskovi toteen, ken unelmaan, ---
%     sama se, kun täysin sa uskot vaan!
%     Sun uskos se juuri on totuutes.
%     Usko poikani unehes!''
%   \endverse
%   \beginverse
%     Miten mielelläin, niin mielelläin
%     hänen luoksensa jäänyt oisin
%     luo Tuonen virtojen viileäin,
%     mut kohtalot päätti toisin.
%     Vielä viimeisen kerran viittasi hän
%     kuin hän vain viitata tiesi.
%     Taas seisoin ma rannalla elämän,
%     mut nyt olin toinen miesi.
%   \endverse
%   \beginverse
%     Nyt tulkaa te murheet ja vastukset,
%     niin saatte te vasten suuta!
%     Nyt raudasta mulla on jänteret,
%     nyt luuni on yhtä luuta.
%     Kas, Apolloa, joka hymyilee,
%     sitä voita ei Olympo jumalineen,
%     ei Tartarus, Pluto, ei Poseidon.
%     Hymyn voima on voittamaton.
%   \endverse
%   \beginverse
%     Meri pauhaa, ukkonen jylisee,
%     Apollo saapuu ja hymyy.
%     Ja katso! Ukkonen vaikenee,
%     tuul' laantuu, lainehet lymyy.
%     Hän hymyllä maailman hallitsee,
%     hän laululla valtansa vallitsee,
%     ja laulunsa korkea, lempeä on.
%     Lemmen voima on voittamaton.
%   \endverse
%   \beginverse
%     Kun aavehet mieltäsi ahdistaa,
%     niin lemmi! --- ja aavehet haihtuu.
%     Kun murheet sun sielusi mustaks saa,
%     niin lemmi! --- ja iloks ne vaihtuu.
%     Ja jos sua häpäisee vihamies,
%     niin lemmellä katko sen kaunan ies
%     ja katso, hän kasvonsa kääntää pois
%     kuin itse hän hävennyt ois.
%   \endverse
%   \beginverse
%     Kuka taitavi lempeä vastustaa?
%     Ketä voita ei lemmen kieli?
%     Sitä kuulee taivas ja kuulee maa
%     ja ilma ja ihmismieli.
%     Kas, povet se aukovi paatuneet,
%     se rungot nostavi maatuneet
%     ja kutovi lehtihin, kukkasiin
%     ja uusihin unelmiin.
%   \endverse
%   \beginverse
%     Ei paha ole kenkään ihminen,
%     vaan toinen on heikompi toista.
%     Paljon hyvää on rinnassa jokaisen,
%     vaikk' ei aina esille loista.
%     Kas, hymy jo puoli on hyvettä
%     ja itkeä ei voi ilkeä;
%     miss' ihmiset tuntevat tuntehin,
%   \endverse
%   % NOTE: verses are missing!
% \endsong


\beginsong{Soutaja}[by={Unto Kupiainen}]
  \beginverse
    Vieras on virta ja vieras on vene, 
    eivät ne unelmies uomia mene. 
    Ilta on ihmisessä ja aamu on outo; 
    illasta aamuun on ihmisen souto. 
    Illasta aamuun on yöllistä matkaa; 
    jos jaksat uskoa, jaksat jatkaa. 
    Taapäin tuijotat, soudat eteen 
    outoa venettä outoon veteen. 
  \endverse
\endsong


%%%%%%%%%%%%%%%%%%%%%%%%%%%%%%%%%%%%%%%%%%%%%%%%%%%%%%%%%%%%%%%%%%%
%%% LATEST PRINTOUT CONTAINED THE SONGS ABOVE.                  %%%
%%%%%%%%%%%%%%%%%%%%%%%%%%%%%%%%%%%%%%%%%%%%%%%%%%%%%%%%%%%%%%%%%%%
%%% Please try to not change the song numbers above this point. %%%
%%% Add new songs only after this point.                        %%%
%%%%%%%%%%%%%%%%%%%%%%%%%%%%%%%%%%%%%%%%%%%%%%%%%%%%%%%%%%%%%%%%%%%


    \interlinepenalty=1000 % songs: back to the default: avoid breaks within verses the most
    \songpos{3} % songs: back to the default: avoid all breaks whenever possible
    \songcolumns{1} % songs: back to the default: one column per page
    \renewcommand{\lyricfont}{\defaultlyricfont} % songs: back to the default font
    \chordson % songs: back to the default: show chords
  \end{songs}

\end{document}
