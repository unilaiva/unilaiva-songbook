% This command will display some text followed by three dots,
% and aligned to the right of the page. The idea is to use it as a signal to go to
% a different part of the song, for example the chorus. You can use it within the
% last verse or between verses (in such a case it will be put on its own line).
\newcommand{\gotochorus}[1]{\hfill {\normalsize\textit{→ #1\ldots}}\par}

% This command will make a small horizontal space. Use this for example in the
% beginning of chorus lines, if you wish the chorus to stand out visually.
\newcommand{\chorusindent}{\hspace{1em}}

% Displays the parameter text centered in a large font. Use within songs, but outside
% verses. Intended to be used for short mantras for recitation.
\newcommand{\showmantra}[1]{{\vspace{1.5em}\begin{center}\Large #1\end{center}}\vspace{1.0em}}

% environment for 'song feeler'
\newenvironment{feeler}{
  \vspace{2em}
  \small
  \par\noindent\ignorespaces
  \begin{center}\begin{em}
}{
  \end{em}\end{center}
  %\par\noindent\ignorespacesafterend
}

% environment for 'song explanation'
\newenvironment{explanation}{
  \vspace{2em}
  \footnotesize
  \par\noindent\ignorespaces
}{
  \par\noindent\ignorespacesafterend
}

% environment for song translation, to be used inside a song
\newenvironment{translation}{
  \newcommand{\nextverse}{\medskip} % some vertical space between verses
  \beginverse*
  \chordsoff
  %\obeylines
  \vspace{1em}
  \small\textit
  %\par\noindent\ignorespaces
}{
  \endverse
  %\par\noindent\ignorespacesafterend
}

% BEGIN MELODY NOTES

% MELODY NOTES feature
%
% These \note<something> commands display an encircled note name hint. They
% are meant to display (sung) melody notes above the lyrics. The commands
% must be called from within a chord definition. For example: \[\note{A}]
% The note must be written as upper case for transposing to work, even though
% the result is actually presented in lower case.
%
% To enable displaying of the notes, define \newcommand{\showmelodynotes}{}
% in the main document. To not display them, comment it out.
% Also, \color{melodynotecolor} must be defined.
%
% Call \notesoff command between verses to disable showing of notes in the
% following verses of the song.
%
% The recommended use for this feature is to display the first sung non-unison
% melody interval of each song. So just specify the first two melody notes and
% use \notesoff after the first verse.
%
% If in doubt of which of these macros to use, use the \note macro.

% Displays encircled note name;
%   Parameter 1: the note name
%   Parameter 2: X coordinate offset (can be negative)
%   Parameter 3: Y coordinate offset (can be negative)
\newcommand{\noteDELTAXY}[3]{%
  \ifdefined\showmelodynotes%
    % Use the real sharp symbol instead of #, because it fits better within a circle:
    \renewcommand{\sharpsymbol}{\ensuremath{^\sharp}}%
    \notenamesout abcdefg%
    \hspace{#2}\raisebox{#3}{\textmd{\color{melodynotecolor}\footnotesize\textcircled{\tiny\transposehere{#1}}}}%
  \fi%
}
% Displays encircled note name (parameter 1). This is the default and can be used
% anywhere. USE THIS ONE if uncertain!
\newcommand{\note}[1]{%
  \noteDELTAXY{#1}{-0.07em}{0.4em}%
}
% Displays encircled note name (parameter 1). The circle is moved up.
% This version is meant to be used ONLY ON THE FIRST LINE OF A VERSE.
% Use this when there is no chord in the same chord block.
\newcommand{\noteU}[1]{%
  \noteDELTAXY{#1}{0em}{1.00em}%
}
% Displays encircled note name (parameter 1). The circle is moved up and left.
% This version is meant to be used ONLY ON THE FIRST LINE OF A VERSE.
% Use this when there is a one letter chord name on the same chord block,
% a major triad.
\newcommand{\noteUL}[1]{%
  \noteDELTAXY{#1}{-0.64em}{1.00em}%
}
% Displays encircled note name (parameter 1). The circle is moved up and left.
% This version is meant to be used ONLY ON THE FIRST LINE OF A VERSE.
% Use this when there is a longer than one letter chord name on the same
% chord block.
\newcommand{\noteULL}[1]{%
  \noteDELTAXY{#1}{-1.44em}{1.00em}%
}
% Displays encircled note name (parameter 1). The circle is moved up.
% This version is meant to be used ONLY ON THE FIRST LINE OF A VERSE,
% with custom X coordinate offset (parameter 2, can be negative).
\newcommand{\noteUDELTAX}[2]{%
  \noteDELTAXY{#1}{#2}{1.00em}%
}

% Use this between verses to disable displaying of the notes in the following
% verses.
\newcommand{\notesoff}{\let\showmelodynotes\undefined}

% END MELODY NOTES

% BEGIN PRELUDE

% Support for extra song information. To show these in song preludes,
% define \newcommand{\showextrainfo}{} in the main document.
% To define this for a song, use ex={} parameter for \beginsong
\newcommand{\extrainfo}{}
\newsongkey{ex}{\def\extrainfo{}}
               {\def\extrainfo{#1\par}}

% Support for phase hints for songs. To show these in song preludes,
% define \newcommand{\showphaselist}{} in the main document.
% To define this for a song, use ph={} parameter for \beginsong
\newcommand{\phaselist}{}%
\newsongkey{ph}{\def\phaselist{}}
               {\def\phaselist{#1}}

% BEGIN PRELUDE/TAGS

% Support for tags for songs. To show them in song preludes,
% define \newcommand{\showtaglist}{} in the main document.
% Tags are defined as tags= keyval for \beginsong. An example:
% \beginsong{Songname}[tags={fire 1, water 1}. Note the weird
% syntax: the tagname must be followed by ' 1'. Valid tags are
% defined in file 'tags.can'.

\newcommand{\taglist}{}% defined below for each song

\makeatletter%

\newsongkey{tags}{\def\taglist{}\def\SB@rawrefs{}\gdef\songrefs{}}%
                 {\def\taglist{\StrSubstitute{#1}{ 1}{}\par}% Show only tag name, remove ' 1'
                  \def\SB@rawrefs{#1}\SB@parsesrefs{#1}}

\renewcommand\SB@makescripindex{%
  \renewenvironment{SB@lgidx}[1]{%
    \gdef\SB@idxcolhead{##1}%
    % Following three lines are commented out to skip the line used for book name
    %\hbox to\hsize{{\idxbook{##1}}\hfil}%
    %\nobreak%
    %\SB@idxheadsep\nointerlineskip%
  }{%
    \mark{\noexpand\relax}%
    \penalty-20\vskip3\p@\@plus3\p@\relax%
  }%
  \renewenvironment{SB@smidx}[1]
    {\begin{SB@lgidx}{##1}}{\end{SB@lgidx}}%
  \renewcommand\idxentry[2]{%
    \SB@ellipspread{\idxscripfont\relax\SB@idxcolhead}% ##1 -> SB@idxcolhead to display 'book name' instead of 'chapter number'
                   {{\idxrefsfont\relax##2}}%
    \SB@toks\expandafter{\SB@idxcolhead}%
    \mark{\noexpand\SB@idxcont{\the\SB@toks}}%
  }%
  \renewcommand\idxaltentry[2]{\SB@erridx{a scripture}}%
  \SB@displayindex%
}

\makeatother%

% END PRELUDE/TAGS


% Redefine displaying of song prelude extension. This version adds
% some of the new song keyvals introduced above. \showrefs is removed,
% because we use the scripture references as tags and they are displayed
% with \taglist.
\makeatletter%
\renewcommand{\extendprelude}{%
  \ifdefined\showtaglist%   following is similar to \showauthors
    {\setbox\SB@box\hbox{\hfill\sffamily\taglist}%
     \ifdim\wd\SB@box>\z@\unhbox\SB@box\par\vspace{-1em}\fi% Go up one line if tag list was non-empty
    }%
  \fi%
  \showauthors%
  \ifdefined\showextrainfo%
    {\sffamily\extrainfo}%
  \fi%
}
\makeatother

% Redefine the song prelude display. Only one line is modified (see comment)
% to show the phase list on the same line with the main title.
\makeatletter
\renewcommand\makeprelude{
  \resettitles%
  \ifslides%
    \hbox to\hsize{{\hfil\stitlefont\relax\songtitle\hfil}}%
    \vskip5\p@%
    \hbox to\hsize{%
      \hfil%
      \vbox{%
        \divide\hsize\tw@\parskip\p@\relax%
        \centering\small\extendprelude%
      }%
      \hfil%
    }%
  \else%
    \ifdim\songnumwidth>\z@%
      \setbox\SB@boxii\hbox{{\SB@colorbox\snumbgcolor{%
        \hbox to\songnumwidth{%
          \printsongnum{\thesongnum}\hfil%
        }%
      }}}%
    \fi%
    \setbox\SB@box\vbox{%
      \ifdim\songnumwidth>\z@%
        \SB@dimen\wd\SB@boxii%
        \advance\SB@dimen3\p@%
        \ifpagepreludes\multiply\SB@dimen\tw@\fi%
        \advance\hsize-\SB@dimen%
      \fi%
      \ifpagepreludes\centering\else\SB@raggedright\fi%
      \offinterlineskip\lineskip\p@%
      {\stitlefont\relax%
       % The following line is modified (everything between \songtitle and \par is added:
       \songtitle{%
         \ifdefined\showphaselist{\normalfont\sffamily\tiny\hfill\raisebox{1em}{\phaselist}}\\\fi%
       }\par%
       \nexttitle%
       \foreachtitle{(\songtitle)\par}}%
      \ifdim\prevdepth=\z@\kern\p@\fi%
      \parskip\p@\relax\tiny%
      \extendprelude%
      \kern\z@%
    }%
    \ifdim\songnumwidth>\z@%
      \hbox{%
        \ifdim\ht\SB@boxii>\ht\SB@box%
          \box\SB@boxii%
          \kern3\p@%
          \vtop{\box\SB@box}%
        \else%
          \SB@colorbox\snumbgcolor{\vbox to\ht\SB@box{{%
            \hbox to\songnumwidth{%
              \printsongnum{\thesongnum}\hfil%
            }\vfil%
          }}}%
          \kern3\p@%
          \box\SB@box%
        \fi%
      }%
    \else%
      \unvbox\SB@box%
    \fi%
  \fi%
}
\makeatother

% END PRELUDE

% The following rewrite of the \SB@@@beginsong command adds songs to the basic TOC also.
% Changes: lines between and including the first and last '\hypersetup' commands were added.
\makeatletter
\renewcommand\SB@@@beginsong{%
  \global\SB@stanzafalse%
  \setbox\SB@chorusbox\box\voidb@x%
  \SB@gotchorusfalse%
  \setbox\SB@songbox\vbox\bgroup\begingroup%
    \ifnum\SB@numcols>\z@\hsize\SB@colwidth\fi%
    \leftskip\z@skip\rightskip\z@skip%
    \parfillskip\@flushglue\parskip\z@skip%
    \SB@raggedright%
    \global\SB@transposefactor\z@%
    \global\SB@cr@{\\}%
    \protected@edef\@currentlabel{\p@songnum\thesongnum}%
    \setcounter{versenum}{1}%
    \SB@prevversetrue%
    \meter44%
    \resettitles%
    \SB@addtoindexes\songtitle\SB@rawrefs\songauthors%
    \nexttitle%
    \foreachtitle{\expandafter\SB@addtotitles\expandafter{\songtitle}}%
    \resettitles%
    \lyricfont\relax%
    \SB@setbaselineskip%
    \hypersetup{bookmarksdepth=0}% the first added line
    \phantomsection%
    \addcontentsline{toc}{subsection}{%
      \numberline{\thesongnum}%
      \songtitle\relax%
      \ifdefined\showphaselistintoc{%
        \hspace{0.15em}\scriptsize\textsuperscript{\tiny\phaselist}}\fi%
      }%
    \hypersetup{bookmarksdepth=2}% the last added line
}

% The following rewrite adds only \color{mbarcolor} command to change the color of the measure bar
\renewcommand\SB@makembar[2]{%
  \ifSB@inverse\else%
    \ifSB@inchorus\else\SB@errmbar\fi%
  \fi%
  \ifhmode%
    \SB@skip\lastskip\unskip%
    \setbox\SB@box\lastbox%
    \copy\SB@box%
    \ifvbox\SB@box%
      \begingroup%
        \setbox\SB@boxii\copy\SB@box%
        \vbadness\@M\vfuzz\maxdimen%
        \setbox\SB@boxii%
          \vsplit\SB@boxii to\maxdimen%
      \endgroup%
      \long\edef\SB@temp{\splitfirstmark}%
      \ifx\SB@temp\SB@measuremark%
        \penalty100\hskip1em%
      \else%
        \penalty100\hskip\SB@skip%
      \fi%
    \else%
      \penalty100\hskip\SB@skip%
    \fi%
  \fi%
  \ifvmode\leavevmode\fi%
  \setbox\SB@box\hbox{{\meterfont\relax#1}}%
  \setbox\SB@boxii\hbox{{\meterfont\relax#2}}%
  \SB@dimen\wd\ifdim\wd\SB@box>\wd\SB@boxii\SB@box\else\SB@boxii\fi%
  \SB@dimenii\baselineskip%
  \advance\SB@dimenii-2\p@%
  \advance\SB@dimenii-\ht\SB@box%
  \advance\SB@dimenii-\dp\SB@box%
  \advance\SB@dimenii-\ht\SB@boxii%
  \advance\SB@dimenii-\dp\SB@boxii%
  \let\SB@temp\relax%
  \ifdim\SB@dimen>\z@%
    \advance\SB@dimenii-.75\p@%
    \def\SB@temp{\kern.75\p@}%
  \fi%
  \SB@maxmin\SB@dimen<{.5\p@}%
  \SB@maxmin\SB@dimenii<\z@%
  \vbox{%
    \mark{\SB@measuremark}%
    \hbox to\SB@dimen{%
      \hfil%
      \box\SB@box%
      \hfil%
    }%
    \nointerlineskip%
    \hbox to\SB@dimen{%
      \hfil%
      \box\SB@boxii%
      \hfil%
    }%
    \SB@temp%
    \nointerlineskip%
    \hbox to\SB@dimen{%
      \hfil%
      \color{mbarcolor}\vrule\@width.5\p@\@height\SB@dimenii%
      \hfil%
    }%
  }%
  \meter{}{}%
}
\let\SB@mbar\SB@makembar
\makeatother
