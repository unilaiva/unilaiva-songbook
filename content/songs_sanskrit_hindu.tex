% Sanskrit language bhajans and mantras from the Indian subcontinent

\beginsong{Om Purnam \\ Purnamadah}[by={traditional, Shantala \& Satyaa \& Sari},ex={from Upaniṣad, the first prayer},ph={I, II}]
  \beginverse
    |\[Am]\[Am/B] |\[Am]\[Am/B] |\[Am]\[Am/B] |\[Am]\[Am/B] |
  \endverse
  \beginverse
    |\[Am\noteULL{A}]Pur\[\noteU{B}]nama\[Am/B]dah |\[Am]Purnami\[Am/B]dam |
    |\[Am]Purnat \[Am/B]Purnama|\[Am]dachya\[Am/B]te |
    |\[Am]Purnasya \[Am/B]Purna|\[Fmaj7]mada\[G]ya |
    |\[Am] Purnam|evava\[Fmaj7]shishya|\[G6]te |
  \endverse
  \beginverse
    |\[Am]Om |\[Fmaj7] \[G6] |
    |\[Am] |\[Fmaj7] \[G6] |\[Gadd9] |\[G(5)] |
  \endverse
  \begin{feeler}
    That is the whole. This is the whole.\\
    From wholeness emerges wholeness.\\
    Wholeness coming from wholeness,\\
    wholeness still remains. \\
  \end{feeler}
  \begin{explanation}
    "\ldots contains the secret of the mystic approach towards life. This small sutra contains the 
    essence of the Upanishadic vision. Neither before nor afterwards has the vision been 
    transcended; it still remains the Everest of human consciousness. The Upanishadic vision is 
    that the universe is a totality, indivisible; it is an organic whole. The parts are not 
    separate, we are all existing in a togetherness: the trees, the mountains, the people, the 
    birds, the stars, howsoever far away they may appear - don't be deceived by the appearance - 
    they are all interlinked, all bridged. Even the smallest blade of grass is connected to the 
    farthest star, and it is as significant as the greatest sun. Nothing is insignificant, nothing 
    is smaller than anything else. The part represents the whole, just as the seed contains the 
    whole\ldots " –Osho: Philosophia Ultima
  \end{explanation}
\endsong


\beginsong{Mahamrityunjaya Mantra}[index={Om Tryambakam},ph={III}]
  \beginverse
    |\[C\noteUL{E}]Om Tryamba\[\noteU{D}]ka|\[G]m Yajamah|\[Dm]e |\[Fmaj7] |
    |\[C] Sugandhim |\[G]Pushti Vardha|\[Dm]nam | -
    Ur|\[F]va Rukamiva |\[C]Bandhanan Mri|\[G]tyor | -
    Muk|\[Dm]shiya Maamritat \echo{Muk|shiya Maamritat}
    Muk|\[G]shiya Maamritat \echo{Muk|shiya Maamritat} |
  \endverse
  \begin{feeler}
    We Meditate on the Three-eyed reality \\
    Which permeates and nourishes all like a fragrance.  \\
    May we be liberated from death for the sake of immortality, \\ 
    Even as the cucumber is severed from bondage to the creeper.
  \end{feeler}
  \begin{explanation}
    Mahamrityunjaya Mantra (maha-mrityun-jaya) is one of the more potent of the ancient Sanskrit 
    mantras. Maha mrityunjaya is a call for enlightenment and is a practice of purifying the karmas 
    of the soul at a deep level. It is also said to be quite beneficial for mental, emotional, and 
    physical health.
  \end{explanation}
\endsong


\beginsong{Moola Mantra}[by={traditional, Seven, Deva Premal},index={Om Satchitananda Parabrahma},tags={source 1},ph={III, IV}]
  \beginchorus
    \[\noteU{E}]Om |\[Am\noteULL{A}]Satchitananda |Parabrahma |
    |Purushothama Para|matma
    Sri |\[F]Bhagavati Same|\[G]tha
    Sri |\[Am]Bhagavate Nama|ha
  \endchorus
  \beginchorus
    Hari |\[Am]Om Tat |Sat, Hari |\[G]Om Tat |Sat
    Hari |\[C]Om Tat |\[G]Sat, Hari |\[Fmaj7]Om Tat |\[Am]Sat
  \endchorus

  \begin{feeler}
    Oh Divine Force, Spirit of All Creation,\\
    Highest Personality, Divine Presence,\\
    manifest in every living being.\\
    Supreme Soul manifested\\
    as the Divine Mother and\\
    as the Divine Father.\\
    I bow in deepest reverence.\\
  \end{feeler}
  \begin{explanation}
    Moola mantra evokes the living God, asking protection and freedom from all sorrow
    and suffering. It is a prayer that adores the great creator and liberator, who out of love and
    compassion manifests, to protect us, in an earthly form.  The calmness that the mantra can
    give is to be experienced, not spoken about. Here is the key with which any door to spiritual
    treasure could be opened. A tool which can be used to achieve all desires. A medicine which
    cures all ills. Just like when you call a person he comes and makes you feel his presence, the
    same manner when you chant this mantra, the supreme energy manifests everywhere around you. As
    the Universe is Omnipresent, the supreme energy can manifest anywhere and any time. It is also
    very important to know that the invocation with all humility, respect and with great necessity
    makes the presence stronger.
    \begin{description}
      \item[Om] Calling on the highest energy of all there is. It is said 'In the beginning was the
        Supreme word and the word created every thing. That word is Om'. If you are meditating in
        silence deeply, you can hear the sound Om within. The whole creation emerged from the sound
        Om. It is the primordial sound or the Universal sound by which the whole universe vibrates.
        This divine sound has the power to create, sustain and destroy, giving life and movement to
        all that exist.
      \item[Sat] Truth. The formless. The all penetrating existence that is formless, shapeless,
        omnipresent, attributeless, and qualityless aspect of the Universe, experienced as emptiness
        of the Universe. The body of the Universe that is static. Everything that has a form and can
        be sensed evolved out of this. So subtle that it is beyond all perceptions. It can only be
        seen when it has become gross and has taken form. We are in the Universe and the Universe is
        in us. We are the effect and Universe is the cause and the cause manifests itself as the
        effect.
      \item[Chit] The Pure Consciousness of the Universe that is infinite, omni-present
        manifesting power of the Universe. Out of this is evolved everything that we call Dynamic
        energy or force. It can manifest in any form or shape. It is the consciousness manifesting
        as motion, as gravitation, as magnetism, etc. Also manifesting as the actions of the body,
        as thought force. The Supreme Spirit.
      \item[Ananda] Pure bliss, love, joy and friendship nature of the Universe. When you experience
        either the Supreme Energy in this Creation (Sat) and become one with the Existence or
        experience the aspect of Pure Consciousness (Chit), you enter into a state of Divine Bliss
        and eternal happiness (Ananda).
      \item[Parabrahma] The Supreme creator being in his Absolute aspect; beyond space and time.
        The essence of the Universe that is with and without form.
      \item[Purushothama] The energy that incarnates as an Avatar in human form to help and guide
        mankind and relate closely to the beloved creation.  This has different meanings. Purusha
        means soul and Uthama means the supreme, the Supreme spirit. It also means the supreme
        energy of force guiding us from the highest world. Purusha also means Man, and Purushothama
        is the energy that incarnates as an Avatar to help and guide Mankind and relate closely to
        the beloved Creation.
      \item[Paramatma] Supreme inner energy that is immanent in every creature and in all beings,
        living and non-living. Who comes to me in my heart, and becomes my inner voice whenever I
        ask. It's the force that can come to you whenever you want and wherever you want to guide
        and help you.
      \item[Sri Bhagavathi] The divine mother, the power aspect of creation. The female aspect,
        which is characterized as the Supreme Intelligence in action, the Power (The Shakti). It is
        referred to the Mother Earth (Divine Mother) aspect of the creation.
      \item[Sametha] Together or in communion with.
      \item[Sri Bhagavathe] The Male aspect of the Creation, which is unchangeable and permanent.
      \item[Namaha] Salutations or prostrations to the Universe that is Om and also has the
        qualities of Sat Chit Ananda, that is omnipresent, unchangeable and changeable at the same
        time, the supreme spirit in a human form and formless, the indweller that can guide and help
        in the feminine and masculine forms with the supreme intelligence. I thank you and
        acknowledge this presence in my life. I seek your presence and guidance all the time.
      \item[Hari om tat sat] God is the truth. Hari is another name of Lord Vishnu.
    \end{description}
  \end{explanation}
\endsong


\beginsong{Gayatri Mantra}[index={Om Bhur Bhuvah Svaha\ldots},tags={enlightenment 1},ph={III}]
  \beginverse
    \[Am] \[\noteU{A}]Om |\[G\noteUL{B}]Bhur Bhuvah Sva|\[Am]ha
    Tat |\[G]Savi|\[Am]tur Va|\[G]ren|\[F]yam
    Bhar|\[G]gho Devasya |\[C]Dheemahi
    Dhi|\[G]yo Yo |\[C]Nah Pra|\[G]choda|\[Am]yat
  \endverse
  \begin{explanation}
    A prayer of praise that awakens the vital energies and gives liberation and deliverance from
    ignorance. This mantra is known to impart wisdom, understanding, and enlightenment. This is
    said to be the oldest and most powerful of mantras, being thousands of years old. It purifies
    the person chanting it as well as the listener as it creates a tangible sense of well being in
    whoever comes across it.

    We meditate on that most adorable, desirable and enchanting luster and brilliance of our
    Supreme Being, our Source Energy, our Collective Consciousness who is our creator, inspirer
    and source of eternal Joy.  May this Light inspire and guide our mind and open our hearts.
    That Divine Illumination which pervades the physical plane astral plane and the celestial
    plane. That which is the most adorable. On that Divine Radiance we Meditate. May that
    Enlighten Our Intellect and Awaken our Spiritual Wisdom.
    \brk
    \paragraph{Om Bhur Bhuvah Svaha} Preamble to the main mantra; means that we invoke in our prayer
      and meditation the One who is our inspirer, our creator and the abode of supreme Joy.  It also
      means, we invoke the earthly, physical world, the world of our mind, and the world of our
      soul.
    \begin{description}
      \item[Om] God/Brahma, Divine Illumination which pervades
      \item[Bhur] Pranic energy
      \item[Bhuvah] Destroyer of sufferings
      \item[Svaha] Happiness bestowing
    \end{description}
    \paragraph{Tat Savitur Varenyam} Divine Illumination, which is the Most Adorable
    \begin{description}
      \item[Tat] THAT, denoting the Supreme Being, God or Spirit
      \item[Savitur] The radiating source of life with the brightness of the Sun. Bright Sun/God
        (also a deva some call upon using this mantra)
      \item[Varenyam] Most adorable, most desirable, greatest
    \end{description}
    \paragraph{Bhargho Devasya Dheemahi}
    \begin{description}
    \item[Bhargho] Luster and splendor; destroyer of misdeeds
      \item[Devasya] Divine or Supreme God
      \item[Dheemahi] "We meditate upon"; knowledge imparted/understood
    \end{description}
    \paragraph{Dhiyo Yo Nah Prachodayat}
    \begin{description}
      \item[Dhiyo] Our understanding of reality, our intellect, our intention; Intelligence
      \item[Yo] He who
      \item[Nah] Our
      \item[Prachodayat] May he inspire, guide; enlightenment
    \end{description}
  \end{explanation}
\endsong


\beginsong{Om Asato Ma \\ Pavamana Mantra}[ex={from Bṛhadāraṇyaka Upaniṣad},tags={transcendence 1},ph={III}]
  \beginverse
    |\[Am\noteULL{A}\noteU{G}]O|\[Em\noteULL{E}]om asato ma |\[Am]sat gamaya |
    |\[Em] Tamaso ma |\[C]jyotir gama|\[G]ya
    Mrit|\[Em]yor ma amri|\[Am]tam gamaya | |
  \endverse
  \textnote{suomeksi:}
  \beginverse
    \notesoff
    |^O|^om Johda minut |^totuuteen |
    |^ Pimeydestä |^va|^loon
    Tiedot|^tomuudesta tietoi|^suuteen | |
  \endverse
  \begin{translation}
    Lead me from illusion to reality,
    from darkness to light,
    from death to immortality.
  \end{translation}
\endsong


\beginsong{Prabhu Aap Jago}[by={Carioca},tags={source 1},ph={III}]
  \newchords{chords_prabhu_a}\newchords{chords_prabhu_b}
  \beginchorus\memorize[chords_prabhu_a]
    \[^\noteU{C}]Pra\[^\noteU{D}]bhu |\[C\noteUL{E}]Aap Jago Prabhu |\[Fmaj7]Aap Jago
    Prabhu |\[Dm]Aap Jago Para|\[G]mathma Jago
  \endchorus
  \notesoff
  \beginverse\memorize[chords_prabhu_b]
    Mere |\[Em]Sarva Jago Sar|\[Am]vatra Jago
    Prabhu |\[Dm]Aap Jago Para|\[G]mathma Jago
    Mere |\[Em]Sarva Jago Sar|\[Am]vatra Jago
    Prabhu |\[Dm]Aap Jago Para|\[G]mathma Ja|\[C]go | -
  \endverse
  \textnote{in English:}
  \beginchorus\replay[chords_prabhu_a]
    |^Cease the cause of |^suffering
    Illumi|^nate the cause of |^joy
  \endchorus
  \beginverse\replay[chords_prabhu_b]
    |^Cease the cause of |^suffering
    Illumi|^nate the cause of |^love
    |^Cease the cause of |^suffering
    Illumi|^nate the |^cause of |^love | -
  \endverse
  \begin{feeler}
    God awaken, God awaken in me, God awaken everywhere.\\
    May love awaken; may love awaken everywhere.
  \end{feeler}
\endsong


\beginsong{Shakti Kundalini \\ Om Mata Om Kali}[tags={Divine Mother 1},ph={III, IV}]
  \meter{4}{4}
  \beginchorus
    \[\noteU{F}]Om |\[Dm\noteULL{E}]Ma\[\noteU{D}]ta Om |Kali |
    |\[C]Durga Devi na|\[Dm]mo namaha
  \endchorus
  \beginchorus
    |\[Dm]Shakti kunda|\[C]lini |\[B&]Jagadambe Ma|\[A]ta |
    |\[Dm]Shakti kunda|\[C]lini |\[B&]Jagadam\[A]be Ma|\[Dm]ta
  \endchorus
  \begin{feeler}
    I bow unto the Divine Mother and Her many feminine aspects: Kali, remover of delusion and
    ignorance; Divine Goddess Durga; Shakti, universal life force and consort to Shiva; and
    Kundalini, the Goddess energy that rises within us. Praise to the Mother of the World!
  \end{feeler}
\endsong


\beginsong{Jay Shri Ma \\ Ananda Ma \\ Kali Ma}[tags={Divine Mother 1},ph={IV}]
  \beginchorus
    |\[Dm\noteULL{D}]Jay Shri \[F\noteUL{A}]Ma |Kali Kali \[Gm]Ma |
    |\[Gm]Jay Shri \[Dm]Ma | |
  \endchorus
  \beginchorus
    |\[F]Ananda Ma |\[C]Durga Devi |
    |\[Gm]Jagadambe Shri |\[Dm]Ma |
  \endchorus
\endsong


\beginsong{Jay Ambe}[tags={Divine Mother 1},ph={IV}]
  \beginchorus
    |\[Dm\noteULL{D}]Jay Am\[\noteU{A}]be |\[C]Jagadam\[Am]be |
    |\[F]Mata Bha\[C]vani ki |\[Dm]Jay Ambe |
  \endchorus
  \beginchorus
    |\[Dm] Durgati Nashini |\[F]Durga Jaya Jaya |
    |\[C] Kala Vinashini |\[Dm]Kali Jaya Jaya |
  \endchorus
  \beginchorus
    |\[C]Uma Rama Brah|\[F]mani Jaya Jaya |
    |\[C]Radha \[Am]Rukamani |\[Dm]Sita Jaya Jaya |
  \endchorus
\endsong


\beginsong{Saraswati}[tags={Divine Mother 1},ph={II, III}]
  \beginverse
    \[\noteU{D}]Sa\[\noteU{E}]ras|\[Dm\noteULL{F}\noteU{E}]wa\[\noteU{D}]ti | Maha|laxmi | -
    Durga |\[F]Devi |\[C]Nama|\[Dm]ha | -
  \endverse
  \beginverse
    Saras|\[Gm]wati | Maha|\[F]laxmi | -
    Durga |\[B&]Devi |\[C]Nama|\[Dm]ha | -
  \endverse
\endsong


\beginsong{Devi Mantra \\ Sarva Mangala}[tags={Divine Mother 1, Shiva 1},ph={III, IV}]
  % \capo{3}
  \textnote{part A: Devi Mantra}
  \beginchorus
    |\[Am\noteULL{A}]Sarva Mangala |Man\[\noteU{C}]ga\[\noteU{A}]ly\[Em\noteULL{B}]e |
    |\[G]Shive Sarv|\[Em]artha Sadhi\[Am]ke |
    |Sharanye Tryambake |Gau\[Em]ri
    Nara|\[Am]yan\[G]i Na|\[Em]mostut\[Am]e
    Nara|\[Am]yan\[G]i Na|\[Em]mostut\[Am]e | | |
  \endchorus
  \textnote{part B: Om Namah Shivaya}
  \beginverse
    \chorusindent \lrep |\[Am]Om Namah Shivaya, |\[C]Om \[G]Nama Shiva | \rrep\rep{3}
    \chorusindent |\[Am]Om Namah Shivaya, |\[C]Om \[G]Nama Shiv|\[Am]aya | \textsuperscript{1}( |)
  \endverse
    \textnote{D.C. al Fine}
  \beginverse
    \chorusindent \lrep |\[Am]Shivaya, Shivaya, |\[Em]Shivay\[Am]a | \rrep\rep{3}
    \chorusindent |Shivaya, Shivaya |
    \chorusindent \lrep |\[C]Om \[G]Namah Shiv|\[Am]aya | \rrep
  \endverse
  \begin{feeler}
     \textbf{Devi Mantra:}\par
     Welcome to you O Narayani; who is the positiveness in all the auspicious,
     one who is so auspicious herself and has all auspicious qualities,
     The provider of protection, the one with 3 eyes and a beautiful face;
     we salute you, O Narayani.
  \end{feeler}
\endsong


\beginsong{Mataji}[by={trad., \textmd{combined by:} Elisabet Just},tags={Divine Mother 1},ex={saṃskṛtam, português},ph={IV}]
  %\capo{3}
  \beginverse
    |\[Am\noteULL{A}]Ayi Ayi Ayi Ayi Di|wa\[\noteU{B}]li \[\noteU{A}]Hai Ye Ayi |
    |\[G]Aise Shubhawa\[C]sar Par Hum |\[Em]Puje Maha \[Am]Lakshmi |
    |\[Am]Sabhi Devi Devata |Aapa Hi Ko Puje |
    |\[C]Nirmala \[Asus2/E]Ma, O Maiya |\[G]Nirmala \[Am]Ma
    \lrep O Chindra|wara, Wali Maha, Laksh|\[(C)]mi Mata\[Am]ji \rrep
  \endverse
  \beginchorus
    \lrep Ô iê iê i|\[Am]ê, ô iê iê |Xoroodô \rrep {\small\hfill \textsuperscript{*}\lrep |Ayi Ayi\ldots\rrep}
    Olomi ai|\[G]ê Xorô \[C] Ómanféé |\[Em] Xoroo\[Am]dô {\small\hfill \textsuperscript{*}|Aise Shubhawa\ldots}
  \endchorus
  \beginverse
    Ô iê iê i|\[Am]ê, ô iê iê |Xoroodô {\small\hfill \textsuperscript{*}|Sabhi Devi\ldots}
    Ô iê iê i|\[C]ê, \[Asus2/E]ô iê iê |\[G]Xoroo\[Am]dô {\small\hfill \textsuperscript{*}|Nirmala\ldots}
    \lrep Ô iê iê i|\[Am]ê, ô iê iê |Xoroodô \rrep {\small\hfill \textsuperscript{*}\lrep |Ayi Ayi\ldots \rrep}
  \endverse
  \beginchorus\musicnote{decelerando}
    |\[Am]Om Mani Padme Hum | | {\small\hfill \textsuperscript{*}Ô iê iê i|ê, ô iê iê |Xoroodô}
  \endchorus
  \beginchorus
    |\[Am]Om Mani Padme Hum |\[G]Om Mani Padme Hum |
    |\[F]Om Mani \[Em]Padme |\[Am]Hum |
    \rep{4}
  \endchorus
  \beginchorus
    |\[Am]Om Om Mani |\[G]Padme \[Am]Hum |
    \rep{4}
  \endchorus
  \beginchorus
    |\[Am]Om Mani Padme Hum |\[G]Om Mani Padme Hum |
    |\[F]Om Mani \[Em]Padme |\[Am]Hum
  \endchorus
  \beginchorus
    Eu |\[Am]vi mamãe Oxum na cacho|\[G6]eira sen-| {\small\hfill \textsuperscript{*1}|Om Mani\ldots}
    |\[F]tada na bei\[G7]ra do r|\[Am]io {\small\hfill \textsuperscript{*1}|Om Mani\ldots}
  \endchorus
  \beginchorus
    Colhendo |\[Dm]lírio, lírio lê \[G] colhendo |
    |\[C]lírio, lírio lá \[F] colhendo |
    |\[Bm7&5]lírio pra enfei\[Em]tar o seu con|\[Am]gá |
    \rep{4}
  \endchorus
  \gotochorus{Om Mani Padme Hum}
  \begin{translation}
    The day of Diwali has arrived.
    On this auspicious occasion allow us to worship Shri Mahalaxmi.
    All the Gods and Goddesses worship you,
    oh Mahalaxmi Mataji Goddess of Chindwara.
    \nextverse
    I saw Mother Oxum at the waterfall
    sitting on the river's edge
    gathering lilies, there picking lilies,
    lilies to adorn her congá.
  \end{translation}
  \begin{explanation}
    \begin{description}
    \item[Mataji:] a Hindi term meaning "respected mother".
    \item[Diwali:] the yearly Hindu festival of lights, when prayers are offered to
    \textbf{Lakshmi}, goddess of prosperity and fortune.
    \item[Oxum:] \emph{see song \emph{Ide Were} for explanation.}
    \end{description}
  \end{explanation}
\endsong


\beginsong{Shiva Shambo}[tags={Shiva 1, Divine Mother 1},ph={III, IV}]
  \beginchorus\memorize
    \lrep |\[Am\noteULL{A}]Jaya Jaya Shiva \[^\noteU{B}]Sham|bo, |\[G]Jaya Jaya Shiva Sham|\[Am]bo | \rrep
    \lrep |\[Am]Maha Devi Sham|bo, |\[G]Maha Devi Sham|\[Am]bo | \rrep
  \endchorus
  \notesoff
  \beginchorus
    \lrep |^Shiva Shiva Shiva Sham|bo, |^Shiva Shiva Shiva Sham|^bo | \rrep
    \lrep |^Maha Deva Sham|bo, |^Maha Deva Sham|^bo | \rrep
  \endchorus
\endsong


\beginsong{Haidakandhi}[tags={Shiva 1, Vishnu 1},ph={III}]
  \meter{4}{4}
  \beginchorus
    |\[Am] \[\noteU{A}]Om Namah \[\noteU{B}]Shi|\[F]vaya Namah |\[C]Om |\[G]Haidakandhi |
  \endchorus
  \notesoff
  \beginchorus
    |\[Am] Hari Hari |\[F] Hari Hari |\[C] Hari Hari |\[G]Shankara |
  \endchorus
\endsong


\beginsong{Om Namah Shivaya}[tags={Shiva 1},ph={IV}]
  \beginchorus\memorize % memorize chords even though in 'chorus'
    \textnote{intro:}
    |\[Am\noteULL{A}]Om Namah Shi|\[F\noteUL{C}]vaya; |\[G]Om Namah Shi|\[Am]vaya |
  \endchorus
  \notesoff
  \beginchorus
    |^ Shivaya |^Namaha; |^ Shivaya |^Namaho |
  \endchorus
  \beginchorus
    |^Sham Bol ^Shankara |^Namah ^Shivaya; |\replay 
    |^Girija ^Shankara |^Namah ^Shivaya |
  \endchorus
  \beginchorus
    |^Aruna^chala Shiva |^Namah Shi^vaya; |\replay 
    |^Aruna^chala Shiva |^Namah ^Shivaya
  \endchorus
  \beginchorus
    Hari |\[C]Om Namah Shi|\[G]vaya; |\[F]Om Namah Shi|\[Am]vaya |
  \endchorus
  \beginchorus
    \textnote{outro:}
    |^Om Namah Shi|^vaya; |^Om Namah Shi|^vaya |
  \endchorus
\endsong


\beginsong{Ganesha Mantra \\ Removing of obstacles Mantra }[index={Om Gam Ganapatayei Namaha},by={Prembabanda},tags={Ganesha 1},ph={I, IV}]
  \showmantra{Om Gam Ganapatayei Namaha}
  \begin{feeler}
    Salutations to the remover of obstacles.
  \end{feeler}
  \begin{explanation}
    This sound formula assists us in the removal of obstacles. In order for that to happen there
    is no need to know the exact nature of the hindrances. Just the awareness and recognition that
    there are obstacles and then chanting this mantra with the intention for resolve is enough.
    This mantra unifies us within. When there is oneness there are no obstacles. This mantra is
    also used for the beginning of any endeavor. Whenever we start anything anew we can bless the
    project with the energy of Ganesh through this mantra.
    \vspace{2em}
    \begin{description}
      \item[Gam] is the seed sound for Ganesh.
      \item[Ganapati] is another name for Ganesh - the Remover of Obstacles and the of
        Oneness/Unity.
      \item[Yei] is a sound that activates shakti/energy.
    \end{description}
  \end{explanation}
  \begin{center}%
    \vfill%
    \includegraphics[width=0.382\textwidth]{ganesha_bw_transparent_background_1280x1232.png}%
  \end{center}
  \textnote{song: part A}
  \beginchorus\memorize % memorize chords even though in 'chorus'
    |\[Em\noteULL{E}]Om \[^\noteU{B}]Parvati Patayei |\[Bm]Hara Hara Hara Mahadeva |
    |\[C] Gajana\[D]nam Bu|\[Em]ta |
  \endchorus
  \notesoff
  \beginchorus
    |^Ganadi Sevatam |^ Kapitha Jambu |
    |^ Phalacha^ru |^Bhakshanam |
  \endchorus
  \beginchorus
    |^Umasutam Shoka |^ Vinasha Karakam |
    |^ Namami ^Vigneshvara |^Pada Pankajam |
  \endchorus
  \textnote{part B}
  \beginchorus
    |\[Em]Om Gam Ganapata|yei Nama\[Bm]ha |
    |\[Em]Om Gam Ganapata|yei Nama\[Bm]ha |
    |\[G]Om Gam Ganapata|yei Nama\[Bm]ha |
    |\[Em]Om Gam Ganapata|yei Nama\[Bm]ha |
  \endchorus
\endsong


\beginsong{Govinda Hari Om}[tags={Vishnu 1, Krishna 1},ph={III}]
  \meter{4}{4}
  \beginverse
    |\[Am\noteULL{A}]Go\[\noteU{E}]vin|\[Dm]da | Hari |\[Am]Om Hari |\[E]Hari |
    |\[Am]Gopa|\[Dm]la | Hari |\[Am]Om|\[G] |
    |\[C]Sada |\[Dm]Sadhana | Ananda |\[Am]Bhavana |
    \lrep |\[Dm]Vishnu |\[Am]Sadhana |\[E]Hari |\[Am]Om | \rrep
  \endverse
\endsong


\beginsong{Om Shanti Om \\ Peace Mantra}[tags={peace 1},ph={I, II, III}]
  \beginchorus\memorize
    |\[Am\noteULL{A}]Om |\[Em\noteULL{B}]Shanti |\[Am]Om | -
  \endchorus
  \notesoff
  \beginverse
    Om |\[Dm]Shanti |\[F]Shanti |\[Am]Shantihi | |
  \endverse
  \textnote{Outro:}
  \beginverse
    |^Om |^Shanti |^Om | |
  \endverse
  \begin{feeler}
    Peace in my heart, peace with each other, peace in the cosmos.
  \end{feeler}
\endsong


\beginsong{Dhanvantre Mantra \\ Healing Mantra}[index={Om Shree Dhanvantre},tags={healing 1},ph={III}]
  \showmantra{Om Shree Dhanvantre Namaha}
  \begin{feeler}
    Salutations to the being and power of the Celestial Healer.
  \end{feeler}
  \begin{explanation}
    \textbf{Dhanvantari} is the celestial healer. This mantra helps us find the right path to 
    healing, or directs us to the right health practitioner. In India it is also commonly chanted 
    during cooking in order for the food to be charged with healing vibrations – either to prevent 
    disease or assist in healing for those who are sick. This mantra can be chanted for any 
    situation that one would like to be healed or remedied. Good to remember and be open to the 
    path of healing not necessarily looking the way we expect it!
  \end{explanation}
\endsong


\beginsong{Om Namo Bhagavate Vasudevaya}[tags={liberation 1},ph={II, III}]
  \showmantra{Om Namo Bhagavate Vasudevaya}
  \begin{feeler}
    Salutations to the Indweller who is omnipresent, omnipotent, immortal and divine.
  \end{feeler}
  \begin{explanation}
    \textbf{Vasudeva} is the individual aspect of God that dwells inside of us. This mantra frees 
    our minds and spirits from negative patterns in this life. Regular and consistent practice of 
    this mantra gives us a complete spiritual freedom: it frees us from the cycle of rebirth and 
    helps us realize ourselves as a manifestation of transcendent divinity. It can also help bring 
    in an advanced spiritual soul if chanted by the mother during pregnancy.
  \end{explanation}
\endsong


\beginsong{Hari Om Shiva Om\ldots \\ Cosmic vibration Mantra}[tags={Vishnu 1, Shiva 1},ph={II}]
  \showmantra{Hari Om Shiva Om Shiva Om Hari Om}
  \begin{explanation} 
    \textbf{Hari} is another name of Lord Vishnu. Can also be translated as The Remover of ego. 
    Universal mantra of cosmic vibration.
  \end{explanation}
\endsong


\beginsong{Om Eim Saraswatyei Namaha}[tags={learning 1},ph={II}]
  \showmantra{Om Eim Saraswatyei Namaha}
  \begin{explanation}
    Salutations to Saraswati, the goddess of music, poetry, the arts, education, 
    learning and divine speech. Opens us towards education, learning, and the artistic world of 
    music and poetry. Whenever you find yourself moved to tears by a piece of music, or touched 
    by the words of the great poets and sages, you are in the presence of Saraswati. May we be 
    at ease while learning the wonders of the unfolding mystery of Life.
  \end{explanation}
\endsong


\beginsong{Om Namo Narayana}[tags={transcendence 1},ph={II}]
  \showmantra{Om Namo Narayana}
  \begin{explanation}
    I bow to the divine. Salutes the all-pervading aspect of the Great Spirit anchored 
    in our hearts and in all beings. Destroys barriers, obstacles, afflictions, and difficulties. 
    Leads to self-realisation. Traditionally chanted to assist the dying as they make their 
    transition, the mantra asks prayerfully, that we may all merge into the grace of divine light.
  \end{explanation}
\endsong


% Image to show on the empty recto page (remove if no longer empty!)
\begin{intersong}%
  \subsection*{Chakras}
  \begin{center}
    \vfill%
    \includegraphics[width=0.618\textwidth]{Chakras_map_1000px.png}%
    \vfill
  \end{center}
  \begin{description}
    \item[Sahasrāra:] "thousand-petaled", the crown chakra
    \item[Ājñā:] "command", the third eye chakra
    \item[Viśuddha:] "especially pure", the throat chakra
    \item[Anāhata:] "unstruck", the heart chakra
    \item[Maṇipūra:] "jewel city", the solar plexus chakra
    \item[Svādhiṣṭhāna:] "one's own base", the sacral chakra
    \item[Mūlādhāra:] "root support", the root chakra
  \end{description}
\end{intersong}

