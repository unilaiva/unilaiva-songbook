% Santo Daime: O Cruzeiro (Mestre Irineu)
% =======================================
%
% The following sets the song number for the first song in this file.
% The number will automatically be incremented by one for each song.
% Please do not change this! Changing would make different versions of
% the songbook to have different numbers for the same songs, and it
% would totally mess up the selection booklets causing them to have
% wrong songs in them. (For the same reason, add new songs only to the
% end of each songs_ file.)
\setcounter{songnum}{1}


\beginsong{Lua Branca}[by={Mestre Irineu Serra},tags={valsa},key={C},gk={C, (B)--(C)}]
  \audio[key=C]{https://soundcloud.com/user-6451069-977349974/01-lua-branca?in=user-6451069-977349974/sets/o-cruzeiro-universal-mestre-irineu}
  \newchords{chords_lua_branca_a}\newchords{chords_lua_branca_b}\newchords{chords_lua_branca_c}
  \meter{3}{4}
  \mnbeginverse\memorize[chords_lua_branca_a]
    \[^\mn{G}]Deus |\[\mnc{C}C]te \[^\mn{E}]Sal|\[^\mn{G}]ve oh! Lu\[^\mn{A}]a |\[\mncadj{+2ex}{G}Am7]Bran|\[\mnc{B}G]ca, |\[^\mn{G}]Da \[^\mn{B}]luz \[^\mn{D}]tão |\[\mnc{F}F]pra\[^\mn{A}]te|\[\mncadj{+1ex}{G}Am]a|\[\mnc{C}C]da
    \mnendverse\glueverses\mnbeginchorus\memorize[chords_lua_branca_b]
    |\[\mnc{C}C]Tu \[^\mn{E}]sois \[^\mn{G}]mi|nha \[^\mn{A}]Pro\[^\mn{C}]te|\[\mnc{B}G]to|\[\mnc{A}Am]ra, |\[\mnc{B}G]De \[^\mn{A}]Deus \[^\mn{G}]tu |\[^\mn{D}]sois \[^\mn{F}]es\[^\mn{B}]ti|\[\mnc{D}G7]ma|\[\mnc{C}C]da
    \mnendchorus\glueverses\mnbeginverse\memorize[chords_lua_branca_c]
    |\[\mnc{A}Am]Oh \[^\mn{C}]Mãe \[^\mn{B}]Di|\[^\mn{A}]vina |\[\mnc{G}G]do \[^\mn{C}]co\[^\mn{B}]ra|\[^\mn{G}]ção, |Lá \[^\mn{B}]nas \[^\mn{D}]al|\[\mnc{F}F]turas \[^\mn{G}]on\[^\mn{B}]de es|\[\mnc{A}Am7]tás
    \[^\mn{G}]Mi\[^\mn{E}]nha |\[\mnc{C}C]Mãe, \[^\mn{A}]lá \[^\mn{C}]no |\[\mnc{B}G]Céu, |\[\mnc{A}Am]Dai-\[^\mn{C}]me o \[^\mn{B}]per|\[^\mn{A}]dão |\[\mnc{G}G] \e
  \mnendverse
  \notesoff
  \beginverse\replay[chords_lua_branca_a]
    |^Das flores |do meu pa|^ís, |^ |Tu sois a |^mais deli|^ca|^da
    \endverse\glueverses\beginchorus\replay[chords_lua_branca_b]
    |^De todo |meu cora|^ção, |^ |^Tu sois de |Deus esti|^ma|^da
    \endchorus\glueverses\beginverse\replay[chords_lua_branca_c]
    |^Oh Mãe Di|vina |^do cora|ção, |Lá nas al|^turas onde es|^tás
    Minha |^Mãe, lá no |^Céu, |^Dai-me o per|dão |^ \e
  \endverse
  \beginverse\replay[chords_lua_branca_a]
    |^Tu sois |a flor mais |^be|^la, |Aonde |^Deus pôs a |^mão |^ \e
    \endverse\glueverses\beginchorus\replay[chords_lua_branca_b]
    |^Tu sois mi|nha Advo|^ga|^da, |^Oh! Virgem |da Concei|^ção |^ \e
    \endchorus\glueverses\beginverse\replay[chords_lua_branca_c]
    |^Oh Mãe Di|vina |^do cora|ção, |Lá nas al|^turas onde es|^tás
    Minha |^Mãe, lá no |^Céu, |^Dai-me o per|dão |^ \e
  \endverse
  \beginverse\replay[chords_lua_branca_a]
    |^Estrela |do Uni|^ver|^so, |Que me pa|^rece um jar|^dim |^ \e
    \endverse\glueverses\beginchorus\replay[chords_lua_branca_b]
    |^Assim co|mo sois bri|^lhan|^te, |^Quero que |brilhes à |^mim |^ \e
    \endchorus\glueverses\beginverse\replay[chords_lua_branca_c]
    |^Oh Mãe Di|vina |^do cora|ção, |Lá nas al|^turas onde es|^tás
    Minha |^Mãe, lá no |^Céu, |^Dai-me o per|dão |^ \e
  \endverse
  \begin{lilywrap}\begin{lilypond}[]
    % transcribed by larva, latest update on 2024-03
    \include "tex/lp-include-head.ly"
    % \header {
    %   title = "1. Lua Branca"
    %   composer = "Mestre Raimundo Irineu Serra"
    % }
    theMelody = \relative g' {
      \key c \major \slurDashed
      \set melismaBusyProperties = #'()
      \time 3/4 \partial 4
      \sectionLabel "A"
      \parenthesize g4 \pomark | c4( c4) e4 | g g a | \once\slurSolid a2( g4)( | b,2.)
      g4 b4 d4 | \once\slurDown f4( << f4) \single\altcol g4 >> a4 | \once\slurSolid a2( g4)( | c,2.)
      \sectionLabel "B"
      \repeat volta 2 {
        | c4 e g | g a c | b2.( a2.) | b4 a g | d f b,
        d2.( c2.)
      }
      \sectionLabel "C"
      a4 c b | a2 a4 | g4 c b | g2. | g4 b d | f8 f g4 b
      | a g e | c a c | b2. | a4 c b | \once\slurSolid a2.( | g2)
      \fine
    }
    theLyricsOne = \lyricmode {
      \set stanza = "1."
      Deus | te __ _ Sal -- | ve~oh! Lu -- a | Bran _ -- | ca;
      | Da luz tão | pra -- _ te -- | a -- _ | da.
      \repeat volta 2 {
        | Tu sois mi -- | nha Pro -- te -- | to -- | ra;
        | De Deus tu | sois es -- ti -- | ma -- | da.
      }
      | Oh Mãe Di -- | vi -- na | do co -- ra -- | ção;
      | Lá nas al -- | tu -- ras on -- de~es -- | tás;
      Mi -- nha | Mãe, lá no | Céu;
      | Dai -- me~o per -- | dão. __ | _
    }
    theLyricsTwo = \lyricmode {
      \set stanza = "2."
      \skip 1 | Das flo -- res | do meu pa -- | ís; __ _ | _
      | Tu sois a | mais \altcol de -- li -- | ca -- _ | da.
      \repeat volta 2 {
        | De to -- do | meu co -- ra -- | ção; __ | _
        | Tu sois de | Deus es -- ti -- | ma -- | da.
      }
    }
    theLyricsThree = \lyricmode {
      \set stanza = "3."
      \skip 1 | Tu __ _ sois | a flor mais | be -- _ | la;
      | A -- on -- de | Deus \altcol pôs a | mão. __ _ | _
      \repeat volta 2 {
        | Tu sois mi -- | nha~A -- d -- vo -- | ga -- | da;
        | Oh! Vir -- gem | da Con -- cei -- | ção. __ | _
      }
    }
    theLyricsFour = \lyricmode {
      \set stanza = "4."
      \skip 1 | Es -- tre -- la | do U -- ni -- | ver -- _ | so;
      | Que me pa -- | re -- \altcol ce~um jar -- | dim; __ _ | _
      \repeat volta 2 {
        | As -- sim co -- | mo sois bri -- | lhan -- | te;
        | Que -- ro que | bri -- lhes à | mim. __ | _
      }
    }
    theChords = \chordmode {
      s4 | c2. | c | a:m7 | g
      | g | f | a:m | c
      \repeat volta 2 {
        | c | c | g | a:m
        | g | g | g:7 | c
      }
      a:m | a:m | g | g
      | g | f | a:m7 | c
      | g | a:m | a:m | g2
    }
%    \layout { #(layout-set-staff-size 15) } % for better fit
   \include "tex/lp-include-tail-lyricsbelow.ly"
  \end{lilypond}\end{lilywrap}
  \ifshowlilypond\translationoff\fi % to fit whole song on one spread
  \begin{translation}[EN]
    God salutes You, oh! White Moon of such silvery light
    You are my Protectress, You are esteemed by God
    \nextverse
    Oh! Divine Mother of the heart, there in the heights where You are
    My Mother, there in Heaven, give me forgiveness
    \nextverse
    Oh! Amongst the flowers of my country, You are the most delicate
    With all my heart, You are esteemed by God
    \nextverse
    You are the most beautiful flower, where God put His hand
    You are my Advocate, oh! Virgin of Conception
    \nextverse
    Star of the Universe that looks like a garden
    Being bright as You are, I want You to shine upon me
  \end{translation}
\endsong


\beginsong{Tuperci}[by={Mestre Irineu Serra},tags={3x},key={C},gk={D, C--D}]
  \audio[key=C]{https://soundcloud.com/user-6451069-977349974/02-tuperci-1?in=user-6451069-977349974/sets/o-cruzeiro-universal-mestre-irineu}
  \mnbeginchorus\prep{3}
    \[\mn{A}]Tu\[\mn{E}]per|\[\mnc{G}G7]ci \[\bm] \[\mn{F}]não \[\mn{E}]me \[\mn{B}]co|\[\mnc{D}C]nhe\[\mn{C}]ce \[\bm] \altchords{\id[1]{(D)}|A7 |D}
    \[\mn{B}]Tu \[\mn{A}]não |\[G7]sa\[\mn{G}]bes\[\bm] \[\mn{F}]me a\[\mn{E}]pre\[\mn{D}]ci|\[\mnc{G}C]ar \[\bm] \altchords{|A7 |D}
    \[\mn{A}]Tu \[\mn{E}]não |\[\mnc{G}G7]sabes\[\bm] \[\mn{F}]me \[\mn{E}]com\[\mn{B}]preen|\[\mncii{D}{C}C]der \[\bm] \altchords{|A7 |D}
    A \[\mn{B}]mi\[\mn{A}]nha |\[\mncadj{+2ex}{G}G7]flor\[\bm] \[\mn{F}]cor \[\mn{E}]de \[\mn{D}]Ja|\[\mnc{G}C]ci \[\bm] \altchords{|A7 |D}
  \mnendchorus
  \begin{translation}[EN]
    Tuperci doesn't know me
    You don't know how to appreciate me
    You don't know how to understand me
    My flower, the color of Jaci
  \end{translation}
  \ifshowlilypond\hardbrk\fi
  \begin{lilywrap}\begin{lilypond}[]
    \include "tex/lp-include-head.ly"
    % % transcribed by larva, latest update on 2024-03
    % \header {
    %   title = "2. Tuperci"
    %   composer = "Mestre Raimundo Irineu Serra"
    % }
    theMelody = \relative b'' {
      \key c \major \slurSolid
      \set melismaBusyProperties = #'()
      \time 4/4 \partial 4
      \repeat volta 3 {
        \mark \markup {"3x"}
        a8 \pomark e | g2~ g8 f e b | d4 c2
        b8 a | a4 g4. f'8 e d | g2.
        a8 e | g4 g4. f8 e b | d4( c4.)
        c8 b a | a4( g4.) f'8 e d | g2.
      }
      \fine
    }
    theLyricsOne = \lyricmode {
      Tu -- per -- | ci _ não me co -- | nhe -- ce;
      Tu não | sa -- bes me~a -- pre -- ci -- | ar;
      Tu não | sa -- bes me com -- preen -- | der; _
      A mi -- nha | flor _ cor de Ja -- | ci.
    }
    theChords = \chordmode {
      s4
      | g1:7 | c
      | g:7 | c
      | g:7 | c
      | g:7 | c2.
    }
%    \layout { #(layout-set-staff-size 15) } % for better fit
   \include "tex/lp-include-tail-lyricsbelow.ly"
  \end{lilypond}\end{lilywrap}
  \begin{explanation}[PT]
    \emph{Tuperci} pode ser considerado um dos muitos caboclos ou entidades encantadas presentes na doutrina tradicional da Ayahuasca, que são invocados do plano astral através de cânticos durante rituais geralmente para fornecer orientação. No entanto, por que uma entidade pouco evoluída seria chamada para se apresentar? Afinal, Tuperci ``não sabe, não aprecia, não compreende'' a flor de cor Jaci, que parece simbolizar a revelação da doutrina do Santo Daime. \emph{Jaci} significa ``a lua`` em algumas línguas indígenas. Ou será que Tuperci significa ``tu de per si'' --- que você sozinho, isolado, ``não me conhece''?
  \end{explanation}%
  \begin{explanation}[EN]
    \emph{Tuperci} can be considered to be one of the many caboclos or enchanted entities present in the traditional Ayahuasca doctrine, who are invoked from the astral plane by chanting during rituals usually to provide guidance. However, why would an unevolved entity be called upon to present itself? After all, Tuperci ``does not know, does not appreciate, does not understand'' the flower of color Jaci, which seems to symbolise the revelation of the Santo Daime doctrine. \emph{Jaci} means ``the moon'' in some indigenous languages. Or does Tuperci mean ``you by yourself'' (\emph{tu de per si}) --- that you alone, isolated, ``do not know me''?
  \end{explanation}
\endsong


\beginsong{Ripi}[by={Mestre Irineu Serra},tags={3x},key={Dm},gk={Dm, Cm--Em}]
  \audio[key=Dm]{https://soundcloud.com/user-6451069-977349974/03-ripi-1?in=user-6451069-977349974/sets/o-cruzeiro-universal-mestre-irineu}
  \mnbeginchorus
    \prep{3}
    \[\mn{D}]Ri|\[\mnc{E}A]pi, \[\mn{C#}]Ri\[\bmc\mn{D}]pi, \[\mn{E}]Ri|\[\mnc{F}Dm]pi \[\bm] \altchords{\id[4]{alt. (Dm)}|A |Dm}
    \[\mn{A}]Ri|\[\mnc{G}Gm]pi, \[\mn{E}]Ri\[\bmc\mn{F}]pi, \[\mn{G}]Ia|\[\mnc{A}A]iá \[\bm] \altchords{|A7 |Dm}
    Se |\[\mnc{G}Gm]vo\[\mn{F}]cê \[\bmc\mn{E}]não \[\mn{D}]que|\[\mnc{F}Dm]ri\[\mn{A}]a \[\bm] \altchords{|A7 |Dm}
    Para |\[A]que \[\mn{C#}]veio \[\mnc{E}A7]me en\[\mn{C#}]ga|\[\mnc{D}Dm]nar? \[\bm] \altchords{|A7 |Dm}
  \mnendchorus
  \ifchorded\ifshowaltchords
    \beginverse
      \altchords{\id[5]{alt. (Dm)}|A Dm |F}
      \altchords{|Gm Gm7 |A}
      \altchords{|Gm Gm6 |F}
      \altchords{|A A7 |Dm}
    \endverse
  \fi\fi
  \begin{translation}[EN]
   Ripi, Ripi, Ripi, Ripi, Ripi, Iaiá
   If you did not want to, why did you come to deceive me?
  \end{translation}
  \ifshowlilypond\hardbrk\fi
  \begin{lilywrap}\begin{lilypond}[]
    \include "tex/lp-include-head.ly"
    % % transcribed by larva, latest update on 2024-03
    % \header {
    %   title = "3. Ripi"
    %   composer = "Mestre Raimundo Irineu Serra"
    % }
    theMelody = \relative d'' {
      \key d \minor \slurSolid
      \set melismaBusyProperties = #'()
      \time 4/4 \partial 4
      \repeat volta 3 {
        \mark \markup {"3x"}
        d4 \pomark | e cis d e | f2.
        a4 | g e f g | a2.
        a4 | g f e d | f a,2
        a8 a8 | a4 cis e cis | d2.
      }
      \fine
    }
    theLyricsOne = \lyricmode {
      \repeat volta 3 {
        Ri -- | pi, Ri -- pi, Ri -- | pi;
        Ri -- | pi, Ri -- pi, Ia -- | iá;
        Se | vo -- cê não que -- | ri -- a;
        Pa -- ra | que veio me~en -- ga -- | nar?
      }
    }
    theChords = \chordmode {
      \repeat volta 3 {
        s4
        | a1 | d:m
        | g:m | a
        | g:m | d:m
        | a2 a2:7 | d2.:m
      }
    }
    \layout { #(layout-set-staff-size 15) } % for better fit
    \include "tex/lp-include-tail-lyricsbelow.ly"
  \end{lilypond}\end{lilywrap}
  \begin{explanation}[PT]
    Este hino também pode ser considerado um chamado para entidades astrais.
    Dona Percília Matos da Silva avalia que este hino é um aviso para casos
    em que alguém participa, mas não procura aprender e até acaba atrapalhando
    os outros. ``Ripi, Ripi, Ripi Ripi, Ripi, Iaiá'' pode ser apenas um
    cantarolar preliminar introdutório à mensagem de alerta.
  \end{explanation}%
  \begin{explanation}[EN]
    This hymn can also be considered a call to astral entities. Dona Percília
    Matos da Silva evaluates that this hymn is a notice for cases in which
    someone participates, but doesn't seek to learn, and even ends up hindering
    others. ``Ripi, Ripi, Ripi Ripi, Ripi, Iaiá'' might just be a preliminary
    humming introductory to the alerting message.
  \end{explanation}
\endsong


\beginsong{Formosa}[by={Mestre Irineu Serra},key={A lyd.},gk={A lyd., (A lyd.) -- (B\flt{} lyd.)}]
  \audio[key=A]{https://soundcloud.com/user-6451069-977349974/04-formosa-1?in=user-6451069-977349974/sets/o-cruzeiro-universal-mestre-irineu}
  \mnbeginchorus
    \[\mn{G#}]For|\[\mnc{E}E]mo\[\bmc\mn{F#}]sa, \[\mn{G#}]for|\[\mn{E}]mo\[\bmc\mn{F#}]sa \altchords{\id[1]{(G lyd.)}|D | \e}
    \[\mn{G#}]For|\[\mn{E}]mosa, é \[\bmc\mn{F#}]bem \[\mn{G#}]for|\[\mnc{A}A]mo\[\bm]sa \altchords{| - |G}
    \[\mn{B}]For|\[\mnc{G#}E]mo\[\mn{E}]sa, é \[\bmc\mn{F#}]bem \[\mn{G#}]for|\[\mnc{F#}F#m]mosa \altchords{|D |Em}
    \[\bmc\mn{E}]Ta\[\mn{C#}]ru|\[\mn{A}]mim, \[\mn{F#}]tu \[\bmc\mn{G#}]sois \[\mn{A}]for|\[\mnc{B}B7]mo\[\bm]sa \altchords{| - |A7}
    For|mosa, é \[\bmc\mn{C#}]bem \[\mn{D#}]for|\[\mnc{E}E]mo\[\bm]sa \altchords{| - |D}
  \mnendchorus
  \mnbeginverse
    \[\mn{G#}]For|\[\mnc{E}E]mo\[\bmc\mn{F#}]sa, \[\mn{G#}]for|\[\mn{E}]mo\[\bmc\mn{F#}]sa \altchords{|D | \e}
    \[\mn{G#}]For|\[\mn{E}]mosa, é \[\bmc\mn{F#}]bem \[\mn{G#}]for|\[\mnc{A}A]mo\[\bm]sa \altchords{| - |G}
  \mnendverse
  \mnbeginchorus
    \[\mn{A}]Ta\[\mn{B}]ru|\[\mnc{G#}E]mim, \[\mn{E}]eu es\[\bmc\mn{F#}]tou \[\mn{G#}]com |\[\mnc{F#}F#m]sede \altchords{|D |Em}
    \[\bmc\mn{E}]Ta\[\mn{C#}]ru|\[\mn{A}]mim, \[\mn{F#}]tu \[\bmc\mn{G#}]me \[\mn{A}]dá |\[\mnc{B}B7]á\[\bm]gua \altchords{| - |A7}
    Taru|mim, tu \[\bmc\mn{C#}]sois \[\mn{D#}]Mãe |\[\mnc{E}E]D'á\[\bm]gua \altchords{| - |D}
    Ta\[\mn{G#}]ru|\[\mn{E}]mim, tu \[\bmc\mn{F#}]sois \[\mn{G#}]for|\[\mn{E}]mo\[\bmc\mn{F#}]sa \altchords{| - | \e}
    \[\mn{G#}]For|\[\mn{E}]mo\[\bmc\mn{F#}]sa, \[\mn{G#}]for|\[\mn{E}]mosa, é \[\bmc\mn{F#}]bem \[\mn{G#}]for|\[\mnc{A}A]mo\[\bm]sa \altchords{| - | - |G}
  \mnendchorus
  \begin{translation}[EN]
    Beautiful, beautiful; beautiful, is very beautiful
    Beautiful, is very beautiful; Tarumin, you are beautiful
    Beautiful, is very beautiful
    \nextverse
    Beautiful, beautiful; beautiful, is very beautiful
    \nextverse
    Tarumin, I am thirsty; Tarumin, you give me water
    Tarumin, you are the Mother of the Waters; Tarumin, you are beautiful
    Beautiful, beautiful, is very beautiful
  \end{translation}
  \begin{lilywrap}\begin{lilypond}[]
    \include "tex/lp-include-head.ly"
    % % transcribed by larva, latest update on 2024-03
    % \header {
    %   title = "4. Formosa"
    %   composer = "Mestre Raimundo Irineu Serra"
    % }
    theMelody =  \relative c''' {
      \key a \major \slurSolid
      \set melismaBusyProperties = #'()
      \time 4/4 \partial 4
      \sectionLabel "A"
      \repeat volta 2 {
        gis4 | e2 fis4 gis | e2 fis4
        gis | e e8 e fis4 gis | a2 a4
        b4 | gis e8 e fis4 gis | fis4. fis8
        e4. cis8 | a4 fis gis a | b2 b4
        b4 | b b8 b8 cis4 dis | e2 e4 %was: d4
      }
      \break
      \sectionLabel "B"
      gis4 | e2 fis4 gis | e2 fis4
      gis | e e8 e fis4 gis | a2 a4
      \break
      \sectionLabel "C"
      \repeat volta 2 {
        a8 \pomark b | gis4 e8 e fis4
        gis | fis4. fis8 e4. cis8 | a4 fis gis a | b2 b4
        b8 b | b4 b cis dis | e2 e4
        e8 gis | e4 e fis gis | e2 fis4
        gis | e2 fis4 gis | e e8 e fis4 gis | a2 a4
      }
      \fine
    }
    theLyricsOne = \lyricmode {
      \repeat volta 2 {
        For -- | mo -- sa, for -- | mo -- sa;
        For -- | mo -- sa, é bem for -- | mo -- sa;
        For -- | mo -- sa, é bem for -- | mo -- sa;
        Ta -- ru -- | mim, tu sois for -- | mo -- sa;
        For -- | mo -- sa, é bem for -- | mo -- sa.
      }
      For -- | mo -- sa, for -- | mo -- sa;
      For -- | mo -- sa, é bem for -- | mo -- sa.
      \repeat volta 2 {
        Ta -- ru -- | mim, eu es -- tou com | se -- de;
        Ta -- ru -- | mim, tu me dá | á -- gua
        Ta -- ru -- | mim, tu sois Mãe | D'á -- gua;
        Ta -- ru -- | mim, tu sois for -- | mo -- sa;
        For -- | mo -- sa, for -- | mo -- sa, é bem for -- | mo -- sa.
      }
    }
    theChords = \chordmode {
      \repeat volta 2 {
        s4
        | e1 | e
        | e | a
        | e | fis:m
        | fis:m | b:7
        | b:7 | e2.
      }
      s4
      | e1 | e
      | e | a2.
      \repeat volta 2 {
        s4
        | e1 | fis:m
        | fis:m | b:7
        | b:7 | e
        | e | e
        | e | e | a2.
      }
    }
    \layout { #(layout-set-staff-size 15) } % for better fit
    \include "tex/lp-include-tail-lyricsbelow.ly"
  \end{lilypond}\end{lilywrap}
\endsong


\sclearpage\chapcornermarkdashedtrue
\beginsong{Refeição}[by={Mestre Irineu Serra},key={A},gk={A, (A) -- (B\flt{})}]
  \audio[key=G]{https://www.youtube.com/watch?v=KTK-h0rBWTg}
  \textnotefornext{Este hino não é cantado durante os trabalhos. Em vez disso, é cantado antes (ou depois, com letras alteradas) de uma refeição.}*
  \textnotefornext{This hino is not sung in the works. Instead, it is sung before (or after with altered lyrics) a meal.}*
  \vspace{2ex}
  \mnbeginverse\memorize
    \[^\mn{E}]Pa\[^\mn{F#}]pai \[^\mn{G#}]do |\[\mnc{A}A]céu \[\mnc{G#}E] \[^\mn{B}]do \[^\mn{G#}]co\[^\mn{E}]ra|\[\mnc{D}D]ção
    Que \[\mnc{C#}A/C#]hoje \[^\mn{E}]neste |\[\mnc{A}A]dia
    É \[^\mn{G#}]quem \[\mnc{F#}F#m]{dá o} \[^\mn{C#}]nos\[^\mn{C}]so |\[\mnc{B}Bm]pão \altlyr{Foi quem deu o nosso |pão}
    \[\mnc{E}A/E]Graças \[^\mn{A}]{a Ma}|\[\mnc{E}E]mãe \[\bm]
  \mnendverse
  \notesoff
  \beginverse
    Mamãe do |^céu ^ do cora|^ção
    Que ^hoje neste |^dia
    É quem ^{dá o} nosso |^pão \altlyr{Foi quem deu o nosso |pão}
    Lou^vado seja |^Deus ^
  \endverse
  \begin{translation}[EN]
    Papa of heaven of the heart
    Who, today on this day
    Is who gives us our bread \altlyr{{\scriptsize was who gave us}}
    Thanks to Mama
    \nextverse
    Mama of heaven of the heart
    Who, today on this day \altlyr{{\scriptsize was who gave us}}
    Is who gives us our bread
    Praised be God
  \end{translation}
  \begin{lilywrap}\begin{lilypond}[]
    \include "tex/lp-include-head.ly"
    % % latest update on 2024-03
    % \header {
    %   title = "5. Refeição"
    %   composer = "Mestre Raimundo Irineu Serra"
    % }
    theMelody =  \relative c'' {
      \key a \major \slurDashed
      \set melismaBusyProperties = #'()
      \time 4/4 \partial 4.
      e8 fis gis | \once\slurSolid a2( gis8)
      b gis e | d4.
      d8 cis cis e e | a, a
      a gis fis fis cis' bis | b4.( \parenthesize b8)
      e8 e a a | e2~8
      \fine
    }
    theLyricsOne = \lyricmode {
      \set stanza = "1."
      Pa -- pai do | céu _ do co -- ra -- | ção;
      Que ho -- je nes -- te | di -- a;
      << { É quem dá } \new Lyrics { \set associatedVoice = "theMelody" \override LyricText.color = #grey (Foi quem deu) } >>
        o nos -- so | pão; __ _
      Gra -- ças a Ma -- | mãe. _
    }
    theLyricsTwo = \lyricmode {
      \set stanza = "2."
      Ma -- mãe do | céu _ do co -- ra -- | ção;
      Que ho -- je nes -- te | di -- a;
      É quem dá o nos -- so | pão;
      Lou -- va -- do se -- ja | Deus. _
    }
    theChords = \chordmode {
      s4.
      | a2 e | d a:/cis
      | a fis:m
      | b:m a:/e
      | e2~8
    }
%    \layout { #(layout-set-staff-size 15) } % for better fit
    \include "tex/lp-include-tail-lyricsbelow.ly"
  \end{lilypond}\end{lilywrap}
\endsong
\sclearpage\chapcornermarkdashedfalse

