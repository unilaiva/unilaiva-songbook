
\begin{intersong}
  \section*{Mantras} % Note: must use non-numbered section* here within 'songs'!
  
  When you chant even without knowing the meaning, that itself carries power. But when you know 
  the meaning and chant with that feeling in your heart then the energy would flow million times 
  more powerful. Therefore it is essential to know the meaning of the Mantra when you use it. 
  Mantra is like calling a name. The meaning isn't as important as the vibrations of the words 
  have on the body and energy centers.

  Each mantra is recommended to be chanted for an entire mala – 108 times. According to the Vedic 
  scriptures, our bodies –  physical and subtle – contain 72,000 energy channels, called Nadis 
  ’There are 108 major nadis that meet in the ‘sacred heart’ (hrit padma). By chanting a mantra 
  108 times the energy permeates the entire body and energy body.  There are six senses (sight, 
  sound, smell, taste, touch, and consciousness) multiplied by three reactions (positive, 
  negative, or indifference) making 18 "feelings." Each of these feelings can be either attached 
  to pleasure or detached from pleasure making 36 "passions", each of which may be manifested in 
  the past, present, or future. All the combinations of all these things makes a total of 108, 
  which are represented by the beads.

  Choose a particular mantra and make it part of your daily life for at least for 21 days – or 
  the auspicious number of 40 days. Chant at dawn and/or dusk, or when you can. 
  \textbf{Be focused, and enjoy each repetition, being as present as you can.}

  Buddhism takes the view that the nature of everything in its most restful state is the blissful 
  union of wisdom and compassion. This is symbolized in Tibetan tantric practice in the union of 
  male and female energies. Within this state resides our pure consciousness. Mantra is its sound. 
  Relaxing with conscious mantra recitation enables pure consciousness to materialise. 

  As the word ‘mantra’ suggests, it becomes a tool that holds the mind together. The power 
  generated in this concentrated affirmation is believed to cut through the vision of impure 
  self-perception, regarded as the root of all suffering.  
  
  \paragraph{Common terms}
  \begin{description}
   \item[OM] usually chanted at the beginning of every mantra. It is known as a 'seedsound' that 
     is extremely potent and expresses a particular energy. A translation will always fall short 
     and is actually impossible. Om is the sound of the sixth chakra (third eye). Here is where 
     the masculine and feminine energies meet. It is called the Soundless Sound, or the Sound of 
     the Universe.
   \item[Namaha] a common ending to many mantras, “I offer”, giving thanks.
  \end{description}

  \paragraph{Sources}

  Collected from various sources over time. This is work in progress, and made for private 
  purposes to support mantra recitation.
\end{intersong}


\beginsong{Om Purnam \\ Purnamadah}[by={traditional, Shantala \& Satyaa \& Sari},sr={Upanishad, the first prayer}]
  \beginverse
    |\[Am]\[Am/B]|\[Am]\[Am/B]|\[Am]\[Am/B]|\[Am]\[Am/B]|
  \endverse
  \beginverse
    |\[Am]Purnama\[Am/B]dah |\[Am]Purnami\[Am/B]dam|
    |\[Am]Purnat \[Am/B]Purnamu|\[Am]dachya\[Am/B]te|
    |\[Am]Purnasya \[Am/B]Purna|\[Fmaj7]mada\[G]ya|
    |\[Am] Purnam|evava\[Fmaj7]shishya|\[G6]te|
  \endverse
  \beginverse
    |\[Am]Om|\[Fmaj7] \[G6]|
    |\[Am]|\[Fmaj7] \[G6]|\[Gadd9]|\[G(5)]|
  \endverse
  \begin{feeler}
    That is the whole. \\
    This is the whole. \\
    From wholeness emerges wholeness. \\
    Wholeness coming from wholeness, \\
    Wholeness still remains. \\
  \end{feeler}
  \begin{explanation}
    ``\ldots contains the secret of the mystic approach towards life. This small sutra contains the 
    essence of the Upanishadic vision. Neither before nor afterwards has the vision been 
    transcended; it still remains the Everest of human consciousness. The Upanishadic vision is 
    that the universe is a totality, indivisible; it is an organic whole. The parts are not 
    separate, we are all existing in a togetherness: the trees, the mountains, the people, the 
    birds, the stars, howsoever far away they may appear - don't be deceived by the appearance - 
    they are all interlinked, all bridged. Even the smallest blade of grass is connected to the 
    farthest star, and it is as significant as the greatest sun. Nothing is insignificant, nothing 
    is smaller than anything else. The part represents the whole, just as the seed contains the 
    whole\ldots `` –Osho: Philosophia Ultima
  \end{explanation}
\endsong


\beginsong{Mahamrityunjaya Mantra \\ Om Tryambakam}[ah={339}]
  \capo{2}
  \beginverse
    |\[C]Om Tryamba|\[G]kam Yajama|\[Dm]he |\[Fmaj7] |
    |\[C] Sugandhim |\[G]Pushti Vardha|\[Dm]nam | -
    Ur|\[F]va Rukamiva |\[C]Bandhanan Mri|\[G]tyor | -
    Muk|\[Dm]shiya Maamritat \echo{Muk|shiya Maamritat}
    Muk|\[G]shiya Maamritat \echo{Muk|shiya Maamritat} |
  \endverse
  \begin{feeler}
    We Meditate on the Three-eyed reality \\
    Which permeates and nourishes all like a fragrance.  \\
    May we be liberated from death for the sake of immortality, \\ 
    Even as the cucumber is severed from bondage to the creeper.
  \end{feeler}
  \begin{explanation}
    Mahamrityunjaya Mantra (maha-mrityun-jaya) is one of the more potent of the ancient Sanskrit 
    mantras. Maha mrityunjaya is a call for enlightenment and is a practice of purifying the karmas 
    of the soul at a deep level. It is also said to be quite beneficial for mental, emotional, and 
    physical health.
  \end{explanation}
\endsong


\scleardpage %% starts following content on a new blank even-numbered pages (on the left side)
\beginsong{Moola Mantra}[by={Seven, Deva Premal}]
  \capo{3}
  \beginchorus
    Om |\[Am]Satchitananda |Parabrahma |
    |Purushothama Para|matma
    Sri |\[F]Bhagavati Same|\[G]tha
    Sri |\[Am]Bhagavate Nama|ha
  \endchorus
  \beginchorus     
    Hari |\[Am]Om Tat Sat, Hari |\[G]Om Tat Sat
    Hari |\[C]Om Tat \[G]Sat, Hari |\[Fmaj7]Om Tat \[Am]Sat
  \endchorus 

  \begin{feeler}
    Oh Divine Force, Spirit of All Creation, \\
    Highest Personality, Divine Presence, \\
    manifest in every living being. \\

    Supreme Soul manifested \\
    as the Divine Mother and \\
    as the Divine Father. \\

    I bow in deepest reverence.\\
  \end{feeler}  
  \begin{explanation}  
    \textbf{Moola mantra} evokes the living God, asking protection and freedom from all sorrow 
    and suffering. It is a prayer that adores the great creator and liberator, who out of love and 
    compassion manifests, to protect us, in an earthly form.  The calmness that the mantra can 
    give is to be experienced, not spoken about. Here is the key with which any door to spiritual 
    treasure could be opened. A tool which can be used to achieve all desires. A medicine which 
    cures all ills. Just like when you call a person he comes and makes you feel his presence, the 
    same manner when you chant this mantra, the supreme energy manifests everywhere around you. As 
    the Universe is Omnipresent, the supreme energy can manifest anywhere and any time. It is also 
    very important to know that the invocation with all humility, respect and with great necessity 
    makes the presence stronger.
    
    \begin{description}
      \item[Om] Calling on the highest energy of all there is. It is said 'In the beginning was the
        Supreme word and the word created every thing. That word is Om'. If you are meditating in 
        silence deeply, you can hear the sound Om within. The whole creation emerged from the sound 
        Om. It is the primordial sound or the Universal sound by which the whole universe vibrates. 
        This divine sound has the power to create, sustain and destroy, giving life and movement to 
        all that exist.
      \item[Sat] Truth. The formless. The all penetrating existence that is formless, shapeless, 
        omnipresent, attributeless, and qualityless aspect of the Universe, experienced as emptiness  
        of the Universe. The body of the Universe that is static. Everything that has a form and can 
        be sensed evolved out of this. So subtle that it is beyond all perceptions. It can only be 
        seen when it has become gross and has taken form. We are in the Universe and the Universe is 
        in us. We are the effect and Universe is the cause and the cause manifests itself as the 
        effect.      
      \item[Chit] The Pure Consciousness of the Universe that is infinite, omni-present 
        manifesting power of the Universe. Out of this is evolved everything that we call Dynamic 
        energy or force. It can manifest in any form or shape. It is the consciousness manifesting 
        as motion, as gravitation, as magnetism, etc. Also manifesting as the actions of the body, 
        as thought force. The Supreme Spirit.     
      \item[Ananda] Pure bliss, love, joy and friendship nature of the Universe. When you experience
        either the Supreme Energy in this Creation (Sat) and become one with the Existence or
        experience the aspect of Pure Consciousness (Chit), you enter into a state of Divine Bliss 
        and eternal happiness (Ananda). This is the primordial characteristic of the Universe, which 
        is the greatest and most profound state of ecstasy that you can ever experience when you 
        relate with your higher Consciousness.
      \item[Parabrahma] The Supreme creator being in his Absolute aspect; beyond space and time. 
        The essence of the Universe that is with and without form.      
      \item[Purushothama] The energy that incarnates as an Avatar in human form to help and guide 
        mankind and relate closely to the beloved creation.  This has different meanings. Purusha 
        means soul and Uthama means the supreme, the Supreme spirit. It also means the supreme 
        energy of force guiding us from the highest world. Purusha also means Man, and Purushothama 
        is the energy that incarnates as an Avatar to help and guide Mankind and relate closely to 
        the beloved Creation.      
      \item[Paramatma] Supreme inner energy that is immanent in every creature and in all beings, 
        living and non-living. Who comes to me in my heart, and becomes my inner voice whenever I 
        ask. It's the indweller or the Antaryamin who resides formless or in any form desired. It's 
        the force that can come to you whenever you want and wherever you want to guide and help 
        you.      
      \item[Sri Bhagavathi] The divine mother, the power aspect of creation. The female aspect, 
        which is characterized as the Supreme Intelligence in action, the Power (The Shakti). It is 
        referred to the Mother Earth (Divine Mother) aspect of the creation.
      \item[Sametha] Together or in communion with.     
      \item[Sri Bhagavathe] The Male aspect of the Creation, which is unchangeable and permanent.     
      \item[Namaha] Salutations or prostrations to the Universe that is Om and also has the 
        qualities of Sat Chit Ananda, that is omnipresent, unchangeable and changeable at the same 
        time, the supreme spirit in a human form and formless, the indweller that can guide and help 
        in the feminine and masculine forms with the supreme intelligence. I thank you and  
        acknowledge this presence in my life. I seek your presence and guidance all the time.     
      \item[Hari om tat sat] God is the truth. Hari is another name of Lord Vishnu.     
    \end{description}    
  \end{explanation}
\endsong 


\scleardpage
\beginsong{Gayatri Mantra \\ ``Make all the beings on Earth reach enlightenment'' \\ 
           Om Bhur Bhuvah Svaha}[ah={282}]
  \capo{5}
  \beginverse
    \[Am] Om |\[G]Bhur Bhuvah Sva|\[Am]ha
    Tat |\[G]Savi|\[Am]tur Va|\[G]ren|\[F]yam
    Bhar|\[G]gho Devasya |\[C]Dheemahi
    Dhi|\[G]yo Yo |\[C]Nah Pra|\[G]choda|\[Am]yat
  \endverse
  \begin{explanation}
    A prayer of praise that awakens the vital energies and gives liberation and deliverance from 
    ignorance. This mantra is known to impart wisdom, understanding, and enlightenment. This is 
    said to be the oldest and most powerful of mantras, being thousands of years old. It purifies 
    the person chanting it as well as the listener as it creates a tangible sense of well being in 
    whoever comes across it. 
 
    We meditate on that most adorable, desirable and enchanting luster and brilliance of our 
    Supreme Being, our Source Energy, our Collective Consciousness who is our creator, inspirer 
    and source of eternal Joy.  May this Light inspire and guide our mind and open our hearts. 
    That Divine Illumination which pervades the physical plane astral plane and the celestial 
    plane. That which is the most adorable. On that Divine Radiance we Meditate. May that 
    Enlighten Our Intellect and Awaken our Spiritual Wisdom.
    \brk  
    \paragraph{Om Bhur Bhuvah Svaha} Preamble to the main mantra; means that we invoke in our prayer 
      and meditation the One who is our inspirer, our creator and the abode of supreme Joy.  It also 
      means, we invoke the earthly, physical world, the world of our mind, and the world of our 
      soul.
    \begin{description}
      \item[Om] God/Brahma, Divine Illumination which pervades 
      \item[Bhur] Pranic energy
      \item[Bhuvah] Destroyer of sufferings
      \item[Svaha] Happiness bestowing
    \end{description}
    \paragraph{Tat Savitur Varenyam} Divine Illumination, which is the Most Adorable
    \begin{description}  
      \item[Tat] THAT, denoting the Supreme Being, God or Spirit
      \item[Savitur] The radiating source of life with the brightness of the Sun. Bright Sun/God 
        (also a deva some call upon using this mantra)
      \item[Varenyam] Most adorable, most desirable, greatest
    \end{description}    
    \paragraph{Bhargho Devasya Dheemahi}
    \begin{description}
    \item[Bhargho] Luster and splendor; destroyer of misdeeds
      \item[Devasya] Divine or Supreme God
      \item[Dheemahi] ``We meditate upon''; knowledge imparted/understood
    \end{description}
    \paragraph{Dhiyo Yo Nah Prachodayat}
    \begin{description}  
      \item[Dhiyo] Our understanding of reality, our intellect, our intention; Intelligence
      \item[Yo] He who
      \item[Nah] Our 
      \item[Prachodayat] May he inspire, guide; enlightenment  
    \end{description}    
  \end{explanation}
\endsong


\beginsong{Om Asato Ma}[ah={321}]
  \beginverse
    |\[Am]O|\[Em]om asato ma |\[Am]sat gamaya|
    |\[Em] Tamaso ma |\[C]jyotir gama|\[G]ya
    Mrit|\[Em]yor ma amri|\[Am]tam gamaya| |
  \endverse
  \textnote{suomeksi:}
  \beginverse
    |^O|^om Johda minut |^totuuteen|
    |^ Pimeydestä |^va|^loon
    Tiedot|^tomuudesta tietoi|^suuteen| |
  \endverse  
  \begin{translation}
    Lead me from illusion to reality,
    from darkness to light,
    from death to immortality.
  \end{translation}
\endsong


\beginsong{Shakti Kundalini \\ Om Mata Om Kali}[ah={329}]
  \meter{4}{4}
  \beginchorus
    Om |\[Dm]Mata Om |Kali|
    |\[C]Durga devi na|\[Dm]mo namaha|
  \endchorus
  \beginchorus
    |\[Dm]Shakti kunda|\[C]lini |\[B&]jagadambe ma|\[A]ta|
    |\[Dm]Shakti kunda|\[C]lini |\[B&]jagadam\[A]be ma|\[Dm]ta|
  \endchorus  
  \begin{feeler}
    I bow unto the Divine Mother and Her many feminine aspects: Kali, remover of delusion and
    ignorance; Divine Goddess Durga; Shakti, universal life force and consort to Shiva; and
    Kundalini, the Goddess energy that rises within us. Praise to the Mother of the World!
  \end{feeler}  
  \footnotesize
  \paragraph{} Extended ending, append: \\
    Hey ma Durga hey ma Durgaya (2x) \\
    Jagadambe jagadambe jai jai ma 
\endsong


\beginsong{Jay Shri Ma \\ Ananda Ma \\ Kali Ma}[ah={332}]
  \beginchorus
    |\[Dm]Jay shri \[F]ma |Kali Kali \[Gm]ma|
    |\[Gm]Jay shri \[Dm]ma| |
  \endchorus
  \beginchorus
    |\[F]Ananda ma |\[C]Durga devi|
    |\[Gm]Jagadambe shri |\[Dm]ma| 
  \endchorus  
  \begin{feeler}
    Victory to the Holy Mother Kali, the blissful Mother Divine Durga, the Holy Mother of 
    the Universe.
  \end{feeler}  
\endsong


\beginsong{Jay Ambe}[ah={328}]
  \beginchorus
    |\[Dm]Jay Ambe |\[C]Jagadam\[Am]be|
    |\[F]Mata Bha\[C]vani ki |\[Dm]Jay Ambe|
  \endchorus
  \beginchorus
    |\[Dm]Durgati Nashini |\[F]Durga Jaya Jaya|
    |\[C]Kala Vinashini |\[Dm]Kali Jaya Jaya|
  \endchorus  
  \beginchorus
    |\[C]Uma Rama Brah|\[F]mani Jaya Jaya|
    |\[C]Radha \[Am]Rukamani |\[Dm]Sita Jaya Jaya|
  \endchorus  
\endsong


\beginsong{Om Namah Shivaya}[]
  \beginchorus\memorize % memorize chords even though in 'chorus'
    \textnote{intro:}
    |\[Am]Om Namah Shi|\[F]vaya; |\[G]Om Namah Shi|\[Am]vaya |
  \endchorus
  \beginchorus
    |^ Shivaya |^Namaha; |^ Shivaya |^Namaho |
  \endchorus
  \beginchorus
    |^Sham Bol ^Shankara |^Namah ^Shivaya; \replay |^Girija ^Shankara |^Namah ^Shivaya |
  \endchorus
  \beginchorus
    |^Aruna^chala Shiva |^Namah Shi^vaya; \replay |^Aruna^chala Shiva |^Namah ^Shivaya
  \endchorus
  \beginchorus
    Hari |\[C]Om Namah Shi|\[G]vaya; |\[F]Om Namah Shi|\[Am]vaya |
  \endchorus
  \beginchorus
    \textnote{outro:}
    |^Om Namah Shi|^vaya; |^Om Namah Shi|^vaya |
  \endchorus
\endsong


\beginsong{Haidakandhi}[ah={283}]
  \meter{4}{4}
  \beginverse
    |\[Am] Om namah shi|\[F]vaya namah |\[C]om |\[G]haidakandhi|
    |\[Am] Hari hari |\[F] hari hari |\[C] hari hari |\[G]shankara|
  \endverse
\endsong


\beginsong{Govinda Hari Om}[ah={334}]
  \meter{4}{4}
  \beginverse
    |\[Am]Govin|\[Dm]da | Hari |\[Am]Om Hari |\[E]Hari|
    |\[Am]Gopa|\[Dm]la | Hari |\[Am]Om|\[G] |
    |\[C]Sada |\[Dm]Sadhana | Ananda |\[Am]Bhavana|
    \lrep|\[Dm]Vishnu |\[Am]Sadhana |\[E]Hari |\[Am]Om|\rrep\rep{2}
  \endverse
\endsong


\beginsong{Ganesha Mantra \\ Om Gam Ganapatayei Namaha \\ Removing of obstacles Mantra }[by={Prembababnda}]
  \textnote{part A}
  \beginchorus\memorize % memorize chords even though in 'chorus'
    |\[Em]Om Parvati Patayei |\[Bm]Hara Hara Hara Mahadeva |
    |\[C] Gajana\[D]nam Bu|\[Em]ta |
  \endchorus
  \beginchorus
    |^Ganadi Sevatam |^ Kapitha Jambu |
    |^ Phalacha^ru |^Bhakshanam |
  \endchorus
  \beginchorus
    |^Umasutam Shoka |^ Vinasha Karakam |
    |^ Namami ^Vigneshvara |^Pada Pankajam |
  \endchorus
  \textnote{part B}
  \beginchorus
    |\[Em]Om Gam Ganapata|yei Nama\[Bm]ha |
    |\[Em]Om Gam Ganapata|yei Nama\[Bm]ha |
    |\[G]Om Gam Ganapata|yei Nama\[Bm]ha |
    |\[Em]Om Gam Ganapata|yei Nama\[Bm]ha |
  \endchorus
  \begin{feeler}
    Salutations to the remover of obstacles.
  \end{feeler}
  \begin{explanation}
    This sound formula assists us in the removal of obstacles. In order for that to happen there 
    is no need to know the exact nature of the hindrances. Just the awareness and recognition that 
    there are obstacles and then chanting this mantra with the intention for resolve is enough. 
    This mantra unifies us within. When there is oneness there are no obstacles. This mantra is 
    also used for the beginning of any endeavor. Whenever we start anything anew we can bless the 
    project with the energy of Ganesh through this mantra.
    \begin{description}
      \item[Gam] is the seed sound for Ganesh.
      \item[Ganapati] is another name for Ganesh - the Remover of Obstacles and the of 
        Oneness/Unity.
      \item[Yei] is a sound that activates shakti/energy.
    \end{description}
  \end{explanation}  
  
\endsong


\beginsong{Om Shanti Om \\ Peace Mantra}[]
  \beginchorus
    Om Shanti Om
  \endchorus
  \beginverse
    Om Shanti Shanti Shantihi
  \endverse
  \begin{feeler}
    Peace in my heart, peace with each other, peace in the cosmos.
  \end{feeler}
\endsong


\beginsong{Om Shree Dhanvantre Namaha \\ Healing Mantra}[]
  \beginverse
    Om Shree Dhanvantre Namaha
  \endverse
  \begin{feeler}
    Salutations to the being and power of the Celestial Healer.
  \end{feeler}
  \begin{explanation}
    \textbf{Dhanvantari} is the celestial healer. This mantra helps us find the right path to 
    healing, or directs us to the right health practitioner. In India it is also commonly chanted 
    during cooking in order for the food to be charged with healing vibrations – either to prevent 
    disease or assist in healing for those who are sick. This mantra can be chanted for any 
    situation that one would like to be healed or remedied. Good to remember and be open to the 
    path of healing not necessarily looking the way we expect it!
  \end{explanation}
\endsong


\beginsong{Om Namo Bhagavate Vasudevaya \\ Liberation Mantra}[]
  \beginverse
    Om Namo Bhagavate Vasudevaya
  \endverse
  \begin{feeler}
    Salutations to the Indweller who is omnipresent, omnipotent, immortal and divine.
  \end{feeler}
  \begin{explanation}
    \textbf{Vasudeva} is the individual aspect of God that dwells inside of us. This mantra frees 
    our minds and spirits from negative patterns in this life. Regular and consistent practice of 
    this mantra gives us a complete spiritual freedom: it frees us from the cycle of rebirth and 
    helps us realize ourselves as a manifestation of transcendent divinity. It can also help bring 
    in an advanced spiritual soul if chanted by the mother during pregnancy.
  \end{explanation}
\endsong


\beginsong{Hari Om Shiva Om Shiva Om Hari Om \\ Cosmic vibration Mantra}[]
  \beginverse
    Hari Om Shiva Om Shiva Om Hari Om
  \endverse
  \begin{explanation} 
    \textbf{Hari} is another name of Lord Vishnu. Can also be translated as The Remover of ego. 
    Universal mantra of cosmic vibration.
  \end{explanation}
\endsong


\beginsong{Om Eim Saraswatyei Namaha \\ Music and learning Mantra}[]
  \beginverse
    Om Eim Saraswatyei Namaha
  \endverse
  \begin{explanation}
    Salutations to Saraswati, the goddess of music, poetry, the arts, education, 
    learning and divine speech. Opens us towards education, learning, and the artistic world of 
    music and poetry. Whenever you find yourself moved to tears by a piece of music, or touched 
    by the words of the great poets and sages, you are in the presence of Saraswati. We offer this 
    mantra as a gift to our children, that they may be at ease while learning the wonders of 
    this unfolding mystery of Life.
  \end{explanation}
\endsong


\beginsong{Om Namo Narayana}[]
  \beginverse
    Om Namo Narayana
  \endverse
  \begin{explanation}
    I bow to the divine. Salutes the all-pervading aspect of the Great Spirit anchored 
    in our hearts and in all beings. Destroys barriers, obstacles, afflictions, and difficulties. 
    Leads to self-realisation. Traditionally chanted to assist the dying as they make their 
    transition, the mantra asks prayerfully, that we may all merge into the grace of divine light.
  \end{explanation}
\endsong

