% Note that this file is not supposed to be wrapped in 'songs' environment.

\begin{figure}[!ht]
  \caption{R. W. Ekman: Väinämöisen soitto (1866)}
  \centering
  \includegraphics[width=1.0\textwidth]{Ekman_-_Vainamoisen_soitto.jpg}
\end{figure}  

\clearpage
\subsection{Luonnon nostatus}
  \begin{large}
    \begin{center}
      Luontoani nostattelen \\
      haastattelen haltijata \\
      Nouse luontoni lovesta \\
      Syntyni syvästä maasta \\
      Syntyni syvästä maasta \\
    \end{center}
    \begin{center}
      Maasta hauna haltijani \\
      Nouse niinkuin nousit ennen \\
      Minun nostatellessani \\
    \end{center}
    \begin{center}
      Isoni luonto, emoni luonto \\
      Luonto valta vanhempani \\
      Oman luontoni lisäksi \\
      Nosta Ukon voima taivahasta \\
      Maasta maan Emoisen voima \\
    \end{center}
    \begin{center}
      Tuekseni turvakseni väekseni voimakseni \\
      Tulkaa tarvittaessa, käykää kutsuttaessa \\
      Terveyttä tekemään rauhaa rakentamaan \\  
    \end{center}
  \end{large}

  \paragraph{}
  \begin{em}    
    Eräs kielemme vanhimmista sanoista on noita, joka on alunperin tarkoittanut tietäjää eli 
    shamaania. Kristillisellä aikakaudella noita sai kielteisen kaiun. Sitä ennen tietäjät 
    olivat arvokas osa yhteisöä, jonka jäseniä he tukivat sairauksissa ja elämän murroskohdissa.       
      
    Sekä loveen lankeaminen että haltioituminen olivat tietäjien tapoja saavuttaa taianomainen 
    mielentila, jossa he saivat apua esivanhemmilta tai suojelushengiltä.    
  \end{em}
 
  \paragraph{Lovi} yliluonnollinen paikka tai olotila. Aukko arkitodellisuuden ja alisen maailman 
    välillä. Aliseen maailmaan kuului vainajala taikka tuonela, jossa esi-isät asustivat. Lovi 
    sijaitsi joko maan tai veden alla. Ilmeisesti myös naisten hameen alta, sillä sieltäkin 
    löytyi ovi tuonpuoleiseen. Osa muinaisista hautajaistavoistakin viittaa siihen, että vainaja 
    pyrittiin palauttamaan takaisin kohtuun. Vainajalle tehtiin vertauskuvallisesti se, mitä 
    sielulle toivottiin käyvän. Pohjolan emäntä Louhi, mahtava noita, tunnetaan myös muun muassa 
    nimillä Lovetar ja Loviatar. Lovetar ilmenee syntysanoissa ja loitsun sanoissa. Louhi pystyy 
    muuttamaan muotoaan, parantamaan, käskemään säätä, kuuta ja aurinkoa sekä synnyttämään mitä 
    ihmeellisempiä olentoja. Hänen kotinsa myyttinen Pohjola on pahojen asioiden, sairauden ja 
    pakkasen lähde. Monet Pohjolan ongelmat, sairaudet ja harmit ovat itse Louhesta lähtöisin.
  \paragraph{Loveen lankeaminen} tietäjän matkaa vainajalaan kutsuttiin loveen lankeamiseksi. 
    Hurmoksessa tietäjä teki sielunmatkan tuonpuoleiseen ja kävi kysymässä neuvoa esivanhemmilta.  
    Suomalaiset tietäjät saattoivat vajota hurmokseen esimerkiksi raivoamalla, kun taas 
    saamelainen noaidi käytti apunaan rumpua. Loveen lankeamisessa saatettiin ehkä käyttää apuna 
    myös laulamista (vrt. Väinämöinen laulaa Joukahaisen (Lapin tietäjän) suohon) ja 
    soittamista (Väinämöisen jättihauenluinen kannel, jota kuunnelleessaan luomakunta lumoutui 
    ja liikuttui). Väinämöinen lähtee Tuonelaan hakemaan puuttuvaa tietoa veneen rakentamiseen 
    tai reen korjaamiseen. Joissain kertomuksissa hän hakee myös työvälineitä. Matkaa Tuonelaan 
    kuvataan hyvin vaikeaksi ja vaivalloiseksi. Poiskaan sieltä ei ole helppo päästä, mutta 
    Väinämöinen pakenee muuttumalla vesikäärmeeksi ja uimalla Tuonelan joen ylitse. Tämä 
    viitannee muutokseen sieluneläimeksi, joiden joukossa käärmeet olivat suosittuja. Muutos 
    sieluneläimeksi oli yksi tietäjän keinoista.
  \paragraph{Lovesta nosto} Loitsimalla saatettiin nostaa suvun esivanhempia eli omakuntaisia 
    vainajalasta tälle puolen esimerkiksi suojelushaltijoiksi eli luonnoiksi.
  \paragraph{Luonto} on ihmisen tai muun tietoiseksi koetun olennon tai asian suojelushaltija. 
    Alkuperäiseen merkitykseen viittaa ajatus jonkun luontumisesta. Luonnon katsottiin 
    seuraavan ihmistä, suojelevan häntä ja tuovan hänelle onnea. Voimakasluontoiset eli ne 
    joilla oli voimakas oma haltija pärjäsivät elämässä heikkoluontoisia paremmin. Luonto 
    saattoi olla esimerkiksi vainajasta peräisin oleva esivanhempi eli syntyinen.
  \paragraph{Luonnon kutsunta} Jos ihmisellä oli heikko luonto, hän saattoi joko kutsua itselleen 
    vahvempaa luontoa tai voimistaa ja karaista luontoaan. Luontoa kutsuttiin myös kateita, 
    vihollisia ja sairauksia vastaan, ja antamaan lisää ruumiillisia, henkisiä tai 
    yliluonnollisia kykyjä. Loitsuissa luonto nostetaan yleensä lovesta tai syvästä paikasta, 
    vainajalasta. Joskus sitä kutsutaan nousemaan haon eli uppopuun alta. Joskus taas luontoa 
    kutsutaan nousemaan "haudan alta", joka sekin viittaa siihen, että luonto on vainajalasta. 
    Erään tiedon mukaan luontoaan saattoi karaista siten, että kohotti juhlayönä järvestä 
    hakoa ja kutsui loitsulla luontoaan. Luontoa puhuteltaessa hänet ilmaistaan usein 
    haltijaksi ja synnyksi. Tässä synty merkitsee joko myyttistä syntyä, esivanhemman sielua 
    syntyistä tai molempia. Esimerkkejä loitsuista:

    \begin{center}\begin{em}
      Nouse luontoni lovesta, \\
      Syntyni syvästä maasta, \\
      Ha'on alta haltijani \\
      Vastuksia voittamah, \\
      Katehia kaatamah, \\
      Sotisia sortamahan... \\
    \end{em}\end{center}

  \paragraph{Haltioituminen} Ihmisen haltioituessa hänen luontonsa eli haltijansa hallitsi häntä. 
    Tietäjät saattoivat tavoitella haltioitumisen tilaa sairauksia parantaakseen tai muihin 
    yliluonnollisiin tehtäviin. Nykyajan psykologiassa haltioituminen vertautuu lähinnä niin 
    sanottuun flow-tilaan, jossa ihminen on syvästi keskittynyt tavoitteisiinsa. Joskus 
    haltioituminen on käsitetty myös loveen lankeamisen vastineeksi tai taikojen 
    harjoittamiseksi. 
  \paragraph{Emuu} Synnyttänyt tai luonut jonkin kasvi- tai eläinlajin, ja vastuussa tämän lajin 
    toiminnasta sekä huolenpidosta. Metsästäjät pyysivät loitsuin kantavanhemmilta 
    metsästysonnea. Ajateltiin, että kantavanhempi päättää, kenelle antaa lapsiaan saaliiksi.
    Pahaa voitiin yrittää karkottaa vanhempansa luo, tai uhata, että jos olento ei tottele, 
    hänen käytöksestään kerrotaan hänen vanhemmalleen. Myös kantavanhempaa saatettiin käskeä 
    panemaan lapsensa kuriin.
  \paragraph{Herättäjä} herättää ihmisen vaaran uhatessa. Jotkut väittävät omistavansa oman 
    herättäjän, joka valvoo heidän untaan. 
  \paragraph{Juhlia} Nykyisistä juhlapyhistä laskiainen, helasunnuntai, juhannus, pyhäinpäivä ja 
    joulu olivat alun perin suomalaiseen alkuperäisuskontoon kuuluneita juhlia.


\subsection{Päivä, Aurinko, Kuu} 
  
  \paragraph{Päivätär} oli elämää ja valoa hallinnut jumala. Kristillisellä kaudella hänet 
    korvasi Neitsyt Maria. Auringon ja päivän jumalatar ja kaunis neito, jonka toveri tai 
    kaksois\-olento on yhtä lailla kaunis kuun jumalatar Kuutar. 

  Päivätär ja Kuutar omistavat kuun kultaa ja auringon hopeaa, joita näkyy kuun ja auringon 
  hohteessa, ja kultareunaisissa pilvissä auringonlaskun aikaan. Päivätär ja Kuutar näkyvät 
  toisinaan Pohjolan tyttären tavoin taivaalla kehräämässä kulta- ja hopealankaa ja kutomassa 
  siitä kultaisia ja hopeisia koruja ja vaatteita joita neidot pyytävät Päivättäreltä ja 
  Kuuttarelta kaunistuksekseen.
  
  \paragraph{Ilmarisen} mainitaan olleen mukana kosmisessa luomisessa, takomalla taivaankantta 
    ja asettamalla taivaankappaleita.

  \begin{em}
    Tuli vanha Väinämöinen, ovelle asetteleikse.
    Sanan virkkoi, noin nimesi: "Oi on seppo veikkoseni!
    Mitä paukutat pajassa, ajan kaiken kalkuttelet?"
    Se on seppo Ilmarinen sanan virkkoi, noin nimesi:
    "Kuuta kullaista kuvoan, hope'ista aurinkoa
    tuonne taivahan laelle, päälle kuuen kirjokannen."
  \end{em}



\subsection{Maa}

  \paragraph{Akka} Ukko ylijumalan nainen. Joskus tulkittu hedelmällisyyden jumalattareksi. Luonnon 
    naisellinen puoli, maaemonen, jonka Ukko hedelmöittää, ja sade saa viheriöimään. Näin saatiin 
    aikaan maanviljelyn kannalta suotuisat ilmat. 8034 Akka, Akan mukaan nimetty asteroidi.
  \paragraph{Maan haltija} on perhettä, taloa, satoa, karjaa ja pihapiiriä hoitava ja vartioiva 
    hyvä haltija, nais- tai miespuolinen. Jos se on naispuolinen, menestyy karja hyvin, jos 
    miespuolinen, menestyvät hevoset. Joskus haltijoita on pariskunta. Tällöin talo on erityisen 
    onnekas. Haltijaa tulee kunnioittaa, niin hän hoitaa työnsä hyvin. Haltijalle on usein 
    luvattu, että joka vuosi jonkun juhlan aikaan se saa osansa juhlaruuasta tai sadosta. Pyhä 
    pihapuu on uhripaikka, jonka luona haltijaa kumarretaan ja sille viedään uhrilahjoja. Jos 
    haltija on hyvissä vaatteissa ja iloinen, ei hän ole onnettomuuden aiheuttaja, ja on vain 
    varoittamassa. Jos haltija on pahoissa vaatteissa eikä näytä kasvojaan, on hän vihastunut, ja 
    aikomassa aiheuttaa onnettomuuden. Erään perinteen mukaan tällöin on kiireesti mentävä 
    paikalle, jossa haltija on näyttäytynyt, ja uhrattava alasti. Kun perustetaan uutta taloa, 
    tehdään taikoja, jotta saataisiin haltija. Muuten voivat pahat haltijat tai maanväki asettua 
    pihapiiriin. Erään kansanperinteen mukaan mukaan pitää polttaa tulta kolme päivää sillä 
    paikalla, johon haluaa talon perustaa. Kolmantena päivänä haltija ilmestyy unessa. Silloin 
    näkee, tuleeko mies vai nais-haltija vaiko pariskunta.  
  \paragraph{Maahiset} eli \textbf{maanväki} ovat pieniä, ihmisenmuotoisia olentoja. Asuvat maan 
    alla omassa maailmassaan. Usein maahisten maailma koetaan todellisen maailman peilikuvaksi, 
    joka voi olla ylösalaisin. Maahiset voivat olla ilkikurisia. Metsässä saattaa joutua maahisten 
    lumoihin ja eksyä nurinkuriseen maailmaan, jota metsänpeitoksikin on kutsuttu. Maan väki 
    voidaan käsittää myös taikavoimaksi, jota on maassa. Maanpäällisessä maailmassa maahiset ovat 
    yleensä näkymättömiä. Maahisten kerrotaan myös lumoavan ihmisiä jäämään luokseen, jos ottaa 
    vastaan heidän tarjoamiaan lahjoja, ruokaa tai juomaa.
  \paragraph{Sampsa Pellervoinen} kylvää maan kasvillisuuden, kaikenlaiset metsät, suot, ahot 
    ja kivikotkin. Kylvö tapahtuu sammon murusten avulla. Hedelmällisyyden haltija, joka on 
    rituaalisesti herätettävä joka kevät. 
  \paragraph{Äkräs} on monipuolinen kasvillisuuden haltija. \begin{em}Loi herneet, pavut ja 
    nauriit sekä antoi kaalit, pellavat ja hamput: Egres hernet Pawudh Naurit loi / Caalit Linat 
    ia Hamput edestoi.\end{em}
  \paragraph{Jumi} on yliluonnollisten ilmiöiden, lähinnä panteistisen maailmanhengen nimitys. 
    Marit viettävät vuosittain Jumolle omistettua juhlapäivää uhraten lihaa ja viljaa. Jumi 
    aiheuttaa esimerkiksi eläimelle äkillisen taudin ampumalla näkymättömän nuolen. Arvoituksissa 
    jumi on jokin, jolle on ominaista ehdoton paikallaanolo tai jota on mahdoton kiertää.
    
  \subsubsection{Kivet}
    \paragraph{Kyllikki} kerrotaan kivien emuuksi loitsuissa, joilla kiven aiheuttamia vammoja 
      hoidetaan
  \subsubsection{Käärme}
    \paragraph{Käres} käärmeitten emuu  
    \paragraph{Mammotar} matojen emuu, käärmeitten synnyttäjä


    
\subsection{Meri ja Vesi}
  \paragraph{Ahti} Veden jumala tai haltija. 
  \paragraph{Veen Emonen, Veden emo, Vellamo} on hyvä ja arvostettu vedenhaltija, Ahdin puoliso, 
    joka asui veden alla Ahtolassa. Kuului kalansaaliin toivossa palvottuihin vedenhenkiin. Ohjaa 
    kalat verkkoon. Aaltojen nostaminen tai purjehdussäähän vaikuttaminen. Kaunis; sininen lakki, 
    kaislainen paita ja vaahtoinen vaippa.
  \paragraph{Väinämöinen}-nimen uskotaan juontuvan väinä-sanasta, joka tarkoittaa suvantoa tai 
    hiljalleen virtaavaa vettä, salmea tai joensuuta. Väinämöisen lisämääreenä esiintyy myös 
    \textbf{suvantolainen}. Väinämöinen on Ilmattaren, ilman immen ja meren poika. Hän viettää 
    meressä kelluvan äitinsä kohdussa kolmekymmentä vuotta ja alkaa synnyttyään autella maailman 
    luomisessa. Taidokas veneenveistäjä. Hän tietää melkein kaiken tarpeellisen veneen tekoon, 
    mutta ei kolmea ratkaisevan tärkeää sanaa, luotetta. Hän lähtee hakemaan niitä Tuonelasta, 
    kuolleiden maasta. Väinämöinen käy myös ammoin kuolleen tietäjä Vipusen vatsassa tietoa 
    hakemassa. Matka kuolleiden maahan on kuviteltu hyvin vaaralliseksi, vain mahtavien 
    tietäjien on ajateltu pystyvän siihen ja tulemaan takaisin. Tässä on uistumia rituaaleista, 
    joissa tietäjä vajoaa transsiin, ja hänen sielunsa liikkuu kuolleiden maailmassa suorittamassa 
    tehtävää; suomalaisessa mytologiassa kerrotaan loveen lankeamisesta ja lovesta nostosta. 
    Tuonen tytti saattaa veneellään vainajat Tuonen mustan virran yli. Väinämöisen tytti kuitenkin 
    huomaa olevan elossa. Väinämöinen valehtelee kuolleensa, mutta ei vakuuta. Tytti tietää 
    millaisia ovat eri tavoin kuolleet - rautaan kuolleet ovat verisiä, hukkuneet vetisiä ja 
    palaneet kärventyneitä. Väinämöinen kuitenkin lopulta pääsee Tuonelle. Tuonella hänet 
    laitetaan nukkumaan vastenmieliseen sänkyyn tai paikkaan, joka on tehty käärmeistä tai 
    täynnä käärmeitä. Hän onnistuu kuitenkin pakenemaan. Hän muuttuu vesieläimeksi, ja ui Tuonen 
    joen poikki. Tuonen poika virittää rautaverkon veteen, mutta ei saa Väinämöistä tarttumaan. 
    Väinämöinen palaa Tuonelta saatuaan haluamansa, ja kertoo eläville Tuonelan oloista. 
    Joissain runoissa kerrotaan, että Väinämöinen menee veneellään Rutjan koskeen, tuliseen 
    pyörteeseen. Pyörre on usein tulkittu reitiksi Tuonelaan, vauhdikkaammaksi versioksi Tuonen 
    joesta. Väinämöinen menee siis elävänä kuolleiden maahan. Väinämöisen veneenjäljeksi 
    kutsutaan tyyntä kohtaa muuten aaltoilevalla vedenpinnalla. Vedenpinnan nimitysten lisäksi 
    Väinämöiseen liittyviä nimityksiä löytyy luonnosta tähtitaivaalta, kuten Väinämöisen miekka 
    tai viikate (Orion) ja Väinämöiset tai Väinämöisen virsut (Seulaset). Näitä tähdistöjä on 
    käytetty suunnistamiseen vesillä. Väinämöisen arvellaankin alun perin liittyneen kiinteästi 
    vesillä liikkumiseen. Kalevalassa Väinämöinen syntyy Ilmattaresta vanhana miehenä. 
    Kansanrunoissa Ilmatar ei synnytä Väinämöistä, vaan Väinämöinen syntyy yksin tai Iro-neidosta.


    
\subsection{Tuli}

  \paragraph{}Samanistisessa kosmologiassa ihmisen maailma sijaitsi henkien asuttamien 
    monikerroksisten ylä- ja alamaailmojen välissä. Joissakin se kuvattiin kodan pohjaksi, josta 
    taivaankupoli rajasi henkien asuinsijat. Molemmissa ajatusrakennelmissa taivasta kannatti 
    maailmanpylväs ja taivaan napaa edusti Pohjantähti. Käsitys sielusta kuului ikivanhaan 
    pohjoiseen samanismiin. Sielun ja ruumiin pitkästä rinnakkaisuudesta kertoo muun muassa 
    sukulaiskansojemme mansien ja hantien sana is, joka on sukua suomen sanalle itse. Tähän 
    liittyy löyly, joka sukukielissämme viittaa kylpemisen ohella sieluun. 
  \paragraph{Ukko ylijumala} on muinainen sään ja sadon, ilman, oikeuden sekä 
    \textbf{ukkosen jumala}. Pyydettiin avuksi taisteluun tai \textbf{taikuuteen} ryhtyessä. 
    Ukon lisänimen ”ylijumala” voikin tulkita kahdella tapaa; joko että hän todella oli mahtavin, 
    ”ylin”, jumalista, tai pelkästään, että hän asui ylhäällä taivaalla. Vahinkoa tekevän 
    salamaniskun ajateltiin olevan Ukon rangaistus tai vihan ilmaus, ja elämää tuovan sateen 
    suopeuden osoitus. Ukkoa muistuttava ukkosenjumala tunnettiin latvian kielessä nimellä 
    Perkons ja liettuan kielessä nimellä Perkūnas, joista on peräisin suomen sana Perkele. 

    Iskee salamoita kirveellä, vasaralla, nuolella tai miekalla. Kokonaisen ukonilman hän saa 
    aikaan puimalla riihtä, kyntäen, jyristellen vaunuillaan taivaissa, makaamalla naispuolisen 
    jumaluuden kanssa, taikka kolisuttamalla konkeloa eli kelopuuta. Savukvartsia, josta 
    iskettäessä syntyi kipinöitä ja palaneen hajua, nimitettiin ukonkiveksi. 

    Ukko käveli pitkin askelin pilvien ja monikerroksisen taivaan yläpuolella, ja katseli 
    ylhäältä maailmaa. Mahtavuudestaan huolimatta Ukko ei ollut kaikkitietävä tai kaikkivaltias, 
    vaan muilla henkiolennoilla ja ihmisillä oli valtaa ja toimintatilaa. Ukon taivaallisessa 
    valtakunnassa oli heikompia henkiolentoja, joista perinteet tuntevat muun muassa päivättäret, 
    kuuttaret, ilman immet, kapeet, kuumet, tuulettaret ja muita. 

    Ukon kerrotaan myös pitäneen pilvissä käräjiä, joten voisi olettaa, että taivaalla asui 
    tai ainakin vieraili muitakin merkittäviä henkiolentoja, joiden kanssa Ukko päätti asioista. 
    Ukkoa rukoiltiin pitämään käräjiä eri ongelmien voittamiseksi.

  \paragraph{Kokko} tai \textbf{vaakalintu} on jättiläismäinen kotka, sukua yleismaailmalliselle 
    ukkos\-lintu-hengelle, joka tunnetaan Euroopasta aina Amerikan intiaaneille asti. Se voi olla 
    muistumaa ihmishahmoista jumaluutta edeltäneestä käsityksestä, jonka mukaan taivasta ja 
    ukkosta hallitsee ukkoslintu. Ukkoslintuihin uskovat paitsi jotkin suomalais-ugrilaiset myös 
    monet muut kansat. Kokko ja Ilmarinen osallistuvat yhä joissain tulen syntysanoissa 
    ensimmäisen tulen iskentään, joka toisissa perinteissä on Ukon tehtävä.Toisinaan 
    tarusankarien ystävä, toisinaan vihollinen. Joskus sen tehtävä on vartioida. Kokko kuvataan 
    joskus rautaiseksi, joskus tuliseksi. Pystyy kantamaan ihmistä. Iskee tulta auttaakseen 
    Väinämöistä polttamaan kasken. Myös kokon sulkia käytetään tulen iskemiseen. Joissain tulen 
    syntykertomuksissa tuli on isketty kokon sulilla. Kokon mittasuhteet kuvataan joskus 
    valtaviksi: toinen siipi haroo taivasta kun toinen osuu meren luotoihin. Kokkoja voi olla 
    useita. Eräs näistä pelastaa Väinämöisen merihädästä palkkioksi siitä, että kaskea 
    kaataessaan Väinämöinen jätti koivun linnuille istumapuuksi. Ilmarinen ja Louhi tekevät 
    omat kokkonsa. Ilmarinen tekee metallisen kokon, joka pyydystää suomuhauen. Louhi taas 
    muuttuu itse kokoksi rakentamalla siivet ja pyrstön laivan osista ja ottamalla viikatteet 
    kynsiksi. Sanassa vaakalintu esiintyvä ’vaaka’ tulee mahdollisesti sanasta vaa’as, joka 
    tarkoittaa myyttistä tulta, aaltoa ja kipua. Toinen mahdollisuus on vuokko, saamelaisten 
    kertomusten tietäjän apulintu.
  \paragraph{Panu} on tulen henki, auringon poika. Panuun voi vedota loitsuissa, kun ollaan 
    tekemisissä tulen kanssa. 

  
  
\subsection{Ilma ja Tuuli}

  \paragraph{Ilmattaret} eli \textbf{Ilman immet} ovat taivaalla asuvia jumalaisia neitoja, jotka 
    muistuttavat Pohjan neitoja, sillä nekin istuvat ajoittain taivaankannella. Ilmatar liittyy 
    Väinämöisen ja maailman syntymään. Ilmatar tylsistyy oloonsa impenä taivaalla ja laskeutuu 
    meren selälle. Tuuli hedelmöittää hänet. Sotka etsii pesäpaikkaa ja huomaa Ilmattaren polven, 
    jolle se tekee pesän ja munii siihen munansa. Polvea alkaa kuumoittaa ja Ilmatar heilauttaa 
    sitä, jolloin munat putoavat ja särkyvät. Särkyneistä munankuorista syntyy maailma. 
  \paragraph{Seppo Ilmarinen} Tuulen, sään ja ilman jumala, mukana maailman syntymässä. 
    Seppäsankari, jolla on myös jumalallisia piirteitä. Iro-neito synnytti Ilmarisen yöllä ja jo 
    päivällä Ilmarinen teki pajan. Palkeet liittyvät Ilmarisen asemaan tuulen jumalana, mutta 
    toisaalta myös tämän rooliin kosmisena seppänä. Ilmarisen taontatyö ei onnistu ennen kuin hän 
    tarttuu palkeisiin orjien sijasta itse.  Ilmarinen takoi maailman alussa taivaankantta niin 
    taidokkaasti, etteivät näy pihtien pitämät, eivätkä tunnu vasaran iskut. Ilmarisen 
    kädenjälkeä ovat myös revontulet, aamu- ja iltaruskon värit. Myös raudan keksiminen ja 
    alkutulen iskeminen ovat Ilmarisen saavutuksia. Hän asuu Väinölässä. Takoo monia esineitä, 
    kuten sammon, Kultaisen naisen, ja yrittää takoa uuden Auringon ja Kuun. Suomalaisilla lienee 
    ollut Ukkoa aikaisempi, omaperäinen taivaan jumala. Tämän syrjäydyttyä Ukon tieltä siitä 
    kehittyi kalevalaisen perinteen seppäsankari Ilmarinen. Ilmarisen asemasta taivaan 
    jumaluutena on säilynyt muistumia myytteihin, kuten uskomukset, että hän takoi taivaankannen 
    ja Sammon, joka alkujaan käsitettiin taivaan tukipylvääksi.
  \paragraph{Sielulintu} oli sielun koti ja vertauskuva. Lintu ehkä toi sielun syntymässä, ja 
    vei sen kuoleman hetkellä. Joidenkin perinteiden mukaan nukkuessa oli hyvä olla lähellä 
    puusta veistetty sielulintu, joka pitäisi huolta sielusta unen aikana, ettei se lähtisi 
    omille teilleen. Ihmisen kuoltua hänen puinen sielulintunsa laitettiin ortodoksisen 
    hautaristin yläpuolelle. Lintujen ruokkiminen jouluna on vanha tapa. Kuolleet eli 
    sielulinnut olivat elävien kanssa mukana keskitalven juhlassa. Samanistisessa ihmiskuvassa 
    sielu oli usein monikerroksinen. Sen osista yksi saattoi unessa tai transsissa liikkua 
    kehon ulkopuolella esimerkiksi linnun hahmossa. Lintujen merkitys suomalais-ugrilaisille 
    näkyy siinäkin, että taivaan halkaisevan galaktisen vyön nimi on Linnunrata. 
  \paragraph{Tuuletar} tarkoittaa naispuolista tuulta hallitsevaa luonnonhenkeä tai tuulen 
    personoitumaa. Tuulettaria on erilaisia. Tuuletar saattaa olla yksittäinen tuulenpuuska, 
    vihuri tai tuulispää, tai tietynlainen jatkuva tuuli. Tuuletar voi myös olla tuulen jumala, 
    jolta pyytämällä saa suotuisaa ilmaa. Tietty tuuletar voi myös vastata tietystä 
    ilmansuunnasta tulevaa tuulta.
  \paragraph{Tapiotar} lintujen emuu 



\subsection{Metsä}

  \paragraph{Tapio} on metsän haltija. Hallitsee metsäistä valtakuntaansa Tapiolaa. Tapion väki 
    kaunistivat ja siivosivat metsää ja huolehtivat kasveista ja eläimistä. Tapion tyttäriä ovat 
    ihastuttavat Tellervo, Tyytikki, Tuulikki ja Annikki. Tapiota tai hänen perhettään 
    kuvaillaan ihmishahmoisiksi ja joko alastomiksi tai kauniisti pukeutuneiksi. Joidenkin mukaan 
    Tapion parta oli puuta ja silmät kuin kaksi pohjatonta järveä.  
  \paragraph{Mielikki} Tapion vaimo. Metsästäjien oli joskus puheltava ja laulettava 
    viettelevästi ja imartelevasti Mielikille saadakseen tältä lahjana saalista. Metsän 
    "taloustöiden" eli siistimisen, koristelun ja kaunistamisen hoitaja. Työn tuloksia, metsän 
    kauneutta, kuten myös Mielikin omaa kauneutta, kannatti metsässä liikkuessa kehua. Uskottiin 
    lepyttävän miestään Tapiota, jos tämä tuli huonolle tuulelle ja usutti voimat metsämiehiä 
    vastaan. Harvoin ihmisille näyttäytyessääm Mielikki usein huvikseen pukeutuu Tapion 
    harmaaseen naavaturkkiin ja -hattuun. Katsojan silmissä kuusikossa kulkee höperö vanhus, 
    joka laskee mättäiden marjoja.  Mielikki on taitava parantaja. Hän hoitaa ansoihin jääneet 
    käpälät ja tassut, pesästä pudonneet linnunpoikaset ja metsokukkojen taisteluhaavat. Metsän 
    parantavat kasvit hän kerää huolellisesti talteen, ja niinpä hänellä on sopivia rohtoja 
    myös ihmisten vaivoihin, jos joku vain keksii käydä pyytämässä. Mielikin käyttämiä kasveja 
    ovat muun muassa kanerva ja kataja. Pienriistan pyytäjän, sienestäjän ja marjastajan 
    kannattaa lausua metsään mennessään: "Siniviitta, viidan eukko, / mieluinen metsän 
    emäntä! / Anna tie, avaa portti / minun metsällä käydessäni." Mielikin nimi juontuu onnea 
    ja kohtaloa merkinneestä mielu-sanasta.
  \paragraph{Tyytikki} oravien emuu; Tapion ja hänen puolisonsa Mielikin tytär
  \paragraph{Metsän haltijat ja olennot} Metsässä oli myös arvaamattomampia tai vihamielisiä 
    olentoja, kuten maahisia, metsähiisiä, menninkäisiä ja keijuja. Nämä saattoivat sairastuttaa, 
    eksyttää tai lumota, jos tuli näiden valtapiirille. Menninkäiset saattoivat eksyttää metsässä 
    kulkevan nurinkuriseen maailmaan. Metsässä saattoi olla myös noidankehä, alue, johon joutunut 
    lumoutui. Tätä saattoi merkitä esimerkiksi sienien kehä. Väkeä asuu muun muassa 
    muurahaiskeoissa, puunkoloissa, kivenkoloissa, juurakoissa ja kannoissa. Metsän väen 
    taikavoimaa saattoi löytää myrskyn katkaisemien puiden murtumakohdista tai yhteenkasvaneista 
    puista. Avoimella paikalla kasvava yksinäinen puu oli tärkeä metsänväen kokoontumispaikka.
  \paragraph{Metsänneito} (myös \textbf{metsänneitsyt}, \textbf{metsänpiika}, \textbf{sinipiika}) 
    ilmaantui joskus metsässä liikkuville tai yöpyville miehille. Saattoi tulla tanssimaan 
    nuotiolle tai kävellä vastaan. Metsänneitsyt oli edestäpäin ihastuttavan kaunis, 
    harsopukuinen ja pitkähiuksinen, mutta olennon selkäpuoli oli ontto, tai takaa se oli vain 
    puupökkelö. Tämän huomasi kauhukseen mies, jos yritti nähdä neidon selkäpuolen. Tällöin 
    metsänneitsyt pelästyi ja lähti.
  \paragraph{Menninkäinen} on yksinäisillä paikoilla asustava, yleensä ihmisille suopea, pieni 
    ja pimeästä pitävä. Ilmeisesti alun perin tarkoittanut vainajaa ja manalaista. Menninkäiset 
    eivät välttämättä kestä päivänvaloa. Vieraita ja outoja olentoja, joiden motiivit ovat 
    tuntemattomat ihmisille, päinvastoin kuin tiettyihin elementteihin liittyvien väkien. 
    Menninkäiset ovat voineet tulla mellastamaan kirkkoon öisin, jolloin niitä on pidetty 
    pikku paholaisina. Ne saattavat pälyillä ihmisiä ikkunan tai puunrungon takaa, tai 
    istuskella ryhmänä kivellä ihmistä tuijottaen. Menninkäiset järjestävät mielellään pitoja, 
    joissa syödään, juodaan ja tanssitaan. Menninkäiset pitävät kiiltävistä esineistä.
  \paragraph{Ajattara} paha naispuolinen metsään liittyvä. Ajoi metsämiehiä ja metsästäjiä 
    harhaan. 
  \paragraph{Äimätär} susien emuu 

  

\subsection{Puut}

  \paragraph{Kati}  puiden emuu metsän kaunis ja nuori jumalatar, joka synnyttää puita 
  \paragraph{Elämänpuu} Nainen nähdään elämän ja kuoleman sekä niitä vastaavien ilmansuuntien 
    etelän ja pohjoisen hallitsijana. Tämän äitihahmon mielikuvastoon liittyvät myös aurinko ja 
    elämänpuu, joka usein miellettiin koivuksi. Kalevalassa Iso Tammi kohosi peittämään koko 
    taivaankannen ja se lopulta kaadettiin. Aihelmaa on selitetty Linnunradan syntynä, sillä 
    Linnunrata muistuttaa muodoltaan kaadettua puuta.
  \paragraph{Koivu} (vesi), Suojelu, puhdistaminen. Koivunoksia on käytetty kautta aikain 
    karkottamaan pahoja henkiä ihmisestä vihtomalla. 
  \paragraph{Poppeli} (vesi) Vauraus, lentäminen
  \paragraph{Haapa} (ilma) suojelee varkauksilta, parantaa ilmaisukykyä
  \paragraph{Mänty} (ilma) Parantaminen, hedelmällisyys, vauraus
  \paragraph{Pihlaja} (tuli) Psyykkiset voimat, parantaminen, voima, menestyminen, suojelu
  \paragraph{Tammi} (tuli) Suojelu, terveys, vauraus, paraneminen, potenssi, hedelmällisyys, onni
  \paragraph{Saarni} (tuli) Suojelu, terveys, mereen liittyvät rituaalit, vauraus. 
  \paragraph{Kataja} (tuli) Suojelu, varkauksien esto, rakkaus, terveys



\subsection{Tieto}

  \paragraph{Antero Vipunen} on maan alla makaava vainaja tai jättiläinen; tietäjä, jolla on 
    hallussaan arvokkaita ikiaikaisia loitsuja tai tietoja. Väinämöisen loitsusta puuttuu kolme 
    sanaa eli luotetta. Ne saadakseen hän menee herättämään nukkuvan Vipusen joko hakkaamalla 
    puut tämän haudalta, tai menemällä suusta vatsaan. Vipunen voi myös nielaista Väinämöisen. 
    Vatsassa Väinämöinen takoo niin kovasti, että Vipunen luovuttaa, ja antaa sanat vatsasärystä 
    päästäkseen. Viron kielessä vibu on jousipyssy, joten Virossa Vipunen on käsitetty taitavaksi 
    jousimieheksi. Useissa kertomuksen versioissa Vipusen mainitaan olevan ansastaja, kuten 
    nimestä Vipunen voi päätellä. Arvellaan lainatun saamelaisista tarinoista, joissa käytiin 
    noita Antereeuksen haudalla hakemassa tietoja.  
  \paragraph{Väinämöinen} on taidokas loitsujen laulaja ja kanteleen soittaja. Veistää veneen 
    laulamalla. Tekee itselleen kanteleen. Sotajoukon laiva pysähtyy jättiläismäisen suomuhauen 
    selkään. Hauki tapetaan, ja sen leukaluusta Väinämöinen tekee kanteleen. Kanteleen kielet hän 
    saa jonkin Hiiden olennon hiuksista. Väinämöinen soittaa kanneltaan niin taidokkaasti, että 
    ihmiset, eläimet ja jumalolennotkin tulevat kuuntelemaan, eikä ole karskeintakaan urosta, joka 
    ei liikuttuisi kyyneliin. Loitsuaa kateellisen Joukahaisen suohon. 
  \paragraph{Väinämöinen} Runonlaulaja ja suuri tietäjä: Vaka vanha Väinämöinen, tietäjä 
    iänikuinen. Auttaa maailman luomisessa. Eräässä yleisessä kansanrunossa Väinämöinen syntyy 
    yöllä, tekee päivällä pajan, takoo rautaisen hevosen, ja ratsastaa sillä veden päällä. 
    Joukahainen on nuori ja laiha Lapin tietäjä kadehti Väinämöisen laulutaitoja ja matkusti 
    kolme päivää haastamaan tämän miekan mittelöön. Väinämöinen ei suostunut miekkailemaan. He 
    loihtivat kilpaa, jonka päätteeksi Väinämöinen lauloi hänet suohon. Pelastautuakseen 
    Joukahainen lupasi Väinämöiselle siskonsa Ainon puolisoksi. Aino hukuttautui, sillä hän ei 
    halunnut vaimoksi vanhalle Väinämöiselle. Väinämöinen ratsastaa veden päällä, ja Joukahainen 
    ampuu hänet kostoksi alkumereen. Aiemmissa runoissa Väinämöinen ja Joukahainen ovat saman 
    äidin, Iro-neidon lapsia. He lähtevät yhtä matkaa kulkemaan, mutta ajautuvat erilleen ja 
    käyvät toistensa kimppuun. Väinämöinen voittaa taikansa avulla. Maailmankaikkeus syntyy, kun 
    sotka munii munansa Väinämämöisen polvelle, kun Joukahainen on suistanut hänet veteen. 
    Väinämöinen kelluu vedessä, kun vesilintu, sotka, pesii hänen polvelleen. Haudonta polttaa 
    polvea, jolloin Väinämöinen vavahduttaa sitä. Linnun munat joutuvat mereen, hajoavat ja 
    synnyttävät maailman. Kalevalassa sotka munii Väinämöistä odottavan ilmattaren polvelle, 
    eikä Väinämöisen polvelle kuten kansanrunoissa. 
  \paragraph{Väinämöisen paluu ?} Neitsyt Marjatta, saa poikalapsen tai lapsi löytyy metsästä, 
    jolloin etsitään turhaan myös äitiä. Väinämöinen määrää lapsen äpäränä suolle vietäväksi ja 
    puulla päähän lyötäväksi. Sylilapsi alkaa puhua, syyttää Väinämöistä pahemmista synneistä, 
    ja huomauttaa, ettei Väinämöistäkään ole niiden takia viety suolle. Lapsi kastetaan 
    Kaukomieleksi Karjalan kuninkaaksi. Väinämöinen suuttuu ja häpeää ja poistuu vaskisella ja 
    kuparisella veneellään, ja sanoo palaavansa, kun häntä tarvitaan, etsitään ja kaivataan. 
    Samaan tapaan vesitse ja kristinuskon ahdistamina ovat poistuneet myös Kalevanpojat, mutta 
    he soutivat kivellä.


\subsection{Hiisi} 

  Pyhä kulttipaikka, pyhä lehto ja mahdollisesti kalmisto. Hiideksi on myös myöhemmin alettu 
  kutsua kulttipaikalla palvottua henkiolentoa, kalmiston vainajien yhteensulautuneiden henkien 
  muodostamaa kollektiivia. Kristinuskon jälkeen hiisi on ollut paha henkiolento ja paha paikka. 
  Hiisistä muodostui kansantarinoihin pieniä pahoja tai vähintään tuhmia olentoja, joiden 
  kotipaikka oli myös nimeltään Hiisi, tai joskus Hiitola. Metsähiisi ja vesihiisi olivat 
  metsässä ja vedessä asuvia hiisiä, mutta myös sairauksien nimiä. Myös pyhästä hauta-alueesta 
  tai helvetin kaltaisesta paikasta saatettiin puhua hiitenä. Myöhemmin hiisi-sana alkoi viitata 
  pakanalliseen henkiolentoon, pienikokoiseen ilkeään haltiaan tai peikkoon. Kalevalassa 
  Väinämöinen joutui taistelemaan Lempoa, Pahaa ja Hiittä vastaan. Hiisien kotipaikka Hiitola 
  sijaitsi vaikeakulkuisessa maastossa, syrjässä ihmisasutuksesta. Hiittä pahana paikkana on 
  peräpohjolassa kutsuttu sanalla helsinki. Joskus rinnastettiin myös jättiläisiin tai 
  vuorenpeikkoihin. Hiitolaan oli rakennettu Hiiden linna, jossa Hiisi asui Hiiden emännän 
  kanssa. Hiiden emännältä Väinämöinen sai kanteleeseensa kielet. Hiiden linnassa asui myös 
  hiitolaisia - joiden hiukset olivat käärmeitä - sekä Hiiden raivoava rakki ja Hiiden kissa, 
  Kipinätär nimeltään. Hiidellä oli myös Hiiden ruuna, hiisien nopea hevonen ja yksi ainoa 
  tytär, Hippe. Hipen tehtävänä oli laittaa varkaat palauttamaan varastetun omaisuuden oikeille 
  omistajilleen. 
  
  \paragraph{Hiiden hirvi} Vaikeasti pyydystettävä, voimakas ja nopea. Voi olla samaa kantaa 
    kuin samojedien ja obinugrilaisten taivaallinen peura, kuusijalkainen olento, jonka 
    ensimmäinen samaani pyydysti. 



\subsection{Pohjola}
  Pohjolan emännän valtakunta kuvataan pahaksi ja kylmäksi maaksi kaukana pohjoisessa. Pohjola 
  oli sekä sankarien vihollinen, että pahojen asioiden, kuten pakkasen ja sairauksien, alkulähde. 
  Pohjolan hyviä oloja kuitenkin kadehditaan, eivätkä kielteiset käsitykset estä kalevalaisia 
  sankareita matkaamasta kohti Louhen valtakuntaa ja kosiskelemasta hänen yliluonnollisen 
  kauniita tyttäriään.
  
  \paragraph{Pohjolan emäntä} eli \textbf{Pohjan akka} johtaa myyttistä Pohjolaa. Kalevalassa 
  Pohjolan emännän nimi on \textbf{Louhi}, mutta tunnetaan muitakin nimiä, \textbf{Lovetar}, 
  \textbf{Loviatar}, \textbf{Louheatar} ja \textbf{Lovehetar}. Viitataan säeparilla: Louhi 
  Pohjolan emäntä Pohjan akka harvahammas. Pohjolan emännällä on suunnattomat taikavoimat: hän 
  pystyy muuttamaan muotoaan, käskemään säätä, säätämään auringon ja kuun kulkua, parantamaan 
  ja on kykeneväinen synnyttämään mitä ihmeellisimpiä olioita. Hän varustaa sotaveneen soutajineen 
  ja sotaväkineen, ja veneen tuhouduttua ottaa veneen laidat siivikseen, ja muuntautuu 
  jättimäiseksi Kokko-linnuksi, jonka selkään soturit nousevat. Pohjolan emäntä on synnyttänyt 
  monia ongelmista, jotka tulevat Pohjolasta ihmisten harmiksi. Hänet on usein portoksi mainittu 
  siksi, etteivät nämä "lapset" ole avioliitosta peräisin. Hänet hedelmöittää tuuli, kun hän 
  paljastaa alapäänsä pohjoiseen päin, tai sen tekee Iku-Tursas meren kuohuilla kovilla. Louhi 
  mainitaan usein myös ihmeellisten Pohjolan neitojen äidiksi. Pohjolan isäntä jää yleensä 
  sivuosaan. Yleensä miestä ei mainita ollenkaan, ja paheksuen korostetaan, että emäntä luo 
  synnyttämänsä asiat ja olennot luonnottomasti ja aviottomasti. 
 


\subsection{Sampo} 
  Ihmeellinen, rikkauksia tekevä mylly. Kirjokansi, ihmekone, joka jauhaa rahaa, viljaa ja 
  suolaa. Sampo jauhaa, sillä maailmanpylvään on ajateltu kiertyvän taivaankannen mukana. Sillä 
  arvellaan myös olevan jalat, joiden avulla se seisoo, sekä reiät tai nokat. Sammon juuret ovat 
  syvällä maaemässä, ja Sammon yleinen toisintonimi on kirjokansi, joka viittaa taivaankanteen. 
  Sammas on patsas, joka kannattaa taivaan kantta. Kultanappi tai kultainen naula oli pohjantähti, 
  joka kiinnitti taivaankannen ja maan. Taivaankansi pyöri naulan ympäri. Tästä hän käytti myös 
  ilmaisua ”sammas jauho”. Sammas ulottui yhdeksän sylen syvyyteen maan alle, ja nojautui siellä 
  kupariseen vuoreen. Vilhunen kertoi myös seisauttaneensa verenvuotoja iskemällä puukon seinään 
  Pohjantähteä vasten. Sampo tulee kultanaulalla kiinnitetystä taivaan kantta kannattavasta 
  pylväästä, jonka ympäri kirjokansi kiertää. Kultanaulan kohdalla on napatähti, esimerkiksi 
  Pohjantähti tai Vega, joka oli 12 000 vuotta sitten Pohjantähden kohdalla. Myös 
  indoeurooppalaisilla kansoilla on maailmanpylväsmyytti. Pylvästä kannatteli kilpikonna 
  esimerkiksi Intiassa.

  Sampo-tarut ovat olleet paitsi ihmeellisiä kertomuksia kahden kansan taistelusta ja 
  ihme-esineen vaiheista, alkujaan kosmisia kuvauksia. Selittäneet maanviljelyksen syntyä ja 
  muidenkin asioiden myyttistä alkuperää. Sampo on esiintynyt myös nimillä sampa, sammas, sampi, 
  sampu, samppu, samppo ja sammut. Sanana sampo on sukua sammakselle eli pylväälle. Laajalti 
  esiintyviä sammon piirteitä ovat hyvyys, kirjokanteen viittaus ja yhteys merelliseen Pohjolaan. 
  Sampo liitetään seppien tuomaan uudenlaiseen hyvinvointiin, joka uhkaa sysäyttää Väinämöisen 
  ja tietäjäkulttuurin.

  Väinämöinen lähtee Pohjolaan kosimaan Pohjolan tytärtä. Pohjolan emäntä a\-settaa kosijalle 
  ehdoksi, että Väinämöisen on taottava sampo. Väinämöinen ei tähän pysty, ja hän lähtee 
  pettyneenä kotiin. Hän kuitenkin lähettää sepän, Ilmarisen, takomaan Pohjolaan sammon. Seppo 
  Ilmarinen takoo Sammon Pohjolan emännälle vastineeksi Pohjolan tyttärestä. Pohjolan emäntä 
  Louhi ottaa Sammon vastaan, mutta kieltäytyy luovuttamasta tyttöä Ilmarisen vaimoksi. Louhi vie 
  Sammon kivisen mäen sisään, juurruttaa sen maaperään, ja sulkee lukkojen taakse. Sen jälkeen 
  sampo jauhaa rikkautta Pohjolan väelle. Tästä suuttuneina kalevalaiset hyökkäävät Pohjolaan ja 
  anastavat Sammon. Syntyneessä taistelussa sampo tuhoutuu. Pohjan Akka saa Sammon kannen. Sammon 
  sirpaleita ajelehtii kalevalaisten rantaan, ja ne saavat maanviljelyn aikaan.
 
 
  \begin{figure}[!hb]
    \caption{Gallen-Kallela: Sammon puolustus (1896)}
    \centering
    \includegraphics[width=0.66\textwidth]{Gallen-Kallela_-_Sammon_puolustus.png}
  \end{figure}  

  \begin{figure}[!hb]
    \caption{Pohjantähti naulana liittää yhteen kuvun muotoisen taivaankannen ja sampaan eli maailmanpylvään. E. N. Setälän (1932) mukaan.}
    \centering
    \includegraphics[width=0.66\textwidth]{Pohjannaula.jpg}
  \end{figure}  
  
  
  
\subsection{Kirjallisuutta}

  \begin{itemize}
    \item Siikala, Anna-Leena.: Itämerensuomalaisten mytologia. Hämeenlinna: SKS, 2012. 
    \item Haavio, Martti: Suomalainen mytologia. Porvoo Helsinki: WSOY, 1967. 
    \item Suurimpia sampoa käsitteleviä teoksia on Emil Nestor Setälän 650-sivuinen tutkimus Sammon arvoitus.
  \end{itemize}

  \paragraph{}

  "Nimi-instituution puuttumisesta siis johtunee, ettei vanhimmilla ihmishahmoisilla 
  jumaluuksillakaan ollut erityistä nimeä, vaan naisjumaluudet olivat akkoja, jotka erotettiin
  toisistaan asuinsijojensa, tehtäviensä tai aseittensa mukaan. Toisin sanoen: akka-jumaluudet 
  tuntuisivat siis olevan ajalta, jolloin uralilaiset/suomalais-ugrilaiset eivät vielä olleet 
  omaksuneet kantaeuroopasta lainasanaa nimi ja siihen liittyvää nimenantoa ja proprin-käyttöä. 
  Mutta akkojen aika saattoi senkin jälkeen jatkua yksinomaisena pitkään, kunnes kantasuomalaisten 
  taivaalle ilmaantui kantaeurooppalainen ukkosenjumala Ilmamoinen; hän tarvitsi ja sai nimen. 
  Akkojen aika kesti siten koko neljännen vuosituhannen ja suuren osan kolmattakin. Eivätkä 
  akkajumalat hävinneet senkään jälkeen: Pohjan akka ja akka manteren alainen elivät kalevalaisessa 
  perinteessä entiseen tapaan nimettöminä lähes nykyaikaan asti." - Unto Salo, "Tuoni, Pohjola, 
  Taivas - Arkeologian ja kalevalaisten runojen tuonelat (Kalevalaiset myytit ja 
  uskomukset III), s. 115
 