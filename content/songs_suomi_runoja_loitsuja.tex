\songcolumns{2} % two columns!

\beginsong{Aamulla}[tags={Aurinko 1}]
  \beginverse
    Terve kasvos näyttämästä,
    Päivä kulta koittamasta,
    Aurinko ylenemästä!
    Pääsit ylös alltoin alta
    Yli männistön ylenit,
    Nousit kullaisna käkenä,
    Hopeaisna kyyhkyläisnä
    Tasaiselle taivahalle,
    Elollesi entiselle,
    Matkoillesi muinaisille.
  \endverse
  \beginverse
    Nouse aina aikoinasi
    Perästä tämänki päivän,
    Tuo meille tuliaisiksi
    Anna täyttä terveyttä,
    Siirrä saama saatavihin,
    Pyytö päähän peukalomme,
    Onni onkemme nenähän;
    Käy kaaresi kaunihisti,
    Päätä päivän matkuesi,
    Pääse illalla ilohon!
  \endverse
\endsong


\beginsong{Tuulen sanat}[tags={tuuli 1}]
  \beginverse
    Terve kuu, terve päivä,
    Terve ilma, terve tuulet,
    Pohjois- ja etelätuuli,
    Itätuuli, länsituuli
    Lapintuuli, luoetuuli
    Suvituuli, lounaistuuli,
    Päivän nousu- ja laskutuuli
    Ja kaikki väliset tuulet!
    Lepy tuuli leppeäksi
    Lauhu ilma lauhkeaksi
    Kuu kirkas kumottamahan,
    Päivä lämmin paistamahan;
    Sivu tuulet tuulekohot,
    Sivu saakohot satehet,
    Kohti kuut kumottakohot,
    Kohti päivät paistakohot!
  \endverse
\endsong


\beginsong{Löylyn sanat: terve löyly}[tags={sauna 1}]
  \beginverse
    Terve löyly, terve lämmin
    terve henkäys kiukainen,
    kylpy lämpimäin kivisten,
    hiki vanhan Väinämöisen.
    Löylystä vihannan vihdan,
    tervan voimasta terveiden.
  \endverse
  \beginverse
    Löyly kiukahan kivestä,
    löyly saunan sammalista.
    Tervehyttä tekemähän,
    rauhoa rakentamahan,
    kipehille voitehiksi,
    pahoille parantehiksi.  
  \endverse 
\endsong


\beginsong{Löylyn sanat: tule löylyhyn}[tags={sauna 1}]
  \musicnote{Melodia: Kalevala-sävelmä tai esim. Hedingarna: Täss' on nainen}
  \beginverse
    Tule löylyhyn, Jumala, 
    Iso ilman, lämpimähän,
    Terveyttä tekemähän,
    Rauhoa rakentamahan
  \endverse
  \beginverse
    Lyötä maahan liika löyly
    Paha löyly pois lähetä
    Ettei polta tyttöjäsi
    Turmele tekemiäsi
  \endverse
  \beginverse
    Minkä vettä viskaelen
    Noille kuumille kivillen
    Se medeksi muuttukohon
    Simaksi sirahtakohon
  \endverse
  \beginverse
    Juoskohon joki metinen
    Simalampi laikkukohon
    Läpi kiukahan kivisen
    Läpi saunan sammalisen! 
  \endverse 
\endsong


\beginsong{Varjele vakainen luoja}[by={Kalevala: 43. runo}]
  \beginverse
    Anna Luoja, suo Jumala
    anna onni ollaksemme.
    Hyvin ain’ eleäksemme,
    Kunnialla kuollaksemme.
    Suloisessa Suomenmaassa
    Kaunihissa Karjalassa!
  \endverse
  \beginverse
    Varjele, vakainen Luoja
    Kaitse, kaunoinen Jumala,
    Ole puolla poikiesi,
    Aina lastesi apuna,
    Aina yöllisnä tukena,
    Päivällisnä vartiana.
  \endverse  
\endsong


\beginsong{Ihmisen synty}[]
  \beginverse
    Ihminen ihala ilme,
    Sukukunnan suuri luomus,
    Tehty on mullan kakkarasta,
    Mullan kaakusta rakettu,
    (Sille Herra hengen antoi,
    Luoja suustahan sukesi.)
  \endverse
\endsong


\beginsong{Karhun synty}[]
  \beginverse
    Otsoseni, ainoiseni,
    Mesikämmen kaunoiseni,
    Kyllä mä sukusi tieän,
    Miss' oot otso syntynynnä,
    Saatuna sinisaparo,
    Jalka kyntinen kyhätty:
    Tuoll' oot otso syntynynnä
    Ylähällä taivosessa,
    Kuun kukuilla, päällä päivän,
    Seitsentähtien selällä,
    Ilman impien tykönä,
    Luona luonnon tyttärien.
  \endverse
  \beginverse
    Tuli läikkyi taivahasta,
    Ilma kääntyi kehrän päällä,
    Otsoa suettaessa,
    Mesikkiä luotaessa.
    Sieltä maahan laskettihin
    Vierehen metisen viian,
    Hongattaren huolitella,
    Tuomettaren tuu'itella,
    Juurella nyrynärehen,
    Alla haavan haaralatvan,
    Metsän linnan liepehellä,
    Korven kultaisen kotona.
  \endverse
  \beginverse
    Siitä otso ristittihin,
    Karvahalli kastettihin,
    Metisellä mättähällä,
    Sarajoen salmen suulla,
    Pohjan tyttären sylissä.
    Siinä se valansa vannoi
    Pohjan eukon polven päässä,
    Essä julkisen Jumalan,
    Alla parran autuahan,
    Tehä ei syytä syyttömälle,
    Vikoa viattomalle,
    Käyä kesät kaunihisti,
    Soreasti sorkutella,
    Elellä ajat iloiset
    Suon selillä, maan navoilla,
    Kilokangasten perillä;
    Käyä kengättä kesällä,
    Sykysyllä syylingittä,
    Asua ajat pahemmat,
    Talvikylmät kyhmästellä,
    Tammisen tuvan sisässä,
    Havulinna liepehellä,
    Kengällä komean kuusen,
    Katajikon kainalossa.
  \endverse
\endsong


\beginsong{Kiven synty}[]
  \beginverse
    Ken kiven kiveksi tiesi,
    Kun oli otraisna jyvänä,
    Nousi maasta mansikkana,
    Puun juuresta puolukkana,
    Taikka häilyi hattarassa,
    Piili pilvien sisässä,
    Tuli maahan taivahasta,
    Putosi punakeränä,
    Kaaloi kakraisna kapuna,
    Vieri vehnäisnä mykynä,
    Läpi pilvipatsahien,
    Puhki kaarien punaisten,
    Hullu huutavi kiveksi,
    Maan munaksi mainitsevi.
  \endverse
\endsong


\beginsong{Noidan synty}[]
  \beginverse
    Kyllä tieän noian synnyn,
    Sekä alun arpojia:
    Tuoll' on noita syntynynnä,
    Tuolla alku arpojien,
    Pohjan penkeren takana,
    Lapin maassa laakeassa;
    Siell' on noita syntynynnä,
    Siellä arpoja sikesi,
    Hakoisella vuotehella,
    Kivisellä pääalalla.
  \endverse
\endsong


\beginsong{Puiden synty}[]
  \beginverse
    Sampsa poika Pellervoinen
    Kesät kentällä makasi
    Keskellä jyväketoa,
    Jyväparkan parmahalla;
    Otti kuusia jyviä,
    Seitsemiä siemeniä,
    Yhen nää'än nahkasehen,
    Koipehen kesäoravan,
    Läksi maita kylvämähän,
    Toukoja tihittämähän.
  \endverse
  \beginverse
    Kylvi maita kyyhätteli,
    Kylvi maita, kylvi soita,
    Kylvi auhtoja ahoja,
    Panettavi paasikoita.
    Kylvi kummut kuusikoiksi,
    Mäet kylvi männiköiksi,
    Kankahat kanervikoiksi,
    Notkont nuoriksi vesoiksi.
    Noromaille koivut kylvi,
    Lepät maille leyhkeille,
    Kylvi tuomet tuorehille,
    Pihlajat pyhille maille,
    Pajut maille paisuville,
    Raiat nurmien rajoille,
    Katajat karuille maille,
    Tammet virran vierimaille.
  \endverse
  \beginverse
    Läksi puut ylenemähän,
    Vesat nuoret nousemahan,
    Tuuliaisen tuu'ittaissa,
    Ahavaisen liekuttaissa,
    Kasvoi kuuset kukkalatvat,
    Lautui lakkapäät petäjät,
    Nousi koivuset noroilla,
    Lepät mailla leyhkeillä,
    Tuomet mailla tuorehilla,
    Pihlajat pyhillä mailla,
    Pajut mailla paisuvilla,
    Raiat mailla raikkahilla,
    Katajat karuilla mailla,
    Tammet virran vieremillä.
  \endverse
\endsong


\sclearpage
\beginsong{Höyhensaaret}[by={Eino Leino}]
  \beginverse
    Mitä siitä jos nuorna ma murrunkin
    tai taitun ma talvisäihin,
    moni murtunut onpi jo ennemmin
    ja jäätynyt elämän jäihin.
    Kuka vanhana vaappua tahtoiskaan?
    Ikinuori on nuoruus laulujen vaan
    ja kerkät lemmen ja keväimen,
    ilot sammuvi ihmisten.
  \endverse
  \beginverse
    Mitä siitä jos en minä sammukaan
    kuin rauhainen, riutuva liesi,
    jos sammun kuin sammuvat tähdet vaan
    ja vaipuvi merillä miesi.
    Kas, laulaja tähtiä laulelee
    ja hän meriä suuria seilailee
    ja hukkuvi hyrskyhyn, ennen kuin
    käy purjehin reivatuin.
  \endverse
  \beginverse
    Mitä siitä jos en minä saanutkaan,
    mitä toivoin ma elämältä,
    kun sain minä toivehet suuret vaan
    ja kaihojen kantelen hältä.
    Ja vaikka ma laps olen pieni vain,
    niin jumalten riemut ma juoda sain
    ja juoda ne täysin siemauksin--
    niin riemut kuin murheetkin.
  \endverse
  \beginverse
    Ja vaikka ma laps olen syksyn vaan
    ja istuja pitkän illan,
    sain soittaa ma kielillä kukkivan maan
    ja hieprukan hivuksilla.
    Niin mustat, niin mustat ne olivat;
    ja suurina surut ne tulivat,
    mut kaikuos riemu nyt kantelen
    vielä kertasi viimeisen!
  \endverse
  \beginverse
    Oi, kantelo pitkien kaihojen,
    sinä aarteeni omani, ainoo!
    Me kaksi, me kuulumme yhtehen,
    jos kuin mua kohtalot vainoo.
    Me kuljemme kylästä kylähän näin,
    ohi kylien koirien räkyttäväin,
    ja keskellä raition raakuuden
    sävel soipa on keväimen.
  \endverse
  \beginverse
    Me kaksi, me tulemme metsästä
    ja me metsien ilmaa tuomme,
    me laulamme nuoresta lemmestä
    ja lempemme kuvan me luomme,
    me luomme sen maailman tomusta niin
    kuin Luoja loi ihmisen Eedeniin
    ja korvesta kohoitamme me sen
    kun vaskisen käärmehen.
  \endverse
  \beginverse
    Te ystävät, joiden rinnassa kyyt
    yön-pitkät pistää ja kalvaa,
    te, joita jäytävi sydämen syyt
    ja elämä harmaja halvaa,
    oi, helise heille mun kantelein,
    oi, helise onnea haavehein
    ja unta silmihin unettomiin
    mun silmäni suljit sa niin.
    Kas, ylläpä mustien murheiden
    on kaunihit taivaankaaret
    ja kaukana keskellä aaltojen
    on haaveiden höyhensaaret
    ja ken sinne lapsosen kaarnalla käy,
    ei sille ne aavehet yölliset näy,
    vaan rinnoin hän uinuvi rauhaisin
    kuin äitinsä helmoihin.
  \endverse
  \beginverse
    Mitä siitä jos valhetta onkin ne vaan
    ja kestä ei päivän terää!
    Me uinumme siksi kuin valveutaan
    ja vaivat ne jällehen herää.
    Moni nukkui nuorihin toiveisiin
    ja heräsi hapsihin hopeisiin;
    hän katsahti ympäri kummissaan
    ja -- uinahti uudestaan.
    Miks ihmiset tahtoa, taistella
    ja koittaa korkealle?
    Me olemme kaikki vain lapsia
    ja murrumme murheen alle.
    Miks emme me kaikki vois uinahtaa
    ja hyviä olla ja hymytä vaan
    ja katsoa katsehin kirkkahin
    vain sielumme syvyyksiin?
  \endverse
  \beginverse
    Oi, unessa murheet ne unhottuu
    ja rauhaton rauhan saapi,
    oi, unessa vankikin vapautuu,
    sen kahlehet katkeaapi,
    ja köyhä on rikas kuin kuningas maan
    ja kevyt on valtikka kuninkaan
    ja kaikki, kaikki on veljiä vaan--
    oi, onnea unelmain!
  \endverse
  \beginverse
    Oi, onnea uinua uudelleen
    ne lapsuen päivät lauhat
    ja itkeä jällehen yksikseen
    ne riemut ja rinnan rauhat;
    taas uskoa, että on lapsi vaan
    ja että voi alkaa uudestaan
    ja uskoa uusihin toiveisiin
    sekä vanhoihin ystäviin!
  \endverse
  \beginverse
    Taas uskoa riemuhun, keväimeen
    ja lippuhun pilvien linnan
    ja uskoa lempehen puhtaaseen
    taas kahden puhtahan rinnan,
    taas uskoa itsensä rikkahaks
    ja maailman suureks ja avaraks--
    voi, kuinka se sentään on ihanaa,
    kun sen nuorena uskoa saa!
  \endverse
  \beginverse
    Voi, kuinka se sille on ihanaa,
    joka kaiken sen kadotti kerran,
    joka häkistä katseli maailmaa
    ja näki vain vaaksan verran,
    joka etsi kauneutta, elämää,
    ja näki vain markkinavilinää,
    ja näki räyhäävän raakuuden, tyhmyyden--
    niit' aikoja unhota en.
  \endverse
  \beginverse
    Kun muistelen, kuinka ma kerjännyt
    olen koirana lempeä täällä,
    miten rikasten portailla pyydellyt
    olen tuiskulla, tuulissäällä,
    vain lämpöä hiukkasen, hiukkasen vain
    ja kun minä muistelen, mitä mä sain
    ja mitä mä nielin ja vaikenin
    ja mitä mä ajattelin!
  \endverse
  \beginverse
    Miten olen minä kulkenut, uskonut,
    ett'eivät ne unhoitukaan!
    Ja sentään ne olen minä unhoittanut
    kuin unhoittaa voi kukaan.
    Ja sentään se nousi, niin kohtalot kaas,
    ja sentään ma seppona seison taas
    ja taivahan kansia taon ja lyön--
    oi, onnea tähtisen yön!
  \endverse
  \beginverse
    Ne saapuvat, saapuvat uudestaan
    mun onneni orhit valkeet,
    ne painavat vanhalla voimallaan
    mun rintani jättipalkeet.
    Ja kirkas on taivas ja kukkii maa
    ja säkenet suustani suitsuaa
    ja ääneni on kuni ukkosen--
    oi, onnea unelmien!
  \endverse
  \beginverse
    Mitä siitä jos haaveeni verkot vaan
    on verkkoja hämähäkin!
    Mitä siitä jos omieni viittova vaan
    on laulua laineiden näkin!
    Moni nukkui nuorihin toiveisiin
    ja heräsi hapsihin hopeisiin
    tai herännyt täällä ei milloinkaan.
    Missä? Milloin? Helmassa maan.
    Minä tahdon riemuja keväimen
    ja onnesta osani kerta!
    Olen imenyt rintoja totuuden,
    mut niistä vaan tuli verta.
    Siis, tulkaa te utaret unelmien,
    minä vaivun riemunne rinnoillen
    ja uskon päivähän, aurinkohon.
    Unen maito on loppumaton.
  \endverse
  \beginverse
    Oi, kauniisti mulle te kaartukaa,
    mun syömeni sateenkaaret!
    Mua hiljaa, hiljaa tuudittakaa,
    te haaveiden höyhensaaret!
    Mua katsokaa: olen lapsi vaan,
    olen riisunut päältäni riemut maan
    ja pyytehet kullan ja kunnian.
    Uni onni on laulajan.
  \endverse
  \beginverse
    Minä tahdon vain uinua yksikseen.
    En tahtois vielä mä kuolla.
    Mut kuulkaa, jo äitini huhuilee
    Tuonen aaltojen tuolla puolla.
    Oi, odota hetkinen, äityein!
    En viel' olis valmis ma matkallein,
    mun syömeni on niin syyllinen.
    Suo että mä pesen sen.
  \endverse
  \beginverse
    Suo että mä ensin huuhdon vaan
    nämä synkeät, huonot aatteet,
    suo että mä päälleni ensin saan
    ne puhtahat, valkeat vaatteet,
    jotk' ompeli onneni impynen,
    hän, hämärän impeni ihmeellinen,
    min kuvaa kannan ma sydämessäin
    siit' asti kuin hänet mä näin.
  \endverse
  \beginverse
    Me tulemme, äitini armahain!
    Oi katso, meitä on kaksi!
    Oi katso, mik' on mulla rinnassain!
    Niin oisitko rikkahaksi
    sinä uskonut koskaan kuopustas?
    Ja katso, me pyydämme siunaustas,
    sun poikasi synkeä, syyllinen,
    ja mun impeni puhtoinen.
  \endverse
  \beginverse
    Katso, kuin hän on kaunis ja valkoinen
    ja muistuttaa niin sua!
    Hän on niin hellä ja herttainen,
    vaikk'ei hän lemmi mua,
    Elä kysele hältä, miks tänne mun toi,
    mut usko, se niin oli parhain, oi!
    Ja usko, nyt ett' olen onnellinen
    kuin aikoina lapsuuden.
  \endverse
  \beginverse
    Elä kysele multa sa laaksoista maan!
    Ei olleet ne luodut mulle.
    Mut jos sinun silmäsi tutkii vaan,
    voin laulaa ma laulun sulle
    kuin lauloin ma lapsen aikoihin--
    kas, lauluna sujuu se paremmin
    ja kyynelet kuuluvat kantelehen.
    Niitä muuten ma ilmoita en.
  \endverse
\endsong


\beginsong{Hymyilevä Apollo}[by={Eino Leino}]
  \beginverse
    Näin lauloin ma kuolleelle äidillein
    ja äiti mun ymmärsi heti.
    Hän painoi suukkosen otsallein
    ja sylihinsä mun veti:
    »Ken uskovi toteen, ken unelmaan, --
    sama se, kun täysin sa uskot vaan!
    Sun uskos se juuri on totuutes.
    Usko poikani unehes!»
  \endverse
  \beginverse
    Miten mielelläin, niin mielelläin
    hänen luoksensa jäänyt oisin
    luo Tuonen virtojen viileäin,
    mut kohtalot päätti toisin.
    Vielä viimeisen kerran viittasi hän
    kuin hän vain viitata tiesi.
    Taas seisoin ma rannalla elämän,
    mut nyt olin toinen miesi.
  \endverse
  \beginverse
    Nyt tulkaa te murheet ja vastukset,
    niin saatte te vasten suuta!
    Nyt raudasta mulla on jänteret,
    nyt luuni on yhtä luuta.
    Kas, Apolloa, joka hymyilee,
    sitä voita ei Olympo jumalineen,
    ei Tartarus, Pluto, ei Poseidon.
    Hymyn voima on voittamaton.
  \endverse
  \beginverse
    Meri pauhaa, ukkonen jylisee,
    Apollo saapuu ja hymyy.
    Ja katso! Ukkonen vaikenee,
    tuul' laantuu, lainehet lymyy.
    Hän hymyllä maailman hallitsee,
    hän laululla valtansa vallitsee,
    ja laulunsa korkea, lempeä on.
    Lemmen voima on voittamaton.
  \endverse
  \beginverse
    Kun aavehet mieltäsi ahdistaa,
    niin lemmi! -- ja aavehet haihtuu.
    Kun murheet sun sielusi mustaks saa,
    niin lemmi! -- ja iloks ne vaihtuu.
    Ja jos sua häpäisee vihamies,
    niin lemmellä katko sen kaunan ies
    ja katso, hän kasvonsa kääntää pois
    kuin itse hän hävennyt ois.
  \endverse
  \beginverse
    Kuka taitavi lempeä vastustaa?
    Ketä voita ei lemmen kieli?
    Sitä kuulee taivas ja kuulee maa
    ja ilma ja ihmismieli.
    Kas, povet se aukovi paatuneet,
    se rungot nostavi maatuneet
    ja kutovi lehtihin, kukkasiin
    ja uusihin unelmiin.
  \endverse
  \beginverse
    Ei paha ole kenkään ihminen,
    vaan toinen on heikompi toista.
    Paljon hyvää on rinnassa jokaisen,
    vaikk' ei aina esille loista.
    Kas, hymy jo puoli on hyvettä
    ja itkeä ei voi ilkeä;
    miss' ihmiset tuntevat tuntehin,
  \endverse
\endsong


\beginsong{Soutaja}[by={Unto Kupiainen}]
  \beginverse
    Vieras on virta ja vieras on vene, 
    eivät ne unelmies uomia mene. 
    Ilta on ihmisessä ja aamu on outo; 
    illasta aamuun on ihmisen souto. 
    Illasta aamuun on yöllistä matkaa; 
    jos jaksat uskoa, jaksat jatkaa. 
    Taapäin tuijotat, soudat eteen 
    outoa venettä outoon veteen. 
  \endverse
\endsong

\songcolumns{1} % back to one column
