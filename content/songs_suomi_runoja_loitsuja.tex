% Finnish spells and poems
%
\beginsong{Maailmansynty\-runo}[by={Kalevala: 1. runo}]
  \beginverse
    Mieleni minun tekevi,
    aivoni ajattelevi
    lähteäni laulamahan,
    saa'ani sanelemahan,
    sukuvirttä suoltamahan,
    lajivirttä laulamahan.
    Sanat suussani sulavat,
    puhe'et putoelevat,
    kielelleni kerkiävät,
    hampahilleni hajoovat.
    Veli kulta, veikkoseni,
    kaunis kasvinkumppalini!
    Lähe nyt kanssa laulamahan,
    saa kera sanelemahan
    yhtehen yhyttyämme,
    kahta'alta käytyämme!
    Harvoin yhtehen yhymme,
    saamme toinen toisihimme
    näillä raukoilla rajoilla,
    poloisilla Pohjan mailla.
  \endverse
  \beginverse
    Lyökämme käsi kätehen,
    sormet sormien lomahan,
    lauloaksemme hyviä,
    parahia pannaksemme,
    kuulla noien kultaisien,
    tietä mielitehtoisien,
    nuorisossa nousevassa,
    kansassa kasuavassa:
    noita saamia sanoja,
    virsiä virittämiä
    vyöltä vanhan Väinämöisen,
    alta ahjon Ilmarisen,
    päästä kalvan Kaukomielen,
    Joukahaisen jousen tiestä,
    Pohjan peltojen periltä,
    Kalevalan kankahilta.
    Niit' ennen isoni lauloi
    kirvesvartta vuollessansa;
    niitä äitini opetti
    väätessänsä värttinätä,
    minun lasna lattialla
    eessä polven pyöriessä,
    maitopartana pahaisna,
    piimäsuuna pikkaraisna.
    Sampo ei puuttunut sanoja
    eikä Louhi luottehia:
    vanheni sanoihin sampo,
    katoi Louhi luottehisin,
    virsihin Vipunen kuoli,
    Lemminkäinen leikkilöihin.
  \endverse
  \beginverse
    Viel' on muitaki sanoja,
    ongelmoita oppimia:
    tieohesta tempomia,
    kanervoista katkomia,
    risukoista riipomia,
    vesoista vetelemiä,
    päästä heinän hieromia,
    raitiolta ratkomia,
    paimenessa käyessäni,
    lasna karjanlaitumilla,
  \endverse
  \beginverse
    metisillä mättähillä,
    kultaisilla kunnahilla,
    mustan Muurikin jälessä,
    Kimmon kirjavan keralla.
    Vilu mulle virttä virkkoi,
    sae saatteli runoja.
    Virttä toista tuulet toivat,
    meren aaltoset ajoivat.
    Linnut liitteli sanoja,
    puien latvat lausehia.
    Ne minä kerälle käärin,
    sovittelin sommelolle.
    Kerän pistin kelkkahani,
    sommelon rekoseheni;
    ve'in kelkalla kotihin,
    rekosella riihen luoksi;
    panin aitan parven päähän
    vaskisehen vakkasehen.
    Viikon on virteni vilussa,
    kauan kaihossa sijaisnut.
  \endverse
  \beginverse
    Veänkö vilusta virret,
    lapan laulut pakkasesta,
    tuon tupahan vakkaseni,
    rasian rahin nenähän,
    alle kuulun kurkihirren,
    alle kaunihin katoksen,
    aukaisen sanaisen arkun,
    virsilippahan viritän,
    kerittelen pään kerältä,
    suorin solmun sommelolta?
  \endverse
  \beginverse
    Niin laulan hyvänki virren,
    kaunihinki kalkuttelen
    ruoalta rukihiselta,
    oluelta ohraiselta.
    Kun ei tuotane olutta,
    tarittane taarivettä,
    laulan suulta laihemmalta,
    vetoselta vierettelen
    tämän iltamme iloksi,
    päivän kuulun kunniaksi,
    vaiko huomenen huviksi,
    uuen aamun alkeheksi.
    * * *
  \endverse
  \beginverse
    Noin kuulin saneltavaksi,
    tiesin virttä tehtäväksi:
    yksin meillä yöt tulevat,
    yksin päivät valkeavat;
    yksin syntyi Väinämöinen,
    ilmestyi ikirunoja
    kapehesta kantajasta,
    Ilmattaresta emosta.
  \endverse
  \beginverse
    Olipa impi, ilman tyttö,
    kave luonnotar korea.
    Piti viikoista pyhyyttä,
    iän kaiken impeyttä
    ilman pitkillä pihoilla,
    tasaisilla tanterilla.
    Ikävystyi aikojansa,
    ouostui elämätänsä,
    aina yksin ollessansa,
    impenä eläessänsä
  \endverse
  \beginverse
    ilman pitkillä pihoilla,
    avaroilla autioilla.
    Jop' on astuiksen alemma,
    laskeusi lainehille,
    meren selvälle selälle,
    ulapalle aukealle.
    Tuli suuri tuulen puuska,
    iästä vihainen ilma;
    meren kuohuille kohotti,
    lainehille laikahutti.
  \endverse
  \beginverse
    Tuuli neittä tuuitteli,
    aalto impeä ajeli
    ympäri selän sinisen,
    lakkipäien lainehien:
    tuuli tuuli kohtuiseksi,
    meri paksuksi panevi.
    Kantoi kohtua kovoa,
    vatsantäyttä vaikeata
    vuotta seitsemän satoa,
    yheksän yrön ikeä;
  \endverse
  \beginverse
    eikä synny syntyminen,
    luovu luomatoin sikiö.
    Vieri impi veen emona.
    Uipi iät, uipi lännet,
    uipi luotehet, etelät,
    uipi kaikki ilman rannat
    tuskissa tulisen synnyn,
    vatsanvaivoissa kovissa;
    eikä synny syntyminen,
    luovu luomatoin sikiö.
  \endverse
  \beginverse
    Itkeä hyryttelevi;
    sanan virkkoi, noin nimesi:
    "Voi poloinen, päiviäni,
    lapsi kurja, kulkuani!
    Jo olen joutunut johonki:
    iäkseni ilman alle,
    tuulen tuuiteltavaksi,
    aaltojen ajeltavaksi
    näillä väljillä vesillä,
    lake'illa lainehilla!
  \endverse
  \beginverse
    "Parempi olisi ollut
    ilman impenä eleä,
    kuin on nyt tätä nykyä
    vierähellä veen emona:
    vilu tääll' on ollakseni,
    vaiva värjätelläkseni,
    aalloissa asuakseni,
    veessä vierielläkseni.
    "Oi Ukko, ylijumala,
    ilman kaiken kannattaja!
  \endverse
  \beginverse
    Tule tänne tarvittaissa,
    käy tänne kutsuttaessa!
    Päästä piika pintehestä,
    vaimo vatsanvääntehestä!
    Käy pian, välehen jou'u,
    välehemmin tarvitahan!"
  \endverse
  \beginverse
    Kului aikoa vähäisen,
    pirahteli pikkaraisen.
    Tuli sotka, suora lintu;
    lenteä lekuttelevi
  \endverse
  \beginverse
    etsien pesän sijoa,
    asuinmaata arvaellen.
    Lenti iät, lenti lännet,
    lenti luotehet, etelät.
    Ei löyä tiloa tuota,
    paikkoa pahintakana,
    kuhun laatisi pesänsä,
    ottaisi olosijansa.
  \endverse
  \beginverse
    Liitelevi, laatelevi;
    arvelee, ajattelevi:
  \endverse
  \beginverse
    "Teenkö tuulehen tupani,
    aalloillen asuinsijani?
    Tuuli kaatavi tupasen,
    aalto vie asuinsijani."
    Niin silloin ve'en emonen,
    veen emonen, ilman impi,
    nosti polvea merestä,
    lapaluuta lainehesta
    sotkalle pesän sijaksi,
    asuinmaaksi armahaksi.
  \endverse
  \beginverse
    Tuo sotka, sorea lintu,
    liiteleikse, laateleikse.
    Keksi polven veen emosen
    sinerväisellä selällä;
    luuli heinämättähäksi,
    tuoreheksi turpeheksi.
    Lentelevi, liitelevi,
    päähän polven laskeuvi.
    Siihen laativi pesänsä,
    muni kultaiset munansa:
  \endverse
  \beginverse
    kuusi kultaista munoa,
    rautamunan seitsemännen.
  \endverse
  \beginverse
    Alkoi hautoa munia,
    päätä polven lämmitellä.
    Hautoi päivän, hautoi toisen,
    hautoi kohta kolmannenki.
    Jopa tuosta veen emonen,
    veen emonen, ilman impi,
    tuntevi tulistuvaksi,
    hipiänsä hiiltyväksi;
  \endverse
  \beginverse
    luuli polvensa palavan,
    kaikki suonensa sulavan.
    Vavahutti polveansa,
    järkytti jäseniänsä:
    munat vierähti vetehen,
    meren aaltohon ajaikse;
    karskahti munat muruiksi,
    katkieli kappaleiksi.
  \endverse
  \beginverse
    Ei munat mutahan joua,
    siepalehet veen sekahan.
  \endverse
  \beginverse
    Muuttuivat murut hyviksi,
    kappalehet kaunoisiksi:
    munasen alainen puoli
    alaiseksi maaemäksi,
    munasen yläinen puoli
    yläiseksi taivahaksi;
    yläpuoli ruskeaista
    päivöseksi paistamahan,
    yläpuoli valkeaista,
    se kuuksi kumottamahan;
    mi munassa kirjavaista,
    ne tähiksi taivahalle,
    mi munassa mustukaista,
    nepä ilman pilvilöiksi.
  \endverse
  \beginverse
    Ajat eellehen menevät,
    vuoet tuota tuonnemmaksi
    uuen päivän paistaessa,
    uuen kuun kumottaessa.
    Aina uipi veen emonen,
    veen emonen, ilman impi,
  \endverse
  \beginverse
    noilla vienoilla vesillä,
    utuisilla lainehilla,
    eessänsä vesi vetelä,
    takanansa taivas selvä.
    Jo vuonna yheksäntenä,
    kymmenentenä kesänä
    nosti päätänsä merestä,
    kohottavi kokkoansa.
    Alkoi luoa luomiansa,
    saautella saamiansa
  \endverse
  \beginverse
    selvällä meren selällä,
    ulapalla aukealla.
    Kussa kättä käännähytti,
    siihen niemet siivoeli;
    kussa pohjasi jalalla,
    kalahauat kaivaeli;
    kussa ilman kuplistihe,
    siihen syöverit syventi.
  \endverse
  \beginverse
    Kylin maahan kääntelihe:
    siihen sai sileät rannat;
  \endverse
  \beginverse
    jaloin maahan kääntelihe:
    siihen loi lohiapajat;
    pä'in päätyi maata vasten:
    siihen laitteli lahelmat.
    Ui siitä ulomma maasta,
    seisattelihe selälle:
    luopi luotoja merehen,
    kasvatti salakaria
    laivan laskemasijaksi,
    merimiesten pään menoksi.
  \endverse
  \beginverse
    Jo oli saaret siivottuna,
    luotu luotoset merehen,
    ilman pielet pistettynä,
    maat ja manteret sanottu,
    kirjattu kivihin kirjat,
    veetty viivat kallioihin.
    Viel' ei synny Väinämöinen,
    ilmau ikirunoja.
    Vaka vanha Väinämöinen
    kulki äitinsä kohussa
  \endverse
  \beginverse
    kolmekymmentä keseä,
    yhen verran talviaki,
    noilla vienoilla vesillä,
    utuisilla lainehilla.
    Arvelee, ajattelevi,
    miten olla, kuin eleä
    pimeässä piilossansa,
    asunnossa ahtahassa,
    kuss' ei konsa kuuta nähnyt
    eikä päiveä havainnut.
  \endverse
  \beginverse
    Sanovi sanalla tuolla,
    lausui tuolla lausehella:
    "Kuu, keritä, päivyt, päästä,
    otava, yhä opeta
    miestä ouoilta ovilta,
    veräjiltä vierahilta,
    näiltä pieniltä pesiltä,
    asunnoilta ahtahilta!
    Saata maalle matkamiestä,
    ilmoillen inehmon lasta,
    kuuta taivon katsomahan,
    päiveä ihoamahan,
    otavaista oppimahan,
    tähtiä tähyämähän!"
    Kun ei kuu kerittänynnä
    eikä päivyt päästänynnä,
    ouosteli aikojansa,
    tuskastui elämätänsä:
    liikahutti linnan portin
    sormella nimettömällä,
  \endverse
  \beginverse
    lukon luisen luikahutti
    vasemmalla varpahalla;
    tuli kynsin kynnykseltä,
    polvin porstuan ovelta.
    Siitä suistui suin merehen,
    käsin kääntyi lainehesen;
    jääpi mies meren varahan,
    uros aaltojen sekahan.
  \endverse
  \beginverse
    Virui siellä viisi vuotta,
    sekä viisi jotta kuusi,
  \endverse
  \beginverse
    vuotta seitsemän, kaheksan.
    Seisottui selälle viimein,
    niemelle nimettömälle,
    manterelle puuttomalle.
    Polvin maasta ponnistihe,
    käsivarsin käännältihe.
    Nousi kuuta katsomahan,
    päiveä ihoamahan,
    otavaista oppimahan,
    tähtiä tähyämähän.
  \endverse
  \beginverse
    Se oli synty Väinämöisen,
    rotu rohkean runojan
    kapehesta kantajasta,
    Ilmattaresta emosta.
  \endverse
\endsong


\beginsong{Aamulla}[tags={aamu 1, Aurinko 1}]
  \beginverse
    Terve kasvos näyttämästä,
    Päivä kulta koittamasta,
    Aurinko ylenemästä!
    Pääsit ylös altoin alta
    Yli männistön ylenit,
    Nousit kullaisna käkenä,
    Hopeaisna kyyhkyläisnä
    Tasaiselle taivahalle,
    Elollesi entiselle,
    Matkoillesi muinaisille.
  \endverse
  \beginverse
    Nouse aina aikoinasi
    Perästä tämänki päivän,
    Tuo meille tuliaisiksi
    Anna täyttä terveyttä,
    Siirrä saama saatavihin,
    Pyytö päähän peukalomme,
    Onni onkemme nenähän;
    Käy kaaresi kaunihisti,
    Päätä päivän matkuesi,
    Pääse illalla ilohon!
  \endverse
\endsong


\beginsong{Tuulen sanat}[tags={tuuli 1}]
  \beginverse
    Terve kuu, terve päivä,
    Terve ilma, terve tuulet,
    Pohjois- ja etelätuuli,
    Itätuuli, länsituuli
    Lapintuuli, luoetuuli
    Suvituuli, lounaistuuli,
    Päivän nousu- ja laskutuuli
    Ja kaikki väliset tuulet!
    Lepy tuuli leppeäksi
    Lauhu ilma lauhkeaksi
    Kuu kirkas kumottamahan,
    Päivä lämmin paistamahan;
    Sivu tuulet tuulekohot,
    Sivu saakohot satehet,
    Kohti kuut kumottakohot,
    Kohti päivät paistakohot!
  \endverse
\endsong


\beginsong{Löylyn sanat: terve löyly}[tags={sauna 1}]
  \beginverse
    Terve löyly, terve lämmin
    terve henkäys kiukainen,
    kylpy lämpimäin kivisten,
    hiki vanhan Väinämöisen.
    Löylystä vihannan vihdan,
    tervan voimasta terveiden.
  \endverse
  \beginverse
    Löyly kiukahan kivestä,
    löyly saunan sammalista.
    Tervehyttä tekemähän,
    rauhoa rakentamahan,
    kipehille voitehiksi,
    pahoille parantehiksi.
  \endverse
\endsong


\beginsong{Löylyn sanat: tule löylyhyn}[tags={sauna 1}]
  \musicnote{Melodia: Kalevala-sävelmä tai esim. Hedingarna: Täss' on nainen}
  \beginverse
    Tule löylyhyn, Jumala,
    Iso ilman, lämpimähän,
    Terveyttä tekemähän,
    Rauhoa rakentamahan
  \endverse
  \beginverse
    Lyötä maahan liika löyly
    Paha löyly pois lähetä
    Ettei polta tyttöjäsi
    Turmele tekemiäsi
  \endverse
  \beginverse
    Minkä vettä viskaelen
    Noille kuumille kivillen
    Se medeksi muuttukohon
    Simaksi sirahtakohon
  \endverse
  \beginverse
    Juoskohon joki metinen
    Simalampi laikkukohon
    Läpi kiukahan kivisen
    Läpi saunan sammalisen!
  \endverse
\endsong


\beginsong{Ihmisen synty}[]
  \beginverse
    Ihminen ihala ilme,
    Sukukunnan suuri luomus,
    Tehty on mullan kakkarasta,
    Mullan kaakusta rakettu,
    (Sille Herra hengen antoi,
    Luoja suustahan sukesi.)
  \endverse
\endsong


\beginsong{Karhun synty}[]
  \beginverse
    Otsoseni, ainoiseni,
    Mesikämmen kaunoiseni,
    Kyllä mä sukusi tieän,
    Miss' oot otso syntynynnä,
    Saatuna sinisaparo,
    Jalka kyntinen kyhätty:
    Tuoll' oot otso syntynynnä
    Ylähällä taivosessa,
    Kuun kukuilla, päällä päivän,
    Seitsentähtien selällä,
    Ilman impien tykönä,
    Luona luonnon tyttärien.
  \endverse
  \beginverse
    Tuli läikkyi taivahasta,
    Ilma kääntyi kehrän päällä,
    Otsoa suettaessa,
    Mesikkiä luotaessa.
    Sieltä maahan laskettihin
    Vierehen metisen viian,
    Hongattaren huolitella,
    Tuomettaren tuu'itella,
    Juurella nyrynärehen,
    Alla haavan haaralatvan,
    Metsän linnan liepehellä,
    Korven kultaisen kotona.
  \endverse
  \beginverse
    Siitä otso ristittihin,
    Karvahalli kastettihin,
    Metisellä mättähällä,
    Sarajoen salmen suulla,
    Pohjan tyttären sylissä.
    Siinä se valansa vannoi
    Pohjan eukon polven päässä,
    Essä julkisen Jumalan,
    Alla parran autuahan,
    Tehä ei syytä syyttömälle,
    Vikoa viattomalle,
    Käyä kesät kaunihisti,
    Soreasti sorkutella,
    Elellä ajat iloiset
    Suon selillä, maan navoilla,
    Kilokangasten perillä;
    Käyä kengättä kesällä,
    Sykysyllä syylingittä,
    Asua ajat pahemmat,
    Talvikylmät kyhmästellä,
    Tammisen tuvan sisässä,
    Havulinna liepehellä,
    Kengällä komean kuusen,
    Katajikon kainalossa.
  \endverse
\endsong


\beginsong{Kiven synty}[]
  \beginverse
    Ken kiven kiveksi tiesi,
    Kun oli otraisna jyvänä,
    Nousi maasta mansikkana,
    Puun juuresta puolukkana,
    Taikka häilyi hattarassa,
    Piili pilvien sisässä,
    Tuli maahan taivahasta,
    Putosi punakeränä,
    Kaaloi kakraisna kapuna,
    Vieri vehnäisnä mykynä,
    Läpi pilvipatsahien,
    Puhki kaarien punaisten,
    Hullu huutavi kiveksi,
    Maan munaksi mainitsevi.
  \endverse
\endsong


\beginsong{Noidan synty}[]
  \beginverse
    Kyllä tieän noian synnyn,
    Sekä alun arpojia:
    Tuoll' on noita syntynynnä,
    Tuolla alku arpojien,
    Pohjan penkeren takana,
    Lapin maassa laakeassa;
    Siell' on noita syntynynnä,
    Siellä arpoja sikesi,
    Hakoisella vuotehella,
    Kivisellä pääalalla.
  \endverse
\endsong


\beginsong{Puiden synty}[]
  \beginverse
    Sampsa poika Pellervoinen
    Kesät kentällä makasi
    Keskellä jyväketoa,
    Jyväparkan parmahalla;
    Otti kuusia jyviä,
    Seitsemiä siemeniä,
    Yhen nää'än nahkasehen,
    Koipehen kesäoravan,
    Läksi maita kylvämähän,
    Toukoja tihittämähän.
  \endverse
  \beginverse
    Kylvi maita kyyhätteli,
    Kylvi maita, kylvi soita,
    Kylvi auhtoja ahoja,
    Panettavi paasikoita.
    Kylvi kummut kuusikoiksi,
    Mäet kylvi männiköiksi,
    Kankahat kanervikoiksi,
    Notkont nuoriksi vesoiksi.
    Noromaille koivut kylvi,
    Lepät maille leyhkeille,
    Kylvi tuomet tuorehille,
    Pihlajat pyhille maille,
    Pajut maille paisuville,
    Raiat nurmien rajoille,
    Katajat karuille maille,
    Tammet virran vierimaille.
  \endverse
  \beginverse
    Läksi puut ylenemähän,
    Vesat nuoret nousemahan,
    Tuuliaisen tuu'ittaissa,
    Ahavaisen liekuttaissa,
    Kasvoi kuuset kukkalatvat,
    Lautui lakkapäät petäjät,
    Nousi koivuset noroilla,
    Lepät mailla leyhkeillä,
    Tuomet mailla tuorehilla,
    Pihlajat pyhillä mailla,
    Pajut mailla paisuvilla,
    Raiat mailla raikkahilla,
    Katajat karuilla mailla,
    Tammet virran vieremillä.
  \endverse
\endsong


\beginsong{Tammen synty}[]
  \beginverse
    Oli ennen neljä neittä,
    Kolme kuulua tytärtä,
    Sininurmen niitännässä,
    Korttehen kokoannassa,
    Nenässä utuisen niemen,
    Päässä saaren terhenisen.
    Niitit päivän, niitit toisen,
    Niitit kohta kolmannenki,
    Minkä niitit, sen haravoit,
    Kaikki karhille vetelit,
    Laitit heinät lallosille,
    Sataisille saprasille,
    Siitä suovahan kokosit,
    Saatoit sankapieleksihin.
  \endverse
  \beginverse
    Jo oli nurmi niitettynä,
    Heinät luotu pielin pystyin,
    Tuli Turjan lappalainen,
    Nimeltä tulinen Tursas,
    Tunki heinäset tulehen,
    Paiskasi panun väkehen.
  \endverse
  \beginverse
    Tuli tuhkia vähäinen,
    Kypeniä pikkarainen,
    Tytöt tuossa arvelevat,
    Neiet neuvoa pitävät,
    Kunne tuhkat koottanehen,
    Poron pohjat pantanehen:
    "Noistapa puuttuvi poroa,
    Lipeätä liuvahtavi,
    Pestä päätä Päivän poian,
    Silmiä hyvän urohon".
  \endverse
  \beginverse
    Tuli tuuli tunturista,
    Kova ilma koillisesta,
    Tuonne tuuli tuhkat kantoi,
    Porot koillinen kokosi,
    Nenästä utuisen niemen,
    Päästä saaren terhenisen,
    Korvalle tulisen kosken,
    Pyhän virran vieremille.
    Tuuli tuopi tammen terhon,
    Kantoi maalta kaukaiselta
    Korvalle tulisen kosken,
    Pyhän virran vieremille,
    Heitti paikalle hyvälle,
    Maan lihavan liepehelle.
    Nousi tuosta nuori taimi,
    Vesa verraton vetihe,
    Siitä kasvoi kaunis tammi,
    Yleni rutimon raita,
    Latva täytti taivahille,
    Oksat ilmoille olotti.
  \endverse
\endsong


\beginsong{Tulen synty}[tags={tuli 1}]
  \beginverse
    Ei tuli syviltä synny,
    Eikä kasva karkealta,
    Tuli syntyi taivosessa,
    Seitsentähtyen selällä,
    Siell' on tulta tuu'iteltu,
    Valkeaista vaapoteltu,
    Kultaisessa kursikossa,
    Kultakunnahan kukulla.
  \endverse
  \beginverse
    Kasi kaunis, neito nuori,
    Tulityttö taivahinen,
    Tuopa tulta tuu'ittavi,
    Vaapottavi valkeata,
    Tuolla taivahan navoilla,
    Yllä taivahan yheksän,
    Hopeaiset nuorat notkui,
    Koukku kultainen kulisi,
    Neien tulta tuu'ittaissa,
    Vaapottaissa valkeaista.
  \endverse
  \beginverse
    Putosi tuli punainen,
    Kirposi kipuna yksi,
    Kultaisesta kursikosta,
    Hopeaisesta sulusta,
    Ilmalta yheksänneltä,
    Kaheksannen kannen päältä,
    Läpi taivahan tasaisen,
    Halki tuon ihalan ilman,
    Läpi ramppalan ovista,
    Läpi lapsen vuotehesta;
    Paloi polvet poikuelta,
    Paloi paarmahat emolta.
  \endverse
  \beginverse
    Se lapsi meni manalle,
    Katopoika tuonelahan,
    Kun oli luotu kuolemahan,
    Katsottu katoamahan,
    Tuskissa tulen punaisen,
    Angervoisen ailuissa;
    Märäten meni manalle,
    Torkahellen tuonelahan,
    Tuonen tyttöjen torua,
    Manan lasten lausuella.
  \endverse
  \beginverse
    Emopa ei manalle mennyt;
    Akka oli viisas villikerta,
    Se tunsi tulen lumoa,
    Valkeaisen vaivutella,
    Läpi pienen neulan silmän,
    Halki kirvehen hamaran,
    Puhki kuuman tuuran putken,
    Kerivi tulen kerälle,
    Suorittavi sommelolle,
    Kierähyttävi keräsen
    Pitkin pellon pientaretta,
    Läpi maan, läpi manuen,
    Työnti Tuonelan jokehen,
    Manalan syväntehesen.
  \endverse
\endsong

%% % Tulen synty, toinen versio
%%\beginsong{Tulen synty B}[tags={tuli 1}]
%%  \beginverse
%%    Tulta iski ilman Ukko,
%%    Valahutti valkeata,
%%    Miekalla tuliterällä
%%    Säilällä säkenevällä,
%%    Ylisessä taivosessa,
%%    Tähtitarhojen takana.
%%  \endverse
%%  \beginverse
%%    Saipa tulta iskemällä,
%%    Kätkevi tulikipunan
%%    Kultaisehen kukkarohon,
%%    Hopeaisehen kehä'än,
%%    Antoi neien tuu'itella,
%%    Ilman immen vaapotella.
%%  \endverse
%%  \beginverse
%%    Neiti pitkän pilven päällä,
%%    Impi ilman partahalla,
%%    Tuota tulta tuu'ittavi,
%%    Valkeaista vaapottavi,
%%    Kultaisessa kätkyessä,
%%    Hihnoissa hopeisissa;
%%    Hopeiset hihnat helkkyi,
%%    Kätkyt kultainen kulisi,
%%    Pilvet liikkui, taivot naukui,
%%    Taivon kannet kallistihe,
%%    Tulta tuu'iteltaessa,
%%    Valkeata vaapottaissa.
%%  \endverse
%%  \beginverse
%%    Impi tulta tuu'itteli,
%%    Valkeaista vaapotteli,
%%    Tulta sormin suoritteli,
%%    Käsin vaali valkeaista,
%%    Tuli tuhmalta putosi,
%%    Valkea varattomalta,
%%    Kätösistä käänteliän,
%%    Sormilta somittelian.
%%  \endverse
%%  \beginverse
%%    Kirposi tulikipuna,
%%    Suikahti punasoronen,
%%    Läpi läikkyi taivosista,
%%    Puhki pilvistä putosi.
%%    Päältä taivahan yheksän,
%%    Halku kuuen kirjokannen.
%%  \endverse
%%  \beginverse
%%    Tuikahti tulikipuna,
%%    Putosi punasoronen,
%%    Luojan luomilta tiloilta,
%%    Ukon ilman iskemiltä,
%%    Puhki reppänän retuisen,
%%    Kautta kuivan kurkihirren,
%%    Tuurin uutehen tupahan,
%%    Palvosen laettomahan;
%%    Sitten sinne tultuansa
%%    Tuurin uutehen tupahan,
%%    Panihe pahoille töille,
%%    Löihe töille törkeille:
%%    Riipi rinnat tyttäriltä,
%%    Käsivarret neitosilta,
%%    Turmeli pojilta polvet,
%%    Isännältä parran poltti.
%%  \endverse
%%  \beginverse
%%    Äiti lastansa imetti
%%    Kätkyessä vaivaisessa
%%    Alla reppänän retuisen;
%%    Siihen tultua tulonen
%%    Poltti lapsen kätkyestä,
%%    Puhki paarmahat emolta,
%%    Meni siitä mennessänsä,
%%    Vieri vieriellessänsä,
%%    Ensin poltti paljon maita,
%%    Paljon maita, paljon soita,
%%    Poltti auhtoja ahoja,
%%    Sekä korpia kovasti,
%%    Viimein vieprahti vetehen,
%%    Aaltoihin Aluejärven.
%%  \endverse
%%  \beginverse
%%    Tuosta tuo Aluejärvi
%%    Oli syttyä tulehen,
%%    Säkehinä säihkyellä,
%%    Tuon tuiman tulen käsissä,
%%    Ärtyi päälle äyrästensä,
%%    Kuohui päälle korpikuusten,
%%    Kuohui kuiville kalansa,
%%    Arinoille ahvenensa.
%%  \endverse
%%  \beginverse
%%    Viel' ei viihtynyt tulonen,
%%    Aalloista Aluejärven,
%%    Karkasi katajikkohon,
%%    Niin paloi katajakangas,
%%    Kohahutti kuusikkohon,
%%    Poltti kuusikon komean,
%%    Vieri vieläki etemmä,
%%    Poltti puolen Pohjanmaata,
%%    Sakaran Savon rajoa,
%%    Kappalehen Karjalata.
%%  \endverse
%%  \beginverse
%%    Kävi siitä kätkösehen,
%%    Pillojansa piilemähän,
%%    Heittihe lepeämähän
%%    Kahen kannon juuren alle,
%%    Lahokannon kainalohon,
%%    Leppäpökkelön povehen,
%%    Sieltä tuotihin tupihin,
%%    Honkaisihin huonehisin,
%%    Päivällä käsin pi'ellä
%%    Kivisessä kiukahassa,
%%    Yöllä lie'essä levätä
%%    Hiilisessä hinkalossa.
%%  \endverse
%%\endsong


\beginsong{Veden synty}[tags={vesi 1}]
  \beginverse
    Tiettävä on vetosen synty,
    Kanssa kastehen sijentö:
    Vesi on tullut taivosesta,
    Pilvistä pisarehina,
    Siitä vuoressa sikesi,
    Kasvoi kallion lomassa.
    Vesiviitta Vaitan poika,
    Suoviitta Kalevan poika,
    Veen kaivoi kalliosta,
    Veen vuoresta valutti,
    Kepillänsä kultaisella,
    Sauvallansa vaskisella.
  \endverse
  \beginverse
    Vuoresta valuttuansa,
    Kalliosta saatuansa,
    Vesi heilui hettehenä,
    Kulki pieninä puroina,
    Siitä suureksi sukeni,
    Sai jokena juoksemahan,
    Virtana vipajamahan,
    Koskena kohajamahan,
    Tuonne suurehen merehen,
    Alaisehen aukehesen.
  \endverse
\endsong


\beginsong{Raudan synty}[]
  \beginverse
    Itse ilmoinen Jumala,
    Tuo Ukko, ylinen Luoja,
    Hieroi kahta kämmentänsä
    Vasemmassa polven päässä,
    Siitä syntyi neittä kolme,
    Koko kolme Luonnotarta,
    Rauan ruostehen emoiksi,
    Suu sinervän siittäjiksi.
  \endverse
  \beginverse
    Neiet käyä notkutteli,
    Astui immet ilman äärtä,
    Utarilla uhkuvilla,
    Nännillä pakottavilla,
    Lypsit maalle maitojansa,
    Uhkutit utariansa,
    Lypsit maille, lypsit soille,
    Lypsit vienoille vesille.
    Yksi lypsi mustan maion,
    Vanhimpainen neitoksia,
    Toinen puikutti punaisen,
    Keskimmäinen neitosia,
    Kolmas valkean valutti,
    Nuorimpainen neitosia.
    Ku on lypsi mustan maion,
    Siitä syntyi melto rauta,
    Ku on puikutti punaisen,
    Siit' on saatu rääkyrauta,
    Ku on valkean valutti,
    Siit' on tehtynä teräkset.
  \endverse
  \beginverse
    Oli aikoa vähäisen,
    Rauta tahtovi tavata
    Vanhempata veljeänsä,
    Käyä tulta tuntemassa.
    Tuli tuhmaksi repesi,
    Kovin kasvoi kauheaksi,
    Poltti soita, poltti maita,
    Poltti korpia kovia,
    Oli polttoa poloisen
    Rauta raukan veikkosensa;
    Rauta pääsevi pakohon,
    Pakohon ja piilemähän
    Pimeähän Pohjolahan,
    Lapin laajalle perälle,
    Suurimmalle suon selälle,
    Tuiman tunturin laelle,
    Jossa joutsenet munivat,
    Hanhi poiat hautelevi.
  \endverse
  \beginverse
    Rauta suossa soikottavi,
    Vetelässä vellottavi,
    Piili vuoen, piili toisen,
    Piili kohta kolmannenki,
    Ei toki pakohon pääsnyt
    Tulen tuimista käsistä,
    Piti käyä toisen kerran,
    Lähteä tulen tuville,
    Astalaksi tehtäessä,
    Miekaksi taottaessa.
  \endverse
  \beginverse
    Susi juoksi suota myöten,
    Karhu kangasta samosi,
    Suo nousi suen jaloissa,
    Kangas karhun kämmenissä,
    Kasvoi rautaiset karangot,
    Teräksiset tierottimet,
    Suen sorkkien sijoille,
    Karhun kannan kaivamille.
  \endverse
  \beginverse
    Tuop' on seppo Ilmarinen,
    Taki taitava takoja,
    Oli teitensä käviä,
    Matkojensa mitteliä,
    Joutuvi suen jälille,
    Karhun kantapään sijoille.
    Näki rautaiset orahat,
    Teräksiset tierottimet,
    Suen suurilla jälillä,
    Karhun kannan kääntämillä,
    Sanovi sanalla tuolla:
    "Voi sinua rauta raukka,
    Kun olet kurjassa tilassa,
    Alahaisessa asussa,
    Suolla sorkissa sutosen,
    Aina karhun askelissa',
    Kasvaisitko kaunihiksi,
    Koreaksi korkenisit,
    jos sun suosta suorittaisin,
    Sekä saattaisin pajahan,
    Tunkisin tulisijahan,
    Ahjohon asettelisin?"
  \endverse
  \beginverse
    Rauta raukka säpsähtihe,
    säpsähtihe, säikähtihe,
    Kun kuuli tulen sanomat,
    Tulen tuiman maininnaiset.
  \endverse
  \beginverse
    Sanoi seppo Ilmarinen:
    "Et sä synny rauta raukka,
    Ei sinun suku sukeu,
    Eikä kasva heimokunta,
    Ilman tuimatta tuletta,
    Ilman viemättä pajahan,
    Ahjohon asettamatta,
    Lietsimellä lietsomatta;
    Vaan ellös sitä varatko,
    Ellös olko milläskänä,
    Tuli ei polta tuttuansa,
    Herjaele heimoansa;
    Kun tulet tulen tuville,
    Hiilisehen hinkalohon,
    Siellä kasvat kaunihiksi,
    Ylenet ylen ehoksi,
    Miesten miekoiksi hyviksi,
    Naisten nauhan päättimiksi."
  \endverse
  \beginverse
    Senpä päivyen perästä
    Rauta suosta sotkettihin,
    Vetelästä vellottihin,
    Saatihin saven seasta;
    Itse seppo suossa seisoin
    Polvin mustassa murassa
    Rauan suosta raaettaissa,
    Maan murasta muokattaissa,
    Otti rautaiset orahat,
    Teräksiset tierottimet,
    Suen suurista jälistä,
    Karhun kantapään tiloista.
  \endverse
  \beginverse
    Se on seppo Ilmarinen
    Siihen painoi palkehensa,
    Siihen ahjonsa asetti,
    Suurille suen jälille,
    Karhun kannan hiertimille;
    Rauan tunkevi tulehen,
    Lietsoi yön levähtämättä,
    Päivän umpehuttamatta,
    Lietsoi päivän, lietsoi toisen,
    Lietsoi kohta kolmannenki,
    Rauta vellinä venyvi,
    Kuonana kohaelevi,
    Venyi vehnäisnä tahasna,
    Rukihisna taikinana,
    Sepon suurissa tulissa,
    Ilmi valkean väessä.
  \endverse
  \beginverse
    Siitä seppo Ilmarinen
    Katsoi ahjonsa alusta,
    Mitä ahjo antanevi,
    Palkehensa painanevi.
    Ensin saapi rääkyrauan,
    Sitten kuonaisen kuletti,
    Siitä valkean valutti
    Alisesta lietsimestä.
  \endverse
  \beginverse
    Siinä huuti rauta raukka:
    "Oi sie seppo Ilmarinen
    Ota pois minua täältä
    Tuskista tulen vihaisen!"
  \endverse
  \beginverse
    Sanoi seppo Ilmarinen:
    "Jos otan sinun tulesta,
    Ehkä kasvat kauheaksi,
    Kovin raivoksi rupeat,
    Vielä veistät veljeäsi,
    Lastuat emosi lasta."
  \endverse
  \beginverse
    Siinä vannoi rauta raukka,
    Vannoi vaikean valansa,
    Ahjossa, alaisimella,
    Vasaroilla valkkamilla:
    "En mä liikuta lihoa,
    Enkä verta vierehytä;
    On mun puuta purrakseni,
    Haukatakseni hakoa,
    Närettä näpätäkseni,
    Kiven syäntä syöäkseni,
    Ett' en veistä veljeäni,
    Lastua emoni lasta.
    Parempi on ollakseni,
    Ehompi eleäkseni,
    Kulkialla kumppalina,
    Käyvällä käsiasenna,
    Kuin sukua suin piellä,
    Heimoani herjaella."
  \endverse
  \beginverse
    Silloin seppo Ilmarinen,
    Takoja ijän ikuinen,
    Rauan tempasi tulesta,
    Asetti alaisimelle,
    Rakentoa raukeaksi,
    Takoa teräkaluiksi,
    Keihä'iksi, kirvehiksi,
    Kaikenlaisiksi kaluiksi.
    Takoa taputtelevi,
    Lyöä helkähyttelevi,
    Vaan ei kiehu rauan kieli,
    Ei sukeu suu teräksen,
    Rauta ei kasva karkeaksi,
    Terä rauan tenhosaksi.
  \endverse
  \beginverse
    Siitä seppo Ilmarinen
    Arvellen ajatelevi,
    Mitä tuohon tuotanehen
    Ja kuta ve'ettänehen
    Teräksen tekomujuiksi,
    Rauan karkaisuvesiksi.
    Laati pikkuisen poroa,
    Lipeäistä liuvotteli,
    Siitä koitti kielellänsä,
    Hyvin maistoi mielellänsä,
    Itse tuon sanoiksi virkki:
    "Ei nämät hyvät minulle
    Teräksen tekovesiksi,
    Rautojen rakento-maiksi."
  \endverse
  \beginverse
    Mehiläinen maasta nousi
    Sinisiipi mättähästä,
    Lentelevi, liitelevi,
    Ympäri sepon pajoa;
    Senp' on seppo Ilmarinen
    Käski käyä metsolassa,
    Tuoa mettä metsolasta,
    Simoa simasalosta,
    Teräksille tehtäville,
    Rauoille rakettaville.
  \endverse
  \beginverse
    Herhiläionen Hiien lintu,
    Hiien lintu, Lemmon katti,
    Lensi ympäri pajoa
    Kipujansa kaupotellen,
    Lentelevi, kuuntelevi
    Sepon selviä sanoja
    teräksistä tehtävistä,
    Rauoista rakettavista.
  \endverse
  \beginverse
    Tuo oli siiviltä sivakka,
    Kynäluilta luikkahampi,
    Tuop' on ennätti e'ellä,
    Nouti Hiien hirmuloita,
    Kantoi käärmehen kähyjä,
    Maon mustia mujuja,
    Kusiaisen kutkelmoita,
    Sammakon salavihoja,
    Teräksen tekomujuihin,
    Rauan karkaisuvetehen.
  \endverse
  \beginverse
    Itse seppo Ilmarinen,
    Takoja alinomainen,
    Luulevi, ajattelevi,
    Mehiläisen tulleheksi,
    Tuon on mettä tuoneheksi,
    Kantaneheksi simoa,
    Sanan virkkoi, noin nimesi:
    "Kas nämät hyvät minulle
    Teräksen tekovesiksi,
    Rautojen rakennusmaiksi."
    Siihen kasti rauta raukan,
    Siihen tempaisi teräksen,
    Pois tulesta tuotaessa,
    Ahjosta otettaessa;
    Siitä sai teräs pahaksi,
    Rauta raivoksi rupesi,
    Veisti raukka veljeänsä,
    Sukuansa suin piteli,
    Veren laski vuotamahan,
    Hurmehen hurajamahan.
  \endverse
\endsong


\beginsong{Varjele vakainen luoja}[by={Kalevala: 43. runo}]
  \beginverse
    Anna Luoja, suo Jumala
    anna onni ollaksemme.
    Hyvin ain’ eleäksemme,
    Kunnialla kuollaksemme.
    Suloisessa Suomenmaassa
    Kaunihissa Karjalassa!
  \endverse
  \beginverse
    Varjele, vakainen Luoja
    Kaitse, kaunoinen Jumala,
    Ole puolla poikiesi,
    Aina lastesi apuna,
    Aina yöllisnä tukena,
    Päivällisnä vartiana.
  \endverse
\endsong


% \sclearpage
% \beginsong{Höyhensaaret}[by={Eino Leino}]
%   \beginverse
%     Mitä siitä jos nuorna ma murrunkin
%     tai taitun ma talvisäihin,
%     moni murtunut onpi jo ennemmin
%     ja jäätynyt elämän jäihin.
%     Kuka vanhana vaappua tahtoiskaan?
%     Ikinuori on nuoruus laulujen vaan
%     ja kerkät lemmen ja keväimen,
%     ilot sammuvi ihmisten.
%   \endverse
%   \beginverse
%     Mitä siitä jos en minä sammukaan
%     kuin rauhainen, riutuva liesi,
%     jos sammun kuin sammuvat tähdet vaan
%     ja vaipuvi merillä miesi.
%     Kas, laulaja tähtiä laulelee
%     ja hän meriä suuria seilailee
%     ja hukkuvi hyrskyhyn, ennen kuin
%     käy purjehin reivatuin.
%   \endverse
%   \beginverse
%     Mitä siitä jos en minä saanutkaan,
%     mitä toivoin ma elämältä,
%     kun sain minä toivehet suuret vaan
%     ja kaihojen kantelen hältä.
%     Ja vaikka ma laps olen pieni vain,
%     niin jumalten riemut ma juoda sain
%     ja juoda ne täysin siemauksin ---
%     niin riemut kuin murheetkin.
%   \endverse
%   \beginverse
%     Ja vaikka ma laps olen syksyn vaan
%     ja istuja pitkän illan,
%     sain soittaa ma kielillä kukkivan maan
%     ja hieprukan hivuksilla.
%     Niin mustat, niin mustat ne olivat;
%     ja suurina surut ne tulivat,
%     mut kaikuos riemu nyt kantelen
%     vielä kertasi viimeisen!
%   \endverse
%   \beginverse
%     Oi, kantelo pitkien kaihojen,
%     sinä aarteeni omani, ainoo!
%     Me kaksi, me kuulumme yhtehen,
%     jos kuin mua kohtalot vainoo.
%     Me kuljemme kylästä kylähän näin,
%     ohi kylien koirien räkyttäväin,
%     ja keskellä raition raakuuden
%     sävel soipa on keväimen.
%   \endverse
%   \beginverse
%     Me kaksi, me tulemme metsästä
%     ja me metsien ilmaa tuomme,
%     me laulamme nuoresta lemmestä
%     ja lempemme kuvan me luomme,
%     me luomme sen maailman tomusta niin
%     kuin Luoja loi ihmisen Eedeniin
%     ja korvesta kohoitamme me sen
%     kun vaskisen käärmehen.
%   \endverse
%   \beginverse
%     Te ystävät, joiden rinnassa kyyt
%     yön-pitkät pistää ja kalvaa,
%     te, joita jäytävi sydämen syyt
%     ja elämä harmaja halvaa,
%     oi, helise heille mun kantelein,
%     oi, helise onnea haavehein
%     ja unta silmihin unettomiin
%     mun silmäni suljit sa niin.
%     Kas, ylläpä mustien murheiden
%     on kaunihit taivaankaaret
%     ja kaukana keskellä aaltojen
%     on haaveiden höyhensaaret
%     ja ken sinne lapsosen kaarnalla käy,
%     ei sille ne aavehet yölliset näy,
%     vaan rinnoin hän uinuvi rauhaisin
%     kuin äitinsä helmoihin.
%   \endverse
%   \beginverse
%     Mitä siitä jos valhetta onkin ne vaan
%     ja kestä ei päivän terää!
%     Me uinumme siksi kuin valveutaan
%     ja vaivat ne jällehen herää.
%     Moni nukkui nuorihin toiveisiin
%     ja heräsi hapsihin hopeisiin;
%     hän katsahti ympäri kummissaan
%     ja --- uinahti uudestaan.
%     Miks ihmiset tahtoa, taistella
%     ja koittaa korkealle?
%     Me olemme kaikki vain lapsia
%     ja murrumme murheen alle.
%     Miks emme me kaikki vois uinahtaa
%     ja hyviä olla ja hymytä vaan
%     ja katsoa katsehin kirkkahin
%     vain sielumme syvyyksiin?
%   \endverse
%   \beginverse
%     Oi, unessa murheet ne unhottuu
%     ja rauhaton rauhan saapi,
%     oi, unessa vankikin vapautuu,
%     sen kahlehet katkeaapi,
%     ja köyhä on rikas kuin kuningas maan
%     ja kevyt on valtikka kuninkaan
%     ja kaikki, kaikki on veljiä vaan ---
%     oi, onnea unelmain!
%   \endverse
%   \beginverse
%     Oi, onnea uinua uudelleen
%     ne lapsuen päivät lauhat
%     ja itkeä jällehen yksikseen
%     ne riemut ja rinnan rauhat;
%     taas uskoa, että on lapsi vaan
%     ja että voi alkaa uudestaan
%     ja uskoa uusihin toiveisiin
%     sekä vanhoihin ystäviin!
%   \endverse
%   \beginverse
%     Taas uskoa riemuhun, keväimeen
%     ja lippuhun pilvien linnan
%     ja uskoa lempehen puhtaaseen
%     taas kahden puhtahan rinnan,
%     taas uskoa itsensä rikkahaks
%     ja maailman suureks ja avaraks ---
%     voi, kuinka se sentään on ihanaa,
%     kun sen nuorena uskoa saa!
%   \endverse
%   \beginverse
%     Voi, kuinka se sille on ihanaa,
%     joka kaiken sen kadotti kerran,
%     joka häkistä katseli maailmaa
%     ja näki vain vaaksan verran,
%     joka etsi kauneutta, elämää,
%     ja näki vain markkinavilinää,
%     ja näki räyhäävän raakuuden, tyhmyyden ---
%     niit' aikoja unhota en.
%   \endverse
%   \beginverse
%     Kun muistelen, kuinka ma kerjännyt
%     olen koirana lempeä täällä,
%     miten rikasten portailla pyydellyt
%     olen tuiskulla, tuulissäällä,
%     vain lämpöä hiukkasen, hiukkasen vain
%     ja kun minä muistelen, mitä mä sain
%     ja mitä mä nielin ja vaikenin
%     ja mitä mä ajattelin!
%   \endverse
%   \beginverse
%     Miten olen minä kulkenut, uskonut,
%     ett'eivät ne unhoitukaan!
%     Ja sentään ne olen minä unhoittanut
%     kuin unhoittaa voi kukaan.
%     Ja sentään se nousi, niin kohtalot kaas,
%     ja sentään ma seppona seison taas
%     ja taivahan kansia taon ja lyön ---
%     oi, onnea tähtisen yön!
%   \endverse
%   \beginverse
%     Ne saapuvat, saapuvat uudestaan
%     mun onneni orhit valkeet,
%     ne painavat vanhalla voimallaan
%     mun rintani jättipalkeet.
%     Ja kirkas on taivas ja kukkii maa
%     ja säkenet suustani suitsuaa
%     ja ääneni on kuni ukkosen ---
%     oi, onnea unelmien!
%   \endverse
%   \beginverse
%     Mitä siitä jos haaveeni verkot vaan
%     on verkkoja hämähäkin!
%     Mitä siitä jos omieni viittova vaan
%     on laulua laineiden näkin!
%     Moni nukkui nuorihin toiveisiin
%     ja heräsi hapsihin hopeisiin
%     tai herännyt täällä ei milloinkaan.
%     Missä? Milloin? Helmassa maan.
%     Minä tahdon riemuja keväimen
%     ja onnesta osani kerta!
%     Olen imenyt rintoja totuuden,
%     mut niistä vaan tuli verta.
%     Siis, tulkaa te utaret unelmien,
%     minä vaivun riemunne rinnoillen
%     ja uskon päivähän, aurinkohon.
%     Unen maito on loppumaton.
%   \endverse
%   \beginverse
%     Oi, kauniisti mulle te kaartukaa,
%     mun syömeni sateenkaaret!
%     Mua hiljaa, hiljaa tuudittakaa,
%     te haaveiden höyhensaaret!
%     Mua katsokaa: olen lapsi vaan,
%     olen riisunut päältäni riemut maan
%     ja pyytehet kullan ja kunnian.
%     Uni onni on laulajan.
%   \endverse
%   \beginverse
%     Minä tahdon vain uinua yksikseen.
%     En tahtois vielä mä kuolla.
%     Mut kuulkaa, jo äitini huhuilee
%     Tuonen aaltojen tuolla puolla.
%     Oi, odota hetkinen, äityein!
%     En viel' olis valmis ma matkallein,
%     mun syömeni on niin syyllinen.
%     Suo että mä pesen sen.
%   \endverse
%   \beginverse
%     Suo että mä ensin huuhdon vaan
%     nämä synkeät, huonot aatteet,
%     suo että mä päälleni ensin saan
%     ne puhtahat, valkeat vaatteet,
%     jotk' ompeli onneni impynen,
%     hän, hämärän impeni ihmeellinen,
%     min kuvaa kannan ma sydämessäin
%     siit' asti kuin hänet mä näin.
%   \endverse
%   \beginverse
%     Me tulemme, äitini armahain!
%     Oi katso, meitä on kaksi!
%     Oi katso, mik' on mulla rinnassain!
%     Niin oisitko rikkahaksi
%     sinä uskonut koskaan kuopustas?
%     Ja katso, me pyydämme siunaustas,
%     sun poikasi synkeä, syyllinen,
%     ja mun impeni puhtoinen.
%   \endverse
%   \beginverse
%     Katso, kuin hän on kaunis ja valkoinen
%     ja muistuttaa niin sua!
%     Hän on niin hellä ja herttainen,
%     vaikk'ei hän lemmi mua,
%     Elä kysele hältä, miks tänne mun toi,
%     mut usko, se niin oli parhain, oi!
%     Ja usko, nyt ett' olen onnellinen
%     kuin aikoina lapsuuden.
%   \endverse
%   \beginverse
%     Elä kysele multa sa laaksoista maan!
%     Ei olleet ne luodut mulle.
%     Mut jos sinun silmäsi tutkii vaan,
%     voin laulaa ma laulun sulle
%     kuin lauloin ma lapsen aikoihin ---
%     kas, lauluna sujuu se paremmin
%     ja kyynelet kuuluvat kantelehen.
%     Niitä muuten ma ilmoita en.
%   \endverse
% \endsong


% \beginsong{Hymyilevä Apollo}[by={Eino Leino}]
%   \beginverse
%     Näin lauloin ma kuolleelle äidillein
%     ja äiti mun ymmärsi heti.
%     Hän painoi suukkosen otsallein
%     ja sylihinsä mun veti:
%     "Ken uskovi toteen, ken unelmaan, ---
%     sama se, kun täysin sa uskot vaan!
%     Sun uskos se juuri on totuutes.
%     Usko poikani unehes!"
%   \endverse
%   \beginverse
%     Miten mielelläin, niin mielelläin
%     hänen luoksensa jäänyt oisin
%     luo Tuonen virtojen viileäin,
%     mut kohtalot päätti toisin.
%     Vielä viimeisen kerran viittasi hän
%     kuin hän vain viitata tiesi.
%     Taas seisoin ma rannalla elämän,
%     mut nyt olin toinen miesi.
%   \endverse
%   \beginverse
%     Nyt tulkaa te murheet ja vastukset,
%     niin saatte te vasten suuta!
%     Nyt raudasta mulla on jänteret,
%     nyt luuni on yhtä luuta.
%     Kas, Apolloa, joka hymyilee,
%     sitä voita ei Olympo jumalineen,
%     ei Tartarus, Pluto, ei Poseidon.
%     Hymyn voima on voittamaton.
%   \endverse
%   \beginverse
%     Meri pauhaa, ukkonen jylisee,
%     Apollo saapuu ja hymyy.
%     Ja katso! Ukkonen vaikenee,
%     tuul' laantuu, lainehet lymyy.
%     Hän hymyllä maailman hallitsee,
%     hän laululla valtansa vallitsee,
%     ja laulunsa korkea, lempeä on.
%     Lemmen voima on voittamaton.
%   \endverse
%   \beginverse
%     Kun aavehet mieltäsi ahdistaa,
%     niin lemmi! --- ja aavehet haihtuu.
%     Kun murheet sun sielusi mustaks saa,
%     niin lemmi! --- ja iloks ne vaihtuu.
%     Ja jos sua häpäisee vihamies,
%     niin lemmellä katko sen kaunan ies
%     ja katso, hän kasvonsa kääntää pois
%     kuin itse hän hävennyt ois.
%   \endverse
%   \beginverse
%     Kuka taitavi lempeä vastustaa?
%     Ketä voita ei lemmen kieli?
%     Sitä kuulee taivas ja kuulee maa
%     ja ilma ja ihmismieli.
%     Kas, povet se aukovi paatuneet,
%     se rungot nostavi maatuneet
%     ja kutovi lehtihin, kukkasiin
%     ja uusihin unelmiin.
%   \endverse
%   \beginverse
%     Ei paha ole kenkään ihminen,
%     vaan toinen on heikompi toista.
%     Paljon hyvää on rinnassa jokaisen,
%     vaikk' ei aina esille loista.
%     Kas, hymy jo puoli on hyvettä
%     ja itkeä ei voi ilkeä;
%     miss' ihmiset tuntevat tuntehin,
%   \endverse
%   % NOTE: verses are missing!
% \endsong


\beginsong{Soutaja}[by={Unto Kupiainen}]
  \beginverse
    Vieras on virta ja vieras on vene, 
    eivät ne unelmies uomia mene. 
    Ilta on ihmisessä ja aamu on outo; 
    illasta aamuun on ihmisen souto. 
    Illasta aamuun on yöllistä matkaa; 
    jos jaksat uskoa, jaksat jatkaa. 
    Taapäin tuijotat, soudat eteen 
    outoa venettä outoon veteen. 
  \endverse
\endsong
