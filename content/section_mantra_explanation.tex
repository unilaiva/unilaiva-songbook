% Mantra chapter explanation
%
% (Note that this file is not supposed to be wrapped in 'songs' environment.)

  \subsection*{Mantras}

  When you chant even without knowing the meaning, that itself carries power. But when you know
  the meaning and chant with that feeling in your heart then the energy would flow million times
  more powerful. Therefore it is essential to know the meaning of the Mantra when you use it.
  Mantra is like calling a name. The meaning isn't as important as the vibrations of the words
  have on the body and energy centers.

  Each mantra is recommended to be chanted for an entire \emph{mala}: 108 times. According to the Vedic
  scriptures, our bodies ---  physical and subtle --- contain 72,000 energy channels, called Nadis.
  There are 108 major nadis that meet in the ``sacred heart'' \emph{(hrit padma)}. By chanting a mantra
  108 times the energy permeates the entire body and energy body.  There are six senses (sight,
  sound, smell, taste, touch, and consciousness) multiplied by three reactions (positive,
  negative, or indifference) making 18 ``feelings''. Each of these feelings can be either attached
  to pleasure or detached from pleasure making 36 ``passions'', each of which may be manifested in
  the past, present, or future. All the combinations of all these things makes a total of 108,
  which are represented by the beads.

  Choose a particular mantra and make it part of your daily life for at least for 21 days --- or
  the auspicious number of 40 days. Chant at dawn and/or dusk, or when you can.
  \textbf{Be focused, and enjoy each repetition, being as present as you can.}

  Buddhism takes the view that the nature of everything in its most restful state is the blissful
  union of wisdom and compassion. This is symbolized in Tibetan tantric practice in the union of
  male and female energies. Within this state resides our pure consciousness. Mantra is its sound.
  Relaxing with conscious mantra recitation enables pure consciousness to materialise.

  As the word \emph{mantra} suggests, it becomes a tool that holds the mind together. The power
  generated in this concentrated affirmation is believed to cut through the vision of impure
  self-perception, regarded as the root of all suffering.

  \subsection*{Bhajans}
  In addition to mantras, this chapter contains varied kinds of devotional songs from the Indian
  subcontinent or inspired by the spirituality found therein. They are collectively called
  \emph{bhajans}. Literally, the term means ``sharing''.

  \subsection*{Common terms}
  \begin{description}
   \item[OM:] usually chanted at the beginning of every mantra. It is known as a ``seedsound'' that
     is extremely potent and expresses a particular energy. A translation will always fall short
     and is actually impossible. Om is the sound of the sixth chakra (third eye). Here is where
     the masculine and feminine energies meet. It is called the Soundless Sound, or the Sound of
     the Universe.
   \item[Namaha:] a common ending to many mantras, ``I offer'', giving thanks.
  \end{description}
%   % Commented out, as this image is now used as the chapter's front image
%   \begin{center}%
%     \vspace*{\fill}%
%     % 0.618^2 ~ 0.382 (Golden Ratio)
%     \includegraphics[width=0.382\textwidth]{lotus_om_bw_transparent_bg_2000px.png}%
%     \vspace*{\fill}%
%   \end{center}%
