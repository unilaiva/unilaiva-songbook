% Songs in other languages

\beginsong{Ide Were}[by={Yoruba, traditional}]
  \beginverse
    Ide |\[Em]were were nita O|\[D]shun
    Ide |\[C]were were|\[C]-
    Ide |\[Em]were were nita O|\[D]shun
    Ide |\[C]were were nita ya|
    |\[C] Ocha kini|ba nita O|\[D]shun
    Che|\[G]ke cheke che|\[C]ke nita |\[D]ya
    Ide |\[C]were were|\[C]-
  \endverse
  \begin{explanation}
    This Yoruba chant is dedicated to \textbf{Oshun}, the Goddess of Love, 
    happiness and prosperity. She brings to us all the good things of life, 
    and is defender of the poor and the mother of all orphans; Goddess 
    Ochun brings to them their needs in this life.

    Oshun (also known as Ochun or Oxum in Latin America) is an orisha, a highly 
    benevolent spirit or deity that reflects one of the expressions of God in 
    the Ifa and Yoruba religions (Nigeria). 

    Thought to be the most beautiful of the female Orixas. No one can resist 
    her charming laugh, her graceful dancing, and her lips that taste like 
    honey. She has a lush womanly figure with full hips, which suggest 
    fertility and eroticism.

    She exhibits all of the attributes connected with fresh flowing water: 
    lively, sparking, refreshing, vivacious. She is the \underline{Goddess 
    of sweet water} and can be discovered where there is fresh water, at 
    rivers, ponds, lakes, and particularly waterfalls. 

    She is also a healer of the sick. Teacher, who taught the Yoruba culture, 
    agriculture and mysticism. The art of divination using cowrie shells. The 
    bringer of song, music and dance, healing chants and meditations taught 
    to her by her father Obatala, the first of the created Orishi.

    This chant speaks of a necklace as a symbol of initiation into love. 
  \end{explanation}
\endsong
