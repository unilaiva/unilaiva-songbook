% Songs in other languages

\beginsong{Ancient Aramaic Prayer}
  \chordsoff % there are no chords
  \vskip 1em
  \begin{center} % center the lines
    Abwûn d'bwaschmâja
    \vskip 1em
    Nethkâdasch schmach
    \vskip 1em
    Têtê malkuthach.
    \vskip 1em
    Nehwê tzevjânach aikâna d'bwaschmâja af b'arha.
    \vskip 1em
    Hawvlân lachma d'sûnkanân jaomâna.
    \vskip 1em
    Waschboklân chaubên wachtahên aikâna \\
    daf chnân schwoken l'chaijabên.
    \vskip 1em
    Wela tachlân l'nesjuna
    \vskip 1em
    ela patzân min bischa.
    \vskip 1em
    Metol dilachie malkutha wahaila wateschbuchta l'ahlâm almîn.
    \vskip 1em
    Amên.
    \vskip 2em
    % Ancient aramaic symbol:
    %   - upper dot: God (mind)
    %   - left dot:  Son (wisdom)
    %   - right dot: Spirit (life)
    %   - below(?): One Universal God
    % The symbol has been used by ancient Near Eastern scribes
    % to indicate that the writing was of a sacred nature
    \includegraphics[width=0.12\textwidth]{ancient_aramaic_symbol_bw_transparent_bg_184x225px.png}
  \end{center}
  \brk % to suggest putting a page break here
  \begin{translation}
    \vskip 8em % try to align with the original prayer
    %\normalsize % scale up from ordinary translation, to align (there is space on the page)
    \begin{center}
      Oh Thou, from whom the breath of life comes,
      who fills all realms of sound, light and vibration.
      \nextverse
      May Your light be experienced in my utmost holiest.
      \vskip 0.5em % to align
      \nextverse
      Your Heavenly Domain approaches.
      \vskip 0.5em % to align
      \nextverse
      Let Your will come true - in the universe (all that vibrates)
      just as on earth (that is material and dense).
      \nextverse
      Give us wisdom (understanding, assistance)
      for our daily need.
      \nextverse
      Detach the fetters of faults that bind us, (karma)
      like we let go the guilt of others.
      \nextverse
      Let us not be lost in superficial things
      (materialism, common temptations),
      \nextverse
      but let us be freed from that what keeps us off from
      our true purpose.
      \nextverse
      From You comes the all-working will, the lively strength to act,
      the song that beautifies all and renews itself from age to age.
      \nextverse
      Sealed in trust, faith and truth.
      (I confirm with my entire being.)
    \end{center}
  \end{translation}
\endsong


\beginsong{Lecha Eli}[by={Rabbi Avraham Iebn Ezra, Yair Gadassi},ex={hebrew},tags={source 1}]
  \beginverse
    |\[Am] Lecha E|li | teshuka|\[G]ti |
    | Becha chesh|\[Dm]ki |\[Em] ve'ahava|\[Am]ti | |
    |\[Am] Lecha li|bi | vechilyo|\[G]tai |
    | Lecha ru|\[Dm]chi |\[Em] venishma|\[Am]ti | |
  \endverse
  \beginchorus
    \chorusindent |\[Dm] Hashive|ni va'ashu|\[G]va | \textsuperscript{2}( | |)
    \chorusindent | Vetirtzeh |\[Dm] |\[Em]et teshuva|\[Am]ti| |
  \endchorus
  \beginverse
    |^ Lecha ya|dai | lecha rag|^lai |
    | Umimach |^hee |^ techuna|^ti | |
    |^ Lecha atz|mi | lecha da|^mi |
    | Ve'ori |^im |^ geviya|^ti | | \gotochorus{Hashiveni}
  \endverse
  \beginchorus
    |\[Am]Oh |ho oh ho ho ho |\[G]ho | |
    |\[Dm]Oh |\[Em]ho oh ho ho ho |\[Am]ho | |
  \endchorus
  \beginverse
    |^ Lecha ez'|ak | becha ed|^bak |
    | Adei shu|^vi |^ le'adma|^ti | |
    |^ Lecha a|ni | be'odi |^chai |
    | Ve'af ki |^a- |^ charei mo|^ti | | \gotochorus{Hashiveni}
  \endverse
  \begin{translation}
    For You my God is my passion
    In You is my desire and my love
    Yours are my heart and my organs
    Yours are my spirit and my soul
    \nextverse
    \chorusindent Bring me back to You and I will return
    \chorusindent And You shall want my repentance
    \nextverse
    Yours are my hands and legs
    And from You is my character
    Yours are my bones and my blood
    And my skin and my body
    \nextverse
    Oh ho oh ho ho ho ho
    \nextverse
    To You I will call and to You I will cling
    Until I return to my land
    I give myself to You whilst I still live
    And even after I die
  \end{translation}
\endsong


\beginsong{Ishq Allāh\\Love, Lover and Beloved}[by={James Burgess},tags={source 1, love 1},ex={arabic, english}]
  \beginchorus
    \chorusindent |\[Bm]Ishq Allāh ma'|būd Allāh
    \chorusindent Ishq Al|lāh ma'\[A]būd Al|\[Bm]lāh |
  \endchorus
  \beginverse
    |\[A]God is Love, |\[Bm]Lover and Beloved |
    |\[A] Love, Lover and Be|\[Bm]loved |
    |\[A]I am Love, |\[Bm]Lover and Beloved |
    |\[A] Love, Lover and Be|\[Bm]loved |
  \endverse
  \begin{explanation}
    \begin{description}
      \item[Ishq Allāh ma'būd Allāh] translates literally to ``love God adored God''
           which can be interpreted as ``God is Love and God is the Beloved'' -- and more
           poetically as ``God is Love, Lover and Beloved''.
    \end{description}
  \end{explanation}
\endsong


\beginsong{Beautiful Names of God}[tags={source 1},ex={arabic}]
  \meter{3}{4}
  \beginverse
    \[.]Bismil|\[Am]lah, \[.] \[.]Al|\[C]lāh, \[.] \[.]Raḥ|\[G]mān, \[.] \[.]Ra|\[Am]ḥīm \[.]
    \[.]Mā|\[.]lik, \[.] \[.]Qud|\[Dm]dūs, \[.] \[.]Sa|\[C]laām, \[.]Mu’\[.]min, |\[E]Muhay\[.]min \[.] | \[.]- \[.]
    \[.]A|\[Am]zīz, \[.] \[.]Jab|\[E]bār, \[.] \[.]Muta |\[C]kab\[Dm]bir, \[.]Khā|\[E]liq \[.] \[.] | \[.]- \[.]
%    % Original, rhythmically stranger version below:
%    \[.]Bismil|\[Am]lah, \[.] \[.]Al|\[C]lāh, \[.] \[.]Raḥ|\[G]mān, \[.] \[.]Ra|\[Am]ḥīm \[.]
%    \[.]Mā|\[.]lik, \[.] \[.]Qud|\[Dm]dūs, \[.] \[.]Sa|\[C]laām, \[.]Mu’\[.]min, |\[E]Muhay\[.]min \[.] |
%    |\[.] \[.]A\[Am]zīz, |\[.] \[.]Jab\[E]bār, |\[.] \[.]Muta \[C]kab|\[Dm]bir, \[.]Khā\[E]liq | \[.] \[.]
  \endverse
  \begin{translation}
    \textit{In Qur'an:} Begin in the name of God, the One, Compassion, Mercy;
    Sovereign, Holy, Peace, Guarantor, Guardian; 
    Allmighty, Powerful, Tremendous, Creator
  \end{translation}
\endsong


\beginsong{Mash Allah}[tags={you 1, source 1},ex={arabic, english}]
  \beginchorus
    Through your |\[Em]eyes shines the light
    Mash Al|lah mash Allah |
    |\[D]Wonder of \[B7]God in |\[Em]You
  \endchorus
  \beginverse
    |\[G]Mash Al|\[Am]lah mash Allah |
    |\[D7]Mash Al|\[Em]lah mash Allah |
    |\[G]Mash Al|\[Am]lah mash Allah |
    |\[B7]Wonder of God in |\[Em]You |
    |\[B7]Wonder of God in |\[Em]You |
  \endverse
  \begin{explanation}
    \begin{description}
      \item[Mash Allah] is Arabic and means "\textit{as God willed it}". It is used to express
      thankfulness, appreciation or joy for what was just mentioned.
    \end{description}
  \end{explanation}
\endsong


\beginsong{Asse wana hey wana \\ Hey niketi}[ex={hopílavayi, english},tags={heart 1, circle 1}]
  \beginchorus
    |\[Em]Asse wana |\[Am]hey wana |\[D]asse wana |\[Em]hey wana |
  \endchorus
  \beginchorus
    |\[Em]Hey niketi |\[D]hey wana |\[Bm]hey niketi |\[Em]hey wana |
  \endchorus
  \beginchorus
    |\[Em]Hey sister |\[Am]we are one, |\[D]hey brother |\[Em]we are one |
  \endchorus
  \beginchorus
    |\[Em]No matter |\[D]where we're going to |\[Bm]no matter |\[Em]where we're coming from |
  \endchorus
  \begin{explanation}
    \begin{description}
     \item[Wana] is a Hopi word for ``heart''. We are all connected in our hearts.
    \end{description}
  \end{explanation}
\endsong


\beginsong{Weha Ehayo}[by={Lakota},ex={lakȟótiyapi, español, english}]
  \beginverse % 25 beats in this verse
    \chorusindent \[D]Weha eh\[.]ay\[.]o \[A]weha eh\[.]ay\[.]o
    \chorusindent W\[.]eha e\[C]hay\[.]o \[G]weha eh\[.]ay\[.]o
    \chorusindent W\[.]eha e\[C]hay\[.]o \[G]weha \[.]eha\[D]yo! \[.] \[.] \[.] \[.] \[.] \[.] \[.]
  \endverse
  \beginverse\memorize % 36 beats in this verse
    \[D]Gran Esp\[.]írit\[.]u \[A]yo voy \[.]a ped\[.]ir, ó\[.]yem\[.]e \[.] \[.]
    A\[.]l uni\[C]vers\[.]o \[G]yo voy \[.]a ped\[.]ir, ó\[.]yem\[.]e \[.] \[.]
    Par\[.]a mi \[C]puebl\[.]o \[G]que sobrev\[.]iv\[.]a
    y\[.]o he d\[.]icho \[D]hey! \[.] \[.] \[.] \[.] \[.] \[.] \[.] \gotochorus{Weha ehayo}
  \endverse
  \beginverse
    ^Pacham^am^a ^yo voy ^a ped^ir, ó^yem^e ^ ^
    ^a Wira^coch^a ^yo voy ^a ped^ir, ó^yem^e ^ ^
    par^a mi ^puebl^o ^que siempre v^iv^a
    y^o he d^icho ^hey! \[.] \[.] \[.] \[.] \[.] \[.] \[.] \gotochorus{Weha ehayo}
  \endverse
  \beginverse
    ^Great Sp^ir^it ^I am ^going to ^plead, ^hear my ^call ^ ^
    T^o the ^univ^erse ^I am ^going to ^plead, ^hear my ^call ^ ^
    F^or the sur^viv^al ^of our p^eop^le
    ^I am s^aying ^hey! \[.] \[.] \[.] \[.] \[.] \[.] \[.] \gotochorus{Weha ehayo}
  \endverse
  \begin{explanation}
    \begin{description}
     \item[Wiracocha] is the great creator deity in the pre-Inca and Inca mythology in the Andes.
    \end{description}
  \end{explanation}
\endsong


\beginsong{O la Mama \\ Ancient Mother}[tags={Divine Mother 1, Mother Earth 1},by={trad. African},ex={some african language, english}]
  \beginchorus\memorize
    |\[Em] O la |\[Am]Mama, |\[D] wa ha su |\[Em]kola |
    |\[Bm11/D] O la |\[C]Mama, |\[D] wa ha su |\[Em]wam |
    \vspace{1em}
    |\[Em] O la |\[Am]Mama, |\[D] kow wey ha |\[Em]ha ha ha |
    |\[Bm11/D] O la |\[C]Mama, |\[D] ta te ka|\[Em]yee |
  \endchorus
  \beginchorus\memorize
    |^ Ancient |^Mother, |^ I hear you |^calling |
    |^ Ancient |^Mother, |^ I hear your |^song |
    \vspace{1em}
    |^ Ancient |^Mother, |^ I hear your |^laughter |
    |^ Ancient |^Mother, |^ I taste your |^tears |
  \endchorus
  % This second verse is a more unknown addon by somebody:
  \beginchorus
    |^ Ancient |^Mother, |^ I feel you |^calling |
    |^ Ancient |^Mother, |^ I sing your |^song |
    \vspace{1em}
    |^ Ancient |^Mother, |^ I share your |^laughter |
    |^ Ancient |^Mother, |^ I dry your |^tears |
  \endchorus
\endsong


\beginsong{Ide Were}[by={traditional},tags={water 1},ex={yoruba}]
  \beginverse
    Ide |\[Em]were were nita Osh|\[D]un
    Ide |\[C]were were | -
    Ide |\[Em]were were nita Osh|\[D]un
    Ide |\[C]were were nita ya |
    |\[C] Ocha kini|ba nita Osh|\[D]un
    Chek|\[G]e cheke chek|\[C]e nita |\[D]ya
    Ide |\[C]were were | | -
  \endverse
  \begin{explanation}
    This Yoruba chant is dedicated to \textbf{Oshun}, the Goddess of Love, 
    happiness and prosperity. She brings to us all the good things of life, 
    and is defender of the poor and the mother of all orphans; Goddess 
    Ochun brings to them their needs in this life.

    Oshun (also known as Ochun or Oxum in Latin America) is an orisha, a highly 
    benevolent spirit or deity that reflects one of the expressions of God in 
    the Ifa and Yoruba religions (Nigeria). 

    Thought to be the most beautiful of the female Orixas. No one can resist 
    her charming laugh, her graceful dancing, and her lips that taste like 
    honey. She has a lush womanly figure with full hips, which suggest 
    fertility and eroticism.

    She exhibits all of the attributes connected with fresh flowing water: 
    lively, sparking, refreshing, vivacious. She is the \underline{Goddess 
    of sweet water} and can be discovered where there is fresh water, at 
    rivers, ponds, lakes, and particularly waterfalls. 

    She is also a healer of the sick. Teacher, who taught the Yoruba culture, 
    agriculture and mysticism. The art of divination using cowrie shells. The 
    bringer of song, music and dance, healing chants and meditations taught 
    to her by her father Obatala, the first of the created Orishi.

    This chant speaks of a necklace as a symbol of initiation into love. 
    
    According to the Yoruba elders, Oshun [also Osun, Oxum] is the “unseen 
    mother present at every gathering”, because Oshun is the Yoruba 
    understanding of the cosmological forces of water, moisture, and 
    attraction. Therefore she is omnipresent and omnipotent. Her power is 
    represented in another Yoruba scripture which reminds us that “no one is 
    an enemy to water” and therefore everyone has need of and should respect 
    and revere Oshun, as well as her followers.
    
    Oshun is the force of harmony. Harmony we see as beauty, feel as love, 
    and experience as ecstasy. Oshun according to the ancients was the only 
    female Irunmole amongst the original 16 sent from the spirit realm to 
    create the world. As such, she is revered as “Yeye” — the sweet mother 
    of us all. When the male Irunmole attempted to subjugate Oshun due to 
    her femaleness she removed her divine energy, called ase by the Yoruba, 
    from the project of creating the world and all subsequent efforts at 
    creation were in vain. It was not until visiting with the Supreme Being, 
    Olodumare, and begging Oshun pardon under the advice of Olodumare that 
    the world could continue to be created. But not before Oshun had given 
    birth to a son. This son became Elegba, the great conduit of ase in the 
    Universe and also the eternal and infernal trickster.
    
    Oshun is known as Iyalode, the “(explicitly female) chief of the realm.” 
    She is also known as Laketi, she who has ears, because of how quickly 
    and effectively she answers prayers. When she possesses her followers, 
    she dances, flirts and then weeps — because no one can love her enough 
    and the world is not as beautiful as she knows it could be.\\
  \end{explanation}

  \begin{center}%
    \vfill%
    \includegraphics[width=0.618\textwidth]{oshun_bw_transparent_bg_650x811px.png}%
    \vfill
  \end{center}
\endsong

