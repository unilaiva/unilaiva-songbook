% Songs in other languages

\beginsong{Ancient Aramaic Prayer}[ph={I}]
  % NOTE: try to align the translation with the prayer with \vskips. Check this
  % after style changes etc.
  \chordsoff % there are no chords
  \beginverse\justifycenter
    \vskip 3em
    Abwûn d'bwaschmâja
    \vskip 1em
    Nethkâdasch schmach
    \vskip 1em
    Têtê malkuthach.
    \vskip 1em
    Nehwê tzevjânach aikâna d'bwaschmâja af b'arha.
    \vskip 1em
    Hawvlân lachma d'sûnkanân jaomâna.
    \vskip 1em
    Waschboklân chaubên wachtahên aikâna \\
    daf chnân schwoken l'chaijabên.
    \vskip 1em
    Wela tachlân l'nesjuna
    \vskip 1em
    ela patzân min bischa.
    \vskip 1em
    Metol dilachie malkutha wahaila wateschbuchta l'ahlâm almîn.
    \vskip 1em
    Amên.
    \vskip 5em
    \imagec[4]{ancient_aramaic_symbol_bw_transparent_bg_184x225px.png}
  \endverse
  \brk % to suggest putting a page break here
  \begin{translation}\justifycenter
    \- % needed here, otherwise the first vskip has no effect
    \vskip 7.5em % try to align with the original prayer
    %\normalsize % scale up from ordinary translation, to align (there is space on the page)
    Oh Thou, from whom the breath of life comes,
    who fills all realms of sound, light and vibration.
    \vskip 0.5em % to align
    \nextverse
    May Your light be experienced in my utmost holiest.
    \vskip 1em % to align
    \nextverse
    Your Heavenly Domain approaches.
    %\vskip 0.5em % to align
    \nextverse
    Let Your will come true --- in the universe \emph{(all that vibrates)}
    just as on Earth \emph{(that is material and dense)}.
    \nextverse
    Give us wisdom \emph{(understanding, assistance)}
    for our daily need.
    \nextverse
    Detach the fetters of faults that bind us \emph{(karma)},
    like we let go the guilt of others.
    \vskip 0.5em % to align
    \nextverse
    Let us not be lost in superficial things
    \emph{(materialism, common temptations)},
    \nextverse
    but let us be freed from that what keeps us off from
    our true purpose.
    \nextverse
    From You comes the all-working will, the lively strength to act,
    the song that beautifies all and renews itself from age to age.
    \vskip 0.5em % to align
    \nextverse
    Sealed in trust, faith and truth.
    \emph{(I confirm with my entire being.)}
  \end{translation}
  \begin{explanation}
    The symbol has been used by ancient Near Eastern scribes to indicate that
    the writing is of a sacred nature.
    \begin{description}
      \item[upper dot:] God (mind)
      \item[left dot:] Son (wisdom)
      \item[right dot:] Spirit (life)
      \item[bottom dot:] One Universal God
    \end{description}
  \end{explanation}
\endsong


\beginsong{Lecha Eli}[by={Rabbi Avraham Iebn Ezra, Yair Gadassi},ex={hebrew},tags={source 1},ph={II}]
  \beginverse
    |\[Am] \[^\noteU{A}]Lecha \[^\noteU{E}]E|li | teshuka|\[G]ti
    | Becha chesh|\[Dm]ki |\[Em] ve'ahava|\[Am]ti | \e
    |\[Am] Lecha li|bi | vechilyo|\[G]tai
    | Lecha ru|\[Dm]chi |\[Em] venishma|\[Am]ti | \e
  \endverse
  \beginchorus
    \chorusindent |\[Dm] Hashive|ni va'ashu|\[G]va \up{2}(| | \e)
    \chorusindent | Vetirtzeh |\[Dm] |\[Em]et teshuva|\[Am]ti | \e
  \endchorus
  \beginverse
    |^ Lecha ya|dai | lecha rag|^lai
    | Umimach |^hee |^ techuna|^ti | \e
    |^ Lecha atz|mi | lecha da|^mi
    | Ve'ori |^im |^ geviya|^ti | \e  \gotochorus{Hashiveni}
  \endverse
  \beginchorus
    |\[Am]Oh |ho oh ho ho ho |\[G]ho | \e
    |\[Dm]Oh |\[Em]ho oh ho ho ho |\[Am]ho | \e
  \endchorus
  \beginverse
    |^ Lecha ez'|ak | becha ed|^bak
    | Adei shu|^vi |^ le'adma|^ti | \e
    |^ Lecha a|ni | be'odi |^chai
    | Ve'af ki |^a- |^ charei mo|^ti | \e  \gotochorus{Hashiveni}
  \endverse
  \begin{translation}
    For You my God is my passion
    In You is my desire and my love
    Yours are my heart and my organs
    Yours are my spirit and my soul
    \nextverse
    \chorusindent Bring me back to You and I will return
    \chorusindent And You shall want my repentance
    \nextverse
    Yours are my hands and legs
    And from You is my character
    Yours are my bones and my blood
    And my skin and my body
    \nextverse
    Oh ho oh ho ho ho ho
    \nextverse
    To You I will call and to You I will cling
    Until I return to my land
    I give myself to You whilst I still live
    And even after I die
  \end{translation}
\endsong


\beginsong{Ishq Allāh\\Love, Lover and Beloved}[by={James Burgess},tags={source 1, love 1},ex={arabic, english},ph={IV}]
  \beginchorus
    \chorusindent |\[Bm\noteULL{B}]Ishq \[\noteU{F#}]Allāh ma'|būd Allāh
    \chorusindent Ishq Al|lāh ma'\[A]būd Al|\[Bm]lāh
  \endchorus
  \beginverse
    |\[A]God is Love, |\[Bm]Lover and Beloved
    |\[A] Love, Lover and Be|\[Bm]loved
    |\[A]I am Love, |\[Bm]Lover and Beloved
    |\[A] Love, Lover and Be|\[Bm]loved
  \endverse
  \begin{explanation}
    \begin{description}
      \item[Ishq Allāh ma'būd Allāh] translates literally to ``love God adored God''
        which can be interpreted as ``God is Love and God is the Beloved'' --- and more poetically
        as ``God is Love, Lover and Beloved''.
    \end{description}
  \end{explanation}
\endsong


\beginsong{Beautiful Names of God}[tags={source 1},ex={arabic},ph={I}]
  \meter{3}{4}
  \beginverse
    \[ .\noteUL{B}]Bis\[\noteU{A}]mil|\[Am]lah, \[ .] \[ .]Al|\[C]lāh, \[ .] \[ .]Raḥ|\[G]mān, \[ .] \[ .]Ra|\[Am]ḥīm \[ .]
    \[ .]Mā|\[ .]lik, \[ .] \[ .]Qud|\[Dm]dūs, \[ .] \[ .]Sa|\[C]laām, \[ .]Mu’\[ .]min, |\[E]Muhay\[ .]min \[ .] | \[ .]\e \[ .]
    \[ .]A|\[Am]zīz, \[ .] \[ .]Jab|\[E]bār, \[ .] \[ .]Muta |\[C]kab\[Dm]bir, \[ .]Khā|\[E]liq \[ .] \[ .] | \[ .]\e \[ .]
%    % Original(?), rhythmically stranger version below:
%    \[ .\noteUL{B}]Bis\[\noteU{A}]mil|\[Am]lah, \[ .] \[ .]Al|\[C]lāh, \[ .] \[ .]Raḥ|\[G]mān, \[ .] \[ .]Ra|\[Am]ḥīm \[ .]
%    \[ .]Mā|\[ .]lik, \[ .] \[ .]Qud|\[Dm]dūs, \[ .] \[ .]Sa|\[C]laām, \[ .]Mu’\[ .]min, |\[E]Muhay\[ .]min \[ .]
%    |\[ .] \[ .]A\[Am]zīz, |\[ .] \[ .]Jab\[E]bār, |\[ .] \[ .]Muta \[C]kab|\[Dm]bir, \[ .]Khā\[E]liq | \[ .] \[ .]
  \endverse
  \begin{translation}
    \emph{In Qur'an:} Begin in the name of God, the One, Compassion, Mercy;
    Sovereign, Holy, Peace, Guarantor, Guardian; 
    Allmighty, Powerful, Tremendous, Creator
  \end{translation}
\endsong


\beginsong{Mash Allāh}[tags={you 1, source 1},ex={arabic, english},ph={IV}]
  \beginchorus
    \[\noteU{E}]Through \[\noteU{F#}]your |\[Em\noteULL{G}]eyes shines the light
    Mash Al|lāh mash Allāh
    |\[D]Wonder of \[B7]God in |\[Em]You
  \endchorus
  \beginverse
    |\[G]Mash Al|\[Am]lāh mash Allāh
    |\[D7]Mash Al|\[Em]lāh mash Allāh
    |\[G]Mash Al|\[Am]lāh mash Allāh
    |\[B7]Wonder of God in |\[Em]You
    |\[B7]Wonder of God in |\[Em]You
  \endverse
  \begin{explanation}
    \textbf{Mash Allāh} is Arabic and means ``as God willed it''. It is used to express thankfulness,
    appreciation or joy for what was just mentioned.
  \end{explanation}
\endsong


\beginsong{Asse Wana Hey Wana \\ Hey Niketi}[ex={hopílavayi, english},tags={heart 1, circle 1},ph={III, IV}]
  \beginchorus
    |\[Em\noteULL{B}]Asse \[\noteU{A}]wa\[\noteU{G}]na |\[Am\noteULL{A}]hey wana |\[D]asse wana |\[Em]hey wana
  \endchorus
  \notesoff
  \beginchorus
    |\[Em]Hey niketi |\[D]hey wana |\[Bm]hey niketi |\[Em]hey wana
  \endchorus
  \beginchorus
    |\[Em]Hey sister |\[Am]we are one, |\[D]hey brother |\[Em]we are one
  \endchorus
  \beginchorus
    |\[Em]No matter |\[D]where we're going to |\[Bm]no matter |\[Em]where we're coming from
  \endchorus
  \begin{explanation}
    \begin{description}
     \item[Wana] is a Hopi word for ``heart''. We are all connected in our hearts.
    \end{description}
  \end{explanation}
\endsong


\beginsong{Weha Ehayo}[by={Lakota},ex={lakȟótiyapi, español, english},ph={III}]
  \beginverse % 25 beats in this verse
    \chorusindent \[D\noteUL{D}]Weha eh\[.]ay\[.]o \[A\noteUL{C#}]weha eh\[.]ay\[.]o
    \chorusindent W\[.]eha e\[C]hay\[.]o \[G]weha eh\[.]ay\[.]o
    \chorusindent W\[.]eha e\[C]hay\[.]o \[G]weha \[.]eha\[D]yo! \[.] \[.] \[.] \[.] \[.] \[.] \[.]
  \endverse
  \beginverse\memorize % 36 beats in this verse
    \[D]Gran Esp\[.]írit\[.]u \[A]yo voy \[.]a ped\[.]ir, ó\[.]yem\[.]e \[.] \[.]
    A\[.]l uni\[C]vers\[.]o \[G]yo voy \[.]a ped\[.]ir, ó\[.]yem\[.]e \[.] \[.]
    Par\[.]a mi \[C]puebl\[.]o \[G]que sobrev\[.]iv\[.]a
    y\[.]o he d\[.]icho \[D]hey! \[.] \[.] \[.] \[.] \[.] \[.] \[.] \gotochorus{Weha ehayo}
  \endverse
  \beginverse
    ^Pacham^am^a ^yo voy ^a ped^ir, ó^yem^e ^ ^
    ^a Wira^coch^a ^yo voy ^a ped^ir, ó^yem^e ^ ^
    par^a mi ^puebl^o ^que siempre v^iv^a
    y^o he d^icho ^hey! \[.] \[.] \[.] \[.] \[.] \[.] \[.] \gotochorus{Weha ehayo}
  \endverse
  \beginverse
    ^Great Sp^ir^it ^I am ^going to ^plead, ^hear my ^call ^ ^
    T^o the ^univ^erse ^I am ^going to ^plead, ^hear my ^call ^ ^
    F^or the sur^viv^al ^of our p^eop^le
    ^I am s^aying ^hey! \[.] \[.] \[.] \[.] \[.] \[.] \[.] \gotochorus{Weha ehayo}
  \endverse
  \begin{explanation}
    \begin{description}
     \item[Wiracocha] is the great creator deity in the pre-Inca and Inca mythology in the Andes.
    \end{description}
  \end{explanation}
\endsong


\beginsong{He yama yo}[by={Lakota, version by Curawaka},ex={lakȟótiyapi},tags={thankfulness 1},ph={III, IV, V}]
  % See: https://www.youtube.com/watch?v=SofBQbI38PQ
  \transpose{-5}
  \beginchorus
    \[\noteU{A}]He |\[Dm\noteULL{D}]yama yo, wa\[\noteU{E}]na |\[\noteU{F}]he\[\noteU{E}]ne \[\noteU{D}]yo
    He |\[A7]yama yo, | wana hene y|\[Dm]o | \e
  \endchorus
  \beginverse
    Wa|\[Dm]hee | ya ya |\[C]na, | he he he |\[Dm]ho | \e
    Wa|\[A7]hee | he |\[B&]he he he he h|o, wa|\[Dm]hee | \e
  \endverse
\endsong


\beginsong{Mahk Jchi}[by={Pura Fé},ph={IV, V},ex={tutelo-saponi},tags={canon 1}]
  % See: http://thebirdsings.com/mahkjchi/
  % See: https://www.youtube.com/v=bOn4vIybDU8
  % See: 
  \meter{3}{4}
  \beginchorus
    \[\noteU{D}]Mahk |\[Dm\noteULL{D}]Jchi tahm |boo-\[\noteU{A}]ee
    yahm |\[G]pi-gih-dee | \e
    Mahk |\[F]Jchi tahm |\[C]boo-ee
    kahn |\[Dm]speh-wah eh-|bi
  \endchorus
  \beginverse
    Mahm-pi |\[F]wah |ho-ka yi|\[G]i nonk | \e
    tah |\[F]hond tah-|\[C]ni kih-|\[G]yee tai-yee | \e
    Ghee weh |\[F]meh |yee-tai-|\[G]yee | \e
  \endverse\glueverses\beginchorus
    Nan-ka |\[F]yaht yah |\[C]moo-ni-yeh |\[Dm]wah-jhi-seh | \e \up{2}(| | \e)
  \endchorus
  \begin{translation}
    Our hearts are full and our minds are good.
    Our ancestors come and give us strength.
    Stand tall, sing, dance and never forget who you are
    Or where you come from.
  \end{translation}
\endsong


\beginsong{Marirí}[ex={quechua},tags={protection 1},ph={I, II}]
  \beginchorus\meter{5}{4}
    \up{*}\[ .\noteUL{E}]Lu\[ .]pu|\[C\noteUL{G}]nita \[ .]supay \[ .]callam\[ .]puntay \[ .]man|\[Fmaj7\noteUDELTAX{A}{-3em}]tay m\[.\noteUL{G}]ar\[C\noteUL{E}]í
  \endchorus\glueverses
  \beginchorus
    \up{*}\[ .]Lu\[ .]pu|\[Am]nita \[ .]supay \[ .]callam\[ .]puntay \[ .]man|\[Dm]tay \[ .]mar\[Am]í \[\up{2}(.)]
  \endchorus
  \beginchorus\meter{4}{4}
    \[ .\noteUL{E}]Mari|\[C]rí \[ .]mari\[Fmaj7]rí \[ .]mari|\[Am]rí \[.] \[.] \rep{3}
  \endchorus\glueverses
  \beginverse
    \[.] |\[C] \[.] \[Fmaj7] \[.] |\[Am] \[.] \e
  \endverse\chordsoff
  \altlyr{Irapay, Ashpasuri, Bobinsana, Chiringa, Ayahuasca, Chacrunera,
    Chaliponga, Huanilla, Catahua, Chihuahuaco, Tomapende, Aguaje, Palmicha, Wicungo, Remocaspi,
    Huachumita, Iboga, Peyotito, Hongocitos, Sapotito, Boa Boa, Otorongo, Urcututo, Yanguntoro,
    Aguilita, Condorcito, Lucero, Quillaruna, Chulla Chaqui, Wiracocha, Pachamama, Taita Inti\ldots
  }
  \begin{translation}
    \up{*}\textit{Lupunita}, with the tip of my tongue I call on your power.
  \end{translation}
  \begin{explanation}
    \begin{description}
      \item[Marirí] is a powerful protecting and healing spirit that lives in \textit{yachay},
        the phlegm that contains the essence of a curandera's power. The spirit can be passed
        to another by a curandero, who regurgitates it, or by another nature spirit. It is
        nurtured by tobacco smoke. Marirí is very important in the practices of curanderos of
        the Upper Amazon.
      \item[Lupunita:] wolf
      \item[Irapay:] plant spirits
      \item[Ashpasuri:] animal spirits
      \item[Bobinsana \ldots  Sapotito:] plants
      \item[Boa Boa \ldots  Condorcito:] animals
      \item[Lucero:] stars
      \item[Quillaruna:] Moon
      \item[Chulla Chaqui \ldots Taita Inti:] deities
    \end{description}
  \end{explanation}
\endsong


\beginsong{Ayahuasca Takimuyki}[by={Don José Campos},ex={quechua, español},tags={Aya 1},ex={quechua, español},ph={II}]
  \beginchorus\memorize
    \meter{3}{4}|\[C\noteUL{E}\noteU{C}]Ayaw\[ .]aska \[ .]urqu\meter{2}{4}|\[F\noteUL{D}]man\[ .\noteUL{C}]ta \meter{3}{4}|\[Am\noteULL{A}]taki \[ .]taki\[.\noteU{C}]muy\meter{1}{4}|\[Em\noteULL{E}]ki
  \endchorus
  \notesoff
  \beginchorus
    \chorusindent \meter{3}{4}|\[Em]Chuyay \[ .]chuyay \[ .]hampi\meter{2}{4}|\[Am]kuy\[ .]niy
    \chorusindent \meter{3}{4}|\[F]Mişki \[ .]ñuñu \[ .]kurku\meter{2}{4}|\[Am]chay\[ .]paq \up{4}(\meter{3}{4}|\[.] \[.] \[.] )
    \rep{4}
  \endchorus
  \beginchorus
    \meter{3}{4}|^Ayah^uasca ^curan\meter{2}{4}|^de^ra \meter{3}{4}|^taki ^taki^muy\meter{1}{4}|^ki
  \endchorus
  \gotochorus{Chuyay chuyay}
  \beginchorus
    \meter{3}{4}|^Ayah^uasca ^luce\meter{3}{4}|^rito ^man\[ .]ta \meter{3}{4}|^taki ^taki^muy\meter{1}{4}|^ki
  \endchorus
  \gotochorus{Chuyay chuyay}
  \beginchorus
    \meter{3}{4}|^Ayah^uasca ^chacru\meter{2}{4}|^ne^ra \meter{3}{4}|^taki ^taki^muy\meter{1}{4}|^ki
  \endchorus
  \gotochorus{Chuyay chuyay}
  \beginchorus
    \meter{3}{4}|^Ayah^uasca ^pintu\meter{2}{4}|^re^ra \meter{3}{4}|^taki ^taki^muy\meter{1}{4}|^ki
  \endchorus
  \gotochorus{Chuyay chuyay}
  \begin{translation}
    Ayahuasca of the mountain \emph{\small(healer, from stars, and Chacruna, vision painter)}
    Singing, singing, I come to you
    \nextverse
    Cleanse, cleanse, our medicine
    Sweet milk for my little body
  \end{translation}
  \vspace{-0.5em} % to fit on one page
  \musicnote{alternate, simplified, rhythm: make any/every measure 3/4}
  % NOTE: Our version has the first verse in quechua, and the rest in quechua/spanish.
  % Adapted lyrics in 100% quechua, without spanishisms:
  %   Ayawaska urqumanta takitakimuyki,
  %   chuyay chuyay hanpikuyniy
  %   mişki ñuñu kurkuchaypaq.
  %   Ayawaska hanpiqniyku...
  %   Ayawaska chaska quyllurmanta...
  %   Ayawaska chaqrunallay...
  %   Ayawaska chaqru panqa...
  %   Ayawaska llimpiqchallay...
  % Adapted lyrics in 100% spanish:
  %   Ayahuasca de la montaña, te vengo cantando y cantando,
  %   limpia limpia medicina nuestra
  %   dulce leche para mi cuerpecito.
  %   Ayahuasca curandera...
  %   Ayahuasca de lucerito...
  %   Ayahuasca chacrunera...
  %   Ayahuasca hoja de chacru...
  %   Ayahuasca pinturera...
\endsong


\beginsong{Paparuy}[by={Don Aquilino Chujandama},ph={II}]
  \beginverse
    \[^\noteU{D}]Papa|\[Dm\noteULL{A}]ruy pa\[^\noteU{G}]pa|\[^\noteU{A}]ruy pa\[^\noteU{C}]pa|\[Gm\noteULL{G}]ruy pa\[^\noteU{F}]pa|\[^\noteU{G}]ruy
    Papa|\[Dm]ruy papa|ruy papa|\[C]ruy papa|\[Dm]ruy | \e
  \endverse
  \notesoff
  \beginverse
    Wayru|^ruy wayru|ruy wayru|^ruy wayru|ruy
    Wayru|^ruy wayru|ruy wayru|^ruy wayru|^ruy | \e
  \endverse
  \beginverse
    Oto|^rongo wa|waí oto|^rongo wa|waí
    Shinapuri |^kungi wa|waí iwa|^waí kawa|^waí | \e
  \endverse
  \beginverse
    Side|^rachi wa|waí Side|^rachi wa|waí
    Shinapuri |^kungi wa|waí iwa|^waí kawa|^waí | \e
  \endverse
  \beginverse
    Churi|^chiyu wa|waí churi|^chiyu wa|waí
    Shinapuri |^kungi wa|waí iwa|^waí kawa|^waí | \e
  \endverse
  \beginverse
    Tibi|^rungi wa|waí tibi|^rungi waw|aí
    Shinapuri |^kungi wa|waí iwa|^waí kawa|^waí | \e
  \endverse
  \begin{translation}
    Bird, bird\ldots
    \nextverse
    Tree, tree\ldots
    \nextverse
    Spirit of \emph{jaguar/deer/\ldots}, take me with you.
    Take me while you roam and make me part of your kingdom.
    % As the jaguar/deer/... raises it's little one; so that it walks; what are you looking?
  \end{translation}
  \begin{explanation}
    \textbf{Paparuy}: bird; \textbf{otorongo}: jaguar; \textbf{wayruruy}: tree; \textbf{churichiuy}: deer
  \end{explanation}
\endsong


\beginsong{Eskawatã kayawey}[by={pajé Agostinho Inkamuru},ex={hancha kuin},ph={III, IV}]
  \transpose{7}
  \beginchorus\memorize
    |\[Dm] \[^\noteU{A}]Nukun |mãnã Yu\[^\noteU{C}]xi|\[^\noteU{A}]bu bu \[^\noteU{G}]be\[F\noteUL{F}]tã
    |Eskawatã kaya|\[Am]wey ki|\[Dm]ki | \e
  \endchorus\glueverses
  \notesoff
  \beginchorus
    |^ Eskawatã ka|^ya, kaya|wey kaya|^wey, | kayawey ki|^ki | \e
  \endchorus
  \beginchorus
    |^ Nukun |\up{*}niwe Yuxi|bu bu be^tã
    |Eskawatã kaya|^wey ki|^ki | \e
  \endchorus\glueverses
  \beginchorus
    |^ Eskawatã ka|^ya, kaya|wey kaya|^wey, | kayawey ki|^ki | \e
  \endchorus
  \altlyr{shina, kãna, bari, ushe, vixi, yamã, shava, yura, muká\ldots}
  \begin{explanation}
    This song from the \textit{Huni Kuin (Kaxinawá)} tribe calls the spirits of nature, the elements, the sun, the moon, the stars, and \textit{Yuxibu} to bring transformation.
    \begin{description}
      \item[Eskawatã kayawey:] transformation
      \item[Yuxibu:] the creator (energy)
    \end{description}
  \end{explanation}
  % The version above is the Curawaka's version.
  % See: https://www.youtube.com/watch?v=DU64jmOPL5k
  % Lyrics for another, more original(?) version, go something like this:
  %   Nukun mana Yoxibu, yubã mana Yoxibu, mana Yoxibu bãta,
  %     eskawata kayawê Eskawata kayawá (2x)
  %   Nukun shina Yoxibu, yubã shina Yoxibu, shina Yoxibu bãta,
  %     eskawata kayawê Eskawata kayawá (2x)
  %   Nukun kãna Yoxibu, yubã kãna Yoxibu, kãna Yoxibu bãta,
  %     eskawata kayawê Eskawata kayawá (2x)
  %   Nukun bari Yoxibu, yubã bari Yoxibu, bari Yoxibu bãta,
  %     eskawata kayawê Eskawata kayawá
  %   Nukun ushe Yoxibu, yubã ushe Yoxibu, ushe Yoxibu bãta,
  %     eskawata kayawê Eskawata kayawá
  %   Nukun vixi Yoxibu, yubã vixi Yoxibu, vixi Yoxibu bãta,
  %     eskawata kayawê Eskawata kayawá
  %   Nukun yamã Yoxibu, yubã yamã Yoxibu, yamã Yoxibu bãta,
  %     eskawata kayawê Eskawata kayawá
  %   Nukun shava Yoxibu, yubã shava Yoxibu, shava Yoxibu bãta,
  %     eskawata kayawê Eskawata kayawá
  %   Nukun yura Yoxibu, yubã yura Yoxibu, yura Yoxibu bãta,
  %     eskawata kayawê Eskawata kayawá
  %   Nukun muká Yoxibu, yubã muká Yoxibu, muká Yoxibu bãta,
  %     eskawata kayawê Eskawata kayawá
  % See: https://www.youtube.com/watch?v=1xRclkh6kUg
\endsong


\beginsong{Pahuene}[ex={yawanawa},ph={II, III}]
  % See: https://www.youtube.com/watch?v=mjGxJkavh0k
  \beginchorus
    |\[C\noteUL{E}]Cura y no\[\noteU{D}]ron|\[\noteU{E}]dé |\[\noteU{E}]cura y no\[\noteU{G}]ron|\[\noteU{E}]dé
    |\[Dm\noteULL{D}]Cura y \[\noteU{E}]to\[\noteU{D}]to|\[C\noteUL{C}]to |\[Dm\noteULL{D}]cura y \[\noteU{E}]to\[\noteU{D}]to|\[C\noteUL{C}]to
  \endchorus
  \beginchorus
    |\[C\noteUL{C}]Icama\[\noteU{D}]hi |\[\noteU{E}]pa\[\noteU{D}]hue\[Am\noteULL{C}]ne \[\noteU{A}]i|\[C\noteUL{C}]pa\[\noteU{D}]hue\[\noteU{C}]ne \up{1}(| \e)
  \endchorus\glueverses
  \beginchorus
    |\[Dm\noteULL{D}]Pahue |\[C\noteUL{E}]pa\[\noteU{D}]hue\[Am\noteULL{C}]ne \[\noteU{A}]i|\[C\noteUL{C}]pa\[\noteU{D}]hue\[\noteU{C}]ne \up{2}(| \e)
  \endchorus
\endsong


\beginsong{Wacomaia}[ex={yawanawa},ph={IV}]
  % See: https://www.youtube.com/watch?v=27OtT76yyAg
  % See: https://www.youtube.com/watch?v=7RzhsVTOcEk
  % TODO: check different versions. Most (original?) seem to have more 'wacomaia' words
  %       in the first and last verses of the song, perhaps different chords, too...
  \beginchorus
    \[G\noteUL{G}]Waco\meter{4}{4}|maia, Wa\[\noteU{A}]co|\[Am\noteULL{B}]mai\[\noteU{A}]a, |\[\noteU{G}]Wa\[\noteU{A}]co\[\noteU{B}]mai|\[\noteU{A}]a,
    Waco|\[C]maia, |Wacomai\meter{2}{4}|\[Am]a, \meter{4}{4}|\[G]heé! | | \e
  \endchorus
  \beginchorus
    Wa|comai|\[Am]a, tone |pinda|\[C]ke pinda |kanarô |\[G]ho | | \e
  \endchorus
  \beginchorus
    To|zake hô pa|\[Am]rá tone |pinda|\[C]ke, pinda |kanarô |\[G]ho | | \e
  \endchorus
  \beginchorus
    Ahe, e|he, ya he, |\[Am]waya, |Wacomai|a,
    Waco|\[C]maia, |Wacomai\meter{2}{4}|\[Am]a, \meter{4}{4}|\[G]heé! | | \e
  \endchorus
  \begin{explanation}
    \begin{description}
      \item[Wacomaia] means ``joy, happiness''
    \end{description}
  \end{explanation}
\endsong


\beginsong{Kanô Kanô}[ex={yawanawa},ph={IV}]
  % See: https://www.youtube.com/watch?v=0Po_Q1Ft9aY
  % See: https://www.youtube.com/watch?v=Vrj9wXKuyDk
  % See: https://soundcloud.com/iranviene-edderv/yawanawa-kano-kano
  \beginchorus\memorize
   \[^G] \[^\noteU{G}]Ka|\[^\noteU{A}]nô \[^\noteU{G}]Ka|\[Am\noteULL{A}]nô, | | | |\[C] \[^\noteU{G}]Ka|\[^\noteU{A}]nô \[^\noteU{G}]Ka|\[G]nô | | \e
  \endchorus
  \notesoff
  \beginchorus
    He ka|nore kano|^rê, He Ka|nore Kano|rê
    He ka|nore kano|^rê, He Ka|nore Kano|^rê | | \e
  \endchorus
  \beginverse
    Mara|cá inakai|^nã Mara|cá inakai|nã
    Mara|cá inakai|^nã Mara|cá inakai|^nã | | \e
  \endverse
  \beginverse
    Mara|cá ioio|^iô Mara|cá ioio|iô
    Mara|cá ioio|^iô Mara|cá ioio|^iô | | \e
  \endverse
  \beginverse
    Io|iô ioio |^iô, Io|iô ioio |iô
    Io|iô ioio |^iô, Io|iô ioio |^iô | | \e
  \endverse
  \beginverse
    Para|ra iranoi |^rã, Para|ra iranoi |rã
    Para|ra iranoi |^rã, Para|ra iranoi |^rã | | \e
  \endverse
  \beginverse
    Ra|no rano i|^rã, Ra|no rano i|rã
    Ra|no rano i|^rã, Ra|no rano i|^rã | | \e
  \endverse
  \beginverse
    Para|ra ioiô |^iô, Para|ra ioiô |iô
    Para|ra ioiô |^iô, Para|ra ioiô |^iô | | \e
  \endverse
\endsong


\beginsong{Ide Were}[by={traditional},tags={water 1},ex={yoruba},ph={IV}]
  \beginverse
    \[\noteU{C}]Ide |\[Em\noteULL{E}]were were nita Osh|\[D]un
    Ide |\[C]were were | \e
    Ide |\[Em]were were nita Osh|\[D]un
    Ide |\[C]were were nita ya
    |\[C] Ocha kini|ba nita Osh|\[D]un
    Chek|\[G]e cheke chek|\[C]e nita |\[D]ya
    Ide |\[C]were were | | \e
  \endverse
  \begin{explanation}
    This Yoruba chant is dedicated to \textbf{Oshun}, the Goddess of Love,
    happiness and prosperity. She brings to us all the good things of life,
    and is defender of the poor and the mother of all orphans; Goddess
    Oshun brings to them their needs in this life.
    \par
    Oshun (also known as Ochun or Oxum in Latin America) is an orixá, a highly
    benevolent spirit or deity that reflects one of the expressions of God in
    the Ifa and Yoruba religions (Nigeria).
    \par
    Thought to be the most beautiful of the female Orixás. No one can resist
    her charming laugh, her graceful dancing, and her lips that taste like
    honey. She has a lush womanly figure with full hips, which suggest
    fertility and eroticism.
    \par
    She exhibits all of the attributes connected with fresh flowing water:
    lively, sparking, refreshing, vivacious. She is the \underline{Goddess
    of sweet water} and can be discovered where there is fresh water, at
    rivers, ponds, lakes, and particularly waterfalls.
    \par
    She is also a healer of the sick. Teacher, who taught the Yoruba culture,
    agriculture and mysticism. The art of divination using cowrie shells. The
    bringer of song, music and dance, healing chants and meditations taught
    to her by her father Obatala, the first of the created Orishi.
    \par
    This chant speaks of a necklace as a symbol of initiation into love.
    \par
    According to the Yoruba elders, Oshun [also Osun, Oxum] is the ``unseen
    mother present at every gathering'', because Oshun is the Yoruba
    understanding of the cosmological forces of water, moisture, and
    attraction. Therefore she is omnipresent and omnipotent. Her power is
    represented in another Yoruba scripture which reminds us that ``no one is
    an enemy to water'' and therefore everyone has need of and should respect
    and revere Oshun, as well as her followers.
    \par
    Oshun is the force of harmony. Harmony we see as beauty, feel as love,
    and experience as ecstasy. Oshun according to the ancients was the only
    female Irunmole amongst the original 16 sent from the spirit realm to
    create the world. As such, she is revered as ``Yeye'' --- the sweet mother
    of us all. When the male Irunmole attempted to subjugate Oshun due to
    her femaleness she removed her divine energy, called ase by the Yoruba,
    from the project of creating the world and all subsequent efforts at
    creation were in vain. It was not until visiting with the Supreme Being,
    Olodumare, and begging Oshun pardon under the advice of Olodumare that
    the world could continue to be created. But not before Oshun had given
    birth to a son. This son became Elegba, the great conduit of ase in the
    Universe and also the eternal and infernal trickster.
    \par
    Oshun is known as Iyalode, the ``(explicitly female) chief of the realm.''
    She is also known as Laketi, she who has ears, because of how quickly
    and effectively she answers prayers. When she possesses her followers,
    she dances, flirts and then weeps --- because no one can love her enough
    and the world is not as beautiful as she knows it could be.
  \end{explanation}
  \imagecc[1]{oshun_bw_transparent_bg_648x811px.png}%
\endsong


\beginsong{O la Mama \\ Ancient Mother}[tags={Divine Mother 1, Mother Earth 1},by={trad. African},ex={some african language, english},ph={I, II}]
  \beginchorus\memorize
    \[^\noteU{E}]O \[^\noteU{B}]la |\[Am\noteULL{A}]Mama, |\[D] wa ha su |\[Em]kola |\[Bm11/D]
    O la |\[C]Mama, |\[D] wa ha su |\[Em]wam | \e
    \vspace{1em}
    O la |\[Am]Mama, |\[D] kow wey ha |\[Em]ha ha ha |\[Bm11/D]
    O la |\[C]Mama, |\[D] ta te ka|\[Em]yee | \e
  \endchorus
  \notesoff
  \textnote{in English:}
  \beginchorus
    Ancient |^Mother, |^ I hear you |^calling |^
    Ancient |^Mother, |^ I hear your |^song | \e
    \vspace{1em}
    Ancient |^Mother, |^ I hear your |^laughter |^
    Ancient |^Mother, |^ I taste your |^tears | \e
  \endchorus
  % This second verse is a more unknown addon by somebody:
  \beginchorus
    Ancient |^Mother, |^ I feel you |^calling |^
    Ancient |^Mother, |^ I sing your |^song | \e
    \vspace{1em}
    Ancient |^Mother, |^ I share your |^laughter |^
    Ancient |^Mother, |^ I dry your |^tears | \e
  \endchorus
\endsong


\beginsong{Emamaa \\ Moder Jord}[by={trad., Tane Mahuta},ex={eesti, suomi, svenska; based on a Swedish folk song},tags={Mother Earth 1},ph={III}]
  % TODO: check song origin, find out the full swedish lyrics,
  %       and perhaps better finnish ones as well; ting?
  \textnote{eesti keeles:}
  \beginchorus\memorize
    \lrep \[^\noteU{A}]Ema |\[Dm\noteULL{D}]Maa, ema \[C]Maa, Maa|\[Dm]e\[Am]ma \rrep
    Su |\[F]sees elab \[C]ürgne |\[Dm]vägi mis |toob mul \[Am]lohu|\[Dm]tust
  \endchorus
  \notesoff
  \beginchorus
    |\[Dm]Kummar\[Am]dan su |\[Dm]ees kallis \[Am]ema
    Et |\[F]südames \[C]mind ikka |\[Dm]kannad
  \endchorus
  \textnote{suomeksi:}
  \beginchorus
    \lrep Äiti |^Maa, äiti ^Maa, Maa|^äi^ti \rrep
    Sun |^sisälläs on ^väkevä |^voima, se |minua ^lohdut|^taa
  \endchorus
  \beginchorus
    |\[Dm]Kumar\[Am]ran edes|\[Dm]säsi rakas \[Am]äiti
    Sua |\[F]sydämes\[C]säin aina |\[Dm]kannan
  \endchorus
  \textnote{på svenska:}
  \beginchorus
    \lrep Moder |^Jord, moder ^Jord, Jord|^mo^der \rrep
    Du |^glöder så ^varmt i ditt |^inre, din |hetta ^ger mig |^lust
  \endchorus
\endsong

