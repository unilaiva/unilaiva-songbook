% Songs in other languages

\scleardpage
\beginsong{Ide Were}[by={Yoruba, traditional}]
  \beginverse
    Ide |\[Em]were were nita O|\[D]shun
    Ide |\[C]were were|\[C]-
    Ide |\[Em]were were nita O|\[D]shun
    Ide |\[C]were were nita ya|
    |\[C] Ocha kini|ba nita O|\[D]shun
    Che|\[G]ke cheke che|\[C]ke nita |\[D]ya
    Ide |\[C]were were|\[C]-
  \endverse
  \begin{explanation}
    This Yoruba chant is dedicated to \textbf{Oshun}, the Goddess of Love, 
    happiness and prosperity. She brings to us all the good things of life, 
    and is defender of the poor and the mother of all orphans; Goddess 
    Ochun brings to them their needs in this life.

    Oshun (also known as Ochun or Oxum in Latin America) is an orisha, a highly 
    benevolent spirit or deity that reflects one of the expressions of God in 
    the Ifa and Yoruba religions (Nigeria). 

    Thought to be the most beautiful of the female Orixas. No one can resist 
    her charming laugh, her graceful dancing, and her lips that taste like 
    honey. She has a lush womanly figure with full hips, which suggest 
    fertility and eroticism.

    She exhibits all of the attributes connected with fresh flowing water: 
    lively, sparking, refreshing, vivacious. She is the \underline{Goddess 
    of sweet water} and can be discovered where there is fresh water, at 
    rivers, ponds, lakes, and particularly waterfalls. 

    She is also a healer of the sick. Teacher, who taught the Yoruba culture, 
    agriculture and mysticism. The art of divination using cowrie shells. The 
    bringer of song, music and dance, healing chants and meditations taught 
    to her by her father Obatala, the first of the created Orishi.

    This chant speaks of a necklace as a symbol of initiation into love. 
    
    According to the Yoruba elders, Oshun [also Osun, Oxum] is the “unseen 
    mother present at every gathering”, because Oshun is the Yoruba 
    understanding of the cosmological forces of water, moisture, and 
    attraction. Therefore she is omnipresent and omnipotent. Her power is 
    represented in another Yoruba scripture which reminds us that “no one is 
    an enemy to water” and therefore everyone has need of and should respect 
    and revere Oshun, as well as her followers.
    
    Oshun is the force of harmony. Harmony we see as beauty, feel as love, 
    and experience as ecstasy. Oshun according to the ancients was the only 
    female Irunmole amongst the original 16 sent from the spirit realm to 
    create the world. As such, she is revered as “Yeye” — the sweet mother 
    of us all. When the male Irunmole attempted to subjugate Oshun due to 
    her femaleness she removed her divine energy, called ase by the Yoruba, 
    from the project of creating the world and all subsequent efforts at 
    creation were in vain. It was not until visiting with the Supreme Being, 
    Olodumare, and begging Oshun pardon under the advice of Olodumare that 
    the world could continue to be created. But not before Oshun had given 
    birth to a son. This son became Elegba, the great conduit of ase in the 
    Universe and also the eternal and infernal trickster.
    
    Oshun is known as Iyalode, the “(explicitly female) chief of the realm.” 
    She is also known as Laketi, she who has ears, because of how quickly 
    and effectively she answers prayers. When she possesses her followers, 
    she dances, flirts and then weeps — because no one can love her enough 
    and the world is not as beautiful as she knows it could be.    
  \end{explanation}
\endsong
