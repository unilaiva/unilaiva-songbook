% Songs in other languages
% ========================
%
% The following sets the song number for the first song in this file.
% The number will automatically be incremented by one for each song.
% Please do not change this! Changing would make different versions of
% the songbook to have different numbers for the same songs, and it
% would totally mess up the selection booklets causing them to have
% wrong songs in them. (For the same reason, add new songs only to the
% end of each songs_ file.)
\setcounter{songnum}{500}


\beginsong{Ancient Aramaic Prayer}[ph={I}]
  % NOTE: try to align the translation with the prayer with \vskips. Check this
  % after style changes etc.
  \chordsoff % there are no chords
  \beginverse\justifycenter
    \vspace{3em}
    Abwûn d'bwaschmâja
    \vspace{1em}
    Nethkâdasch schmach
    \vspace{1em}
    Têtê malkuthach.
    \vspace{1em}
    Nehwê tzevjânach aikâna d'bwaschmâja af b'arha.
    \vspace{1em}
    Hawvlân lachma d'sûnkanân jaomâna.
    \vspace{1em}
    Waschboklân chaubên wachtahên aikâna\\
    daf chnân schwoken l'chaijabên.
    \vspace{1em}
    Wela tachlân l'nesjuna
    \vspace{1em}
    ela patzân min bischa.
    \vspace{1em}
    Metol dilachie malkutha wahaila wateschbuchta l'ahlâm almîn.
    \vspace{1em}
    Amên.
    \vspace{5em}
    \imagec[4]{ancient_aramaic_symbol_bw_transparent_bg_184x225px.png}
  \endverse
  \brk % to suggest putting a page break here
  \begin{translation}\justifycenter
    \vspace{-2em} % to align
    Oh Thou, from whom the breath of life comes,
    who fills all realms of sound, light and vibration.
    \vspace{1em} % to align
    May Your light be experienced in my utmost holiest.
    \vspace{1.5em} % to align
    Your Heavenly Domain approaches.
    \vspace{0.5em} % to align
    Let Your will come true --- in the universe \emph{(all that vibrates)}
    just as on Earth \emph{(that is material and dense)}.
    \vspace{0.5em} % to align
    Give us wisdom \emph{(understanding, assistance)}
    for our daily need.
    \vspace{0.5em} % to align
    Detach the fetters of faults that bind us \emph{(karma)},
    like we let go the guilt of others.
    \vspace{1em} % to align
    Let us not be lost in superficial things
    \emph{(materialism, common temptations)},
    \vspace{0.5em} % to align
    but let us be freed from that what keeps us off from
    our true purpose.
    \vspace{0.5em} % to align
    From You comes the all-working will, the lively strength to act,
    the song that beautifies all and renews itself from age to age.
    \vspace{1em} % to align
    Sealed in trust, faith and truth.
    \emph{(I confirm with my entire being.)}
  \end{translation}
  \begin{explanation}
    The symbol has been used by ancient Near Eastern scribes to indicate that
    the writing is of a sacred nature.
    \begin{description}
      \item[upper dot:] God (mind)
      \item[left dot:] Son (wisdom)
      \item[right dot:] Spirit (life)
      \item[bottom dot:] One Universal God
    \end{description}
  \end{explanation}
\endsong


\beginsong{\texorpdfstring{Wild Gazelle --- \textpersian{اهوی وحشی}}{Wild Gazelle}}
    [by={\texorpdfstring{Faramarz Aslani --- \textpersian{فرامرز اصلانی}, Hafez --- \textpersian{حافظ}}{Faramarz Aslani, Hafez}},
     ex={persian},
     key={Am},
     gk={Am, Gm--C\shrp{}m}]
  \audio[key=Em]{https://soundcloud.com/armin-poyamanesh/fymww4dufmfn}
  \meter{12}{8} %
  \capo{3}
  %TODO: needs refining!
  \beginverse
    \[\mn{B}]Äl|\[\mncii{D}{B}Em]{a e}|-\[\mn{A}]i a\[\mn{B}]huy|\[\mncii{C}{A}Am]e vähshi | - \[B7]{} \[\mn{G}]ko\[\mn{A}]ja|\[\mnc{B}Em]ii \hfill{\scriptsize\textpersian{الا ای آهوی وحشی کجائی}}
    | |\[Cmaj7]{} |\[B7]{} \e
    Mära |\[Em]ba tost | chä|\[Am]{-ndin} | - \[B7]{} ashena|\[Em]ii \hfill{\scriptsize\textpersian{مـرا با تـُست چندین آشـنائی}}
    | |\[Cmaj7]{} |\[B7]{} \e
  \endverse
  \beginverse
    \[\mn{A}]Do tän|\[Am]hao do särgar\[\mn{G}]dan \[\mn{A}]do |\[\mnc{B}Em]bi\[\mn{G}]käs \hfill{\scriptsize\textpersian{دو تنها و دو سرگردان دو بیکس}}
    Dädo |\[C]damät kämin az pisho a|\[Am]{z päs} \hfill{\scriptsize\textpersian{دد و دامت کمین از پیش و از پس}}
    Bi|\[Am]a ta hale yek digär |\[Em]bedanim \hfill{\scriptsize\textpersian{بـیا تا حـال یکدیگر بدانیم}}
    Mor|\[C]ade häm bejuiim ar tä|\[Am]vanim \hfill{\scriptsize\textpersian{مـراد هـم بجوییم اَر توانیم}}
  \endverse
  \beginverse
    \[\mn{F#}]Ke mi|\[B7]binäm ke in däshte mosh|\[\mnc{A}Em]ä\[\mn{G}]väsh \hfill{\scriptsize\textpersian{که می‌بینم که این دشت مشوش}}
    chär|\[Am]agahi näda\[B7]räd khorrä|\[Em]mo khäsh |\[B7]{} |\[Em]{} \e \hfill{\scriptsize\textpersian{چراگاهی ندارد خرم و خـَوش}}
  \endverse
  \beginverse
    \ind \[\mn{B}]Ke kha|\[Em]häd shod beguiid \[\mn{A}]ey \[\mn{B}]räf|\[\mncii{C}{A}Am]ighan \hfill{\scriptsize\textpersian{که خواهد شد بگویید ای رفیقان}}
    \ind Räfig|\[C]he bikäsan yare ghär|\[Am]iban \hfill{\scriptsize\textpersian{رفیق بیکسان یار غریبان}}
  \endverse
  \beginverse
    \ind \[\mn{A}]Mägär |\[B7]Khezre mobaräk pei |\[\mnc{B}Em]dä\[\mn{A}]ra\[\mn{G}]yd \hfill{\scriptsize\textpersian{مگر خضر مبارک پی درآید}}
    \ind ze io|\[C]mne hemmätäsh kari gosha|\[Am]{-yäd}, \[B7]{} \hfill{\scriptsize\textpersian{ز یـُمن همتش کاری گشاید}}
    \ind gosha|\[Em]{-yäd}
  \endverse
  \beginverse
    |\[Em]{} | | |\[Am]{} | |\[Em]{} |\[B7]{} \e
    |\[Cmaj7]{} |\[Am]{} |\[B7]{} |\[Em]{} | \e
  \endverse
  \beginverse
    Cho an |\[Em]särve | räva|\[Am]{-n} shod | - \[B7]{} karevan|\[Em]i \hfill{\scriptsize\textpersian{چوآن سرو روان شد کاروانی}}
    | |\[Cmaj7]{} |\[B7]{} \e
    zesh|\[Em]akhe särv | - m|\[Am]i kon | - \[B7]{} sayeb|\[Em]ani \hfill{\scriptsize\textpersian{ز شاخ سرو می‌کن سایه بانی}}
    | |\[Cmaj7]{} |\[B7]{} \e
  \endverse
  \beginverse
    Läbe s|\[Am]är cheshmeii o tärfe |\[Em]juii \hfill{\scriptsize\textpersian{لب سر چشمه‌ای و طرفِ جوئی}}
    näme |\[C]äshki o ba khod gofte|\[Am]guii \hfill{\scriptsize\textpersian{نم اشکی و با خود گفت و گوئی}}
    Be ya|\[Am]{-de} räftegan o dust|\[Em]daran \hfill{\scriptsize\textpersian{به یاد رفتگان و دوستداران}}
    mov|\[C]afegh gärd ba äbre bä|\[Am]haran \hfill{\scriptsize\textpersian{موافق گرد با ابر بهاران}}
  \endverse
  \beginverse
    Chon|\[B7]alan ayädät abe rä|\[Em]van pish \hfill{\scriptsize\textpersian{چو نالان آیدت آب روان پیش}}
    mäd|\[Am]äd bäkhshäsh ze \[B7]abe dide|\[Em]ye khish \hfill{\scriptsize\textpersian{مدد بخشش ز آب دیدۀ خویش}}
    |\[B7]{} |\[Em]{} \e
  \endverse
  \beginverse
    \ind \[\mn{B}]Näkärd |\[Em]an hamdame dir\[\mn{A}]in \[\mn{B}]mo|\[\mnc{C}Am]da\[\mn{A}]ra \hfill{\scriptsize\textpersian{نکرد آن همدم دیرین مدارا}}
    \ind mosä|lmanan, mosä\[B7]lmanan khod|\[Em]ara \hfill{\scriptsize\textpersian{مسلمانان مسلمانان خدا را}}
  \endverse
  \beginverse
    \ind \[\mn{G}]Mägä|\[C]r Khezre mobaräk pey \[\mn{A}]tä|\[\mnc{B}Am]va\[\mn{A}]näd \[B7]{} \hfill{\scriptsize\textpersian{مگر خضر مبارک ‌پی تواند}}
    \ind Ke in |\[Am]tänha be an \[B7]tänha res|\[Em]anäd \hfill{\scriptsize\textpersian{که این تنها به آن تنها رساند}}
  \endverse
  \beginchorus
    |\[Cmaj7]{} |\[Am]{} \[B7]{} |\[Am]{} \[B7]{} |\[Em]
  \endchorus
  \begin{translation}
    O wild gazelle, where can I find you?
    We have lots in common, me and you!
    \nextverse
    We are both alone, wanderers, we both have no one
    In behind and before, beasts and traps ambush [begun]
    \nextverse
    Let's ask each other, ``how do you feel?''
    Grant each other's wishes, if we can deal
    \nextverse
    Alas! As far as I can see, this troubled meadow
    Has no place to graze happily, free from sorrow
    \nextverse
    O friends! Let me know who [on earth] will be the one
    to help outcasts, to company the ones having no one
    \nextverse
    May green feet \emph{Khizr} come [strolling]
    And his mighty will get things rolling
    \nextverse
    As that fleeting tall as cedar [beauty] did get going
    Let's shelter [from longing] by the cedar tree's twig
    \nextverse
    Let's sit by a spring fountain, next to a brook
    Let's have moist in eyes, to myself let me talk
    \nextverse
    Let's recall memories of the bygone [people], of the friendlier
    And the spring cloud will accompany us in tear
    \nextverse
    As the running wailing water gets close by
    Give it help by the [running] water of your eye
    \nextverse
    She didn't get along with me, that old fond friend!
    [Help,] O people of faith! Help! For the love of God!
    \nextverse
    No one but the green feet Khizr can be the one
    To unite this lonely one with the other lonely one
  \end{translation}
  \begin{explanation}
    Lyrics for this song are based on a poem by the famous poet
    \textpersian{حافظ} (Hafez).\par\vspace{2ex}
    \textbf{Khizr} is a figure described, but not mentioned by name, in the
    Quran as a righteous servant of God possessing great wisdom or mystic
    knowledge. The miracle of Khizr was, that if he did sit on a dry
    land, green vegetation would appear on the ground beneath him and that
    land would become green. The name of the prophet means ``green'' in
    Arabic.
    \vfill
    \begin{center}
      \tiny Original English translation by BlueBird under Creative Commons
      License (\href{https://creativecommons.org/licenses/by/4.0/}{CC BY 4.0}).
    \end{center}
  \end{explanation}
\endsong


\beginsong{Beautiful Names of God}[tags={source},ex={arabic},ph={I}]
  \meter{3}{4}
  \beginverse
    \[\bmc \mnc{B}]Bis\[\mn{A}]mil|\[\bmc Am]lah, \[\bm]{} \[\bm\mn{C}]Al|\[\bmc C]lāh, \[\bm]{} \[\bm]Raḥ|\[\bmc\mnc{B}G]mān, \[\bm]{} \[\bm\mn{G}\mn{A}]Ra|\[\bmc Am]ḥīm \[\bm]
    \[\bm]Mā|\[\bm]lik, \[\bm]{} \[\bm]Qud|\[\bmc Dm]dūs, \[\bm]{} \[\bm]Sa|\[\bmc C]laām, \[\bm]Mu’\[\bm]min, |\[\bmc E]Muhay\[\bm]min \[\bm]{} | \[\bm]{} \[\bm]{} \e
    \[\bm]A|\[\bmc Am]zīz, \[\bm]{} \[\bm]Jab|\[\bmc E]bār, \[\bm]{} \[\bm]Muta |\[\bmc C]kab\[\bmc Dm]bir, \[\bm]Khā|\[\bmc E]liq \[\bm]{} \[\bm]{} | \[\bm]{} \[\bm]{} \e
%    % Original(?), rhythmically stranger version below:
%    \[\bmc \mnc{B}]Bis\[\mn{A}]mil|\\bmc [Am]lah, \[\bm]{} \[\bm]Al|\[\bmc C]lāh, \[\bm]{} \[\bm]Raḥ|\[\bmc G]mān, \[\bm]{} \[\bm]Ra|\[\bmc Am]ḥīm \[\bm]
%    \[\bm]Mā|\[\bm]lik, \[\bm]{} \[\bm]Qud|\[\bmc Dm]dūs, \[\bm]{} \[\bm]Sa|\[\bmc C]laām, \[\bm]Mu’\[\bm]min, |\[\bmc E]Muhay\[\bm]min \[\bm]
%    |\[\bm]{} \[\bm]A\[\bmc Am]zīz, |\[\bm]{} \[\bm]Jab\[\bmc E]bār, |\[\bm]{} \[\bm]Muta \[\bmc C]kab|\[\bmc Dm]bir, \[\bm]Khā\[\bmc E]liq | \[\bm]{} \[\bm]{} \e
  \endverse
  \begin{translation}
    \emph{In Qur'an:} Begin in the name of God, the One, Compassion, Mercy;
    Sovereign, Holy, Peace, Guarantor, Guardian; 
    Allmighty, Powerful, Tremendous, Creator
  \end{translation}
\endsong


\beginsong{Ishq Allāh\\Love, Lover and Beloved}[by={James Burgess},tags={source, love},ex={arabic, english},ph={IV}]
  \beginchorus
    \ind |\[\mnc{B}Bm]Ishq \[\mn{F#}]Allāh ma'|\[\mn{B}]būd \[\mn{F#}]Allāh
    \ind Ishq Al|lāh ma'\[A]būd Al|\[Bm]lāh
  \endchorus
  \beginverse
    |\[A]God is Love, |\[Bm]Lover and Beloved
    |\[A] Love, Lover and Be|\[Bm]loved
    |\[A]I am Love, |\[Bm]Lover and Beloved
    |\[A] Love, Lover and Be|\[Bm]loved
  \endverse
  \begin{explanation}
    \begin{description}
      \item[Ishq Allāh ma'būd Allāh] translates literally to ``love God adored God''
        which can be interpreted as ``God is Love and God is the Beloved'' --- and more poetically
        as ``God is Love, Lover and Beloved''.
    \end{description}
  \end{explanation}
\endsong


\beginsong{Mash Allāh}[tags={you, source},ex={arabic, english},ph={IV}]
  \beginchorus
    \[\mn{E}]Through \[\mn{F#}]your |\[\mnc{G}Em]eyes shines the light
    Mash Al|lāh mash Allāh
    |\[D]Wonder of \[B7]God in |\[Em]You
  \endchorus
  \beginverse
    |\[G]Mash Al|\[Am]lāh mash Allāh
    |\[D7]Mash Al|\[Em]lāh mash Allāh
    |\[G]Mash Al|\[Am]lāh mash Allāh
    |\[B7]Wonder of God in |\[Em]You
    |\[B7]Wonder of God in |\[Em]You
  \endverse
  \begin{explanation}
    \textbf{Mash Allāh} is Arabic and means ``as God willed it''. It is used to express thankfulness,
    appreciation or joy for what was just mentioned.
  \end{explanation}
\endsong


\beginsong{Shalom Aleichem}[ph={I, II}]
  \audio[key={Dm}]{https://www.youtube.com/watch?v=913jZFL1bdE}
  \beginverse
    |\[\mnc{A}Dm]Sha\[\mn{F}]lom \[\mn{E}]a\[\mn{D}]leichem |\[\mnc{C#}A7]mal'\[\mn{D}]a\[\mn{E}\mn{D}]chei \[\mn{C#}]has\[\mn{B&}]ha\[\mn{A}]lom
    |\[B&]Mal'achei e|\[A7]lyon
    |\[Dm]Mimelech |\[A7]malchei hamelachim
    Ha-|\[Gm]kadosh baruch |\[A7]hu
  \endverse
  \beginverse
    |\[\mnc{F}F]{\up{*}(Bo'a}\[\mn{A}]chem) l'shalom |\[\mnc{G}C]mal'\[\mn{F}]a\[\mn{E}]chei \[\mn{F}]has\[\mn{G}\mn{A}]ha\[\mn{G}]lom
    |\[B&]Mal'achei e|\[A7]lyon
    |\[Gm]Mimelech |\[A7]malchei ha\[Dm]melachim
    Ha-|\[B&]kadosh \[A7]baruch |\[Dm]hu
  \endverse
  \altlyr{Barchuni, Tseitchem}
  \begin{translation}
    Peace be unto you, ye ministeri​​​​ng Angels, Angels of the
    most High, ye that come from the Supreme King of Kings,
    the Holy One, blessed be He.
    \nextverse
    May your coming be in peace, ye ministeri​​​​ng Angels\ldots
    \nextverse
    Bless​ me with peace, ye ministeri​​​​ng Angels\ldots
    \nextverse
    Go ye forth in peace, ye ministeri​​​​ng Angels\ldots
  \end{translation}
\endsong


\beginsong{Lecha Eli}[by={Rabbi Avraham Iebn Ezra, Yair Gadassi},ex={hebrew},tags={source},ph={II}]
  \audio[]{https://soundcloud.com/shimonlevtahor/vimala-zohar-gad-lecha-eli}
  \beginverse
    |\[Am]{} \[^\mn{A}]Lecha \[^\mn{E}]E|li | \[^\mn{A}]teshu\[^\mn{E}]ka|\[G\mn{D}\mn{C}\mn{D}]ti
    | Becha chesh|\[Dm]ki |\[Em]{} ve'ahava|\[Am]ti | \e
    |\[Am]{} Lecha li|bi | vechilyo|\[G]tai
    | Lecha ru|\[Dm]chi |\[Em]{} venishma|\[Am]ti | \e
  \endverse
  \beginchorus
    \ind |\[Dm]{} Hashive|ni va'ashu|\[G]va \up{2}(| | \e)
    \ind | Vetirtzeh |\[Dm]{} |\[Em]et teshuva|\[Am]ti | \e
  \endchorus
  \notesoff
  \beginverse
    |^ Lecha ya|dai | lecha rag|^lai
    | Umimach |^hee |^ techuna|^ti | \e
    |^ Lecha atz|mi | lecha da|^mi
    | Ve'ori |^im |^ geviya|^ti | \e  \goto{Hashiveni}
  \endverse
  \beginchorus
    |\[Am]Oh |ho oh ho ho ho |\[G]ho | \e
    |\[Dm]Oh |\[Em]ho oh ho ho ho |\[Am]ho | \e
  \endchorus
  \beginverse
    |^ Lecha ez'|ak | becha ed|^bak
    | Adei shu|^vi |^ le'adma|^ti | \e
    |^ Lecha a|ni | be'odi |^chai
    | Ve'af ki |^a- |^ charei mo|^ti | \e  \goto{Hashiveni}
  \endverse
  \begin{translation}
    For You my God is my passion, in You is my desire and my love
    Yours are my heart and my organs, tours are my spirit and my soul
    \nextverse
    \ind Bring me back to You and I will return
    \ind And You shall want my repentance
    \nextverse
    Yours are my hands and legs, and from You is my character
    Yours are my bones and my blood, and my skin and my body
    \nextverse
    Oh ho oh ho ho ho ho
    \nextverse
    To You I will call and to You I will cling, until I return to my land
    I give myself to You whilst I still live, and even after I die
  \end{translation}
\endsong


\beginsong{Lev Tahor}[ph={II, III}, key={Am}, gk={Bm, Am--Em}]
  \audio[key=Cm]{https://soundcloud.com/bettinamaureenji/lev-tahor}
  % in Am the notes range from G to C' (or E' if going the high route :))
  \beginverse
    \[\mn{A}]Lev \[\mn{B}]ta|\[\mnc{C}C]ho|\[\mnc{B}G]r b'ra \[\mn{G}]li \[\mn{B}]E\[\mn{C}]lo|\[\mnc{A}Am]him; | \e \altchords{\id[1]{(Bm)}|D |A |Bm | \e}
    v'ruach na|\[C]cho|\[G]n chadesh b'ki|\[Am]rbi. | \e \altchords{|D |A |Bm | \e}
    \vspace{.5em}
    Al tashli|\[F]cheni|\[G]{} mil'fa|\[Am]necha; | \e \altchords{|G |A |Bm | \e}
    v'ruach kodshe|\[C]cha|\[G]{} al tikach mi|\[Am]meni. | \e \altchords{|D |A |Bm | \e}
  \endverse
  \imagerc[3]{lev_tahor_hebrew_script_bw_transparent_bg_1477x300px.png}%
  \begin{translation}
    Create me a clean heart, O God;
    and renew a stedfast spirit within me.
    \nextverse
    Cast me not away from Thy presence;​
    and take not Thy holy spirit from me.
  \end{translation}
\endsong


\beginsong{Asse Wana Hey Wana \\ Hey Niketi}[ex={hopílavayi, english},tags={heart, circle},ph={III, IV}]
  \audio[]{https://www.youtube.com/watch?v=WsEoKmEGe18}
  \beginchorus
    |\[\mnc{G}Em]Asse \[\mn{F#}]wa\[\mn{E}]na |\[\mnc{A}Am]hey wana |\[\mnc{F#}D]asse \[\mn{E}]wa\[\mn{D}]na |\[\mnc{E}Em]hey wana
  \endchorus
  \beginchorus
    \ind |\[\mnc{B}Em]Hey ni\[\mn{A}]ke\[\mn{G}]ti |\[\mnc{A}D]hey wana |\[Bm]hey ni\[\mn{G}]ke\[\mn{F#}]ti |\[\mnc{G}Em]hey \[\mn{E}]wana
  \endchorus
  \notesoff
  \beginchorus
    |\[Em]Hey sister |\[Am]we are one, |\[D]hey brother |\[Em]we are one
  \endchorus
  \beginchorus
    \ind |\[Em]No matter |\[D]where we're going to |\[Bm]no matter |\[Em]where we're coming from
  \endchorus
  \begin{explanation}
    \begin{description}
     \item[Wana] is a Hopi word for ``heart''. We are all connected in our hearts.
    \end{description}
  \end{explanation}
\endsong


\beginsong{Weha Ehayo}[by={Lakota},ex={lakȟótiyapi, español, english},ph={III}]
  \beginverse % 25 beats in this verse
    \ind \[\bmc\mnc{D}D]Weha eh\[\bmadj{-.5ex}]ay\[\bmadj{-.5ex}]o \[\bmc\mnc{C#}A]weha eh\[\bmadj{-.5ex}]ay\[\bmadj{-.5ex}]o
    \ind W\[\bmadj{-.5ex}]eha e\[\bmc C]hay\[\bmadj{-.5ex}]o \[\bmc G]weha eh\[\bmadj{-.5ex}]ay\[\bmadj{-.5ex}]o
    \ind W\[\bmadj{-.5ex}]eha e\[\bmc C]hay\[\bmadj{-.5ex}]o \[\bmc G]weha \[\bmadj{-.5ex}]eha\[\bmc D]yo! \[\bm]{} \[\bm]{} \[\bm]{} \[\bm]{} \[\bm]{} \[\bm]{} \[\bm]
  \endverse
  \beginverse\memorize % 36 beats in this verse
    \[\bmc D]Gran Esp\[\bmadj{-.5ex}]írit\[\bmadj{-.5ex}]u \[\bmc A]yo voy \[\bm]a ped\[\bmadj{-.5ex}]ir, ó\[\bmadj{-.5ex}]yem\[\bmadj{-.5ex}]e \[\bm]{} \[\bm]
    A\[\bmadj{-.5ex}]l uni\[\bmc C]vers\[\bmadj{-.5ex}]o \[\bmc G]yo voy \[\bm]a ped\[\bmadj{-.5ex}]ir, ó\[\bmadj{-.5ex}]yem\[\bmadj{-.5ex}]e \[\bm]{} \[\bm]
    Par\[\bmadj{-.5ex}]a mi \[\bmc C]puebl\[\bmadj{-.5ex}]o \[\bmc G]que sobrev\[\bmadj{-.5ex}]iv\[\bmadj{-.5ex}]a
    y\[\bmadj{-.5ex}]o he d\[\bmadj{-.5ex}]icho \[\bmc D]hey! \[\bm]{} \[\bm]{} \[\bm]{} \[\bm]{} \[\bm]{} \[\bm]{} \[\bm]{} \goto{Weha ehayo}
  \endverse
  \beginverse
    ^Pacham^am^a ^yo voy ^a ped^ir, ó^yem^e ^ ^
    ^a Wira^coch^a ^yo voy ^a ped^ir, ó^yem^e ^ ^
    par^a mi ^puebl^o ^que siempre v^iv^a
    y^o he d^icho ^hey! \[\bm]{} \[\bm]{} \[\bm]{} \[\bm]{} \[\bm]{} \[\bm]{} \[\bm]{} \goto{Weha ehayo}
  \endverse
  \beginverse
    ^Great Sp^ir^it ^I am ^going to p^lead, h^ear my c^all ^ ^
    T^o the ^univ^erse ^I am ^going to p^lead, h^ear my c^all ^ ^
    F^or the sur^viv^al ^of our p^eop^le
    ^I am s^aying ^hey! \[\bm]{} \[\bm]{} \[\bm]{} \[\bm]{} \[\bm]{} \[\bm]{} \[\bm]{} \goto{Weha ehayo}
  \endverse
  \begin{explanation}
    \begin{description}
     \item[Wiracocha] is the great creator deity in the pre-Inca and Inca mythology in the Andes.
    \end{description}
  \end{explanation}
\endsong


\beginsong{He yama yo}[by={Lakota, version by Curawaka}, ex={lakȟótiyapi}, tags={thankfulness}, ph={III, IV, V}, key={Dm}, gk={Cm, Cm--Em}]
  \audio[key=Dm]{https://www.youtube.com/watch?v=SofBQbI38PQ}
  % In Dm the notes go from A to A'
  \beginchorus
    \[\mn{A}]He |\[\mnc{D}Dm]yama yo, wa\[\mn{E}]na |\[\mn{F}]he\[\mn{E}]ne \[\mn{D}]yo \altchords{\id[1]{(Am)}|Am | \e}
    He |\[A7]yama yo, | wana hene y|\[Dm]o | \e \altchords{|E7 | - |Am | \e}
  \endchorus
  \beginverse
    \[\mn{A}]Wa|\[\mnc{A}Dm]hee | \[\mn{G}]ya \[\mn{F}]ya |\[\mnc{G}C]na, | he \[\mn{F}]he \[\mn{E}]he |\[\mnc{F}Dm]ho | \e \altchords{|Am | - |G | - |Am | \e}
    Wa|\[A7]hee | he |\[B&]he he he he h|o, wa|\[Dm]hee | \e \altchords{|E7 | - |F | - |Am | \e}
  \endverse
\endsong


\beginsong{Mahk Jchi}[by={Pura Fé}, ph={IV, V}, ex={tutelo-saponi}, tags={canon}, key={Am}, gk={Am, Am--Bm}]
  \audio[]{http://thebirdsings.com/mahkjchi/}
  \audio[]{https://www.youtube.com/v=bOn4vIybDU8}
  \meter{3}{4}
  \transpose {-5} % Transpose to Am, where notes go from G to A'
  \beginchorus
    \[\mn{D}]Mahk |\[\mnc{D}Dm]Jchi tahm |boo-\[\mn{A}]ee
    yahm |\[G]pi-gih-dee | \e
    Mahk |\[F]Jchi tahm |\[C]boo-ee
    kahn |\[Dm]speh-wah eh-|bi
  \endchorus
  \beginverse
    Mahm-pi |\[F]wah |ho-ka yi|\[G]i nonk | \e
    tah |\[F]hond tah-|\[C]ni kih-|\[G]yee tai-yee | \e
    Ghee weh |\[F]meh |yee-tai-|\[G]yee | \e
  \endverse\glueverses\beginchorus
    Nan-ka |\[F]yaht yah |\[C]moo-ni-yeh |\[Dm]wah-jhi-seh | \e \up{2}(| | \e)
  \endchorus
  \begin{translation}
    Our hearts are full and our minds are good.
    Our ancestors come and give us strength.
    Stand tall, sing, dance and never forget who you are
    Or where you come from.
  \end{translation}
\endsong


\beginsong{Marirí}[ex={quechua},tags={protection},ph={I, II}]
  \beginchorus\meter{5}{4}
    \up{*}\[\bmc\mnc{E}]Lu\[\bm]pu|\[\bmc\mnc{G}C]nita \[\bm]supay \[\bm]callam\[\bm]puntay \[\bm]man|\[\bmc\mnc{A}Fmaj7]tay \[\bmc\mnc{G}]mar\[\bmc\mnc{E}C]í
  \endchorus\glueverses
  \beginchorus
    \up{*}\[\bm]Lu\[\bm]pu|\[\bmc Am]nita \[\bm]supay \[\bm]callam\[\bm]puntay \[\bm]man|\[\bmc Dm]tay \[\bm]mar\[\bmc Am]í
  \endchorus
  \beginchorus\meter{4}{4}
    \[\bm]{} \[\bmc\mnc{E}]Mari|\[\bmc C]rí \[\bm]mari\[\bmc Fmaj7]rí \[\bm]mari|\[\bmc Am]rí \[\bm]\rep{3}
  \endchorus\glueverses
  \beginverse
    \[\bm]{} \[\bm]{} |\[\bmc C]{} \[\bm]{} \[\bmc Fmaj7]{} \[\bm]{} |\[\bmc Am]{} \[\bm]{} \e
  \endverse\chordsoff
  \altlyr{Irapay, Ashpasuri, Bobinsana, Chiringa, Ayahuasca, Chacrunera,
    Chaliponga, Huanilla, Catahua, Chihuahuaco, Tomapende, Aguaje, Palmicha, Wicungo, Remocaspi,
    Huachumita, Iboga, Peyotito, Hongocitos, Sapotito, Boa Boa, Otorongo, Urcututo, Yanguntoro,
    Aguilita, Condorcito, Lucero, Quillaruna, Chulla Chaqui, Wiracocha, Pachamama, Taita Inti\ldots
  }
  \begin{translation}
    \up{*}\textit{Lupunita}, with the tip of my tongue I call on your power.
  \end{translation}
  \begin{explanation}
    \begin{description}
      \item[Marirí] is a powerful protecting and healing spirit that lives in \textit{yachay},
        the phlegm that contains the essence of a curandera's power. The spirit can be passed
        to another by a curandero, who regurgitates it, or by another nature spirit. It is
        nurtured by tobacco smoke. Marirí is very important in the practices of curanderos of
        the Upper Amazon.
      \item[Lupunita:] wolf
      \item[Irapay:] plant spirits
      \item[Ashpasuri:] animal spirits
      \item[Bobinsana \ldots  Sapotito:] plants
      \item[Boa Boa \ldots  Condorcito:] animals
      \item[Lucero:] stars
      \item[Quillaruna:] Moon
      \item[Chulla Chaqui \ldots Taita Inti:] deities
    \end{description}
  \end{explanation}
\endsong


\beginsong{Ayahuasca Takimuyki}[by={Don José Campos},ex={quechua, español},tags={Aya},ex={quechua, español},ph={II},key={Bm},gk={Cm, (Bm)--Cm--C\shrp{}m}]
  % in Am the notes range from E to G'
  % in Bm the notes range from F# to A'
  \transpose{2}
  \beginchorus\memorize
    \meter{3}{4}|\[\bmc\mncii{E}{C}C]Ayaw\[\bm]aska \[\bm]urqu\meter{2}{4}|\[\bmc\mncii{D}{C}F]man\[\bm]ta \meter{3}{4}|\[\bmc\mnc{A}Am]taki \[\bm]taki\[\bmc\mnc{C}]muy\meter{1}{4}|\[\bmc\mnc{E}Em]ki \altchords{\id[1]{(Am)}|C . . |F . |Am . . |Em}
  \endchorus
  \mnbeginchorus\rep{4}
    \ind \meter{3}{4}|\[\bmc\mnc{E}Em]Chuyay \[\bm]chuyay \[\bmc\mn{G}]ham\[\mn{E}]pi\meter{2}{4}|\[\bmc\mncii{D}{C}Am]kuy\[\bm]niy\altchords{|Em . . |Am . }
    \ind \meter{3}{4}|\[\bmc\mnc{A}F]Mişki \[\bm]ñuñu \[\bm]kur\[\mn{G}]ku\meter{2}{4}|\[\bmc\mnc{A}Am]chay\[\bm]paq \up{4}(\meter{3}{4}|\[\bm]{} \[\bm]{} \[\bm]{} ) \altchords{|F . . |Am . \up{4}(| . . .)}
  \mnendchorus
  \notesoff
  \beginchorus
    \meter{3}{4}|^Ayah^uasca ^curan\meter{2}{4}|^de^ra \meter{3}{4}|^taki ^taki^muy\meter{1}{4}|^ki
  \endchorus
  \goto{Chuyay chuyay}
  \beginchorus
    \meter{3}{4}|^Ayah^uasca ^luce\meter{3}{4}|^rito ^man\[\bm]ta \meter{3}{4}|^taki ^taki^muy\meter{1}{4}|^ki
  \endchorus
  \goto{Chuyay chuyay}
  \beginchorus
    \meter{3}{4}|^Ayah^uasca ^chacru\meter{2}{4}|^ne^ra \meter{3}{4}|^taki ^taki^muy\meter{1}{4}|^ki
  \endchorus
  \goto{Chuyay chuyay}
  \beginchorus
    \meter{3}{4}|^Ayah^uasca ^pintu\meter{2}{4}|^re^ra \meter{3}{4}|^taki ^taki^muy\meter{1}{4}|^ki
  \endchorus
  \goto{Chuyay chuyay}
  \begin{translation}
    Ayahuasca of the mountain \emph{\small(healer, from stars, and Chacruna, vision painter)}
    Singing, singing, I come to you
    \nextverse
    Cleanse, cleanse, our medicine
    Sweet milk for my little body
  \end{translation}
  \vfill%
  \musicnotefornext{alternate, simplified, rhythm: make any/every measure 3/4}
  \noendsongvfill
  % NOTE: Our version has the first verse in quechua, and the rest in quechua/spanish.
  % Adapted lyrics in 100% quechua, without spanishisms:
  %   Ayawaska urqumanta takitakimuyki,
  %   chuyay chuyay hanpikuyniy
  %   mişki ñuñu kurkuchaypaq.
  %   Ayawaska hanpiqniyku...
  %   Ayawaska chaska quyllurmanta...
  %   Ayawaska chaqrunallay...
  %   Ayawaska chaqru panqa...
  %   Ayawaska llimpiqchallay...
  % Adapted lyrics in 100% spanish:
  %   Ayahuasca de la montaña, te vengo cantando y cantando,
  %   limpia limpia medicina nuestra
  %   dulce leche para mi cuerpecito.
  %   Ayahuasca curandera...
  %   Ayahuasca de lucerito...
  %   Ayahuasca chacrunera...
  %   Ayahuasca hoja de chacru...
  %   Ayahuasca pinturera...
\endsong


\beginsong{Paparuy}[by={Don Aquilino Chujandama},ph={II},key={Em (2 oct)},gk={Bm, Am--C\shrp{}m}]
  \audio[key=Dm]{https://soundcloud.com/ayahuapu/paparuy-en-vivo}
  \audio[key=Em]{https://www.youtube.com/watch?v=iZljniFESak}
  % in Dm the notes range from C to C'
  % in Em the notes range from D to D'
  % in Am the notes range from G to G'
  % in Bm the notes range from A to A'
  \transpose{2} % to Em for singing in two octaves
  \mnbeginverse
    \[^\mn{D}]Papa|\[\mnc{A}Dm]ruy pa\[^\mn{G}\mn{A}]pa|ruy pa\[^\mn{C}]pa|\[\mnc{G}Gm]ruy pa\[^\mn{F}\mn{G}]pa|ruy \altchords{\id[1]{(Bm)}|Bm | - |Em | \e}
    \[^\mn{D}]Papa|\[\mnc{F}Dm]ruy \[^\mn{D}]pa\[^\mn{C}\mn{D}]pa|ruy papa|\[\mnc{E}C]ruy \[^\mn{F}]pa\[^\mn{E}\mn{D}]pa|\[Dm]ruy | \e \altchords{|Bm | - |A |Bm | \e}
  \mnendverse
  \notesoff
  \beginverse
    Wayru|^ruy wayru|ruy wayru|^ruy wayru|ruy
    Wayru|^ruy wayru|ruy wayru|^ruy wayru|^ruy | \e
  \endverse
  \beginverse
    Oto|^rongo wa|waí oto|^rongo wa|waí
    Shinapuri |^kungi wa|waí iwa|^waí kawa|^waí | \e
  \endverse
  \beginverse
    Side|^rachi wa|waí Side|^rachi wa|waí
    Shinapuri |^kungi wa|waí iwa|^waí kawa|^waí | \e
  \endverse
  \beginverse
    Churi|^chiyu wa|waí churi|^chiyu wa|waí
    Shinapuri |^kungi wa|waí iwa|^waí kawa|^waí | \e
  \endverse
  \beginverse
    Tibi|^rungi wa|waí tibi|^rungi waw|aí
    Shinapuri |^kungi wa|waí iwa|^waí kawa|^waí | \e
  \endverse
  \begin{translation}
    Bird, bird\ldots
    \nextverse
    Tree, tree\ldots
    \nextverse
    Spirit of \emph{jaguar/deer/\ldots}, take me with you.
    Take me while you roam and make me part of your kingdom.
    % As the jaguar/deer/... raises it's little one; so that it walks; what are you looking?
  \end{translation}
  \begin{explanation}
    \textbf{Paparuy}: bird; \textbf{otorongo}: jaguar; \textbf{wayruruy}: tree; \textbf{churichiuy}: deer
  \end{explanation}
\endsong


\beginsong{Eskawatã kayawey}[by={pajé Agostinho Inkamuru},ex={hancha kuin},ph={III, IV}]
  \transpose{7}
  \beginchorus\memorize
    |\[Dm]{} \[^\mn{A}]Nukun |mãnã Yu\[^\mn{C}]xi|\[^\mn{A}]bu bu \[^\mn{G}]be\[\mnc{F}F]tã
    |Eskawatã kaya|\[Am]wey ki|\[Dm]ki | \e
  \endchorus\glueverses
  \notesoff
  \beginchorus
    |^ Eskawatã ka|^ya, kaya|wey kaya|^wey, | kayawey ki|^ki | \e
  \endchorus
  \beginchorus
    |^ Nukun |\up{*}niwe Yuxi|bu bu be^tã
    |Eskawatã kaya|^wey ki|^ki | \e
  \endchorus\glueverses
  \beginchorus
    |^ Eskawatã ka|^ya, kaya|wey kaya|^wey, | kayawey ki|^ki | \e
  \endchorus
  \altlyr{shina, kãna, bari, ushe, vixi, yamã, shava, yura, muká\ldots}
  \begin{explanation}
    This song from the \textit{Huni Kuin (Kaxinawá)} tribe calls the spirits of nature, the elements, the sun, the moon, the stars, and \textit{Yuxibu} to bring transformation.
    \begin{description}
      \item[Eskawatã kayawey:] transformation
      \item[Yuxibu:] the creator (energy)
    \end{description}
  \end{explanation}
  % The version above is the Curawaka's version.
  \audio[]{https://www.youtube.com/watch?v=DU64jmOPL5k}
  % Lyrics for another, more original(?) version, go something like this:
  %   Nukun mana Yoxibu, yubã mana Yoxibu, mana Yoxibu bãta,
  %     eskawata kayawê Eskawata kayawá (2x)
  %   Nukun shina Yoxibu, yubã shina Yoxibu, shina Yoxibu bãta,
  %     eskawata kayawê Eskawata kayawá (2x)
  %   Nukun kãna Yoxibu, yubã kãna Yoxibu, kãna Yoxibu bãta,
  %     eskawata kayawê Eskawata kayawá (2x)
  %   Nukun bari Yoxibu, yubã bari Yoxibu, bari Yoxibu bãta,
  %     eskawata kayawê Eskawata kayawá
  %   Nukun ushe Yoxibu, yubã ushe Yoxibu, ushe Yoxibu bãta,
  %     eskawata kayawê Eskawata kayawá
  %   Nukun vixi Yoxibu, yubã vixi Yoxibu, vixi Yoxibu bãta,
  %     eskawata kayawê Eskawata kayawá
  %   Nukun yamã Yoxibu, yubã yamã Yoxibu, yamã Yoxibu bãta,
  %     eskawata kayawê Eskawata kayawá
  %   Nukun shava Yoxibu, yubã shava Yoxibu, shava Yoxibu bãta,
  %     eskawata kayawê Eskawata kayawá
  %   Nukun yura Yoxibu, yubã yura Yoxibu, yura Yoxibu bãta,
  %     eskawata kayawê Eskawata kayawá
  %   Nukun muká Yoxibu, yubã muká Yoxibu, muká Yoxibu bãta,
  %     eskawata kayawê Eskawata kayawá
  \audio[]{https://www.youtube.com/watch?v=1xRclkh6kUg}
\endsong

\sclearpage % to put on the same page with the next song (yawanawa, too)
\beginsong{Pahuene}[ex={yawanawa},ph={II, III}]
  \audio[]{https://www.youtube.com/watch?v=mjGxJkavh0k}
  \beginchorus
    |\[\mnc{E}C]Cura y no\[\mn{D}]ron|\[\mn{E}]dé |\[\mn{E}]cura y no\[\mn{G}]ron|\[\mn{E}]dé
    |\[\mnc{D}Dm]Cura y \[\mn{E}]to\[\mn{D}]to|\[\mnc{C}C]to |\[\mnc{D}Dm]cura y \[\mn{E}]to\[\mn{D}]to|\[\mnc{C}C]to
  \endchorus
  \beginchorus
    |\[\mnc{C}C]Icama\[\mn{D}]hi |\[\mn{E}]pa\[\mn{D}]hue\[\mnc{C}Am]ne \[\mn{A}]i|\[\mnc{C}C]pa\[\mn{D}]hue\[\mn{C}]ne \up{1}(| \e)
  \endchorus\glueverses
  \beginchorus
    |\[\mnc{D}Dm]Pahue |\[\mnc{E}C]pa\[\mn{D}]hue\[\mnc{C}Am]ne \[\mn{A}]i|\[\mnc{C}C]pa\[\mn{D}]hue\[\mn{C}]ne \up{2}(| \e)
  \endchorus
\endsong


\beginsong{Te Nande}[ex={yawanawa},ph={III, IV}]
  \audio[]{https://www.youtube.com/watch?v=MhXOAJ1UGxw}
  \audio[]{https://www.youtube.com/watch?v=89hzAlPXJAQ}
  \beginchorus\memorize
    \[^\mn{B}]Ta|\[\mnc{E}Em]raouaca um|bari \[^\mn{F#}\mn{D}]kara|\[^\mn{E}]nê | \e
    Mo|rouane um|bari sara|\[C]nê | \e
    Ka|\[D]ho anze um|bari te nan|\[Em]de | | | \e
  \endchorus
  \notesoff
  \beginchorus
    \ind |^Te nan|de te nan|^de | \e
    \ind |^Te nan|de te nan|^de | \up{2}(| |) \e
  \endchorus
  \beginchorus
    Kapa |^xô mina ha|ro | \e
    Keho |ande te ne|^dê | \e
    |^Yo rain|de yo rain|^dê | | | \e
  \endchorus
  \goto{Te nande}
\endsong


\beginsong{Wacomaia}[ex={yawanawa},ph={IV}]
  \audio[]{https://www.youtube.com/watch?v=27OtT76yyAg}
  \audio[]{https://www.youtube.com/watch?v=7RzhsVTOcEk}
  % TODO: check different versions. Most (original?) seem to have more 'wacomaia' words
  %       in the first and last verses of the song, perhaps different chords, too...
  \beginchorus
    \[\mnc{G}G]Waco\meter{4}{4}|maia, Wa\[\mn{A}]co|\[\mnc{B}Am]mai\[\mn{A}]a, |\[\mn{G}]Wa\[\mn{A}]co\[\mn{B}]mai|\[\mn{A}]a,
    Waco|\[C]maia, |Wacomai\meter{2}{4}|\[Am]a, \meter{4}{4}|\[G]heé! | | \e
  \endchorus
  \beginchorus
    Wa|comai|\[Am]a, tone |pinda|\[C]ke pinda |kanarô |\[G]ho | | \e
  \endchorus
  \beginchorus
    To|zake hô pa|\[Am]rá tone |pinda|\[C]ke, pinda |kanarô |\[G]ho | | \e
  \endchorus
  \beginchorus
    Ahe, e|he, ya he, |\[Am]waya, |Wacomai|a,
    Waco|\[C]maia, |Wacomai\meter{2}{4}|\[Am]a, \meter{4}{4}|\[G]heé! | | \e
  \endchorus
  \begin{explanation}
    \begin{description}
      \item[Wacomaia] means ``joy, happiness''
    \end{description}
  \end{explanation}
\endsong


\beginsong{Kanô Kanô}[ex={yawanawa},ph={IV}]
  \audio[]{https://www.youtube.com/watch?v=0Po\_Q1Ft9aY}
  \audio[]{https://www.youtube.com/watch?v=Vrj9wXKuyDk}
  \audio[]{https://soundcloud.com/iranviene-edderv/yawanawa-kano-kano}
  \beginchorus\memorize
   \[^G]{} \[^\mn{G}]Ka|\[^\mn{A}]nô \[^\mn{G}]Ka|\[\mnc{A}Am]nô, | | | |\[C]{} \[^\mn{G}]Ka|\[^\mn{A}]nô \[^\mn{G}]Ka|\[G]nô | | \e
  \endchorus
  \notesoff
  \beginchorus
    He ka|nore kano|^rê, He Ka|nore Kano|rê
    He ka|nore kano|^rê, He Ka|nore Kano|^rê | | \e
  \endchorus
  \beginverse
    Mara|cá inakai|^nã Mara|cá inakai|nã
    Mara|cá inakai|^nã Mara|cá inakai|^nã | | \e
  \endverse
  \beginverse
    Mara|cá ioio|^iô Mara|cá ioio|iô
    Mara|cá ioio|^iô Mara|cá ioio|^iô | | \e
  \endverse
  \beginverse
    Io|iô ioio |^iô, Io|iô ioio |iô
    Io|iô ioio |^iô, Io|iô ioio |^iô | | \e
  \endverse
  \beginverse
    Para|ra iranoi |^rã, Para|ra iranoi |rã
    Para|ra iranoi |^rã, Para|ra iranoi |^rã | | \e
  \endverse
  \beginverse
    Ra|no rano i|^rã, Ra|no rano i|rã
    Ra|no rano i|^rã, Ra|no rano i|^rã | | \e
  \endverse
  \beginverse
    Para|ra ioiô |^iô, Para|ra ioiô |iô
    Para|ra ioiô |^iô, Para|ra ioiô |^iô | | \e
  \endverse
\endsong


\beginsong{Emamaa \\ Moder Jord}[by={trad., Tane Mahuta},ex={eesti, suomi, svenska; based on a Swedish folk song},tags={Mother Earth},ph={III}]
  % TODO: check song origin, find out the full swedish lyrics,
  %       and perhaps better finnish ones as well; ting?
  \textnotefornext{eesti keeles:}
  \beginchorus\memorize
    \lrep \[^\mn{A}]Ema |\[\mnc{D}Dm]Maa, ema \[C]Maa, Maa|\[Dm]e\[Am]ma \rrep
    Su |\[F]sees elab \[C]ürgne |\[Dm]vägi mis |toob mul \[Am]lohu|\[Dm]tust
  \endchorus
  \notesoff
  \beginchorus
    |\[Dm]Kummar\[Am]dan su |\[Dm]ees kallis \[Am]ema
    Et |\[F]südames \[C]mind ikka |\[Dm]kannad
  \endchorus
  \textnotefornext{suomeksi:}
  \beginchorus
    \lrep Äiti |^Maa, äiti ^Maa, Maa|^äi^ti \rrep
    Sun |^sisälläs on ^väkevä |^voima, se |minua ^lohdut|^taa
  \endchorus
  \beginchorus
    |\[Dm]Kumar\[Am]ran edes|\[Dm]säsi rakas \[Am]äiti
    Sua |\[F]sydämes\[C]säin aina |\[Dm]kannan
  \endchorus
  \textnotefornext{på svenska:}
  \beginchorus
    \lrep Moder |^Jord, moder ^Jord, Jord|^mo^der \rrep
    Du |^glöder så ^varmt i ditt |^inre, din |hetta ^ger mig |^lust
  \endchorus
\endsong


\beginsong{O la Mama \\ Ancient Mother}[tags={Divine Mother, Mother Earth},by={trad. African},ex={some african language, english},ph={I, II}]
  \beginchorus\memorize
    \[^\mn{E}]O \[^\mn{B}]la |\[\mnc{A}Am]Mama, |\[D]{} wa ha su |\[Em]kola |\[Bm11/D]
    O la |\[C]Mama, |\[D]{} wa ha su |\[Em]wam | \e
    \vspace{1em}
    O la |\[Am]Mama, |\[D]{} kow wey ha |\[Em]ha ha ha |\[Bm11/D]
    O la |\[C]Mama, |\[D]{} ta te ka|\[Em]yee | \e
  \endchorus
  \notesoff
  \textnotefornext{in English:}
  \beginchorus
    Ancient |^Mother, |^ I hear you |^calling |^
    Ancient |^Mother, |^ I hear your |^song | \e
    \vspace{1em}
    Ancient |^Mother, |^ I hear your |^laughter |^
    Ancient |^Mother, |^ I taste your |^tears | \e
  \endchorus
  % This second verse is a more unknown addon by somebody:
  \beginchorus
    Ancient |^Mother, |^ I feel you |^calling |^
    Ancient |^Mother, |^ I sing your |^song | \e
    \vspace{1em}
    Ancient |^Mother, |^ I share your |^laughter |^
    Ancient |^Mother, |^ I dry your |^tears | \e
  \endchorus
\endsong


\beginsong{Ide Were}[by={traditional},tags={water},ex={yoruba},ph={III, IV},key={Am},gk={Gm, Gm--Bm}]
  \audio[key=Gm]{https://www.youtube.com/watch?v=YUeIFRkthb8}
  \transpose{5}%\preferflats
  \mnbeginverse
    \[\mn{B}]Ide |\[\mnc{E}Em]were were \[\mn{D}]nita \[\mn{B}]Osh|\[\mnc{A}D]un \altchords{\id[1]{(Em) \capo{5}}|Em |D}
    \[\mn{G}]I\[\mn{A}]de |\[\mnc{B}Gmaj7]were \[\mn{D}]we\[\mn{B}]re | \e \altchords{|Gmaj7 | \e}
    Ide |\[\mnc{E}Em]were were \[\mn{D}]nita \[\mn{B}]Osh|\[\mnc{A}D]un \altchords{|Em |D}
    \[\mn{G}]I\[\mn{A}]de |\[\mnc{B}G]were \[\mn{D}]we\[\mn{B}]re \[\mn{G}]nita \[\mn{E}]ya | \e \altchords{|G | \e}
    \[\mn{G}]Ocha kini|\[\mnc{E}C]ba ni\[\mn{G}]ta \[\mn{E}]Osh|\[\mnc{A}D]un \altchords{|C |D}
    \[\mn{B}]Chek|\[G]e \[\mn{G}]cheke \[\mn{E}]chek|\[C]e ni\[\mn{G}\mn{E}]ta |\[\mnc{A}D]ya \altchords{|G |C |D}
    \[\mn{G}]I\[\mn{A}]de |\[\mnc{B}Bm]were \[\mn{D}]we\[\mn{B}]re |\[B7]{} \e \altchords{|Bm |B7}
  \mnendverse
  \begin{explanation}
    This Yoruba chant is dedicated to \textbf{Oshun}, the Goddess of Love,
    happiness and prosperity. She brings to us all the good things of life,
    and is defender of the poor and the mother of all orphans; Goddess
    Oshun brings to them their needs in this life.
    \par
    Oshun (also known as Ochun or Oxum in Latin America) is an orixá, a highly
    benevolent spirit or deity that reflects one of the expressions of God in
    the Ifa and Yoruba religions (Nigeria).
    \par
    Thought to be the most beautiful of the female Orixás. No one can resist
    her charming laugh, her graceful dancing, and her lips that taste like
    honey. She has a lush womanly figure with full hips, which suggest
    fertility and eroticism.
    \par
    She exhibits all of the attributes connected with fresh flowing water:
    lively, sparking, refreshing, vivacious. She is the \underline{Goddess
    of sweet water} and can be discovered where there is fresh water, at
    rivers, ponds, lakes, and particularly waterfalls.
    \par
    She is also a healer of the sick. Teacher, who taught the Yoruba culture,
    agriculture and mysticism. The art of divination using cowrie shells. The
    bringer of song, music and dance, healing chants and meditations taught
    to her by her father Obatala, the first of the created Orishi.
    \par
    This chant speaks of a necklace as a symbol of initiation into love.
    \par
    According to the Yoruba elders, Oshun [also Osun, Oxum]{} is the ``unseen
    mother present at every gathering'', because Oshun is the Yoruba
    understanding of the cosmological forces of water, moisture, and
    attraction. Therefore she is omnipresent and omnipotent. Her power is
    represented in another Yoruba scripture which reminds us that ``no one is
    an enemy to water'' and therefore everyone has need of and should respect
    and revere Oshun, as well as her followers.
    \par
    Oshun is the force of harmony. Harmony we see as beauty, feel as love,
    and experience as ecstasy. Oshun according to the ancients was the only
    female Irunmole amongst the original 16 sent from the spirit realm to
    create the world. As such, she is revered as ``Yeye'' --- the sweet mother
    of us all. When the male Irunmole attempted to subjugate Oshun due to
    her femaleness she removed her divine energy, called ase by the Yoruba,
    from the project of creating the world and all subsequent efforts at
    creation were in vain. It was not until visiting with the Supreme Being,
    Olodumare, and begging Oshun pardon under the advice of Olodumare that
    the world could continue to be created. But not before Oshun had given
    birth to a son. This son became Elegba, the great conduit of ase in the
    Universe and also the eternal and infernal trickster.
    \par
    Oshun is known as Iyalode, the ``(explicitly female) chief of the realm.''
    She is also known as Laketi, she who has ears, because of how quickly
    and effectively she answers prayers. When she possesses her followers,
    she dances, flirts and then weeps --- because no one can love her enough
    and the world is not as beautiful as she knows it could be.
  \end{explanation}
  \imagecc[1]{Oshun_ed_by_larva_aa_1290x1613px.png}%
\endsong


\beginsong{O Mileko}[ex={swahili},ph={I, IV}]
  \meter{2}{4}
  \beginverse
    \[\mn{C}]Aka|túm\[\mn{A}]bale \[\mn{G}]aka|túm\[\mn{E}]bale \emph{\[\mn{C}]Be|\[\mn{E}]lele \[\mn{C}]be|\[\mn{E}]lele}
    Be|le tzimi be|le tzimi \emph{Tzi|mimi tzi|mimi}
  \endverse
  \beginverse
    Atzi|mi tzaya atzi|mi tzaya \emph{Tza|yaya tza|yaya}
    Tza|ya butu tza|ya butu \emph{Bu|tutu bu|tutu}
  \endverse
  \beginverse
    Abu|tu gnda abu|tu gnda \emph{Gn|danga gn|danga}
    Kun|da leli kun|da leli \emph{Ale|lila ale|lila}
  \endverse
  \beginverse
    Ale|li manga ale|li manga \emph{Man|ganga man|ganga}
    \up{*}\emph{o man|ganga man|ganga o} | | \e
  \endverse
    \altlyr{\emph{slow down}}
  \beginchorus
    |O | mi|leko, | |o | mi|leko | \e
  \endchorus
\endsong


%%%%%%%%%%%%%%%%%%%%%%%%%%%%%%%%%%%%%%%%%%%%%%%%%%%%%%%%%%%%%%%%%%%
%%% LATEST PRINTOUT CONTAINED THE SONGS ABOVE.                  %%%
%%%%%%%%%%%%%%%%%%%%%%%%%%%%%%%%%%%%%%%%%%%%%%%%%%%%%%%%%%%%%%%%%%%
%%% Please try to not change the song numbers above this point. %%%
%%% Add new songs only after this point.                        %%%
%%%%%%%%%%%%%%%%%%%%%%%%%%%%%%%%%%%%%%%%%%%%%%%%%%%%%%%%%%%%%%%%%%%


