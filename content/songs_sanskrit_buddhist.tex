% Buddhist mantras
% ================
%
% The following sets the song number for the first song in this file.
% The number will automatically be incremented by one for each song.
% Please do not change this! Changing would make different versions of
% the songbook to have different numbers for the same songs, and it
% would totally mess up the selection booklets causing them to have
% wrong songs in them. (For the same reason, add new songs only to the
% end of each songs_ file.)
\setcounter{songnum}{470}


\beginsong{Compassion Mantra \\ Mantra of Avalokiteshvara}[index={Om Mani Padme Hum},tags={compassion},ph={I, II}]
  \showmantra{Om Mani Padme Hum}
  \vspace{1em}
  \textnotefornext{in Pāḷi:}
  % move the next one up (this is a special case of two \showmantra macros):
  \vspace{-2em}
  \showmantra{Om Mani Peme Hung}
  \begin{feeler}
    OM,\\
    the jewel (method; MANI)\\
    in the lotus (wisdom; PADME)\\
    indivisible (HUM).\\\vspace{1em}
    Hail to the Jewel in the Lotus.
  \end{feeler}
  \begin{explanation}
    % From a lecture given by The Dalai Lama at the Kalmuck Mongolian Buddhist
    % Center, New Jersey:
    \textbf{The 14th Dalai Lama:} ``It is very good to recite the mantra OM
    MANI PADME HUM, but while doing it, think the meaning of the six syllables
    which is great and vast.
    \par
    The first, OM [\ldots] symbolizes the practitioner's impure body, speech,
    and mind; it also symbolizes the pure exalted body, speech, and mind of a
    Buddha. [\ldots]
    \par
    The path is indicated by the next four syllables. MANI, meaning jewel,
    symbolizes the factors of method: the altruistic intention to become
    enlightened, compassion, and love. [\ldots]
    \par
    The two syllables, PADME, meaning lotus, symbolize wisdom. Just as a lotus
    grows forth from mud but is not sullied by the faults of mud, so wisdom is
    capable of putting you in a situation of non-contradiction where as there
    would be contradiction if you did not have wisdom. [\ldots]
    \par
    Purity must be achieved by an indivisible unity of method and wisdom,
    symbolized by the final syllable HUM, which indicates indivisibility. [\ldots]
    \par
    Thus the six syllables mean that in dependence on the practice of a path
    which is an indivisible union of method and wisdom, you can transform your
    impure body, speech, and mind into the pure exalted body, speech, and mind
    of a Buddha. [\ldots]''
    % % Commented out for ..mm.. reasons
    %\par
    %\textbf{Lama Thubken Trinley:} ``These six syllables prevent rebirth into
    %the six realms of cyclic existence. It translates as 'OM the jewel in the
    %lotus HUM'. OM prevents rebirth in the God realm, MA prevents rebirth in
    %the Asura (Titan) realm, NI prevents rebirth in the Human realm, PA
    %prevents rebirth in the Animal realm, ME prevents rebirth in the Hungry
    %Ghost realm, and HUM prevents rebirth in the Hell Realm.''
  \end{explanation}
\endsong


\beginsong{Jewel in the Lotus Flower}[index={Om Mani Padme Hum},tags={compassion},ph={I, II}]
  \meter{4}{4}
  \beginverse
    \[\mn{A}]There's \[^\mn{C}]a |\[\mnc{D}Dm]jewel in \[^\mn{C}]the \[^\mn{D}]Lo\[^\mn{C}]tus |\[^\mn{D}]flower
    Unfolding |\[C]deep with\[Am]in my |\[Dm]soul
    To be a |jewel in a Lotus |flower
    Unfolding |\[C]is the \[Am]highest |\[Dm]goal
  \endverse
  \notesoff
  \beginchorus
    ^Hari |^Om Mani Padme |Hum
    Om Mani |^Om Mani ^Padme |^Hum
  \endchorus
  \imagecc[1]{om_mani_padme_hum_script_bw_transparent_bg_2000px.png}
\endsong


\beginsong{Padmasambhava Mantra \\ Vajra Guru Mantra}[index={Om Ah Hum},ph={I, II}]
  \showmantra{Om Ah Hum Vajra Guru Padme Siddhi Hum}
  \vspace{1em}
  \textnotefornext{in Pāḷi:}
  % move the next one up (this is a special case of two \showmantra macros):
  \vspace{-2em}
  \showmantra{Om Ah Hung Benza Guru Peme Siddhi Hung}
  \vspace{1em}\vfill
  \textnotefornext{song:}
  \beginchorus
    |\[\mnc{B}Em]Om A\[\mn{A}]h |\[\mn{B}]Hum |Vajra \[\mn{C}]Gu\[\mn{A}]ru |\[Asus2]Padme \[\mn{C}]Sid\[\mn{D}]dhi |\[\mnc{B}Em]Hum | \e
  \endchorus
  \begin{explanation}
    \textbf{Padmasambhava} was a historical teacher in the 8th century, who
    is regarded as the founder of the Nyingma tradition. He is said to have
    been a renowned scholar, meditator, and magician --- the `second Buddha'
    in the minds of many in Tibet.
    \begin{description}
      \item Dilgo Khyentse Rinpoche:\par
        ``It is said that the twelve syllables Om Ah Hum Vajra Guru Padme
        Siddhi Hum carry the entire blessing of the twelve types of teaching
        taught by Buddha, which are the essence of His 84000 Dharmas\ldots''
      \item Jamyang Khyentse Wangpo:\par
        ``It begins with \textbf{OM AH HUM}, which are the seed syllables of
        the three vajras (of body, speech and mind).
        \par
        \textbf{VAJRA} signifies the \emph{dharmakaya} (\emph{Truth body}
        which embodies the very principle of enlightenment and knows no limits
        or boundaries) since, like the adamantine vajra, it cannot be `cut'
        or destroyed by the elaborations of conceptual thought.
        \par
        \textbf{GURU} signifies the \emph{sambhogakaya}
        (\emph{body of mutual enjoyment} which is a body of bliss or clear
        light manifestation), which is `heavily' laden with the qualities of
        the seven aspects of union.
        \par
        \textbf{PADME} signifies the \emph{nirmanakaya} (\emph{created body}
        which manifests in time and space), the radiant awareness of the wisdom
        of discernment arising as the lotus family of enlightened speech.
        \par
        Remembering the qualities of the great Guru of Oddiyana
        (Padmasambhava), who is inseparable from these three \emph{kayas},
        pray with the devotion that is the intrinsic display of the nature of
        mind, free from the elaboration of conceptual thought.
        \par
        All the supreme and ordinary accomplishments --- \textbf{SIDDHI} ---
        are obtained through the power of this prayer, and by thinking,
        `\textbf{HUM}! May they be bestowed upon my mindstream, this very
        instant!'''
    \end{description}
  \end{explanation}
  \yesendsongvfill
\endsong


\beginsong{Om Namo Amitābhaya}[ph={I}]
  \beginchorus
    |\[\bmc\mnc{A}Am]Om na\[\mnc{E}]mo\[\bmadj{-.5ex}] Ami|\[\mnc{F}\bmc G]tā\[\mn{E}]bha\[\mn{D}]ya\[\bmadj{-.7ex}]
    |\[\bmc C]Buddhaya, \[\bmc Em7]Dharmaya, |\[\bmc Am]Sanghaya\[\bmadj{-.7ex}]
  \endchorus
  \beginchorus
    |\[\bmc Am]Om na\[\bm]mo, |\[\bmc G]Om na\[\bm]mo
    |\[\bmc F]Om na\[\bm]mo Ami|\[\bmc E]tābhaya\[\bmadj{-.7ex}]
  \endchorus
  \begin{explanation}
    \textbf{Amitābha} is the principal buddha in Pure Land Buddhism, a branch
    of East Asian Buddhism. In Vajrayana Buddhism, Amitābha is known for his
    longevity attribute, magnetising red fire element, the aggregate of
    discernment, pure perception and the deep awareness of emptiness of
    phenomena. According to these scriptures, Amitābha possesses infinite
    merits resulting from good deeds over countless past lives as a bodhisattva
    named Dharmakāra. Amitābha means ``Infinite Light'' so Amitābha is also
    called ``The Buddha of Immeasurable Life and Light''.\\
    \par
    \noindent Buddhists take refuge in the \emph{Three Jewels} or
    \emph{Triple Gem}, which are:
    \begin{description}
      \item[\hspace{2em} Buddha:] the fully enlightened one
      \item[\hspace{2em} Dharma:] the teachings (expounded by the Buddha)
      \item[\hspace{2em} Sangha:] the spiritual community (the monastic order
        of Buddhism that practices the Dharma)
    \end{description}
  \end{explanation}
\endsong


\beginsong{Perfection Mantra \\ Gate Gate}[index={Teyata Gate Gate},ph={II}]
  \showmantra{Teyata Gate Gate Paragate Para Samgate Bodhi \brk So Ha}
  \begin{feeler}
    Gone, gone, gone far beyond to the awakened state.
  \end{feeler}
  \vfill
  \textnotefornext{song:}
  \meter{6}{8}
  \beginchorus
    \[\mn{B}]Ga\[\mn{D}]te |\[\mnc{E}Em]Gate Pa\[\mn{D}]ra|\[D]gate
    Para Sam|\[Bm]gate Bodhi |\[Em]So Ha
  \endchorus
  \beginchorus
    Gate |\[G]Gate Para|\[D]gate
    Para Sam|\[Bm]gate Bodhi |\[Em]So Ha
  \endchorus
  \begin{explanation}
    The path that takes us to enlightenment comprises the six arts of
    perfection. This mantra helps us to be generous, patient,
    conscientious, diligent, focused and wise.
  \end{explanation}
\endsong


\beginsong{Tara Mantra}[index={Om Tare Tu Tare},by={traditional, Deva Premal},ph={III}]
  \showmantra{Om Tare Tu Tare Ture Mama Ah Yuh Pune Jana Putim Kuru So Ha}
  \begin{feeler}
    The liberator of suffering shines light upon me to create\\
    an abundance of merit and wisdom for long life and happiness.
  \end{feeler}
  \vfill
  \textnotefornext{song:}
  \beginchorus
    \[\mn{E}]Om |\[\mnc{A}Am]Tare Tu |\[\mnc{C}Fmaj7]Tare Tu|\[\mnc{D}G]re \[\mn{F}\mn{D}]So |\[\mnc{E}C]Ha
    Om |\[Dm]Tare Tu |\[Em]Tare Tu|\[Fmaj7]re So |\[Am]Ha
  \endchorus
  \begin{explanation}
    Long life and good health for oneself and others is sought through
    recitation of this mantra thus making one’s life and particularly the
    spiritual journey meaningful.
    \par
    \emph{Tara}, who Tibetans also call \emph{Dolma}, is commonly thought to
    be a Bodhisattva or Buddha of compassion and action, a protector who comes
    to our aid to relieve us of physical, emotional and spiritual suffering.
    \par
    Tara has 21 forms, of which two are especially popular among Tibetan
    people: \emph{White Tara}, who is associated with compassion and long life,
    and \emph{Green Tara}, who is associated with enlightened activity and
    abundance.
  \end{explanation}
\endsong


\beginsong{Medicine Buddha Mantra \\ Healing Mantra}[index={Teyata Om Bekanze Bekanze},tags={health},ph={III}]
  \showmantra{Teyata Om Bekanze Bekanze Maha Bekanze Radza Samut Gate So Ha}
  \begin{feeler}
    I invoke the healing buddha inside me by going all the way to the supreme heights\\
    to remove the pain of illness and spiritual ignorance.
  \end{feeler}
  \vfill
  \textnotefornext{song:}
  \beginchorus
    |\[\mnc{D}Dm]Teyata Om |\[\mnc{F}F]Bekanze Bekanze |\[C]{} \[\mn{E}]Ma\[\mn{D}]ha \[\mn{E}]Be\[\mn{D}]kan\[\mn{E}]ze
    | Radza Samut Gate |\[Dm]So Ha | \e
  \endchorus
  \begin{explanation}
    The practical purpose of spirituality is to help others deal with their
    various life issues. Sickness represents a major problem. Reciting this
    mantra may contribute to healing on many levels adding to the effectiveness
    of medical treatment and medicines.
  \end{explanation}
\endsong


\beginsong{Purification Mantra}[index={Om Benza Satto Hung},ph={I}]
  \showmantra{Om Benza Satto Hung}
  \begin{explanation}
    \ldots which is the short version of the 100 syllable Mantra:\\
    \\
    OM BENZA SATVO SA MA YA MA NU PALA YA BHENZA SATTO TENO PA TISHTHA DRIDHO
    ME BHAWA SUTOKHAYO ME BHAWA SUPOKHAYO ME BHAWA ANURAKTO ME BHAWA SARVA
    SIDDHI ME PRAYACCHA SARVA KARMA SUTSA ME TSITTAM SHREYANG KURU HUNG HA HA
    HA HA HO BHAGWAN SARVA TATHAGATA BENZA MA ME MUCCHA BHENZE BHAWA MAHA
    SAMAYASATTVA AH HUNG PHET
  \end{explanation}
  \begin{feeler}
    Buddha of Purification within me, embodying all the Buddhas, please
    protect my resolve to purify all my karmas and always bestow on me the
    ability to make my mind good, virtuous, auspicious and immeasurably loving
    with the indestructible strength of a diamond.
  \end{feeler}
  \begin{explanation}
    Even though our potential remains obscure in the darkness of negativity,
    it need not be permanent. This mantra helps transform negative karma
    created over many lifetimes.
  \end{explanation}
\endsong


\beginsong{Teacher Buddha Mantra}[index={Om Muni Muni},tags={teacher},ph={I}]
  \showmantra{Om Muni Muni Maha Muni So Ha}
  \begin{feeler}
    To the teacher, teacher, the great teacher, I pay homage.
  \end{feeler}
  \begin{explanation}
    Shakyamuni, the historical Buddha, cast as the overall teacher of the
    tradition, illustrates the point that without a good teacher in the
    beginning there can be no success in spiritual training. Reciting this
    mantra therefore helps us find a good teacher to lead us towards clarity
    of mind and ultimately discovery of our own pure consciousness which is
    the real guru.
  \end{explanation}
\endsong


\beginsong{Wisdom Mantra}[index={Om Ah Ra Pa Tsa Na},tags={wisdom},ph={II}]
  \showmantra{Om Ah Ra Pa Tsa Na Dhi Dhi Dhi\ldots}
  \begin{feeler}
    Amidst the chaos, everything is pure by nature.
  \end{feeler}
  \begin{explanation}
    The pinnacle of spiritual success is to achieve enlightenment (regardless
    of what it means or whether such a thing is possible or not). This depends
    on recognition of our potential. The mantra confirms that each of us has
    the capacity to replace ignorance with wisdom.
  \end{explanation}
\endsong


%%%%%%%%%%%%%%%%%%%%%%%%%%%%%%%%%%%%%%%%%%%%%%%%%%%%%%%%%%%%%%%%%%%
%%% LATEST PRINTOUT CONTAINED THE SONGS ABOVE.                  %%%
%%%%%%%%%%%%%%%%%%%%%%%%%%%%%%%%%%%%%%%%%%%%%%%%%%%%%%%%%%%%%%%%%%%
%%% Please try to not change the song numbers above this point. %%%
%%% Add new songs only after this point.                        %%%
%%%%%%%%%%%%%%%%%%%%%%%%%%%%%%%%%%%%%%%%%%%%%%%%%%%%%%%%%%%%%%%%%%%

