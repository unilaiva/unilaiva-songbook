% Scribblings by larva (perhaps to be removed later)
% ==================================================
%
% The following sets the song number for the first song in this file.
% The number will automatically be incremented by one for each song.
% Please do not change this! Changing would make different versions of
% the songbook to have different numbers for the same songs, and it
% would totally mess up the selection booklets causing them to have
% wrong songs in them. (For the same reason, add new songs only to the
% end of each songs_ file.)
\setcounter{songnum}{670}


\beginsong{Puhdista}[by={larva}, tags={suitsutus, suojelus}, ph={I}, key={Bm}, sks={Bm, Bm--Cm}]
  \transpose{2}
  \beginchorus
    |\[\mnc{A}Am]Happi yhtyy |\[\mnc{C}Dm7]hii\[\mn{A}]leen, |\[\mnc{C}C]lämpö nousee | \e \altchords{\id[1]{(Am)}|Am |Dm7 |C | \e}
    |\[Am]Henki koskee |\[Dm7]ainetta, |\[F]tietoisuus \[G]koho|\[Am]aa\altchords{|Am |Dm7 |F G |Am}
  \endchorus
  \beginverse
    \ind |\[Am]Puhdis|\[Em]ta, |\[G]puhdis|\[Am]ta \altchords{|Am |Em |G |Am}
    \endverse\glueverses\beginchorus
    \ind |\[Dm]Savuna ilmaan, |\[Am]uhrilahja \altchords{|Dm |Am}
    \ind |\[G]Puhdis|\[Am]ta \altchords{|G |Am}
  \endchorus
  \beginverse
    |\[Em]Kutsun suojelusta, |kutsun suojelusta \altchords{|Em | \e}
    |\[Am]Paikka on pyhä | \e \altchords{|Am | \e}
    |\[Em]Kutsun suojelusta, |kutsun suojelusta \altchords{|Em | \e}
    |\[Am]Aika on pyhä | \e \altchords{|Am | \e}
  \endverse
  \beginverse
    \ind |\[Am]Puhdis|\[Em]ta, |\[G]puhdis|\[Am]ta \altchords{|Am |Em |G |Am}
    \endverse\glueverses\beginchorus
    \ind |\[Dm]Savuna ilmaan, |\[Am]uhrilahja \altchords{|Dm |Am}
    \ind |\[G]Puhdis|\[Am]ta \altchords{|G |Am}
  \endchorus
  \textnote{outro, fade out:}
  \beginchorus
    |\[G]Puhdis|\[Am]ta \altchords{|G|Am}
  \endchorus
\endsong


\beginsong{Matkustan}[by={larva},tags={sydän},ph={II}]
  \beginverse
    |\[\mnc{A}Am]Mat\[^\mn{C}]kus|\[Em]tan, |\[Am]matkus|\[Em]tan
    |\[Am]Mieleni |\[Em]sisään, |\[Dm]syvemmäl|\[Am]le
    |\[Am]Matkus|\[Em]tan, |\[Am]matkus|\[Em]tan
    |\[Am]Ajatusten |\[Em]taakse, |\[Dm]yti|\[Am]meen
  \endverse
  \beginverse
    |\[Am]Kaikenlaista |\[Dm]tulee vastaan;
    |\[C]mikä siitä \[Em]on tärke|\[Am]ää?
  \endverse
  \beginchorus
    \lrep |\[Am/E]Sydämeen voi |\[G]luottaa \rrep
    |\[Em]Siellä se \up{1}sisällä \up{2}(syvällä) |\[Am]on
  \endchorus
\endsong


\begin{intersong}
  \begin{feeler}
    ``An old alchemist gave the following consolation to one of his disciples: `No matter how
    isolated you are and how lonely you feel, if you do your work truly and conscientiously,
    unknown friends will come and seek you.'''\\
    --- \emph{Carl Jung} (1875--1961)
  \end{feeler}
  %\vfill
\end{intersong}


\beginsong{Avaruus aukeaa sisältä}[by={larva}, tags={sydän, avaruus}, ph={II, III}, key={Am}]
  \transpose{5}
  \beginchorus
    \lrep |\[\bmc\mnc{E}Em]Joskus luulen \[\bm]olevani |\[\bmc\mnc{F#}D]jotain mitä \up{1}\[\bm]min' en |\[\bmc Em]oo \[\bm]|\[\bm]\[\bm]\e \rrep \altlyr[2]{en vaan}
    |\[\bmc Am]Kun sen \[\bm]huomaan niin |\[\bmc C]suuntaan ta\[\bmc Bm]kaisin ko|\[\bmc Em]tiin: \[\bm] |\[\bm]\[\bm]\e
    |\[\bmc D]Sy-\[\bm]ydä|\[\bmc Em]meen, \[\bm] |\[\bmc D]ke-\[\bm]eskel|\[\bmc Em]le \[\bm]
    |\[\bmc D]Sy-\[\bm]y-|\[\bm]y-\[\bm]ydä|\[\bmc Em]meen \[\bm] |\[\bm]\[\bm]\e
    |\[\bmc D]Ke-\[\bm]e-|\[\bm]e-\[\bm]eskel|\[\bmc Em]le \[\bm] |\[\bm]\[\bm]\e
    \lrep |\[\bmc Am] \[\bmc C] |\[\bmc Bm] \[\bm] |\[\bmc Em] \[\bm] |\[\bm]\[\bm]\e \rrep
  \endchorus
  \beginchorus
    \lrep |\[\bmc G]A-\[\bm]va|\[\bmc C]ru-\[\bm]u-|\[\bmc Em]uus \[\bm] |\[\bm]\[\bm]\e \rrep
    |\[\bmc G]A-\[\bm]va|\[\bmc C]ruus \[\bm]auke|\[\bmc Am]aa \[\bmc Am6/E]keskel|\[\bmc Em]tä \[\bm]
    |\[\bmc D]Sy-\[\bm]yväl|\[\bmc Em]tä |\[\bmc D]si-\[\bm]isäl|\[\bmc Em]tä\[\bm]
    |\[\bmc D]Sy-\[\bm]y-|\[\bm]y-\[\bm]yväl|\[\bmc Em]tä \[\bm] |\[\bm]\[\bm]\e
    |\[\bmc D]Si-\[\bm]i-|\[\bm]i-\[\bm]isäl|\[\bmc Em]tä \[\bm] |\[\bm]\[\bm]\e
    \lrep |\[\bmc Am] \[\bmc C] |\[\bmc Bm] \[\bm] |\[\bmc Em] \[\bm] |\[\bm]\[\bm]\e \rrep
  \endchorus
\endsong


\beginsong{Ajan takaa}[by={larva},tags={rakkaus, lähde}]
  \beginchorus
    \lrep |\[\bmc\mnc{E}Am]A-\[\bm] \[\mnc{D}Am7]jan |\[\bmc Em]takaa,\[\bm] |\[\bmc Dm]totu\[\bm]uden |\[\bmc Am]luota\[\bm] \rrep
    \lrep |\[\bmc Em]To-\[\bm] |\[\bm] \[\bm]otuu|\[\bmc Am]den \[\bm]aijjai|\[\bm]jajajaja\[\bm]jajajaja \rrep
  \endchorus
  \beginchorus
    \lrep |\[\bmc Am]A-\[\bm] \[Am7]jan |\[\bmc Em]takaa,\[\bm] |\[\bmc Dm]totu\[\bm]uden |\[\bmc Am]luota\[\bm] \rrep
    \lrep |\[\bmc Dm]Jospa saisin \[\bm]sieltä |\[\bmc G]mukaani \[\bm]palan |\[\bmc Am]rakkaut\[\bm]ta |\[\bm]\[\bm]\e
    |\[\bmc Em]Sitä kylväi\[\bm]sin |\[\bmc G] \[\bm]maail|\[\bmc Am]maan\[\bm] |\[\bm]\[\bm]\e \rrep
  \endchorus
  \imagecc[3]{bufo_alvarius_bw_transparent_bg_300x240px.png}
\endsong


\beginsong{Hetki}[by={larva},tags={kiitollisuus},ph={III, IV}]
  \beginchorus\memorize
    |\[\mnc{D}Dm]Mie\[^\mn{A}]li |\[Gm]kuljettaa |\[Am]mennee\[Am7]seen ja |\[Dm]tulevaan
    |\[Dm]Hetki |\[Gm]katoaa |\[Am]aja\[C]tusten |\[Dm]mukana
  \endchorus
  \notesoff
  \beginchorus
    |\[Dm]Oi kuinka \[C]kaipaan |\[Dm]niin
    |\[B&]Sitä mitä \[C]en osaa |\[Dm]saavuttaa
  \endchorus
  \beginchorus
    \up{1}(mutta) |\[C#]Kiitos kiitos kiitos kiitos |kiitos tästä hetkes|\[Dm]tä | \e
  \endchorus
  \beginchorus
    Se |^opettaa |^elämään |^elämää ^ |^tässä vaan
    |^Opettaa |^elämään |^e-^elä|^mää!
  \endchorus
\endsong


\beginsong{Minne olenkaan matkalla}[by={larva},ph={III}]
  \beginchorus
    |\[\mnc{A}Am]Minne mä \[\mn{E}]olenkaan |\[F]matkalla,
    |\[C]tiedä sitä |\[Em]en
  \endchorus
  \beginverse
    Mutta |\[C]tahdon |\[G]luottaa,
    sillä |\[Em]kaikkeus ihme |\[Am]on
  \endverse
  \beginchorus
    pada-\[Dm7]diida-diida-|Diida-diida-diida-diida-|Daida-
    \[Em]dam-pada-|Dam padadam-|Padam-\[Am]dam | \e \up{2}(| | \e)
  \endchorus
  \beginverse
    \ind |\[Dm]Sisältäni | löydän |\[Am]surua, | \e
    \ind |\[Dm]paljon | on myös |\[Am]iloa | \e
    \ind \lrep |\[C]Elämä |\[G]kaunis |\[Am]on | \e \rrep
  \endverse
  \beginverse
    \ind |\[Dm]Ympärilläni | kohtaan |\[Am]pelkoa, | \e
    \ind |\[Dm]kaikkialla | kuitenkin |\[Am]rakkautta | \e
    \ind \lrep |\[C]Elämä |\[G]kaunis |\[Am]on | \e \rrep
  \endverse
\endsong


\beginsong{Lämpö, löyly}[by={larva},tags={sauna}]
  \beginchorus\memorize
    |\[Am] \[^\mn{A}]Lämpö, |\[^\mn{E}]löyly, \[G]iha|\[C]nainen
    |\[Dm] Poista |\[E]kiire aina|\[Am]hinen
  \endchorus
  \beginchorus
    |^ Täytä s|ielu, t^yhjää |^mieli
    |^ Anna |^ajatusten |^sulaa
  \endchorus
  \beginchorus
    |^ Tuli, |vesi, ^vasta|^jaiset
    |^ Yhes |^löylyn tänne |^tuovat
  \endchorus
  \beginchorus
    \ind |\[F] Ilon, |\[E]rauhan, meille |\[Am]suovat
  \endchorus
\endsong

\nextcol % Jump to the next page; can be removed when there wouldn't be an empty page
\beginsong{Kyynikolle pelastus}[by={larva}]
  \beginchorus
    |\[\mnc{C}C]Mikä tääll' \[\mn{B}]on |\[F]aitoo?
    Ku |\[C]rakkauskin vaatii |\[G]taitoo.
  \endchorus
  \beginchorus
    |\[C]Anna mulle |\[F]jotakin pientä:
    |\[G]lientä tai vaikkapa |\[C]sientä.
  \endchorus
  \noendsongvfill% % because of the intersong below
\endsong


\begin{intersong} % Fibonacci sequence
  \vfill
  \begin{math}
    % \mathbb comes from 'amssymb' package (the rest doesn't require any pkgs)
    (F_{n})_{n\in\mathbb{N}} = \left\lbrace
      \begin{array}{l}
        F_0 = 0\\
        F_1 = 1\\
        F_n = F_{n-1} + F_{n-2}, n>1
      \end{array}
    \right\rbrace = (0, 1, 1, 2, 3, 5, 8, 13, 21, 34, \ldots)
  \end{math}
  \vfill
  %\vfill
  %\footnotesize
  %\begin{center}
  %  1, 1, 2, 3, 5, 8, 13, 21, 34, 55, 89, 144, 233, 377, 610, 987, 1597, \ldots
  %\end{center}
\end{intersong}


%%%%%%%%%%%%%%%%%%%%%%%%%%%%%%%%%%%%%%%%%%%%%%%%%%%%%%%%%%%%%%%%%%%
%%% LATEST PRINTOUT CONTAINED THE SONGS ABOVE.                  %%%
%%%%%%%%%%%%%%%%%%%%%%%%%%%%%%%%%%%%%%%%%%%%%%%%%%%%%%%%%%%%%%%%%%%
%%% Please try to not change the song numbers above this point. %%%
%%% Add new songs only after this point.                        %%%
%%%%%%%%%%%%%%%%%%%%%%%%%%%%%%%%%%%%%%%%%%%%%%%%%%%%%%%%%%%%%%%%%%%

