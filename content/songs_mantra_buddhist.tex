\beginsong{Om Mani Padme Hum \\ Compassion Mantra \\ Mantra of Avalokiteshvara}[]
  \showmantra{Om Mani Padme Hum}
  {\small\textnote{Tibetan:} }
  % move the next one up (this is a special case, using two \showmantra after each other):
  \vspace{-2em}
  \showmantra{Om Mani Peme Hung}
  \begin{feeler}
    Om, the jewel (method; mani) in the lotus (wisdom; padme), indivisible (hum). 
    Hail to the Jewel in the Lotus.
  \end{feeler}
  \begin{explanation}
    Lama Thubken Trinley: "These six syllables prevent rebirth into the six realms of cyclic 
    existence. It translates as 'OM the jewel in the lotus HUM'. OM prevents rebirth in the God 
    realm, MA prevents rebirth in the Asura (Titan) realm, NI prevents rebirth in the Human realm, 
    PA prevents rebirth in the Animal realm, ME prevents rebirth in the Hungry Ghost realm, and 
    HUM prevents rebirth in the Hell Realm."

    The 14th Dalai Lama: "It is very good to recite the mantra Om mani padme hum, but while doing 
    it, think the meaning of the six syllables which is great and vast... The first, Om [\ldots ] 
    symbolizes the practitioner's impure body, speech, and mind; it also symbolizes the pure 
    exalted body, speech, and mind of a Buddha[\ldots ]" "The path is indicated by the next four 
    syllables. Mani, meaning jewel, symbolizes the factors of method: (the) altruistic intention 
    to become enlightened, compassion, and love." 

    "Purity must be achieved by an indivisible unity of method and wisdom, symbolized by the final 
    syllable HUM, which indicates indivisibility" 

    "Thus the six syllables mean that in dependence on the practice of a path which is an 
    indivisible union of method and wisdom, you can transform your impure body, speech, and mind 
    into the pure exalted body, speech, and mind of a Buddha[\ldots ]"
  \end{explanation}
  \vfill% Fill the page, as it's very close to the bottom anyway on A5
\endsong


\beginsong{Jewel in the Lotus Flower \\ Om Mani Padme Hum}[ah={335}]
  \meter{4}{4}
  \beginverse
    There is a |\[Dm]jewel in the Lotus |flower
    Unfolding |\[C]deep with\[Am]in my |\[Dm]soul
    To be a |jewel in a Lotus |flower
    Unfolding |\[C]is the \[Am]highest |\[Dm]goal
  \endverse
  \beginchorus
    Hari |^om mani padme |hum
    Om mani |^om mani ^padme |^hum
  \endchorus
  \begin{center}%
    \vfill%
    \includegraphics[width=0.618\textwidth]{om_mani_padme_hum_script_bw_transparent_bg_2000px.png}
    \vfill%
  \end{center}
\endsong


\beginsong{Padmasambhava mantra \\ Vajra Guru mantra}[]
  \showmantra{Om Ah Hum Vajra Guru Padme Siddhi Hum}
  {\small\textnote{Tibetan:} }
  % move the next one up (this is a special case, using two \showmantra after each other):
  \vspace{-2em}
  \showmantra{Om Ah Hung Benza Guru Peme Siddhi Hung}
  \vspace{2em}
  \textnote{song:}
  \beginchorus
    \[Em]Om Ah |Hum |Vajra Guru |\[Asus2]Padme Siddhi |\[Em]Hum | |
  \endchorus
  \begin{explanation}
    \textbf{Padmasambhava} was a historical teacher in the 8th century, who is regarded
    as the founder of the Nyingma tradition. He is said to have been a renowned
    scholar, meditator, and magician -- the 'second Buddha' in the minds of many
    in Tibet.
    \begin{description}
      \item{Dilgo Khyentse Rinpoche}:
        ``It is said that the twelve syllables Om Ah Hum Vajra Guru Padme Siddhi Hum carry
        the entire blessing of the twelve types of teaching taught by Buddha, which are the
        essence of His 84000 Dharmas\ldots''
      \item{Jamyang Khyentse Wangpo}:
        ``It begins with \textbf{OM AH HUM}, which are the seed syllables of the three vajras (of body,
        speech and mind).

        \textbf{VAJRA} signifies the \textit{dharmakaya} [\textit{Truth body} which embodies the very
        principle of enlightenment and knows no limits or boundaries] since, like the adamantine vajra,
        it cannot be 'cut' or destroyed by the elaborations of conceptual thought.

        \textbf{GURU} signifies the \textit{sambhogakaya} [\textit{body of mutual enjoyment} which is
        a body of bliss or clear light manifestation], which is 'heavily' laden with the qualities of the
        seven aspects of union.

        \textbf{PADME} signifies the \textit{nirmanakaya} [\textit{created body} which manifests in time
        and space], the radiant awareness of the wisdom of discernment arising as the lotus family of
        enlightened speech.

        Remembering the qualities of the great Guru of Oddiyana [Padmasambhava], who is inseparable from these
        three kayas, pray with the continuous devotion that is the intrinsic display of the nature
        of mind, free from the elaboration of conceptual thought.

        All the supreme and ordinary accomplishments — \textbf{SIDDHI} — are obtained through the power of
        this prayer, and by thinking, '\textbf{HUM}! May they be bestowed upon my mindstream, this very
        instant!'''
    \end{description}
  \end{explanation}
\endsong


\beginsong{Om Tare Tu Tare\ldots \\ White Tara Mantra}[by={traditional, Deva Premal}]
  \showmantra{Om Tare Tu Tare Ture Mama Ah Yuh Pune Jana Putim Kuru So Ha}
  \begin{feeler}
    The liberator of suffering shines light upon me to create an abundance of merit and wisdom for 
    long life and happiness.
  \end{feeler}
  \begin{explanation}
    Long life and good health for oneself and others is sought through recitation of this mantra 
    thus making one’s life and particularly the spiritual journey meaningful.\\
  \end{explanation}
  \vspace{2em}
  \textnote{song:}
  \capo{3}
  \beginchorus
    Om |\[Am]Tare Tu |\[Fmaj7]Tare Tu|\[G]re So |\[C]Ha
    Om |\[Dm]Tare Tu |\[Em]Tare Tu|\[Fmaj7]re So |\[Am]Ha
  \endchorus
\endsong


\beginsong{Gate Gate \\ Perfection Mantra}[]
  \showmantra{Teyata Gate Gate Paragate Para Samgate Bodhi So Ha}
  \begin{feeler}
    Gone, gone, gone far beyond to the awakened state.
  \end{feeler}
  \begin{explanation}
    The path that takes us to enlightenment comprises the six arts of perfection. This mantra
    helps us to be generous, patient, conscientious, diligent, focused and wise.\\ 
  \end{explanation}
  \vspace{2em}
  \textnote{song:}
  \meter{6}{8}
  \beginchorus
    Gate |\[Em]Gate Para|\[D]gate
    Para Sam|\[Bm]gate Bodhi |\[Em]So Ha
  \endchorus
  \beginchorus
    Gate |\[G]Gate Para|\[D]gate
    Para Sam|\[Bm]gate Bodhi |\[Em]So Ha
  \endchorus  
\endsong


\beginsong{Teyata Om Bekanze Bekanze\ldots \\ Healing Mantra \\ Medicine Buddha Mantra}[tags={healing 1}]
  \showmantra{Teyata Om Bekanze Bekanze Maha Bekanze Radza Samut Gate So Ha}
  \begin{feeler}
    I invoke the healing buddha inside me by going all the way to the supreme heights to remove 
    the pain of illness and spiritual ignorance.
  \end{feeler}
  \begin{explanation}
    The practical purpose of spirituality is to help others deal with their various life issues. 
    Sickness represents a major problem. Reciting this mantra may contribute to healing on
    many levels adding to the effectiveness of medical treatment and medicines. 
  \end{explanation}
  \vspace{2em}
  \textnote{song:}
  \beginchorus
    |\[Dm]Teyata Om |\[F]Bekanze Bekanze |\[C] Maha Bekanze |
    | Radza Samut Gate |\[Dm]So Ha | |
  \endchorus
\endsong


\beginsong{Om Benza Satto Hung \\ Purification Mantra}[]
  \showmantra{Om Benza Satto Hung}
  \begin{explanation}
    \ldots which is the short version of the 100 syllable Mantra: \\
    OM BENZA SATVO SA MA YA MA NU PALA YA BHENZA SATTO TENO PA TISHTHA DRIDHO ME BHAWA SUTOKHAYO ME 
    BHAWA SUPOKHAYO ME BHAWA ANURAKTO ME BHAWA SARVA SIDDHI ME PRAYACCHA SARVA KARMA SUTSA ME
    TSITTAM SHREYANG KURU HUNG HA HA HA HA HO BHAGWAN SARVA TATHAGATA BENZA MA ME MUCCHA BHENZE 
    BHAWA MAHA SAMAYASATTVA AH HUNG PHET
  \end{explanation}
  \begin{feeler}
    Buddha of Purification within me, embodying all the Buddhas, please protect my resolve to 
    purify all my karmas and always bestow on me the ability to make my mind good, virtuous, 
    auspicious and immeasurably loving with the indestructible strength of a diamond.
  \end{feeler}
  \begin{explanation}
    Even though our potential remains obscure in the darkness of negativity, it need not be
    permanent. This mantra helps transform negative karma created over many lifetimes.  
  \end{explanation}
\endsong


\beginsong{Om Muni Muni Maha Muni So Ha \\ Teacher Buddha Mantra}[tags={teacher 1}]
  \showmantra{Om Muni Muni Maha Muni So Ha}
  \begin{feeler}
    To the teacher, teacher, the great teacher, I pay homage.
  \end{feeler}
  \begin{explanation}
    Shakyamuni, the historical Buddha, cast as the overall teacher of the tradition, illustrates 
    the point that without a good teacher in the beginning there can be no success in spiritual 
    training. Reciting this mantra therefore helps us find a good teacher to lead us towards 
    clarity of mind and ultimately discovery of our own pure consciousness which is the real guru.
  \end{explanation}
\endsong


\beginsong{Om Ah Ra Pa Tsa Na Dhi Dhi Dhi\ldots}[tags={wisdom 1}]
  \showmantra{Om Ah Ra Pa Tsa Na Dhi Dhi Dhi\ldots}
  \begin{feeler}
    Amidst the chaos, everything is pure by nature.
  \end{feeler}
  \begin{explanation}
    The pinnacle of spiritual success is to achieve enlightenment. This depends on recognition of 
    our potential. The mantra confirms that each of us has the capacity to replace ignorance with 
    wisdom. 
  \end{explanation}
\endsong


\begin{intersong}%
  \begin{center}
    \vfill%
    \includegraphics[width=0.618\textwidth]{Prokofiev_-_Golden_Ratio_fractal_rotated_bw_transparent_bg_944x1675.png}%
    \vfill
  \end{center}
    {\scriptsize Fractal pattern with Hausdorff dimension \(\frac{log \varphi}{log \sqrt[\varphi]{\varphi}} = \varphi \approx 1.618034 \),
    the Golden Ratio. The construction is similar to Heighway's dragon, except for the similarity
    ratios. It is generated from an IFS consisting of two similarities of ratios: \(r\) and \(r^2\),
    with \(r=\frac{1}{\varphi^{\left(\frac{1}{\varphi}\right)}}\). By: Prokofiev.}
\end{intersong}

