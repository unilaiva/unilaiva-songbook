% Finnish songs
% =============
%
% The following sets the song number for the first song in this file.
% The number will automatically be incremented by one for each song.
% Please do not change this! Changing would make different versions of
% the songbook to have different numbers for the same songs, and it
% would totally mess up the selection booklets causing them to have
% wrong songs in them. (For the same reason, add new songs only to the
% end of each songs_ file.)
\setcounter{songnum}{600}


\beginsong{Kalevala-sävelmä}[tags={(chords missing)}]
  % Present the melody on a staff using Lilypond
  \begin{lilywrap}\begin{lilypond}[] \include "tex/lilypond-settings-include.ly"
    {\key a \minor \time 2/4
      a'4 a' | b' b' | c'' e'' | b'2 | b'2
      c''4 a'| d'' c'' | b' c'' | a'2 a'2 \bar "|."
    }\addlyrics {Va -- ka van -- ha Väi -- nä -- möi -- nen tie -- tä -- jä i -- än i -- kui -- nen}
  \end{lilypond}
  \textnote{Haikea versio:}
  \begin{lilypond}[] \include "tex/lilypond-settings-include.ly"
    {\key a \minor \time 2/4
      d''4 d'' | d'' a' | d'' c'' | b'2 | b'2
      d''4 d''| d'' c'' | b' c'' | a'2 a'2 \bar "|."
    }\addlyrics {Va -- ka van -- ha Väi -- nä -- möi -- nen tie -- tä -- jä i -- än i -- kui -- nen}
  \end{lilypond}\end{lilywrap}
  \yesendsongvfill% to balance vspace before and after lilypond, as there is no other content
\endsong


\beginsong{Juurilaulu \\ Kuulumme piiriin}[tags={piiri},ph={I, V}]
  % Present the melody on a staff using Lilypond
  \begin{lilywrap}\begin{lilypond}[] \include "tex/lilypond-settings-include.ly"
    theMelody = \relative a' {
      \key a \minor \time 4/4 \partial 2
      \repeat volta 2 {
        a4 a8 a8 | g8( a8) a2.~
        | a2 a4 a8 b | c( a) a2.~
        | a2 a4 a8 b | c c~ c2.~
        | c2 c8 b4 g8 | g4 a2.~ | a2
      }
    }
    theLyricsOne = \lyricmode {
      Kuu -- lum -- me pii -- riin __
      Kuu -- lum -- me pii -- riin __
      Ai -- ko -- jen ta -- kaa __
      Ta -- kai -- sin kier -- toon __
    }
    theChords = \chordmode {
      s2 | a1:m | a:m | a:m | a:m | c | c | a:m | a:m
    }
    \layout { #(layout-set-staff-size 16) } % to fit better
    \include "tex/lilypond-score-chords-melody-tabs.ly"
  \end{lilypond}\end{lilywrap}
  \yesendsongvfill% to balance vspace before and after lilypond, as there is no other content
\endsong


\beginsong{Lampaanpolska \\ Kekrilaulu \\ Yksi kaksi kolme neljä}[ph={III}]
  \meter{3}{4}
  \beginverse
    |\[\mnc{A}Am]Yksi kaksi kolme |\[\mnc{B}E]neljä, |\[\mnc{C}Am]anna \[\mn{A}]i\[\mnc{G#}E]lon |\[\mnc{A}Am]olla.
    Ja |\[Am]kun suru |\[E]tulee, |\[Am]anna hä\[E]nen |\[Am]mennä.
  \endverse
  \beginverse
    |\[Am/E]Paarmat ne |\[Dm]laulaa, |\[C]neljä hiirtä |\[E]hyppelee.
    |\[Am]Kissi lyöpi |\[E]trummun päälle ja |\[Am]koko maa\[E]ilma |\[Am]pauhaa.
  \endverse
  % Image downloaded from: https://www.maxpixel.net/Musical-Instruments-Drum-Music-Jazz-Cat-1287910
  \imagecc[2]{cat_drumming_bw_transparent_bg_CC0_1264x1788px.png}
\endsong


\beginsong{Laulu oravasta}[by={Aleksis Kivi, Otto Kotilainen}]
  % NOTE: this version is composed by Aapo Similä
  \beginverse
    \musicnote{intro:}
    \up{*}\meter{3}{4}|\[C] \[F] \[Em]
  \endverse
  \beginverse\memorize
    \meter{3}{4}|\[\mnc{E}Am]Make\[^\mn{A}]as\[^\mn{C}]ti |\[\mncii{B}{A}Em]ora\[^\mn{G}]vai\[^\mn{E}]nen
    \meter{5}{4}|\[F]Makaa sammalhuonees\[G]sansa;
    \meter{3}{4}|\[Em]Sinnepä ei |\[C]Hallin hammas
    \meter{5}{4}|\[F]Eikä metsä\[Em]miehen \[Am]ansa
    \up{*}\meter{3}{4}|\[C]Ehtineet \[F]milloin\[Em]kaan, |\[C]ei \[F]milloin\[Em]kaan
  \endverse
  \notesoff
  \beginverse
    \meter{3}{4}|^Kammiostaan |^korkeasta
    \meter{5}{4}|^Katselee hän mailman ^piirii,
    \meter{3}{4}|^Taisteloa |^allans' monta;
    \meter{5}{4}|^Havuoksan ^rauhan^viiri
    \up{*}\meter{3}{4}|^Päällänsä ^liepoit^taa. |^ ^ ^
  \endverse
  \beginverse
    \meter{3}{4}|^Mikä elo |^onnellinen
    \meter{5}{4}|^Keinuvassa kehto^linnass'!
    \meter{3}{4}|^Siellä kiikkuu |^oravainen
    \meter{5}{4}|^Armaan kuusen ^äitin^rinnass':
    \up{*}\meter{3}{4}|^Metsolan ^kantele ^soi! |^ ^ ^
  \endverse
  \beginverse
    \meter{3}{4}|^Siellä torkkuu |^heiluhäntä
    \meter{5}{4}|^Akkunalla pienoi^sella,
    \meter{3}{4}|^Linnut laulain |^taivaan alla
    \meter{5}{4}|^Saattaa hänen ^ilta^sella
    \up{*}\meter{3}{4}|^Unien ^Kulta^laan. |^ ^ ^
  \endverse
  \musicnote{\up{*}grave}
\endsong


\beginsong{Taivas on sininen ja valkoinen}[by={Kansanlaulu},ph={II}]
  \meter{2}{4}
  % declare new (global) named chord-replay registers:
  \newchords{chords_taivas_a}\newchords{chords_taivas_b}
  \beginchorus\memorize[chords_taivas_a] % memorize chords into a named register
    |\[\mnc{A}Am]Taivas \[^\mn{C}]on \[^\mn{E}]sininen ja |\[E7]valkoi\[Am]nen ja
    |\[Am]tähtö\[Dm]si\[G7]ä|\[C]täyn\[Em]nä
  \endchorus
  \notesoff
  \beginchorus\memorize[chords_taivas_b] % memorize chords into a named register
    |\[Am]Niin on \[F]nuori |\[G7]sydä\[C]me\[F]ni
    |\[Am]aja\[E7]tuksia |\[Am]täynnä
  \endchorus
  \vspace{1em}
  \beginchorus\replay[chords_taivas_a] % replay chords from a named register
    |^Enkä mä muille |^ilmoi^ta mun
    |^sydän^su^ru|^ja^ni
  \endchorus
  \beginchorus\replay[chords_taivas_b] % replay chords from a named register
    |^Synkkä ^metsä ja |^kirkas ^tai^vas ne
    |^tuntee mun ^huoli|^ani
  \endchorus
\endsong


\beginsong{Haltin häät}[by={Hannu Seppänen, Arto Alaspää}]
  \beginverse\memorize
    Kun |\[\mnc{C}C]ihmiskunnan aamu \[^\mn{D}]vas\[^\mn{E}]ta |\[Em]alkoi sarastaa
    Ja |\[F]Lappi oli jättiläisten |\[G]maana
    |\[C]Kaunis Malla-neito alkoi |\[Em]häitään valmistaa
    |\[F]Sulhasenaan nuori uljas |\[G]Saana
  \endverse
  \notesoff
  \beginverse
    |^Kaikkialta kansaa saapui |^Haltiin juhlimaan
    Ja |^kirkonkellot häitä alkoi |^soittaa
    |^Silloin astui kirkkoon tumma |^Pältsa Ruotsinmaan
    Hän |^vaimokseen myös Mallan tahtoi |^voittaa
  \endverse
  \beginverse
    \ind Hän |\[Am]aikoi estää häät ja kutsui |\[Em]velhot avukseen
    \ind Ja |\[Am]pian saikin juhlakansa |\[Em]kuulla kauhukseen
    \ind Kun |\[F]pohjoisesta vyöryi |\[C/G]jää ja yltyi |\[G]tuuli
  \endverse
  \beginverse
    |^Kirkkokansa pakeni ja |^Mallaa sylissään
    Myös |^Saana alkoi juosten turvaan |^kantaa
    He |^kauas eivät ehtineet kun |^jäivät alle jään
    Ja |^jähmettyivät Kilpisjärven |^rantaan
  \endverse
  \beginverse
    \ind On |\[Am]aikakaudet tuntureiksi |\[Em]heidät muuttaneet
    \ind Ja |\[Am]Kilpisjärven kasvattaneet |\[Em]Mallan kyyneleet
    \ind Kun |\[F]jäinen pohjoistuuli |\[C/G]soi myös itkee |\[G]Saana
  \endverse
\endsong


\beginsong{Finlandia-hymni}[by={Jean Sibelius, Veikko Koskenniemi}]
  % TODO: better chords?
  \beginverse
     \[\mnc{F#}D]Oi \[\mnc{E}A]Suo\[\mnc{F#}D]mi, |\[\mnc{G}G]kat\[^\mn{F#}]so, |\[\mnc{E}A]si\[^\mn{F#}]nun \[\mnc{D}G]päi\[^\mn{E}]väs' |\[A]koit\[\mnc{F#}D]taa,
    |\[D] yön \[A]uh\[D]ka |\[G]karkoi|\[A]tettu \[G]on \[A]jo |\[D]pois,
    |\[D] ja \[A/C#]]aamun |\[Bm]kiuru |\[D/F#]kirkkaudessa |\[A]soit\[Em]taa
    |\[Em] kuin \[D/F#]itse |\[G]taiva|\[D]han kan\[A]si |\[F#]sois.
    |\[D] Yön \[A/C#]vallat |\[Bm]aamun |\[D/F#]valkeus \[A]jo |voit\[Em]taa,
    |\[Em] sun \[D/F#]päiväs |\[G]koittaa, |\[A7]oi syn\[D]nyin|maa! | \e
  \endverse
  \notesoff
  \beginverse
    ^Oi ^nou^se, |^Suomi, |^nosta ^korke|^al^le
    |^ pääs' ^sep^pe|^löimä |^suurten ^muis^to|^jen,
    |^ oi ^nouse, |^Suomi, |^näytit maail|^mal^le
    |^ sa ^että |^karkoi|^tit or^juu|^den
    |^ ja ^ettet |^taipu|^nut sa sor^ron |al^le,
    |^ on ^aamus |^alka|^nut, syn^nyin|maa! | \e
  \endverse
  % \begin{translation} % comment out: takes too much space
  %   Finland, behold, your day has now come dawning;
  %   Banished is night, its menace gone with light,
  %   Larks' song again in morning-brightness ringing,
  %   Filling the air to heaven's great height,
  %   And morning's glow, night's darkness overcoming;
  %   Your day is come, o my native land.
  %   \nextverse
  %   O Finland, rise, stand proud, the future facing,
  %   Your valiant deeds recalling, once again;
  %   O Finland rise, in the world's sight erasing
  %   From your fair brows vile slavery's stain.
  %   You were not broken by oppressors ruling;
  %   Your morning's come, o my native land.
  % \end{translation}
\endsong


\beginsong{Päivänsäde ja menninkäinen}[by={Reino Helismaa},ph={IV}]
  \beginverse
    |\[\mnc{C}Am]Aurin\[^\mn{B}]ko \[^\mn{A}]kun \[Dm]päätti retken, |\[Am]siskoistaan jäi \[Dm]jälkeen hetken
    |\[Am]päivänsäde \[E7]viimei|\[Am]nen.
    |\[Dm]Hämärä jo \[Am]metsään hiipi, |\[Dm]päivänsäde \[Am]kultasiipi
    |\[D]juuri aikoi \[D7]lentää eestä |\[G]sen,
    kun |\[C]menninkäisen \[Am]pienen näki |\[Dm]vastaan tule\[G7]van;
    se |\[C]juuri oli \[D7]noussut luolas|\[G]taan.
    Kas |\[C]menninkäinen \[C7]ennen päivän |\[F]laskua ei \[F#\textdegree7]voi
    mil|\[C]loinkaan olla \[Dm7]pääl\[G7]lä  |\[C]maan.
  \endverse
  \notesoff
  \beginverse
    |^Katselivat ^toisiansa; |^menninkäinen ^rinnassansa
    |^tunsi kummaa ^leiskun|^taa.
    |^Sanoi: poltat ^silmiäni, |^mut' en ole ^eläissäni
    |^nähnyt mitään ^yhtä iha|^naa!
    Ei |^haittaa vaikka ^loisteesi mun |^sokeaksi ^saa;
    on |^pimeässä ^hyvä asus|^taa.
    Käy |^kanssani, niin ^kotiluolaan |^näytän sulle ^tien
    ja |^sinut armaak^se^ni  |^vien!
  \endverse
  \beginverse
    |^Säde vastas: ^peikko kulta, |^pimeys vie ^hengen multa,
    |^enkä toivo ^kuole|^maa.
    |^Pois mun täytyy ^heti mennä, |^ellen kohta ^valoon lennä,
    |^niin en hetke^äkään elää |^saa.
    Niin |^lähti kaunis ^päivänsäde, |^mutta vielä^kin,
    kun |^menninkäinen ^öisin tallus|^taa,
    hän |^miettii, miksi ^toinen täällä |^valon lapsi ^on,
    ja |^toinen yötä ^ra^kas|^taa.
  \endverse
\endsong


% Force this song on its own page to not have an empty page after
% the song after this one. Remove this when chords are added or the
% songs are rearranged.
\sclearpage
\beginsong{Viatonten valssi}[by={Einojuhani Rautavaara, Eila Kivikk'aho},tags={(chords missing)}]
  \chordsoff % do not show empty line for non-existing chords
  \beginverse
    Kun kesäinen yö oli kirkkain ja tyyninä valvoivat veet
    ja helisi soittimet sirkkain kuin viulut ja kanteleet.
    Viisi pientä piru parkaa aivan ujoa ja arkaa
    sievin kumarruksin tohti käydä enkeleitä kohti.
  \endverse
  \beginverse
    Univormunsa karvaiset heitti, he sarvet ja saparovyön,
    oli lanteilla vain lukinseitti ja helisi harput yön.
    Enkelitkin sulkapaidan jätti tuonne, taakse aidan.
    Siellä häntä, siellä siipi toisiansa tervehtiipi.
  \endverse
  \beginverse
    Ja niinhän he, nostaen jalkaa, niin nätisti tanssia alkaa
    yli kallion kasteisen. Ja se yö oli onnellinen.
    Missäs sika, --- jos ei kerää kärsäänsä se yhtäperää ---
    siivet karvat, ynnä muuta, vielä maiskutellen suuta.
  \endverse
  \beginverse
    Sill' aikaa enkelit tanssi niin ujosti varpaillaan
    Vain pukuna pikkuinen kranssi, viis pirua toverinaan.
    Oi, pienoiset, ettehän arvaa, moni vaihtaa nahkaa ja karvaa.
    Mut harppua sirkat lyö, yhä kun on kesäyö.
  \endverse
  \beginverse
    Kerran tuli Aamunkoitto. Loppui tanssi, loppui soitto.
    Pirut, niinkuin enkelitkin, tunnusmerkkejänsä itki.  
  \endverse
\endsong


\beginsong{Täss' on nainen}[by={Hedningarna},tags={(chords missing)},ph={II}]
    % TODO: chords, melody (note: melody is good for Kalevala-type stuff)
  \meter{5}{8}
  \beginchorus
    \lrep |\[\mnlow{A}]Täss' \[\mnlow{G}]on \[\mnlow{A}]nainen |tuu\[\mnlow{G}]len \[\mnlow{A}]tuoma
    |\[\mnlow{B}]Tuulen \[\mnlow{C}]tuoma |\[\mnlow{B}]ve'en \[\mnlow{C}]ve\[\mnlow{A}]tämä \rrep
    \lrep |\[\mnlow{E}]Me\[\mnlow{D}]ren \[\mnlow{E}]a\[\mnlow{D}]al\[\mnlow{E}]to|\[\mnlow{D}]jen \[\mnlow{E}]a\[\mnlow{A}]jama
    |\[\mnlow{B}]Meren \[\mnlow{C}]tyrskyn |\[\mnlow{B}]työn\[\mnlow{C}]te\[\mnlow{A}]lemä \rrep
  \endchorus
  \notesoff
  \beginchorus
    \lrep |Kuin mie käynen |laulamahan
    |Laulan mie me|ret mesiksi \rrep
    \lrep |Suoloiksi me|ren somerot
    |Meren hiekat |hernehiksi \rrep
  \endchorus
  \beginchorus
    \lrep |Yhen vyöni |vyötännällä
    |Yhen paita|ni panolla \rrep
    \lrep |Solkeni so|littamalla
    |Polkimeni |painamalla \rrep
  \endchorus
  \beginverse
    \lrep |Nouse luonto|ni lovesta
    |Syntyni sy|västä maasta \rrep
    \lrep |Syntyni sy|västä maasta
    |Haavan alta |haltiainen! \rrep
  \endverse
\endsong


\beginsong{Sadelaulu}[by={Sanna Kurki-Suonio},tags={vesi},ph={II}]
  % notes: |\[d]Sa\[a]de \[a]\[a]syök\[a]sy\[e]y|\[f]vi \[a]sy\[g]li-\[f]i\[e]hin
  %        |\[a]Aa-\[b]a\|[g#]aa-\[e]a|\[a]aa\[g#]a\[a]a|\[b]a\[c]a\|[b]a\[a]a\[b]|a\|[g#]aa\[e]a
  \beginverse
    |\[\mnc{D}Dm]Sa\[^\mn{A}]de syöksyy|\[F]vi sy\[C]lihin |\[Dm]pisaraiset |\[F]paian \[C]päälle
    |\[Dm]Vesi vihmo|\[F]en ve\[C]tävi |\[Dm]kaiken alleen |\[F]kaste\[C]levi
    |\[Dm]Minä vain sa|\[F]teessa \[C]seison |\[Dm]satehessa |\[F]suloi\[C]sessa
  \endverse
  \notesoff
  \beginverse
    |^Oi kaalinna... |^ ^ |^ |^ ^
  \endverse
  \beginverse
    |^Vesi mulle |^voiman ^tuopi |^voiman vahvan |^ja vä^kevän
    |^Pyyhkii pois pö|^lyiset ^mietteet |^ajatukset |^auvot^taapi
  \endverse
  \beginchorus
    \ind |\[Am]Aaa... |\[E] |\[Am] |\[E]
  \endchorus
  \beginverse
    |^Ukko heittävi |^vasa^moitaan |^säästele ei |^sala^moitaan
    |^Minä vain sa|^teessa ^seison |^satehessa |^suloi^sessa
  \endverse
  \beginchorus
    |^Oi kaalinna... |^ ^ |^ |^ ^
  \endchorus
  \beginchorus
    \ind |\[Am]Aaa.. |\[E] |\[Am] |\[E] \rep{4}
  \endchorus
  \beginverse
    |^Puhdista ve|^si puh^dista |^ajatukse|^ni kir^kasta
    |^Syän surusta |^sulat^tele |^tuskan tunteet |^tunnol^tani
    |^Aatteet alhaiset |^aivois^tani |^puhdista pi|^sara ^pieni
  \endverse
  \beginchorus
    \ind |\[Am]Aaa.. |\[E] |\[Am] | \[E]
  \endchorus
  \beginverse
    |^Virtaa vesi, |^vihmo ^vesi |^voimaa tuot sä |^mulle ^vesi
    |^Virtaa vesi, |^vihmo ^vesi |^voimaa tuot sä |^mulle ^vesi
  \endverse
  \beginchorus
    |^Oi kaalinna... |^ ^ |^ |^ ^ \rep{5}
  \endchorus % after this there are four measures with just chord C
  % Image downloaded from: https://imgbin.com/png/psfPETyB/water-png
  % Image license: Free for non-commercial use
  \imagecc[4]{water_bw_transparent_bg_254x784px.png}
\endsong


\beginsong{Sisältäni portin löysin}[by={Pekka Streng},ph={III, IV}]
  \beginverse\memorize
    |\[A] \[^\mn{A}]Sisäl|täni \[^\mn{B}]por\[^\mn{A}]tin |\[D]löy\[^\mn{F#}]sin | \e
    |\[A] melkein |huomaamattom|\[D]an. | \e
    |\[G] Kun sen |läpi hiljaa |\[D]nousen, | \e
    |\[G] näen |toisin \[A] maail|\[D]man. | \e
  \endverse
  \notesoff
  \beginverse
    |^ Värit k|auniit vasta h|^uomaan, | \e
    |^ kuulen |äänet kirkkaam|^mat. | \e
    |^ Jätän |soinnuttomat lu|^olat, | \e
    |^ jätän |varjot ^ hoippuv|^at. | \e
  \endverse
  \noteson
  \beginverse
    \ind |\[\mnc{A}A]Aaa\ldots |\[\mn{E}] |\[\mnc{F#}D] |\[\mn{D}] \[\mn{A}] |\[\mnc{C#}A] | |\[D]\[\mn{D}] |\[\mn{F#}] \[\mn{A}]
    \ind |\[A]Aaa\ldots | |\[D] | |\[A] | |\[D] | \e
  \endverse
  \notesoff
  \beginverse
    |^ Jokin |säteilee ja loi|^staa | \e
    |^ alta k|uoren synkänk|^in. | \e
    |^ Kun sen |huomaa kevyem|^min | \e
    |^ ajatuk|set ^ liikkuv|^at. | \e
  \endverse
  \beginverse
    |^ Meidän |värit ylös vi|^rtaa | \e
    |^ ja |yhteen sulaut|^uu. | \e
    |^ Kaikki to|istaan kosket|^taa, | \e
    |^ kaikki a|amuun ^ kurkot|^tuu. | \e
  \endverse
  \goto{Aaa}
\endsong


\beginsong{Äidin laulu}[index={Laulan sinulle lapsoseni},by={Marika Salo},tags={Äiti Maa},ph={III}]
  \meter{4}{4}
  \beginverse
    |\[\mnc{A}Am]Laulan \[\mnc{B}G]sinulle |\[Am]lapsoseni |\[Em]laulan sinulle |\[Am]laulun
  \endverse\glueverses
  \beginchorus
    |\[Am]Kuule \[G]minua |\[Am]lapsoseni, kun |\[Em]äitisi laulaa |\[Am]sulle
  \endchorus
  \notesoff
  \beginverse
    |^Missä ^ikinä |^kuljetkin |^siellä olen |^aina
  \endverse\glueverses
  \beginchorus
    |^Olen ^jalkojes |^alla |^metsän puissa ja |^tuulessa
  \endchorus
  \beginverse
    |^Vuoret on ^syntyneet |^kupeistani |^laaksot rintojen |^välistä
  \endverse\glueverses
  \beginchorus
    |^Meret ja ^joet |^kohdustani |^veri on värjännyt |^maan
  \endchorus
  \beginverse
    |^Hyvä sun on ^täällä |^kulkea |^maan ja taivaan |^väliä
  \endverse\glueverses
  \beginchorus
    |^Äitisi ^silittää |^varpaitasi ja |^isäs silittää |^päätä
  \endchorus
  \beginverse
    Ja |^jos sattuis ^lapseni |^käymään niin, että |^ilmaan tipah|^taisit
  \endverse\glueverses
  \beginchorus
    Niin |^älä sinä ^lapseni |^huolta kanna |^isäs ottaa |^kopin
  \endchorus
  \beginverse
    |^Ei ole ^harha-|^askelia |^ei ole |^virheitä
  \endverse\glueverses
  \beginchorus
    |^Kauneutta ^kohti |^kuljet vain täällä |^ikuisessa |^sylissä
  \endchorus
  \beginverse
    Ja |^vielä ^kerron |^sinulle |^kuuntele vielä |^hetki
  \endverse\glueverses
  \beginchorus
    |^Aina oot ^ollut |^toivottu ja |^tänne terve|^tullut
  \endchorus
\endsong


\beginsong{Olkoon niin}[by={Laura Iso-Metsälä},ph={III}]
  \audio[]{https://soundcloud.com/arulai/olkoon-niin-live-in-temppeli}
  \beginverse
    |\[\mnc{B}Bm]Ol\[\mn{C#}]koon |\[\mn{D}]niin, olkoon |\[A]ni\[\mn{C#}]in | \e
    |\[Bm]Olkoon |niin, olkoon |\[A]niin | \e
    |\[Bm]Olkoon |niin, olkoon |\[A]niin | \e
    Että olet |\[Bm]terve, tur|vassa ja |\[A]vapaa | \e
  \endverse
  \beginverse
    Että olet |\[D]onnellin|en ja |\[F#m]rauhallin|en
    Olet |\[A]turvas|sa nyt ja |\[Bm]aina | \e
    Että olet |\[D]onnellin|en ja |\[F#m]rauhallin|en
    Olet |\[A]ter|ve ja |\[Bm]vapaa | \e
  \endverse
  \beginchorus
    \ind Ahee a|\[Bm]hoo, hee a|hoo, hee a|\[A]hoo | \e
    \rep{4}
  \endchorus
  \begin{feeler}
    May all beings be peaceful.\\
    May all beings be happy.\\
    May all beings be safe.\\
    May all beings awaken to\\
    the light of their true nature.\\
    May all beings be free.
  \end{feeler}
\endsong


\beginsong{Kiitos elämälle}[by={Tiia Ilomäki},tags={kiitollisuus},ph={V}]
  \beginverse
    |\[Am] \[\mn{E}]Tämä on |lau\[\mn{A}]lu e\[\mn{C}]lä|\[Em]mäl\[\mn{B}]le, | \e
    |\[Am] sen valoil|le ja |\[Em]varjoil|le.
    |\[Am] Tämä on |laulu tun|\[Em]teille, | \e
    |\[Am] joskus niin |helvetin |\[Em]tukahdutta|ville.
    |\[Am] | | | \e
    |\[Am] Tämä on |laulu tans|\[Em]sille, | \e
    |\[Am] liik|keelle niin |\[Em]kauniil|le.
    |\[Am] Tämä on |laulu nau|\[Em]rulle, | \e
    |\[Am] het|kille |\[Em]yhtei|sille.
    |\[Am] Tämä on |laulu ihmi|\[Em]sille, | \e
    |\[Am] rak|kaudelle |\[Em]jaetul|le.
    |\[Am] |\[C] |\[Am] |\[C] \e
    |\[Em] | | | \e
  \endverse
  \beginchorus
    \[\mn{B}]Mä laulan |\[G]kii\[\mn{E}]tos | \[\mn{G}]elä|\[\mnc{F#}Em]mäl\[\mn{E}]le | \e
    Mä laulan |\[G]kiitos | tun|\[Em]teille | \e
    Mä laulan |\[G]kiitos | tans|\[Em]sille | \e
    Mä laulan |\[G]kiitos | nau|\[Em]rulle | \e
    Mä laulan |\[G]kiitos | ihmi|\[Em]sille | \e
    |\[C] Olemme |tulleet tänne | luomaan |uutta
    |\[Em] maail|maa | | \e
    \up{*}\echo{maail|\[G]maa, | | | \e
    maail|\[Em]maa, | | | \e
    maail|\[G]maa, | | | \e
    maail|\[Em]maa | | | \e} \altlyr[*]{Vocalize on 2nd repeat only}
  \endchorus
  \beginverse
    |\[Am] Tämä on |laulu elä|\[Em]mälle, | \e
    |\[Am] sen valoil|le ja |\[Em]varjoil|le.
  \endverse
\endsong


\beginsong{Tupakkarulla}[by={Kansanlaulu},tags={uni}]
  \meter{2}{4}
  \beginverse
    |\[\mnc{D}Dm]Tuu |\[^\mn{A}]tuu |tupakka|\[A]rul|la \brk|\[Gm]mistäs |\[Dm]tiesit |\[A7]tänne |\[Dm]tul|la?
    |\[Dm]Tulin |pitkin |\[Gm7]Turun |\[A]tie|tä, \brk|\[A7]hämä|\[Dm]läisten |\[A7]härkä|\[Dm]tie|tä.
  \endverse
  \notesoff
  \beginverse
    |^Mistäs |tunsit |meidän |^por|tin? \brk|^Siitä |^tunsin |^uuden |^por|tin:
    |^haka |alla, |^pyörä |^pääl|lä \brk|^karhun |^talja |^portin |^pääl|lä
  \endverse
  \beginverse
    |^Uni |kysyi |uunin |^pääl|tä, \brk|^unen |^poika |^porstu|^as|ta:
    |^Onko |lasta |^kätky|^es|sä, \brk|^pientä |^peittei|^den si|^säs|sä?
  \endverse
  \beginverse
    |^Tuoppa |unta |tuokko|^ses|sa, \brk|^kanna |^vaski |^vakka|^ses|sa,
    |^sillä |silmät |^sive|^le,| \brk|^näky|^miset |^näppä|^e|le.
  \endverse
  \beginverse
    |^Nuku |nuku |nurmi|^lin|tu, \brk|^väsy |^väsy |^västä|^räk|ki,
    |^nuku |kun mi|^nä nu|^ku|tan, \brk|^väsy |^kun mi|^nä vä|^sy|tän.
  \endverse
  % Present the melody on a staff using Lilypond
  \begin{lilywrap}\begin{lilypond}[] \include "tex/lilypond-settings-include.ly"
    {\key d \minor \time 2/4
      d'2 | a'2 | a'8 a'4 f'8 | e'2 | e'2
      g'4 g'4 | a'4 f'4 | e'4 f'4 | d'2 | d'2
      f'4 d'4 | e'4 f'4 | g'4 f'4 | e'2 | e'2
      a'4. g'8 | f'4 f'4 | e'4 f'4 | d'2 | d'2 \bar "|."
    }\addlyrics{
      Tuu tuu tu -- pak -- ka -- rul -- la,
      mis -- täs tie -- sit tän -- ne tul -- la?
      Tu -- lin pit -- kin Tu -- run tie -- tä,
      hä -- mä -- läis -- ten här -- kä -- tie -- tä.
    }
  \end{lilypond}\end{lilywrap}
  % Nicely align music notation on both songs of this spread to the bottom.
  % for symmetry. (This is actually the default if Lilypond block is last,
  % but in if it changes, it doesn't have to change here.)
  \noendsongvfill
\endsong


\beginsong{Leppäkerttu}[by={Kansanlaulu},tags={uni},ph={V}]
  \meter{4}{4}
  \beginverse
    |\[\mnc{D}Dm]Lennä, \[^\mn{A}]lennä |\[A]leppäkerttu, |\[Gm]ison \[Dm]kiven |\[A7]juu\[Dm]reen.
    |\[Dm]Lennä leikki|\[Gm7]kedon \[A]kautta |\[A7]unipuuhun |\[Dm]suureen.
  \endverse
  \notesoff
  \beginverse
    |^Kulta-kulta|^lehden alla |^äiti ^puuron |^keit^tää.
    |^Unituutu |^leppä^kertun |^lämpimästi |^peittää.
  \endverse
  \beginverse
    |^Laula, laula, |^unilintu, |^tuoksu, ^tuomen|^tert^tu.
    |^Nuku, puna|^paitu^lainen, |^pikku leppä|^kerttu.
  \endverse
  % Present the melody on a staff using Lilypond
  \begin{lilywrap}
    \imagerb[5]{leppakerttu_transparent_bg_353x279px.png}
    \begin{lilypond}[] \include "tex/lilypond-settings-include.ly"
      {\key d \minor \time 4/4
        d'4 d'4 a'4 a'4 | a'4 a'4 e'4 e'4
        g'4 g'4 f'4 f'4 | e'2 d'2
        f'4 d'4 e'4 f'4 | g'4 f'4 e'4 e'4
        a'4 g'4 f'4 e'4 | d'2 d'2 \bar "|."
      }\addlyrics{
        Len -- nä, len -- nä lep -- pä -- kert -- tu,
        i -- son ki -- ven juu -- reen.
        Len -- nä leik -- ki -- ke -- don kaut -- ta
        u -- ni -- puu -- hun suu -- reen.
      }
    \end{lilypond}
  \end{lilywrap}
  % Nicely align music notation on both songs of this spread to the bottom.
  % for symmetry. (This is actually the default if Lilypond block is last,
  % but in if it changes, it doesn't have to change here.)
  \noendsongvfill
\endsong


\beginsong{Pieni tytön tylleröinen}[tags={uni}]
  \beginverse
    |\[\mnc{D}Dm]Pie\[^\mn{A}]ni \up{*}tytön |\[Gm]tylleröinen |\[Dm]tietä pitkin |kulki.
    |\[B&]Saapui sinne |\[Gm]Nukku-\[E7]Matti, |\[A7]silmät pienet |\[Dm]sulki.
  \endverse
  \notesoff
  \beginverse
    |^Kasvoi kuusi |^kukkalatva, |^käki siinä |kukkui.
    |^Mutta \up{*}tytön |^tylle^röinen |^nurmikolla |^nukkui.
  \endverse
  \beginverse
    |^Pieni \up{*}tytön |^tylleröinen |^sievää unta |näki
    |^että hänen |^ympä^rilleen |^tuli metsän |^väki.
  \endverse
  \beginverse
    |^Tapio ja |^Tellervo ja |^Sinipiika |pieni,
    |^Mustikka ja |^Mansik^ka ja |^suuri metsän |^sieni.
  \endverse
  \beginverse
    |^Sipsutteli |^Sinipiika |^pienen \up{*}tytön |luokse;
    |^otti kiinni |^kädes^tä, |^hyppeli ja |^juoksi.
  \endverse
  \beginverse
    |^Eipä \up{*}tytön |^tylleröinen |^ollut mitään |vailla.
    |^Hauska oli |^oles^kella |^Nukku-Matin |^mailla.
  \endverse
  \altlyr{pojan (palleroinen)}
  \imagecc[4]{sleeping_baby_bw_transparent_bg_1280px.png}%
\endsong


\beginsong{Tuuin yössä muukalaista}[by={Neilikka},tags={uni}]
  \audio[]{https://www.youtube.com/watch?v=OR7YYnrGvPg}
  \meter{3}{4}
  \beginverse
    |\[\mnc{E}Am]Tuu\[^\mn{C}]in \[^\mn{E}]yössä |\[Dm]muu\[^\mn{D}]ka\[^\mn{A}]laista, |\[\mnc{C}Am]tum\[^\mn{A}]ma\[^\mn{C}]sil\[^\mn{E}]mää |\[\mnc{B}E]lasta
    |\[Dm]Äsken tänne |\[Am]kutsumaani, |\[E]maasta iha|\[Am]nasta
  \endverse
  \notesoff
  \beginverse
    |^Unenrihmat |^sinne sitoo |^sen ken tuli |^vasta
    |^Nuku paluun |^sinne laulan |^tuulen suhi|^nasta
  \endverse
  \beginverse
    |^Suvisirkan |^soittelosta, |^hämärästä |^illan
    |^Laulan pilven, |^taivaan mieltä, |^laulan seitti|^sillan
  \endverse
  \beginverse
    |^Lapsen käydä |^univarpain |^kohti onnen |^maata
    |^Sinne äitis |^murhemieli |^seurata ei |^saata
  \endverse
  \beginverse
    \musicnote{interlude:}
    |^ |^ |^ |^ |^ |^ |^ |^
  \endverse
  \beginverse
    |^Siel ei tunnu |^talven tuskat, |^siel on aina |^kesä
    |^Siellä metsän |^joka puussa |^ilolla on |^pesä
  \endverse
  \beginverse
    |^Polut syliin |^satumetsän |^houkuttaa ja |^hukkuu
    |^Siellä lapsi |^itkutonna |^niittyvillaan |^nukkuu
  \endverse
\endsong


\beginsong{Suojelusenkeli}[by={P. J. Hannikainen},tags={suojelus}]
  \meter{3}{8}
  \beginverse
    \[^\mn{B}]Maan |\[\mnc{E}Em]korvessa |kulkevi |\[B]lapsosen |\[Em]tie.
    \[B]Hänt' |\[Em]ihana |\[G]enkeli |\[D]kotihin |\[G]vie.
    Niin |\[Em]pitkä \[C]on |\[D]matka, \[B]ei |\[Em]kotia |\[B]näy, | \e
    vaan |\[Em]ihana |\[C]enke\[B]li |\[Em]vieres\[Am]sä |\[B]käy,
    vaan |\[Em]ihana |\[Am]enkeli |\[Em/B]vieres\[B7]sä |\[Em]käy.
  \endverse
  \notesoff
  \beginverse
    On |^pimeä |korpi ja |^kivinen |^tie,
    ^ja |^usein se |^käytävä |^liukaskin |^lie.
    Oi, |^pian^han |^lapso^nen |^langeta |^vois, | \e
    jos |^käsi ei |^enke^lin |^kädes^sä |^ois,
    jos |^käsi ei |^enkelin |^kädes^sä |^ois.
  \endverse
  \beginverse
    % original: Ja syntikin mustia verkkoja vaan
    Ja |^mielikin |mustia |^verkkoja |^vaan
    ^on |^laajalle |^laskenut |^korpehen |^maan.
    Niin |^pian^han |^niihin^kin |^tarttua |^vois, | \e
    jos |^käsi ei |^enke^lin |^kädes^sä |^ois,
    jos |^käsi ei |^enkelin |^kädes^sä |^ois.
  \endverse
  \beginverse
    Maan |^korvessa |kulkevi |^lapsosen |^tie.
    ^Hänt' |^ihana |^enkeli |^kotihin |^vie.
    Oi, |^laps' et^hän |^milloin^kaan |^ottaa sä |^vois | \e
    sä |^kättäsi |^enke^lin |^kädes^tä |^pois.
    sä |^kättäsi |^enkelin |^kädes^tä |^pois.
  \endverse
\endsong


\beginsong{En etsi valtaa loistoa}[by={Jean Sibelius, Sakari Topelius}]
  \meter{4}{4}
  \beginverse
    \[^\mn{F#}]En |\[D]etsi \[^\mn{G}]val\[^\mn{F#}]taa, |\[Em]loisto\[A7]a, en |\[D]kaipaa \[A7]kul\[D]taa|\[A]kaan,
    mä |\[Em]pyydän \[A7]taivaan |\[D]valoa ja |rauhaa \[Gm]päälle |\[D]maan.
    Se |\[Em]joulu suo, mi |\[F#\textdegree7]onnen tuo ja |\[B7]mielet nostaa |\[Em]Luojan luo.
    Ei |\[A7]val\[D]taa \[A7]ei\[D]kä |\[Em]kultaa\[A7]kaan, vaan |\[D]rauhaa \[A7]päälle |\[D]maan.
  \endverse
  \notesoff
  \beginverse
    Suo |^mulle maja |^rauhai^sa ja |^lasten ^jou^lu|^puu,
    Ju|^malan ^sanan |^valoa, joss' |sieluin ^kirkas|^tuu.
    Tuo |^kotihin, nyt |^pieneenkin, nyt |^joulujuhla |^suloisin,
    Ju|^ma^lan ^sa^nan |^valo^a ja |^mieltä ^jalo|^a.
  \endverse
  \beginverse
    Luo |^köyhän niinkuin |^rikka^han saa, |^joulu ^i^ha|^na!
    Pi|^mey^tehen |^maailman tuo |taivaan ^valo|^a!
    Sua |^halajan, sua |^odotan, sä |^Herra maan ja |^taivahan.
    Nyt |^köy^hän ^niin^kuin |^rikkaan ^luo su|^loinen ^joulus |^tuo!
  \endverse
\endsong


\beginsong{Kosketa minua henki}[by={Ilkka Kuusisto 1979},ex={Virsi 125},ph={III}]
  \transpose{2} % transpose to C for easier chords
  \meter{3}{4}
  \beginverse
    |\[\mnc{D}B&]Kosketa |\[\mnc{F}Dm]minua, |\[\mnc{G}Gm]Hen|\[^\mn{D}]ki, |\[Cm7]kosketa |\[Dm]kirkka|\[E&]us! |\[G7/D]
    |\[Cm]Anna |\[F7]elä|\[B&maj7]mäl|\[G]le |\[Cm7]suunta ja |\[F7]tarkoi|\[B&]tus. | \e
  \endverse
  \notesoff
  \beginverse
    |^Kosketa, |^Jumalan |^Hen|ki, |^syvälle |^sydä|^meen. |^
    |^Sinne |^paina |^hil|^jaa |^luottamus |^rakkau|^teen. | \e
  \endverse
  \beginverse
    |^Rohkaise |^minua, |^Hen|ki, |^murenna |^pelko|^ni. |^
    |^Tässä |^maail|^mas|^sa |^osoita |^paikka|^ni. | \e
  \endverse
  \beginverse
    |^Valaise, |^Jumalan |^Hen|ki, |^silmäni |^aukai|^se, |^
    |^että |^voisin |^ol|^la |^ystävä |^toisil|^le. | \e
  \endverse
  \beginverse
    |^Kosketa |^minua, |^Hen|ki! |^Herätä |^kiittä|^mään, |^
    |^sinun |^lähel|^lä|^si |^armosta |^elä|^mään. | \e
  \endverse
\endsong


\beginsong{Mörri-Möykky}[by={Marjatta Pokela},ph={IV}]
  \newchords{chords_morrimoykky_a}\newchords{chords_morrimoykky_b}
  \beginverse\memorize[chords_morrimoykky_a]
    |\[\mnc{B}Em]Kor\[^\mn{A}]pi\[^\mn{G}]kuusen |\[\mnc{F#}Am]kannon \[\mnc{E}Em]alla on |\[\mnc{F#}B7]Mörri-\[^\mn{B}]Möy\[^\mn{D#}]kyn |\[\mnc{E}Em]kolo
  \endverse\glueverses\beginchorus\memorize[chords_morrimoykky_b]
    |\[Am]Siellä on koti ja |\[Em]siellä on peti
    ja |\[B7]peikolla pehmoinen |\[\up{1}Em\up{2}(E)]olo
  \endchorus
  \notesoff
  \beginverse
    \ind |\[E]Tiu tau tiu tau |tili tali tittan
    \ind |Sirkat soittaa |\[B]salolla
  \endverse\glueverses\beginchorus
    \ind |\[A]Pikkuiset peikot ne |\[E]piilossa pysyy
    \ind |\[B7]kirkkaalla päivän |\[E]valolla
  \endchorus\glueverses\beginverse
    \ind |\[Am] |\[Em] |\[B7] |\[Em]
  \endverse
  \beginverse\replay[chords_morrimoykky_a]
    |^Syksyn tullen |^sieniä ^kasvaa |^karhunkanka|^halla
  \endverse\glueverses\beginchorus\replay[chords_morrimoykky_b]
    |^Mörri-Möykky se |^sateessa istuu
    |^kärpässienen |^alla \goto{Tiu tau tittan}
  \endchorus
  \beginverse\replay[chords_morrimoykky_a]
    |^Ottaisin minä |^Mörri-^Möykyn, |^jos vain kiinni |^saisin
  \endverse\glueverses\beginchorus\replay[chords_morrimoykky_b]
    |^Pieneen koriin |^pistäisin ja
    |^kotiin kuljet|^taisin \goto{Tiu tau tittan}
  \endchorus
  \beginverse\replay[chords_morrimoykky_a]
    Vaan |^eipä taida |^meidän ^äiti |^peikkolasta |^ottaa
  \endverse\glueverses\beginchorus\replay[chords_morrimoykky_b]
    |^Eikä se edes |^usko, että
    |^Mörri-Möykky on |^totta \goto{Tiu tau tittan}
  \endchorus
\endsong

%%%%%%%%%%%%%%%%%%%%%%%%%%%%%%%%%%%%%%%%%%%%%%%%%%%%%%%%%%%%%%%%%%%
%%% LATEST PRINTOUT CONTAINED THE SONGS ABOVE.                  %%%
%%%%%%%%%%%%%%%%%%%%%%%%%%%%%%%%%%%%%%%%%%%%%%%%%%%%%%%%%%%%%%%%%%%
%%% Please try to not change the song numbers above this point. %%%
%%% Add new songs only after this point.                        %%%
%%%%%%%%%%%%%%%%%%%%%%%%%%%%%%%%%%%%%%%%%%%%%%%%%%%%%%%%%%%%%%%%%%%


\beginsong{Nouse luontoni lovesta}[by={Antti Tuonela},ph={I}]
  \beginverse
    |\[\mnc{A}Dm]Nouse \[\mn{G}]luon\[\mn{F}]toni |\[\mn{G}]lo\[\mn{D}]vesta \echo{|Nouse luontoni |lovesta}
    |Syntyni syvästä |maasta \echo{|Syntyni syvästä |maasta}
    |Nouse niin kuin |nousit ennen \echo{|Nouse niin kuin |nousit ennen}
    |Minun nostate|llessani \echo{|Minun nostate|llessani}
  \endverse
  \beginverse
    Nosta |\[Dm]Ukon voima |taivahas\sublyr{Nosta}ta \echo{|Ukon voima |taivahasta}
    |Maasta Maan E|moisen voima \echo{|Maasta Maan E|moisen voima}
    |Nouse niin kuin |nousit ennen \echo{|Nouse niin kuin |nousit ennen}
    |Minun nostate|llessani \echo{|Minun nostate|llessani}
  \endverse
  \beginchorus
    \ind |\[\mnc{G}D#]Tuekseni turvakseni |väekseni \[\mn{F}]voi\[\mn{D#}]makse|\[\mnc{D}Dm]ni | \e
    \ind |\[D#]Tuekseni turvakseni |väekseni voimakse|\[Dm]ni | \e
  \endchorus
  \textnote{\emph{D.C. al Fine}}
  \beginchorus
    |\[Dm]Nouse luontoni |lovesta \echo{|Nouse luontoni |lovesta}
    |\[Dm]Nouse luontoni |lovesta \echo{|Nouse luontoni |lovesta}
  \endchorus
\endsong


\beginsong{Koti mun luona}[by={Malla Maanpiiri}, ph={III, IV}]
  \beginchorus\memorize
    |\[\mnc{C}C]Sulla sulla |\[\mnc{G}G]siskokulta on
    |\[Am]aina koti mun |\[Em]luona | \e
    % Notes on the second line: C, E
  \endchorus
  \notesoff
  \beginchorus\replay
    \ind Me |^ollaan täällä |^näyttämässä,
    \ind |^mitä rakkaus |^on | \e
  \endchorus
  \beginchorus\replay
    |^Sulla sulla |^velikulta on
    |^aina koti mun |^luona | \e
  \endchorus
  \goto{Me ollaan täällä}
  \beginchorus\replay
    |^Sulla sulla |^lapsikulta on
    |^aina koti mun |^luona | \e
  \endchorus
  \goto{Me ollaan täällä}
\endsong


\beginsong{Siunattu voima}[by={Lotta Maija},ph={III}]
  \audio[key=Am]{https://www.youtube.com/watch?v=PHNaFQiBNuU}
  \mnbeginverse
    |\[\mncii{A}{B}Am]Illan \[^\mn{A}]suus\[^\mn{G}]sa |\[^\mn{A}]va\[^\mn{B}\mnc{C}]lon \[^\mn{B}]voi\[^\mn{G}]ma, |\emph{\[^\mn{A}\mn{B}\mn{C}]siunat\[^\mn{D}]tu |\[^\mn{E}]voima}
    |\[\mnc{D}G]Las\[^\mn{C}]ku \[^\mn{B}]au\[^\mn{A}]rin|koi\[^\mn{B}]sen \[^\mn{A}\mn{G}]illan, |\emph{\[\mnciii{A}{B}{A}\emph{Am}]siunat\[^\mn{G}]tu |v\[^\mn{A}]oima}
    |Aurinkoinen |lehvän leuto, |\emph{siunattu |voima}
    |\[G]Kesän henki |koivun hento, |\emph{\[\emph{Am}]siunattu |voima}
  \mnendverse
  \notesoff
  \beginverse
    |^Vetten päälle |valon loiste, |\emph{siunattu |voima}
    |^Valon loiste |sielun kirkkaus, |\emph{^siunattu |voima}
    |Kirkastelee |kimmellellen, |\emph{siunattu |voima}
    |^Valon kanssa |värähdellen, |\emph{^siunattu |voima}
  \endverse
  \beginverse
    |^Syömeen paistaa |herätellen, |\emph{siunattu |voima}
    |^Herätellen |sytytellen, |\emph{^siunattu |voima}
    |Ikiajan |valon voima, |\emph{siunattu |voima}
    |^Kipunoita |kaukaa tuolta, |\emph{^siunattu |voima}
  \endverse
  \beginverse
    |^Lämmön synnyn |loiste ompi, |\emph{siunattu |voima}
    |^Ajan takaa |ikuisempi, |\emph{^siunattu |voima}
  \endverse
\endsong

% \beginsong{O Kriste, kunnian kuningas}
%   \audio[key=Am]{https://www.youtube.com/watch?v=P_elenZRgAA}
%   \newchords{chords_okriste_a}\newchords{chords_okriste_b}
%   \meter{3}{4}
%   \beginverse\memorize[chords_okriste_a]
%     O |\[Am]Kriste, |\[C]kunnian |\[Dm]Kunin|\[Am]gas,
%     Ja |\[Dm]lunas|\[Am]taja |\[Em]laupi|\[Am]as!
%     Kuu|\[Am]le si|\[C]nua |\[Dm]rukoi|\[Am]len,
%     Ve|\[Dm]relläs’ |\[Am]ostet|\[Em]tu o|\[Am]len.
%   \endverse
%   \beginverse\memorize[chords_okriste_b]
%     \ind Suu|\[C]ret syn|\[F]tin’ kyl|\[G]läs tien|\[C]net,
%     \ind Joi|\[Dm]ta vas|\[Am]taas teh|\[Em]nyt lie|\[Am]nen,
%     \ind Koht’ |\[Am]äitin’ |\[C]kohdust’ |\[Dm]tultu|\[Am]an’,
%     \ind Ah |\[Dm]armahda |\[Am]päällen’, |\[Em]Juma|\[Am]la!
%   \endverse
%   \beginverse\replay[chords_okriste_a]
%     |^Muista, |^Herra, va|^las pääl|^len,
%     jon|^ka vah|^vast’ van|^noit meil|^len:
%     Et|^tes suo |^syntist’ |^hukku|^van,
%     vaan |^katu|^van ja |^elä|^vän.
%   \endverse
%   \beginverse\replay[chords_okriste_b]
%     \ind Tun|^nustan, |^synti|^nen o|^len,
%     \ind Huo|^kaan, ka|^dun ty|^kös tu|^len,
%     \ind Ja |^anteeks’ |^anta|^vas toi|^von,
%     \ind Mi|^tä vas|^taas ri|^koin vai|^voin.
%   \endverse
%   \beginverse\replay[chords_okriste_a]
%     Ku|^ningas |^kaikki|^valti|^as,
%     O |^Jesu |^aina |^laupi|^as!
%     Kuu|^le mi|^nua |^vaivais|^ta!
%     Ru|^koilen |^sinua |^hartaas|^ta,
%   \endverse
%   \beginverse\replay[chords_okriste_b]
%     \ind Ken |^paits’ |^sua minu|^a kuu|^lee?
%     \ind Ken |^apuun |^paits’ |^sinua tu|^lee?
%     \ind Jos et |^kuule, |^auta |^minu|^a,
%     \ind Ei |^muilta |^ole |^apu|^a.
%   \endverse
%   \beginverse\replay[chords_okriste_a]
%     Ol|^koon si|^nun, Je|^su kii|^tos!
%     Ai|^na ja |^ijät’ |^ylis|^tys,
%     Kuin |^kaikkein |^päälle |^armah|^dat,
%     Jotk’ |^tykös |^hartaast’ |^huuta|^vat.
%   \endverse
%   \beginverse\replay[chords_okriste_b]
%     \ind Se |^sama |^kunnia |^Isäl|^le,
%     \ind Ja |^ynnä |^Pyhäll’ |^Hengel|^le,
%     \ind Yh|^dell’ kol|^minai|^sell’ Her|^rall’,
%     \ind I|^jäisest’ |^aina |^hallitse|^vall’.
%   \endverse
%   \beginverse\replay[chords_okriste_a]
%     Ó |^Cristo, |^Rei da |^Glóri|^a
%     e |^misericordi|^oso |^Reden|^tor!
%     Es|^cute-me |^por |^fa|^vor
%     Fui |^comprado |^com seu |^san|^gue.
%   \endverse
%   \beginverse\replay[chords_okriste_b]
%     \ind Vo|^cê con|^hece meus |^grandes peca|^dos,
%     \ind que eu |^come|^ti con|^tra vo|^cê
%     \ind des|^de que |^saí do ú|^tero da minha |^mãe.
%     \ind Ah, |^tem misericór|^dia de |^mim, ó De|^us!
%   \endverse
%   \beginverse\replay[chords_okriste_a]
%     Lem|^bre-se, |^ó |^Se|^nhor,
%     o |^pacto |^que você |^fez por |^nós:
%     Que |^você não |^vai deixar um |^pecador mo|^rrer,
%     mas |^se ele se a|^rrepender, e|^le vive|^rá.
%   \endverse
%   \beginverse\replay[chords_okriste_b]
%     \ind Eu |^con|^fesso que sou |^um peca|^dor
%     \ind e |^eu suspiro, eu |^me arrependo |^e vou para vo|^cê
%     \ind pa|^ra buscar |^o seu |^perdã|^o,
%     \ind con|^tra o qual |^eu trans|^gre|^di.
%   \endverse
%   \beginverse\replay[chords_okriste_a]
%     Ó Rei Todo-Poderoso,
%     Jesus sempre misericordioso!
%     Ouça este pobre homem!
%     Eu te imploro com fervor,
%   \endverse
%   \beginverse\replay[chords_okriste_b]
%     \ind Quem mais pode me ouvir?
%     \ind Quem mais pode me ajudar?
%     \ind Se você não me ouvir e me ajudar,
%     \ind Ninguém mais pode me ajudar.
%   \endverse
%   \beginverse\replay[chords_okriste_a]
%     Louvado seja você, Jesus!
%     Louvado seja para todo o sempre
%     porque você é misericordioso
%     para todos aqueles que clamam por você.
%   \endverse
%   \beginverse\replay[chords_okriste_b]
%     \ind E glória ao Pai
%     \ind e ao Espírito Santo,
%     \ind para o Deus triuno
%     \ind que reina para sempre.
%   \endverse
% \endsong


\begin{intersong}
  % Original image downloaded from: https://commons.wikimedia.org/wiki/File:Harmonic_series_to_32.png
  % Image license: Public Domain
  % Edited by: larva
  \imagecc[0]{harmonic_series_to_32_PD__1641x1299px.png}
  % Harmonic series (sum):
  \begin{center}%
    $$\sum_{n=1}^{\infty} \frac{1}{n}$$
    \vfill
  \end{center}
\end{intersong}
