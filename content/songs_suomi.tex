% Finnish songs

\beginsong{Kalevala-sävelmä}[]
  \meter{2}{4}
  \beginverse
    |\[a]Va\[a]ka-|\[b]van\[b]ha |\[c]Väi\[a/e]nä|\[b]möi|\[b]nen|
    |\[c]Tie\[a]tä|\[d]jä \[c]i|\[b]än\[c]i|\[a]kui|\[a]nen|
  \endverse
  \musicnote{Haikea versio: d d d a d c b b ; d d d c b c a a}  
\endsong

\beginsong{Terve löyly}[]
  \chordsoff % do not show empty line for non-existing chords
  \beginverse
    Terve löyly, terve lämmin
    terve henkäys kiukainen,
    kylpy lämpimäin kivisten,
    hiki vanhan Väinämöisen.
    Löylystä vihannan vihdan,
    tervan voimasta terveiden.
  \endverse
  \beginverse
    Löyly kiukahan kivestä,
    löyly saunan sammalista.
    Tervehyttä tekemähän,
    rauhoa rakentamahan,
    kipehille voitehiksi,
    pahoille parantehiksi.  
  \endverse 
\endsong

\beginsong{Tule löylyhyn, Jumala}[]
  \chordsoff % do not show empty line for non-existing chords
  \musicnote{Melodia: Kalevala-sävelmä tai esim. Hedingarna: Täss' on nainen}
  \beginverse
    Tule löylyhyn, Jumala, 
    Iso ilman, lämpimähän,
  \endverse
  \beginverse
    Terveyttä tekemähän,
    Rauhoa rakentamahan
  \endverse
  \beginverse
    Lyötä maahan liika löyly
    Paha löyly pois lähetä
  \endverse
  \beginverse
    Ettei polta tyttöjäsi
    Turmele tekemiäsi
  \endverse
  \beginverse
    Minkä vettä viskaelen
    Noille kuumille kivillen
  \endverse
  \beginverse
    Se medeksi muuttukohon
    Simaksi sirahtakohon
  \endverse
  \beginverse
    Juoskohon joki metinen
    Simalampi laikkukohon
  \endverse
  \beginverse
    Läpi kiukahan kivisen
    Läpi saunan sammalisen! 
  \endverse 
\endsong

\beginsong{Tupakkarulla}
  \meter{2}{4}
  \beginverse
    |\[Dm]Tuu |tuu |tupakka|\[A]rul|la|
    |\[Gm]mistäs |\[Dm]tiesit |\[A7]tänne |\[Dm]tul|la?|
    |\[Dm]Tulin |pitkin |\[Gm7]Turun |\[A]tie|tä,|
    |\[A7]hämä|\[Dm]läisten |\[A7]härkä|\[Dm]tie|tä.|
  \endverse
  \beginverse
    |^Mistäs |tunsit |meidän |^por|tin?|
    |^Siitä |^tunsin |^uuden |^por|tin:|
    |^haka |alla, |^pyörä |^pääl|lä|
    |^karhun |^talja |^portin |^pääl|lä|
  \endverse
  \beginverse
    |^Uni |kysyi |uunin |^pääl|tä,|
    |^unen |^poika |^porstu|^as|ta:|
    |^Onko |lasta |^kätky|^es|sä,|
    |^pientä |^peittei|^den si|^säs|sä?|
  \endverse
  \beginverse
    |^Tuoppa |unta |tuokko|^ses|sa,|
    |^kanna |^vaski |^vakka|^ses|sa,|
    |^sillä |silmät |^sive|^le,| | 
    |^näky|^miset |^näppä|^e|le.|
  \endverse
  \beginverse
    |^Nuku |nuku |nurmi|^lin|tu,|
    |^väsy |^väsy |^västä|^räk|ki,|
    |^nuku |kun mi|^nä nu|^ku|tan,|
    |^väsy |^kun mi|^nä vä|^sy|tän.|
  \endverse
  \beginverse
    |^d | a | a a f | ^e | e|
    |^g g | ^a f | ^e f | ^d |d|
    |^f d | e f | ^g f | ^e | e|
    |^a g | ^f f | ^e f | ^d | d|
  \endverse
\endsong

\beginsong{Lampaanpolska \\ Kekrilaulu \\ Yksi kaksi kolme neljä}[]
  \meter{3}{4}
  \beginverse
    |\[Am]Yksi kaksi kolme |\[E]neljä,|
    |\[Am]anna i\[E7]lon |\[Am]olla.
    Ja |\[Am]kun suru |\[E]tulee,| 
    |\[Am]anna hä\[E]nen |\[Am]mennä.|
  \endverse
  \beginverse
    |\[Am|E]Paarmat ne |\[Dm]laulaa,|
    |\[C]kolme hiirtä |\[E]hyppelee.|
    |\[Am]Kissi lyöpi |\[E]trummun päälle
    ja |\[Am]koko maa\[E7]ilma |\[Am]pauhaa.| | 
  \endverse 
\endsong

\beginsong{Juurilaulu \\ Kuulumme piiriin}[]
  \beginverse
    \[a]Kuu\[a]lum\[a]me \[g]pi\[a]i\[a]riin
    \[a]Kuu\[a]lum\[b]me \[c]pi\[a]i\[a]riin
    \[a]Ai\[a]ko\[b]jen \[c]ta\[c]kaa 
    \[c]Ta\[b]kai\[g]sin \[a]kier\[a]toon
  \endverse
  \musicnote{Duuriversio: c c c b c c ; c c d e c c ; c c d e e e ; e d b b c}
\endsong

\beginsong{Laulu oravasta}[by=Aleksis Kivi]
  \chordsoff % do not show empty line for non-existing chords
  \beginverse
    Makeasti oravainen 
    Makaa sammalhuoneessansa; 
    Sinnepä ei Hallin hammas 
    Eikä metsämiehen ansa 
    Ehtineet milloinkaan.  
  \endverse
  \beginverse
    Kammiostaan korkeasta 
    Katselee hän mailman piirii,
    Taisteloa allans´ monta; 
    Havu-oksan rauhan-viiri 
    Päällänsä liepoittaa.
  \endverse
  \beginverse
    Mikä elo onnellinen
    Keinuvassa kehtolinnass´!
    Siellä kiikkuu oravainen
    Armaan kuusen äitinrinnass´:
    Metsolan kantele soi!
  \endverse
  \beginverse
    Siellä torkkuu heiluhäntä
    Akkunalla pienoisella,
    Linnut laulain taivaan alla 
    Saattaa hänen iltasella
    Unien Kultalaan.   
  \endverse
\endsong


\beginsong{Varjele vakainen luoja}[by={Kalevala, 43. runo}]
  \chordsoff % do not show empty line for non-existing chords
  \beginverse
    Anna Luoja, suo Jumala
    anna onni ollaksemme.
  \endverse
  
  \beginverse
    Hyvin ain’ eleäksemme,
    Kunnialla kuollaksemme.
  \endverse
  
  \beginverse
    Suloisessa Suomenmaassa
    Kaunihissa Karjalassa!
  \endverse
  
  \beginverse
    Varjele, vakainen Luoja
    Kaitse, kaunoinen Jumala,
  \endverse
  
  \beginverse
    Ole puolla poikiesi,
    Aina lastesi apuna,
  \endverse
  
  \beginverse
    Aina yöllisnä tukena,
    Päivällisnä vartiana.
  \endverse  
\endsong


\beginsong{Viatonten valssi}[by={Einojuhani Rautavaara, Eila Kivikk'aho}]
  \chordsoff % do not show empty line for non-existing chords
  \beginverse
    Kun kesäinen yö oli kirkkain ja tyyninä valvoivat veet
    ja helisi soittimet sirkkain kuin viulut ja kanteleet.
    Viisi pientä piru parkaa aivan ujoa ja arkaa
    sievin kumarruksin tohti käydä enkeleitä kohti.
  \endverse
  \beginverse
    Univormunsa karvaiset heitti, he sarvet ja saparovyön,
    oli lanteilla vain lukinseitti ja helisi harput yön.
    Enkelitkin sulkapaidan jätti tuonne, taakse aidan.
    Siellä häntä, siellä siipi toisiansa tervehtiipi.
  \endverse
  \beginverse
    Ja niinhän he, nostaen jalkaa, niin nätisti tanssia alkaa
    yli kallion kasteisen. Ja se yö oli onnellinen.
    Missäs sika, - jos ei kerää kärsäänsä se yhtäperää -
    siivet karvat, ynnä muuta, vielä maiskutellen suuta.
  \endverse
  \beginverse
    Sill' aikaa enkelit tanssi niin ujosti varpaillaan
    Vain pukuna pikkuinen kranssi, viis pirua toverinaan.
    Oi, pienoiset, ettehän arvaa, moni vaihtaa nahkaa ja karvaa.
    Mut harppua sirkat lyö, yhä kun on kesäyö.
  \endverse
  \beginverse
    Kerran tuli Aamunkoitto. Loppui tanssi, loppui soitto.
    Pirut, niinkuin enkelitkin, tunnusmerkkejänsä itki.  
  \endverse  
\endsong


\beginsong{Haltin häät}[by={Hannu Seppänen, Arto Alaspää}]
  \chordsoff % do not show empty line for non-existing chords
  \beginverse
    Kun ihmiskunnan aamu vasta alkoi sarastaa
    Ja Lappi oli jättiläisten maana
    Kaunis Malla-neito alkoi häitään valmistaa
    Sulhasenaan nuori uljas Saana
  \endverse
  \beginverse
    Kaikkialta kansaa saapui Haltiin juhlimaan
    Ja kirkonkellot häitä alkoi soittaa
    Silloin astui kirkkoon tumma Peltsa Ruotsinmaan
    Hän vaimokseen myös Mallan tahtoi voittaa
  \endverse
  \beginverse
    Hän aikoi estää häät ja kutsui velhot avukseen
    Ja pian saikin juhlakansa kuulla kauhukseen
    Kun pohjoisesta vyöryi jää ja yltyi tuuli
  \endverse
  \beginverse
    Kirkkokansa pakeni ja Mallaa sylissään
    Myös Saana alkoi juosten turvaan kantaa
    He kauas eivät ehtineet kun jäivät alle jään
    Ja jähmettyivät Kilpisjärven rantaan
  \endverse
  \beginverse
    On aikakaudet tuntureiksi heidät muuttaneet
    Ja Kilpisjärven kasvattaneet Mallan kyyneleet
    Kun jäinen pohjoistuuli soi myös itkee Saana  
  \endverse 
\endsong


\beginsong{Finlandia-hymni}[by={Jean Sibelius, Veikko Koskenniemi}]
  \chordsoff % do not show empty line for non-existing chords
  \beginverse
    Oi Suomi, katso, sinun päiväs koittaa,
    yön uhka karkoitettu on jo pois,
    ja aamun kiuru kirkkaudessa soittaa
    kuin itse taivahan kansi sois.
    Yön vallat aamun valkeus jo voittaa,
    sun päiväs koittaa, oi synnyinmaa!
  \endverse
  \beginverse
    Oi nouse, Suomi, nosta korkealle
    pääs seppelöimä suurten muistojen,
    oi nouse, Suomi, näytit maailmalle
    sa että karkoitit orjuuden
    ja ettet taipunut sa sorron alle,
    on aamus alkanut, synnyinmaa! 
  \endverse 
\endsong


\beginsong{Leppäkerttu}[by={Kansanlaulu}]
  \meter{4}{4}
  \beginverse
    |\[Dm]Lennä, lennä |\[A]leppäkerttu,|
    |\[Gm]ison \[Dm]kiven |\[A7]juu\[Dm]reen.|
    |\[Dm]Lennä leikki|\[Gm7]kedon \[A]kautta|
    |\[A7]unipuuhun |\[Dm]suureen.|
  \endverse
  \beginverse
    |^Kulta-kulta|^lehden alla|
    |^äiti ^puuron |^keit^tää.|
    |^Unituutu |^leppä^kertun|
    |^lämpimästi |^peittää.|
  \endverse
  \beginverse  
    |^Laula, laula, |^unilintu,|
    |^tuoksu, ^tuomen|^tert^tu.|
    |^Nuku, puna|^paitu^lainen,|
    |^pikku leppä|^kerttu.|
  \endverse
  \beginverse
    |^d d a a |^a a e e|
    |^g g ^f f |^e - ^d -|
    |^f d e f |^g f ^e e|
    |^a g f e |^d - d -|
  \endverse
\endsong



\beginsong{Taivas on sininen ja valkoinen}[by={Kansanlaulu}]
  \meter{2}{4}
  
  \beginverse
    |\[Am]Taivas on  \[Dm]sininen ja |\[E7]valkoi\[Am]nen ja
    |\[Am]tähtö\[Dm]si\[G7]ä|\[C]täyn\[Em]nä
    
    |\[Am]Niin on \[F]nuori |\[G7]sydä\[C]me\[F]ni
    |\[Am]aja\[E7]tuksia |\[Am]täynnä|
    
  \endverse

  \beginverse
    |Enkä mä muille ilmoita
    |mun sydänsuru|jani
    
    |Kirkas metsä ja| kirkas taivas
    |ne tuntee mun| huoliani
  \endverse

\endsong


\beginsong{Laulan sinulle lapsoseni}[by={MaKy}]
  \meter{4}{4}
 
  \beginverse
    Laulan sinulle lapsoseni
    laulan sinulle laulun
    |Kuule minua lapsoseni, kun
    |äitisi laulaa sulle
  \endverse

  \beginverse
    Missä ikinä kuljetkin
    siellä olen aina
    |Olen jalkojes alla
    |metsän puissa ja tuulessa
  \endverse

  \beginverse
    Vuoret on syntyneet kupeistani
    laaksot rintojen välistä
    |Meret ja joet kohdustani
    |veri on värjännyt maan
  \endverse

  \beginverse
    Hyvä sun on täällä kulkea
    maan ja taivaan väliä
    |Äitisi silittää varpaitasi ja
    |isäs silittää päätä
  \endverse

  \beginverse
    Ja jos sattuis lapseni käymään niin,
    että ilmaan tipahtaisit
    |Niin älä sinä lapseni huolta kanna
    |isäs ottaa kopin
  \endverse

  \beginverse
    Ei ole harha-askelia
    ei ole virheitä
    |Kauneutta kohti kuljet vain täällä
    |ikuisessa sylissä
  \endverse

  \beginverse
    Ja vielä kerron sinulle
    kuule vielä hetki
    |Aina oot ollut toivottu
    |ja tänne tervetullut
  \endverse

\endsong
