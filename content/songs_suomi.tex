% Finnish songs

\beginsong{Kalevala-sävelmä}[tags={(chords missing) 1}]
  \begin{lilypond}[fragment] % present melody using lilypond
    {\key a \minor \time 2/4
      a'4 a' | b' b' | c'' e'' | b'2 | b'2 |
      c''4 a'| d'' c'' | b' c'' | a'2 a'2 |
    }\addlyrics {Va -- ka van -- ha Väi -- nä -- möi -- nen tie -- tä -- jä i -- än i -- kui -- nen}
  \end{lilypond}
  \textnote{Haikea versio:}
  \begin{lilypond}[fragment] % present melody using lilypond
    {\key a \minor \time 2/4
      d''4 d'' | d'' a' | d'' c'' | b'2 | b'2 |
      d''4 d''| d'' c'' | b' c'' | a'2 a'2 |
    }\addlyrics {Va -- ka van -- ha Väi -- nä -- möi -- nen tie -- tä -- jä i -- än i -- kui -- nen}
  \end{lilypond}
\endsong


\beginsong{Juurilaulu \\ Kuulumme piiriin}[tags={piiri 1, (chords missing) 1}]
  \begin{lilypond}[fragment] % present melody using lilypond
    {\key a \minor \time 4/4
      a'4 a'8 a'8 g'8( a'8) a'4 | r1 |
      a'4 a'8 b'8 c''8( a'8) a'4 | r1 |
      a'4 a'8 b'8 c''8 c''8~ c''4 | r1 |
      c''8 b'4 g'8 g'4 a'4 | r1 |
    }\addlyrics {
      Kuu -- lum -- me pii -- riin
      Kuu -- lum -- me pii -- riin
      Ai -- ko -- jen ta -- kaa
      Ta -- kai -- sin kier -- toon}
  \end{lilypond}
\endsong


\beginsong{Lampaanpolska \\ Kekrilaulu \\ Yksi kaksi kolme neljä}[]
  \meter{3}{4}
  \beginverse
    |\[Am]Yksi kaksi kolme |\[E]neljä, |\[Am]anna i\[E7]lon |\[Am]olla.
    Ja |\[Am]kun suru |\[E]tulee, |\[Am]anna hä\[E]nen |\[Am]mennä. |
  \endverse
  \beginverse
    |\[Am/E]Paarmat ne |\[Dm]laulaa, |\[C]kolme hiirtä |\[E]hyppelee. |
    |\[Am]Kissi lyöpi |\[E]trummun päälle ja |\[Am]koko maa\[E7]ilma |\[Am]pauhaa. |
  \endverse 
\endsong


\beginsong{Laulu oravasta}[by={Aleksis Kivi, Otto Kotilainen, (Kaj Chydenius)}]
  \newchords{chords_orava_a}\newchords{chords_orava_b} % initialize chord registers
  \beginverse\memorize[chords_orava_a] % memorize chords into a named register
    Make|\[C]asti ora|\[Am]vainen
    Makaa |\[F]sammalhuonees|\[G]sansa;
    Sinne|\[Em]pä ei Hallin |\[Am]hammas
    Eikä |\[F]metsämiehen |\[G]ansa|\[G7] |
    |\[C]Ehtineet mil-|\[D7] \[G7]loin|\[C]kaan. |\[C7] -
  \endverse
  \beginverse\memorize[chords_orava_b] % memorize chords into a named register
    Kammi|\[F]ostaan korke|\[Em]asta
    Katse|\[Dm]lee hän |\[G7]mailman |\[C]piirii\[C7],
    Taiste|\[F]loa allans´ \[Em]monta;
    Havu|\[E&]oksan rauhan|\[Dm7]viiri \[G7] |
    |\[C]Päällänsä |\[D7]lie\[G7]poit|\[C]taa. | -
  \endverse
  \beginverse\replay[chords_orava_a] % replay chords from a named register
    Mikä |^elo onnel|^linen
    Keinu|^vassa kehtolin-|^ nass´!
    Siellä |^kiikkuu ora|^vainen
    Armaan |^kuusen äitin|^rinnass´: ^ |
    |^Metsolan kan-|^ ^tele |^soi! |^ -
  \endverse
  \beginverse\replay[chords_orava_b] % replay chords from a named register
    Siellä |^torkkuu heilu|^häntä
    Akku|^nalla ^pienoi|^sella, ^
    Linnut |^laulain taivaan |^alla
    Saattaa |^hänen iltasel-|^ la ^ |
    |^Unien |^Kul^ta|^laan. | |
  \endverse
\endsong


\beginsong{Taivas on sininen ja valkoinen}[by={Kansanlaulu}]
  \meter{2}{4}
  % declare new (global) named chord-replay registers:
  \newchords{chords_taivas_a}\newchords{chords_taivas_b}
  \beginchorus\memorize[chords_taivas_a] % memorize chords into a named register
    |\[Am]Taivas on sininen ja |\[E7]valkoi\[Am]nen ja |
    |\[Am]tähtö\[Dm]si\[G7]ä|\[C]täyn\[Em]nä |
  \endchorus
  \beginchorus\memorize[chords_taivas_b] % memorize chords into a named register
    |\[Am]Niin on \[F]nuori |\[G7]sydä\[C]me\[F]ni |
    |\[Am]aja\[E7]tuksia |\[Am]täynnä |
  \endchorus
  \vspace{1em}
  \beginchorus\replay[chords_taivas_a] % replay chords from a named register
    |^Enkä mä muille |^ilmoi^ta mun |
    |^sydän^su^ru|^ja^ni |
  \endchorus
  \beginchorus\replay[chords_taivas_b] % replay chords from a named register
    |^Synkkä ^metsä ja |^kirkas ^tai^vas ne |
    |^tuntee mun ^huoli|^ani |
  \endchorus
\endsong


\beginsong{Viatonten valssi}[by={Einojuhani Rautavaara, Eila Kivikk'aho},tags={(chords missing) 1}]
  \chordsoff % do not show empty line for non-existing chords
  \beginverse
    Kun kesäinen yö oli kirkkain ja tyyninä valvoivat veet
    ja helisi soittimet sirkkain kuin viulut ja kanteleet.
    Viisi pientä piru parkaa aivan ujoa ja arkaa
    sievin kumarruksin tohti käydä enkeleitä kohti.
  \endverse
  \beginverse
    Univormunsa karvaiset heitti, he sarvet ja saparovyön,
    oli lanteilla vain lukinseitti ja helisi harput yön.
    Enkelitkin sulkapaidan jätti tuonne, taakse aidan.
    Siellä häntä, siellä siipi toisiansa tervehtiipi.
  \endverse
  \beginverse
    Ja niinhän he, nostaen jalkaa, niin nätisti tanssia alkaa
    yli kallion kasteisen. Ja se yö oli onnellinen.
    Missäs sika, - jos ei kerää kärsäänsä se yhtäperää -
    siivet karvat, ynnä muuta, vielä maiskutellen suuta.
  \endverse
  \beginverse
    Sill' aikaa enkelit tanssi niin ujosti varpaillaan
    Vain pukuna pikkuinen kranssi, viis pirua toverinaan.
    Oi, pienoiset, ettehän arvaa, moni vaihtaa nahkaa ja karvaa.
    Mut harppua sirkat lyö, yhä kun on kesäyö.
  \endverse
  \beginverse
    Kerran tuli Aamunkoitto. Loppui tanssi, loppui soitto.
    Pirut, niinkuin enkelitkin, tunnusmerkkejänsä itki.  
  \endverse  
\endsong


\beginsong{Täss' on nainen}[by={Hedningarna},tags={(chords missing) 1}]
    % TODO: chords, melody (note: melody is good for Kalevala-type stuff)
  \meter{5}{8}
  \beginchorus
    \lrep |Täss'on nainen |tuulen tuoma |
    |tuulen tuoma |ve'en vetämä | \rrep
    \lrep |meren aalto|jen ajama |
    |meren tyrskyn |työntelemä | \rrep
  \endchorus
  \beginchorus
    \lrep |Kuin mie käynen |laulamahan |
    |laulan mie me|ret mesiksi | \rrep
    \lrep |suoloiksi me|ren somerot |
    |meren hiekat |hernehiksi | \rrep
  \endchorus
  \beginchorus
    \lrep |Yhen vyöni |vyötännällä |
    |yhen paita|ni panolla | \rrep
    \lrep |solkeni so|littamalla |
    |polkimeni |painamalla | \rrep
  \endchorus
  \beginverse
    \lrep |Nouse luonto|ni lovesta |
    |syntyni sy|västä maasta | \rrep
    \lrep |syntyni sy|västä maasta |
    |haavan alta |haltiainen! | \rrep
  \endverse
\endsong


\beginsong{Sadelaulu}[by={Sanna Kurki-Suonio},tags={vesi 1}]
  % notes: |\[d]Sa\[a]de \[a]\[a]syök\[a]sy\[e]y|\[f]vi \[a]sy\[g]li-\[f]i\[e]hin |
  %        |\[a]Aa-\[b]a\|[g#]aa-\[e]a|\[a]aa\[g#]a\[a]a|\[b]a\[c]a\|[b]a\[a]a\[b]|a\|[g#]aa\[e]a|
  \beginverse
    |\[Dm]Sade syöksyy|\[F]vi sy\[C]lihin |\[Dm]pisaraiset |\[F]paian \[C]päälle |
    |\[Dm]Vesi vihmo|\[F]en ve\[C]tävi |\[Dm]kaiken alleen |\[F]kaste\[C]levi |
    |\[Dm]Minä vain sa|\[F]teessa \[C]seison |\[Dm]satehessa |\[F]suloi\[C]sessa |
  \endverse
  \beginverse
    |^Oi kaalinna... |^ ^ |^ |^ ^ |
  \endverse
  \beginverse
    |^Vesi mulle |^voiman ^tuopi |^voiman vahvan |^ja vä^kevän |
    |^Pyyhkii pois pö|^lyiset ^mietteet |^ajatukset |^auvot^taapi |
  \endverse
  \beginverse
    \chorusindent |\[Am]Aaa... |\[G#] |\[Am] |\[E] | \rep{2}
  \endverse
  \beginverse
    |^Ukko heittävi |^vasa^moitaan |^säästele ei |^sala^moitaan |
    |^Minä vain sa|^teessa ^seison |^satehessa |^suloi^sessa |
  \endverse
  \beginverse
    |^Oi kaalinna... |^ ^ |^ |^ ^ | \rep{2}
  \endverse
  \beginverse
    \chorusindent |\[Am]Aaa.. |\[G#] |\[Am] |\[E] | \rep{4}
  \endverse
  \beginverse
    |^Puhdista ve|^si puh^dista |^ajatukse|^ni kir^kasta |
    |^Syän surusta |^sulat^tele |^tuskan tunteet |^tunnol^tani |
    |^Aatteet alhaiset |^aivois^tani |^puhdista pi|^sara ^pieni |
  \endverse
  \beginverse
    \chorusindent |\[Am]Aaa.. |\[G#] |\[Am] |\[E] | \rep{2}
  \endverse
  \beginverse
    |^Virtaa vesi, |^vihmo ^vesi |^voimaa tuot sä |^mulle ^vesi |
    |^Virtaa vesi, |^vihmo ^vesi |^voimaa tuot sä |^mulle ^vesi |
  \endverse
  \beginverse
    |^Oi kaalinna... |^ ^ |^ |^ ^ | \rep{5}
    | | | | |
  \endverse
\endsong


\beginsong{Haltin häät}[by={Hannu Seppänen, Arto Alaspää}]
  \beginverse\memorize
    Kun |\[C]ihmiskunnan aamu vasta |\[Em]alkoi sarastaa
    Ja |\[F]Lappi oli jättiläisten |\[G]maana |
    |\[C]Kaunis Malla-neito alkoi |\[Em]häitään valmistaa |
    |\[F]Sulhasenaan nuori uljas |\[G]Saana |
  \endverse
  \beginverse
    |^Kaikkialta kansaa saapui |^Haltiin juhlimaan
    Ja |^kirkonkellot häitä alkoi |^soittaa |
    |^Silloin astui kirkkoon tumma |^Pältsa Ruotsinmaan |
    Hän |^vaimokseen myös Mallan tahtoi |^voittaa
  \endverse
  \beginverse
    Hän |\[Am]aikoi estää häät ja kutsui |\[Em]velhot avukseen
    Ja |\[Am]pian saikin juhlakansa |\[Em]kuulla kauhukseen
    Kun |\[F]pohjoisesta vyöryi |\[C/G]jää ja yltyi |\[G]tuuli |
  \endverse
  \beginverse
    |^Kirkkokansa pakeni ja |^Mallaa sylissään
    Myös |^Saana alkoi juosten turvaan |^kantaa |
    He |^kauas eivät ehtineet kun |^jäivät alle jään
    Ja |^jähmettyivät Kilpisjärven |^rantaan
  \endverse
  \beginverse
    On |\[Am]aikakaudet tuntureiksi |\[Em]heidät muuttaneet
    Ja |\[Am]Kilpisjärven kasvattaneet |\[Em]Mallan kyyneleet
    Kun |\[F]jäinen pohjoistuuli |\[C/G]soi myös itkee |\[G]Saana |
  \endverse 
\endsong


\beginsong{Finlandia-hymni}[by={Jean Sibelius, Veikko Koskenniemi},tags={(chords missing) 1}]
  % \chordsoff % uncomment to not display measure bars nor an empty line for non-existing chords
  \beginverse
     Oi Suomi, |katso, |sinun päiväs' |koittaa, |
    | yön uhka |karkoi|tettu on jo |pois, |
    | ja aamun |kiuru |kirkkaudessa |soittaa |
    | kuin itse |taiva|han kansi |sois. |
    | Yön vallat |aamun |valkeus jo |voittaa, |
    | sun päiväs |koittaa, |oi synnyin|maa! | -
  \endverse
  \beginverse
    Oi nouse, |Suomi, |nosta korke|alle |
    | pääs' seppe|löimä |suurten muisto|jen, |
    | oi nouse, |Suomi, |näytit maail|malle |
    | sa että |karkoi|tit orjuu|den |
    | ja ettet |taipu|nut sa sorron |alle, |
    | on aamus |alka|nut, synnyin|maa! | -
  \endverse 
\endsong


\beginsong{Päivänsäde ja menninkäinen}[by={Reino Helismaa}]
  \beginverse
    |\[Am]Aurinko kun \[Dm]päätti retken, |\[Am]siskoistaan jäi \[Dm]jälkeen hetken |
    |\[Am]päivänsäde \[E7]viimei|\[Am]nen. |
    |\[Dm]Hämärä jo \[Am]metsään hiipi, |\[Dm]päivänsäde \[Am]kultasiipi |
    |\[D]juuri aikoi \[D7]lentää eestä |\[G]sen,
    kun |\[C]menninkäisen \[Am]pienen näki |\[Dm]vastaan tule\[G7]van;
    se |\[C]juuri oli \[D7]noussut luolas|\[G]taan.
    Kas |\[C]menninkäinen \[C7]ennen päivän |\[F]laskua ei \[F#\textdegree7]voi
    mil|\[C]loinkaan olla \[Dm7]pääl\[G7]lä  |\[C]maan. |
  \endverse
  \beginverse
    |^Katselivat ^toisiansa; |^menninkäinen ^rinnassansa |
    |^tunsi kummaa ^leiskun|^taa. |
    |^Sanoi: poltat ^silmiäni, |^mut' en ole ^eläissäni |
    |^nähnyt mitään ^yhtä iha|^naa!
    Ei |^haittaa vaikka ^loisteesi mun |^sokeaksi ^saa;
    on |^pimeässä ^hyvä asus|^taa.
    Käy |^kanssani, niin ^kotiluolaan |^näytän sulle ^tien
    ja |^sinut armaak^se^ni  |^vien! |
  \endverse
  \beginverse
    |^Säde vastas: ^peikko kulta, |^pimeys vie ^hengen multa, |
    |^enkä toivo ^kuole|^maa. |
    |^Pois mun täytyy ^heti mennä, |^ellen kohta ^valoon lennä, |
    |^niin en hetke^äkään elää |^saa.
    Niin |^lähti kaunis ^päivänsäde, |^mutta vielä^kin,
    kun |^menninkäinen ^öisin tallus|^taa,
    hän |^miettii, miksi ^toinen täällä |^valon lapsi ^on,
    ja |^toinen yötä ^ra^kas|^taa. |
  \endverse
\endsong


\beginsong{Laulan sinulle lapsoseni}[by={MaKy},tags={Äiti Maa 1}]
  \meter{4}{4}
  %\chordsoff % do not show space in place of non-existing chords
  \beginverse
    |\[Am]Laulan \[G]sinulle |\[Am]lapsoseni |\[Em]laulan sinulle |\[Am]laulun |
    \lrep |\[Am]Kuule \[G]minua |\[Am]lapsoseni, kun |\[Em]äitisi laulaa |\[Am]sulle | \rrep
  \endverse
  \beginverse
    |^Missä ^ikinä |^kuljetkin |^siellä olen |^aina |
    \lrep |^Olen ^jalkojes |^alla |^metsän puissa ja |^tuulessa | \rrep
  \endverse
  \beginverse
    |^Vuoret on ^syntyneet |^kupeistani |^laaksot rintojen |^välistä |
    \lrep |^Meret ja ^joet |^kohdustani |^veri on värjännyt |^maan | \rrep
  \endverse
  \beginverse
    |^Hyvä sun on ^täällä |^kulkea |^maan ja taivaan |^väliä |
    \lrep |^Äitisi ^silittää |^varpaitasi ja |^isäs silittää |^päätä  \rrep
  \endverse
  \beginverse
    Ja |^jos sattuis ^lapseni |^käymään niin, että |^ilmaan tipah|^taisit
    \lrep Niin |^älä sinä ^lapseni |^huolta kanna |^isäs ottaa |^kopin \rrep
  \endverse
  \beginverse
    |^Ei ole ^harha-|^askelia |^ei ole |^virheitä |
    \lrep |^Kauneutta ^kohti |^kuljet vain täällä |^ikuisessa |^sylissä | \rrep
  \endverse
  \beginverse
    Ja |^vielä ^kerron |^sinulle |^kuuntele hetki |^vielä |
    \lrep |^Aina oot ^ollut |^toivottu ja |^tänne terve|^tullut | \rrep
  \endverse
\endsong


\beginsong{Leppäkerttu}[by={Kansanlaulu},tags={uni 1}]
  \meter{4}{4}
  \beginverse
    |\[Dm]Lennä, lennä |\[A]leppäkerttu, |\[Gm]ison \[Dm]kiven |\[A7]juu\[Dm]reen. |
    |\[Dm]Lennä leikki|\[Gm7]kedon \[A]kautta |\[A7]unipuuhun |\[Dm]suureen. |
  \endverse
  \beginverse
    |^Kulta-kulta|^lehden alla |^äiti ^puuron |^keit^tää. |
    |^Unituutu |^leppä^kertun |^lämpimästi |^peittää. |
  \endverse
  \beginverse
    |^Laula, laula, |^unilintu, |^tuoksu, ^tuomen|^tert^tu. |
    |^Nuku, puna|^paitu^lainen, |^pikku leppä|^kerttu. |
  \endverse
  \vfill%
  \hfill\includegraphics[width=0.08\textwidth]{leppakerttu_354x280px.png}
  \begin{lilypond}[fragment] % present melody using lilypond
    {\key d \minor \time 4/4
      d'4 d'4 a'4 a'4 | a'4 a'4 e'4 e'4 |
      g'4 g'4 f'4 f'4 | e'2 d'2 |
      f'4 d'4 e'4 f'4 | g'4 f'4 e'4 e'4 |
      a'4 g'4 f'4 e'4 | d'2 d'2 |
    }\addlyrics {
      Len -- nä, len -- nä lep -- pä -- kert -- tu, |
      i -- son ki -- ven juu -- reen. |
      Len -- nä leik -- ki -- ke -- don kaut -- ta |
      u -- ni -- puu -- hun suu -- reen. | }
  \end{lilypond}
\endsong


\beginsong{Tupakkarulla}[by={Kansanlaulu},tags={uni 1}]
  \meter{2}{4}
  \beginverse
    |\[Dm]Tuu |tuu |tupakka|\[A]rul|la |
    |\[Gm]mistäs |\[Dm]tiesit |\[A7]tänne |\[Dm]tul|la? |
    |\[Dm]Tulin |pitkin |\[Gm7]Turun |\[A]tie|tä, |
    |\[A7]hämä|\[Dm]läisten |\[A7]härkä|\[Dm]tie|tä. |
  \endverse
  \beginverse
    |^Mistäs |tunsit |meidän |^por|tin? |
    |^Siitä |^tunsin |^uuden |^por|tin: |
    |^haka |alla, |^pyörä |^pääl|lä |
    |^karhun |^talja |^portin |^pääl|lä |
  \endverse
  \beginverse
    |^Uni |kysyi |uunin |^pääl|tä, |
    |^unen |^poika |^porstu|^as|ta: |
    |^Onko |lasta |^kätky|^es|sä, |
    |^pientä |^peittei|^den si|^säs|sä? |
  \endverse
  \beginverse
    |^Tuoppa |unta |tuokko|^ses|sa, |
    |^kanna |^vaski |^vakka|^ses|sa, |
    |^sillä |silmät |^sive|^le,| | 
    |^näky|^miset |^näppä|^e|le. |
  \endverse
  \beginverse
    |^Nuku |nuku |nurmi|^lin|tu, |
    |^väsy |^väsy |^västä|^räk|ki, |
    |^nuku |kun mi|^nä nu|^ku|tan, |
    |^väsy |^kun mi|^nä vä|^sy|tän. |
  \endverse
  \begin{center}%
    \vfill%
    \includegraphics[width=0.382\textwidth]{sleeping_baby_bw_transparent_bg_1280px.png}%
    \vfill%
  \end{center}
  \begin{lilypond}[fragment] % present melody using lilypond
    {\key d \minor \time 2/4
      d'2 | a'2 | a'8 a'4 f'8 | e'2 | e'2 |
      g'4 g'4 | a'4 f'4 | e'4 f'4 | d'2 | d'2 |
      f'4 d'4 | e'4 f'4 | g'4 f'4 | e'2 | e'2 |
      a'4. g'8 | f'4 f'4 | e'4 f'4 | d'2 | d'2 |
    }\addlyrics {
      Tuu tuu tu -- pak -- ka -- rul -- la,
      mis -- täs tie -- sit tän -- ne tul -- la?
      Tu -- lin pit -- kin Tu -- run tie -- tä,
      hä -- mä -- läis -- ten här -- kä -- tie -- tä }
  \end{lilypond}
\endsong


\beginsong{En etsi valtaa loistoa}[by={Jean Sibelius, Sakari Topelius},tags={(chords missing) 1}]
  \meter{4}{4}
  \beginverse
    En etsi valtaa, loistoa, en kaipaa kultaakaan,
    mä pyydän taivaan valoa ja rauhaa päälle maan.
    Se joulu suo, mi onnen tuo ja mielet nostaa Luojan luo.
    Ei valtaa eikä kultaakaan, vaan rauhaa päälle maan.
  \endverse
  \beginverse
    Suo mulle maja rauhaisa ja lasten joulupuu,
    Jumalan sanan valoa, joss´ sieluin kirkastuu.
    Tuo kotihin, nyt pieneenkin, nyt joulujuhla suloisin,
    Jumalan sanan valoa ja mieltä jaloa.
  \endverse
  \beginverse
    Luo köyhän niinkuin rikkahan saa, joulu ihana!
    Pimeytehen maailman tuo taivaan valoa!
    Sua halajan, sua odotan, sä Herra maan ja taivahan.
    Nyt köyhän niinkuin rikkaan luo suloinen joulus tuo!
  \endverse 
\endsong


\beginsong{Kosketa minua henki}[by={Ilkka Kuusisto 1979},ex={Virsi 125}]
  \transpose{2} % transpose to C for easier chords
  \meter{3}{4}
  \beginverse
    |\[B&]Kosketa |\[Dm]minua, |\[Gm]Hen|ki, |\[Cm7]kosketa |\[Dm]kirkka|\[E&]us!|\[G7/D] |
    |\[Cm]Anna |\[F7]elä|\[B&maj7]mäl|\[G]le |\[Cm7]suunta ja |\[F7]tarkoi|\[B&]tus. | |
  \endverse
  \beginverse
    |^Kosketa, |^Jumalan |^Hen|ki, |^syvälle |^sydä|^meen. |^ |
    |^Sinne |^paina |^hil|^jaa |^luottamus |^rakkau|^teen. | |
  \endverse
  \beginverse
    |^Rohkaise |^minua, |^Hen|ki, |^murenna |^pelko|^ni. |^ |
    |^Tässä |^maail|^mas|^sa |^osoita |^paikka|^ni. | |
  \endverse
  \beginverse
    |^Valaise, |^Jumalan |^Hen|ki, |^silmäni |^aukai|^se, |^ |
    |^että |^voisin |^ol|^la |^ystävä |^toisil|^le. | |
  \endverse
  \beginverse
    |^Kosketa |^minua, |^Hen|ki! |^Herätä |^kiittä|^mään, |^ |
    |^sinun |^lähel|^lä|^si |^armosta |^elä|^mään. | |
  \endverse
\endsong


\beginsong{Suojelusenkeli}[by={P. J. Hannikainen},tags={suojelus 1}]
  \meter{3}{8}
  \beginverse
    Maan |\[Em]korvessa |kulkevi |\[B]lapsosen |\[Em]tie.
    \[B]Hänt' |\[Em]ihana |\[G]enkeli |\[D]kotihin |\[G]vie.
    Niin |\[Em]pitkä \[C]on |\[D]matka, \[B]ei |\[Em]kotia |\[B]näy, | -
    vaan |\[Em]ihana |\[C]enke\[B]li |\[Em]vieres\[Am]sä |\[B]käy,
    vaan |\[Em]ihana |\[Am]enkeli |\[Em/B]vieres\[B7]sä |\[Em]käy.
  \endverse
  \beginverse
    On |^pimeä |korpi ja |^kivinen |^tie,
    ^ja |^usein se |^käytävä |^liukaskin |^lie.
    Oi, |^pian^han |^lapso^nen |^langeta |^vois, | -
    jos |^käsi ei |^enke^lin |^kädes^sä |^ois,
    jos |^käsi ei |^enkelin |^kädes^sä |^ois.
  \endverse
  \beginverse
    Ja |^mielikin |mustia |^verkkoja |^vaan  % original: Ja syntikin mustia verkkoja vaan
    ^on |^laajalle |^laskenut |^korpehen |^maan.
    Niin |^pian^han |^niihin^kin |^tarttua |^vois, | -
    jos |^käsi ei |^enke^lin |^kädes^sä |^ois,
    jos |^käsi ei |^enkelin |^kädes^sä |^ois.
  \endverse
  \beginverse
    Maan |^korvessa |kulkevi |^lapsosen |^tie.
    ^Hänt' |^ihana |^enkeli |^kotihin |^vie.
    Oi, |^laps' et^hän |^milloin^kaan |^ottaa sä |^vois | -
    sä |^kättäsi |^enke^lin |^kädes^tä |^pois.
    sä |^kättäsi |^enkelin |^kädes^tä |^pois.
  \endverse
\endsong

