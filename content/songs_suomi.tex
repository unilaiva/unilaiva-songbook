% Finnish songs

\beginsong{Kalevala-sävelmä}[]
  \beginverse
    \[a]Va\[a]ka-\[b]van\[b]ha \[c]Väi\[a/e]nä\[b]möi\[b]nen
    \[c]Tie\[a]tä\[d]jä \[c]i\[b]än\[c]i\[a]kui\[a]nen
  \endverse
  \musicnote{Haikea versio: d d d a d c b b ; d d d c b c a a}  
\endsong

\beginsong{Terve löyly}[]
  \chordsoff % do not show empty line for non-existing chords
  \beginverse
    Terve löyly, terve lämmin
    terve henkäys kiukainen,
    kylpy lämpimäin kivisten,
    hiki vanhan Väinämöisen.
    Löylystä vihannan vihdan,
    tervan voimasta terveiden.
  \endverse
  \beginverse
    Löyly kiukahan kivestä,
    löyly saunan sammalista.
    Tervehyttä tekemähän,
    rauhoa rakentamahan,
    kipehille voitehiksi,
    pahoille parantehiksi.  
  \endverse 
\endsong

\beginsong{Tule löylyhyn, Jumala}[]
  \chordsoff % do not show empty line for non-existing chords
  \musicnote{Melodia: Kalevala-sävelmä tai esim. Hedingarna: Täss' on nainen}
  \beginverse
    Tule löylyhyn, Jumala, 
    Iso ilman, lämpimähän,
  \endverse
  \beginverse
    Terveyttä tekemähän,
    Rauhoa rakentamahan
  \endverse
  \beginverse
    Lyötä maahan liika löyly
    Paha löyly pois lähetä
  \endverse
  \beginverse
    Ettei polta tyttöjäsi
    Turmele tekemiäsi
  \endverse
  \beginverse
    Minkä vettä viskaelen
    Noille kuumille kivillen
  \endverse
  \beginverse
    Se medeksi muuttukohon
    Simaksi sirahtakohon
  \endverse
  \beginverse
    Juoskohon joki metinen
    Simalampi laikkukohon
  \endverse
  \beginverse
    Läpi kiukahan kivisen
    Läpi saunan sammalisen! 
  \endverse 
\endsong

\beginsong{Tupakkarulla}
  \chordsoff % do not show empty line for non-existing chords
  \beginverse
    Tuu tuu tupakkarulla
    mistäs tiesit tänne tulla?
    Tulin pitkin turun tietä,
    hämäläisten härkätietä.
  \endverse
  \beginverse
    Mistäs tunsit meidän portin?
    Siitä tunsin uuden portin:
    haka alla, pyörä päällä
    karhun talja portin päällä
  \endverse
  \beginverse
    Uni kysyi uunin päältä,
    unen poika porstuasta:
    Onko lasta kätkyeessä,
    pientä peitteiden sisässä?
  \endverse
  \beginverse
    Tuoppa unta tuokkosessa,
    kanna vaski vakkasessa,
    sillä silmät sivele, 
    näkymiset näppäele.
  \endverse
  \beginverse
    Nuku nuku nurmilintu,
    väsy väsy västäräkki,
    nuku kun minä nukutan,
    väsy kun minä väsytän.
  \endverse
\endsong

\beginsong{Lampaanpolska \\ Kekrilaulu \\ Yksi kaksi kolme neljä}[]
  \chordsoff % do not show empty line for non-existing chords
  \beginverse
    Yksi kaksi kolme neljä,
    anna ilon olla.
    Ja kun suru tulee, 
    anna hänen mennä.
  \endverse
  \beginverse
    Paarmat ne laulaa,
    neljä hiirtä hyppelee.
    Kissi lyöpi trummun päälle
    ja koko maailma pauhaa. 
  \endverse 
\endsong

\beginsong{Juurilaulu \\ Kuulumme piiriin}[]
  \beginverse
    \[a]Kuu\[a]lum\[a]me \[g]pi\[a]i\[a]riin
    \[a]Kuu\[a]lum\[b]me \[c]pi\[a]i\[a]riin
    \[a]Ai\[a]ko\[b]jen \[c]ta\[c]kaa 
    \[c]Ta\[b]kai\[g]sin \[a]kier\[a]toon
  \endverse
  \musicnote{Duuriversio: c c c b c c ; c c d e c c ; c c d e e e ; e d b b c}
\endsong

\beginsong{Laulu oravasta}[by=Aleksis Kivi]
  \chordsoff % do not show empty line for non-existing chords
  \beginverse
    Makeasti oravainen 
    Makaa sammalhuoneessansa; 
    Sinnepä ei Hallin hammas 
    Eikä metsämiehen ansa 
    Ehtineet milloinkaan.  
  \endverse
  \beginverse
    Kammiostaan korkeasta 
    Katselee hän mailman piirii,
    Taisteloa allans´ monta; 
    Havu-oksan rauhan-viiri 
    Päällänsä liepoittaa.
  \endverse
  \beginverse
    Mikä elo onnellinen
    Keinuvassa kehtolinnass´!
    Siellä kiikkuu oravainen
    Armaan kuusen äitinrinnass´:
    Metsolan kantele soi!
  \endverse
  \beginverse
    Siellä torkkuu heiluhäntä
    Akkunalla pienoisella,
    Linnut laulain taivaan alla 
    Saattaa hänen iltasella
    Unien Kultalaan.   
  \endverse
\endsong

\beginsong{Haltin häät}[by={Hannu Seppänen, Arto Alaspää}]
  \chordsoff % do not show empty line for non-existing chords
  \beginverse
    Kun ihmiskunnan aamu vasta alkoi sarastaa
    Ja Lappi oli jättiläisten maana
    Kaunis Malla-neito alkoi häitään valmistaa
    Sulhasenaan nuori uljas Saana
  \endverse
  \beginverse
    Kaikkialta kansaa saapui Haltiin juhlimaan
    Ja kirkonkellot häitä alkoi soittaa
    Silloin astui kirkkoon tumma Peltsa Ruotsinmaan
    Hän vaimokseen myös Mallan tahtoi voittaa
  \endverse
  \beginverse
    Hän aikoi estää häät ja kutsui velhot avukseen
    Ja pian saikin juhlakansa kuulla kauhukseen
    Kun pohjoisesta vyöryi jää ja yltyi tuuli
  \endverse
  \beginverse
    Kirkkokansa pakeni ja Mallaa sylissään
    Myös Saana alkoi juosten turvaan kantaa
    He kauas eivät ehtineet kun jäivät alle jään
    Ja jähmettyivät Kilpisjärven rantaan
  \endverse
  \beginverse
    On aikakaudet tuntureiksi heidät muuttaneet
    Ja Kilpisjärven kasvattaneet Mallan kyyneleet
    Kun jäinen pohjoistuuli soi myös itkee Saana  
  \endverse 
\endsong

\beginsong{Finlandia-hymni}[by={Jean Sibelius, Veikko Koskenniemi}]
  \chordsoff % do not show empty line for non-existing chords
  \beginverse
    Oi Suomi, katso, sinun päiväs koittaa,
    yön uhka karkoitettu on jo pois,
    ja aamun kiuru kirkkaudessa soittaa
    kuin itse taivahan kansi sois.
    Yön vallat aamun valkeus jo voittaa,
    sun päiväs koittaa, oi synnyinmaa!
  \endverse
  \beginverse
    Oi nouse, Suomi, nosta korkealle
    pääs seppelöimä suurten muistojen,
    oi nouse, Suomi, näytit maailmalle
    sa että karkoitit orjuuden
    ja ettet taipunut sa sorron alle,
    on aamus alkanut, synnyinmaa! 
  \endverse 
\endsong