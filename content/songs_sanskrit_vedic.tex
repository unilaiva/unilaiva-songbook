% Sanskrit language bhajans and mantras from the Indian subcontinent
% ==================================================================
%
% The following sets the song number for the first song in this file.
% The number will automatically be incremented by one for each song.
% Please do not change this! Changing would make different versions of
% the songbook to have different numbers for the same songs, and it
% would totally mess up the selection booklets causing them to have
% wrong songs in them. (For the same reason, add new songs only to the
% end of each songs_ file.)
\setcounter{songnum}{400}


\beginsong{Pūrṇamadaḥ}[by={trad., Shantala},ex={from Upaniṣad, the first prayer},ph={I, II},tags={unity},key={Am},gk={Am, any}]
  \audio[key=Am]{https://www.youtube.com/watch?v=S0nUTYWqlJA}
  \audio[key=Am]{https://soundcloud.com/yogacat-1/purnamadah}
  \audio[key=Am]{https://soundcloud.com/kirtanforthespirit/purnamadah}
  \mnbeginverse
    |\[\mnc{A}Am]Pūr\[\mn{B}]ṇa\[\mn{C}]ma\[\mnc{B}Am/B]daḥ |\[\mnc{A}Am]Pūr\[\mn{B}]ṇa\[\mn{C}]mi\[\mnc{B}Am/B]dam \altchords{\id[1]{(Em) \capo{5}}|Em Em/F\shrp{} |Em Em/F\shrp{}}
    |\[\mnc{A}Am]Pūr\[\mn{B}]ṇa\[\mncii{C}{B}Am/B]{t Pūr}ṇa|\[\mnc{A}Am]mu\[\mn{C}]dacya\[\mnc{B}Am/B]te \altchords{|Em Em/F\shrp{} |Em Em/F\shrp{}}
    |\[\mnc{A}Am]Pūr\[\mn{B}]ṇas\[\mncii{C}{B}Am/B]{ya Pūr}\[\mn{A}]ṇa|\[\mn{B}]mā\[\mn{C}]dā\[Fmaj7]ya \altchords{|Em Em/F\shrp{} | - Cmaj7}
    |\[\mnc{A}Am]Pūr\[\mn{B}]ṇa\[\mn{C}\mncadj{2ex}{B}Am/B]mevā\[\mn{A}]va|\[\mnc{C}Fmaj7]shiṣya\[\mn{B}]te\[G6] \altchords{|Em Em/F\shrp{} |Cmaj7 D6}
  \mnendverse
  \mnbeginverse
    |\[\mnc{A}Am]Oṃ \[\bm]{}{ } |\[Fmaj7]{} \[G6]{} |\[Am]{} \[\bm]{}{ } |\[Fmaj7]{} \[G6] \altchords{|Em |Cmaj7 D6 |Em |Cmaj7 D6}
  \mnendverse
  \begin{lilywrap}\begin{lilypond}[]
    \include "tex/lp-include-head.ly"
    theMelody = \relative a' {
      \key a \minor \time 4/4
      \set melismaBusyProperties = #'()
      \repeat volta 2 {
        | a4 b8 c b2 | a4 b8 c b2
        | a4 b4( c8) b4 b8 | a8 c4 c8 b2
        | a4 b4 c8 b4 a8 | b4 c4 c2
        | a4 b8 c4  b4 a8 | c4 c8 b8~2
        | a1~ | a1 | r1 | r1
      }
    }
    theLyricsOne = \lyricmode {
      \repeat volta 2 {
        | Pūr -- ṇa -- ma -- daḥ | Pūr -- ṇa -- mi -- dam;
        | Pūr -- ṇa -- t Pūr -- ṇa | mu -- da -- cya -- te;
        | Pūr -- ṇas -- ya Pūr -- ṇa | mā -- dā -- ya;
        | Pūr -- ṇa -- me -- vā -- va | shiṣ -- ya -- te. _
        | Oṃ
      }
    }
    theChords = \chordmode {
      \repeat volta 2 {
        | a2:m a2:m/b | a2:m a2:m/b
        | a2:m a2:m/b | a2:m a2:m/b
        | a2:m a2:m/b | a2:m/b f2:maj7
        | a2:m a2:m/b | f2:maj7 g:6
        | a1:m | f2:maj7 g2:6 | a1:m | f2:maj7 g2:6
      }
    }
    \layout { #(layout-set-staff-size 14) } % for better fit
   \include "tex/lp-include-tail-notab.ly"
  \end{lilypond}\end{lilywrap}
  \begin{feeler}
    That is Pūrṇa; this is Pūrṇa.\\
    From Pūrṇa comes Pūrṇa.\\
    Taking Pūrṇa from Pūrṇa, Pūrṇa remains.\\
  \end{feeler}
  \begin{explanation}[EN]
    \begin{description}
      \item[Pūrṇa:] full with Divine Consciousness
    \end{description}
    The outer world is full with Divine Consciousness \emph{(Pūrṇa)};
    The inner world also is full with Divine Consciousness.
    From the fullness of Divine Consciousness the world is manifested.
    Taking \emph{Pūrṇa} from \emph{Pūrṇa}, \emph{Pūrṇa} remains,
    (because Divine Consciousness is non-dual and infinite).
  \end{explanation}
\endsong


\beginsong{Mahāmṛtyuñjaya Mantra \\ Oṃ Tryámbakaṃ}[ph={III},tags={health, liberation},ex={from Rigveda 7.59.12}]
  \beginverse% \quad on the first line is there to not extend melody notes to the next bar
    |\[\mnc{E}C]Oṃ Tryámba\[\mn{D}]ka|\[G]ṃ \[\mn{C}]Ya\[\mn{D}]jā\[\mn{E}]mah|\[\mnciii{D}{C}{A}Dm]e \quad |\[Fmaj7]{} \e
    |\[C]{} Sugandhíṃ |\[G]Puṣṭivárdha|\[Dm]nam | \e
    Ur|\[F]vārukam Íva |\[C]Bandhánān Mṛ|\[G]tyor | \e
    Muk|\[Dm]ṣīya Mā 'Mrtā́t \echo{Muk|ṣīya Mā 'Mṛtā́t}
    Muk|\[G]ṣīya Mā 'Mrtā́t \echo{Muk|ṣīya Mā 'Mṛtā́t}
  \endverse
  \begin{feeler}
    We Meditate on the Three-eyed reality\\
    Which permeates and nourishes all like a fragrance.\\
    May we be liberated from death for the sake of immortality,\\
    Even as the cucumber is severed from bondage to the creeper.
  \end{feeler}
\endsong


\beginsong{Mahāmṛtyuñjaya Mantra 2 \\ Oṃ Tryámbakaṃ 2}[ph={III}, tags={health, liberation}, key={Am}, gk={Cm, Am--Em},ex={from Rigveda 7.59.12}]
  \beginverse
    |\[\mnc{A}Am]Oṃ \[\mn{B}]Tr\[\mn{C}]yám\[\mn{B}]ba\[\mn{A}]ka|ṃ \[\mn{B}]Ya\[\mn{C}]jā\[\mn{A}]ma\[\mn{B}]h|\[Em]e
    Sugandhíṃ |Puṣṭivárdhana|\[Dm]m
    Urvārukam |Íva Bandhánā|\[G]n
    Mṛtyor |Mukṣīya Mā 'Mṛ|\[Am]tā́t | \e
  \endverse
  \begin{explanation}[EN]
    Mahāmṛtyuñjaya Mantra \emph{(lit. ``Great death-defeating mantra``)} is one of the more potent
    of the ancient Sanskrit mantras. Maha Mrityunjaya is a call for enlightenment and is a practice
    of purifying the karmas of the soul at a deep level. It is also said to be quite beneficial for
    mental, emotional, and physical health.
  \end{explanation}
\endsong


\beginsong{Moola Mantra}[index={Om Satchitananda Parabrahma}, tags={source}, ph={I, II}, key={Dm}, gk={Cm, Cm--F\shrp{}m}]
  \transpose{5}\preferflats
  \mnbeginchorus
    \[\mn{E}]Om |\[\mnc{A}Am]Satchitanan\[\mn{G}]da |\[\mn{A}]Para\[\mn{B}]brah\[\mn{A}]ma \altchords{\id[1]{(Am)}|Am | \e}
    |Purushothama Pa\[\mn{G}]ra|\[\mn{A}\mn{B}]mat\[\mn{A}]ma \altchords{| - | \e}
    \[\mn{C}]Sri |\[\mnc{D}F]Bhagavati \[\mn{C}]Sa\[\mn{B}]{me}|\[\mncii{A}{G}G]tha \altchords{|F |G}
    \[\mn{B}]Sri |\[\mnc{A}Am]Bhagavate \[\mn{B}]Na\[\mn{A}]ma|ha \altchords{|Am | \e}
  \mnendchorus
  \mnbeginchorus
    \[\mn{E}]Hari |\[\mnc{A}Am]Om \[\mn{B}]Tat |\[\mn{A}]Sat, Hari |\[\mnc{B}G]Om \[\mn{C}]Tat |\[\mn{B}]Sat \altchords{|Am | - |G | \e}
    \[\mn{A}]Ha\[\mn{B}]ri |\[\mnc{C}F]Om \[\mn{D}]Tat |\[\mnc{B}G]Sat, \[\mn{C}]Ha\[\mn{B}]ri |\[\mncii{A}{G}Em]Om \[\mn{B}]Tat |\[\mnc{A}Am]Sat \altchords{|F |G |Em |Am}
  \mnendchorus
  \begin{feeler}
    Oh Divine Force, Spirit of All Creation,\\
    Highest Personality, Divine Presence,\\
    manifest in every living being.\\
    Supreme Soul manifested\\
    as the Divine Mother and\\
    as the Divine Father.\\
    I bow in deepest reverence.\\
  \end{feeler}
  \begin{explanation}[EN]
    Moola mantra evokes the living God, asking protection and freedom from all sorrow
    and suffering. It is a prayer that adores the great creator and liberator, who out of love and
    compassion manifests, to protect us, in an earthly form.  The calmness that the mantra can
    give is to be experienced, not spoken about. Here is the key with which any door to spiritual
    treasure could be opened. A tool which can be used to achieve all desires. A medicine which
    cures all ills. Just like when you call a person he comes and makes you feel his presence, the
    same manner when you chant this mantra, the supreme energy manifests everywhere around you. As
    the Universe is Omnipresent, the supreme energy can manifest anywhere and any time. It is also
    very important to know that the invocation with all humility, respect and with great necessity
    makes the presence stronger.
    \begin{description}
      \item[Om:] Calling on the highest energy of all there is. It is said 'In the beginning was the
        Supreme word and the word created every thing. That word is Om'. If you are meditating in
        silence deeply, you can hear the sound Om within. The whole creation emerged from the sound
        Om. It is the primordial sound or the Universal sound by which the whole universe vibrates.
        This divine sound has the power to create, sustain and destroy, giving life and movement to
        all that exist.
      \item[Sat:] Truth. The formless. The all penetrating existence that is formless, shapeless,
        omnipresent, attributeless, and qualityless aspect of the Universe, experienced as emptiness
        of the Universe. The body of the Universe that is static. Everything that has a form and can
        be sensed evolved out of this. So subtle that it is beyond all perceptions. It can only be
        seen when it has become gross and has taken form. We are in the Universe and the Universe is
        in us. We are the effect and Universe is the cause and the cause manifests itself as the
        effect.
      \item[Chit:] The Pure Consciousness of the Universe that is infinite, omni-present
        manifesting power of the Universe. Out of this is evolved everything that we call Dynamic
        energy or force. It can manifest in any form or shape. It is the consciousness manifesting
        as motion, as gravitation, as magnetism, etc. Also manifesting as the actions of the body,
        as thought force. The Supreme Spirit.
      \item[Ananda:] Pure bliss, love, joy and friendship nature of the Universe. When you experience
        either the Supreme Energy in this Creation (Sat) and become one with the Existence or
        experience the aspect of Pure Consciousness (Chit), you enter into a state of Divine Bliss
        and eternal happiness (Ananda).
      \item[Parabrahma:] The Supreme creator being in his Absolute aspect; beyond space and time.
        The essence of the Universe that is with and without form.
      \item[Purushothama:] The energy that incarnates as an Avatar in human form to help and guide
        mankind and relate closely to the beloved creation.  This has different meanings. Purusha
        means soul and Uthama means the supreme, the Supreme spirit. It also means the supreme
        energy of force guiding us from the highest world. Purusha also means Man, and Purushothama
        is the energy that incarnates as an Avatar to help and guide Mankind and relate closely to
        the beloved Creation.
      \item[Paramatma:] Supreme inner energy that is immanent in every creature and in all beings,
        living and non-living. Who comes to me in my heart, and becomes my inner voice whenever I
        ask. It's the force that can come to you whenever you want and wherever you want to guide
        and help you.
      \item[Sri Bhagavathi:] The divine mother, the power aspect of creation. The female aspect,
        which is characterized as the Supreme Intelligence in action, the Power (The Shakti). It is
        referred to the Mother Earth (Divine Mother) aspect of the creation.
      \item[Sametha:] Together or in communion with.
      \item[Sri Bhagavathe:] The Male aspect of the Creation, which is unchangeable and permanent.
      \item[Namaha:] Salutations or prostrations to the Universe that is Om and also has the
        qualities of Sat Chit Ananda, that is omnipresent, unchangeable and changeable at the same
        time, the supreme spirit in a human form and formless, the indweller that can guide and help
        in the feminine and masculine forms with the supreme intelligence. I thank you and
        acknowledge this presence in my life. I seek your presence and guidance all the time.
      \item[Hari om tat sat:] God is the truth. Hari is another name of Lord Vishnu.
    \end{description}
  \end{explanation}
\endsong


\beginsong{Gāyatrī Mantra \\ Sāvitrī Mantra}[index={Om Bhūr Bhuvaḥ Svaha},tags={wisdom, liberation, source, Sun},ph={III},ex={from Rigveda 3.62.10},key={Am},gk={Bm, Am--Em}]
  \mnbeginverse
    |\[\mnc{A}Am]Oṃ |\[\mnc{B}G]Bhūr Bhuvaḥ \[\mn{C}]Sva|\[\mncii{B}{A}Am]ha \altchords{\id[1]{(Bm)}|Bm |A |Bm}
    \[\mn{C}]Tat |\[\mnc{B}G]Sa\[\mn{C}]vi|\[\mncii{B}{A}Am]tur \[\mn{C}]Va|\[\mnc{B}G\mn{A}\mn{G}\mn{A}]reṇ|\[F]yaṃ \altchords{|A |Bm |A |G}
    Bhar|\[\mnc{B}G]gho De\[\mn{A}]va\[\mn{B}]sya |\[\mnc{C}C]Dhī\[\mn{D}]ma\[\mn{E}]hi \altchords{|A |D}
    Dhi|\[\mncii{D}{B}G]yo \[\mn{D}]Yo |\[\mncadj{1ex}{C}C]Naḥ Pra|\[\mnc{B}G]cho\[\mn{C}]da|\[\mncii{B}{A}Am]yāt \altchords{|A |D |A |Bm}
  \mnendverse
  \begin{feeler}
    We meditate on the brilliance of that\\
    Being who has produced this universe;\\
    may She enlighten our minds.
  \end{feeler}
  \begin{explanation}[EN]
    A prayer of praise that awakens the vital energies and gives liberation and deliverance from
    ignorance; it directs one's energies from harsh towards subtle. This mantra is known to impart
    wisdom, understanding, and enlightenment. This is said to be the oldest and most powerful of
    mantras, being thousands of years old. It purifies the person chanting it as well as the
    listener as it creates a tangible sense of well being in whoever comes across it.

    We meditate on that most adorable, desirable and enchanting luster and brilliance of
    our Supreme Being, our Source Energy, our Collective Consciousness who is our creator,
    inspirer and source of eternal Joy. May this Light inspire and guide our mind and open
    our hearts. That Divine Illumination which pervades the physical plane, astral plane and
    the celestial plane. That which is the most adorable. On that Divine Radiance we Meditate.
    May that Enlighten our Intellect and Awaken our Spiritual Wisdom.

    The mantra has been translated in many ways. The literal meaning of the words are below:
    \begin{description}
      \item[Oṃ:] the sacred syllable, pranava;
      \item[Bhūr:] Bhuloka (physical plane);
      \item[Bhuvaḥ:] Antariksha (space);
      \item[Svaha:] Svarga Loga (Heaven);
      \item[Tat:] that;
      \item[Savitur:] vivifying power of the Sun, source of life (also identified with Surya, Sun God);
      \item[Vareṇyaṃ:] greatest;
      \item[Bhargho:] brilliance;
      \item[Devasya:] of a Divine Entity;
      \item[Dhīmahi:] ``we meditate upon'' (knowledge imparted/understood);
      \item[Dhiyo:] Buddhi (intellect, our understanding of reality);
      \item[Yo:] which;
      \item[Naḥ:] our;
      \item[Prachodayāt:] enlighten, inspire, propel.
    \end{description}
  \end{explanation}
  \yesendsongvfill
\endsong


\beginsong{Pavamāna Mantra \\ Oṃ Asato Mā}[ex={from Bṛhadāraṇyaka Upaniṣad},tags={transcendence},ph={I, II, III},key={Dm},gk={Dm, Cm--F\shrp{}m}]
  \mnbeginchorus\memorize
    |\[\mnc{D}Dm]Oṃ \[\bm]{}\[^\mn{C}]{ } |\[\mnc{A}Am] asa\[\bmc\mn{D}]to \[^\mn{E}]mā |\[\mnc{D}Dm]sadga\[^\mn{C}]ma\[\bmc\mn{D}]ya, |\[Am]\e
    \[^\mn{A}]Tama\[\bmc\mn{D}]so \[^\mn{E}]mā |\[\mnc{F}F]jyoti\[\bm]r\[^\mn{G}]ga\[^\mn{F}]ma|\[\mnc{E}C]ya, \[\bm]
    Mṛ\[^\mn{F}]t|\[\mnc{E}Am]yor\[^\mn{D}]mā\[\bmc\mn{C}]{'m}\[^\mn{E}]ṛ|\[\mnc{D}Dm]taṃ ga\[^\mn{C}]ma\[\bmc\mn{D}]ya. | \e\[\bm]
  \mnendchorus
  \notesoff
  \textnotefornext{suomeksi:}
  \beginchorus
    |^Joh^{-}da|^ta ^minut |^totuu^teen, |^ \e
    ^Pimeydes|^tä ^kirkkau|^teen,
    ^Tiedotto|^muudes^ta tie|^toisuu^teen. | \e^
  \endchorus
  \begin{lilywrap}\begin{lilypond}[]
    % transcribed by larva, latest update on 2024-02
    % based on Deva Premal's version, but not exactly it
    \include "tex/lp-include-head.ly"
    % \header {
    %   title = "Pavamāna Mantra"
    % }
    theMelody = \relative d'' {
      \key d \minor \slurDashed
      \set melismaBusyProperties = #'()
      %\tempo 4 = 110
      \time 4/4
      \repeat volta 2 {
        | d2.( c4 | a4)( a8 a8) d4 e4 | d8( d8)( d8)( c8) d2~ | d4 \parenthesize a8 \parenthesize a8
        d8( d8) e4 | f4( f4)( f4) g8( f8) | e2( e8 e8)( e8)( f8)
        | e4 d4 c4 e4 | d4 d8( c8) d2~ | d1
      }
      % Deva Premal's version:
      % \repeat volta 2 {
      %   | d2.( c4 | a4)( a8 a8) d4 e4 | d8( d8)( d8)( c8) d2~ | d4 \parenthesize a8 \parenthesize a8
      %   d8( d8) e4 | d2 d4 c8( c8) | a2( a8) e'8( e8)( e8)
      %   | e4 d4 c4 e4 | d4 d8( c8) d2~ | d1
      % }
      \fine
    }
    theLyricsOne = \lyricmode {
      \set stanza = "1."
      \repeat volta 2 {
        | Om __ _ | _ a -- sa -- to mā | sad -- _ ga -- ma -- ya, | _
        Ta -- ma -- so __ _ mā | jyo -- tir -- _ ga -- ma -- | ya, __ _ _
        Mṛt -- _ | yor -- mā -- 'm -- ṛ -- | taṃ ga -- ma -- ya. | _
      }
    }
    theLyricsTwo = \lyricmode {
      % Finnish translation and phrasing by S.
      \set stanza = "2."
      \repeat volta 2 {
        | Joh -- da -- | ta __ _ _ mi -- nut | to -- tuu -- _ _ teen, | _ \skip 1 \skip 1
        Pi -- mey -- des -- | tä __ _ kirk -- kau -- _ | teen,
        Tie -- dot -- _ to -- | muu -- des -- ta tie -- | toi -- suu -- _ teen. | _
      }
    }
    theChords = \chordmode {
      \repeat volta 2 {
        | d1:m | a:m | d:m | a:m
        | f1 | c | a:m | d:m | d:m
      }
      % % Deva Premal's version:
      % \repeat volta 2 {
      %   | d1:m | a:m | d:m | a:m
      %   | f | a:m | a:m | d:m | d:m
      % }
    }
    \layout { #(layout-set-staff-size 15) } % for better fit
   \include "tex/lp-include-tail-notab.ly"
  \end{lilypond}\end{lilywrap}
  \begin{translation}
    Lead me from illusion to reality \emph{(of eternal self)},
    from darkness \emph{(ignorance)} to light \emph{(spiritual understanding)},
    from \emph{(the world of)} death to immortality \emph{(of self-realization)}.
  \end{translation}
\endsong


\beginsong{Prabhu Aap Jago}[by={trad., version by: Carioca},tags={transcendence},ph={III}, key={A},gk={A, G--B--(C)}]
  \audio[key={C}]{https://soundcloud.com/canal-chandra/prabhu-aap-jago}
  \newchords{chords_prabhu_a}\newchords{chords_prabhu_b}
  \transpose{-3} % to A, where the notes range from A to A'
  \mnbeginchorus\memorize[chords_prabhu_a]
    \[^\mn{C}]Pra\[^\mn{D}]bhu |\[\mnc{E}C]Aap Ja\[\bm]go \[^\mn{D}]Pra\[^\mn{E}]bhu |\[\mnc{F}Fmaj7]Aap \[^\mn{E}]Ja\[\bmc\mn{C}]go \altchords{\id[1]{(C)}|C |Fmaj7}
    Prabhu |\[\mnc{D}Dm]Aap Ja\[\bm]go \[^\mn{F}]Para|\[\mnc{E}G]math\[^\mn{D}]ma Ja\[\bm]go \altchords{|Dm |G}
  \mnendchorus
  \mnbeginverse\memorize[chords_prabhu_b]
    \[^\mn{G}]Mere |\[Em]Sar\[^\mn{E}]va Ja\[\bm]go \[^\mn{G}]Sar|\[\mnc{C}Am]va\[^\mn{B}]tra \[^\mn{A}]Ja\[\bmc\mnc{E}]go \altchords{|Em |Am}
    \[^\mn{F}]Prabhu |\[Dm]Aap \[^\mn{E}]Ja\[\bmc\mn{D}]go \[^\mn{F}]Para|\[\mnc{G}G]math\[^\mn{F}]ma \[^\mn{E}]Ja\[\bmc\mn{D}]go \altchords{|Dm |G}
    \[^\mn{G}]Mere |\[Em]Sar\[^\mn{E}]va Ja\[\bm]go \[^\mn{G}]Sar|\[\mnc{C}Am]va\[^\mn{B}]tra \[^\mn{A}]Ja\[\bmc\mn{E}]go \altchords{|Em |Am}
    \[^\mn{F}]Prabhu |\[Dm]Aap \[^\mn{E}]Ja\[\bmc\mn{D}]go \[^\mn{F}]Para|\[\mnc{G}G]math\[^\mn{E}]ma\[\bm]{} \[^\mn{D}]Ja|\[\mnc{C}C]go \[\bm] | { }{ } \[\bm]\e \altchords{|Dm |G |C | \e}
  \mnendverse
  \notesoff
  \textnotefornext{in English:}
  \beginchorus\replay[chords_prabhu_a]
    |^Cease the ^cause of |^sufferin^g
    Illumi|^nate the ^cause of |^joy ^
  \endchorus
  \beginverse\replay[chords_prabhu_b]
    |^Cease the ^cause of |^sufferin^g
    Illumi|^nate the ^cause of |^love ^
    |^Cease the ^cause of |^sufferin^g
    Illumi|^na^{-te} the |^ca^use of |^love^ | { }{ } ^\e
  \endverse
  \begin{feeler}
    God awaken, God awaken in me, God awaken everywhere.\\
    May love awaken; may love awaken everywhere.
  \end{feeler}
\endsong


\beginsong{Shakti Kundalini \\ Om Mata Om Kali}[tags={Divine Mother},ph={I, IV}, key={Dm}, gk={Dm, Gm--F\shrp{}m}]
  \meter{4}{4}
  \mnbeginchorus
    \[\mn{F}]Om |\[\mncii{E}{D}Dm]Mata \[\mn{F}]Om |\[\mn{E}\mn{D}]Kali \altchords{\id[1]{(Am)}|Am | \e}
    |\[\mnc{C}C]Durga \[\mn{E}]Devi \[\mn{F}]na|\[\mnc{E}Dm]mo \[\mn{D}]namaha \altchords{|G |Am}
  \mnendchorus
  \mnbeginchorus
    |\[\mnc{D}Dm]Shakti \[\mn{E}]kun\[\mn{D}]da|\[\mnc{C}C]lini |\[\mnc{B&}B&]Jaga\[\mn{D}]dam\[\mn{C}]be \[\mn{B&}]Ma|\[\mnciii{A}{C#}{E}A]ta \altchords{|Am |G |F |E}
    |\[\mnc{D}Dm]Shakti \[\mn{E}]kun\[\mn{F}]da|\[\mncii{G}{F}C]li\[\mn{E}]ni |\[\mnc{D}B&]Jaga\[\mn{F}]dam\[\mnc{E}Am]be \[\mn{C}]Ma|\[\mnc{D}Dm]ta \altchords{|Am |G |F Em |Am}
  \mnendchorus
  \begin{feeler}
    I bow unto the Divine Mother and Her many feminine aspects: Kali, remover of delusion and
    ignorance; Divine Goddess Durga; Shakti, universal life force and consort to Shiva; and
    Kundalini, the Goddess energy that rises within us. Praise to the Mother of the World!
  \end{feeler}
\endsong


\beginsong{Jay Shri Ma \\ Ananda Ma \\ Kali Ma}[by={trad., Sangita}, tags={Divine Mother}, ph={IV},key={B phr.}, gk={D phr., A phr. -- F\shrp{} phr.}]
  \audio[key={B phr.}]{https://soundcloud.com/yopi-1/jay-shri-ma-live-with-family}
  \mnbeginchorus
    |\[Em]{ }{ }{ } \[\bmc\mn{B}]Jay \[\mn{C}]Shri |\[\mnc{D}G]Ma \[\bm]Kali \[\mn{C}]Ka\[\mn{B}]li |\[\mnc{A}Am]Ma \[\bm]Jay \[\mn{C}]Shri |\[\mnc{B}Em/B]Ma \[\bm]
  \mnendchorus
  \mnbeginchorus
    |\[G]{} \[\mn{D}]Ananda \[\bmc\mn{E}\mn{D}]Ma |\[D]Durga \[\mn{E}]De\[\bmc\mn{D}]vi |\[\mnc{C}Am]Jaga\[\mn{B}]dam\[\bmc\mn{A}]be \[\mn{C}]Shri |\[\mnc{B}Em/B]Ma \[\bm]
  \mnendchorus
\endsong


\beginsong{Jay Ambe}[tags={Divine Mother},ph={II, IV}]
  \beginchorus
    |\[\mnc{D}Dm]Jay Am\[\mn{A}]be |\[\mnc{G}C]Jaga\[\mn{A}]dam\[\mnc{E}Am]be
    |\[F]Mata Bha\[C]vani ki |\[Dm]Jay Ambe
  \endchorus
  \beginchorus
    |\[Dm]{} Durgati Nashini |\[F]Durga Jaya Jaya
    |\[C]{} Kala Vinashini |\[Dm]Kali Jaya Jaya
  \endchorus
  \beginchorus
    |\[C]Uma Rama Brah|\[F]mani Jaya Jaya
    |\[C]Radha \[Am]Rukamani |\[Dm]Sita Jaya Jaya
  \endchorus
\endsong


\beginsong{Saraswati}[tags={Divine Mother, learning},ph={II, III}]
  \beginverse
    \[\mn{D}]Sa\[\mn{E}]ras|\[\mncii{F}{E}Dm]wa\[\mn{D}]ti | \[\mn{D}]Ma\[\mn{E}]ha|\[\mn{F}\mn{E}]lax\[\mn{D}]mi | \e
    Durga |\[F]Devi |\[C]Nama|\[Dm]ha | \e
  \endverse
  \beginverse
    Saras|\[Gm]wati | Maha|\[F]laxmi | \e
    Durga |\[B&]Devi |\[C]Nama|\[Dm]ha | \e
  \endverse
\endsong

\beginsong{Devi Mantra \\ Sarva Mangala}[tags={Divine Mother, Shiva},ph={III, IV}]
  % \capo{3}
  \textnotefornext{part A: Devi Mantra}
  \beginchorus
    |\[\mnc{A}Am]Sarva Mangala |Man\[\mn{C}]ga\[\mn{A}]ly\[\mnc{B}Em]e
    |\[G]Shive Sarv|\[Em]artha Sadhi\[Am]ke
    |Sharanye Tryambake |Gau\[Em]ri
    Nara|\[Am]yan\[G]i Na|\[Em]mostut\[Am]e
    Nara|\[Am]yan\[G]i Na|\[Em]mostut\[Am]e | | \e
  \endchorus
  \textnotefornext{part B: Om Namah Shivaya}
  \beginchorus
    \ind |\[Am]Om Namah Shivaya, |\[C]Om \[G]Nama Shiva \rep{3}
  \endchorus\glueverses\beginverse
    \ind |\[Am]Om Namah Shivaya, |\[C]Om \[G]Nama Shiv|\[Am]aya \up{1}(| \e)
  \endverse
  \dacapo
  \beginchorus
    \ind |\[Am]Shivaya, Shivaya, |\[Em]Shivay\[Am]a \rep{3}
  \endchorus\glueverses\beginverse
    \ind |Shivaya, Shivaya
  \endverse\glueverses\beginchorus
    \ind |\[C]Om \[G]Namah Shiv|\[Am]aya
  \endchorus
  \begin{feeler}
     \textbf{Devi Mantra:}\par
     Welcome to you O Narayani; who is the positiveness in all the auspicious,
     one who is so auspicious herself and has all auspicious qualities,
     The provider of protection, the one with three eyes and a beautiful face;
     we salute you, O Narayani.
  \end{feeler}
\endsong


\beginsong{Mataji}[by={trad., Elisabet Just},tags={Divine Mother},ex={saṃskṛtam, português},ph={IV}]
  \audio[]{https://www.youtube.com/watch?v=\_UEUQZsaDZ4}
  %\capo{3}
  \beginverse
    |\[\mnc{A}Am]Ayi Ayi Ayi Ayi Di|wa\[\mn{B}]li \[\mn{A}]Hai \[\mn{G}]Ye \[\mn{A}\mn{B}\mn{C}]Ayi
    |\[G]Aise Shubhawa\[C]sar Par Hum |\[Em]Puje Maha \[Am]Lakshmi
    |\[Am]Sabhi Devi Devata |Aapa Hi Ko Puje
    |\[C]Nirmala \[E]Ma, O Maiya |\[G]Nirmala \[Am]Ma
  \endverse\glueverses
  \beginchorus
    O Chindra|wara, Wali Maha, Laksh|\[(C)]mi Mata\[Am]ji
  \endchorus
  \beginchorus
    Ô \sublyr{\up{2}(mi)}iê iê i|\sublyr{Ayi Ayi Ayi}\[Am]ê, ô \sublyr{Ayi}iê \sublyr{Di-}iê |\sublyr{wali}Xo\sublyr{Hai}roo\sublyr{Ye}dô \sublyr{A-}
    Ô \sublyr{y-}iê \sublyr{i}iê i|\sublyr{Ayi Ayi Ayi}\[Am]ê, ô \sublyr{Ayi}iê \sublyr{Di-}iê |\sublyr{wali}Xo\sublyr{Hai}roo\sublyr{Ye}dô \sublyr{A-}
    Olomi ai|\sublyr{Aise}\[G]ê \sublyr{Shubhawa}Xorô \sublyr{sar}\[C]{} \sublyr{Par}Ó\sublyr{Hum}manfé|\sublyr{Pu-}\[Em]é \sublyr{je}Xo\sublyr{Ma-}roo \sublyr{ha}dô\sublyr{Laksh-}\[Am]
  \endchorus\glueverses
  \beginverse
    Ô \sublyr{mi}iê iê i|\sublyr{Sabhi}\[Am]ê, \sublyrpush{Devi De}\sublyr{va-}ô \sublyr{ta}iê iê |\sublyr{Aapa}Xo\sublyr{Hi Ko}roodô\sublyr{Puje}
    Ô iê iê i|\sublyr{Nir-}\[C]ê, \sublyrpush{mala} \sublyr{Ma,}\[E]{} \sublyr{O}ô \sublyr{Mai-}iê \sublyr{ya}iê |\sublyr{Nir-}\[G]Xo\sublyr{mala}roodô\sublyr{Ma}\[Am]
  \endverse\glueverses
  \beginchorus
    Ô iê iê i|\sublyr{Ayi Ayi Ayi}\[Am]ê, ô \sublyr{Ayi}iê \sublyr{Di-}iê |\sublyr{wali}Xo\sublyr{Hai}roo\sublyr{Ye}dô \sublyr{A(yi)}
  \endchorus
  \beginchorus\musicnotefornext{decelerando}
    \sublyrpush{Ô iê iê i} |\sublyr{ê,}\[Am]Om Mani Pad\sublyr{ô}me \sublyr{iê iê}Hum |\sublyr{Xoroodô} \e
  \endchorus
  \beginchorus\rep{4}
    \ind \sublyrpush{\up{\textbf{1}}Ô iê iê i} |\sublyr{ê,}\[Am]Om Mani Pad\sublyr{ô}me \sublyr{iê iê}Hum |\sublyr{Xo}\[G]Om \sublyr{roo}Ma\sublyr{dô}ni Pad\sublyr{ô}me \sublyr{iê iê i}Hum
    \ind |\sublyr{ê}\[F]Om Mani \[Em]Pa\sublyr{ô iê iê}dme |\sublyr{Xoroodô}\[Am]Hum
  \endchorus
  \beginchorus\rep{4}
    |\[Am]Om Om Mani |\[G]Padme \[Am]Hum
  \endchorus
  \beginchorus
    \ind |\[Am]Om Mani Padme Hum |\[G]Om Mani Padme Hum
    \ind |\[F]Om Mani \[Em]Padme |\[Am]Hum
  \endchorus
  \beginchorus
    Eu |\sublyr{\up{\textbf{1}}Om}\[Am]vi ma\sublyr{Ma-}mãe \sublyr{ni}O\sublyr{Pad-}xum \sublyr{me}na \sublyr{Hum}cacho|\sublyr{Om}\[G6]ei\sublyr{Mani}ra \sublyrpush{Padme Hum}sen-
    |\sublyr{Om}\[F]tada \sublyr{Ma-}na \sublyr{ni}bei\sublyr{Pad-}\[G7]ra \sublyr{me}do r|\sublyr{Hum}\[Am]io
  \endchorus
  \beginchorus\rep{4}
    Colhendo |\[Dm]lírio, lírio lê \[G]{} colhendo
    |\[C]lírio, lírio lá \[F]{} colhendo
    |\[B\textdegree7]lírio pra enfei\[E7]tar o seu con|\[Am]gá
  \endchorus
  \goto{Om Mani Padme Hum}
  \begin{translation}
    The day of Diwali has arrived.
    On this auspicious occasion allow us to worship Shri Mahalaxmi.
    All the Gods and Goddesses worship you,
    oh Mahalaxmi Mataji Goddess of Chindwara.
    \nextverse
    I saw Mother Oxum at the waterfall
    sitting on the river's edge
    gathering lilies, there picking lilies,
    lilies to decorate her altar.
    % Image downloaded from: https://openclipart.org/detail/185904/lily
    % Original: Drinks of the World --- James Mew and John Ashton, 1892
    % Image license: Public Domain
    \imager[5]{lily_drawing_bw_transparent_bg_PD__1025px.png}%
  \end{translation}
  \begin{explanation}[EN]
    \begin{description}
      \item[Mataji:] a Hindi term meaning ``respected mother''
      \item[Diwali:] the yearly Hindu festival of lights, when prayers are offered to
        \textbf{Lakshmi}, goddess of prosperity and fortune
      \item[Oxum:] see song \emph{Ide Were} for explanation
    \end{description}
  \end{explanation}
\endsong


\beginsong{Shiva Shambho}[tags={Shiva},ph={II, IV}]
  \audio[]{https://www.youtube.com/watch?v=YBso7TPtvJU}
  \beginchorus
    |\[\mnc{A}Am]Jaya jaya Shiva \[\mn{B}]Sham|\[\mn{C}\mn{B}\mn{A}]bho, |\[\mnc{G}G]jaya jaya \[\mn{C}]Shiva \[\mn{B}]sham|\[\mnc{A}Am]bho
  \endchorus
  \beginchorus
    |\[Am]Mahadeva sham|bho, |\[G]Mahadeva sham|\[Am]bho
  \endchorus
  \begin{explanation}[EN]
    \begin{description}
      \item[Mahadeva:] a title for Lord Shiva, meaning ``Great God``
      \item[Shambho:] ``the auspicious one``
    \end{description}
  \end{explanation}
\endsong


\beginsong{Haidakandhi}[tags={Shiva, Vishnu}, ph={III, IV}, key={Am}, gk={Am, Gm--Bm}]
  \meter{4}{4}
  \beginchorus
    |\[Am]{} \[\mn{A}]Om Namah \[\mn{B}]Shi|\[\mnc{C}F]va\[\mn{A}]ya \[\mn{C}]Na\[\mn{D}]mah |\[\mnc{E}C]Om |\[\mnc{D}G]Haida\[\mn{C}]kan\[\mn{B}]dhi
  \endchorus
  \beginchorus
    |\[Am]{} \[\mn{A}]Hari Ha\[\mn{G}\mn{F}]ri |\[F]{} Hari \[\mn{G}]Ha\[\mn{F}\mn{E}]ri |\[C]{} Hari \[\mn{F}]Ha\[\mn{E}]ri |\[\mnc{D}G]Shan\[\mn{C}]ka\[\mn{B}]ra
  \endchorus
  \begin{explanation}[EN]
    \begin{description}
      \item[Hari] is a name for the supreme absolute in the \emph{Vedas} (also
      in \emph{Guru Granth Sahib} and many other sacred texts of South Asia).
      Hari refers to \emph{Vishnu} who takes away all the sorrows of his
      devotees.
      \item[Shankara] is one of the names for Shiva.
    \end{description}
  \end{explanation}
\endsong


\begin{intersong} % A quote from the Upanishads about the fourth state of consciousness
  \begin{feeler}
    ``There must be a fourth state beyond the waking, dreaming and dreamless states in which
    the absolute oneness of Brahman-Atman is what should be known.''\\
    --- \emph{The Upanishads}
  \end{feeler}
  \begin{explanation}[EN]
    \begin{description}
      \item[Brahman] is the highest Universal Principle, the Ultimate Reality. It is the final
        cause of all that exists.
      \item[Atman] is the true inner self, the soul, of an individual.
      \item[The Upanishads] are ancient Sanskrit texts that contain some of the central
        philosophical concepts of Hinduism.
    \end{description}
  \end{explanation}
\end{intersong}


\beginsong{Om Namah Shivaya}[tags={Shiva},ph={IV}]
  \beginchorus\memorize % memorize chords even though in 'chorus'
    \textnotefornext{intro:}*
    |\[\mnc{A}Am]Om Namah Shi|\[\mnc{C}F]vaya; |\[\mnc{D}G]Om Na\[^\mn{C}]mah \[^\mn{B}]Shi|\[\mnc{A}Am]vaya
  \endchorus
  \notesoff
  \beginchorus
    |^ Shivaya |^Namaha; |^ Shivaya |^Namaho
  \endchorus
  \beginchorus
    |^Sham Bol ^Shankara |^Namah ^Shivaya; \replay
    |^Girija ^Shankara |^Namah ^Shivaya
  \endchorus
  \beginchorus
    |^Aruna^chala Shiva |^Namah Shi^vaya; \replay
    |^Aruna^chala Shiva |^Namah ^Shivaya
  \endchorus
  \beginchorus
    Hari |\[C]Om Namah Shi|\[G]vaya; |\[F]Om Namah Shi|\[Am]vaya
  \endchorus
  \beginchorus
    \textnotefornext{outro:}
    |^Om Namah Shi|^vaya; |^Om Namah Shi|^vaya
  \endchorus
  % Image downloaded from: https://imgbin.com/png/VxTaSrxf/shiva-hanuman-art-ganesha-sai-baba-of-shirdi-png
  % Image license: Free for non-commercial use
  \imagecc[3]{shiva_bw_transparent_bg_760x859px.png}%
\endsong


\beginsong{Ganesha Mantra \\ Removing of obstacles Mantra }[index={Om Gam Ganapatayei Namaha},by={Prembabanda},tags={Ganesha},ph={I, IV}]
  \showmantra{Om Gam Ganapatayei Namaha}
  \begin{feeler}
    Salutations to the remover of obstacles.
  \end{feeler}
  \begin{explanation}[EN]
    This sound formula assists us in the removal of obstacles. In order for that to happen there
    is no need to know the exact nature of the hindrances. Just the awareness and recognition that
    there are obstacles and then chanting this mantra with the intention for resolve is enough.
    This mantra unifies us within. When there is oneness there are no obstacles. This mantra is
    also used for the beginning of any endeavor. Whenever we start anything anew we can bless the
    project with the energy of Ganesh through this mantra.
    \vspace{2em}
    \begin{description}
      \item[Gam:] the seed sound for Ganesh
      \item[Ganapati:] another name for Ganesh --- the Remover of Obstacles, and of Oneness/Unity
      \item[Yei:] a sound that activates shakti/energy
    \end{description}
  \end{explanation}
  \imagecb[2]{ganesha_bw_transparent_background_1280x1232.png}%
  \textnotefornext{song: part A}
  \beginchorus\memorize % memorize chords even though in 'chorus'
    |\[\mnc{E}Em]Om \[^\mn{B}]Parvati Patayei |\[\mnc{A}Bm]Hara Ha\[^\mn{F#}]ra \[^\mn{A}]Hara Ma\[^\mn{B}]ha\[^\mn{A}]de\[^\mn{G}]va
    |\[C]{} Gajana\[D]nam Bu|\[Em]ta
  \endchorus
  \notesoff
  \beginchorus
    |^Ganadi Sevatam |^ Kapitha Jambu
    |^ Phalacha^ru |^Bhakshanam
  \endchorus
  \beginchorus
    |^Umasutam Shoka |^ Vinasha Karakam
    |^ Namami ^Vigneshvara |^Pada Pankajam
  \endchorus
  \textnotefornext{part B}
  \beginchorus
    |\[Em]Om Gam Ganapata|yei Nama\[Bm]ha
    |\[Em]Om Gam Ganapata|yei Nama\[Bm]ha
    |\[G]Om Gam Ganapata|yei Nama\[Bm]ha
    |\[Em]Om Gam Ganapata|yei Nama\[Bm]ha
  \endchorus
  \begin{translation}
    O elephant---faced God, Ganesha,
    You are served by the attendants of Shiva.
    \nextverse
    And you eat forest apples and blackberries.
    \nextverse
    You are \emph{Uma}'s son, the destroyer of sorrows.
    I bow to the lotus feet of the remover of obstacles.
  \end{translation}
  \begin{explanation}[EN]
    \begin{description}
      \item[Uma:] ``light'', Lady of the Mountains, also known as \emph{Parvati}
    \end{description}
  \end{explanation}
\endsong


\beginsong{Ganesha Sharanam}[tags={Ganesha},ph={I, IV}]
  \audio[]{https://soundcloud.com/sound-of-light/61-ganesh-sharanam}
  \beginverse
    |\[\mnc{E}Em]Om Bom |Hare
    Na|'maha Shi|\[\mnlow{G}\mnlow{F#}]va\[\mnlow{E}]ya
    |\[C]{} Ganga Par|vati Ma
    |\[D]{} Ganesha |\[B7]Sharanam
  \endverse
\endsong


\beginsong{Jaya Ganesha}[tags={Ganesha},ph={I, IV}]
  \audio[]{https://soundcloud.com/sound-of-light/60-jai-ganesh}
  \beginchorus
    |\[\mnc{C}C]Jaya Ga\[\mn{D}]nesha |\[\mn{E}]jaya Ga\[\mn{F}]nesha |\[\mn{E}]jaya Ga\[\mn{F}]ne\[\mn{E}]sha |\[G]de\[\mn{D}]va
  \endchorus
  \beginchorus
    |\[F]Mata jaki |\[G]Parvati |\[Dm]pita \[G]Maha|\[C]deva
  \endchorus
  \begin{translation}
    Glory to You, O Lord Ganesha!
    \nextverse
    Born of Parvati, daughter of the Himalayas, and the great Shiva.
  \end{translation}
\endsong


\beginsong{Govinda Hari Om}[tags={Vishnu, Krishna},ph={III},key={Am},gk={Am, Gm--C\shrp{}m}]
  % in Am the notes range from A to G'
  \meter{4}{4}
  \beginverse
    |\[\mnc{A}Am]Go\[\mn{E}\mn{D}]vin|\[Dm]da | \[\mn{C}]Ha\[\mn{B}]ri |\[\mnc{A}Am]Om Hari |\[\mnc{B}E]Hari | \altchords{\id[1]{(Bm)}|Bm |Em | - |Bm |F\shrp{}}
    |\[Am]Gopa|\[Dm]la | Hari |\[Am]Om|\[G]{} | \altchords{|Bm |Em | - |Bm |A}
    |\[C]Sada |\[Dm]Sadhana | Ananda |\[Am]Bhavana| \altchords{|D |Em | - |Bm}
    \endverse\glueverses\beginchorus
    |\[Dm]Vishnu |\[Am]Sadhana |\[E]Hari |\[Am]Om|\altchords{|Em |Bm |F\shrp{} |Bm}
  \endchorus
\endsong


\beginsong{Guru Brahma}[by={Adi Sankaracharya},tags={teacher},ph={II, III}]
  \audio[]{https://soundcloud.com/bastiaan-yansa/guru-brahma}
  \transpose{5}
  \beginverse
    \[\mn{E}]Gur|\[C]ur Brahm|\[Am]a Gur|\[\mnc{F#}B7]ur \[\mn{E}]Vishn|\[\mnc{G}Em]u
    Gur|\[C]ur Dev|\[Am]o Mah|\[B7]eshwar|\[Em]ah
    Gu|\[C]ru Saak|\[Am]shaat Par|\[B7]a Brahm|\[Em]a
    Tas|\[C]mai S|\[Am]hri Guruv|\[B7]e Namah|\[Em]a | \e
  \endverse
  \begin{translation}
    Guru that is Brahma, Guru that is Vishnu,
    Guru that is Lord Maheshwara \emph{(Shiva)}.
    Guru that is verily the supreme reality.
    Sublime prostrations to that Guru.
  \end{translation}
  \begin{explanation}[EN]
    This \emph{shloka} (category of verse line developed from the Vedic Anuṣṭubh poetic meter)
    is by \emph{Adi Sankaracharya} (788--820), a Hindu mystic, as a part of \emph{Guru strotam},
    a sacred prayer dedicated to his spiritual guide \emph{Govinda Bhagwadpada}.
  \end{explanation}
\endsong


\beginsong{Om Shanti Om \\ Peace Mantra}[tags={peace},ph={I, II},key={Am},gk={Am, Am--D\shrp{}m}]
  % in Am the notes range from G to F'
  \mnbeginchorus
    |\[\mnc{A}Am]Om |\[\mnc{G}Em]Shan\[\mn{C}\mn{B}]ti |\[\mnc{A}Am]Om | \e \altchords{\id[1]{(Bm)}|Bm |F\shrp{}m |Bm | \e}
    \mnendchorus\glueverses\mnbeginverse
    \[\mn{A}]Om |\[\mnc{D}Dm]Shanti |\[\mnc{F}Dm/F]Shanti |\[\mnc{E}Am/E]Shan\[\mn{D}]ti\[\mn{E}]hi | \e \altchords{|Em |Em/G |Bm/F\shrp{} | \e}
  \mnendverse
  \notesoff
  \textnotefornext{outro:}
  \beginverse
    |\[Am]Om |\[Em]Shanti |\[Am]Om | \e \altchords{|Bm | F\shrp{}m | Bm | \e}
  \endverse
  \begin{feeler}
    Peace in my heart, peace with each other, peace in the cosmos.
  \end{feeler}
\endsong


\beginsong{Sudhossi Budhossi \\ Forever Pure}[by={trad., Shimshai},ex={saṃskṛtam, english, español}, tags={liberation, transcendence},ph={III}]
  \audio[]{https://www.youtube.com/watch?v=AMH7WaiLqWk}
  \newchords{sudhossi_a}\newchords{sudhossi_b}
  \meter{3}{4}
  \beginverse\memorize[sudhossi_a]
    \[^\mn{B}]Su|\[\mnc{E}Em]dhos\[^\mn{F#}]si \[^\mn{E}]bu|\[\mnc{G}C]dhossi ni|\[\mnc{A}D]ran\[^\mn{G}]ja\[^\mn{F#}]no|\[\mnc{E}Em]si
    Sam|\[Am]sara may|\[Em]a pari|\[D]var jito|\[Em]si
  \endverse\glueverses\beginchorus\memorize[sudhossi_b]
    Sam|\[Am]sara swapa|\[Em]nam tria|\[Am]ja mohan ni|\[Em]dram
    Na|\[Am]jamna mri|\[C]tyor tat|\[D]sat swa ru|\[Em]pe
  \endchorus
  \notesoff
  \beginchorus
    \ind |\[D]Na na na\ldots |\[Em]{} |\[D]{} |\[Em]
  \endchorus
  \beginverse\replay[sudhossi_a]
    You |^are forever |^pure, you |^are forever |^true
    And the |^dream of this |^world can |^never touch |^you
  \endverse\glueverses\beginchorus\replay[sudhossi_b]
    So |^give up your at|^tachment and |^give up your con|^fusion
    And |^fly to that |^space that's be|^yond all il|^lusion
    % % Alternate last line (both been used by the author):
    %And a|^bide in the |^truth that's be|^yond all il|^lusion
  \endchorus
  \goto{Na na na}
  \beginverse\replay[sudhossi_a]
    E|^res siempre |^puro e|^res verda|^dero
    Y el |^sueño del |^mundo no |^te toca|^rá
  \endverse\glueverses\beginchorus\replay[sudhossi_b]
    De|^ja los a|^pegos de|^ja la confu|^sión
    Y vi|^ve en la ver|^dad más al|^lá de la ilu|^sión\goto{Na na na}
  \endchorus
  \begin{feeler}
    It is said that this Sanskrit mantra was originally sung every night as\\
    a lullaby by an Indian mother to her 12 children who all became sadhus.
  \end{feeler}
\endsong


\beginsong{Dhanvantre Mantra \\ Healing Mantra}[index={Om Shree Dhanvantre},tags={health},ph={III}]
  \showmantra{Om Shree Dhanvantre Namaha}
  \begin{feeler}
    Salutations to the being and power of the Celestial Healer.
  \end{feeler}
  \begin{explanation}[EN]
    \textbf{Dhanvantari} is the celestial healer. This mantra helps us find the right path to 
    healing, or directs us to the right health practitioner. In India it is also commonly chanted 
    during cooking in order for the food to be charged with healing vibrations – either to prevent 
    disease or assist in healing for those who are sick. This mantra can be chanted for any 
    situation that one would like to be healed or remedied. Good to remember and be open to the 
    path of healing not necessarily looking the way we expect it!
  \end{explanation}
\endsong


\beginsong{Om Namo Bhagavate Vasudevaya}[tags={liberation},ph={II, III}]
  \showmantra{Om Namo Bhagavate Vasudevaya}
  \begin{feeler}
    Salutations to the Indweller who is omnipresent, omnipotent, immortal and divine.
  \end{feeler}
  \begin{explanation}[EN]
    \textbf{Vasudeva} is the individual aspect of God that dwells inside of us. This mantra frees
    our minds and spirits from negative patterns in this life. Regular and consistent practice of
    this mantra gives us a complete spiritual freedom: it frees us from the cycle of rebirth and
    helps us realize ourselves as a manifestation of transcendent divinity. It can also help bring
    in an advanced spiritual soul if chanted by the mother during pregnancy.
  \end{explanation}
\endsong


\beginsong{Hari Om Shiva Om\ldots \\ Cosmic vibration Mantra}[tags={Vishnu, Shiva},ph={II}]
  \showmantra{Hari Om Shiva Om Shiva Om Hari Om}
  \begin{explanation}[EN]
    \textbf{Hari} is another name of Lord Vishnu. Can also be translated as The Remover of ego.
    Universal mantra of cosmic vibration.
  \end{explanation}
\endsong


\beginsong{Om Eim Saraswatyei Namaha}[tags={learning},ph={II}]
  \showmantra{Om Eim Saraswatyei Namaha}
  \begin{explanation}[EN]
    Salutations to Saraswati, the goddess of music, poetry, the arts, education, 
    learning and divine speech. Opens us towards education, learning, and the artistic world of
    music and poetry. Whenever you find yourself moved to tears by a piece of music, or touched
    by the words of the great poets and sages, you are in the presence of Saraswati. May we be
    at ease while learning the wonders of the unfolding mystery of Life.
  \end{explanation}
\endsong


\beginsong{Om Namo Narayana}[tags={transcendence},ph={II}]
  \showmantra{Om Namo Narayana}
  \begin{explanation}[EN]
    I bow to the divine. Salutes the all-pervading aspect of the Great Spirit anchored 
    in our hearts and in all beings. Destroys barriers, obstacles, afflictions, and difficulties.
    Leads to self-realisation. Traditionally chanted to assist the dying as they make their 
    transition, the mantra asks prayerfully, that we may all merge into the grace of divine light.
  \end{explanation}
\endsong


% Image to show on the empty recto page (remove if no longer empty!)
\begin{intersong}%
  \subsection*{Chakras}
  \imagecc[1]{Chakras_map_997x1132px.png}%
  \begin{description}
    \item[Sahasrāra:] ``thousand-petaled'', the crown chakra
    \item[Ājñā:] ``command'', the third eye chakra
    \item[Viśuddha:] ``especially pure'', the throat chakra
    \item[Anāhata:] ``unstruck'', the heart chakra
    \item[Maṇipūra:] ``jewel city'', the solar plexus chakra
    \item[Svādhiṣṭhāna:] ``one's own base'', the sacral chakra
    \item[Mūlādhāra:] ``root support'', the root chakra
  \end{description}
\end{intersong}


\beginsong{Rigveda: Soma}[by={translated to Finnish by Klaus Karttunen},ex={from Rigveda (book 8, hymn 48), ca 1000 BCE}]
  \chordsoff % no vertical space for non-existing chords
  \normalsize % to fit on one spread
  {\noindent\textbf{Somalle:}}\vspace{2em}
  \beginverse
    Makeasta ravinnosta olen ollut osallinen, viisaana,
    hyvin huolehtivasta, parhaan vapauden löytäjästä,
    jonka luo kokoontuvat jumalat ja ihmiset,
    medeksi he sitä nimittävät.
  \endverse
  \beginverse
    Kun olet tunkeutunut sisään, olet kuin Aditi,
    jumalallisen vihan karkottaja,
    oi mehu, Indran kumppanuudesta iloiten
    aja meidät rikkauteen kuin tottelevainen tamma valjaisiin.
  \endverse
  \beginverse
    Olemme juoneet Somaa, olemme tulleet kuolemattomiksi,
    olemme menneet valoon, olemme löytäneet jumalat.
    Mitä nyt meille tekisi vihamielisyys?
    Mitä kuolevaisen pahuus, oi kuolematon?
  \endverse
  \beginverse
    Siunaukseksi tule juotuna sydämellemme, oi mehu,
    ystävällinen ole kuin isä pojalle, oi Soma,
    ymmärtävä kuin ystävä ystävälle, laajamaineinen,
    pidennä elämäämme, Soma, elääksemme.
  \endverse
  \beginverse
    Juotuani nämä ihanat, vapauttavat mehut,
    niveliäni ne sitovat kuin vaunuja nahkahihnat,
    suojelkoot ne minua jalan murtumalta,
    pitäkööt mehut minut erossa katkeamisesta.
  \endverse
  \beginverse
    Sytytä minut kuin sytytetty tuli,
    tee kauasnäkeväksi, tee meidät paremmiksi,
    sillä silloin sinun juopumuksessasi, Soma,
    kuin rikkaana itseäni pidän --- etene menestykseksi.
  \endverse
  \beginverse
    Sinusta puserretusta, innokkain mielin
    nauttisimme kuin perintöomaisuudesta;
    kuningas Soma, pidennä elämäämme,
    kuin Aurinko keväisiä päiviä.
  \endverse
  \beginverse
    Kuningas Soma, ole meille lempeä onneksemme,
    sinun palvojiasi me olemme, tiedä se!
    Nousee kyky ja into, oi mehu,
    älä anna meitä pois vihollisen tahdosta.
  \endverse
  \beginverse
    Sillä sinä olet ruumiittemme suojelija, Soma,
    miesten valvojana olet asettunut kaikkiin jäseniin,
    jos me sinun lupauksesi rikkoisimme,
    ole meille lempeä, hyvä ystävä onneksemme, oi jumala.
  \endverse
  \beginverse
    Olisimmepa lempeän ystävämme kanssa,
    älköön hän minua juotuna vahingoittako, keltakasvoinen,
    tämä Soma, joka on asetettu meihin,
    siksi menen pyytämään Indralta elämäni pidennystä.
  \endverse
  \beginverse
    Pois ovat menneet uupumukset, sairaudet
    vavisten hävisivät, pimentäjättäret ovat pelästyneet,
    mahtava Soma on meihin noussut,
    olemme tulleet sinne, missä elämä pidentyy.
  \endverse
  \beginverse
    Mehu joka juotuna sydämiimme, oi isät,
    kuolematon kuolevaisiin on saapunut,
    sitä Somaa vuodatuksi haluamme kunnioittaa,
    olla hänen armossaan ja suopeamielisyydessään.
  \endverse
  \beginverse
    Sinä, Soma, olet yhdessä isien kanssa
    ulottanut itsesi halki taivaan ja maan,
    sellaista sinua me vuodatuksin haluamme kunnioittaa,
    oi mehu, olisimmepa rikkauksien valtiaita.
  \endverse
  \beginverse
    Puhukaa puolestamme, te suojelevat jumalat,
    älköön meitä uni kukistako, älköön tyhjä puhe,
    me Somalle joka päivä rakkaina
    kokouksessa haluamme puhua hyvine poikinemme.
  \endverse
  \beginverse
    Sinä meille, Soma, joka puolelta voiman antaja,
    sinä valon löytäjä, astu meihin, miesten valvoja,
    sinä meitä, oi mehu, apulaistesi kanssa
    suojele takaapäin ja myöskin edestäpäin.
  \endverse
  \begin{explanation}[EN]
    \begin{description}
      \item[Rigveda] on kokoelma varhaisia intialaisia uskonnollisia hymnejä. Sitä pidetään
        vanhimpana Intian pyhistä teksteistä; se lienee laadittu joskus 1700--1000 eaa.
      \item[Soma] on kasviperäinen enteogeeninen rituaalijuoma, jonka sisältämistä kasveista
        ei ole varmaa tietoa. Rigvedassa Somalle on osoitettu lukuisia hymnejä.
    \end{description}
  \end{explanation}
\endsong


%%%%%%%%%%%%%%%%%%%%%%%%%%%%%%%%%%%%%%%%%%%%%%%%%%%%%%%%%%%%%%%%%%%
%%% LATEST PRINTOUT CONTAINED THE SONGS ABOVE.                  %%%
%%%%%%%%%%%%%%%%%%%%%%%%%%%%%%%%%%%%%%%%%%%%%%%%%%%%%%%%%%%%%%%%%%%
%%% Please try to not change the song numbers above this point. %%%
%%% Add new songs only after this point.                        %%%
%%%%%%%%%%%%%%%%%%%%%%%%%%%%%%%%%%%%%%%%%%%%%%%%%%%%%%%%%%%%%%%%%%%

